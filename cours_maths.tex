% Set up the document's format to A4 and the font's size to 12pt.
\documentclass[a4paper,12pt]{report}

% Set up the document's title, author and date.
\title{Maths -- MP2I}
\author{Romain Bricout}
\date{\today}

% Set up the input's encoding to UTF-8, the document's font and language to T1 (adapted to french) and french (the grammar linter uses this parameter).
\usepackage[utf8]{inputenc}
\usepackage[T1]{fontenc}
\usepackage[french]{babel}

\usepackage[dvipsnames]{xcolor}

% Set up the document's margins.
\usepackage{geometry}
\geometry{hmargin=1.5cm,vmargin=1.5cm}

% The three main maths packages. They are used for a lot of things.
\usepackage{amssymb,amsmath}
\usepackage{mathtools}

% Useful to create nice and easy signs or variations tables.
\usepackage{tkz-tab}

% Useful to create any kind of visual representation (graph functions, illustrate geometry problems, etc)
\usepackage{tikz}
\usetikzlibrary{patterns,angles,quotes,arrows,arrows.meta,bending,matrix,calc}

% Allows to edit the itemize environment's default item document-wide.
\usepackage{enumitem}

% Allows to define \notfoo or \nfoo (not recommended) in order for \not\foo to work as wished.
\usepackage{newtxmath}
\DeclareSymbolFont{CMletters}{OML}{cmm}{m}{it}
\DeclareMathSymbol{\nu}{\mathord}{CMletters}{23}

% Makes the table of contents clickable and gives useful commands for links in general.
\usepackage[hypertexnames=false]{hyperref}
\hypersetup{colorlinks=false,linktoc=all}

% Gives the llbracket and rrbracket commands for integer intervals.
\usepackage{stmaryrd}

% Useful to insert nice-looking quotes.
\usepackage{epigraph}

% Allows to insert chapter-specific table of contents.
\usepackage{minitoc}
\mtcselectlanguage{french}
\setcounter{minitocdepth}{6}

% Useful when units are needed.
\usepackage{siunitx}
\sisetup{
locale=FR,
detect-all,
inter-unit-product=\ensuremath{\cdot},
list-final-separator={et},
list-pair-separator={et},
range-phrase={\ensuremath{\xleftrightarrow{}}},
exponent-product=\ensuremath{\cdot},
per-mode=power-positive-first
}

\usepackage[thmmarks,hyperref]{ntheorem}
\makeatletter
\let\old@thm\@thm
\usepackage[lowercase]{theoremref}
\def\@thm#1#2#3{\def\thmref@currname{#3}\old@thm{#1}{#2}{#3}}
\makeatother

% Allows whiteboard digits with \mathds
\usepackage{dsfont}

\usepackage{needspace}

% Useful for better-looking oneline fractions
\usepackage{nicefrac}

% Set up the horizontal space before the first line of a new paragraph to 2em and the vertical space between two paragraphs to 1em.
\setlength{\parindent}{0pt}
\setlength{\parskip}{1em}

% Adds 0.5em to the vertical space between two lines in an align environment. It looks better.
\addtolength{\jot}{0.5em}

% Allows align environment to break if it's too long to fit in the page where it began.
\allowdisplaybreaks[1]

% Trick to make semicolons considered like relation operators (such as =) and therefore being equidistantly spaced from the two numbers around it.
\mathcode`;=\numexpr\mathcode`;-"3000

% Commands for size-adaptative parentheses, brackets, curly brackets, absolute value and magnitude.
\newcommand{\paren}[1]{\left(#1\right)} % (x)
\newcommand{\croch}[1]{\left[#1\right]} % [x]
\newcommand{\accol}[1]{\left\lbrace#1\right\rbrace} % {x}
\newcommand{\abs}[1]{\left\lvert#1\right\rvert} % |x|
\newcommand{\norme}[1]{\left\|#1\right\|} % ||x||
\newcommand{\floor}[1]{\left\lfloor#1\right\rfloor} % ⌊x⌋
\newcommand{\ceil}[1]{\left\lceil#1\right\rceil} % ⌈x⌉

% Commands for size-adaptative intervals and integer intervals. The commands' roots are "interv" and "interventier" and the added e or i at the end mean "excluded" and "included" respectively.
\newcommand{\intervii}[2]{\left[#1;#2\right]} % [a;b]
\newcommand{\intervee}[2]{\left]#1;#2\right[} % ]a;b[
\newcommand{\intervie}[2]{\left[#1;#2\right[} % [a;b[
\newcommand{\intervei}[2]{\left]#1;#2\right]} % ]a;b]
\newcommand{\interventierii}[2]{\left\llbracket#1;#2\right\rrbracket} % non-ASCII characters needed
\newcommand{\interventieree}[2]{\left\rrbracket#1;#2\right\llbracket} % non-ASCII characters needed
\newcommand{\interventierie}[2]{\left\llbracket#1;#2\right\llbracket} % non-ASCII characters needed
\newcommand{\interventierei}[2]{\left\rrbracket#1;#2\right\rrbracket} % non-ASCII characters needed

% Commands for usually used sets.
\newcommand{\N}{\mathbb{N}} % natural integers
\newcommand{\Ns}{\mathbb{N}^*}

\newcommand{\Z}{\mathbb{Z}} % relative integers
\newcommand{\Zp}{\mathbb{Z}_+}
\newcommand{\Zs}{\mathbb{Z}^*}
\newcommand{\Zps}{\mathbb{Z}_+^*}

\newcommand{\D}{\mathbb{D}} % decimal numbers
\newcommand{\Dp}{\mathbb{D}_+}
\newcommand{\Dm}{\mathbb{D}_-}
\newcommand{\Ds}{\mathbb{D}^*}
\newcommand{\Dps}{\mathbb{D}_+^*}
\newcommand{\Dms}{\mathbb{D}_-^*}

\newcommand{\Q}{\mathbb{Q}} % rational numbers
\newcommand{\Qp}{\mathbb{Q}_+}
\newcommand{\Qm}{\mathbb{Q}_-}
\newcommand{\Qs}{\mathbb{Q}^*}
\newcommand{\Qps}{\mathbb{Q}_+^*}
\newcommand{\Qms}{\mathbb{Q}_-^*}

\newcommand{\R}{\mathbb{R}} % real numbers
\newcommand{\Rp}{\mathbb{R}_+}
\newcommand{\Rm}{\mathbb{R}_-}
\newcommand{\Rs}{\mathbb{R}^*}
\newcommand{\Rps}{\mathbb{R}_+^*}
\newcommand{\Rms}{\mathbb{R}_-^*}
\newcommand{\Rb}{\overline{\mathbb{R}}}

\newcommand{\C}{\mathbb{C}} % complex numbers
\newcommand{\Cs}{\mathbb{C}^*}

\newcommand{\K}{\mathbb{K}}
\newcommand{\Ks}{\mathbb{K}^*}

\newcommand{\A}{\mathbb{A}}
\renewcommand{\L}[2]{\mathscr{L}\paren{#1,#2}}
\newcommand{\Lendo}[1]{\mathscr{L}\paren{#1}}

\newcommand{\prem}{\mathbb{P}}

\newcommand{\U}{\mathbb{U}} % complex numbers whose modulus is 1
\renewcommand{\H}{\mathbb{H}} % quaternions (\H normally prints slanted quotation marks)
\renewcommand{\O}{\mathbb{O}} % octonions (\O normally prints a slashed capital o : Ø)

\renewcommand{\P}[1]{\mathscr{P}\paren{#1}} % subsets of a set
\newcommand{\F}[2]{\mathscr{F}\paren{#1,#2}} % functions from 1 to 2
\newcommand{\V}[1]{\mathscr{V}\paren{#1}} % neighborhood of a number

% Redefines \Re and \Im to print Re and Im (the same way as ln or lim) instead of fraktur R and I which don't look nice and are less readable.
\renewcommand{\Re}{\operatorname{Re}}
\renewcommand{\Im}{\operatorname{Im}}
\newcommand{\Card}{\operatorname{Card}}

% Command to print an upright e for the exponential instead of a slanted e and put the exponent.
\newcommand{\e}[1]{\mathrm{e}^{#1}}

% Command to print the imaginary i with a little space on the right. This way, the exponents don't look confusing. \i normally prints a dotless i.
\renewcommand{\i}{i\mkern1mu}

% Redefines \vec such that the arrow covers the whole name of the vector.
%\renewcommand{\vec}[1]{\overrightarrow{#1}}

% Commands for 2D and 3D vectors' coordinates
\newcommand{\dcoords}[2]{\begin{pmatrix}#1\\#2\end{pmatrix}}
\newcommand{\tcoords}[3]{\begin{pmatrix}#1\\#2\\#3\end{pmatrix}}

% Redefines binom to print nicer parentheses around the numbers.
\renewcommand{\binom}[2]{\begin{pmatrix}#2\\#1\end{pmatrix}}

% Command for a QED black square. It automatically prints a whitespace before the square such that it looks nice.
\newcommand{\cqfd}{\text{ }\blacksquare}

% Commands with more explicit names for the best way to express divisibility (mid and nmid).
\newcommand{\divise}{\mid}
\newcommand{\notdivise}{\nmid}

% Commands that do the exact same thing but with explicit names for a complex number's conjugate and an event's negation in probability.
\newcommand{\conj}[1]{\overline{#1}}

% Command for a size-adaptative middle bar meaning "such that" (with spacing around it in order to look nice).
\newcommand{\tq}{\;\middle|\;}

% Command with an explicit name for the scalar product.
\newcommand{\scal}{\cdot}
\newcommand{\vecto}{\operatorname{_\wedge}}

% Shortcut for forcing displaystyle in inline mode.
\newcommand{\ds}{\displaystyle}

% Make the not version of implies, impliedby and iff look nice.
\newcommand{\notimp}{\centernot{\imp}}
\newcommand{\notimpr}{\centernot{\impr}}
\newcommand{\notssi}{\centernot{\ssi}}

\renewcommand{\subset}{\subseteq}
\renewcommand{\supset}{\supseteq}
\newcommand{\notsubset}{\centernot{\subset}}
\newcommand{\notsupset}{\centernot{\supset}}

% Shortcut for P(event).
\newcommand{\proba}[1]{\mathbb{P}\paren{#1}}
\newcommand{\probacond}[2]{\mathbb{P}_{#2}\paren{#1}}

% More explicit names for land (logical and) and lor (logical or).
\newcommand{\et}{\land}
\newcommand{\ou}{\lor}
\newcommand{\non}{\lnot}

% Explicitly named environment for tkz-tab tables. Automatically centers the table and handles the tikzpicture environment.
\newenvironment{tkz}
{
\begin{tikzpicture}
}
{
\end{tikzpicture}
}

% More explicitly named commands for the creation of tkz-tab tables.
\newcommand{\tableauinit}[2]{\tkzTabInit{#1}{#2}}
\newcommand{\tableausignes}[1]{\tkzTabLine{#1}}
\newcommand{\tableauvariations}[1]{\tkzTabVar{#1}}

% Shortcut for the curve and the domain of the given function.
\newcommand{\graphe}[1]{\Gamma_{#1}}
\newcommand{\ensembledef}[1]{\mathcal{D}_{#1}}

\renewcommand{\S}[1]{\frak{S}_{#1}}
\newcommand{\frakA}[1]{\frak{A}_{#1}}

\newcommand{\semihrule}{\rule{256.074815pt}{0.4pt}}

% Various environments that create boxes. Each one is one type of thing (example, proof, etc). Each type has its own automatic counter.
\theoremstyle{break}
\theorembodyfont{\upshape}
\theoremheaderfont{\itshape}
\theoremprework{\bigskip\needspace{\baselineskip}\color{green}\hrule\color{black}}
\theorempostwork{\bigskip}
\newtheorem{rem}{Remarque}[chapter]

\theoremstyle{break}
\theorembodyfont{\upshape}
\theoremheaderfont{\itshape}
\theoremprework{\bigskip\needspace{\baselineskip}\color{green}\hrule\color{black}}
\theorempostwork{\bigskip}
\newtheorem{ex}[rem]{Exemple}

\theoremstyle{break}
\theorembodyfont{\upshape}
\theoremheaderfont{\itshape}
\theoremprework{\bigskip\needspace{\baselineskip}\color{green}\hrule\color{black}}
\theorempostwork{\bigskip}
\newtheorem{rappel}[rem]{Rappel}

\theoremstyle{break}
\theorembodyfont{\upshape}
\theoremheaderfont{\itshape}
\theoremprework{\bigskip\needspace{\baselineskip}\color{brown}\hrule\color{black}}
\theorempostwork{\bigskip}
\newtheorem{oubli}[rem]{Oubli}

\theoremstyle{break}
\theorembodyfont{\upshape}
\theoremheaderfont{\normalfont\bfseries}
\theoremprework{\bigskip\needspace{\baselineskip}\color{blue}\hrule\color{black}}
\theorempostwork{\bigskip}
\newtheorem{defi}[rem]{Définition}

\theoremstyle{break}
\theorembodyfont{\upshape}
\theoremheaderfont{\normalfont\bfseries}
\theoremprework{\bigskip\needspace{\baselineskip}\color{blue}\hrule\color{black}}
\theorempostwork{\bigskip}
\newtheorem{reform}[rem]{Reformulation}

\theoremstyle{break}
\theorembodyfont{\upshape}
\theoremheaderfont{\normalfont\bfseries}
\theoremprework{\bigskip\needspace{\baselineskip}\color{magenta}\hrule\color{black}}
\theorempostwork{\bigskip}
\newtheorem{exo}[rem]{Exercice}

\theoremstyle{break}
\theorembodyfont{\upshape}
\theoremheaderfont{\normalfont\bfseries}
\theoremprework{\bigskip\needspace{\baselineskip}\color{magenta}\semihrule\color{green}\semihrule\color{black}}
\theorempostwork{\bigskip}
\newtheorem{exoex}[rem]{Exercice/Exemple}

\theoremstyle{break}
\theorembodyfont{\upshape}
\theoremheaderfont{\normalfont\bfseries}
\theoremprework{\bigskip\needspace{\baselineskip}\color{blue}\semihrule\color{red}\semihrule\color{black}}
\theorempostwork{\bigskip}
\newtheorem{defprop}[rem]{Définition/Proposition}

\theoremstyle{break}
\theorembodyfont{\upshape}
\theoremheaderfont{\normalfont\bfseries}
\theoremprework{\bigskip\needspace{\baselineskip}\color{blue}\semihrule\color{red}\semihrule\color{black}}
\theorempostwork{\bigskip}
\newtheorem{deftheo}[rem]{Définition/Théorème}

\theoremstyle{break}
\theorembodyfont{\upshape}
\theoremheaderfont{\normalfont\bfseries}
\theoremprework{\bigskip\needspace{\baselineskip}\color{blue}\hrule\color{black}}
\theorempostwork{\bigskip}
\newtheorem{nota}[rem]{Notation}

\theoremstyle{break}
\theorembodyfont{\upshape}
\theoremheaderfont{\itshape}
\theoremprework{\bigskip\needspace{\baselineskip}\color{blue}\hrule}
\theorempostwork{\hrule\color{black}\needspace{\baselineskip}\bigskip}
\newtheorem*{brouill}{Brouillon}

\theoremstyle{break}
\theorembodyfont{\itshape}
\theoremheaderfont{\normalfont\bfseries}
\theoremprework{\bigskip\needspace{\baselineskip}\color{red}\hrule\color{black}}
\theorempostwork{\bigskip}
\newtheorem{theo}[rem]{Théorème}

\theoremstyle{break}
\theorembodyfont{\itshape}
\theoremheaderfont{\normalfont\bfseries}
\theoremprework{\bigskip\needspace{\baselineskip}\color{red}\hrule\color{black}}
\theorempostwork{\bigskip}
\newtheorem{prop}[rem]{Proposition}

\theoremstyle{break}
\theorembodyfont{\itshape}
\theoremheaderfont{\normalfont\bfseries}
\theoremprework{\bigskip\needspace{\baselineskip}\color{red}\hrule\color{black}}
\theorempostwork{\bigskip}
\newtheorem{cor}[rem]{Corollaire}

\theoremstyle{break}
\theorembodyfont{\itshape}
\theoremheaderfont{\normalfont\bfseries}
\theoremprework{\bigskip\needspace{\baselineskip}\color{red}\hrule\color{black}}
\theorempostwork{\bigskip}
\newtheorem{lem}[rem]{Lemme}

\theoremstyle{break}
\theorembodyfont{\upshape}
\theoremheaderfont{\normalfont\bfseries}
\theoremprework{\bigskip\needspace{\baselineskip}\color{violet}\hrule\color{black}}
\theorempostwork{\bigskip}
\newtheorem{meth}[rem]{Méthode}

\theoremstyle{break}
\theorembodyfont{\upshape}
\theoremheaderfont{\normalfont\bfseries}
\theoremprework{\bigskip\needspace{\baselineskip}\color{violet}\hrule\color{black}}
\theorempostwork{\bigskip}
\newtheorem{appl}[rem]{Application}

\theoremstyle{break}
\theorembodyfont{\upshape}
\theoremheaderfont{\normalfont\bfseries}
\theoremprework{\bigskip\needspace{\baselineskip}\color{violet}\hrule\color{black}}
\theorempostwork{\bigskip}
\newtheorem{abus}[rem]{Abus}

\theoremstyle{break}
\theorembodyfont{\upshape}
\theoremheaderfont{\normalfont\bfseries}
\theoremprework{\bigskip\needspace{\baselineskip}\color{violet}\hrule\color{black}}
\theorempostwork{\bigskip}
\newtheorem{algo}[rem]{Algorithme}

\theoremstyle{break}
\theorembodyfont{\upshape}
\theoremheaderfont{\normalfont\bfseries}
\theoremprework{\bigskip\needspace{\baselineskip}\color{violet}\hrule\color{black}}
\theorempostwork{\bigskip}
\newtheorem{bilan}[rem]{Bilan}

\theoremstyle{break}
\theorembodyfont{\upshape}
\theoremheaderfont{\itshape}
\theoremprework{\bigskip\needspace{\baselineskip}\color{BurntOrange}\hrule\color{black}}
\theorempostwork{\bigskip}
\newtheorem{corr}[rem]{Correction}

\theoremstyle{break}
\theorembodyfont{\upshape}
\theoremheaderfont{\itshape}
\theoremsymbol{\ensuremath{\cqfd}}
\theoremprework{\bigskip\needspace{\baselineskip}\color{yellow}\hrule\color{black}}
\theorempostwork{\bigskip}
\newtheorem{dem}[rem]{Démonstration}

% Trick to make chapter numbering start from 0
\setcounter{chapter}{-1}

% Commands to make proofs easier to write
\newcommand{\impdir}{\fbox{\(\imp\)}~}
\newcommand{\imprec}{\fbox{\(\impr\)}~}
\newcommand{\incdir}{\fbox{\(\subset\)}~}
\newcommand{\increc}{\fbox{\(\supset\)}~}
\newcommand{\leqbox}{\fbox{\(\leq\)}~}
\newcommand{\geqbox}{\fbox{\(\geq\)}~}
\newcommand{\unicite}{\fbox{unicité}~}
\newcommand{\existence}{\fbox{existence}~}
\newcommand{\analyse}{\fbox{analyse}~}
\newcommand{\synthese}{\fbox{synthèse}~}
\newcommand{\conclusion}{\fbox{conclusion}~}

\renewcommand{\to}{\longrightarrow}
\renewcommand{\mapsto}{\longmapsto}

\newcommand{\fonction}[5]{\begin{array}[t]{cccc}#1 : & #2 & \to & #3 \\ & #4 & \mapsto & #5\end{array}}
\newcommand{\fonctionlambda}[4]{\begin{array}[t]{ccc}#1 & \to & #2 \\ #3 & \mapsto & #4\end{array}}

\renewcommand{\leq}{\leqslant}
\renewcommand{\geq}{\geqslant}

\newcommand{\pinf}{+\infty}
\newcommand{\minf}{-\infty}

\newcommand{\id}[1]{\mathrm{id}_{#1}}

\renewcommand{\phi}{\varphi}
\renewcommand{\epsilon}{\varepsilon}

\newcommand{\ind}[1]{\mathds{1}_{#1}}

\newcommand{\iR}{\i\R}

\newcommand{\tcheby}[2]{T_{#1}\paren{#2}}
\newcommand{\utcheby}[2]{U_{#1}\paren{#2}}

\mathcode`l="8000
\begingroup
\makeatletter
\lccode`\~=`\l
\DeclareMathSymbol{\lsb@l}{\mathalpha}{letters}{`l}
\lowercase{\gdef~{\ifnum\the\mathgroup=\m@ne \ell \else \lsb@l \fi}}%
\endgroup

\newcommand{\ensvide}{\varnothing}

\newcommand{\rond}{\circ}

\newcommand{\union}{\cup}
\newcommand{\inter}{\cap}
\newcommand{\bigunion}{\bigcup}
\newcommand{\biginter}{\bigcap}

\newcommand{\ssi}{\iff}
\newcommand{\imp}{\implies}
\newcommand{\impr}{\impliedby}

\newcommand{\excluant}{\setminus}

\newcommand{\littletaller}{\mathchoice{\vphantom{\big|}}{}{}{}}
\newcommand{\restr}[2]{{
\left.\kern-\nulldelimiterspace#1\littletaller\right|_{#2}
}}
\newcommand{\corestr}[2]{{
\left.\kern-\nulldelimiterspace#1\littletaller\right|^{#2}
}}
\newcommand{\restrbar}[1]{{
\left.\kern-\nulldelimiterspace#1\littletaller\right|
}}

\newcommand{\rel}{\mathscr{R}}

\newcommand{\classesdequiv}[1]{\nicefrac{#1}{\sim}}

\newcommand{\majo}[1]{\mathrm{majorants}\paren{#1}}
\newcommand{\mino}[1]{\mathrm{minorants}\paren{#1}}

\newcommand{\ensdiv}[1]{\operatorname{div}\paren{#1}}

\newcommand{\E}[1]{\times 10^{#1}}

\setcounter{secnumdepth}{3}

\newcommand{\guillemets}[1]{\og #1 \fg{}}

\newcommand{\prim}{^{\,\prime}}
\newcommand{\seconde}{^{\,\prime\prime}}

\newcommand{\note}[1]{\textbf{\(\star\star\) #1 \(\star\star\)}}
\newcommand{\cad}{c'est-à-dire }
\newcommand{\Cad}{C'est-à-dire }
\newcommand{\ie}{\textit{i.e.} }
\newcommand{\cf}{\textit{cf.} }
\newcommand{\Cf}{\textit{Cf.} }

\usepackage{xparse}

\NewDocumentCommand{\quantifs}{>{\SplitList{;}}m}{\ProcessList{#1}{\insertquantif}}
\newcommand{\insertquantif}[1]{#1,\;\:}

\DeclareDocumentCommand{\groupe}{m O{+}}{\paren{#1,#2}}
\DeclareDocumentCommand{\anneau}{m O{+} O{\times}}{\paren{#1,#2,#3}}
\DeclareDocumentCommand{\corps}{m O{+} O{\times}}{\paren{#1,#2,#3}}

\DeclareDocumentCommand{\poly}{O{\K} O{X}}{#1\croch{#2}}
\DeclareDocumentCommand{\polydeg}{O{\K} m O{X}}{#1_{#2}\croch{#3}}
\DeclareDocumentCommand{\fracrat}{O{\K} O{X}}{#1\paren{#2}}

\DeclareDocumentCommand{\M}{m O{\K}}{\mathscr{M}_{#1}\paren{#2}}
\DeclareDocumentCommand{\sym}{m O{\K}}{\mathscr{S}_{#1}\paren{#2}}
\DeclareDocumentCommand{\antisym}{m O{\K}}{\mathscr{A}_{#1}\paren{#2}}
\DeclareDocumentCommand{\GL}{m O{\K}}{\operatorname{GL}_{#1}\paren{#2}}
\DeclareDocumentCommand{\Mat}{O{\fami{B}} m}{\operatorname{Mat}_{#1}\paren{#2}}
\newcommand{\pass}[2]{\mathscr{P}_{#1\to#2}}

\DeclareDocumentCommand{\contm}{O{\intervii{a}{b}} O{\K}}{\classe{0}_m\paren{#1,#2}}
\DeclareDocumentCommand{\Esc}{O{\intervii{a}{b}} O{\K}}{\operatorname{Esc}\paren{#1,#2}}

\usepackage{witharrows}

\newcommand{\croix}{^{\times}}

\usepackage{polynom}

\newcommand{\classe}[1]{\mathscr{C}^{#1}}
\newcommand{\ensclasse}[3]{\classe{#1}\paren{#2,#3}}

\newcommand{\deriv}[1]{^{\paren{#1}}}

\usepackage{derivative}
\derivset{\pdv}[delims-eval=.)]

\DeclareMathOperator{\Arctan}{Arctan}
\DeclareMathOperator{\Arcsin}{Arcsin}
\DeclareMathOperator{\Arccos}{Arccos}
\DeclareMathOperator{\cotan}{cotan}
\DeclareMathOperator{\sh}{sh}
\DeclareMathOperator{\ch}{ch}
\DeclareMathOperator{\sg}{sg}
\DeclareMathOperator{\supp}{supp}
\DeclareMathOperator{\Supp}{Supp}
\DeclareMathOperator{\rg}{rg}
\DeclareMathOperator{\tr}{tr}

\newcommand{\Hom}[2]{\operatorname{Hom}\paren{#1,#2}}
\newcommand{\Pol}[2]{\operatorname{Pol}\paren{#1,#2}}
\newcommand{\Aut}[1]{\operatorname{Aut}\paren{#1}}
\DeclareDocumentCommand{\Vect}{O{} m}{\operatorname{Vect}_{#1}\paren{#2}}

\newcommand{\diag}[1]{\operatorname{diag}\paren{#1}}

\usepackage{abstract}
\addto\captionsfrench{\renewcommand{\abstractname}{\Large Introduction}}

\newcommand{\inv}{^{-1}}
\newcommand{\etoile}{^{*}}

\newcounter{orcounter}

\newenvironment{orlist}
{
\begin{array}{|l}
\setcounter{orcounter}{0}
}
{
\end{array}
}

\newcommand{\oritem}[1]{%
\ifthenelse{\theorcounter<1}{}{\\ \text{ou} \\}#1\stepcounter{orcounter}
}

\NewDocumentCommand{\orenv}{>{\SplitList{\\}}m}{%
\begin{orlist}\ProcessList{#1}{\oritem}\end{orlist}}

\usepackage{pgfplots}

\DeclareDocumentCommand{\pgcd}{o o}{
\IfNoValueTF{#1}{\operatorname{pgcd}}{\operatorname{pgcd}\paren{#1,#2}}
}

\DeclareDocumentCommand{\bezout}{o o}{
\IfNoValueTF{#1}{\operatorname{bezout}}{\operatorname{bezout}\paren{#1,#2}}
}

\usepackage{minted}
\newminted{python}{linenos, breaklines, breakanywhere, breakautoindent,tabsize=4,obeytabs}
\newenvironment{code}{\VerbatimEnvironment\begin{pythoncode}}{\end{pythoncode}}

\newcommand{\valp}[2]{v_{#1}\paren{#2}}

\newcommand{\fami}[1]{\mathscr{#1}}

\newcommand{\echange}{\leftrightarrow}

\newcommand{\trans}[1]{\prescript{t}{}{#1}}

\usepackage{mathdots}

\newcommand{\detb}[1]{{\det}_{#1}}

\usepackage{cancel}

\usepackage{nicematrix}

\newcommand{\ps}[2]{\left\langle#1\tq#2\right\rangle}
\newcommand{\ortho}{^{\perp}}

\newcommand{\operp}{\mathrel{%
\begin{tikzpicture}[baseline=-0.25em]
\draw (0,0) circle (0.45em);
\draw (-0.38em,-0.25em) -- (0.38em,-0.25em);
\draw (0,-0.25em) -- (0,0.45em);
\end{tikzpicture}
}%
}

\usepackage{titletoc}
\dottedcontents{section}[5.5em]{}{3.2em}{1pc}

\newcommand{\bouleo}[2]{\mathbb{B}\paren{#1,#2}}
\newcommand{\boulef}[2]{\mathbb{B}\prim\paren{#1,#2}}
\newcommand{\sphere}[2]{\mathbb{S}\paren{#1,#2}}

\newcommand{\vdv}[2]{\operatorname{D}_{#1}#2}

\newcommand{\egqd}[1]{\underset{#1}{=}}
\newcommand{\simqd}[1]{\underset{#1}{\sim}}

\newcommand{\arr}[2]{A_{#2}^{#1}}
\newcommand{\comb}[2]{C_{#2}^{#1}}

\newcommand{\loiuniforme}[1]{\mathscr{U}\paren{#1}}
\newcommand{\loibernoulli}[1]{\mathscr{B}\paren{#1}}
\newcommand{\loibinomiale}[2]{\mathscr{B}\paren{#1,#2}}

\newcommand{\esp}[1]{\operatorname{E}\paren{#1}}
\newcommand{\vari}[1]{\operatorname{V}\paren{#1}}
\newcommand{\cov}[2]{\operatorname{Cov}\paren{#1,#2}}
\newcommand{\ecarttype}[1]{\sigma\paren{#1}}

\begin{document}
\renewcommand{\labelitemi}{\(\bullet\)}
\renewcommand{\labelenumi}{(\arabic{enumi})}

\everymath{\ds}

\maketitle

\begin{abstract}
Ce document réunit l'ensemble de mes cours de Mathématiques de MP2I, ainsi que les TDs (travaux dirigés) les accompagnant. Le professeur était M. Jansou. J'ai adapté certaines formulations me paraissant floues ou ne me plaisant pas mais le contenu pur des cours est strictement équivalent. Le document est organisé selon la hiérarchie suivante : chapitre, I), 1), a).

Les éléments des tables des matières initiale et présentes au début de chaque chapitre sont cliquables (amenant directement à la partie cliquée). C'est également le cas des références à des éléments antérieurs de la forme, par exemple, \guillemets{Démonstration 5.22}.

Dernier chapitre terminé : Chapitre 23 -- Probabilités (trou de 16 à 20).

Dernier TD terminé : Chapitre 23 -- Probabilités (à jour).

Dernier TD corrigé : aucun.
\end{abstract}

\dominitoc\tableofcontents

\part{Cours}

\chapter{Préliminaires}

\minitoc

\section{Logique}

\begin{defi}
Une proposition est une affirmation qui peut être vraie ou fausse.

Exemples : \(1+1=2\) est une proposition vraie ; \(0>1\) est une proposition fausse.
\end{defi}

\begin{defi}
Soient \(P\) et \(Q\) deux propositions. On définit :

\begin{itemize}
\item la proposition \guillemets{\(P\) et \(Q\)} (notée aussi \(P\et Q\)) de table de vérité : \begin{tabular}{c|c||c}
\(P\) & \(Q\) & \(P\et Q\) \\
\hline
V & V & V \\
V & F & F \\
F & V & F \\
F & F & F
\end{tabular}

\item la proposition \guillemets{\(P\) ou \(Q\)} (notée aussi \(P\ou Q\)) de table de vérité : \begin{tabular}{c|c||c}
\(P\) & \(Q\) & \(P\ou Q\) \\
\hline
V & V & V \\
V & F & V \\
F & V & V \\
F & F & F
\end{tabular}

\item la proposition \guillemets{\(P\) implique \(Q\)} (notée aussi \(P\imp Q\)) de table de vérité : \begin{tabular}{c|c||c}
\(P\) & \(Q\) & \(P\imp Q\) \\
\hline
V & V & V \\
V & F & F \\
F & V & V \\
F & F & V
\end{tabular}

Exemple : Soit \(x\) un nombre réel. La proposition \(x\geq0\imp x^2\geq0\) est vraie. La proposition \(x=4\imp x=5\) est fausse si \(x=4\) et vraie sinon.

\item la proposition \guillemets{\(P\) équivaut à \(Q\)} (notée aussi \(P\ssi Q\)) de table de vérité \begin{tabular}{c|c||c}
\(P\) & \(Q\) & \(P\ssi Q\) \\
\hline
V & V & V \\
V & F & F \\
F & V & F \\
F & F & V
\end{tabular}

\item la proposition \guillemets{non \(P\)} (notée aussi \(\non P\)) de table de vérité \begin{tabular}{c||c}
\(P\) & \(\non P\) \\
\hline
V & F \\
F & V
\end{tabular}
\end{itemize}
\end{defi}

\begin{rem}[Contraposition]
Soient \(P\) et \(Q\) deux propositions. Les propositions \(P\imp Q\) et \(\non Q\imp\non P\) sont équivalentes.
\end{rem}

\begin{dem}[Méthode 1]
Il suffit de remarquer que les deux propositions ont la même table de vérité :

\begin{tabular}{c|c||c||c|c||c}
\(P\) & \(Q\) & \(P\imp Q\) & \(\non Q\) & \(\non P\) & \(\non Q\imp\non Q\) \\
\hline
V & V & V & F & F & V \\
V & F & F & V & F & F \\
F & V & V & F & V & V \\
F & F & V & V & V & V
\end{tabular}

~
\end{dem}

\begin{dem}[Méthode 2]
Montrons que \(\paren{P\imp Q}\ssi\paren{\non Q\imp\non P}\) :
\begin{itemize}
\item[\impdir] Supposons \(P\imp Q\). Montrons que \(\non Q\imp\non P\).

Supposons \(\non Q\).

Si \(P\) était vraie, \(Q\) serait vraie aussi. Donc \(P\) est fausse. Donc \(\non P\).

D'où \(\non Q\imp\non P\).

\item[\imprec] Supposons \(\non Q\imp\non P\).

D'après \impdir on a \(\non\paren{\non P}\imp\non\paren{\non Q}\), c'est à dire \(P\imp Q\).
\end{itemize}
\end{dem}

\begin{defi}
Soient \(P\) et \(Q\) deux propositions. Démontrer l'implication \(P\imp Q\) par contraposition, c'est démontrer l'implication contraposée \(\non Q\imp\non P\), c'est à dire supposer \(\non Q\) et montrer \(\non P\).
\end{defi}

\begin{defi}
CS : condition suffisante ; CN : condition nécessaire ; CNS : condition nécessaire et suffisante.
\end{defi}

\begin{ex}
CS pour avoir \(x\geq0\) : \(x=10\) ; CN pour avoir \(x\geq\) : \(x\geq-1\) ; CNS pour avoir \(x\geq0\) : \(x+1\geq1\).
\end{ex}

\begin{rem}
Ne pas utiliser les symboles \(\imp\) et \(\ssi\) comme des abréviations dans du texte en français.
\end{rem}

\begin{ex}
Soit \(x\in\Rp\). Montrons que \(x+1\geq1\).

Écrire \guillemets{On a \(x\geq0\ssi x+1\geq1\)} est faux.

Écrire \guillemets{On a \(x\geq0\) donc \(x+1\geq1\)} est correct.
\end{ex}

\begin{rem}
Abréviations autorisées : CS, CN, CNS et ssi.
\end{rem}

\section{Quantificateurs}

\begin{nota}
\(\in\) : \guillemets{appartient}

\(\exists\) : \guillemets{il existe ... tel que} (quantificateur existentiel)

\(\forall\) : \guillemets{pour tout} (quantificateur universel)

\(\exists!\) : \guillemets{il existe un unique ... tel que}
\end{nota}

\begin{ex}
\(\quantifs{\forall a,b\in\R}\paren{a+b}^2=a^2+2ab+b^2\)

\(\quantifs{\forall x\in\Rp;\exists y\in\R}x=y^2\)

\(\quantifs{\forall x\in\Rp;\exists!y\in\Rp}x=y^2\)
\end{ex}

\begin{rem}
Les quantificateurs se lisent de gauche à droite et leur ordre est important.
\end{rem}

\begin{ex}
On associe à tous réels \(x,y\) une proposition \(P\paren{x,y}\).

Alors \(\quantifs{\exists x\in\R;\forall y\in\R}P\paren{x,y}\imp\quantifs{\forall y\in\R;\exists x\in\R}P\paren{x,y}\).

Dans la proposition de gauche, le \(x\) est valable pour tout \(y\). Dans la proposition de droite, le \(x\) dépend du \(y\).

L'implication réciproque est généralement fausse.

Par exemple, la proposition \(\quantifs{\exists x\in\R;\forall y\in\R}y=x+1\) est fausse mais la proposition \(\quantifs{\forall y\in\R;\exists x\in\R}y=x+1\) est vraie.
\end{ex}

\begin{defi}[Produit cartésien d'ensembles]
Soient \(A,B\) deux ensembles. On note \(A\times B\) l'ensemble des couples de la forme \(\paren{x,y}\) où \(x\in A\) et \(y\in B\).

Plus généralement, soient \(A_1,\dots,A_n\) des ensembles avec \(n\in\Ns\). \(A_1\times\dots\times A_n\) est l'ensemble des n-uplets de la forme \(\paren{x_1,\dots,x_n}\) où \(x_1\in A_1,\dots,x_n\in A_n\).

Soient \(C\) un ensemble et \(m\in\Ns\). On pose \(C^m=\underbrace{C\times\dots\times C}_\text{\(m\) facteurs}\).

On s'autorise alors les identifications suivantes : \(A\times B\times C=A\times\paren{B\times C}=\paren{A\times B}\times C\) et \(\paren{x,y,z}=\paren{x,\paren{y,z}}=\paren{\paren{x,y},z}\).
\end{defi}

\begin{ex}
Dessiner les ensembles \(A=\intervii{1}{2}\times\Rp\), \(B=\N\times\R\) et \(C=\paren{\intervii{0}{1}\cup\accol{2}}\times\Z\).

On a \(\quantifs{\forall\paren{x,y}\in\R^2}\paren{x,y}\in A\ssi\begin{dcases}x\in\intervii{1}{2} \\ y\in\Rp\end{dcases}\) donc :

\begin{center}
\begin{tkz}
\draw[->,gray] (-1,0) -- (3,0);
\draw[->,gray] (0,-1) -- (0,3);
\node[below left] at (0,0) {\(0\)};
\node[below] at (1,0) {\(1\)};
\node[below] at (2,0) {\(2\)};
\draw[blue] (1,0) -- (1,3);
\draw[blue] (2,0) -- (2,3);
\fill[pattern=north east lines,pattern color=blue] (1,0) -- (1,3) -- (2,3) node[below right,blue] {\(A\)} -- (2,0);
\end{tkz}
\end{center}

De même, on a \(\quantifs{\forall\paren{x,y}\in\R^2}\paren{x,y}\in B\ssi\begin{dcases}x\in\N \\ y\in\R\end{dcases}\) donc :

\begin{center}
\begin{tkz}
\draw[->,gray] (-1,0) -- (3,0);
\draw[->,gray] (0,-3) -- (0,3);
\node[below left] at (0,0) {\(0\)};
\node[below left] at (1,0) {\(1\)};
\node[below left] at (2,0) {\(2\)};
\draw[blue] (0,-3) -- (0,3);
\draw[blue] (1,-3) -- (1,3) node[above right] {\(B\)};
\draw[blue] (2,-3) -- (2,3);
\draw[blue] (3,-3) -- (3,3);
\end{tkz}
\end{center}

Enfin, on a \(\quantifs{\forall\paren{x,y}\in\R^2}\paren{x,y}\in C\ssi\begin{dcases}x\in\intervii{0}{1}\cup\accol{2} \\ y\in\Z\end{dcases}\) donc :

\begin{center}
\begin{tkz}
\draw[->,gray] (-1,0) -- (3,0);
\draw[->,gray] (0,-3) -- (0,3);
\node[below left] at (0,0) {\(0\)};
\node[below] at (1,0) {\(1\)};
\node[below] at (2,0) {\(2\)};
\draw[blue] (0,3) -- (1,3);
\draw[blue] (0,2) -- (1,2);
\draw[blue] (0,1) -- (1,1);
\draw[blue] (0,0) -- (1,0);
\draw[blue] (0,-1) -- (1,-1);
\draw[blue] (0,-2) -- (1,-2);
\draw[blue] (0,-3) -- (1,-3);
\filldraw[blue] (2,3) circle (3pt) node[above left,blue] {\(C\)};
\filldraw[blue] (2,2) circle (3pt);
\filldraw[blue] (2,1) circle (3pt);
\filldraw[blue] (2,0) circle (3pt);
\filldraw[blue] (2,-1) circle (3pt);
\filldraw[blue] (2,-2) circle (3pt);
\filldraw[blue] (2,-3) circle (3pt);
\end{tkz}
\end{center}
\end{ex}

\begin{rem}
Les notations \guillemets{\(\forall x,y\in\R\)} et \guillemets{\(\forall\paren{x,y}\in\R^2\)} sont équivalentes.

En revanche, les produits cartésiens sont nécessaires pour le quantificateur \(\exists!\). En effet, on a \(\quantifs{\exists!\paren{x,y}\in\R^2}P\paren{x,y}\imp\quantifs{\exists! x\in\R;\exists! y\in\R}P\paren{x,y}\).
\end{rem}

\section{Raisonnements par analyse-synthèse}

Ils sont utiles pour trouver toutes les solutions à un problème.

\begin{ex}
Soit \(f:\R\to\R\). Montrons que \(f\) s'écrit de façon unique comme la somme d'une fonction paire et d'une fonction impaire. Autrement dit, en notant \(E_0\) l'ensemble des fonctions paires de \(\R\) dans \(\R\) et \(E_1\) l'ensemble des fonctions impaires de \(\R\) dans \(\R\), on a \(\quantifs{\exists!\paren{g,h}\in E_0\times E_1}f=g+h\).

\analyse

Soit \(\paren{g,h}\in E_0\times E_1\) tel que \(f=g+h\).

On a \(\quantifs{\forall x\in\R}\begin{dcases}f\paren{x}=g\paren{x}+h\paren{x} \\ f\paren{-x}=g\paren{x}-h\paren{x}\end{dcases}\)

Donc par somme et différence, \(\quantifs{\forall x\in\R}\begin{dcases}f\paren{x}+f\paren{-x}=2g\paren{x} \\ f\paren{x}-f\paren{-x}=2h\paren{x}\end{dcases}\)

Donc \(\quantifs{\forall x\in\R}\begin{dcases}g\paren{x}=\dfrac{f\paren{x}+f\paren{-x}}{2} \\ h\paren{x}=\dfrac{f\paren{x}-f\paren{-x}}{2}\end{dcases}\)

\synthese

On définit les fonctions \(\fonction{g}{\R}{\R}{x}{\dfrac{f\paren{x}+f\paren{-x}}{2}}\) et \(\fonction{h}{\R}{\R}{x}{\dfrac{f\paren{x}-f\paren{-x}}{2}}\).

On remarque \(\quantifs{\forall x\in\R}\begin{dcases}g\paren{-x}=\dfrac{f\paren{-x}+f\paren{-\paren{-x}}}{2}=\dfrac{f\paren{-x}+f\paren{x}}{2}=g\paren{x} \\ h\paren{-x}=\dfrac{f\paren{-x}-f\paren{-\paren{-x}}}{2}=\dfrac{f\paren{-x}-f\paren{x}}{2}=-h\paren{x}\end{dcases}\)

Donc \(g\) est paire et \(h\) est impaire.

Et \(\quantifs{\forall x\in\R}g\paren{x}+h\paren{x}=\dfrac{f\paren{x}+f\paren{-x}}{2}+\dfrac{f\paren{x}-f\paren{-x}}{2}=f\paren{x}\).

\conclusion

Le seul couple \(\paren{g,h}\in E_0\times E_1\) est \(\paren{\fonctionlambda{\R}{\R}{x}{\dfrac{f\paren{x}+f\paren{-x}}{2}},\fonctionlambda{\R}{\R}{x}{\dfrac{f\paren{x}-f\paren{-x}}{2}}}\).
\end{ex}

\section{Congruences}

\begin{defi}
Soient \(x,y\in\R\) et \(T\in\Rps\).

On dit que \(x\) et \(y\) sont congrus modulo \(T\) si on a \(\quantifs{\exists k\in\Z}x=y+kT\).

On note alors \(x\equiv y\croch{T}\).
\end{defi}

\begin{rem}
Soient \(x,y,z\in\R\) et \(T\in\Rps\). On a \begin{itemize}
\item \(x\equiv y\croch{T}\ssi y\equiv x\croch{T}\) : symétrie

\item \(x\equiv x\croch{T}\) : réflexivité

\item \(\begin{dcases}x\equiv y\croch{T} \\ y\equiv z\croch{T}\end{dcases}\imp x\equiv z\croch{T}\) : transitivité
\end{itemize}
\end{rem}

\chapter{Inégalités, calculs}

\minitoc

\section{Inégalités dans \(\R\)}

\subsection{Parties de \(\R\)}

\begin{defi}
Soient \(A\subset\R\) et \(m,M\in\R\).

On dit que \(M\) est un majorant de \(A\) et que \(M\) majore \(A\) si on a \(\quantifs{\forall x\in A}x\leq M\).

On dit que \(m\) est un minorant de \(A\) et que \(m\) minore \(A\) si on a \(\quantifs{\forall x\in A}m\leq x\).

On dit que la partie \(A\) est majorée si elle admet un majorant.

On dit que la partie \(A\) est minorée si elle admet un minorant.

On dit que la partie \(A\) est bornée si elle admet un majorant et un minorant.

Ainsi, \begin{itemize}
\item \(A\) est majorée \(\ssi\quantifs{\exists\lambda\in\R;\forall x\in A}x\leq\lambda\)
\item \(A\) est minorée \(\ssi\quantifs{\exists\mu\in\R;\forall x\in A}\mu\leq x\)
\item \(A\) est bornée \(\ssi\quantifs{\exists\lambda,\mu\in\R;\forall x\in A}\mu\leq x\leq\lambda\ssi\quantifs{\exists\lambda\in\Rp;\forall x\in A}\abs{x}\leq x\)
\end{itemize}
\end{defi}

\begin{ex}
\begin{itemize}
\item \(\intervii{0}{1}\) est bornée (majorants possibles : \(1\), \(\pi\), ... et minorants possibles : \(0\), \(-10\), ...).

\item \(\N\) est minorée par \(0\) mais non-majorée et donc non-bornée.

\item \(\Z\) n'est ni majorée ni minorée et donc non-bornée.
\end{itemize}
\end{ex}

\begin{rem}
Il n'y a jamais unicité du majorant ou du minorant.
\end{rem}

\begin{defi}
Soient \(A\subset\R\) et \(a\in A\).

On dit que \(a\) est le plus grand élément de \(A\) (ou maximum de \(A\)) si on a \(\quantifs{\forall b\in A}b\leq a\).

On dit que \(a\) est le plus petit élément de \(A\) (ou minimum de \(A\)) si on a \(\quantifs{\forall b\in A}a\leq b\).
\end{defi}

\begin{prop}
S'il existe, le plus grand élément de \(A\) est unique et est noté \(\max A\).

S'il existe, le plus petit élément de \(A\) est unique et est noté \(\min A\).
\end{prop}

\begin{dem}
Montrons l'unicité du plus grand élément de \(A\).

Soient \(a_1,a_2\in A\) tels que \(\quantifs{\forall b\in A}\begin{dcases}b\leq a_1 \\ b\leq a_2\end{dcases}\)

On a en particulier \(\begin{dcases}a_1\leq a_2\text{ avec \(b=a_1\)} \\ a_2\leq a_1\text{ avec \(b=a_2\)}\end{dcases}\)

Donc \(a_1=a_2\). Donc le plus grand élément est unique.

On montre de même l'unicité du plus petit élément.
\end{dem}

\begin{rem}[Plus grand élément \(\imp\) majorant]
Soient \(A\subset\R\) et \(x\in\R\).

\(x\) est le plus grand élément de \(A\) ssi \(\begin{dcases}x\in A \\ x\text{ majore \(A\)}\end{dcases}\)

En particulier, pour que \(A\) admette un plus grand élément, il faut que \(A\) soit majorée.
\end{rem}

\begin{ex}
\begin{itemize}
\item \(\intervii{0}{1}\) admet \(1\) comme plus grand élément et \(0\) comme plus petit élément.

\item \(\Rp\) admet \(0\) comme plus petit élément mais n'admet pas de plus grand élément (car \(\Rp\) n'est pas majorée).

\item \(\intervie{0}{1}\) admet \(0\) comme plus petit élément et est majorée (par \(1\)) mais n'admet pas de plus grand élément.

En effet, par l'absurde :

Soit \(a\in\intervie{0}{1}\). Supposons que \(a\) est le plus grand élément de \(\intervie{0}{1}\).

Comme \(a\in\intervie{0}{1}\), on a \(0\leq a<1\).

Posons \(b=\dfrac{a+1}{2}\).

On a d'une part \(0\leq a+1<2\) donc \(0\leq\dfrac{a+1}{2}<1\) donc \(0\leq b<1\) donc \(b\in\intervie{0}{1}\).

D'autre part, \(a+a<a+1\) donc \(\dfrac{a+a}{2}<\dfrac{a+1}{2}\) donc \(a<b\).

Donc \(a\) ne majore pas \(\intervie{0}{1}\) : contradiction.
\end{itemize}
\end{ex}

\begin{rem}
Toute partie finie admet un plus grand élément.
\end{rem}

\begin{theo}
Toute partie non-vide de \(\N\) admet un plus petit élément.
\end{theo}

\begin{dem}
\note{ADMIS} (fait partie de la définition de \(\N\), hors programme).
\end{dem}

\begin{cor}
Toute partie non-vide de \(\Z\) minorée admet un plus petit élément.

Toute partie non-vide de \(\Z\) majorée admet un plus grand élément.
\end{cor}

\begin{dem}
Soit \(A\subset\Z\) non-vide et minorée.

Soit \(m\in\R\) un minorant de \(A\).

Soit \(m\prim\in\Z\) tel que \(m\prim\leq m\).

On a \(\quantifs{\forall x\in A}m\prim\leq m\leq x\).

Donc \(m\prim\) minore \(A\).

Posons \(B=\accol{x-m\prim}_{x\in A}\).

On remarque \(B\not=\ensvide\) car \(A\not=\ensvide\) et \(B\subset\N\) car \(\quantifs{\forall x\in A}\begin{dcases}x-m\prim\in\Z\text{ car \(x,m\prim\in\Z\)} \\ x-m\prim\geq0\text{ car \(m\prim\leq x\)}\end{dcases}\)

Ainsi, \(B\) est une partie non-vide de \(\N\).

Donc \(B\) admet un plus petit élément \(b_0\in B\) : \(\quantifs{\forall b\in B}b_0\leq b\).

Cet élément \(b_0\) s'écrit \(b_0=a_0-m\prim\) par définition de \(B\).

On a \(\quantifs{\forall x\in A}a_0-m\prim\leq x-m\prim\) donc \(\quantifs{\forall x\in A}a_0\leq x\).

Donc \(a_0=\min A\).

On montre de même que toute partie \(C\) majorée non-vide de \(\Z\) admet un maximum, en considérant un majorant \(M\in\Z\) de \(C\) puis l'ensemble \(D=\accol{M-x}_{x\in C}\).
\end{dem}

\begin{defi}[Intervalle]
Soit \(I\subset\R\).

On dit que \(I\) est un intervalle de \(\R\) si on a \(\quantifs{\forall x,y\in I;\forall z\in\R}x\leq z\leq y\imp z\in I\).

Ie : \(\quantifs{\forall x,y\in I}\intervii{x}{y}\subset I\), en notant \(\intervii{x}{y}=\accol{z\in\R\tq x\leq z\leq y}\) avec donc \(\intervii{x}{y}=\ensvide\) si \(x>y\).
\end{defi}

\begin{ex}
\begin{itemize}
\item \(\ensvide\) et \(\R\) sont des intervalles de \(\R\).

\item \(\intervii{a}{b}\), \(\intervie{a}{b}\), \(\intervee{a}{+\infty}\), ... sont des intervalles de \(\R\) pour tous \(a,b\in\R\).

\item \(\N\), \(\Z\), \(\Q\), ... ne sont pas des intervalles de \(\R\). Par exemple, on a \(0\leq\dfrac{1}{2}\leq1\) mais \(\dfrac{1}{2}\not\in\N\).
\end{itemize}
\end{ex}

\subsection{Manipulation d'inégalités, fonctions}

\begin{prop}
Soit \(I\) un intervalle de \(\R\).

Soit \(f:I\to\R\) dérivable.

\begin{itemize}
\item \(f\) est constante \(\ssi\quantifs{\forall x\in I}f\prim\paren{x}=0\)

\item \(f\) est croissante \(\ssi\quantifs{\forall x\in I}f\prim\paren{x}\geq0\)

\item \(f\) est strictement croissante \(\impr\quantifs{\forall x\in I}f\prim\paren{x}>0\)

\(f\) est strictement croissante \(\ssi\begin{dcases}\quantifs{\forall x\in I}f\prim\paren{x}\geq0 \\ \quantifs{\forall x,y\in I}x<y\imp\quantifs{\exists z\in\intervii{x}{y}}f\prim\paren{z}>0\end{dcases}\)
\end{itemize}
\end{prop}

\begin{dem}
\note{ADMIS} temporairement.
\end{dem}

\begin{rem}
Il ne faut pas oublier l'hypothèse selon laquelle \(I\) est un intervalle.

Par exemple, si \(I=\Rs\) alors \(I\) n'est pas un intervalle de \(\R\) (on a par exemple \(-1\leq0\leq1\) mais \(0\not\in\Rs\)) et les fonctions \(\fonction{f}{\Rs}{\R}{x}{\begin{dcases}1&\text{ si \(x>0\)} \\ -1&\text{ si \(x<0\)}\end{dcases}}\) et \(\fonction{g}{\Rs}{\R}{x}{\dfrac{1}{x}}\) vérifient : \begin{itemize}
\item \(f\prim=0\) mais \(f\) non-constante

\item \(\quantifs{\forall x\in\Rs}g\prim\paren{x}=\dfrac{-1}{x^2}<0\) mais \(g\) n'est pas strictement décroissante.
\end{itemize}
\end{rem}

\begin{rem}
Soient \(A\subset\R\) et \(f:A\to\R\). On considère les propositions suivantes :

\begin{enumerate}
\item \(f\) croissante

\item \(f\) strictement croissante

\item \(\quantifs{\forall x,y\in A}x\leq y\imp f\paren{x}\leq f\paren{y}\)

\item \(\quantifs{\forall x,y\in A}x<y\imp f\paren{x}<f\paren{y}\)

\item \(\quantifs{\forall x,y\in A}x\leq y\ssi f\paren{x}\leq f\paren{y}\)

\item \(\quantifs{\forall x,y\in A}x<y\ssi f\paren{x}<f\paren{y}\)
\end{enumerate}

Les propositions (1) et (3) sont équivalentes.

Les propositions (2), (4), (5) et (6) sont équivalentes.
\end{rem}

\begin{rem}
Quand on rédige un raisonnement par équivalences, la façon correcte d'écrire est la suivante : \guillemets{\(x\leq y\ssi f\paren{x}\leq f\paren{y}\) car \(f\) est strictement croissante}.
\end{rem}

\begin{dem}
\begin{itemize}
\item (1) \(\ssi\) (3) : par définition

\item (2) \(\ssi\) (4) : par définition

\item (6) \(\ssi\) (5) : par contraposition

\item (6) \(\imp\) (4) : clair

\item Il ne reste plus qu'à montrer que (4) \(\imp\) (6).

Supposons (4), montrons (6).

Soient \(x,y\in A\). Montrons que \(x<y\ssi f\paren{x}<f\paren{y}\).

\begin{itemize}
\item[\impdir] Ok selon (4)

\item[\imprec] Supposons \(f\paren{x}<f\paren{y}\). Montrons que \(x<y\).

Si \(x=y\) alors \(f\paren{x}=f\paren{y}\) : impossible.

Si \(x>y\) alors \(f\paren{x}>f\paren{y}\) selon (4) : impossible.

Donc \(x<y\), d'où (6).
\end{itemize}
\end{itemize}
\end{dem}

\begin{prop}
Soient \(a,b,c,d,\lambda\in\R\).

On a \begin{enumerate}
\item \(\begin{dcases}a\leq b \\ c\leq d\end{dcases}\imp a+c\leq b+d\)

\item \(\begin{dcases}a\leq b \\ \lambda\geq0\end{dcases}\imp\lambda a\leq\lambda b\)

\item \(\begin{dcases}a\leq b \\ \lambda\leq0\end{dcases}\imp\lambda a\geq\lambda b\)

\item \(\begin{dcases}0\leq a\leq b \\ 0\leq c\leq d\end{dcases}\imp ac\leq bd\)

\item \(\begin{dcases}a\leq b \\ ab>0\end{dcases}\imp\dfrac{1}{a}\geq\dfrac{1}{b}\)
\end{enumerate}
\end{prop}

\begin{dem}
\begin{enumerate}
\item Supposons \(a\leq b\) et \(c\leq d\).

On a \(a+c\leq b+c\) car \(x\mapsto x+c\) croissante.

Donc \(a+c\leq b+d\) car \(c\leq d\).

\item Supposons \(\lambda\geq0\).

La fonction \(x\mapsto\lambda x\) est de dérivée positive sur l'intervalle \(\R\) donc elle est croissante.

\item Idem.

\item Idem.

\item C'est la décroissance de la fonction inverse sur les intervalles \(\Rps\) et \(\Rms\) (en effet, \(ab>0\ssi\paren{a,b\in\Rps\text{ ou }a,b\in\Rms}\)).
\end{enumerate}
\end{dem}

\subsection{Valeur absolue}

\begin{nota}
On pose \(\quantifs{\forall x\in\R}\abs{x}=\max\accol{x;-x}=\begin{dcases}x&\text{ si \(x\geq0\)} \\ -x&\text{ si \(x<0\)}\end{dcases}\)
\end{nota}

\begin{rem}
On a \(\quantifs{\forall x\in\Rp}\sqrt{x}^2=x\) et \(\quantifs{\forall x\in\R}\sqrt{x^2}=\abs{x}\).

Ainsi, on a \(\quantifs{\forall x,y\in\R}\begin{dcases}x^2=y^2\ssi\abs{x}=\abs{y} \\ x^2\leq y^2\ssi\abs{x}\leq\abs{y}\text{ car \(x\mapsto\sqrt{x}\) strictement croissante}\end{dcases}\)
\end{rem}

\begin{prop}[Inégalité triangulaire]
Soient \(x,y\in\R\).

On a \(\abs{\abs{x}-\abs{y}}\leq\abs{x+y}\leq\abs{x}+\abs{y}\).

Et donc \(\abs{\abs{x}-\abs{y}}\leq\abs{x-y}\leq\abs{x}+\abs{y}\) en remplaçant \(y\) par \(-y\).
\end{prop}

\begin{dem}
\begin{enumerate}
\item Montrons que \(\abs{x+y}\leq\abs{x}+\abs{y}\).

Si \(x\geq0\) et \(y\geq0\) alors \(x+y\geq0\) donc \(\abs{x+y}=x+y=\abs{x}+\abs{y}\).

Si \(x<0\) et \(y<0\) alors \(x+y<0\) donc \(\abs{x+y}=-x-y=\abs{x}+\abs{y}\).

Si \(x\geq0\) et \(y<0\) alors \(\begin{dcases}x+y\leq x-y=\abs{x}+\abs{y}\text{ car \(y<0\)} \\ -x-y\leq x-y=\abs{x}+\abs{y}\text{ car \(x\geq0\)}\end{dcases}\) donc \(\abs{x+y}=\max\accol{x+y;-x-y}\leq\abs{x}+\abs{y}\).

Si \(x<0\) et \(y\geq0\) : idem en échangeant \(x\) et \(y\).

\item On remarque \(\begin{aligned}[t]
\abs{x} &= \abs{x+y-y} \\
&\leq \abs{x+y}+\abs{-y}\text{ selon (1)} \\
&= \abs{x+y}+\abs{y}
\end{aligned}\)

Donc \(\abs{x}-\abs{y}\leq\abs{x+y}\).

De même, \(\abs{y}=\abs{y+x-x}\leq\abs{y+x}+\abs{x}\).

Donc \(\abs{y}-\abs{x}\leq\abs{y+x}\).

Finalement, \(\abs{\abs{x}-\abs{y}}=\max\accol{\abs{x}-\abs{y};\abs{y}-\abs{x}}\leq\abs{x+y}\).
\end{enumerate}
\end{dem}

\begin{prop}
Soient \(x,y\in\R\).

On a \(\abs{xy}\leq\dfrac{x^2+y^2}{2}\).
\end{prop}

\begin{dem}
On a \(\paren{x-y}^2\geq0\) et \(\paren{x+y}^2\geq0\).

Donc \(\begin{dcases}xy\leq\dfrac{x^2+y^2}{2} \\ -xy\leq\dfrac{x^2+y^2}{2}\end{dcases}\)

D'où \(\abs{xy}=\max\accol{xy;-xy}\leq\dfrac{x^2+y^2}{2}\).
\end{dem}

\subsection{Partie entière d'un réel}

\begin{defi}
Soit \(x\in\R\).

On appelle partie entière de \(x\) et on note \(\floor{x}\) le plus grand entier relatif \(n\in\Z\) tel que \(n\leq x\).

Un tel entier existe car \(\accol{n\in\Z\tq n\leq x}\) est une partie non-vide et majorée de \(\Z\).
\end{defi}

\begin{ex}
\(\quantifs{\forall n\in\Z}\floor{n}=n\) ; \(\floor{\dfrac{1}{2}}=1\) ; \(\floor{-\dfrac{1}{2}}=-2\).
\end{ex}

\begin{prop}
Soit \(x\in\R\).

On a \begin{enumerate}
\item \(\floor{x}\leq x<\floor{x}+1\)

\item \(\quantifs{\forall n\in\Z}n\leq x\ssi n\leq\floor{x}\)
\end{enumerate}
\end{prop}

\begin{dem}
\begin{enumerate}
\item On a \(\floor{x}\leq x\) par définition.

De plus, \(\floor{x}\) est le plus grand entier inférieur à \(x\) donc \(\floor{x}+1\) n'est pas un entier inférieur à \(x\).

\item Soit \(n\in\Z\).

\begin{itemize}
\item[\imprec] Claire car \(\floor{x}\leq x\).

\item[\impdir] Si \(n\leq x\) alors \(n\) est plus petit que le plus grand entier inférieur à \(x\).

Donc \(n\leq\floor{x}\).
\end{itemize}
\end{enumerate}
\end{dem}

\section{Sommes et produits}

\begin{nota}
On pose \(\quantifs{\forall a,b\in\Z}\interventierii{a}{b}=\accol{n\in\Z\tq a\leq n\leq b}\).
\end{nota}

\begin{nota}
Soient \(a,b\in\Z\) et \(f:\Z\to\R\).

\begin{itemize}
\item On pose \(\sum_{k=a}^b f\paren{k}=\begin{dcases}f\paren{a}+f\paren{a+1}+\dots+f\paren{b}&\text{ si \(a\leq b\)} \\ 0&\text{ si \(a>b\)}\end{dcases}\)

\item On pose \(\prod_{k=a}^b f\paren{k}=\begin{dcases}f\paren{a}\times f\paren{a+1}\times\dots\times f\paren{b}&\text{ si \(a\leq b\)} \\ 1&\text{ si \(a>b\)}\end{dcases}\)

\item Soit \(E\) un ensemble fini et \(f:E\to\R\).

On note \(\sum_{x\in E}f\paren{x}\) la somme des \(f\paren{x}\) pour \(x\in E\) (\(0\) si \(E=\ensvide\)).

Et \(\prod_{x\in E}f\paren{x}\) le produit des \(f\paren{x}\) pour \(x\in E\) (\(1\) si \(E=\ensvide\)).
\end{itemize}
\end{nota}

\begin{rem}
Les indices \(k\) et \(x\) sont des \guillemets{variables locales}, non définies en dehors des \(\sum\) et \(\prod\).

Ce sont aussi des \guillemets{variables muettes}, la somme ou le produit ne dépend pas du nom de la variable : \(\sum_{k=a}^b f\paren{k}=\sum_{l=a}^b f\paren{l}\).
\end{rem}

\begin{rem}[Changements d'indice]
\begin{itemize}
\item Soit \(n\in\interventierie{2}{\pinf}\).

On a \(\sum_{k=2}^n2^{k-2}\ln\paren{k+1}=\sum_{k=0}^{n-2}2^k\ln\paren{k+3}\).

\item Soit \(\paren{u_k}_{k\in\N}\) une suite de réels.

On a \(\sum_{k=0}^{n-1}u_{1-k}=\sum_{k=1}^n u_k\).
\end{itemize}
\end{rem}

\begin{prop}
Soient \(E\) et \(F\) deux ensembles finis. On suppose que \(E\) et \(F\) sont disjoints (ie \(E\cap F=\ensvide\)).

Soit \(f:E\cup F\to\R\).

Alors \(\sum_{x\in E\cup F}f\paren{x}=\sum_{x\in E}f\paren{x}+\sum_{x\in F}f\paren{x}\).

Et \(\prod_{x\in E\cup F}f\paren{x}=\prod_{x\in E}f\paren{x}\times\prod_{x\in F}f\paren{x}\).
\end{prop}

\begin{prop}[Sommes télescopiques]
Soient \(a,b\in\Z\) tels que \(a\leq b\) et \(\paren{u_n}_{n\in\N}\) une suite de réels.

On a \(\sum_{k=a}^b\paren{u_{k+1}-u_k}=u_{b+1}-u_a\).
\end{prop}

\begin{dem}~\\
On a \(\sum_{k=a}^b\paren{u_{k+1}-u_k}=\sum_{k=a}^b u_{k+1}-\sum_{k=a}^b u_k=\sum_{k=a+1}^{b+1}u_k-\sum_{k=a}^b u_k=u_{b+1}-u_a\).
\end{dem}

\begin{ex}
Soit \(n\in\Ns\). Calculer \(\sum_{k=1}^n\ln\paren{1+\dfrac{1}{k}}\) et \(\sum_{k=1}^n\dfrac{1}{k\paren{k+1}}\).

\begin{itemize}
\item On a \(\begin{aligned}[t]
\sum_{k=1}^n\ln\paren{1+\dfrac{1}{k}}&=\sum_{k=1}^n\ln\dfrac{k+1}{k} \\
&=\sum_{k=1}^n\paren{\ln\paren{k+1}-\ln k} \\
&=\sum_{k=1}^n\ln\paren{k+1}-\sum_{k=1}^n\ln k \\
&=\sum_{k=2}^{n+1}\ln k-\sum_{k=1}^n\ln k \\
&=\ln\paren{n+1}-\ln1 \\
&=\ln\paren{n+1}
\end{aligned}\)

\item On a \(\begin{aligned}[t]
\sum_{k=1}^n\dfrac{1}{k\paren{k+1}}&=\sum_{k=1}^n\dfrac{1+k-k}{k\paren{k+1}} \\
&=\sum_{k=1}^n\paren{\dfrac{1+k}{k\paren{k+1}}-\dfrac{k}{k\paren{k+1}}} \\
&=\sum_{k=1}^n\paren{\dfrac{1}{k}-\dfrac{1}{k+1}} \\
&=\sum_{k=1}^n\dfrac{1}{k}-\sum_{k=1}^n\dfrac{1}{k+1} \\
&=\sum_{k=1}^n\dfrac{1}{k}-\sum_{k=2}^{n+1}\dfrac{1}{k} \\
&=1-\dfrac{1}{n+1} \\
&=\dfrac{n}{n+1}
\end{aligned}\)
\end{itemize}
\end{ex}

\begin{prop}[Produits télescopiques]
Soient \(a,b\in\N\) tels que \(a\leq b\) et \(\paren{u_n}_{n\in\N}\) une suite de réels non-nuls.

Alors, \(\prod_{k=a}^b\dfrac{u_{k+1}}{u_k}=\dfrac{u_{b+1}}{u_a}\).
\end{prop}

\begin{prop}[Doubles sommes]
Soient \(a,b,c,d\in\Z\) et \(f:\Z^2\to\R\).

Alors, \(\sum_{i=a}^b\sum_{j=c}^d f\paren{i,j}=\sum_{\paren{i,j}\in\interventierii{a}{b}\times\interventierii{c}{d}}f\paren{i,j}=\sum_{j=c}^d\sum_{i=a}^b f\paren{i,j}\).

On appelle interversion le passage du membre de gauche au membre de droite.
\end{prop}

\begin{prop}[Produits doubles]
Soient \(a,b,c,d\in\Z\) et \(f:\Z^2\to\R\).

Alors, \(\prod_{i=a}^b\prod_{j=c}^d f\paren{i,j}=\prod_{j=c}^d\prod_{i=a}^b f\paren{i,j}\) (interversion).
\end{prop}

\begin{ex}~\\
On a \(\sum_{i=0}^1\sum_{j=0}^2 j^i=\paren{1+1+1}+\paren{0+1+2}=6\) et \(\sum_{j=0}^2\sum_{i=0}^1 j^i=\paren{1+0}+\paren{1+1}+\paren{1+2}=6\).
\end{ex}

\begin{prop}
Soient \(E\) et \(F\) deux ensembles finis et \(f:E\to\R\) et \(g:F\to\R\).

On a \(\sum_{x\in E}f\paren{x}\times\sum_{y\in F}g\paren{y}=\sum_{x\in E}\sum_{y\in F}f\paren{x}g\paren{y}\).
\end{prop}

\begin{ex}
Soient \(n\in\N\) et \(x_1,\dots,x_n\in\R\).

On a \(\paren{\sum_{k=1}^n x_k}^2=\sum_{k=1}^n x_k^2+2\sum_{1\leq k<l\leq n}x_kx_l\).
\end{ex}

\begin{rem}
La dernière somme pourrait s'écrire \(\sum_{\paren{k,l}\in E}\dots\ \) où \(E=\accol{\paren{k,l}\in\N^2\tq1\leq k<l\leq n}\) ou encore \(\sum_{k=1}^n\sum_{l=k+1}^n\dots\).
\end{rem}

\begin{dem}~\\
On a \(\begin{aligned}[t]
\paren{\sum_{k=1}^n x_k}^2&=\sum_{k=1}^n x_k\times\sum_{l=1}^n x_l \\
&=\sum_{k=1}^n\sum_{l=1}^n x_kx_l \\
&=\sum_{1\leq k=l\leq n}x_kx_l+\sum_{1\leq k<l\leq n}x_kx_l+\sum_{1\leq l<k\leq n}x_kx_l \\
&=\sum_{k=1}^n x_k^2+2\sum_{1\leq k<l\leq n}x_kx_l
\end{aligned}\)

~
\end{dem}

\section{Factorielle, coefficients binomiaux}

\begin{defi}[Factorielle]
Soit \(n\in\N\).

On pose \(n!=\prod_{k=1}^n k\).
\end{defi}

\begin{defi}[Coefficient binomial]
Soient \(n\in\N\) et \(p\in\Z\).

On pose \(\binom{p}{n}=\begin{dcases}\dfrac{n!}{p!\,\paren{n-p}!}&\text{ si \(0\leq p\leq n\)} \\ 0&\text{ si \(0>p\) ou \(p>n\)}\end{dcases}\)

\(\binom{p}{n}\) est le nombre de parties de cardinal \(p\) contenues dans un ensemble de cardinal \(n\).

Supposons \(0\leq p\leq n\). On a \(\dfrac{n!}{\paren{n-p}!}=n\paren{n-1}\dots\paren{n-p+1}\).

Donc \(\binom{p}{n}=\dfrac{n\paren{n-1}\dots\paren{n-p+1}}{p!}\).

On a \(\binom{0}{n}=\dfrac{1}{1}=1\) ; \(\binom{1}{n}=\dfrac{n}{1}=n\) ; \(\binom{2}{n}=\dfrac{n\paren{n-1}}{1\times2}=\dfrac{n\paren{n-1}}{2}\).
\end{defi}

\begin{prop}
Soient \(n\in\N\) et \(p\in\Z\).

On a \begin{enumerate}
\item \(\binom{p}{n}=\binom{n-p}{n}\)

\item \(\binom{p}{n}+\binom{p+1}{n}=\binom{p+1}{n+1}\)

\item Si \(p\not=0\) alors \(\binom{p}{n}=\dfrac{n}{p}\binom{p-1}{n-1}\)
\end{enumerate}
\end{prop}

\begin{dem}
\begin{enumerate}
\item On a \(0\leq p\leq n\ssi0\leq n-p\leq n\).

Si \(p<0\) ou \(n<p\), les deux coefficients binomiaux sont nuls.

Sinon, \(\binom{n-p}{n}=\dfrac{n!}{\paren{n-p}!\,\paren{n-\paren{n-p}}!}=\dfrac{n!}{\paren{n-p}!\,p!}=\binom{p}{n}\).

\item Si \(p<-1\), les trois coefficients binomiaux sont nuls.

Si \(p=-1\), on a bien \(0+1=1\).

Si \(p>n\), les trois coefficients binomiaux sont nuls.

Si \(p=n\), on a bien \(1+0=1\).

Supposons désormais \(0\leq p\leq n-1\).

Les trois coefficients binomiaux sont non-nuls et on a :

\(\begin{aligned}[t]
\binom{p}{n}+\binom{p+1}{n}&=\dfrac{n!}{p!\,\paren{n-p}!}+\dfrac{n!}{\paren{p+1}!\,\paren{n-p+1}!} \\
&=\dfrac{\paren{p+1}n!+\paren{n-p}n!}{\paren{n-p}!\,\paren{p+1}!} \\
&=\dfrac{\paren{n+1}n!}{\paren{n-p}!\,\paren{p+1}!} \\
&=\dfrac{\paren{n+1}!}{\paren{\paren{n+1}-\paren{p+1}}!\,\paren{p+1}!} \\
&=\binom{p+1}{n+1}
\end{aligned}\)

\item Si \(p<0\), on a bien \(0=0\).

Si \(p>n\), idem.

Supposons \(1\leq p\leq n\).

On a \(\binom{p}{n}=\dfrac{n\paren{n-1}\dots\paren{n-p+1}}{p!}=\dfrac{n}{p}\times\dfrac{\paren{n-1}\dots\paren{n-p+1}}{\paren{p-1}!}=\dfrac{n}{p}\binom{p-1}{n-1}\).
\end{enumerate}
\end{dem}

\begin{rem}
Le triangle de Pascal est complété à l'aide de (2) :

\(\begin{array}{c|ccccc}
&0&1&2&3&4 \\
\hline
0&\binom{0}{0}&&&& \\[1em]
1&\binom{0}{1}&\binom{1}{1}&&& \\[1em]
2&\binom{0}{2}&\binom{1}{2}&\binom{2}{2}&& \\[1em]
3&\binom{0}{3}&\binom{1}{3}&\binom{2}{3}&\binom{3}{3}&
\end{array}\) donnant \(\begin{array}{c|ccccc}
&0&1&2&3&4 \\
\hline
0&1&0&0&0&0 \\
1&1&1&0&0&0 \\
2&1&2&1&0&0 \\
3&1&3&3&1&0
\end{array}\)
\end{rem}

\begin{prop}[Formule du binôme de Newton]
Soient \(x,y\in\R\) et \(n\in\N\).

On a \(\paren{x+y}^n=\sum_{k=0}^n\binom{k}{n}x^ky^{n-k}=\sum_{k=0}^n\binom{k}{n}y^kx^{n-k}\)
\end{prop}

\begin{dem}
Par récurrence sur \(n\in\N\).

Pour tout \(n\in\N\), on note \(P\paren{n}\) la proposition \guillemets{\(\paren{x+y}^n=\sum_{k=0}^n\binom{k}{n}x^ky^{n-k}\)}.

\underline{Initialisation :}

On a bien \(\paren{x+y}^0=1\) et \(\sum_{k=0}^0\binom{k}{0}x^ky^{0-k}=\binom{0}{0}x^0y^0=1\).

D'où \(P\paren{0}\).

\underline{Hérédité :}

Soit \(n\in\N\) tel que \(P\paren{n}\). Montrons \(P\paren{n+1}\). On a :

\(\begin{aligned}[t]
\paren{x+y}^{n+1}&=\paren{x+y}\paren{x+y}^n \\
&=\paren{x+y}\sum_{k=0}^n\binom{k}{n}x^ky^{n-k}\text{ selon \(P\paren{n}\)} \\
&=\sum_{k=0}^n\binom{k}{n}x^{k+1}y^{n-k}+\sum_{k=0}^n\binom{k}{n}x^ky^{n-k+1} \\
&=\sum_{k=1}^{n+1}\binom{k-1}{n}x^ky^{n-k+1}+\sum_{k=0}^n\binom{k}{n}x^ky^{n-k+1} \\
&=\sum_{k=0}^{n+1}\binom{k-1}{n}x^ky^{n-k+1}+\sum_{k=0}^{n+1}\binom{k}{n}x^ky^{n-k+1}\text{ car \(\binom{-1}{n}=\binom{n+1}{n}=0\)} \\
&=\sum_{k=0}^{n+1}\paren{\binom{k-1}{n}+\binom{k}{n}}x^ky^{n-k+1} \\
&=\sum_{k=0}^{n+1}\binom{k}{n+1}x^ky^{n-k+1}
\end{aligned}\)

D'où \(P\paren{n+1}\).

\underline{Conclusion :}

On a \(\quantifs{\forall n\in\N}P\paren{n}\).
\end{dem}

\chapter{Révisions de trigonométrie}

\minitoc

\section{Formules}

\subsection{Propriétés fondamentales}

Les fonctions \(\cos\) et \(\sin\) sont définies sur \(\R\) et vérifient \(\cos^2+\sin^2=1\).

La fonctions \(\tan\) est définie en tout réel \(x\) tel que \(x\not\equiv\dfrac{\pi}{2}\croch{\pi}\).

Ces trois fonctions sont dérivables et on a : \(\cos\prim=-\sin\) ; \(\sin\prim=\cos\) ; \(\tan\prim=1+\tan^2=\dfrac{1}{\cos^2}\).

\(\quantifs{\forall a,b\in\R}\cos\paren{a+b}=\cos a\cos b-\sin a\sin b\)

\(\quantifs{\forall a,b\in\R}\sin\paren{a+b}=\sin a\cos b+\sin b\cos a\)

\subsection{Conséquences}

\(\quantifs{\forall a,b\in\R}\begin{dcases}a\not\equiv\dfrac{\pi}{2}\croch{\pi} \\ b\not\equiv\dfrac{\pi}{2}\croch{\pi} \\ a+b\not\equiv\dfrac{\pi}{2}\croch{\pi}\end{dcases}\imp\tan\paren{a+b}=\dfrac{\tan a+\tan b}{1-\tan a\tan b}\)

\(\quantifs{\forall a\in\R}\cos\paren{2a}=2\cos^2a-1\) et \(\cos^2a=\dfrac{1+\cos\paren{2a}}{2}\) et \(\sin^2a=\dfrac{1-\cos\paren{2a}}{2}\)

\(\quantifs{\forall a\in\R}\sin\paren{2a}=2\sin a\cos a\)

\(\quantifs{\forall a\in\R}\begin{dcases}a\not\equiv\dfrac{\pi}{2}\croch{\pi} \\ a\not\equiv\dfrac{\pi}{4}\croch{\dfrac{\pi}{2}}\end{dcases}\imp\tan\paren{2a}=\dfrac{2\tan a}{1-\tan^2a}\)

\(\quantifs{\forall a,b\in\R}\cos a\cos b=\dfrac{\cos\paren{a+b}+\cos\paren{a-b}}{2}\)

\(\quantifs{\forall a,b\in\R}\sin a\sin b=\dfrac{-\cos\paren{a+b}+\cos\paren{a-b}}{2}\)

\(\quantifs{\forall a,b\in\R}\sin a\cos b=\dfrac{\sin\paren{a+b}+\sin\paren{a-b}}{2}\)

\(\quantifs{\forall p,q\in\R}\cos p+\cos q=2\cos\dfrac{p+q}{2}\cos\dfrac{p-q}{2}\)

\(\quantifs{\forall p,q\in\R}\sin p+\sin q=2\sin\dfrac{p+q}{2}\cos\dfrac{p-q}{2}\)

\(\quantifs{\forall p,q\in\R}\cos p-\cos q=-2\sin\dfrac{p+q}{2}\sin\dfrac{p-q}{2}\)

Soit \(\theta\in\R\) tel que \(\theta\not\equiv\pi\croch{2\pi}\). On pose \(t=\tan\dfrac{\theta}{2}\). On a \(\begin{dcases}\cos\theta=\dfrac{1-t^2}{1+t^2} \\ \sin\theta=\dfrac{2t}{1+t^2}\end{dcases}\)

\subsection{Lien avec les nombres complexes}

\(\quantifs{\forall\theta\in\R}\e{\i\theta}=\cos\theta+\i\sin\theta\)

\(\quantifs{\forall\theta\in\R}\cos\theta=\dfrac{\e{\i\theta}+\e{-\i\theta}}{2}\) et \(\sin\theta=\dfrac{\e{\i\theta}-\e{-\i\theta}}{2}\) (formules d'Euler)

\section{Démonstrations et compléments}

Cercle trigonométrique : \(\accol{\paren{x,y}\in\R^2\tq x^2+y^2=1}=\accol{\paren{\cos\theta,\sin\theta}}_{\theta\in\R}\).

\begin{rem}
L'égalité des ensembles signifie :

\begin{itemize}
\item[\increc] \(\quantifs{\forall\theta\in\R}\cos^2\theta+\sin^2\theta=1\)

\item[\incdir] \(\quantifs{\forall\paren{x,y}\in\R^2}x^2+y^2=1\imp\quantifs{\exists\theta\in\R}\paren{x,y}=\paren{\cos\theta,\sin\theta}\)
\end{itemize}
\end{rem}

On a le cercle trigonométrique suivant :

\begin{center}
\begin{tkz}[scale=2]
\draw[->,gray] (0,-1.5) -- (0,1.5);
\draw[->,gray] (-1.5,0) -- (1.5,0) coordinate (A);
\node[below left] at (0,0) {\(0\)};
\draw (0,0) circle (1);
\draw (1,-1.5) node[right] {droite verticale tangente au cercle} -- (1,1.5);
\draw (0,0) coordinate (O) -- (0.707,0.707) coordinate (Theta);
\draw[dotted] (0.707,0.707) -- (0.707,0) node[below] {\(\cos\theta\)};
\draw[dotted] (0.707,0.707) -- (0,0.707) node[left] {\(\sin\theta\)};
\pic[draw,->,"\(\theta\)",angle eccentricity=1.5] {angle = A--O--Theta};
\draw[dotted] (0.707,0.707) -- (1,1) node[right] {\(\tan\theta\)};
\end{tkz}
\end{center}

On a \(\dfrac{\tan\theta}{1}=\dfrac{\sin\theta}{\cos\theta}\) par le théorème de Thalès.

Valeurs remarquables :

\begin{align*}
\begin{array}{cccccc}
\theta & 0 & \dfrac{\pi}{6} & \dfrac{\pi}{4} & \dfrac{\pi}{3} & \dfrac{\pi}{2} \\[1em]
\hline \\
\sin\theta & 0 & \dfrac{1}{2} & \dfrac{\sqrt{2}}{2} & \dfrac{\sqrt{3}}{2} & 1 \\[1em]
\hline \\
\cos\theta & 1 & \dfrac{\sqrt{3}}{2} & \dfrac{\sqrt{2}}{2} & \dfrac{1}{2} & 0 \\[1em]
\hline \\
\tan\theta & 0 & \dfrac{1}{\sqrt{3}} & 1 & \sqrt{3} & \text{indéfini}
\end{array}
\end{align*}

On a \(\quantifs{\forall k\in\Z}\begin{dcases}\cos\paren{\theta+k\pi}=\paren{-1}^k\cos\theta \\ \sin\paren{\theta+k\pi}=\paren{-1}^k\sin\theta\end{dcases}\) et \(\cos\paren{\dfrac{\pi}{2}-\theta}=\sin\theta\) et \(\sin\paren{\dfrac{\pi}{2}-\theta}=\cos\theta\).

Soient \(a,b\in\R\).

\begin{itemize}
\item On a \(\begin{array}[t]{lll}
\tan\paren{a+b} & \text{bien défini ssi} & a+b\not\equiv\dfrac{\pi}{2}\croch{\pi} \\
\tan a & \text{bien défini ssi} & a\not\equiv\dfrac{\pi}{2}\croch{\pi} \\[1em]
\tan b & \text{bien défini ssi} & b\not\equiv\dfrac{\pi}{2}\croch{\pi}
\end{array}\)

On a alors \(\tan\paren{a+b}=\dfrac{\sin\paren{a+b}}{\cos\paren{a+b}}=\dfrac{\sin a\cos b+\sin b\cos a}{\cos a\cos b-\sin a\sin b}=\dfrac{\dfrac{\sin a}{\cos a}+\dfrac{\sin b}{\cos b}}{1-\dfrac{\sin a\sin b}{\cos a\cos b}}=\dfrac{\tan a+\tan b}{1-\tan a\tan b}\).

\item On a \(\begin{array}[t]{lll}
\tan a & \text{bien défini ssi} & a\not\equiv\dfrac{\pi}{2}\croch{\pi} \\
\tan2a & \text{bien défini ssi} & 2a\not\equiv\dfrac{\pi}{2}\croch{\pi}\text{ ssi }a\not\equiv\dfrac{\pi}{4}\croch{\pi}
\end{array}\)

On a alors \(\tan2a=\dfrac{\tan a+\tan a}{1-\tan a\tan a}=\dfrac{2\tan a}{1-\tan^2a}\).

\item \(\cos2a=\cos a\cos a-\sin a\sin a=\cos^2a-\paren{1-\cos^2a}=2\cos^2a-1\)

Donc \(\cos^2a=\dfrac{\cos2a+1}{2}\).

D'où \(\sin^2a=1-\dfrac{\cos2a+1}{2}=\dfrac{1-\cos2a}{2}\).

\item \(\sin2a=\sin a\cos a+\sin a\cos a=2\sin a\cos a\)

\item On a \begin{enumerate}
\item \(\cos\paren{a+b}=\cos a\cos b-\sin a\sin b\)

\item \(\cos\paren{a-b}=\cos a\cos b+\sin a\sin b\)

\item \(\sin\paren{a+b}=\sin a\cos b+\sin b\cos a\)

\item \(\sin\paren{a-b}=\sin a\cos b-\sin b\cos a\)
\end{enumerate}

D'où \begin{itemize}
\item \(\cos a\cos b=\dfrac{\cos\paren{a+b}+\cos\paren{a-b}}{2}\) selon \(\dfrac{(1)+(2)}{2}\).

\item \(\sin a\sin b=\dfrac{\cos\paren{a-b}-\cos\paren{a+b}}{2}\) selon \(\dfrac{(2)-(1)}{2}\).

\item \(\sin a\cos b=\dfrac{\sin\paren{a+b}+\sin\paren{a-b}}{2}\) selon \(\dfrac{(3)+(4)}{2}\).
\end{itemize}

\item Soient \(p,q\in\R\).

On remarque \(\begin{dcases}p=\dfrac{p+q}{2}+\dfrac{p-q}{2} \\ q=\dfrac{p+q}{2}-\dfrac{p-q}{2}\end{dcases}\)

D'où \(\begin{aligned}[t]
\cos p+\cos q&=\cos\paren{\dfrac{p+q}{2}+\dfrac{p-q}{2}}+\cos\paren{\dfrac{p+q}{2}-\dfrac{p-q}{2}} \\
&=\cos\paren{\dfrac{p+q}{2}}\cos\paren{\dfrac{p-q}{2}}-\sin\paren{\dfrac{p+q}{2}}\sin\paren{\dfrac{p-q}{2}}+\cos\paren{\dfrac{p+q}{2}}\cos\paren{\dfrac{p-q}{2}}\notag \\
&\qquad+\sin\paren{\dfrac{p+q}{2}}\sin\paren{\dfrac{p-q}{2}} \\
&=2\cos\paren{\dfrac{p+q}{2}}\cos\paren{\dfrac{p-q}{2}}
\end{aligned}\)

De même, \(\cos p-\cos q=-2\sin\paren{\dfrac{p+q}{2}}\sin\paren{\dfrac{p-q}{2}}\)

Et \(\begin{aligned}[t]
\sin p+\sin q&=\sin\paren{\dfrac{p+q}{2}+\dfrac{p-q}{2}}+\sin\paren{\dfrac{p+q}{2}-\dfrac{p-q}{2}} \\
&=\sin\paren{\dfrac{p+q}{2}}\cos\paren{\dfrac{p-q}{2}}+\sin\paren{\dfrac{p-q}{2}}\cos\paren{\dfrac{p+q}{2}}+\sin\paren{\dfrac{p+q}{2}}\cos\paren{\dfrac{p-q}{2}}\notag \\
&\qquad-\sin\paren{\dfrac{p-q}{2}}\cos\paren{\dfrac{p+q}{2}} \\
&=2\sin\paren{\dfrac{p+q}{2}}\cos\paren{\dfrac{p-q}{2}}
\end{aligned}\)

\item On a \(\theta\not\equiv\pi\croch{2\pi}\) donc \(\dfrac{\theta}{2}\not\equiv\dfrac{\pi}{2}\croch{\pi}\) donc \(t=\tan\dfrac{\theta}{2}\) est bien défini.

On a \(\begin{aligned}[t]
\cos\theta&=\cos\paren{\dfrac{\theta}{2}+\dfrac{\theta}{2}} \\
&=\cos\paren{\dfrac{\theta}{2}}\cos\paren{\dfrac{\theta}{2}}-\sin\paren{\dfrac{\theta}{2}}\sin\paren{\dfrac{\theta}{2}} \\
&=\cos^2\dfrac{\theta}{2}-\sin^2\dfrac{\theta}{2} \\
&=\cos^2\dfrac{\theta}{2}\paren{1-\dfrac{\sin^2\dfrac{\theta}{2}}{\cos^2\dfrac{\theta}{2}}} \\
&=\cos^2\dfrac{\theta}{2}\paren{1-\tan^2\dfrac{\theta}{2}}
\end{aligned}\)

Or \(1+\tan^2=\dfrac{1}{\cos^2}\) donc \(\cos^2=\dfrac{1}{1+\tan^2}\).

Donc \(\cos\theta=\dfrac{1}{1+\tan^2\dfrac{\theta}{2}}\paren{1-\tan^2\dfrac{\theta}{2}}=\dfrac{1-t^2}{1+t^2}\).

De même, \(\begin{aligned}[t]
\sin\theta&=\sin\paren{\dfrac{2\theta}{2}} \\
&=2\sin\paren{\dfrac{\theta}{2}}\cos\paren{\dfrac{\theta}{2}} \\
&=2\dfrac{\sin\dfrac{\theta}{2}}{\cos\dfrac{\theta}{2}}\cos^2\dfrac{\theta}{2} \\
&=2\tan\dfrac{\theta}{2}\times\dfrac{1}{1+\tan^2\dfrac{\theta}{2}} \\
&=\dfrac{2t}{1+t^2}
\end{aligned}\)
\end{itemize}

\begin{rappel}
On a les courbes suivantes :

\(\cos\) (\(2\pi\)-périodique)

\begin{center}
\begin{tkz}
\draw[gray,->] (0,-2) -- (0,2);
\draw[gray,->] (-2*pi,0) -- (2*pi,0);

\draw[domain=-2*pi:2*pi,smooth] plot (\x,{cos(\x r)});

\node[above left,gray] at (0,1) {\(1\)};
\draw[gray] (-0.1,1) -- (0.1,1);

\node[below left,gray] at (0,0) {\(0\)};

\node[left,gray] at (0,-1) {\(-1\)};
\draw[gray] (-0.1,-1) -- (0.1,-1);

\node[below left,gray] at (pi/2,0) {\(\dfrac{\pi}{2}\)};
\draw[gray] (pi/2,-0.1) -- (pi/2,0.1);

\node[below,gray] at (pi,0) {\(\pi\)};
\draw[gray] (pi,-0.1) -- (pi,0.1);

\node[below right,gray] at (3*pi/2,0) {\(\dfrac{3\pi}{2}\)};
\draw[gray] (3*pi/2,-0.1) -- (3*pi/2,0.1);

\node[below right,gray] at (-pi/2,0) {\(-\dfrac{\pi}{2}\)};
\draw[gray] (-pi/2,-0.1) -- (-pi/2,0.1);

\node[below,gray] at (-pi,0) {\(-\pi\)};
\draw[gray] (-pi,-0.1) -- (-pi,0.1);

\node[below left,gray] at (-3*pi/2,0) {\(-\dfrac{3\pi}{2}\)};
\draw[gray] (-3*pi/2,-0.1) -- (-3*pi/2,0.1);

\draw[<->,dotted,blue] (-pi,-1) -- (pi,-1);
\node[below right,blue] at (0,-1) {\(2\pi\)};
\end{tkz}
\end{center}

\(\sin\) (\(2\pi\)-périodique) :

\begin{center}
\begin{tkz}
\draw[gray,->] (0,-2) -- (0,2);
\draw[gray,->] (-2*pi,0) -- (2*pi,0);

\draw[domain=-2*pi:2*pi,smooth] plot (\x,{sin(\x r)});

\node[above left,gray] at (0,1) {\(1\)};
\draw[gray] (-0.1,1) -- (0.1,1);

\node[below left,gray] at (0,0) {\(0\)};

\node[left,gray] at (0,-1) {\(-1\)};
\draw[gray] (-0.1,-1) -- (0.1,-1);

\node[below,gray] at (pi/2,0) {\(\dfrac{\pi}{2}\)};
\draw[gray] (pi/2,-0.1) -- (pi/2,0.1);

\node[below left,gray] at (pi,0) {\(\pi\)};
\draw[gray] (pi,-0.1) -- (pi,0.1);

\node[below,gray] at (3*pi/2,0) {\(\dfrac{3\pi}{2}\)};
\draw[gray] (3*pi/2,-0.1) -- (3*pi/2,0.1);

\node[below,gray] at (-pi/2,0) {\(-\dfrac{\pi}{2}\)};
\draw[gray] (-pi/2,-0.1) -- (-pi/2,0.1);

\node[below left,gray] at (-pi,0) {\(-\pi\)};
\draw[gray] (-pi,-0.1) -- (-pi,0.1);

\node[below,gray] at (-3*pi/2,0) {\(-\dfrac{3\pi}{2}\)};
\draw[gray] (-3*pi/2,-0.1) -- (-3*pi/2,0.1);

\draw[<->,dotted,blue] (-3*pi/2,1) -- (pi/2,1);
\node[above,blue] at (-pi/2,1) {\(2\pi\)};
\end{tkz}
\end{center}

\(\tan\) (\(\pi\)-périodique) :

\begin{center}
\begin{tkz}
\draw[gray,->] (0,-3) -- (0,3);
\draw[gray,->] (-3*pi/2,0) -- (3*pi/2,0);

\foreach \ind in {-1,0,1} {
\pgfmathsetmacro{\starti}{\ind*pi-1.3}
\pgfmathsetmacro{\lefti}{(\ind-0.5)*pi}
\pgfmathsetmacro{\endi}{\ind*pi+1.3}
\draw[dashed,gray] (\lefti,-3) -- (\lefti,3);
\draw[domain=\starti:\endi,smooth] plot (\x,{tan(\x r)});
}

\node[above left,gray] at (0,0) {\(0\)};

\node[below left,gray] at (pi/2,0) {\(\dfrac{\pi}{2}\)};
\draw[gray] (pi/2,-0.1) -- (pi/2,0.1);

\node[below right,gray] at (pi,0) {\(\pi\)};
\draw[gray] (pi,-0.1) -- (pi,0.1);

\node[below left,gray] at (-pi/2,0) {\(-\dfrac{\pi}{2}\)};
\draw[gray] (-pi/2,-0.1) -- (-pi/2,0.1);

\node[below right,gray] at (-pi,0) {\(-\pi\)};
\draw[gray] (-pi,-0.1) -- (-pi,0.1);
\end{tkz}
\end{center}
\end{rappel}

\begin{prop}
On a \(\quantifs{\forall x\in\R}\abs{\sin x}\leq\abs{x}\).
\end{prop}

\begin{dem}
Comme les fonctions \(x\mapsto\abs{\sin x}\) et \(x\mapsto\abs{x}\) sont paires, il suffit de montrer l'inégalité sur \(\Rp\).

C'est à dire \(\quantifs{\forall x\in\Rp}\begin{dcases}\sin x\leq x \\ -\sin x\leq x\end{dcases}\)

Posons \(f:x\mapsto x-\sin x\) et \(g:x\mapsto x+\sin x\).

On a \(\quantifs{\forall x\in\Rp}\begin{dcases}f\prim\paren{x}=1-\cos x\geq0 \\ g\prim\paren{x}=1+\cos x\geq0\end{dcases}\) car \(\quantifs{\forall x\in\Rp}-1\leq\cos x\leq1\).

Donc \(f\) et \(g\) sont croissantes sur \(\Rp\). Donc \(\quantifs{\forall x\in\Rp}f\paren{x}\geq f\paren{0}=0\) et \(\quantifs{\forall x\in\Rp}g\paren{x}\geq g\paren{0}=0\).

Donc \(\quantifs{\forall x\in\Rp}\begin{dcases}x-\sin x\geq0 \\ x+\sin x\geq0\end{dcases}\)

D'où le résultat.
\end{dem}

\begin{rem}
Soient \(a,b\in\R\). On considère la fonction \(f:\theta\mapsto a\cos\theta+b\sin\theta\).

Supposons \(\paren{a,b}\not=\paren{0,0}\).

On a \(\quantifs{\forall\theta\in\R}f\paren{\theta}=\sqrt{a^2+b^2}\paren{\dfrac{a}{\sqrt{a^2+b^2}}\cos\theta+\dfrac{b}{\sqrt{a^2+b^2}}\sin\theta}\).

On remarque \(\paren{\dfrac{a}{\sqrt{a^2+b^2}}}^2+\paren{\dfrac{b}{\sqrt{a^2+b^2}}}^2=1\).

Donc il existe \(\alpha\in\R\) tel que \(\begin{dcases}\dfrac{a}{\sqrt{a^2+b^2}}=\sin\alpha \\ \dfrac{b}{\sqrt{a^2+b^2}}=\cos\alpha\end{dcases}\)

On a alors \(\quantifs{\forall\theta\in\R}\begin{aligned}[t]
f\paren{\theta}&=\sqrt{a^2+b^2}\paren{\sin\alpha\cos\theta+\cos\alpha\sin\theta} \\
&=\sqrt{a^2+b^2}\sin\paren{\theta+\alpha}
\end{aligned}\)
\end{rem}

\chapter{Nombres complexes}

\minitoc

\section{Rappels}

\subsection{Point de vue algébrique}

\begin{rappel}
\begin{itemize}
\item On note \(\C\) l'ensemble des nombres complexes et \(\i\in\C\) le nombre complexe particulier vérifiant \(\i^2=-1\).

\item Tout nombre complexe \(z\in\C\) s'écrit de façon unique sous la forme \(z=a+\i b\) où \(a,b\in\R\) sont appelés respectivement partie réelle et partie imaginaire de \(z\) et notés \(a=\Re z\) et \(b=\Im z\). On dit que \(z=a+\i b\) est l'écriture algébrique de \(z\).

\item Enfin, on dit que \(z\) est un imaginaire pur si \(\Re z=0\).

\item Opérations sur les nombres complexes :

Soient \(z_1,z_2\in\C\). Soient \(a_1,b_1,a_2,b_2\in\R\) tels que \(\begin{dcases}z_1=a_1+\i b_1 \\ z_2=a_2+\i b_2\end{dcases}\)

On pose \(\begin{dcases}z_1+z_2=a_1+a_2+\i\paren{b_1+b_2} \\ z_1z_2=a_1a_2-b_1b_2+\i\paren{a_1b_2+a_2b_1} \\ \conj{z_1}=a_1-\i b_1 \\ \abs{z_1}=\sqrt{a_1^2+b_1^2}\end{dcases}\)

On a alors \(\begin{dcases}\Re z_1=\dfrac{z_1+\conj{z_1}}{2}\text{ et }\Im z_1=\dfrac{z_1-\conj{z_1}}{2} \\ z_1+\conj{z_1}=2\Re z_1\text{ et }z_1\conj{z_1}=\abs{z_1}^2\end{dcases}\)

Si \(z_1\not=0\) alors l'inverse de \(z_1\) est l'unique nombre complexe noté \(z_1^{-1}\) vérifiant \(z_1z_1^{-1}=1\).

On remarque \(z_1^{-1}=\dfrac{\conj{z_1}}{\abs{z_1}^2}\) car \(z_1\dfrac{\conj{z_1}}{\abs{z_1}^2}=1\). Donc \(z_1^{-1}=\dfrac{a_1-\i b_1}{a_1^2+b_1^2}\).

On pose \(\quantifs{\forall n\in\N}z_1^n=\underbrace{z_1\times z_1\times\dots\times z_1}_\text{$n$ facteurs}=\prod_{k=1}^nz_1\).

Si \(z_1\not=0\) on pose aussi \(\quantifs{\forall n\in\N}z_1^{-n}=\dfrac{1}{z_1^n}\).

\item Propriétés :

Soient \(n\in\N\) et \(z_1,\dots,z_n\in\C\).

On a \begin{itemize}
\item \(\Re\paren{\sum_{k=1}^nz_k}=\sum_{k=1}^{n}\Re z_k\)

\item \(\Im\paren{\sum_{k=1}^nz_k}=\sum_{k=1}^{n}\Im z_k\)

\item \(\abs{\prod_{k=1}^nz_k}=\prod_{k=1}^n\abs{z_k}\)

\item \(\conj{\sum_{k=1}^nz_k}=\sum_{k=1}^n\conj{z_k}\) et \(\conj{\prod_{k=1}^nz_k}=\prod_{k=1}^n\conj{z_k}\)

\item \(\quantifs{\forall N\in\N}\paren{\prod_{k=1}^nz_k}^N=\prod_{k=1}^nz_k^N\)

\item \(z_1,\dots,z_n\in\Cs\imp\quantifs{\forall n\in\Z}\paren{\prod_{k=1}^nz_k}^N=\prod_{k=1}^nz_k^N\)
\end{itemize}

\item Nombres complexes de module \(1\) :

On pose \(\U=\accol{z\in\C\tq\abs{z}=1}\).

Soit \(z\in\C\). Soient \(a,b\in\R\) tels que \(z=a+\i b\).

On a \(\begin{aligned}[t]
z\in\U&\ssi\abs{z}=1 \\
&\ssi\sqrt{a^2+b^2}=1 \\
&\ssi a^2+b^2=1 \\
&\ssi\exists\theta\in\R,\begin{dcases}a=\cos\theta \\ b=\sin\theta\end{dcases}
\end{aligned}\)

Donc \(\U=\accol{\cos\theta+\i\sin\theta}_{\theta\in\R}\).

On pose \(\quantifs{\forall\theta\in\R}\e{\i\theta}=\cos\theta+\i\sin\theta\).

Ainsi, \(\U=\accol{\e{\i\theta}}_{\theta\in\R}\).
\end{itemize}
\end{rappel}

\begin{rem}
On a \(\Im z=0\ssi z\in\R\).
\end{rem}

\begin{defprop}
Soit \(z\in\Cs\).

Alors il existe \(\lambda\in\Rps\) et \(\theta\in\R\) tels que \(z=\lambda\e{\i\theta}\) (écriture trigonométrique).

On a \(\lambda=\abs{z}\).

On dit que \(\theta\) est un argument de \(z\).

Si de plus on a \(\theta\in\intervei{-\pi}{\pi}\), on dit que \(\theta\) est l'argument (principal) de \(z\) et on note \(\theta=\arg z\).
\end{defprop}

\begin{dem}
Montrons que \(\lambda\) et \(\theta\) existent.

On a \(z\not=0\).

Donc \(\dfrac{z}{\abs{z}}\) est bien défini et on a \(\abs{\dfrac{z}{\abs{z}}}=1\).

Donc il existe \(\theta\in\R\) tel que \(\dfrac{z}{\abs{z}}=\e{\i\theta}\) car \(z\in\U\).

On a bien \(z=\lambda\e{\i\theta}\) en posant \(\lambda=\abs{z}\).
\end{dem}

\begin{rem}
Soient \(\lambda_1,\lambda_2\in\Rps\) et \(\theta_1,\theta_2\in\R\).

On a \(\lambda_1\e{\i\theta_1}=\lambda_2\e{\i\theta_2}\ssi\begin{dcases}\lambda_1=\lambda_2 \\ \theta_1\equiv\theta_2\croch{2\pi}\end{dcases}\)
\end{rem}

\begin{dem}
\begin{itemize}
\item[\imprec] Claire.

\item[\impdir] Supposons \(\lambda_1\e{\i\theta_1}=\lambda_2\e{\i\theta_2}\).

Donc \(\abs{\lambda_1\e{\i\theta_1}}=\abs{\lambda_2\e{\i\theta_2}}\).

Donc \(\lambda_1=\lambda_2\).

Donc \(\e{\i\theta_1}=\e{\i\theta_2}\) car \(\lambda_1=\lambda_2\not=0\).

Donc \(\cos\theta_1+\i\sin\theta_1=\cos\theta_2+\i\sin\theta_2\).

Donc \(\begin{dcases}\cos\theta_1=\cos\theta_2 \\ \sin\theta_1=\sin\theta_2\end{dcases}\)

Donc \(\theta_1\equiv\theta_2\croch{2\pi}\).
\end{itemize}
\end{dem}

\begin{prop}
Soient \(z_1,z_2\in\Cs\).

On a \begin{itemize}
\item \(\arg\paren{z_1z_2}\equiv\arg z_1+\arg z_2\croch{2\pi}\)

\item \(\arg\dfrac{1}{z_1}\equiv-\arg z_1\croch{2\pi}\)

\item \(\arg\conj{z_1}\equiv-\arg z_1\croch{2\pi}\)

\item \(\quantifs{\forall\lambda\in\Rps}\arg\paren{\lambda z_1}=\lambda\arg z_1\)
\end{itemize}
\end{prop}

\subsection{Point de vue géométrique}

\begin{defi}[Affixe]
Soient \(a,b\in\R\).

On a \(\paren{a,b}\in\R^2\).

On associe à ce point le nombre complexe \(z=a+\i b\).

Ce nombre complexe est appelé l'affixe du point.
\end{defi}

\begin{ex}~ % avoids the theorem header to be centered too
\begin{center}
\begin{tkz}[scale=3]
\draw[->,gray] (-1.3,0) -- (1.3,0) node[right,gray] {\(\R\)};
\draw[->,gray] (0,-1.3) -- (0,1.3) node[above,gray] {\(\iR\)};

\draw (0,0) circle (1) node[below left,gray] {\(0\)};
\node[below left] at (-0.707106,-0.707106) {\(\U\)};

\draw[fill] (1,0) circle (1pt);
\draw[->] (1,0) -- (1.5,-0.5) node[right] {point de \(\R^2\) d'affixe \(1\)};

\draw[fill] (0,1) circle (1pt);
\draw[->] (0,1) -- (-0.5,1.5) node[left] {point de \(\R^2\) d'affixe \(\i\)};

\draw[dotted] (0.5,0.866025) -- (0.5,0) node[gray,below] {\(\dfrac{1}{2}\)};
\draw[dotted] (0.5,0.866025) -- (0,0.866025) node[gray,below left] {\(\dfrac{\sqrt{3}}{2}\)};

\draw[fill] (0.5,0.866025) circle (1pt);
\draw[->] (0.5,0.866025) -- (1,0.866025) node[right] {point de \(\R^2\) d'affixe \(\e{\i\frac{\pi}{3}}=\dfrac{1}{2}+\i\dfrac{\sqrt{3}}{2}\)};
\end{tkz}
\end{center}
\end{ex}

\begin{prop}
Soit \(M\in\R^2\).

On note \(z\) l'affixe de \(M\).

Alors \begin{itemize}
\item \(\abs{z}\) est la distance de \(M\) à l'origine ;

\item \(\conj{z}\) est l'affixe du symétrique de \(M\) par rapport à l'axe des abscisses.
\end{itemize}
\end{prop}

\begin{prop}
Soient \(z,z_1\in\Cs\).

On pose \(z_2=zz_1\).

Considérons les écritures trigonométriques de ces trois complexes : \(\begin{dcases}z=\lambda\e{\i\theta} \\ z_1=\lambda_1\e{\i\theta_1} \\ z_2=\lambda_2\e{\i\theta_2}=\lambda\lambda_1\e{\i\paren{\theta+\theta_1}}\end{dcases}\)
\end{prop}

\begin{defi}[Affixe d'un vecteur]
Soient \(A,B\in\R^2\) deux points du plan d'affixes respectives \(a,b\in\C\).

L'affixe du vecteur \(\vec{AB}\) est le nombre complexe \(b-a\).

Son module \(\abs{b-a}\) est la longueur \(AB\) et son argument est l'angle :

\begin{center}
\begin{tkz}
\draw[->,gray] (-1,0) -- (5,0) coordinate (A);
\draw[->,gray] (0,-1) -- (0,5);

\draw[->] (1,1) node[above left] {\(A\)} -- (4,4) node[above left] {\(B\)};

\draw[dashed] (-1,-1) -- (5,5) coordinate (Theta);

\node[gray, above left] at (0,0) {\(0\)};

\draw (0,0) -- (0,0) coordinate (O);

\draw pic[draw,->,"\(\theta=\arg\paren{b-a}\)"{xshift=40},angle eccentricity=1.8] {angle = A--O--Theta};
\end{tkz}
\end{center}
\end{defi}

\begin{prop}[Inégalité triangulaire pour les complexes]
Soient \(z_1,z_2\in\C\).

On a \(\abs{z_1+z_2}\leq\abs{z_1}+\abs{z_2}\) avec égalité ssi \(\quantifs{\exists\lambda\in\Rp}z_1=\lambda z_2\) ou \(z_2=\lambda z_1\).
\end{prop}

\begin{dem}
On remarque qu'on a \(\quantifs{\forall z\in\C}\Re z\leq\abs{z}\) avec égalité ssi \(z\in\Rp\).

En effet, soit \(z\in\C\) et \(a,b\in\R\) tels que \(z=a+\i b\).

On a \(\Re z=a\leq\abs{a}=\sqrt{a^2}\leq\sqrt{a^2+b^2}=\abs{z}\).

Cas d'égalité : \(\begin{aligned}[t]
\Re z=\abs{z}&\ssi\begin{dcases}a=\abs{a} \\ \sqrt{a^2}=\sqrt{a^2+b^2}\end{dcases} \\
&\ssi\begin{dcases}a\geq0 \\ b=0\end{dcases} \\
&\ssi z\in\Rp
\end{aligned}\)

Montrons l'inégalité triangulaire.

On a \(\begin{aligned}[t]
\abs{z_1+z_2}\leq\abs{z_1}+\abs{z_2}&\ssi\abs{z_1+z_2}^2\leq\paren{\abs{z_1}+\abs{z_2}}^2\text{ car \(t\mapsto t^2\) strictement croissante sur \(\Rp\)} \\
&\ssi\paren{z_1+z_2}\paren{\conj{z_1}+\conj{z_2}}\leq\paren{\abs{z_1}+\abs{z_2}}^2 \\
&\ssi z_1\conj{z_1}+z_1\conj{z_2}+z_2\conj{z_1}+z_2\conj{z_2}\leq\abs{z_1}^2+2\abs{z_1}\abs{z_2}+\abs{z_2}^2 \\
&\ssi2\Re\paren{z_1\conj{z_2}}\leq2\abs{z_1}\abs{z_2} \\
&\ssi\Re\paren{z_1\conj{z_2}}\leq\abs{z_1}\abs{z_2}=\abs{z_1}\abs{\conj{z_2}}=\abs{z_1\conj{z_2}} \\
&\color{white}\ssi\color{black}\text{ce qui est vrai (cf. ci-dessus)}
\end{aligned}\)

Cas d'égalité : montrons que \(\abs{z_1+z_2}=\abs{z_1}+\abs{z_2}\ssi\quantifs{\exists\lambda\in\Rp}z_1=\lambda z_2\) ou \(z_2=\lambda z_1\).

On remarque que l'équivalence est vraie si \(z_1\) ou \(z_2\) est nul.

Supposons \(z_1\) et \(z_2\) non-nuls.

On a \(\begin{aligned}[t]
\abs{z_1+z_2}=\abs{z_1}+\abs{z_2}&\ssi\Re\paren{z_1\conj{z_2}}=\abs{z_1\conj{z_2}} \\
&\ssi z_1\conj{z_2}\in\Rp \\
&\ssi\dfrac{z_1\conj{z_2}}{\abs{z_2}^2}\in\Rp\text{ car }z_2\not=0 \\
&\ssi\dfrac{z_1}{z_2}\in\Rp\text{ car }\dfrac{\conj{z_2}}{\abs{z_2}}=\dfrac{\conj{z_2}}{z_2\conj{z_2}} \\
&\ssi\quantifs{\exists\lambda\in\Rp}\dfrac{z_1}{z_2}=\lambda \\
&\ssi\quantifs{\exists\lambda\in\Rp}z_1=\lambda z_2
\end{aligned}\)

~
\end{dem}

\begin{rem}
Mêmes notations.

Comme on l'a vu pour les réels, on déduit de l'inégalité triangulaire la minoration \(\abs{\abs{z_1}-\abs{z_2}}\leq\abs{z_1+z_2}\).
\end{rem}

\begin{cor}
Soient \(A,B,C\in\R^2\).

On a \(AC\leq AB+BC\) avec égalité ssi \(\exists\lambda\in\Rp,\vec{AB}=\lambda\vec{BC}\) ou \(\vec{BC}=\lambda\vec{AB}\), c'est à dire ssi \(\vec{AB}\) et \(\vec{BC}\) sont colinéaires et de même sens.
\end{cor}

\begin{dem}
Découle de ce qui précède appliqué aux affixes \(z_1,z_2\) respectives de \(\vec{AB}\) et \(\vec{BC}\).
\end{dem}

\begin{prop}
Soient \(A,B,C\in\R^2\) trois points deux à deux distincts d'affixes \(a,b,c\in\C\) respectivement.

Interprétation de \(z=\dfrac{c-a}{b-a}\) : \(\abs{z}=\dfrac{AC}{AB}\) et \(\arg z\equiv\paren{\vec{AB};\vec{AC}}\croch{2\pi}\).
\end{prop}

\begin{ex}
\begin{itemize}
\item \(\begin{aligned}[t]
A,B,C\text{ alignés}&\ssi\arg\dfrac{c-a}{b-a}\equiv0\croch{\pi} \\
&\ssi\dfrac{c-a}{b-a}\in\R
\end{aligned}\)

\item \(\begin{aligned}[t]
\text{Le triangle }ABC\text{ est rectangle en }A&\ssi\arg\dfrac{c-a}{b-a}\equiv\dfrac{\pi}{2}\croch{\pi} \\
&\ssi\dfrac{c-a}{b-a}\in\iR
\end{aligned}\)
\end{itemize}
\end{ex}

\begin{defi}
Soient \(z_0\in\C\) et \(r\in\Rp\).

On définit : \begin{itemize}
\item le cercle de centre \(z_0\) et de rayon \(r\) : \(\accol{z\in\C\tq\abs{z-z_0}=r}\) ;

\item le disque ouvert de centre \(z_0\) et de rayon \(r\) : \(\accol{z\in\C\tq\abs{z-z_0}<r}\) ;

\item le disque fermé de centre \(z_0\) et de rayon \(r\) : \(\accol{z\in\C\tq\abs{z-z_0}\leq r}\).
\end{itemize}
\end{defi}

\begin{defi}[Similitude directe]
On appelle similitude directe de \(\C\) toute fonction de la forme \(\fonctionlambda{\C}{\C}{z}{az+b}\) avec \(\begin{dcases}a\in\Cs \\ b\in\C\end{dcases}\)
\end{defi}

\begin{ex}
\begin{itemize}
\item \(\id{\C}:z\mapsto z\) est une similitude directe de \(\C\) (avec \(a=1\) et \(b=0\)) ;

\item \(\quantifs{\forall b\in\C}z\mapsto z+b\) est une similitude directe de \(\C\) (translation) ;

\item \(\quantifs{\forall a\in\Rps}z\mapsto az\) est une similitude directe de \(\C\) (homothétie de centre \(O\)), en particulier, la symétrie centrale par rapport à \(O\) est une similitude directe de \(\C\) : \(z\mapsto-z\) ;

\item \(\quantifs{\forall\theta\in\R}z\mapsto\e{\i\theta}z\) est une similitude directe de \(\C\) (rotation d'angle \(\theta\) et de centre \(O\)).
\end{itemize}
\end{ex}

\begin{rem}
Soient \(a\in\Cs\) et \(b\in\C\).

On pose \(f:z\mapsto az+b\) une similitude directe de \(\C\).

\begin{itemize}
\item Si \(a=1\) et \(b=0\) alors \(\quantifs{\forall z\in\C}f\paren{z}=z\) (tout complexe \(z\) est un point fixe de \(f\)) ;

\item si \(a=1\) et \(b\not=0\) alors \(f\) n'admet aucun point fixe : \(\quantifs{\forall z\in\C}f\paren{z}\not=z\) ;

\item si \(a\not=1\) alors \(f\) admet un unique point fixe.

En effet, \(\begin{aligned}[t]
f\paren{z}=z&\ssi az+b=z \\
&\ssi\dfrac{b}{1-a}=z
\end{aligned}\)
\end{itemize}
\end{rem}

\begin{rem}
Les fonctions \(z\mapsto\conj{z}\) (symétrie par rapport à \(\R\)) et \(z\mapsto-\conj{z}\) (symétrie par rapport à \(\iR\)) ne sont pas des similitudes directes de \(\C\).
\end{rem}

\subsection{Généralisation de formules connues}

La formule du binôme de Newton, la factorisation de \(a^n-b^n\) et la valeur de \(\sum_{k=a}^b z^k\) restent vraies en prenant des nombres complexes à la place des nombres réels.

\section{Lien avec la trigonométrie}

\subsection{Formules}

On rappelle qu'on a posé \(\quantifs{\forall\theta\in\R}\e{\i\theta}=\cos\theta+\i\sin\theta\).

\begin{prop}\thlabel{prop:prodExpComplexe}
Soient \(\theta_1,\theta_2\in\R\).

On a \(\e{\i\paren{\theta_1+\theta_2}}=\e{\i\theta_1}\e{\i\theta_2}\).
\end{prop}

\begin{dem}
On a \(\begin{aligned}[t]
\e{\i\paren{\theta_1+\theta_2}}&=\cos\paren{\theta_1+\theta_2}+\i\sin\paren{\theta_1+\theta_2} \\
&=\cos\theta_1\cos\theta_2-\sin\theta_1\sin\theta_2+\i\paren{\sin\theta_1\cos\theta_2+\sin\theta_2\cos\theta_1} \\
&=\paren{\cos\theta_1+\i\sin\theta_1}\paren{\cos\theta_2+\i\sin\theta_2} \\
&=\e{\i\theta_1}\e{\i\theta_2}
\end{aligned}\)

~
\end{dem}

\begin{prop}[Formule de Moivre]~\\
On a \(\quantifs{\forall n\in\N;\forall\theta\in\R}\begin{dcases}\cos\paren{n\theta}=\Re\paren{\paren{\cos\theta+\i\sin\theta}^n} \\ \sin\paren{n\theta}=\Im\paren{\paren{\cos\theta+\i\sin\theta}^n}\end{dcases}\)
\end{prop}

\begin{dem}
Soit \(\theta\in\R\).

On déduit de la \thref{prop:prodExpComplexe} que \(\quantifs{\forall n\in\N}\e{\i n\theta}=\paren{\e{\i\theta}}^n\) par récurrence sur \(n\in\N\).

Soit \(n\in\N\). On a donc \(\cos\paren{n\theta}+\i\sin\paren{n\theta}=\paren{\cos\theta+\i\sin\theta}^n\).

On conclut en prenant les parties réelles et imaginaires.
\end{dem}

\begin{ex}
Soit \(\theta\in\R\). Exprimons \(\cos\paren{3\theta}\) en fonction de \(\cos\theta\).

On a \(\begin{aligned}[t]
\cos\paren{3\theta}&=\Re\paren{\paren{\cos\theta+\i\sin\theta}^3} \\
&=\Re\paren{\sum_{k=0}^3\binom{k}{3}\paren{\i\sin\theta}^k\paren{\cos\theta}^{3-k}} \\
&=\Re\paren{\cos^3\theta+3\i\sin\theta\cos^2\theta-3\sin^2\theta\cos\theta-\i\sin^3\theta} \\
&=\cos^3\theta-3\sin^2\theta\cos\theta \\
&=\cos^3\theta-3\paren{1-\cos^2\theta}\cos\theta \\
&=\cos^3\theta-3\cos\theta+3\cos^3\theta \\
&=4\cos^3\theta-3\cos\theta
\end{aligned}\)

Plus tard, on écrira \(\cos\paren{3\theta}=\tcheby{3}{\cos\theta}\) avec \(\tcheby{3}{X}=4X^3-3X\).
\end{ex}

\begin{prop}[Formules d'Euler]
Soit \(\theta\in\R\).

On a \(\begin{dcases}\cos\theta=\dfrac{\e{\i\theta}+\e{-\i\theta}}{2} \\ \sin\theta=\dfrac{\e{i\theta}-\e{-\i\theta}}{2\i}\end{dcases}\)
\end{prop}

\begin{rem}
Soient \(\alpha,\beta\in\R\).

Voyons comment factoriser \(\e{\i\alpha}\pm\e{\i\beta}\).

On remarque \(\begin{dcases}\alpha=\dfrac{\alpha+\beta}{2}+\dfrac{\alpha-\beta}{2} \\ \beta=\dfrac{\alpha+\beta}{2}-\dfrac{\alpha-\beta}{2}\end{dcases}\)

Donc \(\e{\i\alpha}+\e{\i\beta}=\e{\i\frac{\alpha+\beta}{2}}\paren{\e{\i\frac{\alpha+\beta}{2}}+\e{-\i\frac{\alpha-\beta}{2}}}=2\cos\paren{\dfrac{\alpha+\beta}{2}}\e{\i\frac{\alpha+\beta}{2}}\).

Et \(\e{\i\alpha}-\e{\i\beta}=\e{\i\frac{\alpha+\beta}{2}}\paren{\e{\i\frac{\alpha+\beta}{2}}-\e{-\i\frac{\alpha-\beta}{2}}}=2\i\sin\paren{\dfrac{\alpha+\beta}{2}}\e{\i\frac{\alpha+\beta}{2}}\).
\end{rem}

\subsection{Application 1 : sommes particulières}

Soit \(\theta\in\R\) et \(n\in\N\).

Calculons \(S_1=\sum_{k=0}^n\cos\paren{k\theta}\) et \(S_2=\sum_{k=0}^n\sin\paren{k\theta}\).

Calculons d'abord \(S=\sum_{k=0}^n\e{\i k\theta}\).

On a \(S=\sum_{k=0}^n\paren{\e{\i\theta}}^k\).

Si \(\theta\equiv0\croch{2\pi}\) alors \(S=\sum_{k=0}^n1^k=n+1\).

Sinon, \(S=\dfrac{1-\paren{\e{\i\theta}}^{n+1}}{1-\e{\i\theta}}\).

Supposons \(\theta\not\equiv0\croch{2\pi}\).

On a \(\begin{aligned}[t]
S&=\dfrac{1-\paren{\e{\i\theta\paren{n+1}}}}{1-\e{\i\theta}} \\
&=\dfrac{\e{\i\frac{\theta\paren{n+1}}{2}}\paren{\e{-\i\frac{\theta\paren{n+1}}{2}}-\e{\i\frac{\theta\paren{n+1}}{2}}}}{\e{\i\frac{\theta}{2}}\paren{\e{-\i\frac{\theta}{2}}-\e{\i\frac{\theta}{2}}}} \\
&=\dfrac{-2\i\e{\i\frac{\theta\paren{n+1}}{2}}\sin\paren{\frac{\theta\paren{n+1}}{2}}}{-2\i\e{\i\frac{\theta}{2}}\sin\frac{\theta}{2}} \\
&=\dfrac{\sin\frac{\theta\paren{n+1}}{2}}{\sin\frac{\theta}{2}}\times\e{\i\theta\paren{\frac{n+1-1}{2}}} \\
&=\e{\i n\frac{\theta}{2}}\times\dfrac{\sin\paren{\frac{n+1}{2}}\theta}{\sin\frac{\theta}{2}}
\end{aligned}\)

D'où \(S_1=\Re S=\cos\dfrac{n\theta}{2}\times\dfrac{\sin\paren{\dfrac{n+1}{2}\theta}}{\sin\dfrac{\theta}{2}}\).

Et \(S_2=\Im S=\sin\dfrac{n\theta}{2}\times\dfrac{\sin\paren{\dfrac{n+1}{2}\theta}}{\sin\dfrac{\theta}{2}}\).

\subsection{Application 2 : linéarisation}

Soient \(a,b\in\N\) et \(\theta\in\R\).

Linéariser \(\cos^a\theta\sin^b\theta\) c'est écrire cette expression sous la forme \(\sum_{i\in I}\lambda_i\cos\paren{\mu_i\theta}\) ou \(\sum_{i\in I}\lambda_i\sin\paren{\mu_i\theta}\) où \(I\) est un ensemble fini et \(\quantifs{\forall i\in I}\lambda_i,\mu_i\in\R\).

\begin{ex}~\\
On a \(\quantifs{\forall\theta\in\R}\begin{dcases}
\cos^2\theta=\dfrac{1+\cos\paren{2\theta}}{2} \\
\sin^2\theta=\dfrac{1-\cos\paren{2\theta}}{2} \\
\cos^3\theta=\dfrac{\cos\paren{3\theta}+3\cos\theta}{4} \\
\cos\theta\sin\theta=\dfrac{1}{2}\sin\paren{2\theta}
\end{dcases}\)
\end{ex}

Application : linéariser l'expression \(\cos^a\theta\sin^b\theta\) permet de la primitiver.

Méthode : remplacer \(\cos\theta\) et \(\sin\theta\) avec les formules d'Euler puis développer avec le binôme de Newton.

\begin{ex}
Soit \(\fonction{f}{\R}{\R}{\theta}{\sin^4\theta}\).

Linéarisons \(f\paren{\theta}\).

On a \(\begin{aligned}[t]
\sin^4\theta&=\paren{\dfrac{\e{\i\theta}-\e{-\i\theta}}{2\i}}^4 \\
&=\sum_{k=0}^4\binom{k}{4}\paren{\dfrac{\e{\i\theta}}{2\i}}^k\paren{\dfrac{-\e{-\i\theta}}{2\i}}^{4-k} \\
&=1\times1\times\dfrac{\e{-4\i\theta}}{16}+4\times\dfrac{\e{\i\theta}}{2\i}\times\dfrac{-\e{-3\i\theta}}{-8\i}+6\times\dfrac{\e{2\i\theta}}{-4}\times\dfrac{\e{-2\i\theta}}{-4}+4\times\dfrac{\e{3\i\theta}}{-8\i}\times\dfrac{-\e{-\i\theta}}{2\i}+1\times\dfrac{\e{4\i\theta}}{16}\times1 \\
&=\dfrac{\e{4\i\theta}}{16}+\dfrac{\e{-4\i\theta}}{16}+4\paren{\dfrac{-\e{-2\i\theta}}{16}+\dfrac{-\e{2\i\theta}}{16}}+\dfrac{6}{16} \\
&=\dfrac{1}{8}\cos\paren{4\theta}-\dfrac{1}{2}\cos\paren{2\theta}+\dfrac{3}{8}
\end{aligned}\)

On en déduit la primitive \(F\paren{\theta}=\dfrac{1}{32}\sin\paren{4\theta}-\dfrac{1}{4}\sin\paren{2\theta}+\dfrac{3\theta}{8}\).
\end{ex}

\section{Équations algébriques}

\subsection{Fonctions polynomiales}

\begin{defi}
On appelle fonction polynomiale de \(\C\) dans \(\C\) toute fonction de la forme \(\fonction{f}{\C}{\C}{z}{\sum_{k=0}^n\lambda_kz^k}\) où \(n\in\N\) et \(\lambda_0,\dots,\lambda_n\in\C\).

Si \(\lambda_n\not=0\), on dit que la fonction polynomiale est de degré \(n\).

Si \(z_0\in\C\) vérifie \(f\paren{z_0}=0\), on dit que \(z_0\) est une racine de \(f\).
\end{defi}

\begin{prop}
Soient \(n\in\N\) et \(\lambda_0,\dots,\lambda_n\in\C\).

On pose \(\fonction{f}{\C}{\C}{z}{\sum_{k=0}^n\lambda_kz^k}\).

Les propositions suivantes sont équivalentes : \begin{enumerate}
\item \(z_0\) est racine de \(f\)

\item il existe une fonction polynomiale \(g:\C\to\C\) telle que \(\quantifs{\forall z\in\C}f\paren{z}=\paren{z-z_0}g\paren{z}\)
\end{enumerate}
\end{prop}

\begin{rem}
Cette proposition est fausse pour les fonctions continues de \(\R\) dans \(\R\).

Posons par exemple \(\fonction{f}{\R}{\R}{t}{\sqrt{\abs{t}}}\).

On a \(f\paren{0}=0\) donc \(0\) racine de \(f\) donc (1) est vraie.

Pourtant si \(g:\R\to\R\) vérifie \(\forall t\in\R,f\paren{t}=\paren{t-0}g\paren{t}\) alors \(\forall t\in\Rps,g\paren{t}=\dfrac{f\paren{t}}{t}=\dfrac{\sqrt{\abs{t}}}{t}=\dfrac{\sqrt{t}}{t}=\dfrac{1}{\sqrt{t}}\).

On a \(\lim_{t\to0^+}g\paren{t}=\pinf\) : contradiction car \(g\) est continue en \(0\).
\end{rem}

\begin{dem}
Montrons que (1) \(\ssi\) (2).

\begin{itemize}
\item[\imprec] Claire.

\item[\impdir] Supposons \(f\paren{z_0}=0\).

On a \(\begin{aligned}[t]
\quantifs{\forall z\in\C}f\paren{z}&=f\paren{z}-f\paren{z_0} \\
&=\sum_{k=0}^n\lambda_kz^k-\sum_{k=0}^n\lambda_kz_0^k \\
&=\sum_{k=0}^n\lambda_k\paren{z^k-z_0^k} \\
&=\sum_{k=0}^n\paren{\lambda_k\paren{z-z_0}\sum_{l=0}^{k-1}z^lz_0^{k-1-l}} \\
&=\paren{z-z_0}\sum_{1\leq l<k\leq n}\lambda_kz_0^{k-1-l}z^l \\
&=\paren{z-z_0}\sum_{l=0}^{n-1}\paren{\sum_{k=l+1}^n\lambda_kz_0^{k-1-l}}z^l \\
&=\paren{z-z_0}\sum_{l=0}^{n-1}\mu_lz^l
\end{aligned}\)

en posant \(\quantifs{\forall l\in\interventierii{0}{n-1}}\mu_l=\sum_{k=l+1}^{n}\lambda_kz^{k-1-l}\).

D'où (2) en prenant \(g:z\mapsto\sum_{l=0}^{n-1}\mu_lz^l\).
\end{itemize}
\end{dem}

\subsection{Racines nièmes}

\begin{rappel}
Si \(x\in\Rp\), on note \(\sqrt{x}\) ou \(x^{\nicefrac{1}{2}}\) l'unique réel positif de carré \(x\).

On a par exemple \(\sqrt{4}=2\) ; \(\sqrt{2}\) est l'unique réel positif tel que \(\sqrt{2}^2=2\).

Si \(x\in\Rp\) et \(n\in\Ns\), on note \(\sqrt[n]{x}\) ou \(x^{\nicefrac{1}{n}}\) l'unique réel positif dont la puissance nième vaut \(x\).
\end{rappel}

\begin{rem}
On a \(\quantifs{\forall x\in\Rps}\sqrt[n]{x}=\e{\frac{1}{n}\ln x}\).

En effet, \(\e{\frac{1}{n}\ln x}\geq0\) et \(\paren{\e{\frac{1}{n}\ln x}}^n=\e{\ln x}=x\).
\end{rem}

\subsubsection{Division euclidienne dans \(\Z\)}

\begin{prop}
Soient \(a\in\Z\) et \(b\in\Ns\).

Alors \(\quantifs{\exists!\paren{q,r}\in\Z^2}\begin{dcases}a=qb+r \\ 0\leq r<b\end{dcases}\)

\(q\) et \(r\) sont respectivement appelés le quotient et le reste de la division euclidienne de \(a\) par \(b\) .
\end{prop}

\begin{dem}\thlabel{dem:divEucli}
\unicite

Soient \(\paren{q_1,r_1},\paren{q_2,r_2}\in\Z^2\) tels que \(\begin{dcases}
a=q_1b+r_1 \\
a=q_2b+r_2 \\
0\leq r_1<b \\
0\leq r_2<b
\end{dcases}\)

On a d'une part \(q_1b+r_1=q_2b+r_2\) donc \(\paren{q_1-q_2}b=r_2-r_1\).

D'autre part on a \(\begin{dcases}0\leq r_1<b\text{ donc }-b<r_1-r_2<b \\ -b<-r_2\leq0\text{ donc }\abs{r_2-r_1}<b\end{dcases}\)

On en déduit que \(q_1-q_2=0\).

En effet, par l'absurde, si \(q_1-q_2\not=0\) alors \(\abs{q_1-q_2}\geq1\) donc \(\abs{r_2-r_1}=\abs{q_1-q_2}b\geq b\) : contradiction.

Donc \(q_1=q_2\) donc \(r_1=a-q_1b=a-q_2b=r_2\).

D'où l'unicité.

\existence

Montrons d'abord le cas où \(a\in\N\).

On pose \(A=\accol{x\in\N\tq\non\paren{\quantifs{\exists q\in\Z;\exists r\in\interventierii{0}{b-1}}x=bq+r}}\).

Montrons que \(A=\ensvide\). Par l'absurde, supposons \(A\not=\ensvide\).

Comme \(A\) est une partie non-vide de \(\N\), elle admet un plus petit élément \(m\in A\).

\begin{itemize}
\item On remarque que tout \(x\in\interventierii{0}{b-1}\) admet une division euclidienne par \(b\) : \(x=0\times b+x\). Donc \(x\not\in A\).

Donc \(b<m\).

\item Considérons \(m\prim=m-b\). On a \(m\prim<m\).

Donc \(m\prim\not\in A\) et \(m\prim\in\N\) donc il existe \(q\prim\in\Z\) et \(r\prim\in\interventierii{0}{r-1}\) tels que \(m\prim=q\prim b+r\prim\).

Finalement, \(m=m\prim+b=\paren{q\prim+1}b+r\prim\).

Donc \(m\not\in A\) : contradiction.
\end{itemize}

Ainsi on a montré \(A=\ensvide\) donc tout \(x\in\N\) admet une division euclidienne.

On montre ensuite le cas où \(a<0\).

Supposons \(a<0\). On a \(-a\in\N\).

Donc il existe \(q\in\Z\) et \(r\in\interventierii{0}{b-1}\) tels que \(-a=qb+r\).

On a \(a=-qb-r=-qb+b-b-r=\paren{-q-1}b+b-r\) avec \(-b<-r\leq0\) donc \(0<b-r\leq b\).

Donc \(a\) admet une division euclidienne par \(b\).
\end{dem}

\subsubsection{Racines nièmes d'un nombre complexe}

\begin{defi}
Soient \(z\in\C\) et \(n\in\Ns\).

Tout nombre complexe \(\gamma\in\C\) tel que \(\gamma^n=z\) est appelé une racine nième de \(z\).

Les racines nièmes de \(1\) sont appelées racines nièmes de l'unité.

Leur ensemble est noté \(\U_n\). On a \(\U_n=\accol{\omega\in\C\tq\omega^n=1}\).
\end{defi}

\begin{ex}
\begin{itemize}
\item \(\i\in\U_4\) car \(\i^4=1\) ;

\item \(\i\in\U_8\) car \(\i^8=1\) ;

\item \(1+\i\) est une racine carrée de \(2\i\) car \(\paren{1+\i}^2=2\i\) ;

\item \(-1-\i\) est aussi une racine carrée de \(2\i\) car \(\paren{-1-\i}^2=2\i\).
\end{itemize}
\end{ex}

\begin{theo}
Soit \(n\in\Ns\).

Tout nombre complexe non-nul admet exactement \(n\) racines nièmes.

Soient \(\lambda\in\Rps\) et \(\theta\in\R\).

Les racines nièmes du nombre complexe \(z=\lambda\e{\i\theta}\) sont les nombres complexes de la forme \(\gamma_k=\sqrt[n]{\lambda}\e{\i\frac{\theta+2k\pi}{n}}\) avec \(k\in\interventierii{0}{n-1}\).
\end{theo}

\begin{dem}
On rappelle \(\quantifs{\forall r_1,r_2\in\Rps;\forall\theta_1,\theta_2\in\R}r_1\e{\i\theta_1}=r_2\e{\i\theta_2}\ssi\begin{dcases}r_1=r_2 \\ \theta_1\equiv\theta_2\croch{2\pi}\end{dcases}\)

Soient \(r\in\Rp\) et \(\alpha\in\R\). Posons \(\gamma=r\e{\i\alpha}\).

On a \(\begin{aligned}[t]
\gamma^n=z&\ssi r^n\e{\i n\alpha}=\lambda\e{\i\theta} \\
&\ssi\begin{dcases}r^n=\lambda \\ n\alpha\equiv\theta\croch{2\pi}\end{dcases} \\
&\ssi\begin{dcases}r=\sqrt[n]{\lambda} \\ \alpha\equiv\dfrac{\theta}{n}\croch{\dfrac{2\pi}{n}}\end{dcases}
\end{aligned}\)

Avec \(\begin{aligned}[t]
\alpha\equiv\dfrac{\theta}{n}\croch{\dfrac{2\pi}{n}}&\ssi\quantifs{\exists l\in\Z}\alpha=\dfrac{\theta}{n}+\dfrac{2l\pi}{n} \\
&\ssi\quantifs{\exists Q\in\Z;\exists R\in\interventierii{0}{n-1}}\alpha=\dfrac{\theta}{n}+\dfrac{2\paren{Qn+R}\pi}{n} \\
&\ssi\quantifs{\exists R\in\interventierii{0}{n-1};\exists Q\in\Z}\alpha=\dfrac{\theta}{n}+\dfrac{2R\pi}{n}+2Q\pi \\
&\ssi\quantifs{\exists R\in\interventierii{0}{n-1}}\alpha\equiv\dfrac{\theta}{n}+\dfrac{2R\pi}{n}\croch{2\pi}
\end{aligned}\)

Ainsi \(\gamma^n=z\ssi\begin{dcases}r=\sqrt[n]{\lambda} \\ \quantifs{\exists R\in\interventierii{0}{n-1}}\alpha\equiv\dfrac{\theta}{n}+\dfrac{2R\pi}{n}\croch{2\pi}\end{dcases}\)

Les racines nièmes de \(z\) sont donc les complexes de la forme \(\sqrt[n]{\lambda}\e{\i\frac{\theta+2R\pi}{n}}\) où \(R\in\interventierii{0}{n-1}\).
\end{dem}

\begin{ex}
Déterminons \(\U_2\) et \(\U_3\).

\begin{itemize}
\item Les racines carrées de \(1=1\e{0\times\i}\) sont les nombres complexes de la forme \(\sqrt{1}\e{\i\frac{0+2k\pi}{2}}\) où \(k\in\interventierii{0}{1}\), c'est à dire \(1\) et \(1\times\e{\i\frac{2\pi}{2}}=\e{i\pi}=-1\).

Donc \(\U_2=\accol{-1;1}\).

\item Les racines cubiques de \(1\) sont les nombres complexes de la forme \(\e{\i\frac{2k\pi}{3}}\) où \(k\in\interventierii{0}{2}\), c'est à dire \(\e{0\times\i}=1\), \(\e{\i\frac{2\pi}{3}}=j\) et \(\e{\i\frac{4\pi}{3}}=-j\).

Donc \(\U_3=\accol{-;j;-j}\).
\end{itemize}
\end{ex}

\begin{rem}
On a \(j=\e{\i\frac{2\pi}{3}}=\cos\dfrac{2\pi}{3}+\i\sin\dfrac{2\pi}{3}=-\dfrac{1}{2}+\i\dfrac{\sqrt{3}}{2}\).

Et \(j^2=\e{\i\frac{4\pi}{3}}=\e{-\i\frac{2\pi}{3}}=j^{-1}=\conj{j}\).

Et \(1+j+j^2=0\).
\end{rem}

\begin{rem}~\\
On a \(\U_n=\accol{\e{\i\frac{2k\pi}{n}}}_{k\in\interventierii{0}{n-1}}\)
\end{rem}

\begin{rem}
Soient \(m,n\in\Ns\).

On dit que \(n\) divise \(m\) et on note \(n\divise m\) si on a \(\quantifs{\exists k\in\Z}m=kn\).

On a alors \(\U_n\subset\U_m\).
\end{rem}

\begin{dem}
Supposons \(n\divise m\). Soit \(k\in\Z\) tel que \(kn=m\).

Montrons que \(\U_n\subset\U_m\).

Soit \(\omega\in\U_n\). On a \(\omega^n=1\).

Donc \(\omega^m=\omega^{nk}=\paren{\omega^n}^k=1^k=1\).

Donc \(\omega\in\U_m\). Donc \(\U_n\subset\U_m\).
\end{dem}

\begin{prop}
Soient \(z,\gamma\in\Cs\) tels que \(\gamma\) soit une racine nième de \(z\).

Les racines nièmes de \(z\) sont les nombres complexes de la forme \(\omega\gamma\) où \(\omega\in\U_n\).
\end{prop}

\begin{dem}
Soit \(w\in\C\).

On a \(\begin{aligned}[t]
w\text{ racine nième de }z&\ssi w^n=z \\
&\ssi w^n=\gamma^n \\
&\ssi\paren{\dfrac{w}{\gamma}}^n=1 \\
&\ssi\dfrac{w}{\gamma}\in\U_n \\
&\ssi\quantifs{\exists\omega\in\U_n}\dfrac{w}{\gamma}=\omega \\
&\ssi\quantifs{\exists\omega\in\U_n}w=\omega\gamma
\end{aligned}\)

D'où le résultat.
\end{dem}

\begin{ex}
Déterminons les racines quatrièmes de \(4\) et \(-4\).

\begin{itemize}
\item On a \(4=4\e{0\times\i}\) donc les racines quatrièmes de \(4\) sont de la forme \(\sqrt[4]{4}\e{\i\frac{2k\pi+0}{4}}\) avec \(k\in\interventierii{0}{3}\).

C'est à dire \begin{itemize}
\item \(\sqrt{2}\e{0\times\i}=\sqrt{2}\) ;

\item \(\sqrt{2}\e{\i\frac{\pi}{2}}=\i\sqrt{2}\) ;

\item \(\sqrt{2}\e{\i\pi}=-\sqrt{2}\) ;

\item \(\sqrt{2}\e{\i\frac{3\pi}{2}}=-\i\sqrt{2}\).
\end{itemize}

\item On a \(-4=4\e{\i\pi}\) donc les racines quatrièmes de \(-4\) sont de la forme \(\sqrt[4]{4}\e{\i\frac{2k\pi+\pi}{4}}\) avec \(k\in\interventierii{0}{3}\).

C'est à dire \begin{itemize}
\item \(\sqrt{2}\e{\i\frac{\pi}{4}}=1+\i\) ;

\item \(\sqrt{2}\e{\i\frac{3\pi}{4}}=-1+\i\) ;

\item \(\sqrt{2}\e{\i\frac{5\pi}{4}}=-1-\i\) ;

\item \(\sqrt{2}\e{\i\frac{7\pi}{4}}=1-\i\).
\end{itemize}
\end{itemize}
\end{ex}

\subsubsection{Racines carrées sous forme algébrique}

\begin{meth}
Soit \(z=a+\i b\in\Cs\) un nombre complexe non-nul écrit sous forme algébrique avec \(a,b\in\R\) tels que \(\paren{a,b}\not=\paren{0,0}\).

Pour déterminer ses racines carrées sous forme algébrique, on résout l'équation \(\paren{x+\i y}^2=a+\i b\) d'inconnues \(x,y\in\R\) :

\begin{align*}
\paren{x+\i y}^2=a+\i b&\ssi\begin{dcases}\paren{x+\i y}^2=a+\i b \\ \abs{x+\i y}^2=\abs{a+\i b}\end{dcases} \\
&\ssi\begin{dcases}x^2-y^2+2\i xy=a+\i b \\ x^2+y^2=\sqrt{a^2+b^2}\end{dcases} \\
&\ssi\begin{dcases}x^2-y^2=a &L_1 \\ 2xy=b &L_2 \\ x^2+y^2=\sqrt{a^2+b^2} &L_3\end{dcases}
\end{align*}

On obtient \(\abs{x}\) par \(L_1+L_3\), \(\abs{y}\) par \(L_3-L_1\) et les signes par \(L_2\).
\end{meth}

\begin{ex}
Déterminons les racines carrées de \(-5-12\i\).

Soient \(x,y\in\R\). On a \(\begin{aligned}[t]
\paren{x+\i y}^2=-5-12\i&\ssi\begin{dcases}\paren{x+\i y}^2=-5-12\i \\ \abs{x+\i y}^2=\abs{-5-12\i}\end{dcases} \\
&\ssi\begin{dcases}x^2-y^2+2\i xy=-5-12\i \\ x^2+y^2=\sqrt{\paren{-5}^2+\paren{-12}^2}\end{dcases} \\
&\ssi\begin{dcases}x^2-y^2=-5 \\ 2xy=-12 \\ x^2+y^2=13\end{dcases} \\
&\ssi\begin{dcases}2x^2=8 \\ 2y^2=18 \\ xy=-6\end{dcases} \\
&\ssi\begin{dcases}\abs{x}=2 \\ \abs{y}=3 \\ xy=-6\end{dcases} \\
&\ssi\begin{dcases}x=2 \\ y=-3 \\ xy=-6\end{dcases}\text{ ou }\begin{dcases}x=-2 \\ y=3 \\ xy=-6\end{dcases}
\end{aligned}\)

Donc les racines carrées de \(-5-12\i\) sont \(2-3\i\) et \(-2+3\i\).
\end{ex}

\subsection{Équations polynomiales de degré 2}

\subsubsection{Résolution}

\begin{prop}
Soient \(a,b,c\in\C\) avec \(a\not=0\).

On associe à l'équation \(\paren{E}:az^2+bz+c=0\) d'inconnue \(z\in\C\) le discriminant \(\Delta=b^2-4ac\).

Soit \(\delta\in\C\) une racine carrée de \(\Delta\).

Les solutions de \(\paren{E}\) sont \(\dfrac{-b\pm\delta}{2a}\).

Elles sont distinctes ssi \(\Delta\not=0\).
\end{prop}

\begin{dem}
Soit \(z\in\C\).

On a \(\begin{aligned}[t]
\paren{E}&\ssi az^2+bz+c=0 \\
&\ssi a\paren{z^2+\dfrac{b}{a}z+\dfrac{b^2}{4a^2}}-\dfrac{b^2}{4a}+c=0 \\
&\ssi a\paren{\paren{z+\dfrac{b}{2a}}^2-\dfrac{b^2-4ac}{4a^2}}=0 \\
&\ssi\paren{z+\dfrac{b}{2a}}^2-\dfrac{\Delta}{4a^2}=0 \\
&\ssi\paren{z+\dfrac{b}{2a}}^2-\paren{\dfrac{\delta}{2a}}^2=0 \\
&\ssi\paren{z+\dfrac{b}{2a}-\dfrac{\delta}{2a}}\paren{z+\dfrac{b}{2a}+\dfrac{\delta}{2a}}=0 \\
&\ssi z=\dfrac{-b+\delta}{2a}\text{ ou }z=\dfrac{-b-\delta}{2a}
\end{aligned}\)

~
\end{dem}

\begin{ex}
Résolvons les équations \(\paren{A}:z^2-\paren{3+2\i}z+1+3\i=0\) et \(\paren{B}:z^2-\paren{4+\i}z+4+2\i=0\).

\begin{itemize}
\item Résolvons \(\paren{A}\).

On a \(\Delta=\paren{-3-2\i}^2-4\times1\times\paren{1+3\i}=9-2\times\paren{-3}\times2\i-4-4-12\i=1\).

Donc \(\delta=1\). Donc on a \(z=\dfrac{3+2\i\pm1}{2}=2+\i\) ou \(1+\i\).

\item Résolvons \(\paren{B}\).

On a \(\Delta=\paren{-4-\i}^2-4\times1\times\paren{4+2\i}=16-2\times\paren{-4}\times\i-1-16-8\i=-1=\i^2\).

Donc \(\delta=\i\). Donc on a \(z=\dfrac{4+\i\pm\i}{2}=2+\i\) ou \(2\).
\end{itemize}
\end{ex}

\begin{rem}
Si \(\Delta\in\Rm\) alors une racine carrée de \(\Delta\) est \(\i\sqrt{-\Delta}\) donc les solutions sont \(\dfrac{-b\pm\i\sqrt{-\Delta}}{2a}\).
\end{rem}

\begin{rem}
Si \(a,b,c\in\R\) alors \(\Delta\in\R\). On obtient \begin{itemize}
\item deux solutions complexes (non-réelles) conjuguées si \(\Delta<0\) ;

\item une solution réelle si \(\Delta=0\) ;

\item deux solutions réelles distinctes si \(\Delta>0\).
\end{itemize}
\end{rem}

\begin{prop}[Relations coefficients/racines]
Soient \(a,b,c\in\C\) avec \(a\not=0\).

On note \(z_1,z_2\in\C\) les racines de \(f:z\mapsto az^2+bz+c\) et \(\Delta\) le discriminant de \(f\).

On a vu que \(\quantifs{\forall z\in\C}f\paren{z}=a\paren{z-z_1}\paren{z-z_2}\).

On a \(z_1+z_2=-\dfrac{b}{a}\) et \(z_1z_2=\dfrac{c}{a}\).
\end{prop}

\begin{dem}
On note \(\delta\) une racine carrée de \(\Delta\).

On a \(z_1=\dfrac{-b-\delta}{2a}\) et \(z_2=\dfrac{-b+\delta}{2a}\).

On a \(z_1+z_2=\dfrac{-b-\delta-b+\delta}{2a}=-\dfrac{2b}{2a}=-\dfrac{b}{a}\) et \(z_1z_2=\paren{\dfrac{-b-\delta}{2a}}\paren{\dfrac{-b+\delta}{2a}}=\dfrac{b^2-\delta^2}{4a^2}=\dfrac{b^2-\paren{b^2-4ac}}{4a^2}=\dfrac{c}{a}\).
\end{dem}

\begin{rem}
Il est conseillé de vérifier au moins la somme des racines quand on les calcule.
\end{rem}

\begin{prop}
Soient \(z_1,z_2,S,P\in\C\).

Alors \(\begin{dcases}z_1+z_2=S \\ z_1z_2=P\end{dcases}\ssi z_1,z_2\) sont les racines de \(\fonctionlambda{\C}{\C}{z}{z^2-Sz+P}\)
\end{prop}

\begin{dem}
Montrons l'équivalence :

\begin{itemize}
\item[\imprec] Déjà vue.

\item[\impdir] Supposons \(z_1+z_2=S\) et \(z_1z_2=P\).

Alors \(\begin{aligned}[t]
\quantifs{\forall z\in\C}z^2-Sz+P&=z^2-\paren{z_1+z_2}z+z_1z_2 \\
&=z^2-z_1z-z_2z+z_1z_2 \\
&=\paren{z-z_1}\paren{z-z_2}
\end{aligned}\)
\end{itemize}

~
\end{dem}

\begin{ex}~\\
Résolvons le système \(\begin{dcases}a+b=4 \\ ab=5\end{dcases}\) d'inconnues \(a,b\in\C\).

On a \(\begin{dcases}a+b=4 \\ ab=5\end{dcases}\ssi a,b\) sont les racines de \(\fonction{f}{\C}{\C}{z}{z^2-4z+5}\)

On note \(\Delta\) le discriminant de \(f\). On a \(\Delta=\paren{-4}^2-4\times1\times5=16-20=-4\).

Donc on a \(\begin{dcases}a=\dfrac{4+\i\sqrt{4}}{2}=2+\i \\ b=\dfrac{4-\i\sqrt{4}}{2}=2-\i\end{dcases}\) ou \(\begin{dcases}a=2-\i \\ b=2+\i\end{dcases}\)

On a donc \(\S=\accol{\paren{2+\i;2-\i};\paren{2-\i;2+\i}}\).
\end{ex}

\section{Exponentielle complexe}

\begin{defi}\thlabel{defi:expComplexe}
Soit \(z=x+\i y\in\C\) avec \(x,y\in\R\).

On pose \(\exp z=\e{z}=\e{x}\e{\i y}\).

On appelle exponentielle complexe la fonction \(\fonction{\exp}{\C}{\C}{z}{\e{z}}\)
\end{defi}

\begin{prop}
On a \begin{itemize}
\item \(\e{0}=1\) ;

\item \(\quantifs{\forall z\in\C}\abs{\e{z}}=\e{\Re z}\) ;

\item \(\quantifs{\forall z,z\prim\in\C}\e{z+z\prim}=\e{z}\e{z\prim}\) ;

\item \(\quantifs{\forall z\in\C}\e{z}\not=0\) et \(\dfrac{1}{\e{z}}=\e{-z}\) ;

\item \(\quantifs{\forall z\in\C;\forall n\in\Z}\paren{\e{z}}^n=\e{nz}\) ;

\item \(\quantifs{\forall z,z\prim\in\C}\e{z}=\e{z\prim}\ssi z\equiv z\prim\croch{2i\pi}\) ;

\item \(\exp\paren{\C}=\Cs\).
\end{itemize}
\end{prop}

\begin{dem}
\note{EXERCICE}
\end{dem}

\begin{rem}
La \thref{defi:expComplexe} est compatible avec les précédentes définitions de l'exponentielle.
\end{rem}

\begin{defi}
Soient \(I\subset\R\) et \(f:I\to\C\).

On appelle partie réelle de \(f\) la fonction \(\fonction{\Re\paren{f}}{I}{\R}{t}{\Re\paren{f\paren{t}}}\)

On appelle partie imaginaire de \(f\) la fonction \(\fonction{\Im\paren{f}}{I}{\R}{t}{\Im\paren{f\paren{t}}}\)
\end{defi}

\begin{rem}
De même que la donnée d'un nombre complexe revient à celle de deux réels (ses parties réelle et imaginaire), la donnée d'une fonction à valeurs complexes revient à celle de deux fonctions à valeurs réelles (ses parties réelle et imaginaire).

C'est à dire \(\quantifs{\forall t\in I}f\paren{t}=\paren{\Re\paren{f}}\paren{t}+\i\paren{\Im\paren{f}}\paren{t}\).
\end{rem}

\begin{defi}
Soient \(I\) un intervalle de \(\R\) et \(f:I\to\C\).

On dit que \(f\) est continue en un point \(a\in I\) si \(\Re\paren{f}\) et \(\Im\paren{f}\) sont continues en \(a\).

On dit que \(f\) est continue (sur \(I\)) si elle est continue en tout point de \(I\), c'est à dire si \(\Re\paren{f}\) et \(\Im\paren{f}\) sont continues sur \(I\).

On dit que \(f\) est dérivable en un point \(a\in I\) si \(\Re\paren{f}\) et \(\Im\paren{f}\) le sont.

On dit que \(f\) est dérivable (sur \(I\)) si elle est dérivable en tout point de \(I\), c'est à dire si \(\Re\paren{f}\) et \(\Im\paren{f}\) sont dérivables sur \(I\).

Quand \(f\) est dérivable en un point \(a\in I\), on pose \(f\prim\paren{a}=\paren{\Re\paren{f}}\prim\paren{a}+\i\paren{\Im\paren{f}}\prim\paren{a}\).
\end{defi}

\begin{theo}\thlabel{theo:derivExpComplexe}
Soient \(I\) un intervalle de \(\R\) et \(f:I\to\C\) une fonction dérivable sur \(I\).

Alors la fonction \(\fonction{\exp\rond f}{I}{\C}{t}{\e{f\paren{t}}}\) est dérivable et \(\paren{\exp\rond f}\prim=f\prim\times\exp\paren{f}\).
\end{theo}

\begin{dem}
Posons \(f_1=\Re\paren{f}\) et \(f_2=\Im\paren{f}\).

On a \(\begin{aligned}[t]
\quantifs{\forall t\in I}\paren{\exp\rond f}\paren{t}&=\e{f_1\paren{t}+\i f_2\paren{t}} \\
&=\e{f_1\paren{t}}\e{\i f_2\paren{t}} \\
&=\e{f_1\paren{t}}\paren{\cos\paren{f_2\paren{t}}+\i\sin\paren{f_2\paren{t}}} \\
&=\underbrace{\e{f_1\paren{t}}\cos\paren{f_2\paren{t}}}_{\in\R}+\i\underbrace{\e{f_1\paren{t}}\sin\paren{f_2\paren{t}}}_{\in\R}
\end{aligned}\)

D'où \(\quantifs{\forall t\in I}\begin{dcases}\paren{\Re\paren{\exp\rond f}}\paren{t}=\e{f_1\paren{t}}\cos\paren{f_2\paren{t}} \\ \paren{\Im\paren{\exp\rond f}}\paren{t}=\e{f_1\paren{t}}\sin\paren{f_2\paren{t}}\end{dcases}\)

La fonction à valeurs complexes \(\exp\rond f\) est donc dérivable (car ses parties réelle et imaginaire le sont) et on a pour tout \(t\in I\) :

\begin{align*}
\paren{\exp\rond f}\prim\paren{t}&=\paren{f_1\prim\paren{t}\e{f_1\paren{t}}\cos\paren{f_2\paren{t}}-\e{f_1\paren{t}}f_2\prim\paren{t}\sin\paren{f_2\paren{t}}}\notag\\
&\qquad+\i\paren{f_1\prim\paren{t}\e{f_1\paren{t}}\sin\paren{f_2\paren{t}}+\e{f_1\paren{t}}f_2\prim\paren{t}\cos\paren{f_2\paren{t}}} \\
&=\e{f_1\paren{t}}\paren{f_1\prim\paren{t}\croch{\cos\paren{f_2\paren{t}}+\i\sin\paren{f_2\paren{t}}}+f_2\prim\paren{t}\croch{-\sin\paren{f_2\paren{t}}+\i\cos\paren{f_2\paren{t}}}} \\
&=\e{f_1\paren{t}}\paren{f_1\prim\paren{t}\e{\i f_2\paren{t}}+f_2\prim\paren{t}\i\e{\i f_2\paren{t}}} \\
&=\e{f_1\paren{t}}\e{\i f_2\paren{t}}\paren{f_1\prim\paren{t}+\i f_2\prim\paren{t}} \\
&=\e{f\paren{t}}\times f\prim\paren{t}
\end{align*}

~
\end{dem}

\begin{ex}
Soit \(a\in\C\).

La fonction \(\fonction{g}{\R}{\C}{t}{\e{at}}\) est dérivable sur \(\R\) et on a \(\forall t\in\R,g\prim\paren{t}=a\e{at}\).
\end{ex}

\begin{dem}
Cela découle du \thref{theo:derivExpComplexe}, en prenant \(\fonction{f}{\R}{\C}{t}{at}\).
\end{dem}

\begin{theo}
Soient \(f,g\) deux fonctions de \(I\) dans \(\C\), où \(I\) est un intervalle de \(\R\).

Si \(f\) et \(g\) sont dérivables sur \(I\), alors \(f+g\) et \(fg\) le sont aussi et on a \[\paren{f+g}\prim=f\prim+g\prim\text{ et }\paren{fg}\prim=f\prim g+fg\prim\]

Si, de plus, \(g\) ne s'annule pas sur \(I\), alors \(\dfrac{f}{g}\) est dérivable sur \(I\) et on a \[\paren{\dfrac{f}{g}}\prim=\dfrac{f\prim g-fg\prim}{g^2}\]
\end{theo}

\begin{dem}
\note{EXERCICE}
\end{dem}

\begin{theo}
Soient \(I,J\) deux intervalles de \(\R\) et \(f:I\to J\) et \(g:J\to\C\) deux fonctions dérivables.

Alors \(g\rond f\) est dérivable et on a \(\paren{g\rond f}\prim=f\prim\times\paren{g\prim\rond f}\).
\end{theo}

\begin{dem}
\note{EXERCICE}
\end{dem}

\chapter{Notions ensemblistes}

\minitoc

\section{Ensembles}

\subsection{Notations de base}

Ensemble vide : \(\ensvide\) ou \(\accol{}\).

Singletons (ensembles avec un unique élément) : \(\accol{1}\), \(\accol{\sin}\), \(\accol{\ensvide}\), ...

Ensembles finis : \(\accol{1;2;3}\), \(\accol{0;\ensvide;\cos;\R}\), ...

Ensembles classiques (infinis) : \(\N\), \(\Z\), \(\Q\), \(\R\), \(\C\), \(\Qp\), \(\Rps\), ...

Sous-ensembles définis par une condition : \(\accol{x\in\R\tq x\geq0}=\Rp\), ...

Ensembles paramétrés par un autre ensemble : \(\accol{x+\i y}_{\paren{x,y}\in\R^2}\), \(\accol{\cos+\lambda\sin}_{\lambda\in\Rp}\), \(\accol{\fonctionlambda{\R}{\R}{x}{ax+b}}_{\paren{a,b}\in\R^2}\), ...

Appartenance : \guillemets{\(x\in E\)} signifie que l'élément \(x\) appartient à l'ensemble \(E\).

Inclusion : \guillemets{\(A\subset B\)} signifie \(\quantifs{\forall x\in A}x\in B\).

Égalité : \(A=B\ssi\begin{dcases}A\subset B \\ B\subset A\end{dcases}\)

\begin{rem}~\\
\(\begin{dcases}A\subset B \\ B\subset C\end{dcases}\imp A\subset C\).
\end{rem}

\subsection{Opérations sur les ensembles}

\subsubsection{Réunion}

Si \(A,B\) sont deux ensembles, alors leur réunion est l'ensemble des éléments qui appartiennent à \(A\) ou à \(B\).

Elle est définie par \(x\in A\union B\ssi\paren{x\in A\ou x\in B}\) pour tout élément \(x\).

La réunion \(\union\) est une loi associative : \(\paren{A\union B}\union C=A\union\paren{B\union C}\) pour tous ensembles \(A,B,C\). En pratique, on note simplement \(A\union B\union C\).

Si \(I\) est un ensemble et \(\paren{A_i}_{i\in I}\) est une famille d'ensembles indicée par \(I\), alors on définit la réunion \(\bigunion_{i\in I}A_i\) par \(x\in\bigunion_{i\in I}A_i\ssi\exists i\in I,x\in A_i\).

\begin{ex}
\(\bigunion_{n\in\N}\accol{n}=\N\).
\end{ex}

\subsubsection{Intersection}

Si \(A,B\) sont deux ensembles, alors leur intersection est l'ensemble des éléments qui appartiennent à \(A\) et à \(B\). Elle est définie par \(x\in A\inter B\ssi\paren{x\in A\et x\in B}\) pour tout élément \(x\).

L'intersection \(\inter\) est une loi associative : \(\paren{A\inter B}\inter C=A\inter\paren{B\inter C}\) pour tous ensembles \(A,B,C\). En pratique, on note simplement \(A\inter B\inter C\).

Si \(I\) est un ensemble et \(\paren{A_i}_{i\in I}\) est une famille d'ensembles indicées par \(I\), alors on définit l'intersection \(\biginter_{i\in I}A_i\) par \(x\in\biginter_{i\in I}A_i\ssi\forall i\in I,x\in A_i\).

\begin{ex}
\(\biginter_{n\in\N}\intervie{n}{\pinf}=\ensvide\)
\end{ex}

\begin{dem}
\begin{itemize}
\item[\increc] Claire.

\item[\incdir] Supposons \(\biginter_{n\in\N}\intervie{n}{\pinf}\).

On a \(\quantifs{\forall n\in\N}n\leq x\). En particulier, \(\floor{x}+1\leq x\) : impossible.
\end{itemize}
\end{dem}

\begin{ex}~\\
\(\biginter_{n\in\Ns}\intervii{-\dfrac{1}{n}}{\dfrac{1}{n}}=\accol{0}\)
\end{ex}

\begin{dem}
\begin{itemize}
\item[\increc] Claire : on a bien \(\quantifs{\forall n\in\Ns}-\dfrac{1}{n}\leq0\leq\dfrac{1}{n}\).

\item[\incdir] Soit \(x\in\biginter_{n\in\Ns}\intervii{-\dfrac{1}{n}}{\dfrac{1}{n}}\). Montrons que \(x\in\accol{0}\), c'est à dire \(x=0\).

On a \(\quantifs{\forall n\in\Ns}x\in\intervii{-\dfrac{1}{n}}{\dfrac{1}{n}}\).

Donc \(\quantifs{\forall n\in\Ns}-\dfrac{1}{n}\leq x\leq\dfrac{1}{n}\).

Donc \(\quantifs{\forall n\in\Ns}\abs{x}\leq\dfrac{1}{n}\).

Supposons \(x\not=0\). Alors on a \(\quantifs{\forall n\in\Ns}n\leq\dfrac{1}{\abs{x}}\).

En particulier, \(\floor{\dfrac{1}{\abs{x}}}+1\leq\dfrac{1}{\abs{x}}\) : contradiction.

Donc par l'absurde, \(x=0\), d'où l'inclusion.
\end{itemize}
\end{dem}

\begin{rem}
Si \(I\) est un ensemble et \(\paren{A_i}_{i\in I}\) est une famille d'ensembles indicée par \(I\), alors \(\quantifs{\forall j\in I}\biginter_{i\in I}A_i\subset A_j\subset\bigunion_{i\in I}A_i\).
\end{rem}

\subsection{Produit cartésien}

\begin{rappel}
Soient \(E_1,\dots,E_n\) des ensembles avec \(n\in\Ns\). Le produit cartésien \(E_1\times\dots\times E_n\) est l'ensemble des n-uplets \(\paren{x_1,\dots,x_n}\) tels que \(\quantifs{\forall i\in\interventierii{1}{n}}x_i\in E_i\).

Alors si \(E_1,\dots,E_m,F_1,\dots,F_n\) sont des ensembles avec \(m,n\in\Ns\), on identifie les ensembles \(\paren{E_1\times\dots\times E_m}\times\paren{F_1\times\dots\times F_n}\) et \(E_1\times\dots\times E_m\times F_1\times\dots\times F_n\). Autrement dit : \[\quantifs{\forall x_1\in E_1,\dots,\forall x_m\in E_m;\forall y_1\in F_1,\dots,\forall y_n\in F_n}\croch{\paren{x_1,\dots,x_m},\paren{y_1,\dots,y_n}}=\paren{x_1,\dots,x_m,y_1,\dots,y_n}.\]
\end{rappel}

\subsection{Autres opérations}

\begin{defi}
Soit \(E\) un ensemble. Une partie de \(E\) est un ensemble inclus dans \(E\). L'ensemble des parties de \(E\) est noté \(\P{E}\).
\end{defi}

\begin{defi}[Différence ensembliste]
Soient \(A,B\) deux ensembles.

On pose \(A\excluant B=\accol{x\in A\tq x\not\in B}\).
\end{defi}

\begin{ex}
\(\R\excluant\Rm=\Rps\)
\end{ex}

\begin{defi}[Complémentaire]
Soient \(E\) un ensemble et \(A\in\P{E}\).

Le complémentaire de \(A\) (dans \(E\)) est \(E\excluant A\).

Il est noté \(\complement_E A\) ou \(\conj{A}\) ou \(A^C\).
\end{defi}

\subsection{Partitions}

\begin{defi}
Soient \(E\) un ensemble et \(P\subset\P{E}\) un ensemble de parties de \(E\).

On dit que \(P\) est un recouvrement disjoint de \(E\) si on a \(\begin{dcases}\bigunion_{A\in P}A=E\text{ (recouvrement)} \\ \quantifs{\forall A,B\in P}A\not=B\text{ (disjoints deux à deux)}\end{dcases}\)
\end{defi}

\begin{ex}
Si \(E=\interventierii{1}{5}\) alors \(\accol{\accol{1};\accol{2;4};\ensvide;\accol{3;5}}\) est un recouvrement disjoint de \(E\).
\end{ex}

\begin{rem}
On note alors \(E=\bigsqcup_{A\in P}A\).
\end{rem}

\begin{defi}
Soit \(E\) un ensemble et \(P\subset\P{E}\). On dit que \(P\) est une partition de \(E\) si \(P\) est un recouvrement disjoint de \(E\) et \(\ensvide\not\in P\).
\end{defi}

\begin{ex}
Si \(E=\ensvide\) alors \(\P{E}=\accol{\ensvide}\) et \(P=\ensvide\) est l'unique partition de \(\ensvide\).

Si \(E=\accol{1}\) alors \(\P{E}=\accol{\accol{1};\ensvide}\) et \(P=\accol{1}\) est l'unique partition de \(E\).

Si \(E=\accol{1;2}\) alors il existe deux partitions de \(E\) : \(\accol{\accol{1;2}}\) et \(\accol{\accol{1};\accol{2}}\).

Si \(E=\accol{1;2;3}\) alors il existe cinq partitions de \(E\) : \(\accol{\accol{1;2;3}}\), \(\accol{\accol{1;2};\accol{3}}\), \(\accol{\accol{1;3};\accol{2}}\), \(\accol{\accol{2;3};\accol{1}}\) et \(\accol{\accol{1};\accol{2};\accol{3}}\).

Si \(E=\R\), on a notamment les partitions \(\accol{\Rms;\accol{0};\Rps}\) et \(\accol{\R\excluant\Z}\union\bigunion_{n\in\Z}\accol{\accol{n}}\).
\end{ex}

\subsection{Droite réelle achevée}

La droite réelle achevée est l'ensemble \(\Rb=\intervii{\minf}{\pinf}=\R\union\accol{\minf;\pinf}\).

\subsubsection{Somme de deux éléments de \(\Rb\)}

La somme \(a+b\) est définie pour tout couple \(\paren{a,b}\in\Rb^2\excluant\accol{\paren{\minf,\pinf};\paren{\pinf,\minf}}\).

Si \(\paren{a,b}\in\R^2\) alors \(a+b\) est simplement la somme des réels \(a\) et \(b\).

Sinon, on pose \(\begin{dcases}
\paren{\pinf}+\paren{\pinf}=\pinf \\
\quantifs{\forall x\in\R}\paren{\pinf}+x=x+\paren{\pinf}=\pinf \\
\paren{\minf}+\paren{\minf}=\minf \\
\quantifs{\forall x\in\R}\paren{\minf}+x=x+\paren{\minf}=\minf
\end{dcases}\)

\subsubsection{Produit de deux éléments de \(\Rb\)}

Le produit \(a\times b\) est défini pour tout couple \(\paren{a,b}\in\Rb^2\excluant\accol{\paren{\minf,0};\paren{\pinf,0};\paren{0,\minf};\paren{0,\pinf}}\).

Si \(\paren{a,b}\in\R^2\) alors \(a\times b\) est simplement le produit des réels \(a\) et \(b\).

Sinon, on pose \(\begin{dcases}
\paren{\pinf}\times\paren{\pinf}=\paren{\minf}\times\paren{\minf}=\pinf \\
\paren{\minf}\times\paren{\pinf}=\paren{\pinf}\times\paren{\minf}=\minf \\
\quantifs{\forall x\in\Rps}\paren{\pinf}\times x=x\times\paren{\pinf}=\pinf\quad\text{et}\quad\paren{\minf}\times x=x\times\paren{\minf}=\minf \\
\quantifs{\forall x\in\Rms}\paren{\pinf}\times x=x\times\paren{\pinf}=\minf\quad\text{et}\quad\paren{\minf}\times x=x\times\paren{\minf}=\pinf
\end{dcases}\)

\subsubsection{Relation d'ordre sur \(\Rb\)}

On munit l'ensemble \(\Rb\) de la relation d'ordre notée \(\leq\) qui prolonge la relation d'ordre usuelle de \(\R\) et pour laquelle \(\pinf\) est le plus grand élément de \(\Rb\) et \(\minf\) le plus petit élément de \(\Rb\).

On a donc \(\quantifs{\forall x\in\Rb}\minf\leq x\leq\pinf\).

C'est un ordre total.

\section{Fonctions}

\subsection{Notations de base}

Les termes \guillemets{fonction} et \guillemets{application} sont synonymes.

\begin{defi}
Une fonction \(f\) d'un ensemble \(E\) dans un ensemble \(F\) associe à tout élément \(x\) de \(E\) un unique élément de \(F\) noté \(f\paren{x}\) : \(\quantifs{\forall x\in E;\exists! y\in F}y=f\paren{x}\).

On dit que \(E\) et \(F\) sont respectivement l'ensemble de départ (ou de définition) et d'arrivée de \(f\).
\end{defi}

\begin{defi}
Soient \(E,F\) deux ensembles.

L'ensemble des fonctions de \(E\) dans \(F\) est noté \(\F{E}{F}\) ou \(F^E\).
\end{defi}

\begin{defi}[Ensemble image]
Soient \(E,F\) deux ensembles et \(f\in\F{E}{F}\).

On pose \(\Im f=\accol{f\paren{x}}_{x\in E}\).
\end{defi}

\begin{rem}
Il ne faut pas confondre \(\Im f\) et \(F\) : on a \(\Im f\subset F\).
\end{rem}

\begin{defi}[Famille d'éléments d'un ensemble]
Soient \(I,E\) deux ensembles.

Une famille d'éléments de \(E\) indicée par \(I\) est un objet \(\paren{x_i}_{i\in I}\) où \(\quantifs{\forall i\in I}x_i\in E\).

L'ensemble des familles d'éléments de \(E\) indicées par \(I\) est noté \(E^I\).
\end{defi}

\begin{defi}
Soient \(E\) un ensemble et \(A\subset E\).

La fonction indicatrice de \(A\) est la fonction \(\fonction{\ind{A}}{E}{\accol{0;1}}{x}{\begin{dcases}1\text{ si }x\in A \\ 0\text{ sinon}\end{dcases}}\)
\end{defi}

\begin{ex}
\(\ind{\Rps}\) :

\begin{center}
\begin{tkz}
\draw[gray,->] (0,-1) -- (0,2);
\draw[gray,->] (-3,0) -- (3,0);
\draw[{}-{Arc Barb [reversed,length=0.1cm]}] (3,1) -- (-0.1,1) node[left] {\(1\)};
\draw (-3,0) -- (0,0) node[below left] {\(0\)};
\filldraw (0,0) circle (3pt);
\end{tkz}
\end{center}
\end{ex}

\begin{defi}[Image directe, image réciproque]
Soient \(E,F\) deux ensembles et \(f:E\to F\).

Soient \(A\subset E\) et \(B\subset F\).

L'image directe de \(A\) par \(f\) est \(f\paren{A}=\accol{f\paren{x}}_{x\in A}\subset F\).

On a aussi \(f\paren{A}=\accol{y\in F\tq\exists x\in A,y=f\paren{x}}\).

En pratique, \(f\paren{A}\subset F\) et \(\quantifs{\forall y\in F}y\in f\paren{A}\ssi\quantifs{\exists x\in A}f\paren{x}=y\).

L'image réciproque de \(B\) par \(f\) est \(f^{-1}\paren{B}=\accol{x\in E\tq f\paren{x}\in B}\).

En pratique, \(f^{-1}\paren{B}\subset E\) et \(\quantifs{\forall x\in E}x\in f^{-1}\paren{B}\ssi f\paren{x}\in B\).
\end{defi}

\begin{rem}
Mêmes notations.

On a \(\Im f=f\paren{E}\).

Soient \(x\in E\) et \(y\in F\). Si \(y=f\paren{x}\), on dit que \(y\) est l'image de \(x\) par \(f\) et que \(x\) est un antécédent de \(y\) par \(f\).

\(f\paren{A}\) est l'ensemble des images des éléments de \(A\).

\(f^{-1}\paren{B}\) est l'ensemble des antécédents des éléments de \(B\).
\end{rem}

\begin{ex}
Considérons la fonction \(\sin:\R\to\R\).

On a \begin{itemize}
\item \(\Im\sin=\sin\paren{\R}=\intervii{-1}{1}\) ;

\item \(\sin\paren{\Rp}=\intervii{-1}{1}\) ;

\item \(\sin\paren{\intervii{0}{\pi}}=\intervii{0}{1}\) ;

\item \(\sin^{-1}\paren{\accol{0}}=\accol{k\pi}_{k\in\Z}=\pi\Z\)

En effet, \(\begin{aligned}[t]
\forall x\in\R,x\in\sin^{-1}\paren{\accol{0}}&\iff\sin x\in\accol{0} \\
&\iff\sin x=0 \\
&\iff x\equiv0\croch{\pi} \\
&\iff\quantifs{\exists k\in\Z}x=k\pi
\end{aligned}\)

\item \(\sin^{-1}\paren{\intervie{2}{\pinf}}=\ensvide\).
\end{itemize}
\end{ex}

\begin{defi}
Soient \(E,F\) deux ensembles et \(f:E\to F\).

Soit \(A\subset E\).

On appelle restriction de \(f\) à \(A\) la fonction \(\fonction{\restr{f}{A}}{A}{F}{x}{f\paren{x}}\)
\end{defi}

\begin{defi}
Soient \(E,E\prim,F\) trois ensembles tels que \(E\subset E\prim\).

Soient \(f:E\to F\) et \(g:E\prim\to F\).

On dit que \(g\) est un prolongement de \(f\) à \(E\prim\) si on a \(f=\restr{g}{E}\)
\end{defi}

\begin{defi}[Composition]
Soient \(E,F,G\) trois ensembles. Soient \(f:E\to F\) et \(g:F\to G\).

On définit la composée de \(f\) et \(g\) : \(\fonction{g\rond f}{E}{G}{x}{g\paren{f\paren{x}}}\)
\end{defi}

\begin{rem}
Soient \(E,F,G,H\) quatre ensembles. Soient \(f:E\to F\), \(g:F\to G\) et \(h:G\to H\).

On a \(h\rond\paren{g\rond f}=\paren{h\rond g}\rond f\).

On note \(h\rond g\rond f\).
\end{rem}

\subsection{Injectivité, surjectivité, bijectivité}

\subsubsection{Injectivité}

\begin{defi}
Soient \(E,F\) deux ensembles et \(f:E\to F\).

On dit que \(f\) est injective ou que \(f\) est une injection si on a \(\quantifs{\forall x,y\in E}f\paren{x}=f\paren{y}\imp x=y\).
\end{defi}

\begin{ex}
La fonction \(\exp:\R\to\R\) est injective car \(\quantifs{\forall x,y\in \R}\e{x}=\e{y}\imp x=y\).

La fonction \(\sin:\R\to\R\) n'est pas injective car \(\begin{dcases}\sin0=\sin\pi \\ 0\not=\pi\end{dcases}\)
\end{ex}

\begin{rem}
On a aussi \(f\) injective \(\iff\paren{\quantifs{\forall x,y\in E}f\paren{x}=f\paren{y}\iff x=y}\).
\end{rem}

\begin{prop}
Soient \(I\subset\R\) et \(f:I\to\R\) une fonction strictement monotone.

Alors \(f\) est injective.
\end{prop}

\begin{dem}
Soient \(x,y\in I\) tels que \(x\not=y\). Montrons que \(f\paren{x}\not=f\paren{y}\).

Si \(f\) est strictement croissante et \(x<y\) alors \(f\paren{x}<f\paren{y}\) donc \(f\paren{x}\not=f\paren{y}\).

Si \(f\) est strictement croissante et \(x>y\) alors \(f\paren{x}>f\paren{y}\) donc \(f\paren{x}\not=f\paren{y}\).

Si \(f\) est strictement décroissante et \(x<y\) alors \(f\paren{x}>f\paren{y}\) donc \(f\paren{x}\not=f\paren{y}\).

Si \(f\) est strictement décroissante et \(x>y\) alors \(f\paren{x}<f\paren{y}\) donc \(f\paren{x}\not=f\paren{y}\).

Donc \(f\paren{x}\not=f\paren{y}\).

Donc \(f\) injective.
\end{dem}

\begin{prop}[Une composée d'injections est une injection]
Soient \(E,F,G\) trois ensembles. Soient \(f:E\to F\) et \(g:F\to G\) deux injections.

Alors \(g\rond f\) est une injection.
\end{prop}

\begin{dem}
Soient \(x,y\in E\) tels que \(g\rond f\paren{x}=g\rond f\paren{y}\).

Montrons que \(x=y\).

Comme \(g\) injective, on a \(f\paren{x}=f\paren{y}\).

Comme \(f\) injective, on a \(x=y\).

Donc \(g\rond f\) est une injection.
\end{dem}

\subsubsection{Surjectivité}

\begin{defi}
Soient \(E,F\) deux ensembles et \(f:E\to F\).

On dit que \(f\) est surjective ou que \(f\) est une surjection si on a \(\Im f=F\), c'est à dire si \(\quantifs{\forall y\in F;\exists x\in E}f\paren{x}=y\).

Ces deux conditions sont équivalentes car on a \(\Im f\subset F\).

Donc \(\begin{aligned}[t]
\Im f=F&\iff F\subset\Im f \\
&\iff\quantifs{\forall y\in F}y\in\Im f \\
&\iff\quantifs{\forall y\in F;\exists x\in E}y=f\paren{x}
\end{aligned}\)

NB : quand on parle de la surjectivité d'une fonction, il faut que l'ensemble d'arrivée de la fonction soit clair.
\end{defi}

\begin{ex}
\(\sin:\R\to\R\) n'est pas surjective car \(\Im\sin=\intervii{-1}{1}\not=\R\).

\(\sin:\R\to\intervii{-1}{1}\) est une surjection car \(\sin\paren{\R}=\intervii{-1}{1}\).
\end{ex}

\begin{prop}[Une composée de surjections est une surjection]
Soient \(E,F,G\) trois ensembles. Soient \(f:E\to F\) et \(g:F\to G\) deux surjections.

Alors \(g\rond f\) est une surjection de \(E\) dans \(G\).
\end{prop}

\begin{dem}
Montrons que \(\quantifs{\forall z\in G;\exists x\in E}g\rond f\paren{x}=z\).

Soit \(z\in G\).

Comme \(g\) est une surjection, il existe \(y\in F\) tel que \(g\paren{y}=z\).

Comme \(f\) est une surjection, il existe \(x\in E\) tel que \(g\paren{x}=y\).

On a \(z=g\paren{y}=g\paren{f\paren{x}}=g\rond f\paren{x}\).

Donc \(g\rond f\) est surjective.
\end{dem}

\subsubsection{Bijectivité}

\begin{defi}
Soient \(E,F\) deux ensembles et \(f:E\to F\).

On dit que \(f\) est bijective ou que \(f\) est une bijection de \(E\) dans \(F\) si elle est injective et surjective, c'est à dire si on a \(\quantifs{\forall y\in F;\exists! x\in E}f\paren{x}=y\).
\end{defi}

\begin{ex}
\(\exp\) est une bijection de \(\R\) dans \(\Rps\).

\(\ln\) est une bijection de \(\Rps\) dans \(\R\).

\(\sin\) n'est pas une injection et donc pas une bijection.

Mais \(\restr{\sin}{\intervii{-\nicefrac{\pi}{2}}{\nicefrac{\pi}{2}}}\) est une bijection de \(\intervii{-\dfrac{\pi}{2}}{\dfrac{\pi}{2}}\) dans \(\intervii{-1}{1}\).
\end{ex}

\begin{rem}
Soit \(f:E\to F\) une injection avec \(E,F\) deux ensembles.

Alors \(f\) est une bijection de \(E\) dans \(\Im f\).
\end{rem}

\begin{prop}[Une composée de bijections est une bijection]\thlabel{prop:composeeBijectionsEstUneBijection}
Soient \(E,F,G\) trois ensembles. Soient \(f:E\to F\) et \(g:F\to G\) deux bijections.

Alors \(g\rond f:E\to G\) est une bijection.
\end{prop}

\begin{dem}
Comme \(g\) et \(f\) sont des injections, \(g\rond f\) est une injection.

Comme \(g\) et \(f\) sont des surjections, \(g\rond f\) est une surjection.

Donc \(g\rond f\) est une bijection.
\end{dem}

\begin{defprop}[Bijection réciproque]\thlabel{defprop:bijRec}
Soient \(E,F\) deux ensembles et \(f:E\to F\) une bijection. On a \(\quantifs{\forall y\in F;\exists! x\in E}f\paren{x}=y\).

Pour tout \(y\in F\), on note \(f^{-1}\paren{y}\) l'unique antécédent de \(y\) par \(f\).

On obtient alors une fonction \(f^{-1}:F\to E\) appelée bijection réciproque de \(f\).

La fonction \(f^{-1}:F\to E\) est une bijection et vérifie \(\begin{dcases}f\rond f^{-1}=\id{F} \\ f^{-1}\rond f=\id{E}\end{dcases}\)
\end{defprop}

\begin{oubli}
Soit \(E\) un ensemble.

On pose \(\fonction{\id{E}}{E}{E}{x}{x}\) la \guillemets{fonction identité} de \(E\).
\end{oubli}

\begin{rem}
Soit \(E\) un ensemble.

Ne pas confondre : \begin{itemize}
\item si \(A\subset E\), \(f^{-1}\paren{A}\) est l'image réciproque de \(A\) par \(f\) ;

\item si \(y\in E\) et \(f\) bijective, \(f^{-1}\paren{y}\) est l'antécédent de \(y\) par \(f\).
\end{itemize}
\end{rem}

\begin{dem}[De la \thref{defprop:bijRec}]
\begin{itemize}
\item Montrons que \(f^{-1}:F\to E\) est une bijection.

On a \(\begin{aligned}[t]
f^{-1}\text{ bijection}&\ssi\quantifs{\forall x\in E;\exists! y\in F}f^{-1}\paren{y}=x \\
&\ssi\quantifs{\forall x\in E;\exists! y\in F}f\paren{x}=y \\
&\color{white}\iff\color{black}\text{ce qui est vrai}
\end{aligned}\)

\item Pour tout \(y\in F\), \(f^{-1}\paren{y}\) est l'antécédent de \(y\) par \(f\).

Donc \(\quantifs{\forall y\in F}f\paren{f^{-1}\paren{y}}=y\).

Donc \(f\rond f^{-1}=\id{F}\).

De même, pour tout \(x\in E\), l'antécédent de \(f\paren{x}\) par \(f\) est \(x\).

Donc \(\quantifs{\forall x\in E}f^{-1}\paren{f\paren{x}}=x\).

Donc \(f^{-1}\rond f=\id{E}\).
\end{itemize}
\end{dem}

\begin{ex}
\begin{itemize}
\item Déterminons la bijection réciproque de \(\fonction{f}{\R}{\R}{x}{x+1}\)

On a \(\quantifs{\forall x\in\R;\forall y\in\R}\begin{aligned}[t]
f\paren{x}=y&\iff x+1=y \\
&\iff x=y-1
\end{aligned}\)

Comme tout \(y\) admet un unique antécédent, \(f\) est bijective.

De plus, on a obtenu \(\quantifs{\forall y\in\R}f^{-1}\paren{y}=y-1\).

\item Déterminons la bijection réciproque de \(\fonction{\exp}{\R}{\Rp}{x}{\e{x}}\)

On remarque \(\begin{aligned}[t]
\quantifs{\forall y\in\Rps;\forall x\in\R}\exp x=y&\iff\ln\paren{\exp x}=\ln y \\
&\iff x=\ln y
\end{aligned}\)

Ainsi, \(\quantifs{\forall y\in\Rps;\exists! x\in\R}\exp x=y\) donc \(\exp\) est bijective.

De plus, on a obtenu \(\quantifs{\forall y\in\Rps}\exp^{-1}=\ln y\).
\end{itemize}
\end{ex}

\begin{rem}
Soient \(I,J\subset\R\). Soit \(f:I\to J\) bijective.

Le graphe de \(f\) est l'ensemble \(\graphe{f}=\accol{\paren{x,f\paren{x}}}_{x\in I}\).

Autrement dit, \(\quantifs{\forall\paren{x,y}\in\R^2}\paren{x,y}\in\graphe{f}\iff\begin{dcases}x\in I\text{ et }y\in J \\ y=f\paren{x}\end{dcases}\)

De même, \(\quantifs{\forall\paren{y,x}\in\R^2}\begin{aligned}[t]
\paren{y,x}\in\graphe{f^{-1}}&\iff\begin{dcases}y\in J\text{ et }x\in I \\ x=f^{-1}\paren{y}\end{dcases} \\
&\iff\begin{dcases}y\in J\text{ et }x\in I \\ y=f\paren{x}\end{dcases} \\
&\iff\paren{x,y}\in\graphe{f}
\end{aligned}\)

Donc \(\graphe{f^{-1}}\) est le symétrique de \(\graphe{f}\) par rapport à la droite \(\Delta:y=x\).

En effet, la symétrie par rapport à \(\Delta\) est la fonction \(\fonctionlambda{\R^2}{\R^2}{\paren{x,y}}{\paren{y,x}}\)
\end{rem}

\begin{rem}
Soient \(E,F\) deux ensembles. Soit \(B\subset F\). Soit \(f:E\to F\) bijective.

Alors \(f^{-1}\paren{B}\) peut désigner l'image réciproque de \(B\) par \(f\) ou l'image directe de \(B\) par \(f^{-1}\).

En fait, les deux ensembles sont égaux.
\end{rem}

\begin{dem}
Notons \(X=\accol{x\in E\tq f\paren{x}\in B}\) le premier ensemble et \(Y=\accol{f^{-1}\paren{y}}_{y\in B}\) le second.

On a, \(X\subset E\) et \(Y\subset E\) et \(\begin{aligned}[t]
\forall x\in E,x\in X&\iff f\paren{x}\in B \\
&\iff\quantifs{\exists y\in B}f\paren{x}=y \\
&\iff\quantifs{\exists y\in B}x=f^{-1}\paren{y} \\
&\iff x\in Y
\end{aligned}\)

Donc \(X=Y\).
\end{dem}

\begin{prop}
Soient \(E,F,G\) des ensembles. Soient \(f:E\to F\) et \(g:F\to G\).

On a \begin{enumerate}
\item \(g\rond f\text{ injection}\imp f\text{ injection}\) ;

\item \(g\rond f\text{ surjection}\imp g\text{ surjection}\) ;

\item \(g\rond f\text{ bijection}\imp g\text{ surjection et }f\text{ injection}\).
\end{enumerate}
\end{prop}

\begin{dem}
\begin{enumerate}
\item Supposons \(g\rond f\) injection.

Soient \(x,y\in E\). On suppose \(f\paren{x}=f\paren{y}\).

Montrons que \(x=y\).

On a \(g\paren{f\paren{x}}=g\paren{f\paren{y}}\).

Comme \(g\rond f\) injective, on a \(x=y\).

Donc \(f\) injective.

\item Supposons \(g\rond f\) surjection.

Montrons que \(g\) est surjective.

Soit \(z\in G\). Montrons que \(\quantifs{\exists y\in F}g\paren{y}=z\).

Comme \(g\rond f\) surjective, il existe \(x\in E\) tel que \(g\rond f\paren{x}=z\).

On a \(g\paren{f\paren{x}}=z\).

Donc \(f\paren{x}\) est un antécédent de \(z\) par \(g\), donc \(g\) est surjective.

\item Découle de (1) et (2).
\end{enumerate}
\end{dem}

\begin{oubli}
Soient \(E,F\) deux ensembles. Soit \(f:E\to F\) bijective.

On a \(\paren{f^{-1}}^{-1}=f\).
\end{oubli}

\begin{dem}
Soit \(x\in E\).

On a \(x=f^{-1}\paren{f\paren{x}}\).

Donc \(f\paren{x}=\paren{f^{-1}}^{-1}\paren{x}\).
\end{dem}

\begin{prop}
Soient \(E,F,G\) trois ensembles. Soient \(f:E\to F\) et \(g:F\to G\) bijectives.

On a vu que \(g\rond f:E\to G\) est une bijection.

On a \(\paren{g\rond f}^{-1}=f^{-1}\rond g^{-1}\).
\end{prop}

\begin{dem}
Soit \(z\in G\).

On a \(\begin{aligned}[t]
z&=g\paren{g^{-1}\paren{z}} \\
&=g\paren{f\paren{f^{-1}\paren{g^{-1}\paren{x}}}} \\
&=g\rond f\paren{f^{-1}\rond g^{-1}\paren{x}}
\end{aligned}\)

Donc \(f^{-1}\rond g^{-1}\paren{z}=\paren{g\rond f}^{-1}\paren{z}\).
\end{dem}

\section{Relations}

\subsection{Définition}

\begin{defi}
Soit \(E\) un ensemble.

On appelle relation binaire sur \(E\) tout partie \(\rel\subset E\times E\).

On pose alors, \(\quantifs{\forall\paren{x,y}\in E^2}x\rel y\iff\paren{x,y}\in\rel\).

Ainsi, une relation binaire est une proposition qui dépend de \(\paren{x,y}\in E^2\).
\end{defi}

\begin{ex}
Égalité : si \(\rel=\accol{\paren{x,x}}_{x\in E}\) (\guillemets{diagonale} de \(E^2\)), alors on obtient \(\quantifs{\forall\paren{x,y}\in E^2}x\rel y\iff x=y\).

Si \(E=\R\) et \(\rel=\accol{\paren{x,y}\in\R^2\tq x<y}\) alors on obtient la relation binaire \(\quantifs{\forall\paren{x,y}\in\R^2}x\rel y\iff x<y\).
\end{ex}

\begin{defi}
Soient \(E\) un ensemble et \(\rel\) une relation binaire sur \(E\).

On dit que \(\rel\) est réflexive si \(\quantifs{\forall x\in E}x\rel x\).

On dit que \(\rel\) est transitive si \(\quantifs{\forall x,y,z\in E}\paren{x\rel y\et y\rel z}\imp x\rel z\).

On dit que \(\rel\) est symétrique si \(\quantifs{\forall x,y\in E}x\rel y\iff y\rel x\).

On dit que \(\rel\) est antisymétrique si \(\quantifs{\forall x,y\in E}\paren{x\rel y\et y\rel x}\imp x=y\).
\end{defi}

\begin{ex}
La relation \(=\) sur un ensemble \(E\) fini est réflexive, transitive, symétrique et antisymétrique.

La relation \(\leq\) sur \(\R\) est réflexive, transitive et antisymétrique.

La relation \(<\) sur \(\R\) est transitive et antisymétrique.

Soit \(E\) un ensemble. La relation \(\subset\) sur \(\P{E}\) est réflexive, transitive et antisymétrique.
\end{ex}

\subsection{Relations d'équivalence}

\begin{defi}
Soient \(E\) un ensemble et \(\rel\) une relation binaire sur \(E\).

On dit que \(\rel\) est une relation d'équivalence sur \(E\) si elle est réflexive, transitive et symétrique.
\end{defi}

\begin{ex}
\begin{itemize}
\item Soit \(E\) un ensemble. La relation \(=\) est une relation d'équivalence sur \(E\).

\item La relation \(\leq\) n'est pas une relation d'équivalence sur \(\R\) (car elle n'est pas symétrique).

\item Soit \(E\) un ensemble non-vide. La relation \(\subset\) n'est pas une relation d'équivalence sur \(\P{E}\).

\item Soit \(n\in\Ns\). On définit une relation binaire \(\rel\) sur \(\Z\) en posant \(\quantifs{\forall x,y\in\Z}x\rel y\iff x\equiv y\croch{n}\). \(\rel\) est une relation d'équivalence sur \(\Z\).

En effet, \(\rel\) est \begin{itemize}
\item réflexive car \(\quantifs{\forall x\in\Z}x\equiv x\croch{n}\) ;

\item symétrique car \(\quantifs{\forall x,y\in\Z}x\equiv y\croch{n}\iff y\equiv x\croch{n}\) ;

\item transitive car si on a \(x,y,z\in\Z\) tels que \(x\equiv y\croch{n}\) et \(y\equiv z\croch{n}\) alors il existe \(k,l\in\Z\) tels que \(x=y+kn\) et \(y=z+ln\) donc on a \(x=z+\paren{l+k}n\) où \(l+k\in\Z\) donc \(x\equiv z\croch{n}\).
\end{itemize}

\item Idem pour la relation de congruence sur \(\R\) modulo un réel.

\item Soient \(E,F\) deux ensembles et \(f:E\to F\).

On définit la relation binaire \(\rel\) sur \(E\) en posant \(\quantifs{\forall x,y\in E}x\rel y\iff f\paren{x}=f\paren{y}\).

Alors \(\rel\) est une relation d'équivalence sur \(E\).
\end{itemize}
\end{ex}

\begin{rem}
En général, les relations d'équivalence sont notées \(\sim\), \(=\), \(\approx\), \(\equiv\), ...
\end{rem}

\begin{defprop}
Soient \(E\) un ensemble et \(\sim\) une relation d'équivalence sur \(E\).

On associe à tout élément sa classe d'équivalence \(\tilde{x}=\accol{y\in E\tq y\sim x}\).

On note \(\classesdequiv{E}=\accol{\tilde{x}}_{x\in E}\) l'ensemble des classes d'équivalence de \(E\).

On a \(\quantifs{\forall x_1,x_2\in E}\tilde{x_1}=\tilde{x_2}\iff x_1\sim x_2\).
\end{defprop}

\begin{dem}
\begin{itemize}
\item Montrons que \(\classesdequiv{E}\) est une partition de \(E\).

\begin{itemize}
\item Chaque classe d'équivalence est non-vide. En effet, \(\quantifs{\forall x\in E}x\in\tilde{x}\) car \(\sim\) est réflexive.

\item On a \(\bigunion_{\theta\in\classesdequiv{E}}\theta=\bigunion_{x\in E}\tilde{x}=E\).

En effet, d'une part \(\quantifs{\forall x\in E}\tilde{x}\subset E\) donc \(\bigunion_{x\in E}\tilde{x}\subset E\).

Et d'autre part \(\quantifs{\forall y\in E}y\in\tilde{y}\) donc \(\forall y\in E,y\in\bigunion_{x\in E}\tilde{x}\).

\item Montrons que les éléments de \(\classesdequiv{E}\) sont deux à deux disjoints.

Soient \(\theta_1,\theta_2\in\classesdequiv{E}\).

Supposons \(\theta_1\not=\theta_2\). Montrons que \(\theta_1\inter\theta_2=\ensvide\).

Soient \(x_1,x_2\in E\) tels que \(\theta_1=\tilde{x_1}\) et \(\theta_2=\tilde{x_2}\).

Par l'absurde, supposons \(\theta_1\inter\theta_2\not=\ensvide\).

Soit \(x\in\theta_1\inter\theta_2\). On a \(x\in\tilde{x_1}\inter\tilde{x_2}\).

Donc \(x\sim x_1\) et \(x\sim x_2\).

Montrons que \(\theta_1\subset\theta_2\).

Soit \(y\in\theta_1\). On a \(y\sim x_1\).

Or \(x_1\sim x\) et \(x\sim x_2\).

Donc \(y\sim x_2\). Donc \(y\in\theta_2\).

On montre de même \(\theta_2\subset\theta_1\).

Donc \(\theta_1=\theta_2\) : contradiction.
\end{itemize}

Finalement, \(\classesdequiv{E}\) est une partition de \(E\).

\item Montrons que \(\tilde{x_1}=\tilde{x_2}\iff x_1\sim x_2\).

\begin{itemize}
\item[\impdir] Supposons \(\tilde{x_1}=\tilde{x_2}\).

Comme \(x_1\in\tilde{x_1}\), on a \(x_1\in\tilde{x_2}\) donc \(x_1\sim x_2\).

\item[\imprec] Supposons \(x_1\sim x_2\).

Montrons que \(\tilde{x_1}\subset\tilde{x_2}\). Soit \(y\in\tilde{x_1}\).

On a \(y\sim x_1\). Or \(x_1\sim x_2\).

Donc \(y\sim x_2\). Donc \(y\in\tilde{x_2}\). Donc \(\tilde{x_1}\subset\tilde{x_2}\).

On montre de même \(\tilde{x_2}\subset\tilde{x_1}\).

Donc \(\tilde{x_1}=\tilde{x_2}\).
\end{itemize}
\end{itemize}
\end{dem}

\begin{rem}
\(\quantifs{\forall\theta\in\classesdequiv{E};\forall x,y\in\theta}x\sim y\)
\end{rem}

\begin{rem}
Soient \(E\) un ensemble et \(\sim\) une relation d'équivalence sur \(E\).

On note \(\fonction{\pi}{E}{\classesdequiv{E}}{x}{\tilde{x}}\) la surjection canonique de \(E\) vers \(\classesdequiv{E}\).

On a \(\quantifs{\forall x,y\in E}x\sim y\iff\pi\paren{x}=\pi\paren{y}\).
\end{rem}

\begin{dem}
\(\pi\) est une surjection car tout \(\theta\in\classesdequiv{E}\) admet un antécédent : \(\quantifs{\forall\theta\in\classesdequiv{E};\exists x\in E}\theta=\tilde{x}\).

Donc \(\quantifs{\forall\theta\in\classesdequiv{E};\exists x\in E}\pi\paren{x}=\theta\).
\end{dem}

\subsection{Relations d'ordre}

\begin{defi}
Soit \(E\) un ensemble.

On appelle relation d'ordre (partiel) tout relation binaire \(\leq\) sur \(E\) qui est réflexive, transitive et antisymétrique.

On dit alors que \(\paren{E,\leq}\) est un ensemble (partiellement) ordonné.

Si la relation binaire \(\leq\) est claire, on dit parfois que \(E\) est un ensemble ordonné.

Si, de plus, \(\leq\) vérifie \(\quantifs{\forall x,y\in E}x\leq y\) ou \(y\leq x\) alors on dit que \(\leq\) est une relation d'ordre total et que \(\paren{E,\leq}\) est totalement ordonné.
\end{defi}

\begin{ex}
Soit \(E\) un ensemble. La relation \(=\) est une relation d'ordre partiel sur \(E\).

\(\paren{\R,\leq}\) est un ensemble totalement ordonné.

Soit \(X\) un ensemble  et \(E=\P{X}\). \(\paren{E,\subset}\) est ordonné.

Posons \(\quantifs{\forall a,b\in\Z}a\divise b\iff\quantifs{\exists k\in\Z}ak=b\).

La relation \(\divise\) n'est pas une relation d'ordre sur \(\Z\) car elle n'est pas antisymétrique. En effet, \(1\divise-1\) et \(-1\divise1\) mais \(-1\not=1\).

En revanche, \(\divise\) est une relation d'ordre sur \(\N\).
\end{ex}

\begin{defi}
Soient \(\paren{E,\leq}\) un ensemble ordonné et \(A\subset E\).

On dit que \(M\in E\) est un majorant de \(A\) dans \(E\) si \(\quantifs{\forall x\in A}x\leq M\).

On dit que \(m\in E\) est un minorant de \(A\) dans \(E\) si \(\quantifs{\forall x\in A}m\leq x\).

On dit que \(A\) est majorée (dans \(E\)) si elle admet un majorant (dans \(E\)).

On dit que \(A\) est minorée (dans \(E\)) si elle admet un minorant (dans \(E\)).

On dit que \(A\) est bornée (dans \(E\)) si elle admet un majorant et un minorant (dans \(E\)).

On dit que \(M\in E\) est le plus grand élément de \(A\) ssi on a \(M\in A\) et \(\quantifs{\forall x\in A}x\leq M\).

On dit que \(m\in E\) est le plus petit élément de \(A\) ssi on a \(m\in A\) et \(\quantifs{\forall x\in A}m\leq x\).

S'il existe, le plus grand élément de \(A\) est unique et est noté \(\max A\).

S'il existe, le plus petit élément de \(A\) est unique et est noté \(\min A\).
\end{defi}

\begin{dem}
Montrons l'unicité du plus grand élément de \(A\). Soient \(M_1,M_2\in E\).

Supposons \(\begin{dcases}M_1\in A &(1) \\ M_2\in A &(2) \\ \forall x\in A,x\leq M_1 &(3) \\ \forall x\in A,x\leq M_2 &(4)\end{dcases}\)

Montrons que \(M_1=M_2\).

Selon (1) et (4), \(M_1\leq M_2\) et selon (2) et (3), \(M_2\leq M_1\).

Donc \(M_1=M_2\) car \(\leq\) est antisymétrique.

On montre de même l'unicité du plus petit élément.
\end{dem}

\begin{ex}
La partie \(\Rp\) n'est pas majorée dans \(\R\). En revanche, elle est majorée dans \(\Rb\).

\(\pinf\) est le plus grand élément de \(\Rb\).

\(0\) est le plus grand élément de \(\intervei{\minf}{0}\).

Soit \(X\) un ensemble. Dans \(\paren{\P{X},\subset}\), \(\ensvide\) est le plus petit élément et \(X\) est le plus grand élément.

Dans \(\paren{\N,\divise}\), \(1\) est le plus petit élément et \(0\) est le plus grand élément.

Toute partie de \(\paren{\N,\divise}\) est bornée : minorée par \(1\) et majorée par \(0\).

La partie \(\Ns\) admet \(1\) comme plus petit élément mais pas de plus grand élément.
\end{ex}

\begin{defi}
Soient \(\paren{E,\leq}\) un ensemble ordonné et \(A\subset E\).

Notons \(\majo{A}\) l'ensemble des majorants de \(A\) : \(\majo{A}=\accol{M\in E\tq\quantifs{\forall x\in A}x\leq M}\) et \(\mino{A}\) l'ensemble des minorants de \(A\) : \(\mino{A}=\accol{m\in E\tq\quantifs{\forall x\in A}m\leq x}\).

S'il existe, le plus petit élément de \(\majo{A}\) est appelé la borne supérieure de \(A\) et est noté \(\sup A\).

S'il existe, le plus grand élément de \(\mino{A}\) est appelé la borne inférieure de \(A\) et est noté \(\inf A\).
\end{defi}

\begin{ex}
Dans \(\paren{\R,\leq}\) : \begin{itemize}
\item \(\majo{\intervii{0}{1}}=\intervie{1}{\pinf}\) donc \(\sup\intervii{0}{1}=1\).

\item \(\majo{\intervie{0}{1}}=\intervie{1}{\pinf}\) donc \(\sup\intervie{0}{1}=1\).

\item \(\majo{\ensvide}=\R\) donc \(\sup\ensvide\) n'existe pas.
\end{itemize}

Dans \(\paren{\Rb,\leq}\) : \begin{itemize}
\item \(\majo{\R}=\accol{\pinf}\) donc \(\sup\R=\pinf\).

\item \(\majo{\ensvide}=\Rb\) donc \(\sup\ensvide=\minf\).
\end{itemize}

Dans \(\paren{\N,\divise}\) : \begin{itemize}
\item \(\majo{\Ns}=\accol{0}\) donc \(\sup\Ns=0\).

\item Notons \(\mathbb{P}\) l'ensemble des nombres premiers. \(\majo{\mathbb{P}}=\accol{0}\) donc \(\sup\mathbb{P}=0\).
\end{itemize}
\end{ex}

\begin{prop}
Soient \(\paren{E,\leq}\) un ensemble ordonné et \(A\subset E\).

\begin{enumerate}
\item Si \(A\) admet un plus grand élément, alors \(A\) admet une borne supérieure qui est son plus grand élément : \(\sup A=\max A\).

\item Pour que \(A\) admette une borne supérieure, il faut que \(A\) soit majorée.
\end{enumerate}
\end{prop}

\begin{dem}
\begin{enumerate}
\item Supposons que \(A\) admet un plus grand élément \(M\) : \(\begin{dcases}M\in A \\ \quantifs{\forall x\in A}x\leq M\end{dcases}\)

Alors \(M\) majore \(A\) donc \(M\in\majo{A}\).

De plus, \(\quantifs{\forall M\prim\in\majo{A}}M\leq M\prim\) car \(M\in A\).

Donc \(M\) est le plus petit élément de \(\majo{A}\).

Donc \(M=\sup A\).

\item Clair.
\end{enumerate}
\end{dem}

\begin{prop}
Soit \(\paren{E,\leq}\) un ensemble ordonné. Soient \(A\subset E\) et \(y\in E\).

Les propositions suivantes sont équivalentes : \begin{enumerate}
\item \(y\) est la borne supérieure de \(A\)

\item \(\quantifs{\forall z\in E}z\text{ majore }A\iff y\leq z\)
\end{enumerate}
\end{prop}

\begin{dem}
Montrons que \((1)\iff(2)\).

\begin{itemize}
\item[\impdir] Supposons \(y=\sup A\). Soit \(z\in E\).

Montrons que \(z\text{ majore }A\iff y\leq z\).

\begin{itemize}
\item[\impdir] Claire car \(y\) est le plus petit majorant de \(A\).

\item[\imprec] Si \(y\leq z\) alors \(\quantifs{\forall x\in A}x\leq y\leq z\) donc \(z\) majore \(A\).
\end{itemize}

\item[\imprec] Supposons (2).

Comme \(y\leq y\), \(y\) majore \(A\) (selon \imprec).

De plus, tout majorant \(z\in A\) vérifie \(y\leq z\) (selon \impdir).

Donc \(y=\sup A\).
\end{itemize}
\end{dem}

\section{Ensemble ordonné \(\paren{\N,\leq}\)}

\begin{theo}\thlabel{theo:partieNppe}
Toute partie non-vide de \(\N\) admet un plus petit élément.
\end{theo}

\begin{dem}
\note{ADMIS}
\end{dem}

\begin{prop}
Toute partie finie non-vide de \(\N\) admet un plus grand élément.
\end{prop}

\begin{rem}
Plus généralement, tout ensemble totalement ordonné fini admet un plus petit élément et un plus grand élément.
\end{rem}

\begin{theo}[Démonstration par récurrence]\thlabel{theo:demoRec}
Soit \(P\paren{n}\) une proposition dépendant d'un entier \(n\in\N\).

Supposons \(\begin{dcases}P\paren{0} \\ \quantifs{\forall n\in\N}P\paren{n}\imp P\paren{n+1}\end{dcases}\)

Alors \(\quantifs{\forall n\in\N}P\paren{n}\).
\end{theo}

\begin{dem}
On pose \(A=\accol{n\in\N\tq\non P\paren{n}}\). Montrons que \(A=\ensvide\).

Supposons par l'absurde \(A\not=\ensvide\).

Posons \(n_0=\min A\). \(n_0\) existe selon le \thref{theo:partieNppe}. On a \(n_0\not=0\) car \(0\not\in A\) car \(P\paren{0}\) est vraie.

Donc \(n_0\geq1\).

On a \(n_0-1\not\in A\) car \(n_0\) minore \(A\).

Donc \(P\paren{n_0-1}\) est vraie. Donc \(P\paren{n_0}\) est vraie : contradiction.

Donc \(\quantifs{\forall n\in\N}P\paren{n}\).
\end{dem}

\begin{cor}[Démonstration par récurrence forte]
Soit \(P\paren{n}\) une proposition dépendant de \(n\in\N\).

Supposons \(\begin{dcases}P\paren{0} \\ \quantifs{\forall n\in\N}\paren{P\paren{0}\et P\paren{1}\et\dots\et P\paren{n}}\imp P\paren{n+1}\end{dcases}\)

Alors \(\quantifs{\forall n\in\N}P\paren{n}\).
\end{cor}

\begin{dem}
Pour tout \(n\in\N\), on note \(Q\paren{n}\) la proposition \guillemets{\(P\paren{0}\et P\paren{1}\et\dots\et P\paren{n}\)}, c'est à dire \guillemets{\(\quantifs{\forall k\in\interventierii{0}{n}}P\paren{k}\)}.

On a \(Q\paren{0}\) car \(P\paren{0}\) et \(\quantifs{\forall n\in\N}Q\paren{n}\imp P\paren{n+1}\).

Donc \(\quantifs{\forall n\in\N}Q\paren{n}\imp\paren{Q\paren{n}\et P\paren{n+1}}\).

Donc \(\quantifs{\forall n\in\N}Q\paren{n}\imp Q\paren{n+1}\).

D'où selon le \thref{theo:demoRec} : \(\quantifs{\forall n\in\N}Q\paren{n}\) donc \(\quantifs{\forall n\in\N}P\paren{n}\).
\end{dem}

\section{Ensemble ordonné \(\paren{\R,\leq}\)}

\begin{rappel}
On appelle intervalle de \(\R\) toute partie \(I\) de \(\R\) telle que \(\quantifs{\forall x,y\in I;\forall z\in\R}x\leq z\leq y\imp z\in I\).
\end{rappel}

\begin{theo}[Caractérisation de la borne supérieure dans \(\R\)]
Soit \(A\subset\R\). Soit \(y\in\R\).

Alors \(y=\sup A\iff(S):\begin{dcases}\quantifs{\forall x\in A}y\geq x &(1) \\ \quantifs{\forall\epsilon\in\Rps;\exists x\in A}y-\epsilon\leq x &(2)\end{dcases}\)
\end{theo}

\begin{dem}
\begin{itemize}
\item[\impdir] Supposons \(y=\sup A\).

Alors \(y\) majore \(A\) donc \(\quantifs{\forall x\in A}x\leq y\).

De plus, on a \(\quantifs{\forall\epsilon\in\Rps}y-\epsilon<y\).

Donc \(y-\epsilon\) n'est pas un majorant de \(A\).

Donc \(\quantifs{\exists x\in A}y-\epsilon\leq x\).

\item[\imprec] Supposons \((S)\).

\begin{itemize}
\item \(y\) majore \(A\) d'après (1).

\item Montrons que \(y\) est le plus petit majorant de \(A\).

Soit \(z\) un majorant de \(A\). Montrons que \(z\geq y\).

Soit \(\epsilon\in\Rps\).

Soit \(x\in A\) tel que \(y-\epsilon\leq x\).

On a \(y-\epsilon\leq x\leq z\).

Donc \(y-\epsilon\leq z\)

Donc \(y\leq z\). En effet, supposons par l'absurde \(z<y\).

Posons \(\epsilon=\dfrac{y-z}{2}\in\Rps\).

On a \(y-\epsilon\leq z\).

Donc \(y-\dfrac{y-z}{2}<z\).

Donc \(\dfrac{y}{2}<\dfrac{z}{2}\) : contradiction.
\end{itemize}

Donc on a \(y=\sup A\).
\end{itemize}
\end{dem}

\begin{theo}[Caractérisation de la borne inférieure dans \(\R\)]
Soient \(A\subset\R\) et \(y\in\R\).

On a \(y=\inf A\iff\begin{dcases}\quantifs{\forall x\in A}y\leq x \\ \quantifs{\forall\epsilon\in\Rps;\exists x\in A}x\leq y+\epsilon\end{dcases}\)
\end{theo}

\begin{dem}
Idem.
\end{dem}

\begin{theo}
Toute partie non-vide et majorée de \(\R\) admet une borne supérieure.

Toute partie non-vide et minorée de \(\R\) admet une borne inférieure.
\end{theo}

\begin{dem}
\note{ADMIS}
\end{dem}

\section{Fonctions à valeurs dans un ensemble ordonné}

\begin{defi}
Soient \(E\) un ensemble, \(\paren{F,\leq}\) un ensemble ordonné et \(f:E\to F\).

On appelle majorant de \(f\) tout majorant de \(\Im f\).

On appelle minorant de \(f\) tout minorant de \(\Im f\).

Ainsi, \(\begin{aligned}[t]
M\text{ majore }f&\iff\quantifs{\forall y\in\Im f}y\leq M \\
&\iff\quantifs{\forall x\in E}f\paren{x}\leq M
\end{aligned}\)

On appelle borne supérieure de \(f\) la borne supérieure de \(\Im f\) si elle existe.

On appelle borne inférieure de \(f\) la borne inférieure de \(\Im f\) si elle existe.
\end{defi}

\begin{rem}
\(\sup f=\sup_{x\in E}f\paren{x}\)
\end{rem}

\begin{defi}
Soient \(\paren{E,\leq}\) un ensemble ordonné, \(I\) un ensemble et \(\paren{x_i}_{i\in I}\) une famille d'éléments de \(E\).

On appelle majorant de \(\paren{x_i}_{i\in I}\) tout majorant de \(\accol{x_i}_{i\in I}\).

On appelle minorant de \(\paren{x_i}_{i\in I}\) tout minorant de \(\accol{x_i}_{i\in I}\).

On dit que \(\paren{x_i}_{i\in I}\) est majorée si on a \(\quantifs{\exists M\in E;\forall i\in I}x_i\leq M\).

On dit que \(\paren{x_i}_{i\in I}\) est minorée si on a \(\quantifs{\exists m\in E;\forall i\in I}m\leq x_i\).

La borne supérieure de \(\paren{x_i}_{i\in I}\) est le plus petit majorant de la famille.

La borne inférieure de \(\paren{x_i}_{i\in I}\) est le plus grand minorant de la famille.

Elles sont notées \(\sup_{i\in I}x_i\) et \(\inf_{i\in I}x_i\).
\end{defi}

\chapter{Suites}

\minitoc

\section{Suites}

\subsection{Cadre}

On appelle suite réelle toute famille de réels \(\paren{u_n}_{n\in\N}\in\R^\N\) indicée par \(\N\).

On appelle suite complexe toute famille \(\paren{u_n}_{n\in\N}\in\C^\N\) indicée par \(\N\).

On note aussi abusivement \(\paren{u_n}_n\) ces suites.

Plus généralement, on peut appeler suite toute famille de la forme \(\paren{u_n}_{n\in\interventierie{n_0}{\pinf}}\) où \(n_0\in\Z\).

\subsection{Définitions}

Soient \(u=\paren{u_n}_{n\in\N},v=\paren{v_n}_{n\in\N}\in\R^\N\).

On dit que \(u\) est constante si \(\quantifs{\exists c\in\R;\forall n\in\N}u_n=c\).

On dit que \(u\) est stationnaire (ou \guillemets{constante à partir d'un certain rang}) si \(\quantifs{\exists c\in\R;\exists N\in\N;\forall n\in\interventierie{N}{\pinf}}u_n=c\).

On dit que \(u\) est \begin{itemize}
\item croissante si \(\quantifs{\forall n\in\N}u_n\leq u_{n+1}\) ;

\item décroissante si \(\quantifs{\forall n\in\N}u_{n+1}\leq u_n\) ;

\item strictement croissante si \(\quantifs{\forall n\in\N}u_n<u_{n+1}\) ;

\item strictement décroissante si \(\quantifs{\forall n\in\N}u_{n+1}<u_n\) ;

\item monotone si elle est croissante ou décroissante ;

\item strictement monotone si elle est strictement croissante ou strictement décroissante.
\end{itemize}

On dit que \(u\) est \begin{itemize}
\item majorée si \(\quantifs{\exists M\in\R;\forall n\in\N}u_n\leq M\) ;

\item minorée si \(\quantifs{\exists m\in\R,\forall n\in\N}m\leq u_n\) ;

\item bornée si elle est majorée et minorée, c'est à dire si \(\quantifs{\exists K\in\Rp;\forall n\in\N}\abs{u_n}\leq K\).
\end{itemize}

On dit que \(u\) est inférieure ou égale à \(v\) et on note \(u\leq v\) si \(\quantifs{\forall n\in\N}u_n\leq v_n\).

On dit que \(u\) est positive (ou \guillemets{à termes positifs}) et on note \(u\geq0\) si \(\quantifs{\forall n\in\N}u_n\geq0\).

On note \(u+v\) la suite \(\paren{w_n}_{n\in\N}\in\R^\N\) définie par \(\quantifs{\forall n\in\N}w_n=u_n+v_n\). Autrement dit, \(\paren{u_n}_{n\in\N}+\paren{v_n}_{n\in\N}=\paren{u_n+v_n}_{n\in\N}\).

On pose de même \(\paren{u_n}_{n\in\N}\times\paren{v_n}_{n\in\N}=\paren{u_nv_n}_{n\in\N}\).

On pose enfin, si \(\lambda\in\R\) : \(\lambda\paren{u_n}_{n\in\N}=\paren{\lambda u_n}_{n\in\N}\).

\begin{ex}
Supposons \(\quantifs{\forall n\in\N}u_n=\paren{-1}^n\) et \(v_n=\paren{-1}^{n+1}\).

On a \(u+v=\paren{0}_n\) et \(uv=\paren{-1}_n\).
\end{ex}

\subsection{Suite définie en itérant une fonction}

\begin{prop}
Soient \(I\subset\R\), \(x\in I\) et \(f:I\to\R\).

On suppose \(f\paren{I}\subset I\) (\(I\) est stable par \(f\)).

Alors il existe une unique suite \(\paren{u_n}_n\in\R^\N\) telle que \(\begin{dcases}u_0=x \\ \quantifs{\forall n\in\N}u_{n+1}=f\paren{u_n}\end{dcases}\)
\end{prop}

\begin{dem}
\unicite

Soient \(\paren{u_n}_n,\paren{v_n}_n\in\R^\N\) telles que \(\begin{dcases}u_0=v_0=x \\ \quantifs{\forall n\in\N}\begin{dcases}u_{n+1}=f\paren{u_n} \\ v_{n+1}=f\paren{v_n}\end{dcases}\end{dcases}\)

Montrons que \(\quantifs{\forall n\in\N}u_n=v_n\) par récurrence sur \(n\).

On a \(u_0=x=v_0\).

Soit \(n\in\N\) tel que \(u_n=v_n\). On a \(f\paren{u_n}=f\paren{v_n}\) donc \(u_{n+1}=v_{n+1}\).

\existence

On pose \(\quantifs{\forall n\in\N}f^{\rond n}=\begin{dcases}\id{E} &\text{si }n=0 \\ \underbrace{f\rond f\rond f\rond\dots\rond f}_{n\text{ facteurs}} &\text{sinon}\end{dcases}\)

On remarque que la suite \(\paren{f^{\rond n}\paren{x}}_{n\in\N}\) convient.
\end{dem}

\begin{ex}
Avec \(\fonction{f}{\R}{\R}{t}{t^2}\) et \(x=2\), on a \(u_0=2\), \(u_1=4\), \(u_2=16\), \(u_3=256\), ...

Avec \(\fonction{f}{\Rps}{\R}{t}{\ln t}\) et \(x=\e{}\), on a \(u_0=\e{}\), \(u_1=1\), \(u_2=0\), \(u_3=\) problème car \(\Rps\) n'est pas stable par \(f\).
\end{ex}

\subsection{Suites particulières}

\begin{defi}
On dit que \(\paren{u_n}_{n\in\N}\in\R^\N\) est une suite arithmétique si on a \(\quantifs{\exists r\in\R;\forall n\in\N}u_{n+1}=u_n+r\).

Un tel réel \(r\) est unique et est appelé raison de la suite.
\end{defi}

\begin{prop}
Soit \(\paren{u_n}_n\in\R^\N\) une suite arithmétique de raison \(r\in\R\).

On a \(\quantifs{\forall n\in\N}u_n=u_0+nr\).
\end{prop}

\begin{defi}
On dit que \(\paren{u_n}_n\in\R^\N\) est une suite géométrique si on a \(\quantifs{\exists q\in\R;\forall n\in\N}u_{n+1}=qu_n\).

Un tel réel \(q\) est unique si \(v_0\not=0\) et est alors appelé raison de la suite.
\end{defi}

\begin{prop}
Soit \(\paren{u_n}_n\in\R^\N\) une suite géométrique de raison \(q\in\R\).

On a \(\quantifs{\forall n\in\N}u_n=u_0\times q^n\).
\end{prop}

\begin{defi}
On dit que \(\paren{u_n}_n\in\R^\N\) est une suite arithmético-géométrique si on a \(\quantifs{\exists a,b\in\R;\forall n\in\N}u_{n+1}=au_n+b\).
\end{defi}

\begin{prop}
Soit \(\paren{u_n}_n\) une suite arithmético-géométrique. Soient \(a,b\in\R\) tels que \(\quantifs{\forall n\in\N}u_{n+1}=au_n+b\).

Si \(a=1\) alors \(\quantifs{\forall n\in\N}u_n=u_0+nb\).

Supposons désormais \(a\not=1\). Alors la fonction \(\fonction{f}{\R}{\R}{x}{ax+b}\) admet un unique point fixe \(\lambda\in\R\) : \(\quantifs{\exists!\lambda\in\R}f\paren{\lambda}=\lambda\) et la suite \(\paren{u_n-\lambda}_n\) est une suite géométrique.
\end{prop}

\begin{dem}
Soit \(x\in\R\). On a : \[\begin{aligned}
x\text{ point fixe de }f&\ssi f\paren{x}=x \\
&\ssi ax+b=x \\
&\ssi x=\dfrac{b}{1-a}.
\end{aligned}\] Donc \(\lambda=\dfrac{b}{1-a}\) est l'unique point fixe de \(f\).

De plus, on a : \[\begin{aligned}
\quantifs{\forall n\in\N}u_{n+1}-\lambda&=au_n+b-\dfrac{b}{1-a} \\
&=au_n+b\paren{1-\dfrac{1}{1-a}} \\
&=au_n+b\dfrac{-a}{1-a} \\
&=a\paren{u_n-\dfrac{b}{1-a}} \\
&=a\paren{u_n-\lambda}.
\end{aligned}\] Donc \(\paren{u_n-\lambda}_n\) est une suite géométrique de raison \(a\).
\end{dem}

\begin{ex}
Soit \(\paren{u_n}_n\) définie par \(\begin{dcases}u_0=2 \\ \quantifs{\forall n\in\N}u_{n+1}=5u_n+3\end{dcases}\).

Calculons \(u_n\) pour tout \(n\in\N\).

Résolvons l'équation \(x=5x+3\) pour tout \(x\in\R\) : \(x=-\dfrac{3}{4}\).

Ainsi, on a : \[\begin{aligned}
\quantifs{\forall n\in\N}u_{n+1}+\dfrac{3}{4}&=5u_n+3+\dfrac{3}{4} \\
&=5u_n+\dfrac{15}{4} \\
&=5\paren{u_n+\dfrac{3}{4}}.
\end{aligned}\] Donc \(\paren{u_n+\dfrac{3}{4}}_n\) est géométrique de raison \(5\).

Donc \(\quantifs{\forall n\in\N}u_n+\dfrac{3}{4}=\paren{u_0+\dfrac{3}{4}}\times5^n\).

D'où \(\quantifs{\forall n\in\N}u_n=\dfrac{11}{4}\times5^n-\dfrac{3}{4}\).
\end{ex}

\section{Convergence}

\subsection{Définition}

\begin{defprop}
Soient \(\paren{u_n}_n\in\R^\N\) et \(l\in\R\).

On dit que \(\paren{u_n}_n\) converge (ou tend) vers \(l\) si on a : \[\quantifs{\forall\epsilon\in\Rps;\exists N\in\N;\forall n\in\interventierie{N}{\pinf}}\abs{u_n-l}\leq\epsilon.\]

Un tel réel \(l\) est unique et est appelé la limite de \(\paren{u_n}_n\). Il est noté : \[l=\lim_{n\to\pinf}u_n=\lim_nu_n\quad\text{ou}\quad u_n\tendqd{n\to\pinf}l.\]

On dit enfin que \(\paren{u_n}_n\) est convergente. Sinon, on dit que \(\paren{u_n}_n\) est divergente.
\end{defprop}

\begin{rem}
La notation \guillemets{\(\lim_nu_n=l\)} signifie \guillemets{la limite de \(\paren{u_n}_n\) existe et vaut \(l\)}. Sa négation n'est donc pas \(\lim_nu_n\not=l\) mais \[\quantifs{\exists\epsilon\in\Rps;\forall N\in\N;\exists n\in\interventierie{N}{\pinf}}\abs{u_n-l}>\epsilon.\]
\end{rem}

\begin{dem}
Montrons que \(\paren{u_n}_n\) admet au plus une limite.

Soient \(l,l\prim\in\R\) tels que \(\begin{dcases}
\quantifs{\forall\epsilon\in\Rps;\exists N\in\N;\forall n\in\interventierie{N}{\pinf}}\abs{u_n-l}\leq\epsilon \\
\quantifs{\forall\epsilon\in\Rps;\exists N\in\N;\forall n\in\interventierie{N}{\pinf}}\abs{u_n-l\prim}\leq\epsilon
\end{dcases}\).

Supposons \(l\not=l\prim\). Posons \(\epsilon=\dfrac{\abs{l-l\prim}}{3}\). On a \(\epsilon\in\Rps\).

Soit \(N_1\in\N\) tel que \(\quantifs{\forall n\geq N_1}\abs{u_n-l}\leq\epsilon\).

Soit \(N_2\in\N\) tel que \(\quantifs{\forall n\geq N_2}\abs{u_n-l\prim}\leq\epsilon\).

Posons \(N=\max\accol{N_1;N_2}\). On a \(\begin{dcases}\abs{u_N-l}\leq\epsilon \\ \abs{u_N-l\prim}\leq\epsilon\end{dcases}\).

Donc on a : \[\begin{aligned}
\abs{l-l\prim}&\leq\abs{l-u_N}+\abs{u_N-l\prim} \\
&\leq\epsilon+\epsilon \\
&=2\epsilon \\
&=\dfrac{2}{3}\abs{l-l\prim}\quad\text{contradiction.}
\end{aligned}\]

Donc par l'absurde, \(l=l\prim\) donc la limite est unique.
\end{dem}

\begin{ex}
Montrons que \(\lim_{n}\dfrac{1}{n}=0\), \cad : \[\quantifs{\forall\epsilon\in\Rps;\exists N\in\N;\forall n\in\interventierie{N}{\pinf}}\abs{\dfrac{1}{n}-0}\leq\epsilon.\]

Soit \(\epsilon\in\Rps\).

On a : \[\begin{aligned}
\quantifs{\forall n\in\Ns}\abs{\dfrac{1}{n}}\leq\epsilon&\ssi n\geq\dfrac{1}{\epsilon} \\
&\impr n\geq\floor{\dfrac{1}{\epsilon}}+1.
\end{aligned}\]

Donc l'entier \(\floor{\dfrac{1}{\epsilon}}+1\) convient (donc \(N\) existe).
\end{ex}

\begin{rem}
Pour une suite, on a : \[\text{constante}\imp\text{stationnaire}\imp\text{convergente}.\]
\end{rem}

\begin{rem}
Soient \(\paren{u_n}_n,\paren{v_n}_n\in\R^\N\).

Si \(\paren{u_n}_n\) et \(\paren{v_n}_n\) coïncident à partir d'un certain rang, \cad si on a : \[\quantifs{\exists N\in\N;\forall n\geq N}u_n=v_n\] alors on a : \[\paren{u_n}_n\text{ convergente}\ssi\paren{v_n}_n\text{ convergente}\] et les limites sont égales.
\end{rem}

\begin{prop}
Soient \(\paren{u_n}_n\in\R^\N\) et \(l\in\R\).

Alors on a : \[\lim_nu_n=l\ssi\lim_n\paren{u_n-l}=0.\]
\end{prop}

\begin{dem}
On a : \[\begin{aligned}
\lim_nu_n=l&\ssi\quantifs{\forall\epsilon\in\Rps;\exists N\in\N;\forall n\geq N}\abs{u_n-l}\leq\epsilon \\
&\ssi\quantifs{\forall\epsilon\in\Rps;\exists N\in\N;\forall n\geq N}\abs{\paren{u_n-l}-0}\leq\epsilon \\
&\ssi\lim_n\paren{u_n-l}=0
\end{aligned}\]
\end{dem}

\subsection{Convergence et ordre}

\begin{prop}
Toute suite convergente est bornée.
\end{prop}

\begin{dem}\thlabel{dem:suiteReelleConvergenteDoncBornee}
Soit \(\paren{u_n}_n\) une suite convergente. On note \(l\) sa limite.

Soit \(N\in\N\) tel que \(\quantifs{\forall n\geq N}\abs{u_n-l}\leq1\).

On a : \[\begin{aligned}
\quantifs{\forall n\geq N}\abs{u_n}&=\abs{u_n-l+l} \\
&\leq\abs{u_n-l}+\abs{l} \\
&\leq1+\abs{l}.
\end{aligned}\]

Posons \(M=\max\accol{\abs{u_0};\dots;\abs{u_{N-1}};1+\abs{l}}\).

On a \(\quantifs{\forall n\in\N}\abs{u_n}\leq M\).

Donc \(\paren{u_n}_n\) est bornée.
\end{dem}

\begin{prop}
Soit \(\paren{u_n}_n\) une suite convergente de limite \(l\in\R\).

Tout minorant strict de \(l\) minore strictement \(\paren{u_n}_n\) à partir d'un certain rang.
\end{prop}

\begin{dem}
Soit \(m\in\R\) tel que \(m<l\).

On a : \[\quantifs{\exists N\in\N;\forall n\geq N}u_n>m.\]

Posons \(\epsilon=\dfrac{l-m}{2}\). On a \(\epsilon\in\Rps\).

Soit \(N\in\N\) tel que \(\quantifs{\forall n\geq N}\abs{u_n-l}\leq\epsilon\).

On a : \[\begin{aligned}
\quantifs{\forall n\geq N}u_n-l&\geq-\epsilon \\
\quantifs{\forall n\geq N}u_n-l&\geq\dfrac{m-l}{2} \\
\quantifs{\forall n\geq N}u_n&\geq\dfrac{m+l}{2}>m.
\end{aligned}\]

Donc \(N\) convient.
\end{dem}

\begin{ex}
Soient \(\paren{u_n}_n\in\R^\N\) et \(l\in\Rps\).

On a \(0<l\) donc \(\quantifs{\exists N\in\N;\forall n\geq N}0<u_n\).

En particulier, \(\paren{\dfrac{1}{u_n}}_n\) est bien définie à partir d'un certain rang.
\end{ex}

\begin{prop}
Soit \(\paren{u_n}_n\) une suite convergente de limite \(l\in\R\).

Tout majorant strict de \(l\) majore strictement \(\paren{u_n}_n\) à partir d'un certain rang.
\end{prop}

\begin{dem}
Idem.
\end{dem}

\begin{rem}
On ne peut rien dire d'analogue concernant les minorants ou majorants larges de la limite. Par exemple, \(0\) majore \(0\) mais ne majore pas \(\paren{\dfrac{1}{n}}_n\) à partir d'un certain rang.
\end{rem}

\begin{prop}[Passage à la limite dans une inégalité large]\thlabel{prop:passageALaLimiteDansUneInegaliteLargeAvecUnScalaire}
Soit \(\paren{u_n}_n\) une suite convergente de limite \(l\in\R\). Soit \(\lambda\in\R\).

Si \(\quantifs{\forall n\in\N}u_n\leq\lambda\) alors \(l\leq\lambda\).

Si \(\quantifs{\forall n\in\N}u_n\geq\lambda\) alors \(l\geq\lambda\).
\end{prop}

\begin{dem}
Supposons \(\quantifs{\forall n\in\N}u_n\leq\lambda\). Montrons que \(l\leq\lambda\).

Supposons \(\lambda<l\).

Alors \(\quantifs{\exists N\in\N;\forall n\geq N}u_n>l\) : contradiction.

Donc \(l\leq\lambda\) par l'absurde.

On montre de même que si \(\quantifs{\forall n\in\N}u_n\geq\lambda\) alors \(l\geq\lambda\).
\end{dem}

\begin{rem}
C'est faux avec des inégalités strictes. Par exemple, on a : \(\quantifs{\forall n\in\N}0<\dfrac{1}{n}\) mais on n'a pas \(0<\lim_n\dfrac{1}{n}\).
\end{rem}

\begin{theo}[Théorème des gendarmes]\label{theo:gendarmes}
Soient \(\paren{u_n}_n,\paren{v_n}_n,\paren{w_n}_n\in\R^\N\).

On suppose que : \begin{itemize}
\item les suites \(\paren{u_n}_n\) et \(\paren{w_n}_n\) sont convergentes et de même limite \(l\in\R\) ;
\item l'on a \(\quantifs{\forall n\in\N}u_n\leq v_n\leq w_n\). \\
\end{itemize}

Alors \(\paren{v_n}_n\) est convergente de limite \(l\).
\end{theo}

\begin{dem}\thlabel{dem:théorèmeDesGendarmesDansLeCasFiniSuites}
Montrons que \(\quantifs{\forall\epsilon\in\Rps;\exists N\in\N;\forall n\geq N}\abs{v_n-l}\leq\epsilon\).

Soit \(\epsilon\in\Rps\).

Soit \(N_1\in\N\) tel que \(\quantifs{\forall n\geq N_1}l-\epsilon\leq u_n\).

Soit \(N_2\in\N\) tel que \(\quantifs{\forall n\geq N_2}l+\epsilon\geq w_n\).

Posons \(N=\max\accol{N_1;N_2}\).

On a : \[\quantifs{\forall n\geq N}l-\epsilon\leq u_n\leq w_n\leq l+\epsilon.\]

D'où \(\lim_nv_n=l\).
\end{dem}

\begin{ex}~\\
On a vu \(\lim_n\dfrac{1}{n}=0\). On admet \(\lim_n-\dfrac{1}{n}=0\).

Montrons que \(\paren{\dfrac{\paren{-1}^n}{n}}_n\) est convergente et tend vers \(0\).

On a : \[\quantifs{\forall n\in\Ns}-\dfrac{1}{n}\leq\dfrac{\paren{-1}^n}{n}\leq\dfrac{1}{n}\quad\text{et}\quad\lim_n-\dfrac{1}{n}=\lim_n\dfrac{1}{n}=0.\]

Donc selon le théorème des gendarmes, \(\lim_n\dfrac{\paren{-1}^n}{n}=0\).
\end{ex}

\begin{cor}
Soient \(\paren{u_n}_n,\paren{v_n}_n\in\R^\N\). Soit \(l\in\R\).

On suppose \(\begin{dcases}\lim_nv_n=0 \\ \quantifs{\forall n\in\N}\abs{u_n-l}\leq v_n\end{dcases}\)

Alors \(\paren{u_n}_n\) est convergente et \(\lim_nu_n=l\).
\end{cor}

\begin{dem}
On a \(\quantifs{\forall n\in\N}-v_n\leq u_n-l\leq v_n\).

Donc \(\quantifs{\forall n\in\N}l-v_n\leq u_n\leq l+v_n\).

Or \(\lim_n\paren{l-v_n}=\lim_n\paren{l+v_n}=l\).

Donc \(\lim_nu_n=l\) d'après le théorème des gendarmes.
\end{dem}

\begin{nota}
Soit \(\paren{u_n}_n\in\R^\N\). Soit \(l\in\R\).

La notation \(\lim_nu_n=l^+\) signifie que \(\paren{u_n}_n\) converge vers \(l\) par valeurs supérieures, \cad : \[\begin{dcases}\lim_nu_n=l \\ \quantifs{\exists N\in\N;\forall n\geq N}l<u_n\end{dcases}\]

De même, la notation \(\lim_nu_n=l^-\) signifie que \(\paren{u_n}_n\) converge vers \(l\) par valeurs inférieures, \cad : \[\begin{dcases}\lim_nu_n=l \\ \quantifs{\exists N\in\N;\forall n\geq N}l>u_n\end{dcases}\]
\end{nota}

\subsection{Opérations sur les limites}

\begin{prop}
Soient \(\paren{u_n}_n,\paren{v_n}_n\in\R^\N\). Soient \(\lambda,\mu\in\R\). Soient \(l,l\prim\in\R\).

On suppose \(\lim_nu_n=l\) et \(\lim_nv_n=l\prim\).

On a alors :

\begin{enumerate}
\item \(\lim_n\paren{u_n+v_n}=l+l\prim\) \\
\item \(\lim_n\paren{\lambda u_n+\mu v_n}=\lambda l+\mu l\prim\) \\
\item \(\lim_nu_nv_n=ll\prim\)
\end{enumerate}
\end{prop}

\begin{dem}[1]\thlabel{dem:sommeLimitesReellesSuites}
Montrons que \(\quantifs{\forall\epsilon\in\Rps;\exists N\in\N;\forall n\geq N}\abs{u_n+v_n-l-l\prim}\leq\epsilon\).

Soit \(\epsilon\in\Rps\).

Soit \(N_1\in\N\) tel que \(\quantifs{\forall n\geq N_1}\abs{u_n-l}\leq\dfrac{\epsilon}{2}\).

Soit \(N_2\in\N\) tel que \(\quantifs{\forall n\geq N_2}\abs{v_n-l\prim}\leq\dfrac{\epsilon}{2}\).

On pose \(N=\max\accol{N_1;N_2}\).

On a : \[\begin{aligned}
\quantifs{\forall n\geq N}\abs{u_n+v_n-l-l\prim}&\leq\abs{u_n-l}+\abs{v_n-l\prim} \\
&\leq\dfrac{\epsilon}{2}+\dfrac{\epsilon}{2} \\
&=\epsilon.
\end{aligned}\]
\end{dem}

\begin{dem}[2]
Montrons que \(\quantifs{\forall\epsilon\in\Rps;\exists N\in\N;\forall n\geq N}\abs{\lambda u_n+\mu v_n-\lambda l-\mu l\prim}\leq\epsilon\).

Soit \(\epsilon\in\Rps\).

Soit \(N_1\in\N\) tel que \(\quantifs{\forall n\geq N_1}\abs{u_n-l}\leq\dfrac{\epsilon}{2\paren{\abs{\lambda}+1}}\).

Soit \(N_2\in\N\) tel que \(\quantifs{\forall n\geq N_2}\abs{v_n-l\prim}\leq\dfrac{\epsilon}{2\paren{\abs{\mu}+1}}\).

On pose \(N=\max\accol{N_1;N_2}\).

On a : \[\begin{aligned}
\quantifs{\forall n\geq N}\abs{\lambda u_n+\mu v_n-\lambda l-\mu l\prim}&=\abs{\lambda\paren{u_n-l}+\mu\paren{v_n-l\prim}} \\
&\leq\abs{\lambda}\abs{u_n-l}+\abs{\mu}\abs{v_n-l\prim} \\
&\leq\abs{\lambda}\dfrac{\epsilon}{2\paren{\abs{\lambda}+1}}+\abs{\mu}\dfrac{\epsilon}{2\paren{\abs{\mu}+1}} \\
&=\dfrac{\epsilon}{2}\paren{\dfrac{\abs{\lambda}}{\abs{\lambda}+1}+\dfrac{\abs{\mu}}{\abs{\mu}+1}} \\
&\leq\epsilon.
\end{aligned}\]
\end{dem}

\begin{dem}[3]\thlabel{dem:produitLimitesReellesSuites}
Montrons que \(\quantifs{\forall\epsilon\in\Rps;\exists N\in\N;\forall n\geq N}\abs{u_nv_n-ll\prim}\leq\epsilon\).

Soit \(\epsilon\in\Rps\).

On remarque : \[\begin{aligned}
\quantifs{\forall n\in\N}\abs{u_nv_n-ll\prim}&=\abs{u_nv_n-u_nl\prim+u_nl\prim-ll\prim} \\
&\leq\abs{u_nv_n-u_nl\prim}+\abs{u_nl\prim-ll\prim} \\
&=\abs{u_n}\abs{v_n-l\prim}+\abs{u_n-l}\abs{l\prim}.
\end{aligned}\]

Comme \(\paren{u_n}_n\) est convergente, elle est bornée. Soit \(M\in\Rps\) tel que \(\quantifs{\forall n\in\N}\abs{u_n}\leq M\).

Soit \(N_1\in\N\) tel que \(\quantifs{\forall n\geq N_1}\abs{u_n-l}\leq\dfrac{\epsilon}{2\paren{\abs{l'}+1}}\).

Soit \(N_2\in\N\) tel que \(\quantifs{\forall n\geq N_2}\abs{v_n-l\prim}\leq\dfrac{\epsilon}{2M}\).

On pose \(N=\max\accol{N_1;N_2}\).

On a : \[\begin{aligned}
\quantifs{\forall n\geq N}\abs{u_nv_n-ll\prim}&\leq\abs{u_n}\abs{v_n-l\prim}+\abs{u_n-l}\abs{l\prim} \\
&\leq M\dfrac{\epsilon}{2M}+\dfrac{\epsilon}{2\paren{\abs{l\prim}+1}}\abs{l\prim} \\
&\leq\epsilon.
\end{aligned}\]
\end{dem}

\begin{prop}
Soient \(\paren{u_n}_n,\paren{v_n}_n\in\R^\N\).

On suppose \(\lim_nu_n=0\) et \(\paren{v_n}_n\) bornée.

Alors \(\lim_nu_nv_n=0\).
\end{prop}

\begin{dem}\thlabel{dem:limiteProduitSuiteBornéeEtSuiteTendantVersZéroVautZéroSuites}
Montrons que \(\quantifs{\forall\epsilon\in\Rps;\exists N\in\N;\forall n\geq N}\abs{u_nv_n}\leq\epsilon\).

Soit \(\epsilon\in\Rps\).

Soit \(M\in\Rps\) tel que \(\quantifs{\forall n\in\N}\abs{v_n}\leq M\).

Soit \(N\in\N\) tel que \(\quantifs{\forall n\geq N}\abs{u_n}\leq\dfrac{\epsilon}{M}\).

On a \(\quantifs{\forall n\geq N}\abs{u_nv_n}\leq\dfrac{\epsilon}{M}M=\epsilon\).

Donc \(N\) convient.

Donc \(\lim_nu_nv_n=0\).
\end{dem}

\begin{prop}
Soit \(\paren{u_n}_n\in\R^\N\).

On suppose que \(\lim_nu_n=l\in\Rs\).

Alors \(\paren{\dfrac{1}{u_n}}_n\) est \guillemets{bien définie à partir d'un certain rang}.

\Cad : \(\quantifs{\exists N_0\in\N;\forall n\geq N_0}u_n\not=0\).

De plus, \(\lim_n\dfrac{1}{u_n}=\dfrac{1}{l}\).
\end{prop}

\begin{dem}\thlabel{dem:inverseLimiteReelleSuite}
Supposons \(l>0\).

Comme \(0\) minore strictement \(\lim_nu_n\), \(0\) minore strictement \(\paren{u_n}_n\) à partir d'un certain rang \(N_0\in\N\) : \(\quantifs{\forall n\geq N_0}0<u_n\).

Montrons que \(\paren{\dfrac{1}{u_n}}_{n\geq N_0}\) converge vers \(\dfrac{1}{l}\).

Soit \(\epsilon\in\Rps\).

On a \(\quantifs{\forall n\in\N}\abs{\dfrac{1}{u_n}-\dfrac{1}{l}}=\dfrac{\abs{l-u_n}}{u_nl}=\dfrac{1}{\abs{l}}\dfrac{1}{\abs{u_n}}\abs{l-u_n}\).

On a \(\dfrac{l}{2}<\lim_nu_n\).

Donc il existe \(N_1\in\N\) tel que \(\quantifs{\forall n\geq N_1}\dfrac{l}{2}<u_n\).

Soit \(N_2\in\N\) tel que \(\quantifs{\forall n\geq N_2}\abs{u_n-l}\leq\dfrac{\epsilon l^2}{2}\).

On pose \(N=\max\accol{N_1;N_2}\).

On a \(\quantifs{\forall n\geq N}\abs{\dfrac{1}{u_n}-\dfrac{1}{l}}\leq\dfrac{1}{l}\dfrac{2}{l}\dfrac{\epsilon l^2}{2}=\epsilon\).

Donc \(\lim_n\dfrac{1}{u_n}=\dfrac{1}{l}\).

De même si \(l<0\).
\end{dem}

\begin{cor}
Soient deux suites convergentes \(\paren{u_n}_n,\paren{v_n}_n\in\R^\N\) de limites respectives \(l,l\prim\in\R\).

Si \(l\prim\not=0\) alors \(\lim_n\dfrac{u_n}{v_n}=\dfrac{l}{l\prim}\).
\end{cor}

\begin{dem}
On a \(\lim_nv_n=l\prim\not=0\) donc \(\lim_n\dfrac{1}{v_n}=\dfrac{1}{l\prim}\).

Ainsi, on a \(\lim_n\dfrac{u_n}{v_n}=\dfrac{l}{l\prim}\) par produit.
\end{dem}

\section{Limites infinies}

\begin{defi}
Soit \(\paren{u_n}_n\in\R^\N\).

On dit que \(\paren{u_n}_n\) tend (ou diverge) vers \(\pinf\) si on a \[\quantifs{\forall\alpha\in\R;\exists N\in\N;\forall n\geq N}u_n\geq\alpha.\]

On note alors \(\lim_nu_n=\pinf\) et on dit que \(\pinf\) est la limite de \(\paren{u_n}_n\).

De même, on dit que \(\paren{u_n}_n\) diverge (ou tend) vers \(\minf\) si on a \[\quantifs{\forall\alpha\in\R;\exists N\in\N;\forall n\geq N}u_n\leq\alpha.\]
\end{defi}

\begin{rem}
Une suite convergente est une suite admettant une limite finie \(l\in\R\).

Une suite divergente est une suite admettant une limite infinie \(l\in\accol{\minf;\pinf}\) ou n'admettant pas de limite.
\end{rem}

\begin{ex}
On a les limites suivantes :

\begin{itemize}
\item \(\lim_nn=\pinf\) \\

\item \(\lim_nn!=\pinf\) \\

\item \(\lim_n\e{n}=\pinf\) \\

\item \(\lim_n\ln n=\pinf\) \\
\end{itemize}
\end{ex}

\begin{dem}
Montrons que \(\lim_n\ln n=\pinf\), \cad \[\quantifs{\forall\alpha\in\R;\exists N\in\N;\forall n\geq N}\ln n\geq\alpha.\]

Soit \(\alpha\in\R\).

On a \[\begin{aligned}
\quantifs{\forall n\in\Ns}\ln n\geq\alpha&\ssi n\geq\e{\alpha} \\
&\impr n\geq\floor{\e{\alpha}}+1
\end{aligned}\]

Donc l'entier \(N=\floor{\e{\alpha}}+1\) convient.
\end{dem}

\begin{prop}
Toute suite qui tend vers \(\pinf\) est minorée.

Toute suite qui tend vers \(\minf\) est majorée.
\end{prop}

\begin{dem}
Soit \(\paren{u_n}_n\in\R^\N\) une suite de limite \(\pinf\).

Il existe \(N\in\N\) tel que \(\quantifs{\forall n\geq N}0\leq u_n\).

Posons \(m=\min\accol{u_0;\dots;u_{N-1};0}\).

On a \(\quantifs{\forall n\in\N}m\leq u_n\).

Donc \(\paren{u_n}_n\) est minorée.

Idem si \(\lim_nu_n=\minf\).
\end{dem}

\begin{prop}
Soient \(\paren{u_n}_n,\paren{v_n}_n\in\R^\N\).

Si \(\begin{dcases}\lim_nu_n=\pinf \\ \paren{v_n}_n\text{ minorée}\end{dcases}\) alors \(\lim_n\paren{u_n+v_n}=\pinf\).

Si \(\begin{dcases}\lim_nu_n=\minf \\ \paren{v_n}_n\text{ majorée}\end{dcases}\) alors \(\lim_n\paren{u_n+v_n}=\minf\).
\end{prop}

\begin{dem}\thlabel{dem:limiteSommeSuiteMajoréeOuMinoréeEtSuiteDivergente}
Supposons \(\lim_nu_n=\pinf\) et \(\paren{v_n}_n\) minorée.

Soit \(m\in\R\) un minorant de \(\paren{v_n}_n\) : \(\quantifs{\forall n\in\N}m\leq v_n\).

Montrons que \(\lim_n\paren{u_n+v_n}=\pinf\), \cad \[\quantifs{\forall\alpha\in\R;\exists N\in\N;\forall n\geq N}\alpha\leq u_n+v_n.\]

Soit \(\alpha\in\R\).

Soit \(N\in\N\) tel que \(\quantifs{\forall n\geq N}u_n\geq\alpha-v_n\).

On a \(\quantifs{\forall n\geq N}u_n+v_n\geq\alpha-m+m=\alpha\).

Donc \(N\) convient.

Donc \(\lim_n\paren{u_n+v_n}=\pinf\).

Idem pour l'autre implication.
\end{dem}

\section{Suites extraites}

\begin{defi}
Soit \(\paren{u_n}_n\in\R^\N\).

On appelle suite extraite de \(\paren{u_n}_n\) toute suite de la forme \(\paren{u_{\phi\paren{n}}}_{n\in\N}\) où \(\phi:\N\to\N\) strictement croissante.
\end{defi}

\begin{rem}
Soit \(\phi:\N\to\N\) strictement croissante.

On a \(\quantifs{\forall n\in\N}\phi\paren{n}\geq n\).
\end{rem}

\begin{dem}
Par récurrence sur \(n\in\N\) :

On a \(\phi\paren{0}\in\N\) donc \(\phi\paren{0}\geq 0\).

Soit \(n\in\N\) tel que \(\phi\paren{n}\geq n\).

On a \(\phi\paren{n+1}>\phi\paren{n}\) car \(\phi\) strictement croissante.

Donc \(\phi\paren{n+1}\geq\phi\paren{n}+1\geq n+1\).

Ainsi, on a \(\quantifs{\forall n\in\N}\phi\paren{n}\geq n\).
\end{dem}

\begin{ex}
Prenons \(\quantifs{\forall n\in\N}u_n=\paren{-1}^n\).

Si \(\phi:n\mapsto2n\) alors on obtient la suite extraite \(\paren{u_{2n}}_n=\paren{\paren{-1}^{2n}}_n=\paren{1}_n\).

Si \(\phi:n\mapsto2n+1\) alors on obtient la suite extraite \(\paren{u_{2n+1}}_n=\paren{\paren{-1}^{2n+1}}_n=\paren{-1}_n\).

Prenons maintenant \(\quantifs{\forall n\in\Ns}u_n=\dfrac{1}{n}\).

Si \(\phi:n\mapsto2^n\) alors \(\paren{u_{\phi\paren{n}}}_{n\geq1}=\paren{\dfrac{1}{2^n}}_{n\geq1}\).

Si \(\phi:n\mapsto n^2\) alors \(\paren{u_{\phi\paren{n}}}_{n\geq1}=\paren{\dfrac{1}{n^2}}_{n\geq1}\).
\end{ex}

\begin{theo}\thlabel{theo:suiteExtraiteTendVersLaMemeLimite}
Soient \(\paren{u_n}_n\in\R^\N\) et \(l\in\Rb\).

On suppose \(\lim_nu_n=l\).

Alors toute suite extraite de \(\paren{u_n}_n\) tend vers \(l\).
\end{theo}

\begin{dem}
Soit \(\phi:\N\to\N\) strictement croissante.

Montrons que \(\lim_nu_{\phi\paren{n}}=l\).

Supposons \(l\in\R\).

Montrons que \[\quantifs{\forall\epsilon\in\Rps;\exists N\in\N;\forall n\geq N}\abs{u_{\phi\paren{n}}-l}\leq\epsilon.\]

Soit \(\epsilon\in\Rps\).

Soit \(N\in\N\) tel que \(\quantifs{\forall k\geq N}\abs{u_k-l}\leq\epsilon\).

On remarque \(\quantifs{\forall n\geq N}\abs{u_{\phi\paren{n}}-l}\leq\epsilon\) car \(\phi\paren{n}\geq n\geq N\).

Donc \(N\) convient.

Idem si \(l=\pm\infty\).
\end{dem}

\begin{ex}[Exemple d'application]
Montrons que \(\paren{\paren{-1}^n}_{n\in\N}\) n'admet pas de limite.

Par l'absurde, supposons que la suite admet une limite \(l\in\Rb\).

Alors les suites extraites \(\paren{\paren{-1}^{2n}}_{n\in\N}\) et \(\paren{\paren{-1}^{2n+1}}_{n\in\N}\) tendent aussi vers \(l\).

Or, les limites respectives de ces suites sont \(1\) et \(-1\).

Donc par unicité de la limite, \(l=1=-1\) : contradiction.

Donc \(\paren{\paren{-1}^n}_{n\in\N}\) n'admet pas de limite.
\end{ex}

\section{Opérations sur les limites}

\begin{theo}
Soient \(\paren{u_n}_n,\paren{v_n}_n\in\R^\N\). Soient \(l,l\prim\in\Rb\).

On suppose que \(\lim_nu_n=l\) et \(\lim_nv_n=l\prim\).

\begin{enumerate}
\item Si \(\paren{l,l\prim}\not\in\accol{\paren{\pinf,\minf};\paren{\minf,\pinf}}\) alors \[\lim_n\paren{u_n+v_n}=l+l\prim.\] \\

\item Si \(\paren{l,l\prim}\not\in\accol{\paren{0,\pinf};\paren{0,\minf};\paren{\pinf,0};\paren{\minf,0}}\) alors \[\lim_nu_nv_n=ll\prim.\] \\

\item Si \(l\not=0\) alors \[\lim_n\dfrac{1}{u_n}=\begin{dcases}0&\text{si }l\in\accol{\minf;\pinf} \\ \dfrac{1}{l}&\text{sinon}\end{dcases}\] \\

\item Si \(\lim_nu_n=0^+\), respectivement \(0^-\), alors \[\begin{dcases}\paren{\dfrac{1}{u_n}}_n\text{ est définie à partir d'un certain rang} \\ \lim_n\dfrac{1}{u_n}=\pinf\text{, respectivement }\minf\end{dcases}\].
\end{enumerate}
\end{theo}

\begin{dem}[1]
Si \(l,l\prim\in\R\), voir la \thref{dem:sommeLimitesReellesSuites}.

Si \(l=\pinf\) et \(l\prim\not=\minf\) alors \[\begin{dcases}\lim_nu_n=\pinf \\ \paren{v_n}_n\text{ minorée car }\begin{dcases}\paren{v_n}_n&\text{ bornée si }l\prim\in\R \\ \paren{v_n}_n&\text{ minorée si }l\prim=\pinf\end{dcases}\end{dcases}\]

Donc \(\lim_n\paren{u_n+v_n}=\pinf\).

Idem si \(l\prim=\pinf\) et \(l\not=\minf\).

Idem si \(l=\minf\) ou \(l\prim=\minf\).
\end{dem}

\begin{dem}[2]
Si \(l,l\prim\in\R\), voir la \thref{dem:produitLimitesReellesSuites}.

Si \(l=\pinf=l\prim\), montrons que \(\lim_nu_nv_n=\pinf\), \cad \[\quantifs{\forall\alpha\in\Rp;\exists N\in\N;\forall n\geq N}u_nv_n\geq\alpha.\]

Soit \(\alpha\in\Rp\).

Soit \(N_1\in\N\) tel que \(\quantifs{\forall n\geq N_1}u_n\geq\alpha\).

Soit \(N_2\in\N\) tel que \(\quantifs{\forall n\geq N_2}v_n\geq1\).

Posons \(N=\max\accol{N_1;N_2}\).

On a \[\quantifs{\forall n\geq N}u_nv_n\geq\alpha\] donc \(N\) convient.

Si \(l=\pinf\) et \(l\prim\in\Rps\), on a \(\dfrac{l\prim}{2}<\lim_nv_n\).

Donc il existe un rang \(N_1\in\N\) tel que \(\quantifs{\forall n\geq N_1}\dfrac{l\prim}{2}<v_n\).

Montrons que \(\lim_nu_nv_n=\pinf\), \cad \[\quantifs{\forall\alpha\in\Rp;\exists N\in\N;\forall n\geq N}u_nv_n\geq\alpha.\]

Soit \(\alpha\in\Rp\).

Soit \(N_2\in\N\) tel que \(\quantifs{\forall n\geq N_2}u_n\geq\dfrac{2\alpha}{l\prim}\).

On pose \(N=\max\accol{N_1;N_2}\).

On a \(\quantifs{\forall n\geq N}u_nv_n\geq\dfrac{2\alpha}{l\prim}\times\dfrac{l\prim}{2}=\alpha\).

Donc \(\lim_nu_nv_n=\pinf=ll\prim\).

Idem dans les autres cas.
\end{dem}

\begin{dem}[3]
Si \(l\in\R\), voir la \thref{dem:inverseLimiteReelleSuite}.

Si \(l=\pinf\), montrons que \(\lim_n\dfrac{1}{u_n}=0\).

La suite \(\paren{\dfrac{1}{u_n}}_n\) est bien définie à partir d'un certain rang car on a \[\quantifs{\exists N_0\in\N;\forall n\geq N_0}u_n\geq1\not=0.\]

Soit \(N_0\in\N\) tel que \(\quantifs{\forall n\geq N_0}u_n\not=0\).

Montrons que \[\quantifs{\forall\epsilon\in\Rps;\exists N\geq N_0;\forall n\geq N}\abs{\dfrac{1}{u_n}}\leq\epsilon.\]

Soit \(\epsilon\in\Rps\).

Soit \(N_1\geq N_0\) tel que \(\quantifs{\forall n\geq N_1}u_n\geq\dfrac{1}{\epsilon}\).

On a \(\quantifs{\forall n\geq N_1}\abs{\dfrac{1}{u_n}}=\dfrac{1}{u_n}\leq\epsilon\).

Idem si \(l=\minf\)
\end{dem}

\begin{dem}[4]
Si \(\lim_nu_n=0^+\) alors il existe un rang \(N_0\in\N\) tel que \(\quantifs{\forall n\geq N_0}u_n>0\).

En particulier, \(\paren{\dfrac{1}{u_n}}_{n\geq N_0}\) est bien définie.

Montrons que \(\paren{\dfrac{1}{u_n}}_{n\geq N_0}\) tend vers \(\pinf\), \cad \[\quantifs{\forall\alpha\in\Rps;\exists N\geq N_0;\forall n\geq N}\dfrac{1}{u_n}\geq\alpha.\]

Soit \(\alpha\in\Rps\).

Soit \(N\geq N_0\) tel que \(\quantifs{\forall n\geq N}u_n\leq\dfrac{1}{\alpha}\).

On a donc \(\quantifs{\forall n\geq N}0<u_n\leq\dfrac{1}{\alpha}\).

D'où \(\quantifs{\forall n\geq N}\alpha\leq\dfrac{1}{u_n}\) car \(t\mapsto\dfrac{1}{t}\) est décroissante sur \(\Rps\).

Donc \(\lim_n\dfrac{1}{u_n}=\pinf\).

Idem si \(\lim_nu_n=0^-\).
\end{dem}

\begin{theo}[Théorème des gendarmes dans le cas d'une limite infinie]
Soient \(\paren{u_n}_n,\paren{v_n}_n\in\R^\N\).

Si \(\begin{dcases}\lim_nu_n=\pinf \\ \quantifs{\forall n\in\N}u_n\leq v_n\end{dcases}\) alors \(\lim_nv_n=\pinf\).

Si \(\begin{dcases}\lim_nv_n=\minf \\ \quantifs{\forall n\in\N}v_n\geq u_n\end{dcases}\) alors \(\lim_nu_n=\minf\).
\end{theo}

\begin{dem}\thlabel{dem:théorèmeDesGendarmesDansLeCasInfiniSuites}
Supposons \(\lim_nu_n=\pinf\) et \(\quantifs{\forall n\in\N}u_n\leq v_n\).

On a \(\quantifs{\forall\alpha\in\R;\exists N\in\N;\forall n\geq N}\alpha\leq u_n\).

Donc \(\quantifs{\forall\alpha\in\R;\exists N\in\N;\forall n\geq N}\alpha\leq v_n\).

Idem si \(\lim_nv_n=\minf\) et \(\quantifs{\forall n\in\N}v_n\geq u_n\).
\end{dem}

\begin{theo}[Passage à la limite dans les inégalités larges]
Soient \(\paren{u_n}_n,\paren{v_n}_n\) deux suites convergentes.

On suppose \(\quantifs{\forall n\in\N}u_n\leq v_n\).

Alors \(\lim_nu_n\leq\lim_nv_n\).
\end{theo}

\begin{dem}
On sait que \(\paren{v_n-u_n}_n\) est convergente et de limite \(\lim_nv_n-\lim_nu_n\).

Donc on a \(\quantifs{\forall n\in\N}0\leq v_n-u_n\).

Donc \(0\geq\lim_nv_n-\lim_nu_n\) d'après la \thref{prop:passageALaLimiteDansUneInegaliteLargeAvecUnScalaire}.

Donc \(\lim_nu_n\leq\lim_nv_n\).
\end{dem}

\section{Suites monotones}

\begin{theo}[Théorème de la limite monotone]
Toute suite monotone admet une limite.

Soit \(\paren{u_n}_n\in\R^\N\) une suite monotone.

Si \(\paren{u_n}_n\) est croissante et non-majorée alors \(\lim_nu_n=\pinf\).

Si \(\paren{u_n}_n\) est croissante et majorée alors \(\lim_nu_n=\sup_{n\in\N}u_n\).

Si \(\paren{u_n}_n\) est décroissante et non-minorée alors \(\lim_nu_n=\minf\).

Si \(\paren{u_n}_n\) est décroissante et minorée alors \(\lim_nu_n=\inf_{n\in\N}u_n\).
\end{theo}

\begin{dem}
Supposons \(\paren{u_n}_n\) croissante et non-majorée.

Montrons que \(\lim_nu_n=\pinf\), \cad \[\quantifs{\forall\alpha\in\R;\exists N\in\N;\forall n\geq N}u_n\geq\alpha.\]

Soit \(\alpha\in\R\).

Comme \(\paren{u_n}_n\) n'est pas majorée, il existe \(N\in\N\) tel que \(\alpha<u_N\).

On a \(\quantifs{\forall n\geq N}u_n\geq u_N>\alpha\).

Donc \(N\) convient et on a \(\lim_nu_n=\pinf\).

Supposons maintenant \(\paren{u_n}_n\) croissante et majorée.

Comme \(\accol{u_n}_{n\in\N}\) est une partie non-vide et majorée de \(\R\), elle admet une borne supérieure \(S\).

\(S\) vérifie \(\begin{dcases}\quantifs{\forall n\in\N}u_n\leq S \\ \quantifs{\forall\epsilon\in\Rps;\exists N\in\N}S-\epsilon\leq u_N\end{dcases}\)

Montrons que \(\lim_nu_n=S\), \cad \[\quantifs{\forall\epsilon\in\Rps;\exists N\in\N;\forall n\geq N}S-\epsilon\leq u_n\leq S+\epsilon.\]

Soit \(\epsilon\in\Rps\).

Soit \(N\in\N\) tel que \(S-\epsilon\leq u_N\).

On a \(\quantifs{\forall n\geq N}S-\epsilon\leq u_n\leq S+\epsilon\).

Donc \(N\) convient et \(\lim_nu_n=S\).

Idem pour les suites décroissantes.
\end{dem}

\begin{theo}[Théorème des suites adjacentes]
Soient \(\paren{a_n}_n,\paren{b_n}_n\in\R^\N\) telles que \(\paren{a_n}_n\) croissante, \(\paren{b_n}_n\) décroissante et \(\lim_n\paren{b_n-a_n}=0\).

Deux telles suites sont dites adjacentes.

Alors \(\paren{a_n}_n\) et \(\paren{b_n}_n\) sont convergentes et de même limite \(l\).

On a alors \(\quantifs{\forall n\in\N}a_n\leq l\leq b_n\).
\end{theo}

\begin{dem}
La suite \(\paren{b_n-a_n}_n\) est décroissante et convergente donc décroissante et minorée.

Elle converge donc vers sa borne inférieure.

Donc \(\inf_{n\in\N}\paren{b_n-a_n}=0\) donc \(\quantifs{\forall n\in\N}b_n-a_n\geq0\) donc \(\quantifs{\forall n\in\N}a_n\leq b_n\).

De plus, on a \(\paren{a_n}_n\) croissante et \(\paren{b_n}_n\) décroissante.

Donc \(\quantifs{\forall n\in\N}a_0\leq a_n\leq b_n\leq b_0\).

Ainsi, \(\paren{a_n}_n\) est croissante et majorée donc convergente et \(\paren{b_n}_n\) est décroissante et minorée donc convergente.

Enfin, on a \(0=\lim_n\paren{b_n-a_n}=\lim_nb_n-\lim_na_n\).
\end{dem}

\begin{rem}
Toute suite décroissante et convergente est minorée par sa limite.

De même, toute suite croissante et convergente est majorée par sa limite.
\end{rem}

\section{Retour sur les suites extraites}

\begin{prop}
Soit \(\paren{u_n}_n\in\R^\N\). Soit \(l\in\Rb\).

On a \[\lim_nu_n=l\ssi\begin{dcases}\lim_ku_{2k}=l \\ \lim_ku_{2k+1}=l\end{dcases}\]
\end{prop}

\begin{dem}
\impdir Déjà vue (\thref{theo:suiteExtraiteTendVersLaMemeLimite}).

\imprec

Supposons \(\lim_ku_{2k}=\lim_ku_{2k+1}=l\).

Supposons \(l\in\R\).

Montrons que \(\quantifs{\forall\epsilon\in\Rps;\exists N\in\N;\forall n\geq N}\abs{u_n-l}\leq\epsilon\).

Soit \(\epsilon\in\Rps\).

Soit \(K_1\in\N\) tel que \(\quantifs{\forall k\geq K_1}\abs{u_{2k}-l}\leq\epsilon\).

Soit \(K_2\in\N\) tel que \(\quantifs{\forall k\geq K_2}\abs{u_{2k+1}-l}\leq\epsilon\).

Posons \(K=\max\accol{K_1;K_2}\).

On a \(\quantifs{\forall k\geq K}\begin{dcases}\abs{u_{2k}-l}\leq\epsilon \\ \abs{u_{2k+1}-l}\leq\epsilon\end{dcases}\)

Donc \(\quantifs{\forall n\geq 2K+1}\abs{u_n-l}\leq\epsilon\).

En effet, si \(n\) est pair, il existe \(k\in\N\) tel que \(n=2k\) et on a \(2k\geq 2K+1\) donc \(k\geq K\).

Si \(n\) est impair, il existe \(k\in\N\) tel que \(n=2k+1\) et on a \(2k+1\geq 2K+1\) donc \(k\geq K\).

L'entier \(N=2k+1\) convient et on a \(\lim_nu_n=l\).
\end{dem}

\begin{theo}[Théorème de Bolzano-Weierstrass]
De toute suite réelle bornée, on peut extraire une suite convergente.
\end{theo}

\begin{dem}
Soit \(\paren{u_n}_n\in\R^\N\) une suite bornée.

Soit \(M\in\Rp\) tel que \(\quantifs{\forall n\in\N}\abs{u_n}\leq M\).

On a \(\quantifs{\forall n\in\N}u_n\in\intervii{-M}{M}\).

S'il existe une infinité d'indices \(n\in\N\) tels que \(u_n\leq0\), on construit par récurrence la fonction \(\phi_1:\N\to\N\) : \[\begin{dcases}\phi_1\paren{0}=\min\accol{n\in\N\tq u_n\leq0} \\ \quantifs{\forall n\in\Ns}\phi_1\paren{n}=\min\accol{k\in\interventierie{\phi_1\paren{n-1}+1}{\pinf}\tq u_k\leq0}&\text{ (partie non-vide de \(N\))}\end{dcases}\]

On obtient ainsi une fonction \(\phi_1\) strictement croissante telle que \(\quantifs{\forall n\in\N}u_{\phi_1\paren{n}}\leq0\).

Sinon, on définit de même une fonction \(\phi_1:\N\to\N\) strictement croissante telle que \(\quantifs{\forall n\in\N}u_{\phi_1\paren{n}}\geq0\).

On pose \(\begin{dcases}a_0=-M \\ b_0=M\end{dcases}\)

Puis, dans le premier cas \(\begin{dcases}a_1=-M \\ b_1=0\end{dcases}\) et dans le second cas \(\begin{dcases}a_1=0 \\ b_1=M\end{dcases}\)

Ainsi, \(\quantifs{\forall n\in\N}u_{\phi_1\paren{n}}\in\intervii{a_1}{b_1}\).

De même que précédemment :

Si \(\Card\accol{n\in\N\tq u_{\phi_1\paren{n}}\leq\dfrac{a_1+b_1}{2}}=\pinf\) alors on pose \(\begin{dcases}a_2=a_1 \\ b_2=\dfrac{a_1+b_1}{2}\end{dcases}\)

Sinon, on pose \(\begin{dcases}a_2=\dfrac{a_1+b_1}{2} \\ b_2=b_1\end{dcases}\)

Dans les deux cas, on considère \(\phi_2:\N\to\N\) strictement croissante telle que \[\quantifs{\forall n\in\N}u_{\phi_1\rond\phi_2\paren{n}}\in\intervii{a_2}{b_2}.\]

On continue le procédé et on définit ainsi \(\paren{a_n}_n\) une suite croissante, \(\paren{b_n}_n\) une suite décroissante et \(\paren{\phi_k}\) une suite de fonctions strictement croissantes, telles que \[\begin{dcases}\quantifs{\forall n\in\N}b_n-a_n=\dfrac{2M}{2^n} \\ \quantifs{\forall k\in\N;\forall n\in\N}u_{\phi_1\rond\dots\rond\phi_k\paren{n}}\in\intervii{a_k}{b_k}\end{dcases}\]

On pose enfin \(\quantifs{\forall n\in\N}\psi\paren{n}=\phi_1\rond\dots\rond\phi_n\paren{n}\), \cad \[\begin{aligned}
\psi\paren{0}&=0 \\
\psi\paren{1}&=\phi_1\paren{1} \\
\psi\paren{2}&=\phi_1\rond\phi_2\paren{2} \\
\psi\paren{3}&=\phi_1\rond\phi_2\rond\phi_3\paren{3} \\
&\vdots
\end{aligned}\]

Montrons que \(\psi\) est strictement croissante.

Soit \(n\in\N\). Montrons que \(\psi\paren{n}<\psi\paren{n+1}\).

On a \[\begin{aligned}
\psi\paren{n}&=\phi_1\rond\dots\rond\phi_n\paren{n} \\
&<\phi_1\rond\dots\rond\phi_n\paren{n+1} \\
&\leq \phi_1\rond\dots\rond\phi_n\paren{\phi_{n+1}\paren{n+1}}\quad\text{car }\phi_{n+1}\paren{n+1}\geq n+1 \\
&=\psi\paren{n+1}
\end{aligned}\]

Donc la suite \(\paren{u_{\psi\paren{n}}}_n\) est une suite extraite de \(\paren{u_n}_n\).

Appliquons le théorème des suites adjacentes à \(\paren{a_n}_n\) et \(\paren{b_n}_n\) :

On a \(\begin{dcases}\paren{a_n}_n\text{ croissante} \\ \paren{b_n}_n\text{ décroissante} \\ \lim_n\paren{b_n-a_n}=0\end{dcases}\) donc \(\paren{a_n}_n\) et \(\paren{b_n}_n\) sont convergentes de même limite.

Enfin, on a \(\quantifs{\forall n\in\N}a_n\leq u_{\psi\paren{n}}\leq b_n\) donc selon le théorème des gendarmes, \(\paren{u_{\psi\paren{n}}}_n\) est convergente.
\end{dem}

\begin{ex}
De \(\paren{\paren{-1}^n}_n\) on peut extraire une suite convergente : \(\paren{\paren{-1}^{2n}}_n\).

De \(\paren{\sin n}_n\) on peut extraire une suite convergente.
\end{ex}

\section{Densité}

\subsection{Rappels}

\begin{defi}
On appelle nombre rationnel tout réel de la forme \(\dfrac{a}{b}\) avec \(\begin{dcases}a\in\Z \\ b\in\Ns\end{dcases}\)

L'ensemble des rationnels est noté : \[\Q=\accol{\dfrac{a}{b}}_{\paren{a,b}\in\Z\times\Ns}\]

Tout nombre rationnel s'écrit de façon unique \(\dfrac{a}{b}\) (écriture irréductible) avec \(\begin{dcases}a\in\Z \\ b\in\Ns \\ a\text{ et }b\text{ premiers entre eux}\end{dcases}\)
\end{defi}

\begin{defi}
On appelle nombre décimal tout nombre réel de la forme \(\dfrac{a}{10^\alpha}\) où \(a\in\Z\) et \(\alpha\in\N\).
\end{defi}

\begin{rem}
Soit \(x\in\R\). On a \(x\text{ décimal}\imp x\text{ rationnel}\).
\end{rem}

\begin{ex}~\\
\(\dfrac{1}{3}\) est rationnel non-décimal.

\(\sqrt{2},\pi,\e{},\dots\in\R\excluant\Q\).
\end{ex}

\begin{dem}
Par l'absurde, supposons \(\sqrt{2}\in\Q\).

Considérons l'écriture irréductible de \(\sqrt{2}\) : \(\sqrt{2}=\dfrac{a}{b}\) avec \(a\in\Z\) et \(b\in\Ns\) premiers entre eux.

On a \(\sqrt{2}b=a\)

donc \(2b^2=a^2\)

donc \(a^2\) pair

donc \(a\) pair

donc il existe \(k\in\Z\) tel que \(a=2k\)

donc on a \(2b^2=\paren{2k}^2\)

donc \(b^2=2k^2\)

donc \(b\) est pair donc \(2\) divise \(a\) et \(b\) : contradiction.
\end{dem}

\subsection{Densité}

\begin{defi}\thlabel{defi:partieDenseDansR}
Soit \(A\subset\R\).

On dit que \(A\) est dense dans \(\R\) si on a \[\quantifs{\forall a,b\in\R}a<b\imp\intervee{a}{b}\inter A\not=\ensvide.\]
\end{defi}

\begin{ex}
\(\R\) est dense dans \(\R\).

\(\Rs\) est dense dans \(\R\).
\end{ex}

\begin{prop}
\begin{enumerate}
\item L'ensemble des décimaux est dense dans \(\R\). \\

\item \(\Q\) est dense dans \(\R\). \\

\item \(\R\excluant\Q\) est dense dans \(\R\). \\
\end{enumerate}
\end{prop}

\begin{dem}[1]
Notons \(\D\) l'ensemble des décimaux.

Soient \(a,b\in\R\) tels que \(a<b\).

On a \(b-a>0\).

Donc il existe \(\alpha\in\N\) tel que \(10^\alpha\paren{b-a}\geq1\) (il suffit de prendre \(\alpha=\floor{\dfrac{-\ln\paren{b-a}}{\ln10}}+1\)).

Montrons que \(10^\alpha a\leq\floor{10^\alpha b}\leq10^\alpha b\).

On a \(\floor{10^\alpha b}\geq10^\alpha b-1\geq10^\alpha a\), d'où l'encadrement.

Ainsi, \(\dfrac{\floor{10^\alpha b}}{10^\alpha}\in\intervii{a}{b}\).

Donc \(\D\inter\intervii{a}{b}\not=\ensvide\).

Montrons que \(\D\inter\intervee{a}{b}\not=\ensvide\).

On pose \(a\prim=\dfrac{2}{3}a+\dfrac{1}{3}b=a+\dfrac{1}{3}\paren{b-a}\) et \(b\prim=\dfrac{1}{3}a+\dfrac{2}{3}b=a+\dfrac{2}{3}\paren{b-a}\).

On a \(b\prim-a\prim=\dfrac{1}{3}\paren{b-a}>0\).

Donc d'après ce qui précède, \(\D\inter\intervee{a\prim}{b\prim}\not=\ensvide\).

Donc \(\D\inter\intervee{a}{b}\not=\ensvide\).
\end{dem}

\begin{dem}[2]
Comme \(\D\subset\Q\), on en déduit que \(\Q\) est dense aussi.
\end{dem}

\begin{dem}[3]
Montrons que \(\R\excluant\Q\) est dense dans \(\R\).

Soient \(a,b\in\R\) tels que \(a<b\).

On a \(\dfrac{a}{\sqrt{2}}<\dfrac{b}{\sqrt{2}}\).

Comme \(\Q\) est dense dans \(\R\), il existe \(q\in\Q\inter\intervee{\dfrac{a}{\sqrt{2}}}{\dfrac{b}{\sqrt{2}}}\).

On a \(\dfrac{a}{\sqrt{2}}<q<\dfrac{b}{\sqrt{2}}\).

Donc \(a<\sqrt{2}q<b\).

Si \(q\not=0\) alors \(\sqrt{2}q\in\R\).

En effet, par l'absurde, soient \(a_1,a_2\in\Z\) et \(b_1,b_2\in\Ns\) tels que \(q=\dfrac{a_1}{b_1}\) et \(\sqrt{2}q=\dfrac{a_2}{b_2}\). On a \(\dfrac{a_1}{b_1}\sqrt{2}=\dfrac{a_2}{b_2}\). Comme \(q\not=0\), on a \(a_1\not=0\) donc \(\sqrt{2}=\dfrac{a_2b_1}{a_1b_2}\in\Q\) : contradiction.

Donc \(\sqrt{2}q\in\paren{\R\excluant\Q}\inter\intervee{a}{b}\).

Finalement, si \(0\not\in\intervee{a}{b}\) alors \(\paren{\R\excluant\Q}\inter\intervee{a}{b}\not=\ensvide\).

Sinon, on a \(a<0<b\) donc \(\intervee{0}{b}\) contient un irrationnel donc \(\intervee{a}{b}\) aussi.
\end{dem}

\subsection{Caractérisation séquentielle de la densité}

\begin{prop}
Soit \(A\subset\R\).

\(A\) est dense dans \(\R\) si et seulement si tout réel est limite d'une suite d'éléments de \(A\) : \[\quantifs{\forall x\in\R;\exists\paren{a_n}_n\in A^\N}\lim_na_n=x.\]
\end{prop}

\begin{dem}
\imprec

Supposons que tout réel est limite d'une suite d'éléments de \(A\).

Montrons que \(A\) est dense.

Soient \(a,b\in\R\) tels que \(a<b\). Montrons que \(A\inter\intervee{a}{b}\not=\ensvide\).

Soit \(\paren{a_n}_n\in A^\N\) une suite qui tend vers \(\dfrac{a+b}{2}\).

Posons \(\epsilon=\dfrac{b-a}{2}\).

Soit \(N\in\N\) tel que \(\quantifs{\forall n\geq N}\abs{a_n-\dfrac{a+b}{2}}\leq\epsilon\).

On remarque \(\begin{dcases}a_N\in A \\ \abs{a_N-\dfrac{a+b}{2}}<\epsilon\end{dcases}\)

Donc \(a=\dfrac{a+b}{2}-\epsilon<a_N<\dfrac{a+b}{2}+\epsilon=b\).

Ainsi, \(a_N\in A\inter\intervee{a}{b}\).

\impdir

Supposons \(A\) dense dans \(\R\).

Soit \(x\in\R\).

Pour tout \(n\in\Ns\), on a \[A\inter\intervee{x-\dfrac{1}{n}}{x+\dfrac{1}{n}}\not=\ensvide.\]

Donc il existe un élément \(a_n\in A\inter\intervee{x-\dfrac{1}{n}}{x+\dfrac{1}{n}}\).

On définit ainsi une suite \(\paren{a_n}_n\in\R^{\Ns}\) telle que \[\quantifs{\forall n\in\Ns}\begin{dcases}a_n\in A \\ x-\dfrac{1}{n}<a_n<x+\dfrac{1}{n}\end{dcases}\]

De plus, on a \(\lim_n\paren{x-\dfrac{1}{n}}=\lim_n\paren{x+\dfrac{1}{n}}=x\).

Donc d'après le théorème des gendarmes, \(\lim_na_n=x\).

Enfin, \(\paren{a_n}_n\in A^{\Ns}\).
\end{dem}

\section{Remarque}

En général, ne jamais confondre \(<\) et \(\leq\).

Notamment : \begin{itemize}
\item si deux suites convergentes vérifient \(\quantifs{\forall n\in\N}u_n\leq v_n\) alors \(\lim_nu_n\leq\lim_nv_n\) ; \\

\item tout majorant \textbf{strict} de la limite d'une suite convergente majore \textbf{strictement} la suite à partir d'un certain rang. \\
\end{itemize}

Parfois, on a le choix entre \(<\) et \(\leq\). Par exemple : \begin{itemize}
\item \(\quantifs{\forall\epsilon\in\Rps;\exists N\in\N;\forall n\geq N}\abs{u_n-l}\leq\epsilon\) ; \\

\item \(\quantifs{\forall\alpha\in\R;\exists N\in\N;\forall n\geq N}u_n\geq\alpha\).
\end{itemize}

\section{Suites de nombres complexes}

On rappelle que les suites de nombres complexes (ou suites complexes) sont les familles de complexes indicées par \(\N\) (voire par un intervalle de la forme \(\interventierie{n_0}{\pinf}\) où \(n_0\in\Z\)).

L'ensemble des suites complexes est noté \(\C^\N\) (ou \(\C^{\interventierie{n_0}{\pinf}}\)).

\begin{defi}[Opérations algébriques sur les suites]
De manière analogue au cas réel, si \(\paren{u_n}_n,\paren{v_n}_n\) sont deux suites complexes et \(\lambda,\mu\) deux nombres complexes, on définit :

\begin{itemize}
\item la somme de \(\paren{u_n}_n\) et \(\paren{v_n}_n\) : \[\paren{u_n}_n+\paren{v_n}_n=\paren{u_n+v_n}_n\] \\

\item le produit de \(\paren{u_n}_n\) et \(\paren{v_n}_n\) : \[\paren{u_n}_n\times\paren{v_n}_n=\paren{u_nv_n}_n\] \\

\item la combinaison linéaire de \(\paren{u_n}_n\) et \(\paren{v_n}_n\) : \[\lambda\paren{u_n}_n+\mu\paren{v_n}_n=\paren{\lambda u_n+\mu v_n}_n\] \\
\end{itemize}

Dans le cas complexe, on définit de plus :

\begin{itemize}
\item la partie réelle de \(\paren{u_n}_n\) : \[\Re\paren{\paren{u_n}_n}=\paren{\Re\paren{u_n}}_n\] \\

\item la partie imaginaire de \(\paren{u_n}_n\) : \[\Im\paren{\paren{u_n}_n}=\paren{\Im\paren{u_n}}_n\] \\

\item la suite des modules : \[\paren{\abs{u_n}}_n\] \\

\item la suite conjuguée : \[\paren{\conj{u_n}}_n\] \\
\end{itemize}
\end{defi}

\begin{defi}[Suite bornée]
Soit \(\paren{u_n}_n\) une suite complexe. On dit que \(\paren{u_n}_n\) est bornée si on a : \[\quantifs{\exists M\in\Rp;\forall n\in\N}\abs{u_n}\leq M\] \cad si la suite \(\paren{\abs{u_n}}_n\) est majorée.
\end{defi}

\begin{rem}
La notion de \guillemets{suite bornée} pour les suites complexes généralise celle déjà vue pour les suites réelles. En revanche, on ne généralise pas les notions de suite majorée, minorée, croissante, décroissante ou monotone (car \(\C\) n'est pas ordonné \guillemets{canoniquement}).
\end{rem}

\begin{defprop}[Limite d'une suite complexe]
Soient \(\paren{u_n}_n\in\C^\N\) et \(l\in\C\).

On dit que \(\paren{u_n}_n\) converge (ou tend) vers \(l\) si l'on a : \[\quantifs{\forall\epsilon\in\Rps;\exists N\in\N;\forall n\geq N}\abs{u_n-l}\leq\epsilon.\]

On dit alors que \(l\) est la limite de \(\paren{u_n}_n\). Elle est unique et notée \[l=\lim_{n\to\pinf}u_n=\lim_nu_n.\]

On dit alors aussi que \(\paren{u_n}_n\) est convergente. Sinon on dit qu'elle est divergente.
\end{defprop}

\begin{rem}
\begin{enumerate}
\item On a \[\text{constante}\imp\text{stationnaire}\imp\text{convergente}.\] \\

\item Si \(\paren{u_n}_n\) et \(\paren{v_n}_n\) sont deux suites complexes convergentes qui coïncident à partir d'un certain rang, \cad si l'on a : \[\quantifs{\exists N\in\N;\forall n\geq N}u_n=v_n\] alors \(\paren{u_n}_n\) est convergente si, et seulement si, \(\paren{v_n}_n\) est convergente, et dans ce cas leurs limites sont égales. \\
\end{enumerate}
\end{rem}

\begin{prop}[Une façon de se ramener au cas réel]
Soient \(\paren{u_n}_n\in\C^\N\) et \(l\in\C\).

Alors la suite complexe \(\paren{u_n}_n\) converge vers \(l\) si, et seulement si, la suite réelle \(\paren{\abs{u_n-l}}_n\) converge vers \(0\).
\end{prop}

\begin{dem}
Clair.
\end{dem}

\begin{theo}[Une autre façon de se ramener au cas réel]
Soient \(\paren{u_n}_n\in\C^\N\) et \(l\in\C\).

Alors la suite complexe \(\paren{u_n}_n\) converge vers \(l\) si, et seulement si, \[\begin{dcases}\text{la suite réelle }\paren{\Re u_n}_n\text{ converge vers }\Re l \\ \text{la suite réelle }\paren{\Im u_n}_n\text{ converge vers }\Im l\end{dcases}\]
\end{theo}

\begin{dem}~\\
On pose \(\quantifs{\forall n\in\N}\begin{dcases}a_n=\Re u_n \\ b_n=\Im u_n\end{dcases}\) et \(l_1=\Re l\) et \(l_2=\Im l\).

\impdir

Supposons \(\lim_nu_n=l\).

On a \(\lim_n\abs{u_n-l}=0\).

Or \(\quantifs{\forall n\in\N}\begin{dcases}\abs{a_n-l_1}\leq\abs{u_n-l} \\ \abs{b_n-l_2}\leq\abs{u_n-l}\end{dcases}\) car \(\abs{u_n-l}=\sqrt{\paren{a_n-l_1}^2+\paren{b_n-l_2}^2}\).

Donc selon le théorème des gendarmes, \(\lim_na_n=l_1\) et \(\lim_nb_n=l_2\).

\imprec

Supposons \(\lim_na_n=l_1\) et \(\lim_nb_n=l_2\).

On a \(\lim_n\abs{a_n-l_1}=\lim_n\abs{b_n-l_2}=0\).

Donc \(\lim_n\sqrt{\paren{a_n-l_1}^2+\paren{b_n-l_2}^2}=0\).

Donc \(\lim_n\abs{u_n-l}=0\).

Donc \(\lim_nu_n=l\).
\end{dem}

\begin{prop}
Toute suite convergente est bornée.
\end{prop}

\begin{dem}
Voir le cas des suites réelles convergentes : \thref{dem:suiteReelleConvergenteDoncBornee}.
\end{dem}

\begin{prop}
Soient \(\paren{u_n}_n\) et \(\paren{v_n}_n\) deux suites complexes convergentes et \(\lambda\) et \(\mu\) deux nombres complexes.

On pose \(\begin{dcases}l=\lim_nu_n \\ l\prim=\lim_nv_n\end{dcases}\)

Alors :

\begin{enumerate}
\item \(\lim_n\paren{u_n+v_n}=l+l\prim\) \\

\item \(\lim_nu_nv_n=ll\prim\) \\

\item \(\lim_n\paren{\lambda u_n+\mu v_n}=\lambda l+\mu l\prim\) \\

\item \(\lim_n\Re u_n=\Re l\) \\

\item \(\lim_n\Im u_n=\Im l\) \\

\item \(\lim_n\abs{u_n}=\abs{l}\) \\

\item \(\lim_n\conj{u_n}=\conj{l}\) \\

\item Si \(l\not=0\) alors \(\paren{u_n}_n\) est à termes non-nuls à partir d'un certain rang et on a \(\lim_n\dfrac{1}{u_n}=\dfrac{1}{l}\). \\
\end{enumerate}
\end{prop}

\begin{dem}
\note{EXERCICE}
\end{dem}

\begin{theo}[Théorème de Bolzano-Weierstrass]
De toute suite complexe bornée on peut extraire une suite convergente.
\end{theo}

\begin{dem}
Soit \(\paren{u_n}_n\in\C^\N\) une suite bornée.

On pose \(\quantifs{\forall n\in\N}\begin{dcases}a_n=\Re u_n \\ b_n=\Im u_n\end{dcases}\)

Comme \(\paren{u_n}_n\) est bornée, il existe \(M\in\Rp\) tel que \(\quantifs{\forall n\in\N}\abs{u_n}\leq M\).

On a \(\quantifs{\forall n\in\N}\abs{a_n}=\sqrt{a_n^2}\leq\sqrt{a_n^2+b_n^2}=\abs{u_n}\leq M\).

Donc \(\paren{a_n}_n\) est bornée.

De même, on a \(\quantifs{\forall n\in\N}\abs{b_n}\leq M\).

Donc \(\paren{b_n}_n\) est bornée.

Comme \(\paren{a_n}_n\) est bornée, selon le théorème de Bolzano-Weierstrass réel, il existe \(\phi_1:\N\to\N\) strictement croissante telle que \(\paren{a_{\phi_1\paren{n}}}_n\) converge.

Comme \(\paren{b_n}_n\) est bornée, la suite \(\paren{b_{\phi_1\paren{n}}}_n\) est bornée.

Selon le théorème de Bolzano-Weierstrass réel, il existe \(\phi_2:\N\to\N\) strictement croissante telle que \(\paren{b_{\phi_1\rond\phi_2\paren{n}}}_n\) converge.

Posons \(\phi=\phi_1\rond\phi_2\).

On a \(\phi\) strictement croissante (composée de deux fonctions strictement croissantes).

De plus, on a \(\paren{b_{\phi\paren{n}}}_n\) convergente et \(\paren{a_{\phi\paren{n}}}_n\) convergente (car c'est une suite extraite de \(\paren{a_{\phi_1\paren{n}}}_n\) qui est convergente).

Donc \(\paren{u_{\phi\paren{n}}}_n\) converge.
\end{dem}

\chapter{Algèbre générale}

\minitoc

\section{Lois de composition internes}

\subsection{Définition}

\begin{defi}[Loi de composition interne]
Soit \(E\) un ensemble.

On appelle loi de composition interne sur \(E\) toute application de \(E\times E\) dans \(E\).
\end{defi}

\begin{rem}
Les lois de composition internes sont en général notées avec des symboles tels que : \[+\quad\times\quad\cdot\quad\land\quad\rond\quad*\quad\oplus\quad\otimes.\]

Si \(*\) est une loi de composition interne sur un ensemble \(E\) et \(x,y\) sont deux éléments de \(E\), on préfère noter \(x*y\) plutôt que \(*\paren{x,y}\) l'image du couple \(\paren{x,y}\) par \(*\).
\end{rem}

\begin{ex}\thlabel{ex:loisDeCompositionInternes}
\begin{enumerate}
\item L'addition et la multiplication sont des lois de composition internes sur \(\N\), sur \(\Z\), sur \(\Q\), sur \(\R\), sur \(\C\), sur \(\R^\N\), sur \(\C^\N\)... \\

\item Si \(E\) est un ensemble, alors : \begin{itemize}
\item L'intersection est une loi de composition interne sur \(\P{E}\). \\

\item La réunion est une loi de composition interne sur \(\P{E}\). \\

\item La composition est une loi de composition interne sur \(\F{E}{E}\). \\
\end{itemize}

\item Le produit vectoriel est une loi de composition interne sur l'ensemble des vecteurs d'un espace euclidien orienté de dimension 3. \\
\end{enumerate}
\end{ex}

\subsection{Associativité, commutativité}

\begin{defi}
Soit \(E\) un ensemble et \(*\) une loi de composition interne sur \(E\).

On dit que \(*\) est associative si on a : \[\quantifs{\forall x,y,z\in E}x*\paren{y*z}=\paren{x*y}*z.\]
\end{defi}

\begin{ex}
Parmi les lois de composition internes données à l'\thref{ex:loisDeCompositionInternes}, toutes sont associatives, excepté le produit vectoriel.

Par exemple, on a bien \[\quantifs{\forall x,y,z\in\N}\paren{x+y}+z=x+\paren{y+z}.\]

En revanche, si l'on considère une base orthonormée directe \(\paren{i,j,k}\), on a : \[\begin{dcases}\paren{i\vecto j}\vecto j=k\vecto j=-i \\ i\vecto\paren{j\vecto j}=i\vecto0=0\end{dcases}\]
\end{ex}

\begin{defi}
Soient \(E\) un ensemble et \(*\) une loi de composition interne sur \(E\).

On dit que \(*\) est commutative si on a : \[\quantifs{\forall x,y\in E}x*y=y*x.\]
\end{defi}

\begin{ex}
Parmi les lois de composition internes données à l'\thref{ex:loisDeCompositionInternes}, toutes sont commutatives, exceptés le produit vectoriel et la composition.
\end{ex}

\subsection{Élément neutre, inverse}

\begin{defi}[Élément neutre]
Soient \(E\) un ensemble, \(*\) une loi de composition interne sur \(E\) et \(e\) un élément de \(E\).

On dit que \(e\) est un élément neutre pour \(*\) si on a : \[\quantifs{\forall x\in E}e*x=x*e=x.\]

On dit alors que la loi \(*\) admet un élément neutre (ou, par abus, que \(E\) admet un élément neutre).
\end{defi}

\begin{ex}
Parmi les lois de composition internes données à l'\thref{ex:loisDeCompositionInternes}, toutes admettent un élément neutre, excepté le produit vectoriel :

\begin{enumerate}
\item \begin{itemize}
\item La loi \(+\) admet \(0\) (ou \(\paren{0}_n\)) comme neutre. \\

\item La loi \(\times\) admet \(1\) (ou \(\paren{1}_n\)) comme neutre. \\
\end{itemize}

\item \begin{itemize}
\item La loi \(\inter\) admet \(E\) comme neutre. \\

\item La loi \(\union\) admet \(\ensvide\) comme neutre. \\

\item La loi \(\rond\) admet \(\id{E}\) comme neutre. \\
\end{itemize}

\item La loi \(\vecto\) n'admet pas de neutre. Par l'absurde, soit \(v\) un élément neutre. On a \(\quantifs{\forall w\text{ vecteur}}w\vecto v=v\vecto w=w\). D'où, en prenant \(w=v\), \(0=v\). D'où \(\quantifs{\forall w\text{ vecteur}}w=w\vecto0=0\) : contradiction. \\
\end{enumerate}
\end{ex}

\begin{rem}
Soient \(E\) un ensemble, \(*\) une loi de composition interne sur \(E\) et \(e\) un élément de \(E\).

On dit que \(e\) est un élément neutre à droite pour \(*\) si on a : \[\quantifs{\forall x\in E}x*e=x.\]

Il n'est généralement pas suffisant que \(e\) soit un élément neutre à droite pour que \(e\) soit un élément neutre (cela suffit si \(*\) est commutative), comme le montre l'exemple suivant :

\begin{center}
\begin{tikzpicture}
\matrix (mymatrix) [matrix of nodes, nodes={draw, minimum size=6mm, outer sep=0pt}, column sep=-\pgflinewidth, row sep=-\pgflinewidth]
    {  & 0 & 1\\
     0 & 0 & 1\\
     1 & 0 & 0\\};
\draw[->, shorten <=1mm, shorten >=1mm, looseness=1.2]
    (mymatrix-2-1.north west)to[out=90, in=180]node[below right=-3pt]{\(\oplus\)}(mymatrix-1-2.north west);
\end{tikzpicture}
\end{center}

\(\oplus\) est bien une loi de composition interne sur \(\accol{0;1}\) et \(0\) est neutre à gauche mais pas à droite.
\end{rem}

\begin{prop}[Unicité de l'élément neutre]
Soient \(E\) un ensemble et \(*\) une loi de composition interne sur \(E\).

Alors il existe au plus un élément neutre pour \(*\).
\end{prop}

\begin{dem}
Soient \(e,e\prim\in E\) deux éléments neutres pour \(*\). On a : \[\quantifs{\forall x\in E}x*e=e*x=x=e\prim*x=x*e\prim.\]

Montrons que \(e=e\prim\).

On a : \[\begin{aligned}
e&=e*e\prim&\text{car }e\prim\text{ neutre} \\
&=e\prim&\text{car }e\text{ neutre}
\end{aligned}\]
\end{dem}

\begin{defprop}[Inverse d'un élément]
Soient \(E\) un ensemble et \(*\) une loi de composition interne sur \(E\).

On suppose que la loi \(*\) est associative et qu'elle admet un élément neutre noté \(e\).

Soit \(x\) un élément de \(E\).

On appelle inverse de \(x\) tout élément \(y\in E\) tel que : \[y*x=x*y=e.\]

Il existe au plus un tel élément \(y\).

S'il existe, on l'appelle donc l'inverse de \(x\) et on le note \(x\inv\).

On dit alors que \(x\) est inversible.
\end{defprop}

\begin{dem}
Soient \(y_1,y_2\in E\) tels que \(\begin{dcases}y_1*x=x*y_1=e \\ y_2*x=x*y_2=e\end{dcases}\)

Montrons que \(y_1=y_2\).

On a \[\begin{aligned}
\paren{y_1*x}*y_2&=y_1*\paren{x*y_2} \\
e*y_2&=y_1*e \\
y_2&=y_1.
\end{aligned}\]
\end{dem}

\begin{ex}\thlabel{ex:inverseElementNeutreEgalElementNeutre}
Soient \(E\) un ensemble et \(*\) une loi de composition interne associative sur \(E\) et admettant un élément neutre \(e\).

Alors \(e\) est inversible, d'inverse lui-même : \[e\inv=e.\]
\end{ex}

\begin{dem}
On a \(e*e=e*e=e\). Donc \(e\) inversible, d'inverse \(e\).
\end{dem}

\begin{ex}\thlabel{ex:inverseInverseEgalElement}
Soient \(E\) un ensemble et \(*\) une loi de composition interne associative sur \(E\) admettant un élément neutre \(e\).

Soit \(a\) un élément inversible de \(E\).

Alors l'inverse \(a\inv\) de \(a\) est inversible, d'inverse \(a\) : \[\paren{a\inv}\inv=a.\]
\end{ex}

\begin{dem}
On a \(a*a\inv=a\inv*a=e\). Donc \(a\inv\) est inversible, d'inverse \(a\).
\end{dem}

\begin{ex}
Étudions les éléments inversibles des lois de composition internes associatives de l'\thref{ex:loisDeCompositionInternes} qui admettent un élément neutre :

\begin{itemize}
\item Pour \(\paren{\N,+}\), le neutre est \(0\) et il est le seul inversible. \\

\item Pour \(\paren{\N,\times}\), le neutre est \(1\) et il est le seul inversible. \\

\item Pour \(\paren{\Z,+}\), le neutre est \(0\) et tout élément admet un opposé. \\

\item Pour \(\paren{\Z,\times}\), le neutre est \(1\) et \(1\) et \(-1\) sont les seuls éléments inversibles. \\

\item Tout élément admet un opposé dans \(\paren{\Q,+}\), \(\paren{\R,+}\), \(\paren{\C,+}\), \(\paren{\R^\N,+}\) et \(\paren{\C^\N,+}\). \\

\item \(0\) n'est pas inversible par \(\times\) dans \(\Q\), \(\R\), \(\C\), \(\R^\N\) et \(\C^\N\). \\

\item Pour \(\paren{\P{E},\inter}\), le neutre est \(E\). Soit \(A\in\P{E}\). On a \[\begin{aligned}
A\text{ inversible par }\inter&\ssi\quantifs{\exists B\in\P{E}}A\inter B=B\inter A=E \\
&\ssi A=E
\end{aligned}\] donc \(E\) est le seul inversible. \\

\item Pour \(\paren{\P{E},\union}\), de même, \(\ensvide\) est le seul inversible. \\

\item Pour \(\paren{\F{E}{E},\rond}\), le neutre est \(\id{E}\). Soit \(f\in\F{E}{E}\). On a \[\begin{aligned}
f\text{ inversible par }\rond&\ssi\quantifs{\exists g\in\F{E}{E}}f\rond g=g\rond f=\id{E} \\
&\ssi f\text{ est bijective}
\end{aligned}\] donc l'inverse de \(f\) est sa bijection réciproque \(f\inv\). \\
\end{itemize}
\end{ex}

\begin{prop}\thlabel{prop:inverseProduitEgalProduitInversesInverse}
Soient \(E\) un ensemble et \(*\) une loi de composition interne associative sur \(E\) admettant un élément neutre \(e\).

Si \(x\) et \(y\) sont inversibles, alors \(x*y\) est aussi inversible, d'inverse : \[\paren{x*y}\inv=y\inv*x\inv.\]
\end{prop}

\begin{dem}
On a : \[\begin{aligned}
\paren{x*y}*\paren{y\inv*x\inv}&=x*\paren{y*y\inv}*x\inv \\
&=x*e*x\inv \\
&=x*x\inv \\
&=e
\end{aligned}\]

De même : \[\begin{aligned}
y\inv*x\inv*x*y&=y\inv*y \\
&=e
\end{aligned}\]

Donc \(x*y\) est inversible, d'inverse \(y\inv*x\inv\).
\end{dem}

\begin{prop}[Inversible \(\imp\) régulier]
Soit \(E\) un ensemble muni d'une loi de composition interne associative \(*\) et admettant un élément neutre \(e\).

Soient \(x,y,z\in E\). On suppose que \(x\) est inversible.

Alors les conditions suivantes sont équivalentes :

\begin{enumerate}
\item \(y=z\) \\

\item \(xy=xz\) \\

\item \(yx=zx\)
\end{enumerate}

On dit que \(x\) est un élément régulier ou simplifiable.
\end{prop}

\begin{dem}
(1) \(\imp\) (2) et (1) \(\imp\) (3) : Clair.

(2) \(\imp\) (1) : on multiplie à gauche par \(x\inv\).

(3) \(\imp\) (1) : on multiplie à droite par \(x\inv\).
\end{dem}

\begin{nota}
Lorsqu'une loi de composition interne est notée \(*\), \(\times\), \(\cdot\), \(\otimes\) ou \(\rond\), on dit qu'on utilise une notation multiplicative.

Au contraire, les notations telles que \(+\) ou \(\oplus\) sont dites additives.

L'usage est de n'employer des notations additives que pour des lois de composition commutatives.

L'usage est également :

\begin{itemize}
\item de réserver les notations \(1\) ou \(1_E\) pour l'élément neutre aux lois notées multiplicativement. Dans le cas d'une loi notée additivement, on préfère noter l'élément neutre \(0\) ou \(0_E\). \\

\item de réserver la notation \(x\inv\) pour l'inverse d'un élément \(x\) aux lois notées multiplicativement. Dans le cas d'une loi notée additivement, on préfère noter \(-x\) l'inverse de \(x\) et on l'appelle alors plutôt l'opposé de \(x\).
\end{itemize}

Soit \(E\) un ensemble muni d'une loi de composition interne associative notée multiplicativement, par exemple \(*\). Soient \(x\in E\) et \(n\in\Ns\).

Le produit \(x*\dots*x\) de \(n\) facteurs tous égaux à \(x\) est noté \(x^n\).

Si, de plus, \(*\) admet un élément neutre \(1_E\), on pose \(x^0=1_E\).

Enfin, si \(x\) est inversible, on pose \(x^{-n}=\paren{x\inv}^n\).

On fait de même pour les lois notées additivement, mais en utilisant encore des notations différentes :

Soit \(E\) un ensemble muni d'une loi de composition interne associative notée additivement, par exemple \(+\). Soient \(x\in E\) et \(n\in\Ns\).

La somme \(x+\dots+x\) de \(n\) termes tous égaux à \(x\) est notée \(nx\) (ou \(n\cdot x\)).

Si, de plus, \(+\) admet un élément neutre \(0_E\), on pose \(0x=0_E\).

Enfin, si \(x\) est \guillemets{inversible}, on pose \(\paren{-n}x=n\paren{-x}\).

Enfin, on utilisera pour une loi additive la notation \(\sum\) pour une somme et pour une loi multiplicative la notation \(\prod\) pour un produit.

Toutes ces différences entre lois notées multiplicativement et lois notées additivement ne reposent que dans la manière de noter les objets. Il n'y a aucune différence dans les concepts.
\end{nota}

\begin{rem}
Soit \(E\) un ensemble muni d'une loi de composition interne associative \(*\).

Soient \(x,y\in E\) et \(n\in\N\).

Attention à ne pas écrire en général \(\paren{xy}^n=x^ny^n\).

Cette formule n'est vraie que si \(x\) et \(y\) commutent.
\end{rem}

\subsection{Distributivité}

\begin{defi}[Distributivité]
Soit \(E\) un ensemble muni de deux lois de composition internes \(\oplus\) et \(\otimes\).

On dit que la loi \(\otimes\) est distributive par rapport à la loi \(\oplus\) si on a : \[\quantifs{\forall x,y,z\in E}\begin{dcases}x\otimes\paren{y\oplus z}=x\otimes y\oplus x\otimes z \\ \paren{y\oplus z}\otimes x=y\otimes x\oplus z\otimes x\end{dcases}\]
\end{defi}

\begin{rem}
Soit \(E\) un ensemble muni de deux lois de composition internes \(\oplus\) et \(\otimes\).

Si la loi \(\otimes\) est commutative, alors : \[\otimes\text{ est distributive par rapport à }\oplus\ssi\quantifs{\forall x,y,z\in E}x\otimes\paren{y\oplus z}=x\otimes y\oplus x\otimes z.\]
\end{rem}

\begin{dem}
\impdir Claire.

\imprec

Soient \(x,y,z\in E\).

On a \[\begin{aligned}
\paren{y\oplus z}\otimes x&=x\otimes\paren{y\oplus z} \\
&=x\otimes y\oplus x\otimes z \\
&=y\otimes x\oplus z\otimes x
\end{aligned}\]
\end{dem}

\begin{ex}
\begin{itemize}
\item Dans \(\C\), \(\times\) est distributif par rapport à \(+\). \\

\item Dans \(\P{E}\) : \(\quantifs{\forall A,B,C\in\P{E}}\begin{dcases}A\union\paren{B\inter C}=\paren{A\union B}\inter\paren{A\union C} \\ A\inter\paren{B\union C}=\paren{A\inter B}\union\paren{A\inter C}\end{dcases}\)

Donc \(\inter\) est distributive par rapport à \(\union\) et \(\union\) est distributive par rapport à \(\inter\). \\

\item Dans \(\N\), \(+\) n'est pas distributive par rapport à \(\times\) : \[1+\paren{2\times3}\not=\paren{1+2}\times\paren{1+3}.\]
\end{itemize}
\end{ex}

\subsection{Parties stables}

\begin{defi}[Partie stable]
Soient \(E\) un ensemble et \(*\) une loi de composition interne sur \(E\).

On dit qu'un partie \(A\subset E\) est stable par la loi \(*\) si on a : \[\quantifs{\forall x,y\in A}x*y\in A.\]

On peut alors définir une loi de composition interne sur \(A\) : \[\fonctionlambda{A\times A}{A}{\paren{x,y}}{x*y}\] qui est appelée la loi de composition interne induite par \(*\) sur \(A\).

Cette loi est souvent encore notée \(*\) (abusivement).
\end{defi}

\begin{ex}
\begin{itemize}
\item \(\Z\) est une partie stable de \(\R\) par \(+\) et \(\times\). \\

\item Soit \(E\prim\in\P{E}\). Alors \(\P{E\prim}\in\P{\P{E}}\) est stable par \(\union\) et \(\inter\). \\

\item Notons \(\mathscr{P}_p\paren{\interventierii{1}{10}}\) les parties de \(\interventierii{1}{10}\) de cardinal pair. Ce n'est une partie stable de \(\P{\interventierii{1}{10}}\) ni pour \(\union\) ni pour \(\inter\). En effet, on a : \[\accol{1;2}\union\accol{1;3}=\accol{1;2;3}\not\in\mathscr{P}_p\paren{\interventierii{1}{10}}\quad\text{et}\quad\accol{1;2}\inter\accol{1;3}=\accol{1}\not\in\mathscr{P}_p\paren{\interventierii{1}{10}}.\]
\end{itemize}
\end{ex}

\section{Groupes}

\subsection{Définition}

\begin{defi}[Groupe]
Un groupe est un couple \(\groupe{G}[*]\) où \(G\) est un ensemble et \(*\) une loi de composition interne sur \(G\) respectant les conditions suivantes :

\begin{enumerate}
\item La loi \(*\) est associative. \\

\item La loi \(*\) admet un élément neutre. \\

\item Tout élément de \(G\) possède un inverse.
\end{enumerate}

On dit aussi que \(G\) est muni d'une structure de groupe, ou, par abus, que \(G\) est un groupe.

Si, de plus, la loi \(*\) est commutative, on dit que le groupe est abélien ou commutatif.
\end{defi}

\begin{ex}[Groupes abéliens]
\[\groupe{\Qs}[\times]\quad\groupe{\Rs}[\times]\quad\groupe{\Cs}[\times]\quad\groupe{\paren{\Rs}^\N}[\times]\quad\groupe{\paren{\Cs}^\N}[\times]\]
\end{ex}

\begin{ex}
Il existe une unique structure de groupe sur \(\accol{0}\) :

\begin{center}
\begin{tikzpicture}
\matrix (mymatrix) [matrix of nodes, nodes={draw, minimum size=6mm, outer sep=0pt}, column sep=-\pgflinewidth, row sep=-\pgflinewidth]
    {  & 0 \\
     0 & 0 \\};
\draw[->, shorten <=1mm, shorten >=1mm, looseness=1.2]
    (mymatrix-2-1.north west)to[out=90, in=180]node[below right=-3pt]{\(+\)}(mymatrix-1-2.north west);
\end{tikzpicture}
\end{center}

En effet, on a \(0+\paren{0+0}=\paren{0+0}+0=0\) donc \(+\) est une loi de groupe. On a aussi \(0\) neutre et \(0\) opposé de \(0\).

Sur \(\accol{0;1}\), il existe deux structures de groupe :

Si \(0\) neutre :

\begin{center}
\begin{tikzpicture}
\matrix (mymatrix) [matrix of nodes, nodes={draw, minimum size=6mm, outer sep=0pt}, column sep=-\pgflinewidth, row sep=-\pgflinewidth]
    {  & 0 & 1\\
     0 & 0 & 1\\
     1 & 1 & 0\\};
\draw[->, shorten <=1mm, shorten >=1mm, looseness=1.2]
    (mymatrix-2-1.north west)to[out=90, in=180]node[below right=-3pt]{\(\oplus\)}(mymatrix-1-2.north west);
\end{tikzpicture}
\end{center}

Si \(1\) neutre :

\begin{center}
\begin{tikzpicture}
\matrix (mymatrix) [matrix of nodes, nodes={draw, minimum size=6mm, outer sep=0pt}, column sep=-\pgflinewidth, row sep=-\pgflinewidth]
    {  & 0 & 1\\
     0 & 1 & 0\\
     1 & 0 & 1\\};
\draw[->, shorten <=1mm, shorten >=1mm, looseness=1.2]
    (mymatrix-2-1.north west)to[out=90, in=180]node[below right=-3pt]{\(\otimes\)}(mymatrix-1-2.north west);
\end{tikzpicture}
\end{center}
\end{ex}

\begin{ex}
Structure de groupe sur \(\accol{0;1;2}\) dont \(0\) est le neutre :

\begin{center}
\begin{tikzpicture}
\matrix (mymatrix) [matrix of nodes, nodes={draw, minimum size=6mm, outer sep=0pt}, column sep=-\pgflinewidth, row sep=-\pgflinewidth]
    {  & 0 & 1 & 2\\
     0 & 0 & 1 & 2\\
     1 & 1 & 2 & 0\\
     2 & 2 & 0 & 1\\};
\draw[->, shorten <=1mm, shorten >=1mm, looseness=1.2]
    (mymatrix-2-1.north west)to[out=90, in=180]node[below right=-3pt]{\(\oplus\)}(mymatrix-1-2.north west);
\end{tikzpicture}
\end{center}

On remarque que cette loi est commutative.
\end{ex}

\begin{ex}[Produit de deux groupes]
Soient \(\groupe{G_1}[*_1]\) et \(\groupe{G_2}[*_2]\) deux groupes.

L'ensemble \(G_1\times G_2\) est naturellement muni d'une structure de groupe, de loi : \[\fonctionlambda{\paren{G_1\times G_2}\times\paren{G_1\times G_2}}{G_1\times G_2}{\paren{\paren{g_1,g_2},\paren{g_1\prim,g_2\prim}}}{\paren{g_1*_1g_1\prim,g_2*_2g_2\prim}}\]

Déterminons son neutre et l'inverse d'un élément \(\paren{g_1,g_2}\) donné.
\end{ex}

\begin{dem}
On note \(*\) la loi.

Montrons que \(*\) est associative :

Soient \(\paren{g_1,g_2},\paren{g_1\prim,g_2\prim},\paren{g_1\seconde,g_2\seconde}\in G_1\times G_2\).

On a : \[\begin{aligned}
\paren{g_1,g_2}*\paren{\paren{g_1\prim,g_2\prim}*\paren{g_1\seconde,g_2\seconde}}&=\paren{g_1,g_2}*\paren{g_1\prim*_1g_1\seconde,g_2\prim*_2g_2\seconde} \\
&=\paren{g_1*_1\paren{g_1\prim*_1g_1\seconde},g_2*_2\paren{g_2\prim*_2g_2\seconde}} \\
&=\paren{\paren{g_1*_1g_1\prim}*_1g_1\seconde,\paren{g_2*_2g_2\prim}*_2g_2\seconde} \\
&=\paren{g_1*_1g_1\prim,g_2*_2g_2\prim}*\paren{g_1\seconde,g_2\seconde} \\
&=\paren{\paren{g_1,g_2}*\paren{g_1\prim,g_2\prim}}*\paren{g_1\seconde,g_2\seconde}
\end{aligned}\]

Montrons que \(*\) admet un neutre :

On note \(e_1\) le neutre de \(G_1\) et \(e_2\) le neutre de \(G_2\).

On remarque : \[\quantifs{\forall\paren{g_1,g_2}\in G_1\times G_2}\begin{dcases}\paren{g_1,g_2}*\paren{e_1,e_2}=\paren{g_1*_1e_1,g_2*_2e_2}=\paren{g_1,g_2} \\ \paren{e_1,e_2}*\paren{g_1,g_2}=\paren{e_1*_1g_1,e_2*_2g_2}=\paren{g_1,g_2}\end{dcases}\]

Donc \(\paren{e_1,e_2}\) est le neutre de \(*\).

Montrons que tout élément admet un inverse :

Soit \(\paren{g_1,g_2}\in G_1\times G_2\).

On remarque : \[\begin{dcases}\paren{g_1,g_2}*\paren{g_1\inv,g_2\inv}=\paren{g_1*_1g_1\inv,g_2*_2g_2\inv}=\paren{e_1,e_2} \\ \paren{g_1\inv,g_2\inv}*\paren{g_1,g_2}=\paren{g_1\inv*_1g_1,g_2\inv*_2g_2}=\paren{e_1,e_2}\end{dcases}\]

Donc \(\paren{g_1,g_2}\) est inversible, d'inverse \(\paren{g_1\inv,g_2\inv}\).
\end{dem}

\begin{rem}
On a : \[G_1\times G_2\text{ abélien}\ssi\begin{dcases}G_1\text{ abélien} \\ G_2\text{ abélien}\end{dcases}\]
\end{rem}

\begin{ex}[Produit de groupes]
Soient \(\groupe{G_1}[*_1],\dots,\groupe{G_n}[*_n]\) des groupes.

Le produit cartésien \(G_1\times\dots\times G_n\) est naturellement muni d'une structure de groupe, de loi : \[\fonctionlambda{\paren{G_1\times\dots\times G_n}\times\paren{G_1\times\dots\times G_n}}{G_1\times\dots\times G_n}{\paren{\paren{g_1,\dots,g_n},\paren{g_1\prim,\dots,g_n\prim}}}{\paren{g_1*_1g_1\prim,\dots,g_n*_ng_n\prim}}\]

Son neutre est \(\paren{e_1,\dots,e_n}\) où \(\quantifs{\forall i\in\interventierii{1}{n}}e_i\text{ neutre de }G_i\).

L'inverse d'un élément \(\paren{g_1,\dots,g_n}\) est \(\paren{g_1,\dots,g_n}\inv=\paren{g_1\inv,\dots,g_n\inv}\).
\end{ex}

\begin{dem}
Idem.
\end{dem}

\begin{ex}\thlabel{ex:fonctionsDeIDansGroupeEstUnGroupe}
Soit \(I\) un ensemble et \(G\) un groupe.

L'ensemble \(\F{I}{G}\) est naturellement muni d'une structure de groupe.

Notons \(\times\) la loi de \(G\) et \(e\) son neutre.

On pose \[\fonction{*}{\F{I}{G}^2}{\F{I}{G}}{\paren{f,g}}{\fonctionlambda{I}{G}{t}{f\paren{t}\times g\paren{t}}}\]

Montrons que \(*\) est associative :

Soient \(f,g,h\in\F{I}{G}\).

On a : \[\begin{WithArrows}
\quantifs{\forall t\in I}f*\paren{g*h}\paren{t}&=f\paren{t}\times\paren{g*h}\paren{t} \\
&=f\paren{t}\times\paren{g\paren{t}\times h\paren{t}}\Arrow{car \(\times\) est associative} \\
&=\paren{f\paren{t}\times g\paren{t}}\times h\paren{t} \\
&=\paren{f*g}\paren{t}\times h\paren{t} \\
&=\paren{f*g}*h\paren{t}
\end{WithArrows}\]

Donc \(*\) est associative.

Montrons que \(*\) admet un élément neutre :

On pose \(\fonction{f_1}{I}{G}{t}{e}\). Montrons que \(f_1\) est l'élément neutre de \(*\).

Soit \(f\in\F{I}{G}\). Montrons que \(f_1*f=f*f_1=f\).

On a \[\quantifs{\forall t\in I}\begin{dcases}\paren{f_1*f}\paren{t}=f_1\paren{t}\times f\paren{t}=e\times f\paren{t}=f\paren{t} \\ \paren{f*f_1}\paren{t}=f\paren{t}\times f_1\paren{t}=f\paren{t}\times e=f\paren{t}\end{dcases}\]

Donc \(f_1\) est l'élément neutre de \(*\).

Montrons que tout élément de \(\F{I}{G}\) est inversible.

Soit \(f\in\F{I}{G}\). On pose \(\fonction{g}{I}{G}{t}{f\paren{t}\inv}\).

Montrons que \(f*g=g*f=f_1\).

On a : \[\quantifs{\forall t\in I}\begin{dcases}\paren{f*g}\paren{t}=f\paren{t}\times g\paren{t}=f\paren{t}\times f\paren{t}\inv=e=f_1\paren{t} \\ \paren{g*f}\paren{t}=g\paren{t}\times f\paren{t}=f\paren{t}\inv\times f\paren{t}=e=f_1\paren{t}\end{dcases}\]

Donc tout élément de \(\F{I}{G}\) est inversible, d'inverse \(g\).
\end{ex}

\begin{rem}
On a : \[\F{I}{G}\text{ abélien}\ssi\orenv{G\text{ abélien} \\ I=\ensvide}\]
\end{rem}

\subsection{Sous-groupes}

\begin{defi}[Sous-groupe]
Soit \(\groupe{G}[*]\) un groupe. On note \(e\) son élément neutre.

Un sous-groupe de \(\groupe{G}[*]\) (ou, par abus, de \(G\)) est une partie \(H\) de \(G\) telle que :

\begin{enumerate}
\item L'élément neutre \(e\) appartient à \(H\). \\

\item La partie \(H\) est stable par produit : \(\quantifs{\forall h_1,h_2\in H}h_1*h_2\in H\). \\

\item La partie \(H\) est stable par passage à l'inverse : \(\quantifs{\forall h\in H}h\inv\in H\).
\end{enumerate}
\end{defi}

\begin{prop}
Tout sous-groupe d'un groupe \(G\) est naturellement un groupe. Sa loi est la loi induite par celle de \(G\).
\end{prop}

\begin{dem}
On note \(*\) la loi de \(G\).

Soit \(H\) un sous-groupe de \(G\).

Comme \(H\) est stable par \(*\), on peut définir : \(\fonction{*_H}{H^2}{H}{\paren{h_1,h_2}}{h_1*h_2}\)

Montrons que \(*_H\) est associative :

On a bien \[\quantifs{\forall h_1,h_2,h_3\in H}h_1*_H\paren{h_2*_Hh_3}=h_1*\paren{h_2*h_3}=\paren{h_1*h_2}*h_3=\paren{h_1*_Hh_2}*_Hh_3.\]

Montrons que \(*_H\) admet un élément neutre :

On a \(e\in H\) et \[\quantifs{\forall h\in H}\begin{dcases}e*_Hh=e*h=h \\ h*_He=h*e=h\end{dcases}\] donc \(e\) est le neutre de \(*_H\).

Montrons que tout élément de \(H\) est inversible par \(*_H\) :

Soit \(h\in H\). On note \(h\inv\) son inverse dans \(G\).

On a \[\begin{dcases}h\inv\in H \\ h*_Hh\inv=h\inv*_Hh=e\end{dcases}\]

Donc \(\groupe{H}[*_H]\) est un groupe.
\end{dem}

\begin{rem}
On a \[H\text{ abélien}\impr G\text{ abélien}\]
\end{rem}

\begin{rem}
En pratique, on note également \(*\) la loi \(*_H\) de \(H\).
\end{rem}

\begin{prop}
Soit \(\groupe{G}[*]\) un groupe. On note \(e\) son élément neutre.

Soit \(H\) une partie de \(G\).

Alors \(H\) est un sous-groupe de \(G\) si, et seulement si, les conditions suivantes sont vérifiées :

\begin{enumerate}\setcounter{enumi}{3}
\item \(e\in H\) \\

\item \(\quantifs{\forall h_1,h_2\in H}h_1*h_2\inv\in H\) \\
\end{enumerate}
\end{prop}

\begin{dem}
\impdir

Supposons \(H\) sous-groupe de \(G\).

On a clairement (4) selon (1).

Montrons (5).

Soient \(h_1,h_2\in H\).

On a \(h_2\inv\in H\) selon (3) donc \(h_1*h_2\inv\in H\) selon (2).

\imprec

Supposons (4) et (5).

On a (1) selon (4).

Montrons (3).

Soit \(h\in H\).

On a \(e\in H\) selon (4) donc selon (5), \(e*h\inv\in H\) donc \(h\inv\in H\).

Montrons (2).

Soient \(h_1,h_2\in H\).

On a \(h_2\inv\in H\) selon (3) donc \(h_1*\paren{h_2\inv}\inv\in H\) donc \(h_1*h_2\in H\).
\end{dem}

\begin{ex}
Soit \(n\in\Ns\).

L'ensemble \(\U\) des nombres complexes de module \(1\) est un sous-groupe de \(\Cs\).

L'ensemble \(\U_n\) des racines \(n\)-ièmes de l'unité est un sous-groupe de \(\U\).
\end{ex}

\begin{dem}
On sait que \(\groupe{\Cs}[\times]\) est un groupe. Montrons que \(\U\) est un sous-groupe de \(\Cs\).

On a \(1\in\U\) et \(\quantifs{\forall z_1,z_2\in\U}z_1\times z_2\inv\in\U\) (car \(\abs{z_1\times z_2\inv}=\dfrac{\abs{z_1}}{\abs{z_2}}=\dfrac{1}{1}=1\)).

Donc \(\U\) est un sous-groupe de \(\Cs\). Donc \(\U\) est un groupe.

Montrons que \(\U_n\) est un sous-groupe de \(\U\).

On a \(\U_n\subset\U\) et \(1\in\U_n\) et \(\quantifs{\forall z_1,z_2\in\U_n}z_1\times z_2\inv\in\U_n\) car \(\paren{\dfrac{z_1}{z_2}}^n=\dfrac{z_1^n}{z_2^n}=\dfrac{1}{1}=1\).

Donc \(\U_n\) est un sous-groupe de \(\U\). Donc c'est un groupe.
\end{dem}

\subsection{Morphismes de groupe}

\begin{defi}[Morphisme de groupes]
Soient \(\groupe{G_1}[*_1]\) et \(\groupe{G_2}[*_2]\) deux groupes.

On appelle morphisme de groupes de \(\groupe{G_1}[*_1]\) vers \(\groupe{G_2}[*_2]\) (ou, par abus, de \(G_1\) vers \(G_2\)) toute application \(\phi:G_1\to G_2\) telle que : \[\quantifs{\forall g,g\prim\in G_1}\phi\paren{g*_1g\prim}=\phi\paren{g}*_2\phi\paren{g\prim}.\]

L'ensemble des morphismes de groupes de \(G_1\) vers \(G_2\) est noté : \[\Hom{G_1}{G_2}.\]
\end{defi}

\begin{prop}\thlabel{prop:neutreEtInverseParUnMorphismeDeGroupes}
Soient \(\groupe{G_1}[*_1]\) et \(\groupe{G_2}[*_2]\) deux groupes et \(\phi:G_1\to G_2\) un morphisme de groupes de \(\groupe{G_1}[*_1]\) vers \(\groupe{G_2}[*_2]\). On note \(e_1\) et \(e_2\) les éléments neutres respectifs de \(G_1\) et \(G_2\).

On a alors : \[\phi\paren{e_1}=e_2\] et : \[\quantifs{\forall g\in G}\phi\paren{g\inv}=\phi\paren{g}\inv.\]
\end{prop}

\begin{dem}
On a \(e_1*_1e_1=e_1\) donc : \[\begin{aligned}
\phi\paren{e_1*_1e_1}&=\phi\paren{e_1} \\
\phi\paren{e_1}*_2\phi\paren{e_1}&=\phi\paren{e_1} \\
\phi\paren{e_1}\inv*_2\phi\paren{e_1}*_2\phi\paren{e_1}&=\phi\paren{e_1}\inv*_2\phi\paren{e_1} \\
e_2*_2\phi\paren{e_1}&=e_2 \\
\phi\paren{e_1}&=e_2
\end{aligned}\]

Soit \(g\in G_1\).

On a \(g*_1g\inv=e_1\) donc : \[\begin{aligned}
\phi\paren{g*_1g\inv}&=\phi\paren{e_1} \\
\phi\paren{g}*_2\phi\paren{g\inv}&=e_2 \\
\phi\paren{g}\inv*_2\phi\paren{g}*_2\phi\paren{g\inv}&=\phi\paren{g}\inv*_2e_2 \\
\phi\paren{g\inv}&=\phi\paren{g}\inv
\end{aligned}\]
\end{dem}

\begin{prop}\thlabel{prop:imagesDirectesEtReciproquesMorphismeSousGroupes}
Soient \(\groupe{G_1}[*_1]\) et \(\groupe{G_2}[*_2]\) deux groupes et \(\phi:G_1\to G_2\) un morphisme de groupes de \(\groupe{G_1}[*_1]\) vers \(\groupe{G_2}[*_2]\).

Soient \(H_1\) un sous-groupe de \(G_1\) et \(H_2\) un sous-groupe de \(G_2\).

Alors :

\begin{enumerate}
\item L'image directe \(\phi\paren{H_1}\) est un sous-groupe de \(G_2\). \\

\item L'image réciproque \(\phi\inv\paren{H_2}\) est un sous-groupe de \(G_1\).
\end{enumerate}
\end{prop}

\begin{dem}[1]
Montrons que \(\phi\paren{H_1}\) est un sous-groupe de \(G_2\).

On a \(\phi\paren{H_1}\subset G_2\).

On note \(e_1\) l'élément neutre de \(G_1\) et \(e_2\) l'élément neutre de \(G_2\).

On a \(e_1\in H_1\) donc \(\phi\paren{e_1}\in\phi\paren{H_1}\) donc \(e_2\in\phi\paren{H_1}\).

Montrons que \(\quantifs{\forall y_1,y_2\in\phi\paren{H_1}}y_1*_2y_2\inv\in\phi\paren{H_1}\).

Soient \(y_1,y_2\in\phi\paren{H_1}\). Soient \(x_1,x_2\in H_1\) tels que \(\begin{dcases}\phi\paren{x_1}=y_1 \\ \phi\paren{x_2}=y_2\end{dcases}\)

On a \(x_1*_1x_2\inv\in H_1\) donc \(\phi\paren{x_1*_1x_2\inv}\in\phi\paren{H_1}\).

Or \(\phi\paren{x_1*_1x_2\inv}=\phi\paren{x_1}*_2\phi\paren{x_2\inv}=\phi\paren{x_1}*_2\phi\paren{x_2}\inv=y_1*_2y_2\inv\).

Donc \(y_1*_2y_2\inv\in\phi\paren{H_1}\).

Donc \(\phi\paren{H_1}\) est un sous-groupe de \(G_2\).
\end{dem}

\begin{dem}[2]
Montrons que \(\phi\inv\paren{H_2}\) est un sous-groupe de \(G_1\).

On a \(\phi\inv\paren{H_2}\subset G_1\).

On note \(e_1\) l'élément neutre de \(G_1\) et \(e_2\) l'élément neutre de \(G_2\).

On a \(e_1\in\phi\inv\paren{H_2}\) car \(\phi\paren{e_1}=e_2\in H_2\).

Enfin, on a \(\quantifs{\forall x_1,x_2\in\phi\inv\paren{H_2}}\phi\paren{x_1*_1x_2\inv}=\phi\paren{x_1}*_2\phi\paren{x_2\inv}=\phi\paren{x_1}*_2\phi\paren{x_2}\inv\in H_2\) car \(H_2\) est un groupe.

Donc \(\quantifs{\forall x_1,x_2\in\phi\inv\paren{H_2}}x_1*_1x_2\inv\in\phi\inv\paren{H_2}\).

Donc \(\phi\inv\paren{H_2}\) est un sous-groupe de \(G_1\).
\end{dem}

\begin{prop}
Soit \(G\) un groupe et \(H\) un sous-groupe de \(G\).

L'application \[\fonction{i}{H}{G}{h}{h}\] est un morphisme de groupes.
\end{prop}

\begin{dem}
On note \(*\) la loi de \(G\) et \(*_H\) la loi de \(H\) (induite).

On remarque : \[\quantifs{\forall h_1,h_2\in H}i\paren{h_1*_Hh_2}=h_1*_Hh_2=h_1*h_2.\]
\end{dem}

\begin{ex}
Soit \(I\) un ensemble, \(J\) une partie de \(I\) et \(G\) un groupe.

Alors l'application de restriction \[\fonctionlambda{\F{I}{G}}{\F{J}{G}}{f}{\restr{f}{J}}\] est un morphisme de groupes.
\end{ex}

\begin{ex}
La fonction \(\fonction{\phi}{\R}{\U}{\theta}{\e{\i\theta}}\) est un morphisme de groupes de \(\groupe{\R}\) vers \(\groupe{\U}[\times]\).
\end{ex}

\begin{dem}
On a : \[\quantifs{\forall\theta_1,\theta_2\in\R}\phi\paren{\theta_1+\theta_2}=\e{\i\paren{\theta_1+\theta_2}}=\e{\i\theta_1}\e{\i\theta_2}=\phi\paren{\theta_1}\phi\paren{\theta_2}.\]
\end{dem}

\begin{prop}[Composition de morphismes de groupes]\thlabel{prop:composeeMorphismesEstUnMorphisme}
Soient \(G_1\), \(G_2\) et \(G_3\) des groupes et \(\phi:G_1\to G_2\) et \(\psi:G_2\to G_3\) des morphismes de groupes.

Alors \[\psi\rond\phi:G_1\to G_3\] est un morphisme de groupes.
\end{prop}

\begin{dem}
On note \(*_1\) la loi de \(G_1\), \(*_2\) la loi de \(G_2\) et \(*_3\) la loi de \(G_3\).

On a : \[\begin{aligned}
\quantifs{\forall g,g\prim\in G_1}\psi\rond\phi\paren{g*_1g\prim}&=\psi\paren{\phi\paren{g*_1g\prim}} \\
&=\psi\paren{\phi\paren{g}*_2\phi\paren{g\prim}} \\
&=\psi\paren{\phi\paren{g}}*_3\psi\paren{\phi\paren{g\prim}} \\
&=\psi\rond\phi\paren{g}*_3\psi\rond\phi\paren{g\prim}
\end{aligned}\]

Donc \(\psi\rond\phi\) est un morphisme de groupes.
\end{dem}

\begin{defi}[Noyau et image d'un morphisme]
Soient \(G_1\) et \(G_2\) deux groupes et \(\phi:G_1\to G_2\) un morphisme de groupes.

On note \(e_2\) le neutre de \(G_2\).

L'image du morphisme \(\phi\) est l'ensemble image de l'application \(\phi\) : \[\Im\phi=\accol{\phi\paren{g_1}}_{g_1\in G_1}=\accol{g_2\in G_2\tq\quantifs{\exists g_1\in G_1}\phi\paren{g_1}=g_2}.\]

Le noyau du morphisme \(\phi\) est l'ensemble des éléments de \(G_1\) dont l'image par \(\phi\) est le neutre de \(G_2\) : \[\ker\phi=\phi\inv\paren{\accol{e_2}}=\accol{g_1\in G_1\tq\phi\paren{g_1}=e_2}.\]
\end{defi}

\begin{prop}
Soient \(G_1\) et \(G_2\) deux groupes et \(\phi:G_1\to G_2\) un morphisme de groupes.

Alors \(\ker\phi\) est un sous-groupe de \(G_1\) et \(\Im\phi\) est un sous-groupe de \(G_2\).
\end{prop}

\begin{dem}
On a \(\Im\phi=\phi\paren{G_1}\) donc \(\Im\phi\) est l'image directe de \(G_2\) par \(\phi\).

Donc \(\Im\phi\) est un sous-groupe de \(G_2\) selon la \thref{prop:imagesDirectesEtReciproquesMorphismeSousGroupes} car \(\phi\) est un morphisme de groupes et \(G_1\) est un sous-groupe de \(G_1\).

\(\ker\phi\) est l'image réciproque de \(\accol{e_2}\) par \(\phi\).

Donc \(\ker\phi\) est un sous-groupe de \(G_1\) selon la \thref{prop:imagesDirectesEtReciproquesMorphismeSousGroupes} car \(\phi\) est un morphisme de groupes et \(\accol{e_2}\) est un sous-groupe de \(G_2\).
\end{dem}

\begin{theo}\thlabel{theo:morphismeDeGroupeInjectifSsiNoyauNul}
Soit \(\phi:G_1\to G_2\) un morphisme de groupes.

On note \(e_1\) le neutre de \(G_1\).

Alors \(\phi\) est une application injective si, et seulement si, son noyau est \guillemets{nul}, \cad : \[\ker\phi=\accol{e_1}.\]
\end{theo}

\begin{dem}
On note \(e_2\) le neutre de \(G_2\).

\impdir

Supposons \(\phi\) injectif.

Montrons que \(\ker\phi=\accol{e_1}\).

\increc On a \(\phi\paren{e_1}=e_2\) donc \(e_1\in\phi\inv\paren{\accol{e_2}}\) donc \(\accol{e_1}\subset\ker\phi\).

\incdir Soit \(x\in\ker\phi\). On a \(\phi\paren{x}=e_2=\phi\paren{e_1}\). Donc \(x=e_1\) car \(\phi\) est injectif. Donc \(\ker\phi\subset\accol{e_1}\).

Finalement, on a \(\ker\phi=\accol{e_1}\).

\imprec

Supposons \(\ker\phi=\accol{e_1}\).

Montrons que \(\phi\) est injectif.

Soient \(x,y\in G_1\) tels que \(\phi\paren{x}=\phi\paren{y}\). Montrons que \(x=y\).

On a : \[\begin{aligned}
\phi\paren{x}&=\phi\paren{y} \\
\phi\paren{x}\phi\paren{y}\inv&=e_2 \\
\phi\paren{x}\phi\paren{y\inv}&=e_2 \\
\phi\paren{xy\inv}&=e_2
\end{aligned}\]

Donc \(xy\inv\in\ker\phi\) donc \(xy\inv=e_1\) car \(\ker\phi=\accol{e_1}\).

Donc \(x=y\).

Donc \(\phi\) est injectif.
\end{dem}

\subsection{Isomorphismes, endomorphismes, automorphismes}

\begin{defi}[Isomorphisme]
Soient \(G_1\) et \(G_2\) deux groupes.

Un isomorphisme de groupes de \(G_1\) vers \(G_2\) est un morphisme de groupes \(\phi:G_1\to G_2\) bijectif.
\end{defi}

\begin{theo}
Soient \(G_1\), \(G_2\) et \(G_3\) des groupes.

Soient \(\phi:G_1\to G_2\) et \(\phi:G_2\to G_3\) deux isomorphismes de groupes.

Alors \(\psi\rond\phi:G_1\to G_3\) est un isomorphisme de groupes.
\end{theo}

\begin{dem}
Comme \(\psi\) et \(\phi\) sont des morphismes de groupes, d'après la \thref{prop:composeeMorphismesEstUnMorphisme}, \(\psi\rond\phi\) est un morphisme de groupes.

Comme \(\psi\) et \(\phi\) sont des bijections, d'après la \thref{prop:composeeBijectionsEstUneBijection}, \(\psi\rond\phi\) est une bijection.

Donc \(\psi\rond\phi\) est un isomorphisme de groupes.
\end{dem}

\begin{theo}
Soit \(\phi:G_1\to G_2\) un isomorphisme de groupes.

Alors la bijection réciproque \(\phi\inv:G_2\to G_1\) est un isomorphisme de groupes.
\end{theo}

\begin{dem}
On note \(*_1\) la loi de \(G_1\) et \(*_2\) la loi de \(G_2\).

On sait déjà que \(\phi\inv\) est une bijection (\thref{defprop:bijRec}).

Montrons que \(\phi\inv:G_2\to G_1\) est un morphisme de groupes.

Soient \(y,y\prim\in G_2\).

On a \(\phi\rond\phi\inv\paren{y}=y\) et \(\phi\rond\phi\inv\paren{y\prim}=y\prim\).

Donc, comme \(\phi\) est un morphisme de groupes, on a \(\phi\paren{\phi\inv\paren{y}*_1\phi\inv\paren{y\prim}}=y*_2y\prim\).

Donc \(\phi\inv\paren{y}*_1\phi\inv\paren{y\prim}\) est l'antécédent de \(y*_2y\prim\) par \(\phi\), \cad \(\phi\inv\paren{y}*_1\phi\inv\paren{y\prim}=\phi\inv\paren{y*_2y\prim}\).

Donc \(\phi\inv\) est un morphisme de groupes.

Donc \(\phi\inv\) est un isomorphisme de groupes.
\end{dem}

\begin{defi}[Endomorphisme]
Soit \(G\) un groupe.

Un endomorphisme de groupe de \(G\) est un morphisme de groupes \(\phi:G\to G\).
\end{defi}

\begin{defi}[Automorphisme]
Soit \(G\) un groupe.

Un automorphisme de groupe de \(G\) est un isomorphisme de groupes \(\phi:G\to G\), \cad un endomorphisme de \(G\) bijectif.

L'ensemble des automorphismes de groupe de \(G\) est noté \(\Aut{G}\).
\end{defi}

\begin{ex}
Soit \(G\) un groupe.

Alors \(\groupe{\Aut{G}}[\rond]\) est un groupe appelé le groupe des automorphismes de \(G\).
\end{ex}

\begin{dem}
\note{EXERCICE}
\end{dem}

\section{Anneaux}

\subsection{Définition}

\begin{defi}[Anneau]
Un anneau est un triplet \(\anneau{A}\) où \(A\) est un ensemble et \(+\) et \(\times\) sont deux lois de composition internes sur \(A\) tels que les conditions suivantes soient vérifiées :

\begin{enumerate}
\item \(\groupe{A}\) est un groupe commutatif. \\

\item La loi \(\times\) est associative et admet un élément neutre. \\

\item La loi \(\times\) est distributive par rapport à la loi \(+\), \cad : \[\quantifs{\forall a,b,c\in A}a\times\paren{b+c}=a\times b+a\times c\quad\text{et}\quad\paren{b+c}\times a=b\times a+c\times a.\]
\end{enumerate}

On dit aussi que \(A\) est muni d'une structure d'anneau, ou, par abus, que \(A\) est un anneau.

Si, de plus, la loi \(\times\) est commutative, on dit que \(A\) est un anneau commutatif.
\end{defi}

\begin{rem}
La loi \(+\) est appelée l'addition de l'anneau.

La loi \(\times\) est appelée sa multiplication. On s'autorise à ne pas écrire son symbole (\cad écrire \(ab\) au lieu de \(a\times b\)).

L'élément neutre de \(+\) est généralement noté \(0\) ou \(0_A\) ; celui de \(\times\) est noté \(1\) ou \(1_A\).

Si \(a\) est un élément de \(A\), son \guillemets{élément inverse} pour \(+\) est appelé son opposé et est noté \(-a\) (il existe car \(\groupe{A}\) est un groupe commutatif) ; son élément inverse pour \(\times\) est appelé son inverse et est noté \(a\inv\) (s'il existe.)
\end{rem}

\begin{ex}
On vérifie facilement que \(\anneau{\Z}\), \(\anneau{\Q}\), \(\anneau{\R}\), \(\anneau{\C}\), \(\anneau{\R^\N}\) et \(\anneau{\C^\N}\) sont des anneaux commutatifs.
\end{ex}

\begin{ex}\thlabel{ex:produitD'anneauxEstUnAnneau}
Soient \(\anneau{A_1}[+_1][\times_1]\) et \(\anneau{A_2}[+_2][\times_2]\) deux anneaux de neutres \(0_1,1_1\) et \(0_2,1_2\) respectivement.

L'ensemble \(A_1\times A_2\) est naturellement muni d'une structure d'anneau, de lois \[\fonctionlambda{\paren{A_1\times A_2}\times\paren{A_1\times A_2}}{A_1\times A_2}{\paren{\paren{a_1,a_2},\paren{a_1\prim,a_2\prim}}}{\paren{a_1+_1a_1\prim,a_2+_2a_2\prim}}\] et \[\fonctionlambda{\paren{A_1\times A_2}\times\paren{A_1\times A_2}}{A_1\times A_2}{\paren{\paren{a_1,a_2},\paren{a_1\prim,a_2\prim}}}{\paren{a_1\times_1a_1\prim,a_2\times_2a_2\prim}}\]

Ses éléments neutres sont \(\paren{0_1,0_2}\) et \(\paren{1_1,1_2}\).
\end{ex}

\begin{ex}\thlabel{ex:fonctionsD'unEnsembleDansUnAnneauEstUnAnneau}
Soient \(I\) un ensemble et \(A\) un anneau.

\(\F{I}{A}\) est naturellement muni d'une structure d'anneau.

On note \(\begin{dcases}+\text{ et }\times\text{ les lois de }A \\ 0\text{ et }1\text{ les lois de }A\end{dcases}\)

Comme \(\groupe{A}\) est un groupe commutatif, on sait que \(\F{I}{A}\) est un groupe commutatif pour la loi \[\fonction{\oplus}{\F{I}{A}^2}{\F{I}{A}}{\paren{f,g}}{\fonctionlambda{I}{A}{x}{f\paren{x}+g\paren{x}}}\] dont le neutre est \(\fonctionlambda{I}{A}{x}{0}\) (\cf \thref{ex:fonctionsDeIDansGroupeEstUnGroupe}).

On pose d'autre part \[\fonction{\otimes}{\F{I}{A}^2}{\F{I}{A}}{\paren{f,g}}{\fonctionlambda{I}{A}{x}{f\paren{x}\times g\paren{x}}}\]

On vérifie facilement que \(\otimes\) est associative et qu'elle admet pour neutre \(\fonctionlambda{I}{A}{x}{1}\).

Vérifions maintenant que \(\otimes\) est distributive par rapport à \(\oplus\).

Soient \(f,g,h\in\F{I}{A}\).

Montrons que \(\begin{dcases}f\otimes\paren{g\oplus h}=f\otimes g\oplus f\otimes h &\text{(1)} \\ \paren{g\oplus h}\otimes f=g\otimes f\oplus h\otimes f &\text{(2)}\end{dcases}\)

Soit \(x\in I\).

On a : \[\begin{WithArrows}
\croch{f\otimes\paren{g\oplus h}}\paren{x}&=f\paren{x}\times\paren{g\oplus h}\paren{x} \Arrow{par définition de \(\oplus\)} \\
&=f\paren{x}\times\paren{g\paren{x}+h\paren{x}} \Arrow{car \(A\) est un anneau} \\
&=f\paren{x}\times g\paren{x}+f\paren{x}\times h\paren{x} \Arrow{par définition de \(\otimes\)} \\
&=\croch{f\otimes g}\paren{x}+\croch{f\otimes h}\paren{x} \Arrow{par définition de \(\oplus\)} \\
&=\croch{\paren{f\otimes g}\oplus\paren{f\otimes h}}\paren{x}
\end{WithArrows}\]

D'où (1).

On montre de même (2).

Donc \(\otimes\) est distributive par rapport à \(\oplus\).

Donc \(\anneau{\F{I}{A}}\) est un anneau.
\end{ex}

\subsection{Calculs dans un anneau}

\begin{prop}
Soit \(\anneau{A}\) un anneau.

On note \(0\) son neutre pour \(+\).

On a, pour tout élément \(a\) de \(A\) : \[0\times a=a\times 0=0.\]

On dit que l'élément \(0\) est absorbant pour le produit.
\end{prop}

\begin{dem}
Soit \(a\in A\).

On a : \[\begin{WithArrows}
1+0&=1 \Arrow{on multiplie à gauche par \(a\)} \\
a\paren{1+0}&=a\times1 \Arrow{car \(\times\) est distributive par rapport à \(+\)} \\
a\times1+a\times0&=a\times1 \\
a+a\times0&=a \Arrow{on ajoute \(-a\) de chaque côté} \\
a\times0&=0
\end{WithArrows}\]
\end{dem}

\begin{prop}
Soit \(\anneau{A}\) un anneau.

On a : \[\quantifs{\forall a,b\in A}-\paren{a\times b}=\paren{-a}\times b=a\times\paren{-b}\quad\text{et}\quad\paren{-a}\times\paren{-b}=ab.\]
\end{prop}

\begin{dem}
Soient \(a,b\in A\).

On a d'une part \(a\paren{b-b}=a\times0=0\) et d'autre part \(a\paren{b-b}=ab+a\paren{-b}\).

Ainsi, on a \(ab+a\paren{-b}=0\).

D'où, en ajoutant l'opposé de \(ab\) de chaque côté : \(a\paren{-b}=-ab\).

De même, on a : \(\paren{-a}b+ab=\paren{-a+a}b=0\times b=0\) donc \(\paren{-a}b=-ab\).

Enfin, selon ce qui précède : \(\paren{-a}\paren{-b}=-a\paren{-b}=-\paren{-ab}=ab\).
\end{dem}

\begin{defi}[Anneau nul]
Un anneau \(\anneau{A}\) est dit nul si \(A\) est un singleton.
\end{defi}

\begin{prop}
Un anneau \(\anneau{A}\) d'éléments neutres \(0\) et \(1\) est nul si, et seulement si, on a : \(0=1\).
\end{prop}

\begin{dem}
\impdir Si \(A\) est nul alors \(0=1\).

\imprec

Supposons \(0=1\). Montrons que \(A\) est nul.

On a \(\quantifs{\forall a\in A}a=a\times1=a\times0=0\).

Donc \(\Card A=1\).
\end{dem}

\begin{prop}
Soit \(n\in\Ns\). Soient \(A\) un anneau et \(a\) et \(b\) deux éléments de \(A\) tels que : \[ab=ba\] (on dit que les éléments \(a\) et \(b\) commutent).

On a :

\begin{enumerate}
\item La formule du binôme de Newton : \[\paren{a+b}^n=\sum_{k=0}^{n}\binom{k}{n}a^kb^{n-k}\]

\item La factorisation de \(a^n-b^n\) : \[a^n-b^n=\paren{a-b}\sum_{k=0}^{n-1}a^kb^{n-k-1}\]
\end{enumerate}
\end{prop}

\begin{dem}[1]
Pour tout \(n\in\N\) on pose \(\P{n}:\paren{a+b}^n=\sum_{k=0}^{n}\binom{k}{n}a^kb^{n-k}\).

Si \(n=0\), on a : \(\paren{a+b}^0=1\) et \(\sum_{k=0}^{0}=\binom{k}{0}a^kb^{0-k}=\binom{0}{0}a^0b^0=1\) donc \(\P{0}\) est vraie.

Soit \(n\in\N\) tel que \(\P{n}\). Montrons \(\P{n+1}\) :

On a : \[\begin{WithArrows}
\paren{a+b}^{n+1}&=\paren{a+b}\paren{a+b}^n \Arrow{selon \(\P{n}\)} \\
&=\paren{a+b}\sum_{k=0}^{n}\binom{k}{n}a^kb^{n-k} \\
&=\sum_{k=0}^{n}\binom{k}{n}a^{k+1}b^{n-k}+\sum_{k=0}^{n}\binom{k}{n}ba^kb^{n-k} \Arrow{car \(ba=ab\)} \\
&=\sum_{k=1}^{n+1}\binom{k-1}{n}a^kb^{n-k+1}+\sum_{k=0}^{n}\binom{k}{n}a^kb^{n-k+1} \Arrow{car \(\binom{-1}{n}=\binom{n+1}{n}=0\)} \\
&=\sum_{k=1}^{n+1}\binom{k-1}{n}a^kb^{n-k+1}+\sum_{k=0}^{n+1}\binom{k}{n}a^kb^{n-k+1} \\
&=\sum_{k=0}^{n+1}\paren{\binom{k-1}{n}+\binom{k}{n}}a^kb^{n-k+1} \\
&=\sum_{k=0}^{n+1}\binom{k}{n+1}a^kb^{n-k+1}
\end{WithArrows}\]

D'où \(\P{n+1}\).

Donc pour tout \(n\in\N\), on a \(\P{n}\) par récurrence.
\end{dem}

\begin{dem}[2]
On a : \[\begin{WithArrows}
\paren{a-b}\sum_{k=0}^{n-1}a^kb^{n-k-1}&=\sum_{k=0}^{n-1}\paren{a^{k+1}b^{n-k-1}-ba^kb^{n-k-1}} \Arrow{car \(ba=ab\)} \\
&=\sum_{k=0}^{n-1}\paren{a^{k+1}b^{n-k-1}-a^kb^{n-k}} \Arrow{somme téléscopique} \\
&=a^nb^0-a^0b^n \\
&=a^n-b^n
\end{WithArrows}\]
\end{dem}

\begin{ex}
Déterminons toutes les structures d'anneau sur \(\accol{0}\), puis sur \(\accol{0;1}\).

L'unique structure d'anneau sur \(\accol{0}\) est la suivante :

\begin{center}
\begin{tikzpicture}
\matrix (mymatrix) [matrix of nodes, nodes={draw, minimum size=6mm, outer sep=0pt}, column sep=-\pgflinewidth, row sep=-\pgflinewidth]
    {  & 0\\
     0 & 0\\};
\draw[->, shorten <=1mm, shorten >=1mm, looseness=1.2]
    (mymatrix-2-1.north west)to[out=90, in=180]node[below right=-3pt]{\(+\)}(mymatrix-1-2.north west);
\end{tikzpicture}
\hspace{2cm}
\begin{tikzpicture}
\matrix (mymatrix) [matrix of nodes, nodes={draw, minimum size=6mm, outer sep=0pt}, column sep=-\pgflinewidth, row sep=-\pgflinewidth]
    {  & 0\\
     0 & 0\\};
\draw[->, shorten <=1mm, shorten >=1mm, looseness=1.2]
    (mymatrix-2-1.north west)to[out=90, in=180]node[below right=-3pt]{\(\times\)}(mymatrix-1-2.north west);
\end{tikzpicture}
\end{center}

On obtient bien une structure d'anneau :

\begin{itemize}
\item \(\groupe{\accol{0}}\) est bien un groupe commutatif (le groupe nul). \\

\item \(\times\) est associative et possède un neutre. \\

\item \(\times\) est bien distributive par rapport à \(+\) car \[\quantifs{\forall a,b,c\in\accol{0}}a\paren{b+c}=0=0+0=ab+ac.\]
\end{itemize}

Déterminons maintenant les structures d'anneau sur \(A=\accol{0;1}\).

\analyse

Comme \(\Card A=2\), les lois \(+\) et \(\times\) n'admettent pas le même élément neutre.

Si le neutre de \(+\) est \(0\) et celui de \(\times\) est \(1\) :

\begin{center}
\begin{tikzpicture}
\matrix (mymatrix) [matrix of nodes, nodes={draw, minimum size=6mm, outer sep=0pt}, column sep=-\pgflinewidth, row sep=-\pgflinewidth]
    {  & 0 & 1\\
     0 & 0 & 1\\
	 1 & 1 & 0\\};
\draw[->, shorten <=1mm, shorten >=1mm, looseness=1.2]
    (mymatrix-2-1.north west)to[out=90, in=180]node[below right=-3pt]{\(+\)}(mymatrix-1-2.north west);
\end{tikzpicture}
\hspace{2cm}
\begin{tikzpicture}
\matrix (mymatrix) [matrix of nodes, nodes={draw, minimum size=6mm, outer sep=0pt}, column sep=-\pgflinewidth, row sep=-\pgflinewidth]
    {  & 0 & 1\\
     0 & 0 & 0\\
	 1 & 0 & 1\\};
\draw[->, shorten <=1mm, shorten >=1mm, looseness=1.2]
    (mymatrix-2-1.north west)to[out=90, in=180]node[below right=-3pt]{\(\times\)}(mymatrix-1-2.north west);
\end{tikzpicture}
\end{center}

Si le neutre de \(+\) est \(1\) et celui de \(\times\) est \(0\) :

\begin{center}
\begin{tikzpicture}
\matrix (mymatrix) [matrix of nodes, nodes={draw, minimum size=6mm, outer sep=0pt}, column sep=-\pgflinewidth, row sep=-\pgflinewidth]
    {  & 0 & 1\\
     0 & 1 & 0\\
	 1 & 0 & 1\\};
\draw[->, shorten <=1mm, shorten >=1mm, looseness=1.2]
    (mymatrix-2-1.north west)to[out=90, in=180]node[below right=-3pt]{\(+\)}(mymatrix-1-2.north west);
\end{tikzpicture}
\hspace{2cm}
\begin{tikzpicture}
\matrix (mymatrix) [matrix of nodes, nodes={draw, minimum size=6mm, outer sep=0pt}, column sep=-\pgflinewidth, row sep=-\pgflinewidth]
    {  & 0 & 1\\
     0 & 0 & 1\\
	 1 & 1 & 1\\};
\draw[->, shorten <=1mm, shorten >=1mm, looseness=1.2]
    (mymatrix-2-1.north west)to[out=90, in=180]node[below right=-3pt]{\(\times\)}(mymatrix-1-2.north west);
\end{tikzpicture}
\end{center}

\synthese

On admet qu'on obtient bien deux structures d'anneau (on obtient deux anneaux isomorphes à \(\nicefrac{\Z}{2\Z}\)).
\end{ex}

\begin{defi}[Élément inversible d'un anneau]
Soient \(\anneau{A}\) un anneau et \(a\) un élément de \(A\).

On dit que \(a\) est inversible dans l'anneau \(A\) s'il est inversible pour la loi \(\times\).

L'ensemble des éléments inversibles est souvent noté \(A\croix\).
\end{defi}

\begin{defprop}[Groupe des inversibles d'un anneau]\thlabel{defprop:groupeDesInversiblesD'UnAnneau}
Soit \(\anneau{A}\) un anneau.

On appelle groupe des inversibles de l'anneau \(A\) l'ensemble des éléments inversibles de \(A\), muni de la loi \(\times\).

C'est un groupe commutatif si l'anneau \(A\) est commutatif.
\end{defprop}

\begin{dem}
Montrons que l'application \(\fonctionlambda{A\croix\times A\croix}{A\croix}{\paren{a,b}}{ab}\) est bien définie, \cad que \(A\croix\) est une partie de \(A\) stable par \(\times\).

Soient \(a,b\in A\croix\). On sait que \(ab\) est inversible, d'inverse \(\paren{ab}\inv=b\inv a\inv\) (\cf \thref{prop:inverseProduitEgalProduitInversesInverse}).

Donc \(ab\in A\croix\) donc \(A\croix\) est stable par \(\times\).

On a \(\quantifs{\forall a,b,c\in A\croix}a\paren{bc}=\paren{ab}c\) donc \(\times\) est associative.

La loi \(\times\) admet \(1\) comme élément neutre dans \(A\croix\) car \(1\in A\croix\) (\cf \thref{ex:inverseElementNeutreEgalElementNeutre}).

Tout élément de \(A\croix\) est inversible dans \(A\croix\) (\cf \thref{ex:inverseInverseEgalElement}).

Donc \(A\croix\) est un groupe.
\end{dem}

\begin{ex}
Déterminons \(\Z\croix\), \(\R\croix\), \(\C\croix\) et \(\paren{\R^\N}\croix\).

Dans \(\anneau{\Z}\) (anneau de neutres \(0\) et \(1\)), les seuls éléments inversibles sont \(1\) et \(-1\) donc \(\Z\croix=\accol{-1;1}\). Sa loi de groupe est :

\begin{center}
\begin{tikzpicture}
\matrix (mymatrix) [matrix of nodes, nodes={draw, minimum size=6mm, outer sep=0pt}, column sep=-\pgflinewidth, row sep=-\pgflinewidth]
    {  & 1 & -1\\
     1 & 1 & -1\\
	 -1 & -1 & 1\\};
\draw[->, shorten <=1mm, shorten >=1mm, looseness=1.2]
    (mymatrix-2-1.north west)to[out=90, in=180]node[below right=-3pt]{\(\times\)}(mymatrix-1-2.north west);
\end{tikzpicture}
\end{center}

Dans \(\anneau{\Q}\) (anneau de neutres \(0\) et \(1\)), tous les éléments sont inversibles sauf \(0\) donc \(\Q\croix=\Q\excluant\accol{0}\).

Idem dans \(\anneau{\R}\) et \(\anneau{\C}\) : \(\R\croix=\R\excluant\accol{0}\) et \(\C\croix=\C\excluant\accol{0}\).

Dans \(\anneau{\R^\N}\) (anneau de neutres \(\paren{0}_{n\in\N}\) et \(\paren{1}_{n\in\N}\)), tous les éléments sont inversibles sauf les suites dont au moins un terme est nul donc \(\paren{\R^\N}\croix=\paren{\Rs}^\N\).
\end{ex}

\subsection{Sous-anneaux}

\begin{defi}[Sous-anneau]
Soit \(\anneau{A}\) un anneau. On note \(0\) et \(1\) ses neutres.

Un sous-anneau de \(\anneau{A}\) (ou, par abus, de \(A\)) est une partie \(B\) de \(A\) telle que :

\begin{enumerate}[series=sousanneau]
\item Les éléments neutres \(0\) et \(1\) appartiennent à \(B\). \\

\item La partie \(B\) est stable par somme : \[\quantifs{\forall b_1,b_2\in B}b_1+b_2\in B.\]

\item La partie \(B\) est stable par produit : \[\quantifs{\forall b_1,b_2\in B}b_1\times b_2\in B.\]

\item La partie \(B\) est stable par passage à l'opposé : \[\quantifs{\forall b\in B}-b\in B.\]
\end{enumerate}
\end{defi}

\begin{prop}
Tout sous-anneau d'un anneau \(\anneau{A}\) d'éléments neutres \(0\) et \(1\) est naturellement un anneau dont les lois sont celles induites par celles de \(A\).
\end{prop}

\begin{dem}
Soit \(B\) un sous-anneau de \(A\).

Selon (2) et (3), \(B\) est une partie de \(A\) stable par \(+\) et \(\times\) donc \(+\) et \(\times\) induisent des lois \(\oplus\) et \(\otimes\) sur \(B\).

Montrons que \(\groupe{B}[\oplus]\) est un groupe abélien.

On remarque que \(\groupe{B}[\oplus]\) est un sous-groupe de \(\groupe{A}\) selon (1), (2) et (4) donc \(\groupe{B}[\oplus]\) est un groupe abélien.

Montrons que \(\otimes\) est associative et possède un élément neutre :

\(\otimes\) est associative car \(\times\) est associative et \(\otimes\) possède un neutre car \(1\in B\) et \(\quantifs{\forall b\in B}\begin{dcases}1\otimes b=1\times b=b \\ b\otimes1=b\times1=b\end{dcases}\)

\(\otimes\) est distributive par rapport à \(\oplus\) car \(\times\) est distributive par rapport à \(+\).

Donc \(\anneau{B}[\oplus][\otimes]\) est un anneau.
\end{dem}

\begin{prop}
Soient \(\anneau{A}\) d'éléments neutres \(0\) et \(1\) et \(B\) une partie de \(A\).

Alors \(B\) est un sous-anneau de \(A\) si, et seulement si, les conditions suivantes sont vérifiées :

\begin{enumerate}[resume=sousanneau]
\item \(1\in B\) \\

\item \(\quantifs{\forall b_1,b_2\in B}b_1-b_2\in B\) \\

\item \(\quantifs{\forall b_1,b_2\in B}b_1\times b_2\in B\) \\
\end{enumerate}
\end{prop}

\begin{dem}
\impdir

Supposons (1), (2), (3) et (4).

On a clairement (5) et (7) selon (1) et (3).

Montrons (6) :

Soient \(b_1,b_2\in B\). On a \(-b_2\in B\) selon (4) puis \(b_1-b_2=b_1+\paren{-b_2}\in B\) selon (2).

D'où (6).

\imprec

Supposons (5), (6) et (7).

On a clairement (3) selon (7).

On a \(1\in B\) selon (5) donc \(1-1\in B\) selon (6), \cad \(0\in B\). D'où (1).

Montrons (4) :

Soit \(b\in B\). On a \(0-b\in B\) selon (6) donc \(-b\in B\). D'où (4).

Montrons (2) :

Soient \(b_1,b_2\in B\). On a \(-b_2\in B\) selon (4). Donc \(b_1-\paren{-b_2}\in B\) selon (6). Donc \(b_1+b_2\in B\). D'où (2).
\end{dem}

\begin{ex}
On vérifie facilement que \(\Z\) est un sous-anneau de \(\Q\) et de \(\R\), que \(\Q\) est un sous-anneau de \(\R\), que \(\R\) est un sous-anneau de \(\C\), que \(\R^\N\) est un sous-anneau de \(\C^\N\), ...
\end{ex}

\subsection{Morphismes d'anneaux}

\begin{defi}[Morphisme d'anneaux]
Soient \(\anneau{A_1}[+_1][\times_1]\) et \(\anneau{A_2}[+_2][\times_2]\) deux anneaux d'éléments neutres respectifs \(0_1\) et \(1_1\) et \(0_2\) et \(1_2\).

On appelle morphisme d'anneaux de \(\anneau{A_1}[+_1][\times_1]\) vers \(\anneau{A_2}[+_2][\times_2]\) (ou, par abus, de \(A_1\) vers \(A_2\)) toute application \(\phi:A_1\to A_2\) telle que les conditions suivantes sont vérifiées :

\begin{enumerate}
\item \(\phi\paren{1_1}=1_2\) \\

\item \(\quantifs{\forall a,a\prim\in A_1}\phi\paren{a+_1a\prim}=\phi\paren{a}+_2\phi\paren{a\prim}\) \\

\item \(\quantifs{\forall a,a\prim\in A_1}\phi\paren{a\times_1a\prim}=\phi\paren{a}\times_2\phi\paren{a\prim}\) \\
\end{enumerate}
\end{defi}

\begin{prop}
Soient \(\anneau{A_1}[+_1][\times_1]\) et \(\anneau{A_2}[+_2][\times_2]\) deux anneaux d'éléments neutres pour l'addition respectifs \(0_1\) et \(0_2\).

Soit \(\phi:A_1\to A_2\) un morphisme d'anneaux.

Alors on a : \[\phi\paren{0_1}=0_2\] et \[\quantifs{\forall a\in A_1}\phi\paren{-a}=-\phi\paren{a}.\]
\end{prop}

\begin{dem}
Montrons que \(\phi\paren{0_1}=0_2\). Comme \(\phi\) est un morphisme d'anneaux de \(A_1\) vers \(A_2\), c'est un morphisme de groupes de \(\groupe{A_1}[+_1]\) vers \(\groupe{A_2}[+_2]\). On a alors \(\phi\paren{0_1}=0_2\) (\cf \thref{prop:neutreEtInverseParUnMorphismeDeGroupes}).

De même, on a \(\quantifs{\forall a\in A_1}\phi\paren{-a}=-\phi\paren{a}\) car \(\phi\) est un morphisme de groupes de \(\groupe{A_1}[+_1]\) vers \(\groupe{A_2}[+_2]\) (\cf \thref{prop:neutreEtInverseParUnMorphismeDeGroupes}).
\end{dem}

\begin{prop}
Soit \(\anneau{A}\) un anneau dont on note l'élément neutre pour la multiplication \(1\) et \(B\) un sous-anneau de \(A\).

L'application \[\fonction{i}{B}{A}{b}{b}\] est un morphisme d'anneaux.
\end{prop}

\begin{dem}
On a \(i\paren{1}=1\) et \[\quantifs{\forall b_1,b_2\in B}\begin{dcases}i\paren{b_1+b_2}=b_1+b_2=i\paren{b_1}+i\paren{b_2} \\ i\paren{b_1b_2}=b_1b_2=i\paren{b_1}i\paren{b_2}\end{dcases}\]
\end{dem}

\begin{ex}
Soient \(\anneau{A_1}[+_1][\times_1]\) et \(\anneau{A_2}[+_2][\times_2]\) deux anneaux.

Les projections \(p_1:A_1\times A_2\to A_1\) et \(p_2:A_1\times A_2\to A_2\) sont des morphismes d'anneaux.
\end{ex}

\begin{ex}
Soit \(I\) un ensemble, \(J\) une partie de \(I\) et \(A\) un anneau.

Alors l'application de restriction \[\fonctionlambda{\F{I}{A}}{\F{J}{A}}{f}{\restr{f}{J}}\] est un morphisme d'anneaux.
\end{ex}

\begin{prop}[Composition de morphismes d'anneaux]
Soient \(A_1\), \(A_2\) et \(A_3\) des anneaux et \(\phi:A_1\to A_2\) et \(\psi:A_2\to A_3\) des morphismes d'anneaux.

Alors \[\psi\rond\phi:A_1\to A_3\] est un morphisme d'anneaux.
\end{prop}

\begin{dem}
On note \(1_1\), \(1_2\) et \(1_3\) les neutres respectifs de \(A_1\), \(A_2\) et \(A_3\) pour la multiplication et \(+_i\) et \(\times_i\) les lois de \(A_i\) pour tout \(i\in\interventierii{1}{3}\).

On a \(\psi\rond\phi\paren{1_1}=\psi\paren{\phi\paren{1_1}}=\psi\paren{1_2}=1_3\).

Soient \(a,b\in A_1\).

On a : \[\begin{dcases}\psi\rond\phi\paren{a+_1b}=\psi\paren{\phi\paren{a+_1b}}=\psi\paren{\phi\paren{a}+_2\phi\paren{b}}=\psi\paren{\phi\paren{a}}+_3\psi\paren{\phi\paren{b}} \\ \psi\rond\phi\paren{a\times_1b}=\psi\paren{\phi\paren{a}\times_2\phi\paren{b}}=\psi\paren{\phi\paren{a}}\times_3\psi\paren{\phi\paren{b}}\end{dcases}\]

Donc \(\psi\rond\phi\) est un morphisme d'anneaux de \(A_1\) vers \(A_3\).
\end{dem}

\subsection{Isomorphismes, endomorphismes, automorphismes}

\begin{defi}[Isomorphisme]
Soient \(A_1\) et \(A_2\) deux anneaux.

Un isomorphisme d'anneaux de \(A_1\) vers \(A_2\) est un morphisme d'anneaux \(\phi:A_1\to A_2\) bijectif.
\end{defi}

\begin{oubli}
Un composée d'isomorphismes d'anneaux est un isomorphisme d'anneaux.
\end{oubli}

\begin{theo}
Soit un isomorphisme d'anneaux \(\phi:A_1\to A_2\). Alors la bijection réciproque \(\phi\inv:A_2\to A_1\) est un isomorphisme d'anneaux.
\end{theo}

\begin{dem}
On note \(+_1\) et \(\times_1\) les lois de \(A_1\) et \(+_2\) et \(\times_2\) les lois de \(A_2\). On note \(1_1\) et \(1_2\) les neutres de \(\times_1\) et \(\times_2\).

On sait que \(\phi\inv\) est une bijection de \(A_2\) vers \(A_1\).

Montrons que \(\phi\inv\) est un morphisme d'anneaux.

On a \(\phi\paren{1_1}=1_2\) donc \(\phi\inv\paren{1_2}=1_1\).

Soient \(y,y\prim\in A_2\). On a \[\phi\paren{\phi\inv\paren{y}+_1\phi\inv\paren{y\prim}}=\phi\paren{\phi\inv\paren{y}}+_2\phi\paren{\phi\inv\paren{y\prim}}=y+_2y\prim\] donc \(\phi\inv\paren{y}+_1\phi\inv\paren{y\prim}\) est l'antécédent de \(y+_2y\prim\) par \(\phi\), \cad \[\phi\inv\paren{y}+_1\phi\inv\paren{y\prim}=\phi\inv\paren{y+_2y\prim}.\]

On montre de même \(\phi\inv\paren{y}\times_1\phi\inv\paren{y\prim}=\phi\inv\paren{y\times_2y\prim}\).

Donc \(\phi\inv\) est un morphisme d'anneaux.

Donc \(\phi\inv\) est un isomorphisme d'anneaux.
\end{dem}

\begin{defi}[Endomorphisme]
Soit \(A\) un anneau.

Un endomorphisme d'anneau de \(A\) est un morphisme d'anneaux \(\phi:A\to A\).
\end{defi}

\begin{defi}[Automorphisme]
Soit \(A\) un anneau.

Un automorphisme de \(A\) est un isomorphisme d'anneaux \(\phi:A\to A\), \cad un endomorphisme de \(A\) bijectif.

L'ensemble des automorphismes d'anneau de \(A\) est noté \(\Aut{A}\).
\end{defi}

\begin{ex}
Soit \(A\) un anneau.

Alors \(\groupe{\Aut{A}}[\rond]\) est un groupe, appelé le groupe des automorphismes de \(A\).
\end{ex}

\begin{dem}
\note{EXERCICE}
\end{dem}

\subsection{Anneaux intègres}

\begin{defi}[Anneau intègre]
Soit \(\anneau{A}\) un anneau. On note \(0\) son élément neutre pour \(+\).

On dit que \(\anneau{A}\) est un anneau intègre s'il est non-nul, commutatif et \guillemets{sans diviseurs de \(0\)} : \[\begin{dcases}0\not=1 \\ \quantifs{\forall a,b\in A}ab=ba \\ \quantifs{\forall a,b\in A}ab=0\ssi\orenv{a=0 \\ b=0}\end{dcases}\]
\end{defi}

\begin{ex}
Si \(A=\Z^2\) alors \(\paren{1,0}\times\paren{0,1}=\paren{0,0}\) donc \(A\) est un anneau non-intègre.

Si \(A=\F{\R}{\R}\) alors \(\ind{\Rps}\times\ind{\Rms}=0\) donc \(A\) est un anneau non-intègre.

Les anneaux usuels \(\Z\), \(\Q\), \(\R\) et \(\C\) sont intègres.
\end{ex}

\begin{rem}
Un anneau est intègre si, et seulement si, \(\paren{S}:\begin{dcases}\text{il est non-nul} \\ \text{il est commutatif} \\ \text{tout élément non-nul est régulier pour le produit}\end{dcases}\)
\end{rem}

\begin{dem}
\impdir

Soit \(\anneau{A}\) un anneau intègre.

Soit \(a\in A\excluant\accol{0}\). Montrons que \(a\) est régulier pour \(\times\).

Soient \(b,c\in A\) tels que \(ab=ac\).

On a : \[\begin{WithArrows}
ab-ac&=0 \\
a\paren{b-c}&=0 \Arrow{car \(A\) est intègre et \(a\not=0\)} \\
b-c&=0 \\
b&=c
\end{WithArrows}\]

\imprec

Soit \(A\) un anneau tel que \(\paren{S}\) soit vérifié.

Montrons que \(A\) est intègre.

Soient \(a,b\in A\) tels que \(ab=0\). Montrons que \(a=0\) ou \(b=0\).

Si \(a=0\) : OK.

Si \(a\not=0\) alors \(ab=a\times0\) donc \(b=0\) car \(a\) est régulier et non-nul.

Donc \(A\) est intègre.
\end{dem}

\section{Corps}

\begin{defi}[Corps]
Un corps est un anneau non-nul, commutatif et dont tout élément non-nul est inversible, \cad un anneau \(\anneau{K}\) tel que les conditions suivantes soient satisfaites :

\begin{enumerate}
\item \(K\) n'est pas un singleton, \cad \(0\not=1\). \\

\item \(K\) est un anneau commutatif. \\

\item \(K\croix=K\excluant\accol{0}\).
\end{enumerate}

On dit aussi que \(K\) est muni d'une structure de corps, ou, par abus, que \(K\) est un corps.
\end{defi}

\begin{prop}
Si \(K\) est un corps alors \(K\excluant\accol{0}\) est un groupe commutatif.
\end{prop}

\begin{dem}
On sait que le groupe des inversibles d'un anneau est un groupe pour la loi \(\times\) donc \(\groupe{K\croix}[\times]\) est un groupe (\cf \thref{defprop:groupeDesInversiblesD'UnAnneau}).

De plus, on a \(K\croix=K\excluant\accol{0}\) et \(\times\) est commutative.

Donc \(\groupe{K\excluant\accol{0}}[\times]\) est commutatif.
\end{dem}

\begin{prop}
Tout corps est un anneau intègre.
\end{prop}

\begin{dem}
C'est clair car le fait que tous les éléments non-nuls soient inversibles implique le fait que tous les éléments non-nuls soient réguliers.
\end{dem}

\begin{ex}
On vérifie facilement que \(\corps{\Q}\), \(\corps{\R}\) et \(\corps{\C}\) sont des corps.
\end{ex}

\begin{ex}
Soient \(\corps{K_1}[+_1][\times_1]\) et \(\corps{K_2}[+_2][\times_2]\) deux corps (donc deux anneaux dont on note \(0_1\) et \(1_1\) et \(0_2\) et \(1_2\) les neutres).

On a vu que \(K_1\times K_2\) est naturellement muni d'une structure d'anneau (\cf \thref{ex:produitD'anneauxEstUnAnneau}).

Cependant, ce n'est pas un corps.

En effet, il n'est pas intègre : \(\paren{1_1,0_2}\paren{0_1,1_2}=\paren{0_1,0_2}\).
\end{ex}

\begin{ex}
Soit \(I\) un ensemble et \(K\) un corps.

On a vu que \(\F{I}{K}\) est naturellement muni d'une structure d'anneau (\cf \thref{ex:fonctionsD'unEnsembleDansUnAnneauEstUnAnneau}).

Si \(I=\ensvide\) alors \(\F{I}{K}\) n'est pas un corps car c'est l'anneau nul.

Si \(I=\accol{x}\) est un singleton alors \(\fonction{\phi}{K}{\F{\accol{x}}{K}}{\lambda}{\fonctionlambda{\accol{x}}{K}{x}{\lambda}}\) est un isomorphisme d'anneaux. Or \(K\) est un corps donc \(\F{I}{K}\) est un corps.

Si \(I\) contient au moins deux élément \(x\) et \(y\) avec \(x\not=y\) alors \(\F{I}{K}\) est un anneau non-intègre car \(\fonction{f}{I}{K}{z}{\begin{dcases}1&\text{si }z=x \\ 0&\text{sinon}\end{dcases}}\) et \(\fonction{g}{I}{K}{z}{\begin{dcases}1&\text{si }z=y \\ 0&\text{sinon}\end{dcases}}\) vérifient \(f\not=0\), \(g\not=0\) et \(fg=0\).

Donc \(\F{I}{K}\) n'est pas un corps.
\end{ex}

\chapter{Limites de fonctions, continuité}

\minitoc

\section{Voisinages}

\subsection{Définition}

\begin{defi}
Soit \(V\subset\R\).

Soit \(a\in\R\). On dit que \(V\) est un voisinage de \(a\) dans \(\R\) si on a : \[\quantifs{\exists\epsilon\in\Rps}\intervee{a-\epsilon}{a+\epsilon}\subset V.\]

On dit que \(V\) est un voisinage de \(\pinf\) dans \(\R\) si on a : \[\quantifs{\exists\alpha\in\R}\intervee{\alpha}{\pinf}\subset V.\]

On dit que \(V\) est un voisinage de \(\minf\) dans \(\R\) si on a : \[\quantifs{\exists\alpha\in\R}\intervee{\minf}{\alpha}\subset V.\]
\end{defi}

\begin{ex}
\(\R\) est un voisinage de tout point \(a\in\Rb\).

\(\Rp\) est un voisinage de \(\pinf\).

Si \(a\in\Rps\) alors \(\Rp\) est un voisinage de \(a\) car \(\intervee{a-\epsilon}{a+\epsilon}\subset\Rp\) avec \(\epsilon=a\).

\(\Rp\) n'est pas un voisinage de \(0\) car \(\quantifs{\forall\epsilon\in\Rps}\intervee{0-\epsilon}{0+\epsilon}\not\subset\Rp\).

Pour tout \(a\in\Rm\) on a : \(\quantifs{\forall\epsilon\in\Rps}\intervee{a-\epsilon}{a+\epsilon}\not\subset\Rp\) car \(a\not\in\Rp\).

\(\Rp\) n'est pas un voisinage de \(\minf\).

L'ensemble des points dont \(\Rp\) est un voisinage est donc \(\intervei{0}{\pinf}\).

L'ensemble des points dont \(\intervii{0}{1}\) est \(\intervee{0}{1}\).

Soit \(a\in\R\). On a \(\quantifs{\forall\epsilon\in\Rps}\intervee{a-\epsilon}{a+\epsilon}\inter\paren{\R\excluant\Q}\not=\ensvide\) car \(\R\excluant\Q\) est dense dans \(R\).

Donc \(\quantifs{\forall\epsilon\in\Rps}\intervee{a-\epsilon}{a+\epsilon}\not\subset\Q\).

Donc \(\Q\) n'est le voisinage d'aucun réel.

De même, \(\Q\) n'est pas un voisinage de \(\pinf\) ni de \(\minf\).
\end{ex}

\begin{nota}
Soit \(a\in\Rb\).

Dans ce cours, on notera \(\V{a}\) l'ensemble des voisinages de \(a\) dans \(\R\).
\end{nota}

\begin{prop}\thlabel{prop:proprietesVoisinages}
Soient \(a,b\in\Rb\).

\begin{enumerate}
\item Si \(a\in\R\) alors tout voisinage de \(a\) contient \(a\) : \[\quantifs{\forall V\in\V{a}}a\in V.\]

\item L'intersection de deux voisinages de \(a\) est encore un voisinage de \(a\) : \[\quantifs{\forall V,W\in\V{a}}V\inter W\in\V{a}.\]

\item Si \(a\not=b\) alors \(a\) et \(b\) admettent des voisinages respectifs disjoints : \[\quantifs{\exists V\in\V{a};\exists W\in\V{b}}V\inter W=\ensvide.\]
\end{enumerate}
\end{prop}

\begin{dem}
\note{EXERCICE}
\end{dem}

\begin{rem}[Reformulation de la définition d'une partie dense de \(\R\) (\thref{defi:partieDenseDansR})]
Soit \(A\subset\R\).

On a : \[A\text{ dense dans }\R\ssi\quantifs{\forall x\in\R;\forall V\in\V{x}}A\inter V\not=\ensvide.\]
\end{rem}

\subsection{Vocabulaire lié aux voisinages}

Si \(a\in\Rb\), on dit qu'une propriété est vraie \guillemets{au voisinage de \(a\)} si elle est vraie sur un certain voisinage de \(a\).

\begin{defi}
Soient \(a\in\Rb\), \(A\subset\R\) et \(f:A\to\R\).

On dit que \(f\) est bornée au voisinage de \(a\) s'il existe un voisinage \(V\) de \(a\) tel que la restriction \(\restr{f}{A\inter V}\) soit bornée : \[\quantifs{\exists V\in\V{a};\exists M\in\Rp;\forall x\in A\inter V}\abs{f\paren{x}}\leq M.\]
\end{defi}

\begin{ex}
La fonction \(\exp:\R\to\R\) est bornée au voisinage de tout point \(a\in\intervie{\minf}{\pinf}\).

Elle n'est pas bornée au voisinage de \(\pinf\).
\end{ex}

\begin{dem}
On a \(\quantifs{\forall x\in\intervee{a-1}{a+1}}0<\exp x<\e{a+1}\) donc \(\exp\) est bornée au voisinage de \(a\).

De plus, \(\quantifs{\forall x\in\Rm}0<\exp x\leq1\) donc \(\exp\) est bornée au voisinage de \(\minf\) (car \(\Rm\in\V{\minf}\)).
\end{dem}

\begin{defi}
Soient \(A\subset\R\), \(f,g\in\F{A}{\R}\) et \(a\in\Rb\).

On dit que \(f\) et \(g\) coïncident au voisinage de \(a\) si on a : \[\quantifs{\exists V\in\V{a};\forall x\in A\inter V}f\paren{x}=g\paren{x}.\]
\end{defi}

\begin{rappel}[Extremum global]
Soient \(A\subset\R\), \(f:A\to\R\) et \(a\in A\).

On dit que \(f\) admet un maximum (global) en \(a\) si on a : \(\quantifs{\forall x\in A}f\paren{x}\leq f\paren{a}\).

On dit que \(f\) admet un minimum (global) en \(a\) si on a : \(\quantifs{\forall x\in A}f\paren{x}\geq f\paren{a}\).

On dit que \(f\) admet un extremum (global) en \(a\) si \(f\) admet un maximum ou un minimum en \(a\).
\end{rappel}

\begin{defi}[Extremum local]
Soient \(A\subset\R\), \(f:A\to\R\) et \(a\in A\).

On dit que \(f\) admet un maximum local en \(a\) s'il existe un voisinage \(V\) de \(a\) tel que \(\restr{f}{V\inter A}\) admette un maximum global en \(a\), \cad si on a : \[\quantifs{\exists V\in\V{a};\forall x\in V\inter A}f\paren{x}\leq f\paren{a}.\]

On dit que \(f\) admet un minimum local en \(a\) s'il existe un voisinage \(V\) de \(a\) tel que \(\restr{f}{V\inter A}\) admette un minimum global en \(a\), \cad si on a : \[\quantifs{\exists V\in\V{a};\forall x\in V\inter A}f\paren{x}\geq f\paren{a}.\]

On dit que \(f\) admet un extremum local en \(a\) si \(f\) admet un maximum local ou un minimum local en \(a\).
\end{defi}

\begin{ex}
Posons \(\fonction{f}{\intervii{0}{2\pi}}{\R}{x}{\sin x}\) :

\begin{center}
\begin{tkz}
\begin{axis}[axis lines=middle,
xmin=-1,xmax=7,
ymin=-1.5,ymax=1.5,
xtick={pi/2,pi,3*pi/2,2*pi},
xticklabels={\(\dfrac{\pi}{2}\),\(\pi\),\(\dfrac{3\pi}{2}\),\(2\pi\)},
clip=false]
\addplot[domain=0:2*pi,samples=1000,smooth,thick,blue] {sin(deg(x))};
\end{axis}
\end{tkz}
\end{center}

\(f\) admet :

\begin{itemize}
\item un maximum global en \(\dfrac{\pi}{2}\) ; \\

\item un minimum global en \(\dfrac{3\pi}{2}\) ; \\

\item un maximum local en \(2\pi\) ; \\

\item un minimum local en \(0\).
\end{itemize}
\end{ex}

\begin{rem}
Pluriel de maximum : maxima ou maximums.

Pluriel de minimum : minima ou minimums.

Pluriel d'extremum : extrema ou extremums.
\end{rem}

\section{Limite d'une fonction en un point de \(\Rb\)}

Dans ce chapitre, on considère une fonction \(f:A\to\R\) définie sur une partie \(A\subset\R\) et on définit sa limite en un point \(a\in\Rb\).

Pour cela, on suppose qu'il existe des points où la fonction \(f\) est définie et qui sont arbitrairement proches de \(a\) (autrement, la limite n'a pas de sens).

Précisément, on supposera que tout voisinage de \(a\) dans \(\R\) rencontre \(A\) : \[\quantifs{\forall V\in\V{a}}V\inter A\not=\ensvide.\]

Si \(a=\pinf\), cela signifie que \(A\) est une partie non-majorée de \(\R\).

Si \(a=\minf\), cela signifie que \(A\) est une partie non-minorée de \(\R\).

Si \(a\in\R\), cela signifie : \(\quantifs{\forall\epsilon\in\Rps;\exists a\prim\in A}\abs{a\prim-a}\leq\epsilon\).

\subsection{Définition d'une limite}

\begin{defprop}\thlabel{defprop:limiteDeFonctionEnUnPoint}
Soient \(A\subset\R\), \(f:A\to\R\) et \(a,l\in\Rb\).

On suppose que tout voisinage de \(a\) dans \(\R\) rencontre \(A\) : \[\quantifs{\forall V\in\V{a}}V\inter A\not=\ensvide.\]

On dit que \guillemets{\(f\paren{x}\) tend vers \(l\) quand \(x\) tend vers \(a\)} ou que \guillemets{\(f\) tend vers \(l\) en \(a\)} si on a : \[\quantifs{\forall V\in\V{l};\exists W\in\V{a};\forall x\in W\inter A}f\paren{x}\in V,\] \cad : \[\quantifs{\forall V\in\V{l};\exists W\in\V{a}}f\paren{W\inter A}\subset V.\]

Unicité de la limite : il existe au plus un élément \(l\in\Rb\) tel que \(f\) tende vers \(l\) en \(a\). S'il existe, on appelle cet élément \(l\) la limite de \(f\) en \(a\) et on le note : \[\lim_af\quad\text{ou}\quad\lim_{x\to a}f\paren{x}\quad\text{ou}\quad\lim_{\substack{x\to a \\ x\in A}}f\paren{x}.\]
\end{defprop}

\begin{dem}
Soient \(l,l\prim\in\Rb\) deux limites de \(f\) en \(a\). Montrons que \(l=l\prim\).

Par l'absurde, supposons \(l\not=l\prim\).

Soient \(V\in\V{l}\) et \(V\prim\in\V{l\prim}\) tels que \(V\inter V\prim=\ensvide\).

Soit \(W\in\V{a}\) tel que \(f\paren{W\inter A}\subset V\).

Soit \(W\prim\in\V{a}\) tel que \(f\paren{W\prim\inter A}\subset V\prim\).

On pose \(W\seconde=W\inter W\prim\). On a \(W\seconde\in\V{a}\) selon la \thref{prop:proprietesVoisinages}.

De plus : \[\begin{aligned}
f\paren{W\seconde\inter A}&\subset f\paren{W\inter A}\inter f\paren{W\prim\inter A} \\
&\subset V\inter V\prim \\
&=\ensvide\text{ : contradiction.}
\end{aligned}\]
\end{dem}

\begin{rem}
Lorsqu'on utilise une notation de limite dans une formule mathématique, on prétend que la limite existe.

Par exemple, la proposition \guillemets{\(\lim_{x\to a}f\paren{x}\not=0\)} signifie que la limite existe et est non-nulle (ce n'est donc pas la négation de la proposition \guillemets{\(\lim_{x\to a}f\paren{x}=0\)}).

En revanche, si on écrit par exemple \guillemets{\(\lim_{x\to a}f\paren{x}\text{ existe}\ssi\dots\)}, on ne suppose pas a priori que la limite existe.
\end{rem}

\begin{rem}
La limite d'une fonction \(f:\N\to\R\) en \(\pinf\) coïncide avec celle de la suite \(\paren{f\paren{n}}_{n\in\N}\).
\end{rem}

\begin{rem}
Soient \(A\subset\R\), \(f:A\to\R\) et \(a,l\in\Rb\).

Si \(l\in\R\) alors on a \[\lim_{x\to a}f\paren{x}=l\ssi\lim_{x\to a}f\paren{x}-l=0.\]

Si \(a\in\R\) alors on a \[\lim_{x\to a}f\paren{x}=l\ssi\lim_{h\to0}f\paren{a+h}=l.\]
\end{rem}

\begin{rem}\thlabel{rem:fonctionsQuiCoincidentSsiLimitesEgales}
Soient \(A\subset\R\), \(f,g\in\F{A}{\R}\) et \(a\in\Rb\).

Si \(f\) et \(g\) coïncident au voisinage de \(a\) alors \(f\) admet une limite en \(a\) si, et seulement si, \(g\) admet une limite en \(a\). Les deux limites sont alors égales.
\end{rem}

\begin{prop}[Reformulation de la \thref{defprop:limiteDeFonctionEnUnPoint} au cas par cas]
Soient \(A\subset\R\), \(f:A\to\R\) et \(a,l\in\Rb\).

\begin{itemize}
\item Si \(a\in\R\) et \(l\in\R\) : on suppose \(\quantifs{\forall\delta\in\Rps;\exists x\in A}\abs{x-a}\leq\delta\). On a : \[l=\lim_{x\to a}f\paren{x}\ssi\croch{\quantifs{\forall\epsilon\in\Rps;\exists\delta\in\Rps;\forall x\in A}\abs{x-a}\leq\delta\imp\abs{f\paren{x}-l}\leq\epsilon}.\] \\

\item Si \(a\in\R\) et \(l=\pinf\) : on suppose \(\quantifs{\forall\delta\in\Rps;\exists x\in A}\abs{x-a}\leq\delta\). On a : \[l=\lim_{x\to a}f\paren{x}\ssi\croch{\quantifs{\forall\alpha\in\R;\exists\delta\in\Rps;\forall x\in A}\abs{x-a}\leq\delta\imp f\paren{x}\geq\alpha}.\] \\

\item Si \(a\in\R\) et \(l=\minf\) : on suppose \(\quantifs{\forall\delta\in\Rps;\exists x\in A}\abs{x-a}\leq\delta\). On a : \[l=\lim_{x\to a}f\paren{x}\ssi\croch{\quantifs{\forall\alpha\in\R;\exists\delta\in\Rps;\forall x\in A}\abs{x-a}\leq\delta\imp f\paren{x}\leq\alpha}.\] \\

\item Si \(a=\pinf\) et \(l\in\R\) : on suppose la partie \(A\) non-majorée. On a : \[l=\lim_{x\to a}f\paren{x}\ssi\croch{\quantifs{\forall\epsilon\in\Rps;\exists\beta\in\R;\forall x\in A}x\geq\beta\imp\abs{f\paren{x}-l}\leq\epsilon}.\] \\

\item Si \(a=\pinf\) et \(l=\pinf\) : on suppose la partie \(A\) non-majorée. On a : \[l=\lim_{x\to a}f\paren{x}\ssi\croch{\quantifs{\forall\alpha\in\R;\exists\beta\in\R;\forall x\in A}x\geq\beta\imp f\paren{x}\geq\alpha}.\] \\

\item Si \(a=\pinf\) et \(l=\minf\) : on suppose la partie \(A\) non-majorée. On a : \[l=\lim_{x\to a}f\paren{x}\ssi\croch{\quantifs{\forall\alpha\in\R;\exists\beta\in\R;\forall x\in A}x\geq\beta\imp f\paren{x}\leq\alpha}.\] \\

\item Si \(a=\minf\) et \(l\in\R\) : on suppose la partie \(A\) non-minorée. On a : \[l=\lim_{x\to a}f\paren{x}\ssi\croch{\quantifs{\forall\epsilon\in\Rps;\exists\beta\in\R;\forall x\in A}x\leq\beta\imp\abs{f\paren{x}-l}\leq\epsilon}.\] \\

\item Si \(a=\minf\) et \(l=\pinf\) : on suppose la partie \(A\) non-minorée. On a : \[l=\lim_{x\to a}f\paren{x}\ssi\croch{\quantifs{\forall\alpha\in\R;\exists\beta\in\R;\forall x\in A}x\leq\beta\imp f\paren{x}\geq\alpha}.\] \\

\item Si \(a=\minf\) et \(l=\minf\) : on suppose la partie \(A\) non-minorée. On a : \[l=\lim_{x\to a}f\paren{x}\ssi\croch{\quantifs{\forall\alpha\in\R;\exists\beta\in\R;\forall x\in A}x\leq\beta\imp f\paren{x}\leq\alpha}.\]
\end{itemize}
\end{prop}

\begin{rappel}[Définition de \(a^b\)]
Soient \(a,b\in\R\).

L'expression \(a^b\) est définie dans plusieurs cas :

\begin{itemize}
\item Si \(b\in\N\) alors on pose \[a^b=\underbrace{a\times\dots\times a}_{b\text{ facteurs}}.\] Autrement dit (définition par récurrence) : \[\begin{dcases}a^0=1 \\ \quantifs{\forall b\in\N}a^{b+1}=a\times a^b\end{dcases}\]

\item Si \(a\not=0\) et \(b\in\Z\) : on étend la définition précédente en posant, si \(b\) est un entier strictement négatif : \[a^b=\dfrac{1}{a^{-b}}.\]

\item Si \(a>0\) alors on pose \[a^b=\e{b\ln a}.\]

\item Si \(a=0\) et \(b\geq0\) alors on pose : \[0^b=\begin{dcases}1 &\text{si }b=0 \\ 0 &\text{si }b>0\end{dcases}\]
\end{itemize}

Lorsque plusieurs cas s'appliquent, ils donnent le même résultat.
\end{rappel}

\begin{ex}
Soit \(\lambda\in\R\). On a : \[\lim_{x\to\pinf}x^\lambda=\begin{dcases}\pinf &\text{si }\lambda>0 \\ 1 &\text{si }\lambda=0 \\ 0 &\text{si }\lambda<0\end{dcases}\qquad\text{et}\qquad\lim_{\substack{x\to0 \\ x\in\Rps}}x^\lambda=\begin{dcases}0 &\text{si }\lambda>0 \\ 1 &\text{si }\lambda=0 \\ \pinf &\text{si }\lambda<0\end{dcases}\]
\end{ex}

\begin{dem}
Si \(\lambda>0\) :

Montrons que \[\quantifs{\forall\alpha\in\Rps;\exists\beta\in\R;\forall x\in\Rps}x\geq\beta\imp x^\lambda\geq\alpha.\]

Soit \(\alpha\in\Rps\).

On a \[\begin{aligned}
\quantifs{\forall x\in\Rps}x^\lambda\geq\alpha&\ssi\paren{x^\lambda}^{\nicefrac{1}{\lambda}}\geq\alpha^{\nicefrac{1}{\lambda}} \\
&\ssi x\geq\alpha^{\nicefrac{1}{\lambda}}
\end{aligned}\] car \(t\mapsto t^{\nicefrac{1}{\lambda}}=\e{\nicefrac{\ln t}{\lambda}}\) est strictement croissante sur \(\Rps\) car \(\lambda>0\).

Donc le réel \(\beta=\alpha^{\nicefrac{1}{\lambda}}\) convient.

Donc \(\lim_{x\to\pinf}x^\lambda=\pinf\).

Montrons que \[\quantifs{\forall\epsilon\in\Rps;\exists\delta\in\Rps;\forall x\in\Rps}x\leq\delta\imp\abs{x^\lambda-0}\leq\epsilon.\]

Soit \(\epsilon\in\Rps\).

On a \[\begin{aligned}
\quantifs{\forall x\in\Rps}\abs{x^\lambda-0}\leq\epsilon&\ssi x^\lambda\leq\epsilon \\
&\ssi x\leq\epsilon^{\nicefrac{1}{\lambda}}.
\end{aligned}\]

Donc \(\delta=\epsilon^{\nicefrac{1}{\lambda}}\) convient.

Donc \(\lim_{\substack{x\to0 \\ x\in\Rps}}x^\lambda=0\).

Si \(\lambda=0\) :

On a \(\quantifs{\forall x\in\Rps}x^\lambda=1\) donc \[\begin{dcases}\lim_{x\to\pinf}x^\lambda=1 \\ \lim_{\substack{x\to0 \\ x\in\Rps}}x^\lambda=1\end{dcases}\]

Si \(\lambda<0\) :

La fonction \(t\mapsto t^{\nicefrac{1}{\lambda}}=\e{\nicefrac{\ln t}{\lambda}}\) est strictement décroissante sur \(\Rps\).

Montrons que \[\quantifs{\forall\epsilon\in\Rps;\exists\beta\in\R;\forall x\in\Rps}x\geq\beta\imp\abs{x^\lambda}\leq\epsilon.\]

Soit \(\epsilon\in\Rps\).

Le réel \(\beta=\epsilon^{\nicefrac{1}{\lambda}}\) convient.

Donc \(\lim_{n\to\pinf}x^\lambda=0\).

Montrons que \[\quantifs{\forall\alpha\in\Rps;\exists\delta\in\Rps;\forall x\in\Rps}x\leq\delta\imp x^\lambda\geq\alpha.\]

Soit \(\alpha\in\Rps\).

Le réel \(\delta=\alpha^{\nicefrac{1}{\lambda}}\) convient car \(\quantifs{\forall x\in\Rps}x\leq\alpha^{\nicefrac{1}{\lambda}}\imp x^\lambda\geq\alpha\).

Donc \(\lim_{\substack{x\to0 \\ x\in\Rps}}x^\lambda=\pinf\).
\end{dem}

\begin{rem}\thlabel{rem:limiteEnAEgaleFDeA}
Soient \(A\subset\R\), \(f:A\to\R\) et \(a\in A\).

On suppose que \(f\) est définie en \(a\).

Si \(f\) admet une limite en \(a\) alors cette limite est \(f\paren{a}\).
\end{rem}

\begin{dem}
Supposons que \(f\) possède une limite \(l\) en \(a\).

Montrons que \(l=f\paren{a}\).

On a \[\quantifs{\forall V\in\V{l};\exists W\in\V{a};\forall x\in A\inter W}f\paren{x}\in V.\]

Donc \[\quantifs{\forall V\in\V{l}}f\paren{a}\in V\] car \(\begin{dcases}\quantifs{\forall W\in\V{a}}a\in W \\ a\in A\end{dcases}\) donc \(a\in A\inter W\).

Donc \(f\paren{a}=l\) car sinon il existe \(V\prim\in\V{f\paren{a}}\) tel que \(\begin{dcases}V\inter V\prim=\ensvide \\ f\paren{a}\in V\prim\end{dcases}\) donc \(f\paren{a}\not\in V\).
\end{dem}

\subsection{Limites et ordre}

\begin{prop}\thlabel{prop:majorantOuMinorantStrictDeLaLimiteD'UneFonctionMajoreOuMinoreStrictementLaFonction}
Soient \(A\subset\R\), \(f:A\to\R\), \(a,l\in\Rb\) et \(\lambda\in\R\).

On suppose \[\lim_{x\to a}f\paren{x}=l.\]

Si \(\lambda>l\) alors \(\lambda\) majore strictement \(f\) au voisinage de \(a\) : \[\quantifs{\exists V\in\V{a};\forall x\in V\inter A}f\paren{x}<\lambda.\]

Si \(\lambda<l\) alors \(\lambda\) minore strictement \(f\) au voisinage de \(a\) : \[\quantifs{\exists V\in\V{a};\forall x\in V\inter A}f\paren{x}>\lambda.\]
\end{prop}

\begin{dem}
Supposons \(\lambda>l\).

Alors \(V=\intervee{\minf}{\lambda}\in\V{l}\).

Donc il existe \(W\in\V{a}\) tel que \[\quantifs{\forall x\in A}x\in W\imp f\paren{x}\in V.\]

\Cad \(\quantifs{\forall x\in A}x\in W\imp f\paren{x}<\lambda\).

Si \(\lambda<l\) : idem en prenant \(V=\intervee{\lambda}{\pinf}\).
\end{dem}

\begin{cor}\thlabel{cor:limiteFinieDoncFonctionBornée}
Soient \(A\subset\R\), \(f:A\to\R\) et \(a\in\Rb\).

Si \(f\) tend vers une limite finie en \(a\) alors \(f\) est bornée au voisinage de \(a\).

Si \(f\) tend vers \(\pinf\) en \(a\) alors \(f\) est minorée au voisinage de \(a\).

Si \(f\) tend vers \(\minf\) en \(a\) alors \(f\) est majorée au voisinage de \(a\).
\end{cor}

\begin{dem}
Supposons que \(f\) tend vers une limite \(l\in\R\) en \(a\).

On a \(l-1<l<l+1\) donc \[\begin{dcases}\quantifs{\exists V_1\in\V{a};\forall x\in A\inter V_1}l-1<f\paren{x} \\ \quantifs{\exists V_2\in\V{a};\forall x\in A\inter V_2}l+1>f\paren{x}\end{dcases}\] selon la \thref{prop:majorantOuMinorantStrictDeLaLimiteD'UneFonctionMajoreOuMinoreStrictementLaFonction}.

D'où, avec \(V=V_1\inter V_2\) : \[\quantifs{\forall x\in A\inter V}l-1<f\paren{x}<l+1\] donc \(f\) est bornée au voisinage de \(a\).

Supposons \(\lim_{x\to a}f\paren{x}=\pinf\).

On a \(0<\pinf\). Donc \(0\) minore strictement \(f\) au voisinage de \(a\) selon la \thref{prop:majorantOuMinorantStrictDeLaLimiteD'UneFonctionMajoreOuMinoreStrictementLaFonction}.

Idem si \(\lim_{x\to a}f\paren{x}=\minf\).
\end{dem}

\begin{cor}[Passage à la limite dans une inégalité, version 1]\thlabel{cor:passageALaLimiteDansUneInegalitéFonctionsV1}
Soient \(A\subset\R\), \(f:A\to\R\), \(a\in\Rb\) et \(\lambda\in\R\).

On suppose que \(f\) admet une limite en \(a\).

Si \(f\) est minorée par \(\lambda\) au voisinage de \(a\), \cad \(\quantifs{\exists V\in\V{a};\forall x\in A\inter V}\lambda\leq f\paren{x}\), alors \[\lambda\leq\lim_{x\to a}f\paren{x}.\]

Si \(f\) est majorée par \(\lambda\) au voisinage de \(a\), \cad \(\quantifs{\exists V\in\V{a};\forall x\in A\inter V}\lambda\geq f\paren{x}\), alors \[\lambda\geq\lim_{x\to a}f\paren{x}.\]
\end{cor}

\begin{dem}
On suppose \(\lambda\leq f\) au voisinage de \(a\).

Montrons que \(\lambda\leq\lim_{x\to a}f\paren{x}\).

Par l'absurde, supposons \(\lim_{x\to a}f\paren{x}<\lambda\).

Alors \(\lambda\) majore strictement \(f\) au voisinage de \(a\) : \[\quantifs{\exists V_2\in\V{a};\forall x\in A\inter V_2}f\paren{x}<\lambda.\]

Or on a : \[\quantifs{\exists V_1\in\V{a};\forall x\in A\inter V_1}\lambda\leq f\paren{x}.\]

Considérons de tels voisinages \(V_1,V_2\) de \(a\) et posons \(V=V_1\inter V_2\).

On a \(V\in\V{a}\) et \(\quantifs{\forall x\in A\inter V}\begin{dcases}f\paren{x}\geq\lambda \\ f\paren{x}<\lambda\end{dcases}\)

Donc \(A\inter V=\ensvide\) : contradiction avec l'existence de \(\lim_{x\to a}f\paren{x}\).

Idem si \(\lambda\geq f\) au voisinage de \(a\).
\end{dem}

\begin{prop}[Théorème des gendarmes]
Soient \(A\subset\R\), trois fonctions \(f,g,h\in\F{A}{\R}\) et \(a\in\Rb\).

Si on a \(\quantifs{\forall x\in A}g\paren{x}\leq f\paren{x}\leq h\paren{x}\) et si \(g\) et \(h\) admettent la même limite finie \(l\) en \(a\), alors on a \[\lim_{x\to a}f\paren{x}=l.\]

Si on a \(\quantifs{\forall x\in A}g\paren{x}\leq f\paren{x}\) et \(\lim_{x\to a}g\paren{x}=\pinf\), alors on a \[\lim_{x\to a}f\paren{x}=\pinf.\]

Si on a \(\quantifs{\forall x\in A}f\paren{x}\leq h\paren{x}\) et \(\lim_{x\to a}h\paren{x}=\minf\), alors on a \[\lim_{x\to a}f\paren{x}=\minf.\]

NB : dans les trois cas, la proposition assure l'existence de la limite de \(f\) en \(a\).
\end{prop}

\begin{dem}
\note{Exercice} (s'inspirer de la démonstration du théorème des gendarmes pour les suites (\thref{dem:théorèmeDesGendarmesDansLeCasFiniSuites} et \thref{dem:théorèmeDesGendarmesDansLeCasInfiniSuites})).
\end{dem}

\subsection{Compléments}

\subsubsection{Limites à droite/gauche}

\begin{defi}[Limite à droite]
Soient \(A\subset\R\), \(f:A\to\R\) et \(a\in\R\).

On appelle limite à droite de \(f\) en \(a\) la limite en \(a\), si elle existe, de la restriction \(\restr{f}{A\inter\intervee{a}{\pinf}}\). On la note alors : \[\lim_{a^+}f\qquad\text{ou}\qquad\lim_{x\to a^+}f\paren{x}\qquad\text{ou}\qquad\lim_{\substack{x\to a \\ x>a}}f\paren{x}.\]

NB : pour que cette limite existe, il faut en particulier que tout voisinage de \(a\) rencontre l'ensemble \(A\inter\intervee{a}{\pinf}\).
\end{defi}

\begin{defi}[Limite à gauche]
Soient \(A\subset\R\), \(f:A\to\R\) et \(a\in\R\).

On appelle limite à gauche de \(f\) en \(a\) la limite en \(a\), si elle existe, de la restriction \(\restr{f}{A\inter\intervee{\minf}{a}}\). On la note alors : \[\lim_{a^-}f\qquad\text{ou}\qquad\lim_{x\to a^-}f\paren{x}\qquad\text{ou}\qquad\lim_{\substack{x\to a \\ x<a}}f\paren{x}.\]

NB : pour que cette limite existe, il faut en particulier que tout voisinage de \(a\) rencontre l'ensemble \(A\inter\intervee{\minf}{a}\).
\end{defi}

\begin{ex}
On a : \[\lim_{x\to0^+}\dfrac{1}{x}=\pinf\qquad\text{et}\qquad\lim_{x\to0^-}\dfrac{1}{x}=\minf\qquad\qquad\lim_{x\to0^+}\floor{x}=0\qquad\text{et}\qquad\lim_{x\to0^-}\floor{x}=-1.\]
\end{ex}

\begin{prop}
Soient \(A\subset\R\), \(f:A\to\R\), \(a\in\R\) et \(l\in\Rb\).

On suppose que tout voisinage de \(a\) rencontre les ensembles \(A\inter\intervee{\minf}{a}\) et \(A\inter\intervee{a}{\pinf}\).

Si \(f\) n'est pas définie en \(a\) alors \[\lim_af=l\ssi\lim_{a^-}f=\lim_{a^+}f=l.\]

Si \(f\) est définie en \(a\) alors \[\lim_af=l\ssi\lim_{a^-}f=\lim_{a^+}f=f\paren{a}=l.\]
\end{prop}

\subsubsection{Limites par valeurs supérieures/inférieures}

\begin{nota}[Limite par valeurs supérieures]
Soient \(A\subset\R\), \(f:A\to\R\), \(a\in\Rb\) et \(l\in\R\).

La notation \(\lim_af=l^+\) signifie que \(f\) tend vers \(l\) en \(a\) et qu'on a \(f>l\) au voisinage de \(a\) : \[\lim_af=l^+\ssi\begin{dcases}\lim_af=l \\ \quantifs{\exists V\in\V{a};\forall x\in A\inter V}f\paren{x}>l\end{dcases}\]

On dit alors que \(f\) tend vers \(l\) en \(a\) par valeurs supérieures.
\end{nota}

\begin{nota}[Limite par valeurs inférieures]
Soient \(A\subset\R\), \(f:A\to\R\), \(a\in\Rb\) et \(l\in\R\).

La notation \(\lim_af=l^-\) signifie que \(f\) tend vers \(l\) en \(a\) et qu'on a \(f<l\) au voisinage de \(a\) : \[\lim_af=l^-\ssi\begin{dcases}\lim_af=l \\ \quantifs{\exists V\in\V{a};\forall x\in A\inter V}f\paren{x}<l\end{dcases}\]

On dit alors que \(f\) tend vers \(l\) en \(a\) par valeurs inférieures.
\end{nota}

\begin{ex}
On a : \[\lim_{x\to0^+}x=0^+\qquad\lim_{x\to0^-}x=0^-\qquad\qquad\lim_{x\to\pinf}\dfrac{1}{x}=0^+\qquad\lim_{x\to\minf}\dfrac{1}{x}=0^-.\]
\end{ex}

\subsection{Opérations sur les limites}

\begin{prop}\thlabel{prop:limiteProduitOuSommeFonctionMinoréeMajoréeOuBornée}
Soient \(A\subset\R\), \(f,g\in\F{A}{\R}\) et \(a\in\Rb\).

Si \(\lim_{x\to a}f\paren{x}=0\) et \(g\) bornée au voisinage de \(a\) alors \[\lim_{x\to a}f\paren{x}g\paren{x}=0\].

Si \(\lim_{x\to a}f\paren{x}=\pinf\) et \(g\) minorée au voisinage de \(a\) alors \[\lim_{x\to a}f\paren{x}+g\paren{x}=\pinf\].

Si \(\lim_{x\to a}f\paren{x}=\minf\) et \(g\) majorée au voisinage de \(a\) alors \[\lim_{x\to a}f\paren{x}+g\paren{x}=\minf\].
\end{prop}

\begin{dem}
\note{Exercice} (s'inspirer des démonstrations analogues sur les suites (\thref{dem:limiteProduitSuiteBornéeEtSuiteTendantVersZéroVautZéroSuites} et \thref{dem:limiteSommeSuiteMajoréeOuMinoréeEtSuiteDivergente})).
\end{dem}

\begin{prop}[Opérations algébriques sur les limites]\thlabel{prop:opérationsAlgébriquesSurLesLimitesFonctions}
Soient \(A\subset\R\), \(f,g\in\F{A}{\R}\), \(a,l,l\prim\in\Rb\) et \(\lambda,\mu\in\R\).

On suppose \[\lim_af=l\qquad\text{et}\qquad\lim_ag=l\prim.\]

Si la somme \(l+l\prim\) est bien définie dans \(\Rb\) (\cad si \(\paren{l,l\prim}\not\in\accol{\paren{\minf,\pinf};\paren{\pinf,\minf}}\)), alors \[\lim_a\paren{f+g}=l+l\prim.\]

Plus généralement, si la combinaison linéaire \(\lambda l+\mu l\prim\) est bien définie dans \(\Rb\), alors \[\lim_a\paren{\lambda f+\mu g}=\lambda l+\mu l\prim.\]

Si le produit \(ll\prim\) est bien défini dans \(\Rb\) (\cad si \(\paren{l,l\prim}\not\in\accol{\paren{\minf,0};\paren{\pinf,0};\paren{0;\pinf};\paren{0;\minf}}\)), alors \[\lim_afg=ll\prim.\]

Si \(\lim_af=\pinf\) alors \[\lim_a\dfrac{1}{f}=0^+.\]

Si \(\lim_af=\minf\) alors \[\lim_a\dfrac{1}{f}=0^-.\]

Si \(\lim_af=l\in\Rs\) alors \[\lim_a\dfrac{1}{f}=\dfrac{1}{l}.\]

Si \(\lim_af=0^+\) alors \[\lim_a\dfrac{1}{f}=\pinf.\]

Si \(\lim_af=0^-\) alors \[\lim_a\dfrac{1}{f}=\minf.\]
\end{prop}

\begin{dem}[Somme]
Supposons \(l,l\prim\in\R\).

Montrons que \(\lim_a\paren{f+g}=l+l\prim\), \cad : \[\quantifs{\forall\epsilon\in\Rps;\exists V\in\V{a};\forall x\in A\inter V}\abs{f\paren{x}+g\paren{x}-l-l\prim}\leq\epsilon.\]

Soit \(\epsilon\in\Rps\).

Soit \(V_1\in\V{a}\) tel que \(\quantifs{\forall x\in A\inter V_1}\abs{f\paren{x}-l}\leq\dfrac{\epsilon}{2}\).

Soit \(V_2\in\V{a}\) tel que \(\quantifs{\forall x\in A\inter V_2}\abs{g\paren{x}-l\prim}\leq\dfrac{\epsilon}{2}\).

Posons \(V=V_1\inter V_2\in\V{a}\).

On a \[\begin{aligned}
\quantifs{\forall x\in A\inter V}\abs{f\paren{x}+g\paren{x}-l-l\prim}&\leq\abs{f\paren{x}-l}+\abs{g\paren{x}-l\prim} \\
&\leq\dfrac{\epsilon}{2}+\dfrac{\epsilon}{2} \\
&=\epsilon
\end{aligned}\]

Donc \(V\) convient.

Donc \(\lim_a\paren{f+g}=l+l\prim\).

Si \(l\in\accol{\minf;\pinf}\) ou \(l\prim\in\accol{\minf;\pinf}\), alors \(\lim_a\paren{f+g}=l+l\prim\) découle de la \thref{prop:limiteProduitOuSommeFonctionMinoréeMajoréeOuBornée}.
\end{dem}

\begin{dem}[Autres opérations]
\note{Exercice}
\end{dem}

\begin{prop}[Composition de limites]\thlabel{prop:compositionLimitesFonctions}
Soient \(A\subset\R\), \(B\subset\R\), \(f:A\to B\), \(g:B\to\R\) et \(a,b,c\in\Rb\).

On suppose \[\lim_{x\to a}f\paren{x}=b\qquad\text{et}\qquad\lim_{y\to b}g\paren{y}=c.\]

Alors \[\lim_{x\to a}g\rond f\paren{x}=c.\]
\end{prop}

\begin{dem}
Montrons que \(\lim_ag\rond f=c\), \cad \[\quantifs{\forall V\in\V{c};\exists W\in\V{a}}g\rond f\paren{A\inter W}\subset V.\]

Soit \(V\in\V{c}\).

Comme \(\lim_bg=c\), il existe \(W_1\in\V{b}\) tel que \(g\paren{B\inter W_1}\subset V\).

Comme \(\lim_af=b\), il existe \(W\in\V{a}\) tel que \(f\paren{A\inter W}\subset W_1\).

Ainsi, on a \(f\paren{A\inter W}\subset W_1\inter B\) car \(\Im f\subset B\).

Donc \(g\paren{f\paren{A\inter W}}\subset V\).

\Cad \(g\rond f\paren{A\inter W}\subset V\).

Donc \(W\) convient.

D'où \(\lim_{x\to a}g\rond f\paren{x}=c\).
\end{dem}

\begin{rem}
On peut également composer des limites par valeurs supérieures avec des limites à droite ou des limites par valeurs inférieures avec des limites à gauche. Voir la proposition suivante, par exemple.
\end{rem}

\begin{prop}
Soient \(A\subset\R\), \(B\subset\R\), \(f:A\to B\), \(g:B\to\R\), \(a,c\in\Rb\) et \(b\in\R\).

On suppose \[\lim_{x\to a}f\paren{x}=b^+\qquad\text{et}\qquad\lim_{y\to b^+}g\paren{y}=c.\]

Alors \[\lim_{x\to a}g\rond f\paren{x}=c.\]
\end{prop}

\begin{cor}[Passage à la limite dans une inégalité, version 2]
\textit{Version 1 : \thref{cor:passageALaLimiteDansUneInegalitéFonctionsV1}.}

Soient \(A\subset\R\), \(f,g\in\F{A}{\R}\) et \(a\in\Rb\).

On suppose que \(f\) et \(g\) admettent chacune une limite en \(a\) et qu'on a \(f\leq g\) au voisinage de \(a\), \cad : \[\quantifs{\exists V\in\V{a};\forall x\in A\inter V}f\paren{x}\leq g\paren{x}.\]

Alors \[\lim_{x\to a}f\paren{x}\leq\lim_{x\to a}g\paren{x}.\]
\end{cor}

\begin{dem}
Si \(\lim_af\) et \(\lim_ag\) sont finies :

On a \(f\leq g\) au voisinage de \(a\).

Donc \(0\leq g-f\).

Donc \(0\leq\lim_ag-\lim_af\) selon le \thref{cor:passageALaLimiteDansUneInegalitéFonctionsV1}.

Donc \(\lim_af\leq\lim_ag\).

Si \(\lim_af=\pinf\) alors \(\lim_ag=\pinf\) par le théorème des gendarmes.

Si \(\lim_af=\minf\) alors \(\lim_af\leq\lim_ag\).

Idem pour \(g\).
\end{dem}

\subsection{Caractérisation séquentielle de la limite}

\begin{prop}\thlabel{prop:caractérisationSéquentielleDeLaLimite}
Soient \(A\subset\R\), \(f:A\to\R\) et \(a,l\in\Rb\).

On suppose que tout voisinage de \(a\) rencontre \(A\).

On a : \[\lim_{x\to a}f\paren{x}=l\ssi\croch{\quantifs{\forall\paren{x_n}_{n\in\N}\in A^\N}\lim_{n\to\pinf}x_n=a\imp\lim_{n\to\pinf}f\paren{x_n}=l}.\]

Autrement dit : \(f\) tend vers \(l\) en \(a\) si, et seulement si, pour toute suite \(\paren{x_n}_{n\in\N}\) d'éléments de \(A\) qui tend vers \(a\), la suite \(\paren{f\paren{x_n}}_{n\in\N}\) tend vers \(l\).
\end{prop}

\begin{dem}
\impdir

Supposons \(\lim_af=l\).

Soit \(\paren{x_n}_n\in A^\N\) de limite \(a\).

On a \(\begin{dcases}\lim_{n\to\pinf}x_n=a \\ \lim_{x\to a}f\paren{x}=l\end{dcases}\) donc \[\lim_{n\to\pinf}f\paren{x_n}=l.\]

\imprec

Par contraposée :

Supposons que \(f\) ne tende pas vers \(l\) en \(a\).

On a \[\non\croch{\quantifs{\forall V\in\V{l};\exists W\in\V{a};\forall x\in A\inter W}f\paren{x}\in V}.\]

\Cad \[\quantifs{\exists V\in\V{l};\forall W\in\V{a};\exists x\in A\inter W}f\paren{x}\not\in V.\]

Soit \(V\) un tel voisinage.

Supposons \(a\in\R\).

On a \[\quantifs{\forall n\in\Ns;\exists x_n\in A\inter\intervee{a-\dfrac{1}{n}}{a+\dfrac{1}{n}}}f\paren{x_n}\not\in V.\]

En considérant pour tout \(n\in\Ns\) un tel \(x_n\), on obtient une suite \(\paren{x_n}_{n\in\Ns}\in A^{\Ns}\) telle que \[\quantifs{\forall n\in\Ns}\begin{dcases}f\paren{x_n}\not\in V \\ \abs{x_n-a}\leq\dfrac{1}{n}\end{dcases}\]

On en déduit \(\lim_nx_n=a\) par le théorème des gendarmes.

Enfin, \(\paren{f\paren{x_n}}_n\) ne tend pas vers \(l\) car \(\quantifs{\forall n\in\Ns}f\paren{x_n}\not\in V\).

Supposons \(a=\pinf\).

On prend cette fois \(W=\intervie{n}{\pinf}\) et on conclut de la même façon.

Si \(a=\minf\), idem avec \(W=\intervei{\minf}{n}\).
\end{dem}

\begin{ex}
Montrons que \(\sin\) n'admet pas de limite en \(\pinf\).

Par l'absurde, supposons que \(\sin\) admet une limite \(l\in\Rb\) en \(\pinf\).

En prenant \(\quantifs{\forall n\in\N}x_n=n\pi\), on obtient une suite \(\paren{x_n}_n\in\R^\N\) qui tend vers \(\pinf\).

On a donc \(\lim_n\sin x_n=l\).

Or \(\quantifs{\forall n\in\N}\sin x_n=0\).

Donc \(l=0\).

En prenant \(\quantifs{\forall n\in\N}y_n=\dfrac{\pi}{2}+2n\pi\), on obtient une suite \(\paren{y_n}_n\in\R^\N\) qui tend vers \(\pinf\).

On a donc \(\lim_n\sin y_n=l\).

Or \(\quantifs{\forall n\in\N}\sin y_n=1\).

Donc \(l=1\) : contradiction.

Donc \(\sin\) n'admet pas de limite en \(\pinf\).
\end{ex}

\subsection{Théorème de la limite monotone}

\begin{theo}[Théorème de la limite monotone en \(\pinf\)]
Soient \(A\subset\R\) et \(f:A\to\R\).

On suppose que \(f\) est monotone et que \(A\) est une partie de \(\R\) non-majorée.

Alors \(f\) admet une limite en \(\pinf\) : \[\lim_{x\to\pinf}f\paren{x}=\begin{dcases}
\pinf &\text{si \(f\) est croissante et non-majorée} \\
\sup_Af &\text{si \(f\) est croissante et majorée} \\
\minf &\text{si \(f\) est décroissante et non-minorée} \\
\inf_Af &\text{si \(f\) est décroissante et minorée}
\end{dcases}\]
\end{theo}

\begin{dem}
Supposons \(f\) croissante.

Si \(f\) est non-majorée :

Montrons que \(\lim_{\pinf} f=\pinf\), \cad \[\quantifs{\forall\alpha\in\R;\exists\beta\in\R;\forall x\in A}x\geq\beta\imp f\paren{x}\geq\alpha.\]

Soit \(\alpha\in\R\).

Comme \(f\) n'est pas majorée, il existe \(\beta\in A\) tel que \(f\paren{\beta}>\alpha\).

Comme \(f\) est croissante, on a \[\quantifs{\forall x\in A}x\geq\beta\imp f\paren{x}\geq f\paren{\beta}\geq\alpha.\]

Donc \(\beta\) convient et \(\lim_{\pinf}f=\pinf\).

Si \(f\) est majorée :

La partie \(\Im f\) de \(\R\) est non-vide car \(A\) est non-vide (car \(A\) est non-majorée) et majorée car \(f\) est majorée.

Donc \(f\) admet une borne supérieure \(\lambda\in\R\).

Montrons que \(\lim_{\pinf}f=\lambda\), \cad \[\quantifs{\forall\epsilon\in\Rps;\exists\beta\in\R;\forall x\in A}x\geq\beta\imp\abs{f\paren{x}-\lambda}\leq\epsilon.\]

Rappel : \(\lambda\) est caractérisée par \(\begin{dcases}\quantifs{\forall x\in A}f\paren{x}\leq\lambda \\ \quantifs{\forall\epsilon\in\Rps;\exists x\in A}\lambda-\epsilon<f\paren{x}\end{dcases}\)

Soit \(\epsilon\in\Rps\).

D'après la caractérisation de \(\lambda\), il existe un point \(\beta\in A\) tel que \(\lambda-\epsilon<f\paren{\beta}\).

On a \[\quantifs{\forall x\in A}x\geq\beta\imp\lambda-\epsilon<f\paren{\beta}\leq f\paren{x}\leq\lambda.\]

Donc \[\quantifs{\forall x\in A}x\geq\beta\imp\lambda-\epsilon\leq f\paren{x}\leq\lambda+\epsilon.\]

Donc \[\quantifs{\forall x\in A}x\geq\beta\imp\abs{f\paren{x}-\lambda}\leq\epsilon.\]

D'où \(\lim_{\pinf}f=\lambda\).

Idem si \(f\) est décroissante.
\end{dem}

\begin{theo}[Théorème de la limite monotone en \(\minf\)]
Soient \(A\subset\R\) et \(f:A\to\R\).

On suppose que \(f\) est monotone et que \(A\) est une partie de \(\R\) non-minorée.

Alors \(f\) admet une limite en \(\minf\) : \[\lim_{x\to\minf}f\paren{x}=\begin{dcases}
\minf &\text{si \(f\) est croissante et non-minorée} \\
\inf_Af &\text{si \(f\) est croissante et minorée} \\
\pinf &\text{si \(f\) est décroissante et non-majorée} \\
\sup_Af &\text{si \(f\) est décroissante et majorée}
\end{dcases}\]
\end{theo}

\begin{theo}[Théorème de la limite monotone en à gauche d'un réel]
Soient \(A\subset\R\), \(f:A\to\R\) et \(a\in\R\).

On suppose que \(f\) est monotone et que tout voisinage de \(a\) rencontre l'ensemble \(B=A\inter\intervee{\minf}{a}\).

Alors \(f\) admet une limite à gauche en \(a\) : \[\lim_{x\to a^-}f\paren{x}=\begin{dcases}
\pinf &\text{si \(f\) est croissante et non-majorée sur \(B\)} \\
\sup_Bf &\text{si \(f\) est croissante et majorée sur \(B\)} \\
\minf &\text{si \(f\) est décroissante et non-minorée sur \(B\)} \\
\inf_Bf &\text{si \(f\) est décroissante et minorée sur \(B\)}
\end{dcases}\]
\end{theo}

\begin{dem}[Si \(f\) est croissante et non-majorée sur \(B\)]
Comme \(f\) n'est pas majorée sur \(B\), on a \[\quantifs{\forall\alpha\in\R;\exists b\in B}f\paren{b}>\alpha.\]

Donc, comme \(f\) est croissante, on a \[\quantifs{\forall\alpha\in\R;\exists b\in B;\forall x\in A\inter\intervie{b}{a}}f\paren{x}>\alpha.\]

Donc \[\quantifs{\forall\alpha\in\R;\exists\delta\in\Rps;\forall x\in A\inter\intervie{a-\delta}{a}}f\paren{x}>\alpha\] en prenant \(\delta=a-b\).

Finalement : \[\quantifs{\forall\alpha\in\R;\exists\delta\in\Rps;\forall x\in B}\abs{x-a}\leq\delta\imp f\paren{x}\geq\alpha.\]

Donc \(\lim_{a^-}f=\pinf\).
\end{dem}

\begin{dem}[Autres cas]
\note{Exercice}
\end{dem}

\begin{theo}[Théorème de la limite monotone en à droite d'un réel]
Soient \(A\subset\R\), \(f:A\to\R\) et \(a\in\R\).

On suppose que \(f\) est monotone et que tout voisinage de \(a\) rencontre l'ensemble \(B=A\inter\intervee{a}{\pinf}\).

Alors \(f\) admet une limite à droite en \(a\) : \[\lim_{x\to a^+}f\paren{x}=\begin{dcases}
\minf &\text{si \(f\) est croissante et non-minorée sur \(B\)} \\
\inf_Bf &\text{si \(f\) est croissante et minorée sur \(B\)} \\
\pinf &\text{si \(f\) est décroissante et non-majorée sur \(B\)} \\
\sup_Bf &\text{si \(f\) est décroissante et majorée sur \(B\)}
\end{dcases}\]
\end{theo}

\begin{bilan}
Une fonction monotone admet une limite à droite et une limite à gauche partout où son ensemble de définition le permet.
\end{bilan}

\begin{ex}
Étudions les limites de la fonction \guillemets{partie entière} : \[\fonction{f}{\R}{\R}{x}{\floor{x}}\]

Comme \(f\) est croissante sur \(\R\), selon le théorème de la limite monotone, elle admet une limite en \(\pinf\), en \(\minf\) et à droite et à gauche en tout point \(a\in\R\).

On a \(\lim_{\pinf}f=\pinf\).

En effet, cette limite existe et en posant \(\quantifs{\forall n\in\N}x_n=n\), on obtient une suite \(\paren{x_n}_n\) qui tend vers \(\pinf\) et on a \(\lim_{\pinf}f=\lim_nf\paren{x_n}=\pinf\).

On a de même \(\lim_{\minf}f=\minf\) en prenant \(\quantifs{\forall n\in\N}x_n=-n\).

Soit \(a\in\R\).

On a \[\begin{dcases}\lim_{a^+}f=\floor{a} \\ \lim_{a^-}f=\begin{dcases}a-1 &\text{si \(a\in\Z\)} \\ \floor{a} &\text{sinon}\end{dcases}\end{dcases}\]

En effet :

Si \(a\not\in\Z\), on a \[\quantifs{\forall x\in\underbrace{\intervie{\floor{a}}{\floor{a}+1}}_{\text{voisinage de \(a\)}}}f\paren{x}=\floor{a}.\]

Donc au voisinage de \(a\), \(f\) est constante et égale à \(\floor{a}\).

Donc \(\lim_{a^+}f=\lim_{a^-}f=\floor{a}\).

Sinon, on a \[\quantifs{\forall x\in\intervie{a}{a+1}}f\paren{x}=a\] donc \(\lim_{a^+}f=a=\floor{a}\) et \[\quantifs{\forall x\in\intervie{a-1}{a}}f\paren{x}=a-1\] donc \(\lim_{a^-}f=a-1\).
\end{ex}

\section{Continuité}

\subsection{Fonctions continues en un point}

\subsubsection{Définitions}

\begin{defi}[Continuité en un point]
Soient \(A\subset\R\), \(f:A\to\R\) et \(a\in A\).

On suppose que \(f\) est définie en \(a\).

On dit que \(f\) est continue en \(a\) si \(f\) admet une limite en \(a\), \cad, selon la \thref{rem:limiteEnAEgaleFDeA} si \[\lim_{x\to a}f\paren{x}=f\paren{a}.\]
\end{defi}

\begin{rem}
Pour qu'une fonction soit continue en un point, il faut qu'elle soit définie en ce point.

Si une fonction est continue en un point, elle est bornée au voisinage de ce point (selon le \thref{cor:limiteFinieDoncFonctionBornée}).
\end{rem}

\begin{rem}\thlabel{rem:deuxFonctionsQuiCoincidentAuVoisinageD'unPointSontContinuesEnCePointSsiL'autreAussi}
Soient \(A\subset\R\), \(f,g\in\F{A}{\R}\) et \(a\in A\).

Si \(f\) et \(g\) coïncident au voisinage de \(a\), alors \(f\) est continue en \(a\) si, et seulement si, \(g\) est continue en \(a\).
\end{rem}

\begin{dem}
Découle de la \thref{rem:fonctionsQuiCoincidentSsiLimitesEgales}.
\end{dem}

\begin{ex}
Montrons que la fonction \(\sin:\R\to\R\) est continue en \(0\).

On a \[\sin\text{ continu en }0\ssi\lim_{x\to0}\sin x=0.\]

Or on a vu \(\quantifs{\forall x\in\R}\abs{\sin x}\leq\abs{x}\) donc d'après le théorème des gendarmes : \[\lim_{x\to0}\sin x=0.\]

Donc \(\sin\) est continu en \(0\).
\end{ex}

\begin{defi}[Continuité à droite d'un point]
Soient \(A\subset\R\), \(f:A\to\R\) et \(a\in A\).

On dit que \(f\) est continue à droite de \(a\) si on a \[\lim_{x\to a^+}f\paren{x}=f\paren{a}.\]
\end{defi}

\begin{rem}[Reformulation]
Soient \(A\subset\R\), \(f:A\to\R\) et \(a\in A\).

Alors \(f\) est continue à droite en \(a\) si, et seulement si, on a :

\begin{enumerate}
\item Tout voisinage de \(a\) rencontre l'ensemble \(A\inter\intervee{a}{\pinf}\). \\

\item La restriction \(\restr{f}{A\inter\intervee{a}{\pinf}}\) est continue en \(a\).
\end{enumerate}
\end{rem}

\begin{defi}[Continuité à gauche d'un point]
Soient \(A\subset\R\), \(f:A\to\R\) et \(a\in A\).

On dit que \(f\) est continue à droite de \(a\) si on a \[\lim_{x\to a^-}f\paren{x}=f\paren{a}.\]
\end{defi}

\begin{rem}[Reformulation]
Soient \(A\subset\R\), \(f:A\to\R\) et \(a\in A\).

Alors \(f\) est continue à gauche en \(a\) si, et seulement si, on a :

\begin{enumerate}
\item Tout voisinage de \(a\) rencontre l'ensemble \(A\inter\intervee{\minf}{a}\). \\

\item La restriction \(\restr{f}{A\inter\intervee{\minf}{a}}\) est continue en \(a\).
\end{enumerate}
\end{rem}

\begin{ex}
La fonction \guillemets{partie entière} : \(\fonctionlambda{\R}{\R}{x}{\floor{x}}\) est \[\begin{dcases}\text{continue en tout point }a\in\R\excluant\Z \\ \text{continue à droite en tout point} \\ \text{non-continue à gauche}\end{dcases}\]
\end{ex}

\begin{defi}[Prolongement par continuité en un point]
Soient \(A\subset\R\) et \(f:A\to\R\).

Soit \(a\in\R\excluant A\) tel que tout voisinage de \(a\) rencontre \(A\) : \[\quantifs{\forall V\in\V{a}}V\inter A\not=\ensvide.\]

On appelle prolongement par continuité de \(f\) en \(a\) toute fonction \(g:A\union\accol{a}\to\R\) vérifiant : \[\begin{dcases}\restr{g}{A}=f \\ g\text{ est continue en }a\end{dcases}\]
\end{defi}

\begin{prop}
Soient \(A\subset\R\) et \(f:A\to\R\).

Soit \(a\in\R\excluant A\) tel que tout voisinage de \(a\) rencontre \(A\).

Le prolongement par continuité de \(f\) en \(a\) est unique.

Il existe si, et seulement si, la fonction \(f\) admet une limite finie \(l\) en \(a\).

L'unique prolongement par continuité de \(f\) en \(a\) est alors : \[\fonction{g}{A\union\accol{a}}{\R}{x}{\begin{dcases}f\paren{x} &\text{si }x\in A \\ l &\text{si }x=a\end{dcases}}\]
\end{prop}

\begin{dem}
\unicite

Soit \(g:A\union\accol{a}\to\R\) telle que \(\begin{dcases}\restr{g}{A}=f \\ g\text{ continue en }a\end{dcases}\)

On a \[\begin{dcases}\quantifs{\forall x\in A}g\paren{x}=f\paren{x} \\ g\paren{a}=\lim_{x\to a}g\paren{x}=\lim_{x\to a}f\paren{x}\end{dcases}\]

Donc \(g\) est unique et pour que \(g\) existe, il faut que \(f\) admette une limite finie en \(a\).

\existence

Montrons que \[f\text{ prolongeable par continuité}\ssi f\text{ admet une limite finie en }a.\]

\impdir Déjà vu.

\imprec

Supposons \(\lim_af=l\in\R\).

Posons \(\fonction{g}{A\union\accol{a}}{\R}{x}{\begin{dcases}f\paren{x} &\text{si }x\in A \\ l &\text{si }x=a\end{dcases}}\)

On a \[\begin{dcases}\restr{g}{A}=f \\ \lim_{x\to a}g\paren{x}=l=g\paren{a}\end{dcases}\]

Donc \(f\) est prolongeable par continuité, de prolongement \(g\).
\end{dem}

\begin{ex}
Montrons que les fonctions \[\fonction{f}{\Rs}{\R}{x}{x\sin\dfrac{1}{x}}\qquad\text{et}\qquad\fonction{g}{\Rs}{\R}{x}{\sin\dfrac{1}{x}}\] sont respectivement prolongeable par continuité en \(0\) et non-prolongeable par continuité en \(0\).

On remarque \(\quantifs{\forall x\in\Rs}\abs{x\sin\dfrac{1}{x}}\leq\abs{x}\).

Or \(\lim_{x\to0}\abs{x}=0\).

Donc d'après le théorème des gendarmes, \(\lim_{x\to0}x\sin\dfrac{1}{x}=0\in\R\).

Donc \(f\) est prolongeable par continuité en \(0\).

Montrons que \(g\) n'admet aucune limite finie en \(0\).

Supposons par l'absurde \(\lim_{x\to0}g\paren{x}=l\in\R\).

Posons \[\quantifs{\forall n\in\Ns}\begin{dcases}x_n=\dfrac{1}{n\pi} \\ y_n=\dfrac{1}{2n\pi+\nicefrac{\pi}{2}}\end{dcases}\]

On a \(\lim_nx_n=\lim_ny_n=0\) donc \[\lim_ng\paren{x_n}=\lim_ng\paren{y_n}=l.\]

Or on a \(\begin{dcases}\lim_ng\paren{x_n}=0 \\ \lim_ng\paren{y_n}=1\end{dcases}\) donc \(0=1\) : contradiction.

Donc \(g\) n'admet aucune limite finie en \(0\).

Donc \(g\) n'est pas prolongeable par continuité en \(0\).
\end{ex}

\subsubsection{Propriétés}

\begin{prop}[Opérations algébriques]\thlabel{prop:opérationsAlgébriquesSurLesFonctionsContinuesEnUnPoint}
Soient \(A\subset\R\), \(a\in A\), \(\lambda,\mu\in\R\) et \(f,g\in\F{A}{\R}\) continues en \(a\).

La fonction \(f+g\) est continue en \(a\).

La fonction \(\lambda f+\mu g\) est continue en \(a\).

La fonction \(fg\) est continue en \(a\).

Si \(f\paren{a}\not=0\) alors la fonction \(\dfrac{1}{f}\) est continue en \(a\).

Si \(g\paren{a}\not=0\) alors la fonction \(\dfrac{f}{g}\) est continue en \(a\).
\end{prop}

\begin{dem}
Découle de la \thref{prop:opérationsAlgébriquesSurLesLimitesFonctions}.
\end{dem}

\begin{prop}[Composition]\thlabel{prop:compositionDeFonctionsContinuesEnUnPoint}
Soient \(A\subset\R\), \(B\subset\R\), \(f:A\to\R\), \(g:B\to\R\) et \(a\in A\).

On suppose que \(f\) est continue en \(a\) et que \(g\) est continue en \(f\paren{a}\).

Alors \(g\rond f\) est continue en \(a\).
\end{prop}

\begin{dem}
Découle de la \thref{prop:compositionLimitesFonctions}.
\end{dem}

\subsubsection{Caractérisation séquentielle}

\begin{prop}[Caractérisation séquentielle de la continuité en un point]\thlabel{prop:caractérisationSéquentielleDeLaContinuitéEnUnPoint}
Soient \(A\subset\R\), \(f:A\to\R\) et \(a\in A\).

On a : \[f\text{ continue en }a\ssi\croch{\quantifs{\forall\paren{x_n}_{n\in\N}\in A^\N}\lim_{n\to\pinf}x_n=a\imp\lim_{n\to\pinf}f\paren{x_n}=f\paren{a}}.\]

Autrement dit : \(f\) est continue en \(a\) si, et seulement si, pour toute suite \(\paren{x_n}_{n\in\N}\) d'éléments de \(A\) qui tend vers \(a\), la suite \(\paren{f\paren{x_n}}_{n\in\N}\) tend vers \(f\paren{a}\).
\end{prop}

\begin{dem}
Découle de la \thref{prop:caractérisationSéquentielleDeLaLimite}.
\end{dem}

\subsection{Fonctions continues}

\subsubsection{Définition}

\begin{defi}[Continuité]
Soient \(A\subset\R\) et \(f:A\to\R\).

On dit que \(f\) est continue si \(f\) est continue en tout point de \(A\) : \[\quantifs{\forall a\in A}\lim_{x\to a}f\paren{x}=f\paren{a}.\]
\end{defi}

\begin{ex}
Les fonctions \[\fonctionlambda{\R}{\R}{x}{\floor{x}}\qquad\text{et}\qquad\fonctionlambda{\Rs}{\R}{x}{\dfrac{1}{x}}\] sont continues.
\end{ex}

\subsubsection{Propriétés}

\begin{prop}[Restriction]
Soient \(A\subset\R\), \(B\subset A\) et \(f:A\to\R\).

Alors la restriction \(\restr{f}{B}\) est continue.
\end{prop}

\begin{prop}[Opérations algébriques]
Soient \(A\subset\R\), \(f,g\in\F{A}{\R}\) continues et \(\lambda,\mu\in\R\).

La fonction \(f+g\) est continue.

La fonction \(\lambda f+\mu g\) est continue.

La fonction \(fg\) est continue.

Si \(\quantifs{\forall a\in A}f\paren{a}\not=0\) alors la fonction \(\dfrac{1}{f}\) est continue.

Si \(\quantifs{\forall a\in A}g\paren{a}\not=0\) alors la fonction \(\dfrac{f}{g}\) est continue.
\end{prop}

\begin{dem}
Découle de la \thref{prop:opérationsAlgébriquesSurLesFonctionsContinuesEnUnPoint}.
\end{dem}

\begin{prop}[Composition]
Soient \(A\subset\R\), \(B\subset\R\), \(f:A\to\R\) et \(g:B\to\R\).

On suppose que \(f\) est continue (sur \(A\)) et que \(g\) est continue (sur \(B\)).

Alors \(g\rond f\) est continue (sur \(A\)).
\end{prop}

\begin{dem}
Découle de la \thref{prop:compositionDeFonctionsContinuesEnUnPoint}.
\end{dem}

\subsubsection{Caractérisation séquentielle}

\begin{prop}[Caractérisation séquentielle de la continuité]
Soient \(A\subset\R\) et \(f:A\to\R\).

On a : \[f\text{ est continue}\ssi\croch{\quantifs{\forall\paren{x_n}_{n\in\N}\in A^\N}\lim_{n\to\pinf}x_n\in A\imp\lim_{n\to\pinf}f\paren{x_n}=f\paren{\lim_{n\to\pinf}x_n}}.\]

Autrement dit : \(f\) est continue si, et seulement si, pour toute suite convergente \(\paren{x_n}_{n\in\N}\) d'éléments de \(A\) dont la limite \(l\) appartient à \(A\), la suite \(\paren{f\paren{x_n}}_{n\in\N}\) converge vers \(f\paren{l}\).
\end{prop}

\begin{dem}
Découle de la \thref{prop:caractérisationSéquentielleDeLaContinuitéEnUnPoint}.
\end{dem}

\subsubsection{Exemples usuels}

\begin{prop}
Les fonctions \[\exp:\R\to\R\quad\ln:\Rps\to\R\quad\sin:\R\to\R\quad\cos:\R\to\R\quad\tan:\bigunion_{k\in\Z}\intervee{-\dfrac{\pi}{2}+k\pi}{\dfrac{\pi}{2}+k\pi}\to\R\] et, pour tout \(\alpha\in\R\), \[\fonctionlambda{\Rps}{\R}{x}{x^\alpha}\] sont continues.
\end{prop}

\begin{dem}
\note{Admis}
\end{dem}

\subsubsection{Remarque}

\begin{rem}
Soient \(A\subset\R\), \(B\subset\R\) et \(f:A\to\R\).

Ne pas confondre :

\begin{enumerate}
\item La restriction \(\restr{f}{B}\) est continue. \\

\item La fonction \(f\) est continue en tout point de \(B\).
\end{enumerate}

On a (1) \(\imp\) (2) mais l'implication réciproque est fausse en général.

Cela dit, dans le cas particulier où \(B\) est un voisinage d'un réel \(x\), la continuité de \(\restr{f}{B}\) implique la continuité de \(f\) en \(x\) (selon la \thref{rem:deuxFonctionsQuiCoincidentAuVoisinageD'unPointSontContinuesEnCePointSsiL'autreAussi}).
\end{rem}

\begin{ex}
La fonction \guillemets{partie entière} : \[\fonction{f}{\R}{\R}{x}{\floor{x}}.\]

\(\restr{f}{\intervie{0}{1}}\) est continue mais la proposition \[\quantifs{\forall x\in\intervie{0}{1}}f\text{ est continue en }x\] est fausse.
\end{ex}

\subsection{Fonctions continues sur un intervalle}

\subsubsection{Intervalles}

\begin{rappel}
On appelle intervalle de \(\R\) toute partie \(I\subset\R\) telle que : \[\quantifs{\forall a,b\in I;\forall x\in\R}a<x<b\imp x\in I.\]
\end{rappel}

\begin{theo}[Description des intervalles de \(\R\)]\thlabel{theo:descriptionDesIntervallesDeR}
Les intervalles de \(\R\) sont les parties de \(\R\) de la forme :

\begin{itemize}
\item \(\intervii{a}{b}=\accol{x\in\R\tq a\leq x\leq b}\) avec \(a,b\in\R\) \\

\item \(\intervie{a}{b}=\accol{x\in\R\tq a\leq x<b}\) avec \(a\in\R\) et \(b\in\R\union\accol{\pinf}\). \\

\item \(\intervei{a}{b}=\accol{x\in\R\tq a<x\leq b}\) avec \(a\in\R\union\accol{\minf}\) et \(b\in\R\). \\

\item \(\intervee{a}{b}=\accol{x\in\R\tq a<x<b}\) avec \(a\in\R\union\accol{\minf}\) et \(b\in\R\union\accol{\pinf}\).
\end{itemize}

NB : l'ensemble vide est bien un intervalle, on l'obtient en prenant \(a>b\).
\end{theo}

\begin{dem}
\note{Exercice} (\cf \thref{exo:7.17}).
\end{dem}

\begin{defi}[Segment]
On appelle segment tout intervalle de la forme \(\intervii{a}{b}\) où \(a,b\in\R\) tels que \(a\leq b\).
\end{defi}

\begin{rem}
L'ensemble vide est un intervalle mais pas un segment.
\end{rem}

\subsubsection{Théorème des valeurs intermédiaires}

\begin{theo}[Théorème des valeurs intermédiaires]
Soient \(I\) un intervalle de \(\R\), \(f:I\to\R\) continue et \(x_1,x_2\in I\).

Tout réel compris entre \(f\paren{x_1}\) et \(f\paren{x_2}\) admet un antécédent par \(f\) : \[\quantifs{\forall y\in\R}f\paren{x_1}<y<f\paren{x_2}\imp\croch{\quantifs{\exists x\in I}f\paren{x}=y}.\]
\end{theo}

\begin{dem}~\\
On pose \(\begin{dcases}a_0=\min\accol{x_1;x_2} \\ b_0=\max\accol{x_1;x_2}\end{dcases}\)

Quitte à remplacer \(f\) par \(-f\), on suppose \(f\paren{a_0}<f\paren{b_0}\).

Soit \(y\in\intervee{f\paren{a_0}}{f\paren{b_0}}\).

Montrons que \(y\in\Im f\).

On construit par récurrence la suite croissante \(\paren{a_n}_n\) et la suite décroissante \(\paren{b_n}_n\) telles que \[\quantifs{\forall n\in\N}\begin{dcases}f\paren{a_n}\leq y\leq f\paren{b_n} \\ b_n-a_n=\dfrac{b_0-a_0}{2^n}\end{dcases}\]

Soit \(n\in\N\) tel que \(\begin{dcases}f\paren{a_n}\leq y\leq f\paren{b_n} \\ b_n-a_n=\dfrac{b_0-a_0}{2^n}\end{dcases}\)

Si \(f\paren{\dfrac{b_n+a_n}{2}}\leq y\) on pose \(\begin{dcases}a_{n+1}=\dfrac{a_n+b_n}{2} \\ b_{n+1}=b_n\end{dcases}\)

Sinon, on pose \(\begin{dcases}a_{n+1}=a_n \\ b_{n+1}=\dfrac{a_n+b_n}{2}\end{dcases}\)

D'où les deux suites.

\(\paren{a_n}_n\) et \(\paren{b_n}_n\) sont adjacentes donc convergentes et de même limite \(l\).

De plus, on a \(a_0\leq l\leq b_0\) et \(a_0,b_0\in I\) donc \(l\in I\).

On a \(\begin{dcases}\lim_na_n=\lim_nb_n=l\in I \\ f\text{ continue en }l\end{dcases}\) donc \[\lim_nf\paren{a_n}=\lim_nf\paren{b_n}=f\paren{l}.\]

Enfin, on a \(\quantifs{\forall n\in\N}f\paren{a_n}\leq y\leq f\paren{b_n}\) donc par passage à la limite : \[f\paren{l}\leq y\leq f\paren{l}.\]

Donc \(y=f\paren{l}\).

Donc \(y\in\Im f\).
\end{dem}

\begin{cor}[L'image continue d'un intervalle est un intervalle]
Soient \(I\) un intervalle de \(\R\) et \(f:I\to\R\) continue.

Alors l'image directe \(f\paren{I}\) est un intervalle de \(\R\).
\end{cor}

\begin{dem}
Soient \(y_1,y_2\in f\paren{I}\) et \(y\in\R\).

Supposons \(y_1\leq y\leq y_2\).

Montrons que \(y\in f\paren{I}\).

Soient \(x_1,x_2\in I\) tels que \(\begin{dcases}y_1=f\paren{x_1} \\ y_2=f\paren{x_2}\end{dcases}\)

On a \(f\paren{x_1}\leq y\leq f\paren{x_2}\), \(f\) est continue et \(I\) est un intervalle donc selon le théorèmes des valeurs intermédiaires, il existe \(x\in I\) tel que \[f\paren{x}=y.\]

Donc \(y\in f\paren{I}\).
\end{dem}

\begin{rem}
Le corollaire précédent est lui aussi appelé \guillemets{théorèmes des valeurs intermédiaires}.
\end{rem}

\begin{ex}
Montrons par des exemples que :

\begin{itemize}
\item l'image d'un intervalle borné par une fonction continue peut être un intervalle non-borné ; \\

\item l'image d'un intervalle non-borné par une fonction continue peut être un intervalle borné ; \\

\item l'image d'un intervalle ouvert par une fonction continue peut être un segment.
\end{itemize}

Soit \(f:x\mapsto\dfrac{1}{x}\). On a \(f\paren{\intervei{0}{1}}=\intervie{1}{\pinf}\).

On a \(\sin\paren{\R}=\intervii{-1}{1}\).

On a \(\sin\paren{\intervee{0}{4\pi}}=\intervii{-1}{1}\).
\end{ex}

\begin{cor}[TVI appliqué aux fonctions strictement monotones]\thlabel{cor:TVIAppliquéAuxFonctionsStrictementMonotones}
Soient \(a,b\in\Rb\) tels que \(a<b\), \(I\) un intervalle de \(\R\) de bornes \(a\) et \(b\) et \(f:I\to\R\) continue et strictement monotone.

Alors \(f\) est bijective de l'intervalle \(I\) vers l'intervalle \(f\paren{I}\).

On a :

Si \(f\) est croissante et \(I=\intervii{a}{b}\) alors \(f\paren{I}=\intervii{f\paren{a}}{f\paren{b}}\).

Si \(f\) est décroissante et \(I=\intervii{a}{b}\) alors \(f\paren{I}=\intervii{f\paren{b}}{f\paren{a}}\).

Si \(f\) est croissante et \(I=\intervie{a}{b}\) alors \(f\paren{I}=\intervie{f\paren{a}}{\lim_bf}\).

Si \(f\) est décroissante et \(I=\intervie{a}{b}\) alors \(f\paren{I}=\intervei{\lim_bf}{f\paren{a}}\).

Si \(f\) est croissante et \(I=\intervei{a}{b}\) alors \(f\paren{I}=\intervei{\lim_af}{f\paren{b}}\).

Si \(f\) est décroissante et \(I=\intervei{a}{b}\) alors \(f\paren{I}=\intervie{f\paren{b}}{\lim_af}\).

Si \(f\) est croissante et \(I=\intervee{a}{b}\) alors \(f\paren{I}=\intervee{\lim_af}{\lim_bf}\).

Si \(f\) est décroissante et \(I=\intervee{a}{b}\) alors \(f\paren{I}=\intervee{\lim_bf}{\lim_af}\).

{
\small Pour ne pas allonger inutilement l'énoncé, on a supposé implicitement :

\begin{itemize}
\item que \(a\) n'est pas égal à \(\pinf\) (car \(a<b\)), ni à \(\minf\) si \(a\in I\) (car \(I\subset\R\)) ; \\

\item que \(b\) n'est pas égal à \(\minf\) (car \(a<b\)), ni à \(\pinf\) si \(b\in I\) (car \(I\subset\R\)).
\end{itemize}
}
\end{cor}

\begin{dem}[Cas où \(f\) est croissante et \(I=\intervii{a}{b}\)]
\(f\) est continue et \(I\) est un intervalle donc \(f\paren{I}\) est un intervalle selon le théorème des valeurs intermédiaires.

\(f\) est strictement monotone donc c'est une injection et donc une bijection de \(I\) vers \(f\paren{I}\).

Déterminons \(f\paren{I}\) :

Supposons \(f\) croissante et \(I=\intervii{a}{b}\).

On a \(\quantifs{\forall x\in I}a\leq x\leq b\) donc \(\quantifs{\forall x\in I}f\paren{a}\leq f\paren{x}\leq f\paren{b}\).

Donc \(f\paren{I}=\intervii{f\paren{a}}{f\paren{b}}\).
\end{dem}

\begin{dem}[Cas où \(f\) est croissante et \(I=\intervie{a}{b}\)]
\(f\) est continue et \(I\) est un intervalle donc \(f\paren{I}\) est un intervalle selon le théorème des valeurs intermédiaires.

\(f\) est strictement monotone donc c'est une injection et donc une bijection de \(I\) vers \(f\paren{I}\).

Déterminons \(f\paren{I}\) :

Supposons \(f\) croissante et \(I=\intervie{a}{b}\).

On a \(\quantifs{\forall x\in I}f\paren{a}\leq f\paren{x}\) donc \(\min f\paren{I}=f\paren{a}\).

Supposons \(f\) non-majorée, \cad \(f\paren{I}\) non-majoré.

Comme \(f\) est croissante et non-majorée, d'après le théorème de la limite monotone, on a \(f\paren{I}=\intervie{f\paren{a}}{\pinf}=\intervie{f\paren{a}}{\lim_bf}\).

Supposons \(f\) majorée.

D'après le théorème de la limite monotone, on a \(\lim_bf=\sup_If=\sup f\paren{I}\).

Donc \(f\paren{I}=\intervie{f\paren{a}}{\lim_bf}\) ou \(f\paren{I}=\intervii{f\paren{a}}{\lim_bf}\) car \(f\paren{I}\) est un intervalle, \(\min f\paren{I}=f\paren{a}\) et \(\sup f\paren{I}=\lim_bf\).

Montrons que \(\lim_bf\not\in f\paren{I}\).

Par l'absurde, supposons \(\lim_bf\in f\paren{I}\).

Soit \(c\in I\) tel que \(f\paren{c}=\lim_bf\).

On a \[\quantifs{\forall x\in\intervie{c}{b}}f\paren{c}\leq f\paren{x}\leq\sup_If=f\paren{c}.\]

Donc \(f\) est constante sur \(\intervie{c}{b}\) : contradiction.

Donc \(\lim_bf\not\in f\paren{I}\).

Donc \(f\paren{I}=\intervie{f\paren{a}}{\lim_bf}\).
\end{dem}

\begin{dem}[Autres cas]
\note{Exercice}
\end{dem}

\subsection{Fonctions continues sur un segment}

\begin{theo}[Théorème des bornes atteintes]
Soit \(f:\intervii{a}{b}\to\R\) une fonction continue sur un segment \(\intervii{a}{b}\) de \(\R\).

Alors \(f\) admet un minimum et un maximum.
\end{theo}

\begin{rappel}[\Cf \thref{exo:5.15}]
Soit \(A\subset\R\).

Si \(A\) est non-majorée alors il existe une suite d'éléments de \(A\) qui tend vers \(\pinf\).

Si \(\lambda=\sup A\) alors il existe une suite d'éléments de \(A\) qui tend vers \(\lambda\).
\end{rappel}

\begin{dem}
Montrons que \(f\) est majorée :

Par l'absurde, supposons \(f\) non-majorée, \cad \(\Im f\) non-majoré.

Soit \(\paren{y_n}_n\in\paren{\Im f}^\N\) telle que \(\lim_ny_n=\pinf\).

Soit \(\paren{x_n}_n\in\intervii{a}{b}^\N\) telle que \(\quantifs{\forall n\in\N}f\paren{x_n}=y_n\).

On a \(\lim_nf\paren{x_n}=\pinf\).

Comme \(\paren{x_n}_n\) est bornée, d'après le théorème de Bolzano-Weierstrass, il existe \(\phi:\N\to\N\) strictement croissante telle que \(\paren{x_{\phi\paren{n}}}_n\) soit convergente.

On pose \(l=\lim_nx_{\phi\paren{n}}\).

On a \(\quantifs{\forall n\in\N}a\leq x\leq b\) donc par passage à la limite : \(a\leq l\leq b\). Donc \(l\in\intervii{a}{b}\).

Donc \(f\) est continue en \(l\).

Donc \(f\paren{l}=\lim_nf\paren{x_{\phi\paren{n}}}\).

Or \(\paren{f\paren{x_{\phi\paren{n}}}}_n\) tend vers \(\pinf\) car c'est une suite extraite de \(\paren{f\paren{x_n}}_n\) : contradiction.

Donc \(f\) est majorée.

Donc \(f\paren{\intervii{a}{b}}\) est une partie non-vide et majorée de \(\R\), qui admet donc une borne supérieure.

Donc \(\sup_{\intervii{a}{b}}f\) existe.

Comme précédemment, on construit \(\paren{x_n}_n\in\intervii{a}{b}^\N\) telle que \(\lim_nf\paren{x_n}=\sup_{\intervii{a}{b}}f\).

Selon le théorème de Bolzano-Weierstrass, il existe \(\phi:\N\to\N\) strictement croissante telle que \(\paren{x_{\phi\paren{n}}}_n\) soit convergente vers \(l\in\R\).

Comme précédemment, on montre que \(l\in\intervii{a}{b}\).

On a \(\begin{dcases}\lim_nx_{\phi\paren{n}}=l \\ f\text{ continue en }l\end{dcases}\)

Donc \(\lim_nf\paren{x_{\phi\paren{n}}}=f\paren{l}\).

Donc \(\sup_{\intervii{a}{b}}f=f\paren{l}\).

Donc \(\max_{\intervii{a}{b}}f=f\paren{l}\).

On montre de même le minimum.
\end{dem}

\begin{rem}
On énonce parfois ainsi le théorème des bornes atteintes : \guillemets{toute fonction continue sur un segment est bornée et atteint ses bornes}.
\end{rem}

\begin{cor}[L'image continue d'un segment est un segment]
Soit \(f:\intervii{a}{b}:\to\R\) une fonction continue sur un segment \(\intervii{a}{b}\) de \(\R\).

Alors l'image directe \(f\paren{\intervii{a}{b}}\) est un segment de \(\R\).
\end{cor}

\begin{dem}
\(f\paren{\intervii{a}{b}}\) :

\begin{itemize}
\item est un intervalle d'après le théorème des valeurs intermédiaires car \(f\) est continue et \(\intervii{a}{b}\) est un intervalle ; \\

\item admet un maximum et un minimum d'après le théorème des bornes atteintes car \(f\) est continue et \(\intervii{a}{b}\) est un segment.
\end{itemize}

Donc \(f\paren{\intervii{a}{b}}\) est un segment.
\end{dem}

\subsection{Fonctions continues injectives sur un intervalle}

\begin{theo}\thlabel{theo:fonctionContinueEtInjectiveSurUnIntervalleEstMonotone}
Soient \(I\) un intervalle de \(\R\) et \(f:I\to\R\) continue et injective.

Alors \(f\) est monotone.
\end{theo}

\begin{dem}
\note{Exercice} \Cf \thref{exo:7.31}.
\end{dem}

\begin{cor}[Reformulation du \thref{theo:fonctionContinueEtInjectiveSurUnIntervalleEstMonotone}]
Soient \(I\) un intervalle de \(\R\) et \(f:I\to\R\) continue.

Alors \[f\text{ est strictement monotone}\ssi f\text{ est injective.}\]
\end{cor}

\begin{dem}
\impdir Évident.

\imprec \thref{theo:fonctionContinueEtInjectiveSurUnIntervalleEstMonotone}.
\end{dem}

\begin{theo}\thlabel{theo:bijectionReciproqueDeFonctionContinueEtStrictementMonotoneEstContinueEtStrictementDeMêmeMonotonie}
Soient \(I\) un intervalle de \(\R\) et \(f:I\to\R\) continue et strictement monotone.

Alors \(f\) est une injection et donc une bijection de \(I\) vers \(\Im f\).

Sa bijection réciproque \(f\inv:\Im f\to I\) est continue et strictement monotone.

Précisément : \(f\inv\) est strictement croissante si \(f\) est strictement croissante et strictement décroissante si \(f\) est strictement décroissante.
\end{theo}

\begin{dem}
Montrons que \(f\inv\) est strictement monotone.

Supposons \(f\) strictement croissante.

Soient \(y_1,y_2\in\Im f\) tels que \(y_1<y_2\).

On a \(\begin{dcases}f\paren{f\inv\paren{y_1}}<f\paren{f\inv\paren{y_2}} \\ f\text{ strictement croissante}\end{dcases}\)

Donc \(f\inv\paren{y_1}<f\inv\paren{y_2}\).

Donc \(f\inv\) est strictement croissante.

Idem si \(f\) est strictement décroissante.

Montrons que \(f\inv\) est continue.

Supposons \(f\) strictement croissante.

Soit \(y_0\in\Im f\).

Montrons que \(f\inv\) est continue en \(y_0\).

On pose \(x_0=f\inv\paren{y_0}\). Montrons que \(\lim_{y_0}f\inv=x_0\), \cad \[\quantifs{\forall\epsilon\in\Rps;\exists W\in\V{y_0};\forall y\in\Im f\inter W}\abs{f\inv\paren{y}-x_0}\leq\epsilon.\]

Soit \(\epsilon\in\Rps\).

Si \(x_0-\epsilon,x_0+\epsilon\in I\) alors on pose \(W=\intervee{f\paren{x_0-\epsilon}}{f\paren{x_0+\epsilon}}\) un voisinage de \(y_0\) car \(f\) est strictement croissante.

On a \(\quantifs{\forall y\in W}x_0-\epsilon<f\inv\paren{y}<x_0+\epsilon\).

Donc \(W\) convient.

Si \(x_0-\epsilon\in I\) et \(x_0+\epsilon\not\in I\) : idem avec \(W=\intervee{f\paren{x_0-\epsilon}}{\pinf}\).

Etc.

Donc \(f\inv\) continue.
\end{dem}

\begin{ex}[Continuité de la fonction racine carrée]
La bijection \(\fonction{f}{\Rp}{\Rp}{x}{x^2}\) est continue et strictement monotone sur l'intervalle \(\Rp\).

Sa bijection réciproque \(\fonction{f\inv}{\Rp}{\Rp}{x}{\sqrt{x}}\) est donc continue.
\end{ex}

\begin{ex}
Soient \(A=\intervie{0}{1}\union\intervie{2}{3}\) et \(B=\intervie{0}{2}\).

On a \[\fonction{f}{A}{B}{x}{\begin{dcases}x &\text{si }x\in\intervie{0}{1} \\ x-1 &\text{si }x\in\intervie{2}{3}\end{dcases}}\] strictement croissante et continue mais \[\fonction{f\inv}{B}{A}{y}{\begin{dcases}y &\text{si }y\in\intervie{0}{1} \\ y+1 &\text{sinon}\end{dcases}}\] n'est pas continue.

D'où l'importance d'avoir un intervalle.
\end{ex}

\subsection{Fonctions circulaires réciproques}

\subsubsection{\(\Arcsin\)}

\begin{defprop}[Fonction \(\Arcsin\)]
La fonction \[\fonctionlambda{\intervii{\dfrac{-\pi}{2}}{\dfrac{\pi}{2}}}{\R}{\theta}{\sin\theta}\] est continue et strictement croissante sur l'intervalle \(\intervii{\dfrac{-\pi}{2}}{\dfrac{\pi}{2}}\).

Elle est donc bijective de l'intervalle \(\intervii{\dfrac{-\pi}{2}}{\dfrac{\pi}{2}}\) vers l'intervalle \(\intervii{\sin\dfrac{-\pi}{2}}{\sin\dfrac{\pi}{2}}=\intervii{-1}{1}\) (selon le \thref{cor:TVIAppliquéAuxFonctionsStrictementMonotones}).

Sa bijection réciproque est appelée la fonction arcsinus est est notée \(\Arcsin\) : \[\Arcsin:\intervii{-1}{1}\to\intervii{\dfrac{-\pi}{2}}{\dfrac{\pi}{2}}.\]

Elle est continue et strictement croissante (selon le \thref{theo:bijectionReciproqueDeFonctionContinueEtStrictementMonotoneEstContinueEtStrictementDeMêmeMonotonie}).

Allure du graphe :

\begin{center}
\begin{tkz}[scale=1.4]
\begin{axis}[axis lines=middle,
xmin=-pi/2-0.2,xmax=pi/2+0.2,
ymin=-pi/2-0.2,ymax=pi/2+0.2,
xtick={-pi/2,-1,1,pi/2},
ytick={-pi/2,-1,1,pi/2},
xticklabels={\(\dfrac{-\pi}{2}\),\(-1\),\(1\),\(\dfrac{\pi}{2}\)},
yticklabels={\(\dfrac{-\pi}{2}\),\(-1\),\(1\),\(\dfrac{\pi}{2}\)},
legend entries={\(\sin\),\(\Arcsin\)},
legend pos=north west,
legend style={font=\footnotesize},
clip=false]
\addplot[domain=-pi/2:pi/2,samples=1000,smooth,thick,blue] {sin(deg(x))};
\addplot[domain=-1:1,samples=1000,smooth,thick,orange] {asin(x)/180*pi};
\addplot[domain=-pi/2:pi/2,samples=1000,smooth,gray] {x};
\draw[->,green] (axis cs:1,pi/2) -- (axis cs:1,pi/2-0.5);
\draw[->,green] (axis cs:-1,-pi/2) -- (axis cs:-1,-pi/2+0.5);
\draw[->,green] (axis cs:-pi/2,-1) -- (axis cs:-pi/2+0.5,-1);
\draw[->,green] (axis cs:pi/2,1) -- (axis cs:pi/2-0.5,1);
\draw[<->,green] (axis cs:-0.35355,-0.35355) -- (axis cs:0.35355,0.35355);
\end{axis}
\end{tkz}
\end{center}
\end{defprop}

\begin{prop}[Caractérisation]
Si \(x\in\intervii{-1}{1}\), l'angle \(\Arcsin x\) est l'unique angle \(\theta\in\intervii{\dfrac{-\pi}{2}}{\dfrac{\pi}{2}}\) dont le sinus est \(x\).

On a donc : \[\quantifs{\forall x\in\intervii{-1}{1};\forall\theta\in\R}\theta=\Arcsin x\ssi\begin{dcases}\sin\theta=x \\ \dfrac{-\pi}{2}\leq\theta\leq\dfrac{\pi}{2}\end{dcases}\]
\end{prop}

\begin{ex}[Valeurs à connaître]
\[\begin{array}{c|ccccccccc}
x & -1 & \dfrac{-\sqrt{3}}{2} & \dfrac{-1}{\sqrt{2}} & \dfrac{-1}{2} & 0 & \dfrac{1}{2} & \dfrac{1}{\sqrt{2}} & \dfrac{\sqrt{3}}{2} & 1 \\[1em]
\hline \\
\Arcsin x & \dfrac{-\pi}{2} & \dfrac{-\pi}{3} & \dfrac{-\pi}{4} & \dfrac{-\pi}{6} & 0 & \dfrac{\pi}{6} & \dfrac{\pi}{4} & \dfrac{\pi}{3} & \dfrac{\pi}{2}
\end{array}\]
\end{ex}

\begin{prop}
La fonction \(\Arcsin\) est impaire.
\end{prop}

\begin{rem}
On a \[\quantifs{\forall x\in\intervii{-1}{1}}\sin\paren{\Arcsin x}=x\] mais \[\quantifs{\forall\theta\in\R}\Arcsin\paren{\sin x}=\theta\ssi\dfrac{-\pi}{2}\leq\theta\leq\dfrac{\pi}{2}.\]
\end{rem}

\subsubsection{\(\Arccos\)}

\begin{defprop}[Fonction \(\Arccos\)]
La fonction \[\fonctionlambda{\intervii{0}{\pi}}{\R}{\theta}{\cos\theta}\] est continue et strictement décroissante sur l'intervalle \(\intervii{0}{\pi}\).

Elle est donc bijective de l'intervalle \(\intervii{0}{\pi}\) vers l'intervalle \(\intervii{\cos\pi}{\cos0}=\intervii{-1}{1}\) (selon le \thref{cor:TVIAppliquéAuxFonctionsStrictementMonotones}).

Sa bijection réciproque est appelée la fonction arccosinus et est notée \(\Arccos\) : \[\Arccos:\intervii{-1}{1}\to\intervii{0}{\pi}.\]

Elle est continue et strictement décroissante (selon le \thref{theo:bijectionReciproqueDeFonctionContinueEtStrictementMonotoneEstContinueEtStrictementDeMêmeMonotonie}).

Allure du graphe :

\begin{center}
\begin{tkz}[scale=1.4]
\begin{axis}[axis lines=middle,
xmin=-1.2,xmax=1.2,
ymin=-0.2,ymax=pi+0.2,
xtick={-1,1},
ytick={pi},
xticklabels={\(-1\),\(1\)},
yticklabels={\(\pi\)},
legend entries={\(\Arccos\)},
legend pos=north east,
legend style={font=\footnotesize},
clip=false]
\addplot[domain=-1:1,samples=1000,smooth,thick,orange] {acos(x)/180*pi};
\node[above right] at (axis cs:0,pi/2) {\(\dfrac{\pi}{2}\)};
\draw[->,green] (axis cs:-1,pi) -- (axis cs:-1,pi-0.5);
\draw[->,green] (axis cs:1,0) -- (axis cs:1,0.5);
\draw[<->,green] (axis cs:-0.35355,pi/2+0.35355) -- (axis cs:0.35355,pi/2-0.35355);
\end{axis}
\end{tkz}
\end{center}
\end{defprop}

\begin{prop}[Caractérisation]
Si \(x\in\intervii{-1}{1}\), l'angle \(\Arccos x\) est l'unique angle \(\theta\in\intervii{0}{\pi}\) dont le cosinus est \(x\).

On a donc : \[\quantifs{\forall x\in\intervii{-1}{1};\forall\theta\in\R}\theta=\Arccos x\ssi\begin{dcases}\cos\theta=x \\ 0\leq\theta\leq\pi\end{dcases}\]
\end{prop}

\begin{ex}[Valeurs à connaître]
\[\begin{array}{c|ccccccccc}
x & -1 & \dfrac{-\sqrt{3}}{2} & \dfrac{-1}{\sqrt{2}} & \dfrac{-1}{2} & 0 & \dfrac{1}{2} & \dfrac{1}{\sqrt{2}} & \dfrac{\sqrt{3}}{2} & 1 \\[1em]
\hline \\
\Arccos x & \pi & \dfrac{5\pi}{6} & \dfrac{3\pi}{4} & \dfrac{2\pi}{3} & \dfrac{\pi}{2} & \dfrac{\pi}{3} & \dfrac{\pi}{4} & \dfrac{\pi}{6} & 0
\end{array}\]
\end{ex}

\begin{prop}
La fonction \(\Arccos\) n'est ni paire ni impaire.
\end{prop}

\subsubsection{\(\Arctan\)}

\begin{defprop}[Fonction \(\Arctan\)]
La fonction \[\fonctionlambda{\intervee{\dfrac{-\pi}{2}}{\dfrac{\pi}{2}}}{\R}{\theta}{\tan\theta}\] est continue et strictement croissante sur l'intervalle \(\intervee{\dfrac{-\pi}{2}}{\dfrac{\pi}{2}}\).

Elle est donc bijective de l'intervalle \(\intervee{\dfrac{-\pi}{2}}{\dfrac{\pi}{2}}\) vers l'intervalle \(\intervee{\lim_{\paren{\nicefrac{-\pi}{2}}^+}\tan}{\lim_{\paren{\nicefrac{\pi}{2}}^-}\tan}=\R\) (selon le \thref{cor:TVIAppliquéAuxFonctionsStrictementMonotones}).

Sa bijection réciproque est appelée la fonction arctangente est est notée \(\Arctan\) : \[\Arctan:\R\to\intervee{\dfrac{-\pi}{2}}{\dfrac{\pi}{2}}.\]

Elle est continue et strictement croissante (selon le \thref{theo:bijectionReciproqueDeFonctionContinueEtStrictementMonotoneEstContinueEtStrictementDeMêmeMonotonie}).

Allure du graphe :

\begin{center}
\begin{tkz}[scale=1.4]
\begin{axis}[axis lines=middle,
xmin=-13,xmax=13,
ymin=-pi/2-0.2,ymax=pi/2+0.2,
xtick={0},
ytick={-pi/2,pi/2},
yticklabels={\(\dfrac{-\pi}{2}\),\(\dfrac{\pi}{2}\)},
legend entries={\(\Arctan\)},
legend pos=north west,
legend style={font=\footnotesize},
clip=false]
\addplot[domain=-13:13,samples=1000,smooth,thick,orange] {atan(x)/180*pi};
\draw[<->,green] (axis cs:-0.35355,-0.35355) -- (axis cs:0.35355,0.35355);
\draw[dashed,green] (axis cs:0,pi/2) -- (axis cs:13,pi/2);
\draw[dashed,green] (axis cs:0,-pi/2) -- (axis cs:-13,-pi/2);
\end{axis}
\end{tkz}
\end{center}
\end{defprop}

\begin{prop}[Caractérisation]
Si \(x\in\R\), l'angle \(\Arctan x\) est l'unique angle \(\theta\in\intervee{\dfrac{-\pi}{2}}{\dfrac{\pi}{2}}\) dont le tangente est \(x\).

On a donc : \[\quantifs{\forall x\in\R;\forall\theta\in\R}\theta=\Arctan x\ssi\begin{dcases}\tan\theta=x \\ \dfrac{-\pi}{2}<\theta<\dfrac{\pi}{2}\end{dcases}\]
\end{prop}

\begin{ex}[Valeurs à connaître]
\[\begin{array}{c|ccccccc}
x & -\sqrt{3} & -1 & \dfrac{-1}{\sqrt{3}} & 0 & \dfrac{1}{\sqrt{3}} & 1 & \sqrt{3} \\[1em]
\hline \\
\Arctan x & \dfrac{-\pi}{3} & \dfrac{-\pi}{4} & \dfrac{-\pi}{6} & 0 & \dfrac{\pi}{6} & \dfrac{\pi}{4} & \dfrac{\pi}{3}
\end{array}\]
\end{ex}

\begin{prop}
La fonction \(\Arctan\) est impaire.
\end{prop}

\subsubsection{Relations à connaître}

\begin{prop}
On a :

\begin{enumerate}
\item \(\quantifs{\forall x\in\intervii{-1}{1}}\Arcsin x+\Arccos x=\dfrac{\pi}{2}\) \\

\item \(\quantifs{\forall x\in\intervii{-1}{1}}\cos\paren{\Arcsin x}=\sin\paren{\Arccos x}=\sqrt{1-x^2}\)
\end{enumerate}
\end{prop}

\begin{dem}[1]
Soit \(x\in\intervii{-1}{1}\).

Montrons que \(\Arcsin x=\dfrac{\pi}{2}-\Arccos x\).

On a \[\sin\paren{\dfrac{\pi}{2}-\Arccos x}=\cos\paren{\Arccos x}=x.\]

De plus, on a \(0\leq\Arccos x\leq\pi\) donc \[\dfrac{-\pi}{2}\leq\dfrac{\pi}{2}-\Arccos x\leq\dfrac{\pi}{2}.\]

Donc \(\dfrac{\pi}{2}-\Arccos x=\Arcsin x\).

D'où l'égalité.
\end{dem}

\begin{dem}[2]
Soit \(x\in\intervii{-1}{1}\).

Montrons que \(\cos\paren{\Arcsin x}=\sqrt{1-x^2}\).

On a \(\cos^2\paren{\Arcsin x}+\sin^2\paren{\Arcsin x}=1\) donc \[\cos^2\paren{\Arcsin x}=1-\sin^2\paren{\Arcsin x}=1-x^2.\]

De plus, on a \(\dfrac{-\pi}{2}\leq\Arcsin x\leq\dfrac{\pi}{2}\) donc \[\cos\paren{\Arcsin x}\geq0.\]

Donc \(\cos\paren{\Arcsin x}=\sqrt{1-x^2}\).

Idem pour \(\sin\paren{\Arccos x}\).
\end{dem}

\section{Propriétés plus fortes que la continuité}

\subsection{Continuité uniforme}

\begin{defi}[Continuité uniforme]
Soient \(A\subset\R\) et \(f:A\to\R\).

On dit que \(f\) est uniformément continue si on a : \[\quantifs{\forall\epsilon\in\Rps;\exists\delta\in\Rps;\forall x,y\in A}\abs{x-y}\leq\delta\imp\abs{f\paren{x}-f\paren{y}}\leq\epsilon.\]
\end{defi}

\begin{rem}
Ne pas confondre la continuité et la continuité uniforme.

Soient \(A\subset\R\) et \(f:A\to\R\).

La fonction \(f\) est continue si, et seulement si, on a : \[\quantifs{\forall x\in A;\forall\epsilon\in\Rps;\exists\delta\in\Rps;\forall y\in A}\abs{x-y}\leq\delta\imp\abs{f\paren{x}-f\paren{y}}\leq\epsilon.\]
\end{rem}

\begin{prop}
Toute fonction uniformément continue est continue.
\end{prop}

\begin{dem}
Si \(f\) est uniformément continue, on a : \[\quantifs{\forall\epsilon\in\Rps;\exists\delta\in\Rps;\forall x,y\in A}\abs{x-y}\leq\delta\imp\abs{f\paren{x}-f\paren{y}}\leq\epsilon\] où \(\delta\) est indépendant de \(x\).

Cela implique \[\quantifs{\forall x\in A;\forall\epsilon\in\Rps;\exists\delta\in\Rps;\forall y\in A}\abs{x-y}\leq\delta\imp\abs{f\paren{x}-f\paren{y}}\leq\epsilon\] où \(\delta\) dépend de \(x\).
\end{dem}

\begin{exo}
Les fonctions suivantes sont-elles uniformément continues ? \[\fonction{f}{\R}{\R}{x}{x^2}\qquad\fonction{g}{\intervii{0}{1}}{\R}{x}{x^2}\qquad\fonction{h}{\Rp}{\R}{x}{\sqrt{x}}\]

\textit{Indication :} pour l'étude de \(h\), on pourra montrer que \[\quantifs{\forall x,y\in\Rp}\sqrt{x+y}\leq\sqrt{x}+\sqrt{y}.\]
\end{exo}

\begin{corr}[\(f\)]\thlabel{corr:fonctionCarréePasUniformémentContinue}
Posons \(\epsilon=1\).

On a \[\quantifs{\forall x,y\in\R}\abs{f\paren{x}-f\paren{y}}=\abs{x^2-y^2}=\abs{x-y}\abs{x+y}.\]

Soit \(\delta\in\Rps\).

On remarque \[\quantifs{\forall x\in\R}\abs{f\paren{x}-f\paren{x+\delta}}=\abs{2x+\delta}\delta.\]

Donc \(\lim_{x\to\pinf}\abs{f\paren{x}-f\paren{x+\delta}}>1\).

Donc \(f\) n'est pas uniformément continue.
\end{corr}

\begin{corr}[\(g\)]
Montrons que \(g\) est uniformément continue, \cad \[\quantifs{\forall\epsilon\in\Rps;\exists\delta\in\Rps;\forall x,y\in\intervii{0}{1}}\abs{x-y}\leq\delta\imp\abs{g\paren{x}-g\paren{y}}\leq\epsilon.\]

Soit \(\epsilon\in\Rps\).

On a \[\quantifs{\forall x,y\in\intervii{0}{1}}\abs{g\paren{x}-g\paren{y}}=\abs{x+y}\abs{x-y}\leq2\abs{x-y}.\]

Donc \(\delta=\dfrac{\epsilon}{2}\) convient car on a \[\quantifs{\forall x,y\in\intervii{0}{1}}\abs{x-y}\leq\delta\imp\abs{g\paren{x}-g\paren{y}}\leq2\times\dfrac{\epsilon}{2}=\epsilon.\]

Donc \(g\) est uniformément continue.
\end{corr}

\begin{corr}[\(h\)]
On a \(\quantifs{\forall x,y\in\Rp}\sqrt{x+y}\leq\sqrt{x}+\sqrt{y}\).

En effet : \[\begin{aligned}
\quantifs{\forall x,y\in\Rp}\sqrt{x+y}\leq\sqrt{x}+\sqrt{y}&\ssi x+y\leq x+y+2\sqrt{xy} \\
&\ssi0\leq2\sqrt{xy} \\
&\color{white}\ssi\color{black}\text{ce qui est vrai}
\end{aligned}\]

Soient \(a,b\in\Rp\).

Si \(a\leq b\) alors \(0\leq\sqrt{b}-\sqrt{a}=\sqrt{a+b-a}-\sqrt{a}\) donc \(\abs{\sqrt{b}-\sqrt{a}}\leq\sqrt{b-a}\).

Si \(a\geq b\), de même, on a \(\abs{\sqrt{b}-\sqrt{a}}\leq\sqrt{a-b}\).

Donc \(\quantifs{\forall a,b\in\Rp}\abs{\sqrt{a}-\sqrt{b}}\leq\sqrt{\abs{a-b}}\).

Montrons que \(h\) est uniformément continue, \cad \[\quantifs{\forall\epsilon\in\Rps;\exists\delta\in\Rps;\forall x,y\in\Rp}\abs{x-y}\leq\delta\imp\abs{h\paren{x}-h\paren{y}}\leq\epsilon.\]

Soit \(\epsilon\in\Rps\). Posons \(\delta=\epsilon^2\).

On a : \[\quantifs{\forall x,y\in\Rp}\abs{x-y}\leq\delta\imp\abs{h\paren{x}-h\paren{y}}\leq\sqrt{\delta}=\epsilon.\]

Donc \(\delta\) convient.

Donc \(h\) est uniformément continue.
\end{corr}

\begin{theo}[Théorème de Heine]
Soient \(a,b\in\R\) tels que \(a<b\). Soit \(f:\intervii{a}{b}\to\R\) continue.

Alors \(f\) est uniformément continue.
\end{theo}

\begin{dem}
Par l'absurde, supposons \[\quantifs{\exists\epsilon\in\Rps;\forall\delta\in\Rps;\exists x,y\in\intervii{a}{b}}\begin{dcases}\abs{x-y}\leq\delta \\ \abs{f\paren{x}-f\paren{y}}>\epsilon\end{dcases}\]

Soit un tel \(\epsilon\in\Rps\).

On a en particulier \[\quantifs{\forall n\in\Ns;\exists x_n,y_n\in\intervii{a}{b}}\begin{dcases}\abs{x_n-y_n}\leq\dfrac{1}{n} \\ \abs{f\paren{x_n}-f\paren{y_n}}>\epsilon\end{dcases}\]

En considérant de tels \(x_n,y_n\) pour tout \(n\in\Ns\), on obtient deux suites \(\paren{x_n}_n,\paren{y_n}_n\in\intervii{a}{b}^{\Ns}\) telles que \[\quantifs{\forall n\in\Ns}\begin{dcases}\abs{x_n-y_n}\leq\dfrac{1}{n} \\ \abs{f\paren{x_n}-f\paren{y_n}}>\epsilon\end{dcases}\]

Comme \(\paren{x_n}_n\) est bornée, d'après le théorème de Bolzano-Weierstrass, il existe \(\phi:\Ns\to\Ns\) strictement croissante telle que \(\paren{x_{\phi\paren{n}}}_n\) soit convergente.

Comme \(\paren{y_{\phi\paren{n}}}_n\) est bornée, d'après le théorème de Bolzano-Weierstrass, il existe \(\psi:\Ns\to\Ns\) strictement croissante telle que \(\paren{y_{\phi\rond\psi\paren{n}}}\) soit convergente.

Ainsi, \(\paren{x_{\phi\rond\psi\paren{n}}}_n\) converge car c'est une suite extraite d'une suite convergente.

On pose \(x=\lim_nx_{\phi\rond\psi\paren{n}}\) et \(y=\lim_ny_{\phi\rond\psi\paren{n}}\).

On a \[\quantifs{\forall n\in\Ns}\abs{x_{\phi\rond\psi\paren{n}}-y_{\phi\rond\psi\paren{n}}}\leq\dfrac{1}{\phi\rond\psi\paren{n}}.\]

D'où, par passage à la limite : \(\abs{x-y}\leq0\) donc \(x=y\).

D'autre part, on a : \[\quantifs{\forall n\in\Ns}\abs{f\paren{x_{\phi\rond\psi\paren{n}}}-f\paren{y_{\phi\rond\psi\paren{n}}}}\geq\epsilon.\]

D'où, par passage à la limite : \(\abs{f\paren{x}-f\paren{y}}\geq\epsilon\) : contradiction car \(x=y\).
\end{dem}

\subsection{Fonctions lipschitziennes}

\begin{defi}[Fonction lipschitzienne]
Soient \(A\subset\R\), \(f:A\to\R\) et \(k\in\Rp\).

On dit que \(f\) est \(k\)-lipschitzienne si on a : \[\quantifs{\forall x,y\in A}\abs{f\paren{x}-f\paren{y}}\leq k\abs{x-y}.\]

On dit que \(f\) est lipschitzienne si elle est \(K\)-lipschitzienne pour un certain \(K\in\Rp\) : \[\quantifs{\exists K\in\Rp;\forall x,y\in A}\abs{f\paren{x}-f\paren{y}}\leq K\abs{x-y}.\]
\end{defi}

\begin{prop}
Toute fonction lipschitzienne est uniformément continue.
\end{prop}

\begin{dem}
Soient \(A\subset\R\), \(f:A\to\R\) et \(k\in\Rp\).

On suppose que \(f\) est \(k\)-lipschitzienne.

Montrons que \(f\) est uniformément continue, \cad \[\quantifs{\forall\epsilon\in\Rps;\exists\delta\in\Rps;\forall x,y\in A}\abs{x-y}\leq\delta\imp\abs{f\paren{x}-f\paren{y}}\leq\epsilon.\]

Soit \(\epsilon\in\Rps\). On pose \(\delta=\dfrac{\epsilon}{k}\).

On a \[\quantifs{\forall x,y\in A}\abs{x-y}\leq\delta\imp\abs{f\paren{x}-f\paren{y}}\leq k\delta=\epsilon.\]

Donc \(\delta\) convient.

Donc \(f\) est uniformément continue.
\end{dem}

\begin{bilan}
Soient \(A\subset\R\) et \(f:A\to\R\).

On a : \[f\text{ lipschitzienne}\imp f\text{ uniformément continue}\imp f\text{ continue}.\]
\end{bilan}

\begin{exo}
Les fonctions suivantes sont-elles lipschitziennes ? \[\fonction{f}{\R}{\R}{x}{x^2}\qquad\fonction{g}{\intervii{0}{1}}{\R}{x}{x^2}\qquad\fonction{h}{\Rp}{\R}{x}{\sqrt{x}}\]
\end{exo}

\begin{corr}[\(f\)]
On a vu que \(f\) n'est pas uniformément continue donc par contraposée, \(f\) n'est pas lipschitzienne (\cf \thref{corr:fonctionCarréePasUniformémentContinue}).
\end{corr}

\begin{corr}[\(g\)]
Montrons que \(g\) est lipschitzienne, \cad \[\quantifs{\exists k\in\Rp;\forall x,y\in\intervii{0}{1}}\abs{g\paren{x}-g\paren{y}}\leq k\abs{x-y}.\]

On a : \[\begin{aligned}
\quantifs{\forall x,y\in\intervii{0}{1}}\abs{g\paren{x}-g\paren{y}}&=\abs{x^2-y^2} \\
&=\abs{x-y}\abs{x+y} \\
&\leq2\abs{x-y}
\end{aligned}\]

Donc \(g\) est \(2\)-lipschitzienne.

Donc \(g\) est lipschitzienne.
\end{corr}

\begin{corr}[\(h\)]
Montrons que \(h\) n'est pas lipschitzienne.

Par l'absurde, supposons \(h\) lipschitzienne, \cad : \[\quantifs{\exists k\in\Rp;\forall x,y\in\Rp}\abs{\sqrt{x}-\sqrt{y}}\leq k\abs{x-y}.\]

En particulier, on a \[\quantifs{\forall n\in\Ns}\abs{\sqrt{\dfrac{1}{n}}-\sqrt{0}}\leq k\abs{\dfrac{1}{n}-0}.\]

Donc \[\quantifs{\forall n\in\Ns}\dfrac{1}{\sqrt{n}}\leq\dfrac{k}{n}.\]

Donc \(\quantifs{\forall n\in\Ns}\sqrt{n}\leq k\) : contradiction.

Donc \(h\) n'est pas lipschitzienne.
\end{corr}

\chapter{Arithmétique}

\minitoc

\section{Rappels \& compléments}

\subsection{Division euclidienne}

\begin{defprop}[Division euclidienne dans \(\Z\)]
Soient \(a\in\Z\) et \(b\in\Ns\).

Il existe un unique couple \(\paren{q,r}\in\Z^2\) tel que : \[\begin{dcases}a=qb+r \\ 0\leq r<b\end{dcases}\]

L'entier \(q\) est appelé le quotient de la division euclidienne de \(a\) par \(b\).

L'entier \(r\) est appelé le reste de la division euclidienne de \(a\) par \(b\).
\end{defprop}

\begin{ex}
Divisions euclidiennes de \(37\) et \(-37\) par \(10\) : \[37=3\times10+7\quad\text{et}\quad-37=\paren{-4}\times10+3.\]
\end{ex}

\begin{dem}
\textit{Cf.} \thref{dem:divEucli}.
\end{dem}

\subsection{Divisibilité}

\begin{defi}[Divisibilité dans \(\Z\)]
Soient \(a,b\in\Z\). Si on a \[\quantifs{\exists c\in\Z}ac=b\] alors on dit que \begin{itemize}
\item \(a\) divise \(b\) ;

\item \(a\) est un diviseur de \(b\) ;

\item \(b\) est un multiple de \(a\) ;

\item \(b\) est divisible par \(a\).
\end{itemize}
\end{defi}

\begin{nota}
Soient \(a,b\in\Z\).

\begin{itemize}
\item La notation \(a\divise b\) signifie \guillemets{\(a\) divise \(b\)} ;

\item L'ensemble \(\accol{ac}_{c\in\Z}\) des multiples de \(a\) est noté \(a\Z\) : \[a\Z=\accol{ac}_{c\in\Z}=\accol{x\in\Z\tq a\divise x}.\]

\item Dans ce cours, on notera \(\ensdiv{b}\) l'ensemble des diviseurs positifs de \(b\) : \[\ensdiv{b}=\accol{x\in\N\tq x\divise b}.\]
\end{itemize}
\end{nota}

\begin{prop}[Divisibilité dans \(\Z\)]
La relation binaire \(\divise\) sur \(\Z\) est réflexive et transitive (mais pas antisymétrique donc ce n'est pas une relation d'ordre sur \(\Z\)).

On a : \(\quantifs{\forall a,b\in\Z}\paren{a\divise b\quad\text{et}\quad b\divise a}\ssi a=\pm b\).

On a : \begin{itemize}
\item \(\quantifs{\forall a,b\in\Z;\forall n\in\Z}a\divise b\imp na\divise nb\) ;

\item \(\quantifs{\forall a,b\in\Z;\forall n\in\Zs}a\divise b\ssi na\divise nb\).
\end{itemize}
\end{prop}

\begin{prop}[Divisibilité dans \(\N\)]\thlabel{prop:divisibilitéDansN}
La relation binaire \(\divise\) sur \(\N\) est une relation d'ordre sur \(\N\).

Pour cette relation d'ordre, \(0\) est le plus grand élément de \(\N\) et \(1\) le plus petit : \(\quantifs{\forall n\in\N}n\divise0\quad\text{et}\quad1\divise n\).

On a : \(\quantifs{\forall a,b\in\Z}a\divise b\imp\paren{a\leq b\quad\text{ou}\quad b=0}\).
\end{prop}

\subsection{Congruences}

\begin{defi}
Soient \(a,b,n\in\Z\).

On dit que \guillemets{\(a\) est congru à \(b\) modulo \(n\)} et on note \(a\equiv b\croch{n}\) si on a : \[\quantifs{\exists k\in\Z}a=b+kn.\]

On a donc : \[\begin{aligned}
a\equiv b\croch{n}&\ssi\quantifs{\exists k\in\Z}a=b+kn \\
&\ssi n\divise\paren{a-b}.
\end{aligned}\]
\end{defi}

\begin{rem}
Soit \(n\in\Z\).

La relation \guillemets{être congrus modulo \(n\)} est une relation d'équivalence sur \(\Z\).
\end{rem}

\begin{rem}
Soient \(x\in\Z\) et \(n\in\Ns\).

On note \(q\) et \(r\) le quotient et le reste de la division euclidienne de \(x\) par \(n\).

On a : \begin{itemize}
\item \(x\equiv r\croch{n}\) ;

\item \(n\divise x\ssi r=0\).
\end{itemize}
\end{rem}

\begin{prop}[Opérations sur les congruences]
Soient \(a,b,c,d,n\in\Z\).

On suppose \(a\equiv b\croch{n}\text{ et }c\equiv d\croch{n}\).

On a : \begin{itemize}
\item somme : \(a+c\equiv b+d\croch{n}\) ;

\item produit : \(ac\equiv bd\croch{n}\) ;

\item puissance : \(\quantifs{\forall k\in\N}a^k\equiv b^k\croch{n}\).
\end{itemize}
\end{prop}

\begin{prop}
Soient \(a,b,n\in\Z\).

\begin{itemize}
\item Pour tout \(\lambda\in\Zs\), on a : \(a\equiv b\croch{n}\ssi\lambda a\equiv\lambda b\croch{\lambda n}\).

\item Pour tout diviseur \(d\) de \(n\), on a : \(a\equiv b\croch{n}\imp a\equiv b\croch{d}\).
\end{itemize}
\end{prop}

\begin{ex}[Élément non-régulier modulo \(n\)]
Trouvons \(\lambda,x,y,n\in\Z\) tels que : \[\lambda\not=0\quad\text{et}\quad\lambda x\equiv\lambda y\croch{n}\quad\text{et}\quad x\not\equiv y\croch{n}.\]

On remarque que \(\lambda=2\), \(x=3\), \(y=0\) et \(n=6\) conviennent.

En effet, on a \[2\not=0\quad\text{et}\quad2\times3\equiv2\times0\croch{6}\quad\text{et}\quad3\not\equiv0\croch{6}.\]
\end{ex}

\begin{ex}
Calculons \(20.222.023\) modulo \(9\), modulo \(11\) et modulo \(999\).

On a \[\begin{aligned}
20.222.023&=2\E{7}+2\E{5}+2\E{4}+2\E{3}+2\E{1}+3\E{0} \\
&\equiv2+2+2+2+2+3\croch{9}\quad\text{car }10\equiv1\croch{9} \\
&\equiv13\croch{9} \\
&\equiv4\croch{9}.
\end{aligned}\]

De plus, \(10\equiv-1\croch{11}\) donc on a \[\begin{aligned}
20.222.023&\equiv-2-2-2+2-2+3\croch{11} \\
&\equiv-3\croch{11} \\
&\equiv8\croch{11}.
\end{aligned}\]

Enfin, comme \(1.000\equiv1\croch{999}\), on a \[\begin{aligned}
20.222.023&=20\times1000^2+222\times1000^1+23\times1000^0 \\
&\equiv20+222+23\croch{999} \\
&\equiv265\croch{999}.
\end{aligned}\]
\end{ex}

\subsection{Sous-groupes}

\begin{prop}[Intersection de sous-groupes]
Soit \(\paren{G,*}\) un groupe et \(\paren{H_i}_{i\in I}\) une famille de sous-groupes de \(G\).

Alors l'intersection \(\biginter_{i\in I}H_i\) de ces sous-groupes est un sous-groupe de \(G\).
\end{prop}

\begin{dem}
\Cf \thref{exo:6.15}.
\end{dem}

\begin{defprop}[Somme de sous-groupes]\thlabel{defprop:sommeDeSousGroupesEstUnSousGroupe}
Soit \(\paren{G,+}\) un groupe abélien. Soient \(H_1,H_2\) des sous-groupes de \(G\).

On appelle somme de \(H_1\) et \(H_2\) et on note \(H_1+H_2\) l'ensemble : \[\begin{aligned}
H_1+H_2&=\accol{g\in G\tq\quantifs{\exists h_1\in H_1,\exists h_2\in H_2}g=h_1+h_2} \\
&=\accol{h_1+h_2}_{\paren{h_1,h_2}\in H_1\times H_2}
\end{aligned}\]

On a : \begin{enumerate}
\item l'ensemble \(H_1+H_2\) est un sous-groupe de \(G\) ;

\item la loi \(+\) est une loi de composition interne sur l'ensemble des sous-groupes de \(G\) ;

\item cette loi \(+\) est associative et commutative.
\end{enumerate}
\end{defprop}

\begin{dem}[1]
Montrons que \(H_1+H_2\) est un sous-groupe de \(G\).

On note \(0\) le neutre de \(G\).

On a \(H_1+H_2\subset G\).

Comme \(H_1\) et \(H_2\) sont des sous-groupes de \(G\), \(0\in H_1\) et \(0\in H_2\) donc \(0=0+0\in H_1+H_2\).

Soient \(h,h\prim\in H_1+H_2\).

Montrons que \(h-h\prim\in H_1+H_2\).

Soient \(h_1,h_1\prim\in H_1\) et \(h_2,h_2\prim\in H_2\) tels que \[h=h_1+h_2\qquad\text{et}\qquad h\prim=h_1\prim+h_2\prim.\]

On a \[\begin{WithArrows}
h-h\prim&=h_1+h_2-h_1\prim-h_2\prim \Arrow{car groupes abéliens} \\
&=\underbrace{h_1-h_1\prim}_{\in H_1}+\underbrace{h_2-h_2\prim}_{\in H_2}
\end{WithArrows}\]

Donc \(h-h\prim\in H_1+H_2\).

Donc \(H_1+H_2\) est un sous-groupe de \(G\).
\end{dem}

\begin{dem}[2]
Clair.
\end{dem}

\begin{dem}[3]
Montrons que \(+\) est associative :

Soient \(H_1,H_2,H_3\) des sous-groupes de \(G\).

Montrons que \(H_1+\paren{H_2+H_3}=\paren{H_1+H_2}+H_3\).

On a \(H_1+\paren{H_2+H_3}\subset G\) et \(\paren{H_1+H_2}+H_3\subset G\).

Soit \(g\in G\).

On a \[\begin{aligned}
g\in\paren{H_1+H_2}+H_3&\ssi\quantifs{\exists h\in H_1+H_2;\exists h_3\in H_3}g=h+h_3 \\
&\ssi\quantifs{\exists h_1\in H_1;\exists h_2\in H_2;\exists h_3\in H_3}g=\paren{h_1+h_2}+h_3 \\
&\ssi\quantifs{\exists h_1\in H_1;\exists h_2\in H_2;\exists h_3\in H_3}g=h_1+h_2+h_3 \\
&\ssi\quantifs{\exists h_1\in H_1;\exists h\in H_2+H_3}g=h_1+h \\
&\ssi g\in H_1+\paren{H_2+H_3}
\end{aligned}\]

Donc \(+\) est associative.

Montrons que \(+\) est commutative :

On a \(H_1+H_2\subset G\) et \(H_2+H_1\subset G\).

Soit \(g\in G\).

On a \[\begin{aligned}
g\in H_1+H_2&\ssi\quantifs{\exists h_1\in H_1;\exists h_2\in H_2}g=h_1+h_2 \\
&\ssi\quantifs{\exists h_2\in H_2;\exists h_1\in H_1}g=h_2+h_1 \\
&\ssi g\in H_2+H_1
\end{aligned}\]

Donc \(+\) est commutative.
\end{dem}

\begin{theo}[Sous-groupes de \(\Z\)]\thlabel{theo:nZSontLesSousGroupesDeZ}
Les sous-groupes de \(\paren{\Z,+}\) sont les ensembles de la forme \(n\Z\) où \(n\in\N\).
\end{theo}

\begin{dem}
Montrons que les sous-groupes de \(\groupe{\Z}\) sont les parties de la forme \(n\Z=\accol{nk}_{k\in\Z}\) où \(n\in\N\).

On a \[n\Z=\accol{\dots;-2n;-n;0;n;2n;\dots}.\]

\increc

Soit \(n\in\N\).

Montrons que \(n\Z\) est un sous-groupe de \(\groupe{\Z}\).

On a \(n\Z\subset\Z\).

On a \(0\in n\Z\) car \(0=n\times0\).

Soient \(x,y\in n\Z\).

Montrons que \(x-y\in n\Z\).

Soient \(x\prim,y\prim\in\Z\) tels que \(x=nx\prim\) et \(y=ny\prim\).

On a \[x-y=nx\prim-ny\prim=n\underbrace{\paren{x-x\prim}}_{\in\Z}.\]

Donc \(x-y\in n\Z\).

Donc \(n\Z\) est un sous-groupe de \(\groupe{\Z}\).

\incdir

Soit \(H\) un sous-groupe de \(\groupe{\Z}\).

On pose \(E=H\inter\Ns\).

Si \(E=\ensvide\) :

Montrons que \(H=0\Z=\accol{0}\).

Par l'absurde, soit \(x\in H\excluant\accol{0}\).

Si \(x>0\) alors \(x\in E\) : contradiction car \(E=\ensvide\).

Si \(x<0\) alors \(-x\in E\) : contradiction car \(E=\ensvide\).

Donc \(H=\accol{0}=0\Z\).

Supposons \(E\not=\ensvide\).

On pose \(n=\min E\) (\(n\) existe car \(E\) est une partie non-vide de \(\N\)).

Montrons que \(H=n\Z\).

\increc

Montrons que \(\quantifs{\forall k\in\N}nk\in H\) par récurrence sur \(k\).

Initialisation : on a \(0\in H\) et \(0=0\times n\).

Hérédité : soit \(k\in\N\) tel que \(nk\in H\).

On a \(n\paren{k+1}=nk+n\in H\) car \(H\) est un sous-groupe de \(\Z\) et \(nk,n\in H\).

Conclusion : \(\quantifs{\forall k\in\N}nk\in H\).

On en déduit que \(\quantifs{\forall k\in\N}-nk\in H\) car \(H\) est un sous-groupe de \(\Z\).

Donc \(\quantifs{\forall k\in\Z}nk\in H\).

D'où \(n\Z\subset H\).

\incdir

Soit \(h\in H\).

On a \(n\in\Ns\).

On note \(q\) et \(r\) le quotient et le reste de la division euclidienne de \(h\) par \(n\).

On a donc \(\begin{dcases}h=nq+r \\ 0\leq r<n\end{dcases}\)

On a \(h,n\in H\) donc \(r=h-nq\in H\).

Donc \(r=0\) car \(n\) est le plus petit élément strictement positif de \(H\).

Donc \(h=qn\in n\Z\).
\end{dem}

\section{PGCD}

\subsection{PGCD de deux entiers}

\subsubsection{Définition}

\begin{defi}
Soient \(a,b\in\Z\).

On rappelle qu'on note \(\ensdiv{a}\) l'ensemble des diviseurs positifs de \(a\).

Les diviseurs positifs communs à \(a\) et \(b\) sont les éléments de l'ensemble \(\ensdiv{a}\inter\ensdiv{b}\).

Si celui-ci possède un plus grand élément pour la relation d'ordre \(\divise\), on l'appelle le plus grand commun diviseur de \(a\) et \(b\) et on le note \(a\et b\).
\end{defi}

\begin{ex}
On a \[\ensdiv{12}=\accol{1;2;3;4;6;12}\qquad\text{et}\qquad\ensdiv{15}=\accol{1;3;5;15}.\]

Donc les diviseurs communs à \(12\) et \(15\) sont : \[\ensdiv{12}\inter\ensdiv{15}=\accol{1;3}.\]

Donc le PGCD de \(12\) et \(15\) est : \[12\et15=3.\]
\end{ex}

\begin{rem}
Soient \(a,b\in\Z\).

Le PGCD de \(a\) et \(b\) existe si, et seulement si, le PGCD de \(\abs{a}\) et \(\abs{b}\) existe.

On a alors \[a\et b=\abs{a}\et\abs{b}.\]
\end{rem}

\begin{dem}
Clair car \(\ensdiv{a}\inter\ensdiv{b}=\ensdiv{\abs{a}}\inter\ensdiv{\abs{b}}\).
\end{dem}

\begin{rem}
On a, pour tout \(a\in\Z\) : \[a\et0=\abs{a}\qquad\text{et}\qquad a\et1=1.\]
\end{rem}

\begin{rem}
Soient \(a,b\in\Z\).

\begin{enumerate}
\item Le PGCD de \(a\) et \(b\), s'il existe, est le diviseur commun à \(a\) et \(b\) positif et divisible par tous leurs autres diviseurs communs. \\

\item Il est donc caractérisé par : \[a\et b\geq0\qquad\text{et}\qquad\ensdiv{a}\inter\ensdiv{b}=\ensdiv{a\et b}.\] \\

\item Si \(a=b=0\) alors \(0\et0=0\). \\ Sinon, le plus grand diviseur commun à \(a\) et \(b\) pour l'ordre \(\divise\) est aussi le plus grand diviseur commun à \(a\) et \(b\) pour l'ordre \(\leq\) selon la \thref{prop:divisibilitéDansN}.
\end{enumerate}
\end{rem}

\begin{dem}[2]
Soit \(n\in\Z\) tel que \(n\geq0\) et \(\ensdiv{a}\inter\ensdiv{b}=\ensdiv{n}\).

Montrons que \(n\) est le PGCD de \(a\) et \(b\).

On a \(n\in\ensdiv{n}\) donc \(n\in\ensdiv{a}\inter\ensdiv{b}\) donc \[n\divise a\qquad\text{et}\qquad n\divise b.\]

Donc \(n\) est un diviseur commun à \(a\) et \(b\).

Montrons que \(n\) est le plus grand d'entre eux.

Soit \(d\) un diviseur commun à \(a\) et \(b\).

On a \(d\in\ensdiv{a}\inter\ensdiv{b}\) donc \(d\in\ensdiv{n}\).

Donc \(d\divise n\).
\end{dem}

\begin{lem}
Soient \(a,b,n\in\N\) tels que \(a\equiv b\croch{n}\).

Alors \(a\) et \(n\) admettent un PGCD si, et seulement si, \(b\) et \(n\) admettent un PGCD.

On a alors \[a\et n=b\et n.\]
\end{lem}

\begin{dem}
Il suffit de montrer que \(\ensdiv{a}\inter\ensdiv{b}=\ensdiv{b}\inter\ensdiv{a}\).

\incdir

Soient \(d\in\ensdiv{a}\inter\ensdiv{b}\) et \(k\in\Z\) tel que \(b=a+kn\).

On a \(d\divise a\) et \(d\divise kn\) donc \(d\divise b\).

Donc \(d\in\ensdiv{b}\inter\ensdiv{a}\).

\increc Idem.
\end{dem}

\begin{rem}
Soient \(a,b\in\Z\).

On a \[a\et b=\abs{a}\ssi a\divise b.\]
\end{rem}

\begin{dem}
\impdir

Supposons \(a\et b=\abs{a}\).

On a \(a\divise\abs{a}\) et \(\abs{a}\divise b\).

Donc \(a\divise b\).

\imprec

Supposons \(a\divise b\).

On a \(\abs{a}\divise a\) et \(\abs{a}\divise b\).

Donc \(\abs{a}\in\ensdiv{a}\inter\ensdiv{b}\).

De plus, on a \(\quantifs{\forall d\in\ensdiv{a}\inter\ensdiv{b}}d\divise\abs{a}\).

Donc \(\abs{a}=a\et b\).
\end{dem}

\begin{rem}
Soient \(a,b\in\N\).

Alors \(a\et b=\inf\accol{a;b}\) pour l'ordre \(\divise\), sous réserve d'existence.
\end{rem}

\subsubsection{Preuve algébrique}

\begin{prop}
Soient \(a,b\in\Z\).

Alors \(a\) et \(b\) admettent un PGCD.
\end{prop}

\begin{dem}
L'ensemble \(a\Z+b\Z\) est un sous-groupe de \(\Z\) (\cf \thref{defprop:sommeDeSousGroupesEstUnSousGroupe} et \thref{theo:nZSontLesSousGroupesDeZ}).

Il existe donc \(n\in\Z\) tel que \(a\Z+b\Z=n\Z\).

Montrons que \(n\) est le PGCD de \(a\) et \(b\).

On a \(n\geq0\).

De plus, on a \(a\in a\Z+b\Z\) car \(a=1\times a+0\times b\) donc \(a\in n\Z\) donc \(n\divise a\).

De même, \(n\divise b\).

Soient \(d\in\ensdiv{a}\inter\ensdiv{b}\) et \(k,l\in\Z\) tels que \(dk=a\) et \(dl=b\).

On a \(n\in n\Z\) donc \(n\in a\Z+b\Z\).

Soient \(u,v\in\Z\) tels que \(n=ua+vb\).

On a \(n=udk+vdl=d\paren{uk+vl}\) donc \(d\divise n\).

Donc \(n\) est le PGCD de \(a\) et \(b\).
\end{dem}

\subsubsection{Preuve algorithmique, algorithme d'Euclide}

On définit une fonction récursive \(\pgcd:\Z^2\to\N\).

Montrons que la fonction termine toujours et que \(\pgcd[a][b]\) renvoie le PGCD de \(a\) et \(b\) pour tout \(\paren{a,b}\in\Z^2\).

\begin{algo}[Algorithme d'Euclide]\thlabel{algo:EuclideEntiers}
Si \(a<0\) ou \(b<0\), on renvoie \(\pgcd[\abs{a}][\abs{b}]\).

Si \(b=0\), on renvoie \(a\).

Sinon : \begin{itemize}
\item on fait la division euclidienne de \(a\) par \(b\) : \(\begin{dcases}a=qb+r \\ 0\leq r<b\end{dcases}\) ; \\

\item on a \(a\equiv r\croch{b}\) donc on renvoie \(\pgcd[b][r]\).
\end{itemize}
\end{algo}

\begin{dem}[Terminaison]
Montrons que l'algorithme termine toujours.

Soit \(\paren{a,b}\in\Z^2\).

Par l'absurde, supposons que l'algorithme ne termine jamais : \[\pgcd[a][b]\to\underbrace{\pgcd[a_1][b_1]\to\pgcd[a_2][b_2]\to\dots}_{\text{arguments }\geq0}\]

Pour tout \(k\in\Ns\), on a \(\begin{dcases}a_{k+1}=b_k \\ 0\leq b_{k+1}<b_k\end{dcases}\)

Donc \(\paren{b_k}_{k\in\Ns}\) est une suite strictement décroissante d'entiers naturels : contradiction.

Donc l'algorithme termine toujours.
\end{dem}

\begin{ex}
Calculons le PGCD de \(1024\) et \(1000\) : \[\begin{aligned}
1024&=1\times1000+24 \\
1000&=41\times24+16 \\
24&=1\times16+8 \\
16&=2\times8+0
\end{aligned}\]

Donc \(1024\et1000=8\).
\end{ex}

\subsubsection{Propriétés}

\begin{prop}\thlabel{prop:PGCDFoisLambdaEgal}
Soient \(a,b,\lambda\in\Z\).

On a : \[\paren{\lambda a}\et\paren{\lambda b}=\abs{\lambda}\paren{a\et b}.\]
\end{prop}

\begin{dem}
Montrons que \(\lambda\paren{a\et b}\divise\paren{\lambda a}\et\paren{\lambda b}\).

On a \(a\et b\) divise \(a\) et \(b\).

Donc \(\lambda\paren{a\et b}\) divise \(\lambda a\) et \(\lambda b\).

Donc \(\lambda\paren{a\et b}\divise\paren{\lambda a}\et\paren{\lambda b}\).

Montrons que \(\paren{\lambda a}\et\paren{\lambda b}\divise\lambda\paren{a\et b}\).

On a \(\lambda\) divise \(\lambda a\) et \(\lambda b\).

Donc \(\lambda\divise\paren{\lambda a}\et\paren{\lambda b}\).

Soit \(d\in\Z\) tel que \(\paren{\lambda a}\et\paren{\lambda b}=d\lambda\).

On a \(\paren{\lambda a}\et\paren{\lambda b}\) divise \(\lambda a\) et \(\lambda b\).

Donc \(d\lambda\) divise \(\lambda a\) et \(\lambda b\).

Supposons \(\lambda\not=0\).

Alors \(d\) divise \(a\) et \(b\).

Donc \(d\divise a\et b\).

Donc \(\lambda d\divise\lambda\paren{a\et b}\).

Donc \(\paren{\lambda a}\et\paren{\lambda b}\divise\lambda\paren{a\et b}\).

Si \(\lambda=0\) : la proposition est vraie aussi car \(\begin{dcases}\paren{\lambda a}\et\paren{\lambda b}=0 \\ \lambda\paren{a\et b}=0\end{dcases}\)

Finalement, les entiers \(\lambda\paren{a\et b}\) et \(\paren{\lambda a}\et\paren{\lambda b}\) se divisent mutuellement.

Donc \(\abs{\lambda\paren{a\et b}}=\abs{\paren{\lambda a}\et\paren{\lambda b}}\).

Donc \(\abs{\lambda}\paren{a\et b}=\paren{\lambda a}\et\paren{\lambda b}\).
\end{dem}

\subsection{Relation de Bézout}

\subsubsection{Définition}

\begin{defprop}[Relation de Bézout]\thlabel{defprop:relationDeBezout}
Soient \(a,b\in\Z\).

Alors il existe des entiers \(u,v\in\Z\) tels que : \[ua+vb=a\et b.\]

Une telle écriture s'appelle une relation de Bézout.

Elle n'est pas unique.
\end{defprop}

\begin{dem}[Absence d'unicité]
Soient \(u,v\in\Z\) tels que \[ua+vb=a\et b.\]

On a alors \[\paren{u-b}a+\paren{v+a}b=a\et b.\]

Donc le couple \(\paren{u\prim,v\prim}=\paren{u-b,v+a}\) convient aussi.
\end{dem}

\subsubsection{Preuve algébrique}

\begin{prop}
Soient \(a,b\in\Z\).

Alors \(a\) et \(b\) admettent une relation de Bézout.
\end{prop}

\begin{dem}
On a vu que \(a\Z+b\Z=\paren{a\et b}\Z\).

On a \(a\et b\in\paren{a\et b}\Z\) donc \(a\et b\in a\Z+b\Z\).

Donc on a \[\quantifs{\exists u,v\in\Z}a\et b=ua+bv.\]
\end{dem}

\subsubsection{Preuve algorithmique, algorithme d'Euclide étendu}

Écrivons \(\bezout[a][b]\) où \(a,b\in\Z\) et qui renvoie \(u,v,d\) tels que \(\begin{dcases}d=a\et b \\ ua+vb=a\et b\end{dcases}\)

\begin{algo}[Algorithme d'Euclide étendu, en langage Python]\thlabel{algo:EuclideEtenduEntiers}
\begin{code}
def bezout(a, b):
	if a < 0:
		u, v, d = bezout(-a, b)
		return -u, v, d
	if b < 0:
		u, v, d = bezout(a, -b)
		return u, -v, d
	elif b == 0:
		return 1, 0, a
	else:
		q, r = a // b, a % b
		u, v, d = bezout(b, r)
		return v, u - v * q, d
\end{code}
\end{algo}

\begin{ex}
Déterminons le PGCD et une relation de Bézout de \(14\) et \(9\) : \[\begin{WithArrows}
&\bezout[14][9]\text{ (\(q=1\), \(r=5\))} \Arrow[jump=9]{} \\
&\hookrightarrow\bezout[9][5]\text{ (\(q=1\), \(r=4\))} \Arrow[jump=7]{} \\
&\color{white}\hookrightarrow\color{black}\hookrightarrow\bezout[5][4]\text{ (\(q=1\), \(r=1\))} \Arrow[jump=5]{}  \\
&\color{white}\hookrightarrow\hookrightarrow\color{black}\hookrightarrow\bezout[4][1]\text{ (\(q=4\), \(r=0\))} \Arrow[jump=3]{} \\
&\color{white}\hookrightarrow\hookrightarrow\hookrightarrow\color{black}\hookrightarrow\bezout[1][0] \Arrow{} \\
&\color{white}\hookrightarrow\hookrightarrow\hookrightarrow\hookrightarrow\color{black}1,0,1 \\
&\color{white}\hookrightarrow\hookrightarrow\hookrightarrow\color{black}0,1,1 \\
&\color{white}\hookrightarrow\hookrightarrow\color{black}1,-1,1 \\
&\color{white}\hookrightarrow\color{black}-1,2,1 \\
&2,-3,1
\end{WithArrows}\]

Donc \[14\et9=1\qquad\text{et}\qquad1=2\times14-3\times9.\]
\end{ex}

\subsection{PGCD de plusieurs entiers}

\begin{rem}
La loi \(\et\) est une loi de composition interne associative et commutative sur \(\Z\).
\end{rem}

\begin{dem}\thlabel{dem:pgcdLCIsurZ}
La loi \(\et\) est clairement commutative.

Montrons qu'elle est associative.

Soient \(a,b,c\in\Z\).

On a \[\ensdiv{a}\inter\ensdiv{b}\inter\ensdiv{c}=\ensdiv{a}\inter\ensdiv{b\et c}=\ensdiv{a\et\paren{b\et c}}\] et \[\ensdiv{a}\inter\ensdiv{b}\inter\ensdiv{c}=\ensdiv{a\et b}\inter\ensdiv{c}=\ensdiv{\paren{a\et b}\et c}.\]

Donc \(a\paren{b\et c}\) et \(\paren{a\et b}\et c\) se divisent mutuellement.

De plus, ce sont des entiers positifs.

Donc ils sont égaux.
\end{dem}

\begin{defprop}
Soient \(r\in\Ns\) et \(a_1,\dots,a_r\in\Z\).

Les diviseurs positifs communs à \(a_1,\dots,a_r\) sont les diviseurs de \(a_1\et\dots\et a_r\).

Ce nombre est le PGCD de \(a_1,\dots,a_r\).
\end{defprop}

\begin{prop}
Soient \(r\in\Ns\) et \(a_1,\dots,a_r,\lambda\in\Z\).

On a : \[\paren{\lambda a_1}\et\dots\et\paren{\lambda a_r}=\abs{\lambda}\paren{a_1\et\dots\et a_r}.\]
\end{prop}

\begin{dem}
Découle de la \thref{prop:PGCDFoisLambdaEgal} par récurrence sur \(r\in\Ns\).
\end{dem}

\section{Entiers premiers entre eux}\label{sec:entiersPremiersEntreEux}

\subsection{Cas de deux entiers}

\begin{defi}[Entiers premiers entre eux]
Deux entiers \(a,b\in\Z\) sont dits premiers entre eux s'ils vérifient : \[a\et b=1.\]

Cela signifie que leurs seuls diviseurs communs sont \(1\) et \(-1\).
\end{defi}

\begin{theo}[Théorème de Bézout]
Soient \(a,b\in\Z\).

On a : \[a\text{ et }b\text{ premiers entre eux}\ssi\quantifs{\exists u,v\in\Z}ua+vb=1.\]
\end{theo}

\begin{dem}
\impdir Déjà vu (\cf \thref{defprop:relationDeBezout}).

\imprec

Supposons qu'il existe \(u,v\in\Z\) tels que \(ua+bv=1\).

On a \(a\et b\) divise \(a\) et \(b\).

Donc \(a\et b\divise ua+vb\).

Donc \(a\et b\divise1\).

Donc \(a\et b=1\) car \(a\et b\in\N\).
\end{dem}

\begin{prop}
Soient \(\lambda,x,y,n\in\Z\).

On suppose que \(\lambda\) et \(n\) sont premiers entre eux.

Alors : \[x\equiv y\croch{n}\ssi\lambda x\equiv\lambda y\croch{n}.\]
\end{prop}

\begin{dem}
Soient \(u,v\in\Z\) tels que \(u\lambda+vn=1\).

On a \(u\lambda\equiv1\croch{n}\).

Montrons l'équivalence.

\impdir Vraie : il suffit de multiplier par \(\lambda\).

\imprec Vraie : il suffit de multiplier par \(u\).
\end{dem}

\begin{lem}[Lemme de Gauss]
Soient \(a,b,n\in\Z\).

On suppose : \[n\divise ab\qquad\text{et}\qquad n\et b=1.\]

Alors : \[n\divise a.\]
\end{lem}

\begin{dem}
Soient \(u,v\in\Z\) tels que \(un+vb=1\).

On a \(n\divise ab\). Soit \(k\in\Z\) tel que \(nk=ab\).

On a \(a\paren{un+vb}=a\) donc \(nau+vnk=a\).

Donc \(n\paren{au+vk}=a\).

Donc \(n\divise a\).
\end{dem}

\begin{rem}[Forme irréductible d'un rationnel]
Soit \(q\in\Q\).

Il existe un unique couple \(\paren{a,b}\in\Z\times\Ns\) tel que : \[q=\dfrac{a}{b}\qquad\text{et}\qquad a\et b=1.\]
\end{rem}

\begin{dem}\thlabel{dem:formeIrreductibleD'unRationnel}
Soit \(\paren{a,b}\in\Z\times\Ns\) tel que \(q=\dfrac{a}{b}\).

\existence

On pose \(a\prim=\dfrac{a}{a\et b}\) et \(b\prim=\dfrac{b}{a\et b}\) (on a \(a\prim\in\Z\) et \(b\prim\in\Ns\) car \(a\et b\) divise \(a\) et \(b\)).

On a \(q=\dfrac{a\prim}{b\prim}\).

Montrons que \(a\prim\et b\prim=1\).

Comme \(a\et b\geq0\), d'après la \thref{prop:PGCDFoisLambdaEgal}, on a \[a\et b=\paren{\paren{a\et b}a\prim}\et\paren{\paren{a\et b}\et b}=\paren{a\et b}\paren{a\prim\et b\prim}.\]

Or \(a\et b\not=0\) donc \(1=a\prim\et b\prim\).

\unicite

Soit \(\paren{a\seconde,b\seconde}\in\Z\times\Ns\) tel que \(\begin{dcases}a\seconde\et b\seconde=1 \\ q=\dfrac{a\seconde}{b\seconde}\end{dcases}\)

Montrons que \(\paren{a\prim,b\prim}=\paren{a\seconde,b\seconde}\).

On a \(\dfrac{a\prim}{b\prim}=\dfrac{a\seconde}{b\seconde}\) donc \(a\prim b\seconde=b\prim a\seconde\).

Donc \(b\seconde\divise b\prim a\seconde\) et \(b\seconde\et a\seconde=1\) donc \(b\seconde\divise b\prim\) selon le lemme de Gauss.

De même, \(b\prim\divise b\seconde\) donc \(b\prim=b\seconde\) donc \(a\prim=a\seconde\).
\end{dem}

\begin{prop}
Soient \(a,b,n\in\Z\).

On suppose \[a\et n=1\qquad\text{et}\qquad b\et n=1.\]

Alors \[ab\et n=1.\]
\end{prop}

\begin{dem}
Soient \(u,v,u\prim,v\prim\in\Z\) tels que \(\begin{dcases}ua+vn=1 \\ u\prim b+v\prim n=1\end{dcases}\)

On a, par produit : \[\underbrace{uu\prim}_{\in\Z}ab+n\underbrace{\paren{uav\prim+vbu\prim+vv\prim n}}_{\in\Z}=1\]

Donc \(ab\et n=1\).
\end{dem}

\begin{cor}
Soient \(r\in\Ns\) et \(a_1,\dots,a_r,n\in\Z\).

On suppose \[\quantifs{\forall k\in\interventierii{1}{r}}a_k\et n=1.\]

Alors \[\paren{a_1\dots a_r}\et n=1.\]
\end{cor}

\begin{dem}
Par récurrence immédiate sur \(r\in\Ns\).
\end{dem}

\begin{prop}\thlabel{prop:deuxEntiersPremiersEntreEuxDivisantUnNombreImpliqueProduitDiviseCeNombre}
Soient \(a,b,n\in\Z\).

On suppose \[a\divise n\qquad\text{et}\qquad b\divise n\qquad\text{et}\qquad a\et b=1.\]

Alors \[ab\divise n.\]
\end{prop}

\begin{dem}
Comme \(a\divise n\), il existe \(\lambda\in\Z\) tel que \(a\lambda=n\).

On a \(b\divise a\lambda\) et \(b\et a=1\).

Donc selon le lemme de Gauss : \(b\divise\lambda\).

Donc \(ab\divise a\lambda\).

Donc \(ab\divise n\).
\end{dem}

\subsection{Cas de plusieurs entiers}

\begin{defi}
Soient \(r\in\Ns\) et \(a_1,\dots,a_r\in\Z\).

On dit que les entiers \(a_1,\dots,a_r\) sont premiers entre eux deux à deux si on a : \[\quantifs{\forall i,j\in\interventierii{1}{r}}i\not=j\imp a_i\et a_j=1.\]

On dit que les entiers \(a_1,\dots,a_r\) sont premiers entre eux dans leur ensemble si \(a_1\et\dots\et a_r=1\). Cela signifie que leurs seuls diviseurs communs sont \(1\) et \(-1\).
\end{defi}

\begin{rem}
Soient \(r\in\N\excluant\accol{0;1}\) et \(a_1,\dots,a_r\in\Z\).

\guillemets{\(a_1,\dots,a_r\) sont premiers entre eux deux à deux} implique \guillemets{\(a_1,\dots,a_r\) sont premiers entre eux dans leur ensemble}.

L'implication réciproque est fausse, par exemple : \(6\), \(10\) et \(15\) sont premiers entre eux dans leur ensemble mais pas deux à deux.
\end{rem}

\section{PPCM}

\begin{defprop}
Soient \(a,b\in\Z\).

Dans l'ensemble ordonné \(\groupe{\N}[\divise]\), il existe un plus petit multiple commun positif à \(a\) et \(b\) appelé le \guillemets{Plus Petit Commun Multiple} (PPCM) et noté \(a\ou b\).

On a \(\begin{dcases}a\ou b\geq0 \\ a\divise a\ou b \\ b\divise a\ou b \\ \quantifs{\forall m\in\N}\croch{a\divise m\quad\text{et}\quad b\divise m}\imp a\ou b\divise m\end{dcases}\)
\end{defprop}

\begin{dem}
L'ensemble \(a\Z\inter b\Z\) est un sous-groupe de \(\groupe{\Z}\).

Donc il existe \(\mu\in\N\) tel que \(a\Z\inter b\Z=\mu\Z\).

On a \(\mu\geq0\).

Comme \(\mu\in a\Z\inter b\Z\), on a \(\begin{dcases}a\divise\mu \\ b\divise\mu\end{dcases}\)

Enfin, soit \(m\in\N\) tel que \(\begin{dcases}a\divise m \\ b\divise m\end{dcases}\)

On a \(m\in a\Z\inter b\Z\) donc \(m\in\mu\Z\).

Donc \(\mu\divise m\).
\end{dem}

\begin{rem}
On a :

\begin{itemize}
\item \(\quantifs{\forall a,b\in\Z}a\ou b=\abs{a}\ou\abs{b}\) \\

\item \(\quantifs{\forall a,b\in\N}a\ou b=\sup\accol{a;b}\) dans \(\groupe{\N}[\divise]\)
\end{itemize}
\end{rem}

\begin{prop}
\(\ou\) est une loi de composition interne associative et commutative sur \(\Z\).
\end{prop}

\begin{dem}
\(\ou\) est clairement commutative.

Montrons que \(\ou\) est associative.

Soient \(a,b,c\in\Z\).

D'une part, \[a\Z\inter b\Z\inter c\Z=\paren{a\ou b}\Z\inter c\Z=\paren{\paren{a\ou b}\ou c}\Z.\]

D'autre part, \[a\Z\inter b\Z\inter c\Z=a\Z\inter\paren{b\ou c}\Z=\paren{a\ou\paren{b\ou c}}\Z.\]

Donc \(\paren{a\ou b}\ou c\) et \(a\ou\paren{b\ou c}\) se divisent mutuellement.

De plus, ils appartiennent à \(\N\) donc ils égaux.
\end{dem}

\begin{defi}
Soient \(r\in\Ns\) et \(a_1,\dots,a_r\in\Z\).

L'élément \(a_1\ou\dots\ou a_r\) est le plus petit (pour \(\divise\)) multiple commun à \(a_1,\dots,a_r\).

On l'appelle le PPCM de \(a_1,\dots,a_r\).
\end{defi}

\begin{ex}
\(9\ou12=36\) car \(9\N=\accol{0;9;18;27;36;\dots}\) et \(12\N=\accol{0;12;24;36;\dots}\).

On a \[\quantifs{\forall n\in\Z}\begin{dcases}n\ou1=\abs{n} \\ n\ou0=0 \\ n\ou2=\begin{dcases}n &\text{si \(n\) est pair} \\ 2n &\text{sinon}\end{dcases}\end{dcases}\]
\end{ex}

\begin{rem}
Soient \(a,b\in\Z\).

On a \[a\ou b=\abs{b}\ssi a\divise b.\]
\end{rem}

\begin{dem}
\impdir

Supposons \(a\ou b=\abs{b}\).

On a \(a\divise\abs{b}\).

Or \(\abs{b}\divise b\).

Donc \(a\divise b\).

\imprec

Supposons \(a\divise b\).

On a \[\quantifs{\forall m\in\N}\croch{a\divise m\quad\text{et}\quad b\divise m}\ssi b\divise m.\]

Donc \(a\ou b=\abs{b}\).
\end{dem}

\begin{prop}
On a

\begin{enumerate}
\item \(\quantifs{\forall a,b,\lambda\in\Z}\paren{\lambda a}\ou\paren{\lambda b}=\abs{\lambda}\paren{a\ou b}\) \\

\item \(\quantifs{\forall r\in\Ns;\forall a_1,\dots,a_r,\lambda\in\Z}\paren{\lambda a_1}\ou\dots\ou\paren{\lambda a_r}=\abs{\lambda}\paren{a_1\ou\dots\ou a_r}\)
\end{enumerate}
\end{prop}

\begin{dem}[1]
Montrons que \(\paren{\lambda a}\ou\paren{\lambda b}\divise\lambda\paren{a\ou b}\).

On a \(\lambda a\) et \(\lambda b\) divisent \(\lambda\paren{a\ou b}\) donc \(\paren{\lambda a}\ou\paren{\lambda b}\divise\lambda\paren{a\ou b}\).

Montrons que \(\lambda\paren{a\ou b}\divise\paren{\lambda a}\ou\paren{\lambda b}\).

On a \(\lambda\divise\paren{\lambda a}\ou\paren{\lambda b}\) (car \(\lambda\divise\lambda a\divise\paren{\lambda a}\ou\paren{\lambda b}\)).

Soit \(m\in\Z\) tel que \(\lambda m=\paren{\lambda a}\ou\paren{\lambda b}\).

\Cad, en supposant \(\lambda\not=0\) : \(a\ou b\divise m\).

On a \(\lambda a\) et \(\lambda b\) divisent \(\paren{\lambda a}\ou\paren{\lambda b}\) donc \(\lambda a\) et \(\lambda b\) divisent \(\lambda m\).

Donc \(a\) et \(b\) divisent \(m\) car \(\lambda\not=0\).

Donc \(a\ou b\divise m\).

Donc \(\lambda\paren{a\ou b}\divise\lambda m\).

Finalement, \(\paren{\lambda a}\ou\paren{\lambda b}\) et \(\lambda\paren{a\ou b}\) se divisent mutuellement donc ils ont même valeur absolue.
\end{dem}

\begin{dem}[2]
Découle de (1) par récurrence sur \(r\in\Ns\).
\end{dem}

\begin{prop}
Soient \(a,b\in\Z\).

On a \[\paren{a\ou b}\paren{a\et b}=\abs{ab}.\]
\end{prop}

\begin{dem}
Si \(a\et b=1\) :

On a, selon la \thref{prop:deuxEntiersPremiersEntreEuxDivisantUnNombreImpliqueProduitDiviseCeNombre} : \[\quantifs{\forall m\in\Z}\begin{dcases}a\divise m \\ b\divise m\end{dcases}\ssi ab\divise m.\]

Donc \(a\ou b=\abs{ab}\) donc \(\paren{a\ou b}\paren{a\et b}=\abs{ab}\).

Sinon :

On pose \(a\prim=\dfrac{a}{a\et b}\) et \(b\prim=\dfrac{b}{a\et b}\).

Alors \(a\prim\in\Z\) et \(b\prim\in\Ns\) (\cf \thref{dem:formeIrreductibleD'unRationnel}).

On a donc, selon le cas précédent : \(\paren{a\prim\ou b\prim}\paren{a\prim\et b\prim}=\abs{a\prim b\prim}\).

Puis, en multipliant de chaque côté par \(\paren{a\et b}^2\) : \[\croch{\underbrace{\paren{\paren{a\et b}a\prim}}_{a}\et\underbrace{\paren{\paren{a\et b}b\prim}}_b}\croch{\underbrace{\paren{\paren{a\et b}a\prim}}_a\ou\underbrace{\paren{\paren{a\et b}b\prim}}_b}=\abs{\paren{a\et b}a\prim\paren{a\et b}b\prim}=\abs{ab}.\]
\end{dem}

\section{Nombres premiers}

\subsection{Définition}

\begin{defi}[Nombre premier]
On appelle nombre premier tout entier \(p\in\interventierie{2}{\pinf}\) dont les seuls diviseurs positifs sont \(1\) et lui-même.

L'ensemble des nombres premiers est souvent noté \(\prem\) : \[\quantifs{\forall p\in\Z}p\in\prem\ssi\croch{p\geq2\quad\text{et}\quad\ensdiv{p}=\accol{1;p}}.\]
\end{defi}

\begin{defi}[Nombre composé]
On appelle nombre composé tout entier \(n\in\interventierie{2}{\pinf}\) qui n'est pas un nombre premier, \cad vérifiant : \[\quantifs{\exists a,b\in\interventierie{2}{\pinf}}ab=n.\]
\end{defi}

\begin{ex}
Les entiers \(0\) et \(1\) ne sont ni des nombres premiers, ni des nombres composés.

Les entiers \(2\), \(3\), \(5\), \(7\), \(11\) sont des nombres premiers.

Les entiers \(4\), \(6\), \(8\), \(9\), \(10\) sont des nombres composés.
\end{ex}

\begin{rem}
Soient \(p\in\prem\) et \(n\in\Z\).

On a \[p\not\divise n\ssi p\et n=1.\]
\end{rem}

\begin{dem}
Si \(p\divise n\) alors \(p\et n=\abs{n}\).

Si \(p\not\divise n\) alors \(\begin{dcases}p\et n\in\ensdiv{p} \\ p\et n\in\ensdiv{n}\end{dcases}\)

Donc \(\begin{dcases}p\et n\in\accol{1;p} \\ p\et n\not=p\end{dcases}\)

Donc \(p\et n=1\).

Finalement, \(p\et n=1\ssi p\not\divise n\).
\end{dem}

\begin{lem}
Soit \(n\in\interventierie{2}{\pinf}\).

Alors \(n\) admet un diviseur premier.
\end{lem}

\begin{dem}
On raisonne par récurrence forte sur \(n\).

On pose \[\quantifs{\forall n\in\interventierie{2}{\pinf}}\P{n}:\croch{\quantifs{\exists p\in\prem}p\divise n}.\]

On a \(2\divise 2\). D'où \(\P{2}\).

Soit \(n\in\interventierie{3}{\pinf}\) tel que \(\quantifs{\forall k\in\interventierii{2}{n-1}}\P{k}\).

Montrons \(\P{n}\).

Si \(n\in\prem\) alors on a \(\P{n}\) car \(n\divise n\).

Sinon, \(n\) admet un diviseur positif autre que \(1\) et lui-même qu'on note \(d\).

On a \(1<d<n\).

Selon \(\P{d}\), il existe \(p\in\prem\) tel que \(p\divise d\).

D'où \(p\divise n\).

D'où, par récurrence forte, \(\quantifs{\forall n\in\interventierie{2}{\pinf}}\P{n}\).
\end{dem}

\begin{theo}
Il existe une infinité de nombres premiers.
\end{theo}

\begin{dem}
Soit \(n\in\N\).

Posons \(N=n!+1\). On a \(N\geq2\).

Soit \(p\in\prem\) tel que \(p\divise N\) (un tel \(p\) existe selon le lemme précédent).

On a \(p\divise n!+1\).

Montrons que \(p>n\).

Par l'absurde, si \(p\leq n\) alors \(p\divise n!\) donc \(p\divise1\) : contradiction.

On a donc montré \(\quantifs{\forall n\in\N;\exists p\in\prem}p>n\).

Donc \(\prem\) n'est pas majorée.

Donc \(\prem\) n'est pas finie.
\end{dem}

\begin{rem}[Crible d'Eratosthène]
Soit \(N\in\interventierie{2}{\pinf}\).

Tout nombre composé \(n\in\interventierii{1}{N}\) admet un diviseur premier \(p\) tel que \(p\leq N\).
\end{rem}

\begin{dem}
Soit un nombre composé \(n\in\interventierii{1}{N}\).

Il existe \(a,b\in\N\) tels que \(\begin{dcases}ab=n \\ 2\leq a<n \\ 2\leq b<n\end{dcases}\)

Quitte à échanger \(a\) et \(b\), on peut supposer \(a\leq\sqrt{n}\).

Soit, selon le lemme précédent, \(p\in\prem\) tel que \(p\divise a\).

On a \(\begin{dcases}p\divise n &\text{car }p\divise a\text{ et }a\divise n \\ p\leq\sqrt{n} &\text{car }p\leq a\leq\sqrt{n}\end{dcases}\)

On peut ainsi déterminer tous les nombres premiers de \(\interventierii{1}{N}\).
\end{dem}

\begin{ex}
Avec \(N=25\) (et donc \(\sqrt{N}=5\)) :

\begin{center}
\begin{tabular}{ccccc}
\textcolor{red}{1} & \textcolor{green}{2} & \textcolor{green}{3} & \textcolor{red}{4} & \textcolor{green}{5} \\
\textcolor{red}{6} & \textcolor{green}{7} & \textcolor{red}{8} & \textcolor{red}{9} & \textcolor{red}{10} \\
\textcolor{green}{11} & \textcolor{red}{12} & \textcolor{green}{13} & \textcolor{red}{14} & \textcolor{red}{15} \\
\textcolor{red}{16} & \textcolor{green}{17} & \textcolor{red}{18} & \textcolor{green}{19} & \textcolor{red}{20} \\
\textcolor{red}{21} & \textcolor{red}{22} & \textcolor{green}{23} & \textcolor{red}{24} & \textcolor{red}{25}
\end{tabular}
\end{center}
\end{ex}

\subsection{Théorème fondamental}

Le théorème suivant affirme que tout entier strictement positif s'écrit comme produit de nombres premiers, de façon unique à l'ordre des facteurs près.

\begin{theo}[Théorème fondamental de l'arithmétique]
Soit \(n\in\Ns\).

Alors \(n\) s'écrit de façon unique sous la forme : \[n=p_1^{\alpha_1}p_2^{\alpha_2}\dots p_r^{\alpha_r}\qquad\text{où}\qquad\begin{dcases}r\in\N \\ p_1,p_2,\dots,p_r\in\prem \\ p_1<p_2<\dots<p_r \\ \alpha_1,\alpha_2,\dots,\alpha_r\in\Ns\end{dcases}\]
\end{theo}

\begin{dem}
Pour tout \(n\in\Ns\), on note \(\P{n}\) la proposition \guillemets{le théorème est vrai pour \(n\)}.

On raisonne par récurrence forte.

On a clairement \(\P{1}\) car \(1\) est le \guillemets{produit vide} (avec \(r=0\)), et seulement le produit vide.

On a clairement \(\P{2}\) car \(2=p_1^{\alpha_1}\) avec \(r=1\), \(p_1=2\) et \(\alpha_1=1\).

Soit \(n\in\interventierie{3}{\pinf}\) tel que \(\quantifs{\forall k\in\interventierii{1}{n-1}}\P{k}\).

Montrons \(\P{n}\).

\existence

Si \(n\in\prem\) alors l'existence est claire.

Sinon, il existe \(a,b\in\N\) tels que \(\begin{dcases}n=ab \\ 2\leq a<n \\ 2\leq b<n\end{dcases}\)

Selon \(\P{a}\), l'entier \(a\) est produit de nombres premiers.

Selon \(\P{b}\), l'entier \(b\) est produit de nombres premiers.

Donc \(n\) est produit de nombres premiers.

\unicite

Soient \(r,s\in\N\), \(p_1,\dots,p_r,q_1,\dots,q_s\in\prem\) et \(\alpha_1,\dots,\alpha_r,\beta_1,\dots,\beta_s\in\Ns\) tels que \[\begin{dcases}n=p_1^{\alpha_1}\dots p_r^{\alpha_r}=q_1^{\beta_1}\dots q_s^{\beta_s} \\ p_1<\dots<p_r \\ q_1<\dots<q_s\end{dcases}\]

Montrons que \(p_1=q_1\).

Par l'absurde, supposons \(p_1\not=q_1\).

Si \(p_1<q_1\) alors \(\quantifs{\forall l\in\interventierii{1}{s}}p_1<q_l\).

Donc \(\quantifs{\forall l\in\interventierii{1}{s}}p_1\et q_l=1\).

D'où par produit, selon la \thref{prop:deuxEntiersPremiersEntreEuxDivisantUnNombreImpliqueProduitDiviseCeNombre}, \(p_1\et\paren{q_1^{\beta_1}\dots q_s^{\beta_s}}=1\).

Donc \(p_1\et n=1\) et \(p_1\divise n\) : contradiction.

Si \(q_1<p_1\), on montre de même que \(q_1\et n=1\) : contradiction.

Donc \(p_1=q_1\).

Ainsi, \(\dfrac{n}{p_1}=p_1^{\alpha_1-1}\dots p_r^{\alpha_r}=q_1^{\beta_1-1}\dots q_s^{\beta_s}\).

Selon la partie \unicite de \(\P{\dfrac{n}{p_1}}\), on a \[\begin{dcases}r=t \\ \quantifs{\forall k\in\interventierii{2}{s}}\begin{dcases}p_k=q_k \\ \alpha_k=\beta_k\end{dcases} \\ \alpha_1=\beta_1\end{dcases}\]

D'où l'unicité.

D'où \(\P{n}\).

Donc \(\quantifs{\forall n\in\Ns}\P{n}\).
\end{dem}

\subsection{Valuations \(p\)-adiques}

\begin{defi}[Valuation \(p\)-adique]
Pour tout entier strictement positif \(n\in\Ns\) et tout nombre premier \(p\in\prem\), on note \(\valp{p}{n}\) l'exposant de \(p\) dans la décomposition de \(\abs{n}\) en produit de nombres premiers.
\end{defi}

\begin{ex}
On a \(40=2^3\times5\).

Donc \[\valp{2}{40}=3\qquad\valp{5}{40}=1\qquad\quantifs{\forall p\in\prem\excluant\accol{2;5}}\valp{p}{40}=0.\]
\end{ex}

\begin{nota}[Fonction signum]
On pose \[\fonction{\sg}{\R}{\R}{x}{\begin{dcases}1 &\text{si }x>0 \\ 0 &\text{si }x=0 \\ -1 &\text{si }x<0\end{dcases}}\]
\end{nota}

\begin{nota}
Soit \(n\in\Zs\).

La valuation \(p\)-adique de \(n\) est nulle pour tous les nombres premiers \(p\in\prem\) sauf un nombre fini d'entre eux (on dit que la famille \(\paren{\valp{p}{n}}_{p\in\prem}\) est une famille d'entiers \guillemets{presque tous nuls}).

Pour cette raison, on s'autorise à définir le \guillemets{produit infini} : \[\prod_{p\in\prem}p^{\valp{p}{n}}\] comme étant le produit de ses facteurs différents de \(1\).

Sa valeur est donc en fait un produit fini : \[\prod_{p\in\prem}p^{\valp{p}{n}}=\prod_{\substack{p\in\prem \\ \valp{p}{n}\not=0}}p^{\valp{p}{n}}.\]
\end{nota}

\begin{cor}
On a, selon le théorème fondamental de l'arithmétique : \[\quantifs{\forall n\in\Zs}n=\sg\paren{n}\times\prod_{p\in\prem}p^{\valp{p}{n}}.\]
\end{cor}

\begin{prop}[Valuations d'un produit]
Soient \(a,b\in\Zs\).

On a : \[\quantifs{\forall p\in\prem}\valp{p}{ab}=\valp{p}{a}+\valp{p}{b}.\]
\end{prop}

\begin{dem}~\\
On a \(\begin{dcases}a=\prod_{p\in\prem}p^{\valp{p}{a}} \\ b=\prod_{p\in\prem}p^{\valp{p}{b}}\end{dcases}\) donc \[ab=\prod_{p\in\prem}p^{\valp{p}{a}+\valp{p}{b}}=\prod_{p\in\prem}p^{\valp{p}{ab}}.\]

D'où l'égalité, par unicité de l'écriture de \(ab\) en produit de nombres premiers.
\end{dem}

\begin{prop}[Caractérisation de la divisibilité]\thlabel{prop:caractérisationDeLaDivisibilité}
Soient \(a,b\in\Zs\).

On a \[a\divise b\ssi\quantifs{\forall p\in\prem}\valp{p}{a}\leq\valp{p}{b}.\]
\end{prop}

\begin{dem}
\imprec

Supposons \(\quantifs{\forall p\in\prem}\valp{p}{a}\leq\valp{p}{b}\).

Alors on a \[b=\prod_{p\in\prem}p^{\valp{p}{b}}=\prod_{p\in\prem}p^{\valp{p}{a}}\times\prod_{p\in\prem}p^{\valp{p}{b}-\valp{p}{a}}=\abs{a}\times\underbrace{\prod_{p\in\prem}p^{\overbrace{\valp{p}{b}-\valp{p}{a}}^{\text{entiers presque tous nuls}}}}_{\in\N}\]

Donc \(\abs{a}\divise\abs{b}\).

Donc \(a\divise b\).

\imprec

Supposons \(a\divise b\).

Soit \(c\in\Z\) tel que \(ac=b\).

Comme \(b\not=0\), on a \(c\not=0\).

On a \[\paren{\sg\paren{a}\times\prod_{p\in\prem}p^{\valp{p}{a}}}\paren{\sg\paren{c}\times\prod_{p\in\prem}p^{\valp{p}{c}}}=\sg\paren{b}\times\prod_{p\in\prem}p^{\valp{p}{b}}\]

D'où, selon l'unicité de l'écriture d'un entier en produit de nombres premiers : \[\quantifs{\forall p\in\prem}\valp{p}{a}+\underbrace{\valp{p}{c}}_{\in\N}=\valp{p}{b}.\]

Donc \(\quantifs{\forall p\in\prem}\valp{p}{a}\leq\valp{p}{b}\).
\end{dem}

\begin{rem}[Caractérisation de \(\valp{p}{n}\)]
En particulier, si \(n\in\Zs\) et \(p\in\prem\), alors \[\quantifs{\forall\alpha\in\N}\alpha=\valp{p}{n}\ssi\begin{dcases}p^\alpha\divise n \\ p^{\alpha+1}\not\divise n\end{dcases}\]
\end{rem}

\begin{prop}[PGCD et PPCM]
Soient \(a,b\in\Zs\).

On a : \[a\et b=\prod_{p\in\prem}p^{\min\accol{\valp{p}{a};\valp{p}{b}}}\qquad\text{et}\qquad a\ou b=\prod_{p\in\prem}p^{\max\accol{\valp{p}{a};\valp{p}{b}}}\]
\end{prop}

\begin{dem}
Posons \(d=\prod_{p\in\prem}p^{\min\accol{\valp{p}{a};\valp{p}{b}}}\).

Montrons que \(a\et b=d\).

On a \(d\divise a\) et \(d\divise b\) selon la \thref{prop:caractérisationDeLaDivisibilité}.

On a \(d\geq0\).

Enfin, soit \(e\in\N\) tel que \(e\divise a\) et \(e\divise b\).

On a \(e\not=0\) (car \(a\not=0\)).

On a donc, selon la \thref{prop:caractérisationDeLaDivisibilité} : \[\quantifs{\forall p\in\prem}\begin{dcases}\valp{p}{e}\leq\valp{p}{a} \\ \valp{p}{e}\leq\valp{p}{b}\end{dcases}\]

Donc \(\quantifs{\forall p\in\prem}\valp{p}{e}\leq\min\accol{\valp{p}{a};\valp{p}{b}}=\valp{p}{d}\).

D'où \(e\divise d\) selon la \thref{prop:caractérisationDeLaDivisibilité}.

Finalement, \(d=a\et b\).

Idem pour \(a\ou b\).
\end{dem}

\section{Petit théorème de Fermat}

\begin{lem}
Soient \(p\in\prem\) et \(k\in\interventierii{1}{p-1}\).

Alors \(p\divise\binom{k}{p}\).
\end{lem}

\begin{dem}~\\
On a \(\binom{k}{p}=\dfrac{p!}{k!\,\paren{p-k}!}\).

Donc \(p!=\binom{k}{p}k!\,\paren{p-k}!\).

Donc \(p\divise\binom{k}{p}k!\,\paren{p-k}!\).

Or \(p\et k!=1\) car \(k<p\) et \(p\et\paren{p-k}!=1\) car \(k>0\).

Donc \(p\et k!\,\paren{p-k}!=1\).

Donc \(p\divise\binom{k}{p}\) selon le lemme de Gauss.
\end{dem}

\begin{theo}[Petit théorème de Fermat]\thlabel{theo:petitThéorèmeDeFermat}
Soit \(p\in\prem\).

On a, pour tout \(x\in\Z\) : \[x^p\equiv x\croch{p}.\]

De plus, pour tout entier \(x\) qui n'est pas divisible par \(p\) : \[x^{p-1}\equiv1\croch{p}.\]
\end{theo}

\begin{dem}
Montrons que \(\quantifs{\forall x\in\N}\underbrace{x^p\equiv x\croch{p}}_{\P{x}}\) par récurrence sur \(x\in\N\).

On a clairement \(\P{0}\) et \(\P{1}\).

Soit \(x\in\N\) tel que \(\P{x}\).

Montrons \(\P{x+1}\).

On a \[\begin{WithArrows}
\paren{x+1}^p&\equiv\sum_{k=0}^p\binom{k}{p}x^k\croch{p} \Arrow{car \(\binom{k}{n}\equiv0\croch{p}\) si \(1\leq k\leq p-1\)} \\
&\equiv x^0+x^p\croch{p} \Arrow{selon \(\P{x}\)} \\
&\equiv1+x\croch{p}
\end{WithArrows}\]

D'où \(\P{x+1}\).

Ainsi, \(\quantifs{\forall x\in\N}x^p\equiv x\croch{p}\).

Enfin, si \(x\in\Z\) alors il existe \(y\in\N\) tel que \(x\equiv y\croch{p}\).

On a alors \(x^p\equiv y^p\equiv y\equiv x\croch{p}\).
\end{dem}

\chapter{Fonctions dérivables}

\minitoc

Dans tout ce chapitre, on note \(I\) un intervalle de \(\R\) qui contient au moins deux points (\cad un intervalle qui n'est ni l'ensemble vide, ni un singleton).

\section{Étude locale}

\subsection{Définitions}

\begin{defi}[Dérivée]
Soient \(f:I\to\R\) et \(a\in I\).

On dit que \(f\) est dérivable en \(a\) si la limite \[\lim_{\substack{x\to a \\ x\not=a}}\dfrac{f\paren{x}-f\paren{a}}{x-a}\] existe et est finie.

Elle est alors notée \(f\prim\paren{a}\), \(D f\paren{a}\), \(\odv{f}{x}\paren{a}\) ou \(\odv{f}{t}\paren{a}\).

Notons \(J\) l'ensemble des points (de \(I\)) où \(f\) est dérivable.

On appelle dérivée de \(f\) la fonction \[\fonctionlambda{J}{\R}{x}{f\prim\paren{x}}\]

On dit que \(f\) est dérivable (sur \(I\)) si \(f\) est dérivable en tout point de \(I\) (\cad \(J=I\)).
\end{defi}

\begin{defi}[Dérivée à droite, à gauche]
Soient \(f:I\to\R\) et \(a\in I\).

On dit que \(f\) est dérivable à droite en \(a\) si la limite \[\lim_{x\to a^+}\dfrac{f\paren{x}-f\paren{a}}{x-a}\] existe et est finie (il faut donc que l'intervalle \(I\inter\intervee{a}{\pinf}\) soit non vide).

Cette limite est alors notée \(f_d\prim\paren{a}\).

On dit que \(f\) est dérivable à gauche en \(a\) si la limite \[\lim_{x\to a^-}\dfrac{f\paren{x}-f\paren{a}}{x-a}\] existe et est finie (il faut donc que l'intervalle \(I\inter\intervee{\minf}{a}\) soit non vide).

Cette limite est alors notée \(f_g\prim\paren{a}\).
\end{defi}

\begin{rem}
Si \(I=\intervii{a}{b}\), où \(a\) et \(b\) sont deux réels tels que \(a<b\), alors :

La fonction \(f\) est dérivable en \(a\) si, et seulement si, elle est dérivable à droite en \(a\).

La fonction \(f\) est dérivable en \(b\) si, et seulement si, elle est dérivable à gauche en \(b\).

Soit \(c\in\intervee{a}{b}\). La fonction \(f\) est dérivable en \(c\) si, et seulement si, elle est dérivable à droite et à gauche en \(c\) et \(f_d\prim\paren{c}=f_g\prim\paren{c}\).
\end{rem}

\begin{ex}
La fonction \guillemets{carré} \(x\mapsto x^2\) est dérivable sur \(\R\).

En \(0\), la fonction \guillemets{valeur absolue} \(x\mapsto\abs{x}\) est dérivable à droite et à gauche, mais n'est pas dérivable.
\end{ex}

\begin{prop}[Dérivable \(\imp\) continue]
Soient \(f:I\to\R\) et \(a\in I\).

Si \(f\) est dérivable en \(a\), alors \(f\) est continue en \(a\).

Si \(f\) est dérivable, alors \(f\) est continue.
\end{prop}

\begin{dem}
Supposons \(f\) dérivable en \(a\).

Alors \(\lim_{\substack{x\to a \\ x\not=a}}\dfrac{f\paren{x}-f\paren{a}}{x-a}\) existe et est finie et \(\lim_{\substack{x\to a \\ x\not=a}}x-a=0\) donc par produit, \(\lim_{\substack{x\to a \\ x\not=a}}f\paren{x}-f\paren{a}=0\).

Enfin, \(f\paren{x}-f\paren{a}=0\) si \(x=a\).

Donc \(\lim_{x\to a}f\paren{x}=f\paren{a}\).

Donc \(f\) est continue.
\end{dem}

\begin{rem}
Soient \(f:I\to\R\) et \(a\in I\).

Si \(f\) est dérivable à droite en \(a\), alors \(f\) est continue à droite en \(a\).

Si \(f\) est dérivable à gauche en \(a\), alors \(f\) est continue à gauche en \(a\).
\end{rem}

\subsection{Opérations sur les dérivées}

\begin{prop}
Soient \(f,g\in\F{I}{\R}\) et \(a\in I\).

On suppose que \(f\) et \(g\) sont dérivables en \(a\).

Alors :

\begin{enumerate}
\item \(f+g\) est dérivable en \(a\) et on a : \[\paren{f+g}\prim\paren{a}=f\prim\paren{a}+g\prim\paren{a}.\] \\

\item Plus généralement, si \(\lambda,\mu\in\R\) alors \(\lambda f+\mu g\) est dérivable en \(a\) et on a : \[\paren{\lambda f+\mu g}\prim\paren{a}=\lambda f\prim\paren{a}+\mu g\prim\paren{a}.\] \\

\item \(fg\) est dérivable en \(a\) et on a : \[\paren{fg}\prim\paren{a}=f\prim\paren{a}g\paren{a}+f\paren{a}g\prim\paren{a}.\] \\

\item Si \(g\paren{a}\not=0\) alors \(\dfrac{1}{g}\) est dérivable en \(a\) et on a : \[\paren{\dfrac{1}{g}}\prim\paren{a}=\dfrac{-g\prim\paren{a}}{g^2\paren{a}}.\] \\

\item Si \(g\paren{a}\not=0\) alors \(\dfrac{f}{g}\) est dérivable en \(a\) et on a : \[\paren{\dfrac{f}{g}}\prim\paren{a}=\dfrac{f\prim\paren{a}g\paren{a}-f\paren{a}g\prim\paren{a}}{g^2\paren{a}}.\]
\end{enumerate}
\end{prop}

\begin{dem}[1 et 2]
Clair.
\end{dem}

\begin{dem}[3]
On a : \[\begin{aligned}
\quantifs{\forall x\in\R\excluant\accol{a}}\dfrac{f\paren{x}g\paren{x}-f\paren{a}g\paren{a}}{x-a}&=\dfrac{f\paren{x}g\paren{x}-f\paren{x}g\paren{a}+f\paren{x}g\paren{a}-f\paren{a}g\paren{a}}{x-a} \\
&=f\paren{x}\dfrac{g\paren{x}-g\paren{a}}{x-a}+\dfrac{f\paren{x}-f\paren{a}}{x-a}g\paren{a} \\
&\xrightarrow[x\to a]{}f\paren{a}g\prim\paren{a}+f\prim\paren{a}g\paren{a}.
\end{aligned}\]
\end{dem}

\begin{dem}[4]
On a : \[\begin{aligned}
\quantifs{\forall x\in\R\excluant\accol{a}}\dfrac{\frac{1}{g\paren{x}}-\frac{1}{g\paren{a}}}{x-a}&=\dfrac{1}{g\paren{x}}\times\dfrac{1}{g\paren{a}}\times\dfrac{g\paren{a}-g\paren{x}}{x-a} \\
&\xrightarrow[x\to a]{}\dfrac{-g\prim\paren{a}}{g^2\paren{a}}.
\end{aligned}\]

\(g\) est non-nulle au voisinage de \(a\) car \(g\paren{a}\not=0\) et \(g\) est continue en \(a\).
\end{dem}

\begin{dem}[5]~\\
\(f\) et \(\dfrac{1}{g}\) sont dérivables en \(a\) donc leur produit aussi.

On a : \[\begin{aligned}
\paren{f\times\dfrac{1}{g}}\prim\paren{a}&=f\prim\paren{a}\times\dfrac{1}{g\paren{a}}+f\paren{a}\times\paren{\dfrac{1}{g}}\prim\paren{a} \\
&=\dfrac{f\prim\paren{a}}{g\paren{a}}-\dfrac{f\paren{a}g\prim\paren{a}}{g^2\paren{a}} \\
&=\dfrac{f\prim\paren{a}g\paren{a}-f\paren{a}g\prim\paren{a}}{g^2\paren{a}}.
\end{aligned}\]
\end{dem}

\begin{cor}
Soient \(f,g\in\F{I}{\R}\) dérivables.

Alors les fonctions \(f+g\) et \(fg\) sont dérivables sur \(I\) et, si \(g\) ne s'annule pas, les fonctions \(\dfrac{1}{g}\) et \(\dfrac{f}{g}\) sont dérivables sur \(I\).
\end{cor}

\begin{prop}
Soient \(J\) un intervalle de \(\R\) contenant au moins deux points, \(f:I\to J\), \(g:J\to\R\) et \(a\in I\).

On suppose \(f\) dérivable en \(a\) et \(g\) dérivable en \(f\paren{a}\).

Alors \(g\rond f\) est dérivable en \(a\) et on a : \[\paren{g\rond f}\prim\paren{a}=f\prim\paren{a}g\prim\paren{f\paren{a}}.\]
\end{prop}

\begin{dem}
\note{Admis provisoirement} (\cf chapitre \guillemets{Relations de comparaison}).
\end{dem}

\begin{cor}
Soient \(J\) un intervalle de \(\R\) contenant au moins deux points, \(f:I\to J\) et \(g:J\to\R\) dérivables.

Alors \(g\rond f\) est dérivable et on a : \[\quantifs{\forall x\in I}\paren{g\rond f}\prim\paren{x}=f\prim\paren{x}g\prim\paren{f\paren{x}}.\]
\end{cor}

\begin{ex}
Étudions la fonction \(f:x\mapsto\sqrt{1+\ln x}\).

On a : \[\begin{aligned}
\quantifs{\forall x\in\R}f\paren{x}\text{ bien défini}&\ssi1+\ln x\geq0 \\
&\ssi\ln x\geq-1 \\
&\ssi x\geq\dfrac{1}{\e{}}.
\end{aligned}\]

Donc \(\quantifs{\forall x\in\intervee{\dfrac{1}{\e{}}}{\pinf}}f\prim\paren{x}=\dfrac{1}{x}\times\dfrac{1}{2\sqrt{1+\ln x}}=\dfrac{1}{2x\sqrt{1+\ln x}}\).
\end{ex}

\begin{theo}[Dérivée de la bijection réciproque]
Soit \(f:I\to\R\) dérivable et strictement monotone.

La fonction \(f:I\to f\paren{I}\) est bijective ; on note \(f\inv:f\paren{I}\to I\) sa bijection réciproque.

Soit \(y\in f\paren{I}\).

Si \(f\prim\paren{f\inv\paren{y}}\not=0\) alors \(f\inv\) est dérivable en \(y\) et on a : \[\paren{f\inv}\prim\paren{y}=\dfrac{1}{f\prim\paren{f\inv\paren{y}}}.\]

Si \(f\prim\paren{f\inv\paren{y}}=0\) alors \(f\inv\) n'est pas dérivable en \(y\). Son graphe admet une tangente verticale au point \(\paren{y,f\inv\paren{y}}\).
\end{theo}

\begin{dem}
\Cf TD.
\end{dem}

\subsection{Fonctions de classe \(\classe{k}\)}

\begin{defi}[Fonction de classe \(\classe{k}\)]
Soient \(f:I\to\R\) et \(k\in\N\).

On dit que \(f\) est de classe \(\classe{k}\) sur \(I\) si elle est \(k\) fois dérivable sur \(I\) et si sa dérivée \(k\)-ème est continue.

On dit que \(f\) est de classe \(\classe{\infty}\) sur \(I\) si elle est de classe \(\classe{n}\) sur \(I\) pour tout \(n\in\N\).
\end{defi}

\begin{nota}
Soit \(n\in\N\).

L'ensemble des fonctions de classe \(\classe{n}\) (respectivement \(\classe{\infty}\)) de \(I\) dans \(\R\) est noté : \[\ensclasse{n}{I}{\R}\text{ (respectivement \(\ensclasse{\infty}{I}{\R}\))}.\]

Si \(f\) est de classe \(\classe{n}\) sur \(I\), les dérivées successives de \(f\) sont notées : \[f\deriv{0}=f\qquad f\deriv{1}=f\prim\qquad f\deriv{2}=f\seconde\qquad f\deriv{3}\qquad\dots\qquad f\deriv{n}.\]

Elles vérifient la relation de récurrence : \[\quantifs{\forall k\in\interventierii{0}{n-1}}\paren{f\deriv{k}}\prim=\paren{f\prim}\deriv{k}=f\deriv{k+1}.\]
\end{nota}

\begin{rem}
La suite d'ensembles \(\paren{\ensclasse{n}{I}{\R}}_{n\in\N}\) peut être définie par récurrence par : \[\begin{dcases}\ensclasse{0}{I}{\R}\text{ est l'ensemble des fonctions continues de \(I\) dans \(\R\)} \\ \quantifs{\forall n\in\N}\ensclasse{n+1}{I}{\R}=\accol{f\in\F{I}{\R}\tq f\text{ est dérivable et }f\prim\in\ensclasse{n}{I}{\R}}\end{dcases}\]

Enfin, on peut poser : \[\ensclasse{\infty}{I}{\R}=\biginter_{n\in\N}\ensclasse{n}{I}{\R}.\]

On a \[\ensclasse{\infty}{I}{\R}\subset\dots\subset\ensclasse{2}{I}{\R}\subset\ensclasse{1}{I}{\R}\subset\ensclasse{0}{I}{\R}\subset\F{I}{\R}.\]
\end{rem}

\begin{prop}
Soient \(n\in\N\) et \(f,g\in\ensclasse{n}{I}{\R}\).

Alors leur somme \(f+g\) est de classe \(\classe{n}\) et on a : \[\paren{f+g}\deriv{n}=f\deriv{n}+g\deriv{n}.\]

Plus généralement, si \(\lambda,\mu\in\R\), alors \(\lambda f+\mu g\) est de classe \(\classe{n}\) et on a : \[\paren{\lambda f+\mu g}\deriv{n}=\lambda f\deriv{n}+\mu g\deriv{n}.\]
\end{prop}

\begin{dem}
\note{Exercice}
\end{dem}

\begin{prop}[Formule de Leibniz]
Soit \(n\in\N\) et \(f,g\in\ensclasse{n}{I}{\R}\).

Alors leur produit \(fg\) est de classe \(\classe{n}\) et on a : \[\paren{fg}\deriv{n}=\sum_{k=0}^n\binom{k}{n}f\deriv{k}g\deriv{n-k}.\]
\end{prop}

\begin{dem}
Pour tout \(n\in\N\), on note \(\P{n}\) la proposition : \[\quantifs{\forall f,g\in\ensclasse{n}{I}{\R}}fg\in\ensclasse{n}{I}{\R}\text{ et }\paren{fg}\deriv{n}=\sum_{k=0}^n\binom{k}{n}f\deriv{k}g\deriv{n-k}.\]

Si \(f\) et \(g\) sont continues, alors \(fg\) est continue et on a \(\paren{fg}\deriv{0}=fg=\binom{0}{0}f\deriv{0}g\deriv{0}=\sum_{k=0}^0\binom{k}{0}f\deriv{k}g\deriv{n-k}\).

D'où \(\P{0}\).

Soit \(n\in\N\) tel que \(\P{n}\). Montrons \(\P{n+1}\).

Soient \(f,g\in\ensclasse{n+1}{I}{\R}\).

Alors on a \(f,g\in\ensclasse{n}{I}{\R}\) donc selon \(\P{n}\), on a : \(\paren{fg}\deriv{n}=\sum_{k=0}^n\binom{k}{n}f\deriv{k}g\deriv{n-k}\).

Les fonctions \(f\deriv{k}\) et \(g\deriv{n-k}\) sont dérivables pour tout \(k\in\interventierii{0}{n}\).

On a donc : \[\begin{WithArrows}
\paren{\paren{fg}\deriv{n}}\prim&=\sum_{k=0}^n\binom{k}{n}\paren{f\deriv{k+1}g\deriv{n-k}+f\deriv{k}g\deriv{n-k+1}} \\
&=\sum_{k=0}^n\binom{k}{n}f\deriv{k+1}g\deriv{n-k}+\sum_{k=0}^n\binom{k}{n}f\deriv{k}g\deriv{n-k+1} \\
&=\sum_{k=1}^{n+1}\binom{k-1}{n}f\deriv{k}g\deriv{n-k+1}+\sum_{k=0}^n\binom{k}{n}f\deriv{k}g\deriv{n-k+1} \Arrow{car \(\binom{-1}{n}=\binom{n+1}{n}=0\)} \\
&=\sum_{k=0}^{n+1}\binom{k-1}{n}f\deriv{k}g\deriv{n-k+1}+\sum_{k=0}^{n+1}\binom{k}{n}f\deriv{k}g\deriv{n-k+1} \\
&=\sum_{k=0}^{n+1}\binom{k}{n}f\deriv{k}g\deriv{n-k+1}.
\end{WithArrows}\]

De plus, \(f\deriv{k}\) et \(g\deriv{n-k+1}\) sont continues pour tout \(k\in\interventierii{0}{n+1}\) donc \(\paren{fg}\deriv{n+1}\) est continue.

Donc \(fg\) est de classe \(\classe{n+1}\).

D'où \(\P{n+1}\).

Donc \(\quantifs{\forall n\in\N}\P{n}\).
\end{dem}

\begin{rem}
Soit \(n\in\N\).

Les ensembles \(\ensclasse{n}{I}{\R}\) et \(\ensclasse{\infty}{I}{\R}\) sont des sous-anneaux de \(\anneau{\F{I}{\R}}\).
\end{rem}

\begin{dem}
\note{Exercice}
\end{dem}

\begin{prop}[Autres opérations]
Soient \(n\in\N\).

Soient \(f,g\in\ensclasse{n}{I}{\R}\). Si \(g\) ne s'annule pas, alors les fonctions \(\dfrac{1}{g}\) et \(\dfrac{f}{g}\) sont de classe \(\classe{n}\).

Soient \(J\) un intervalle de \(\R\) contenant au moins deux points, \(f\in\ensclasse{n}{I}{J}\) et \(g\in\ensclasse{n}{J}{\R}\). Alors \(g\rond f\) est de classe \(\classe{n}\).

Soient \(J\) un intervalle de \(\R\) contenant au moins deux points et \(f\in\ensclasse{n}{I}{J}\) telle que \[\quantifs{\forall x\in I}f\prim\paren{x}\not=0.\] Alors la bijection réciproque de \(f\) est de classe \(\classe{n}\) sur \(J\).
\end{prop}

\begin{dem}[Composition et bijection réciproque]
\note{Admis}
\end{dem}

\begin{dem}[Quotient, non-exigible]
Pour tout \(n\in\N\), on note \(\P{n}\) la proposition \[\quantifs{\forall f,g\in\ensclasse{n}{I}{\R}}\croch{\quantifs{\forall x\in I}g\paren{x}\not=0}\imp\dfrac{f}{g}\in\ensclasse{n}{I}{\R}.\]

Le quotient de deux fonctions continues est continu donc on a \(\P{0}\).

Soit \(n\in\N\) tel que \(\P{n}\). Montrons \(\P{n+1}\).

Soient \(f,g\in\ensclasse{n+1}{I}{\R}\).

On a \(\paren{\dfrac{f}{g}}\prim=\dfrac{f\prim g-fg\prim}{g^2}\).

Or \(f\prim\) et \(g\prim\) sont de classe \(\classe{n}\) car \(f\) et \(g\) sont de classe \(\classe{n+1}\) et donc de classe \(\classe{n}\).

Donc \(f\prim g-fg\prim\) et \(g^2\) sont de classe \(\classe{n}\).

Donc selon \(\P{n}\), \(\dfrac{f\prim g-fg\prim}{g^2}\) est de classe \(\classe{n}\).

Donc \(\paren{\dfrac{f}{g}}\prim\) est de classe \(\classe{n}\).

Donc \(\dfrac{f}{g}\) est de classe \(\classe{n+1}\).

D'où \(\P{n+1}\).

Donc \(\quantifs{\forall n\in\N}\P{n}\).
\end{dem}

\subsection{Extrema locaux}

\begin{theo}
Soient \(f:I\to\R\) et \(a\in I\).

On suppose que \(f\) admet un extremum local en \(a\), que \(f\) est dérivable en \(a\) et que \(I\) est un voisinage de \(a\) (\cad : \(\quantifs{\exists\epsilon\in\Rps}\intervee{a-\epsilon}{a+\epsilon}\subset I\)).

Alors on a : \[f\prim\paren{a}=0.\]
\end{theo}

\begin{dem}
Soit \(\epsilon\in\Rps\) tel que \(\intervee{a-\epsilon}{a+\epsilon}\subset I\).

Quitte à remplacer \(f\) par \(-f\), on suppose que \(f\) admet un minimum local en \(a\).

Soit \(\delta\in\intervee{0}{\epsilon}\) tel que \(\quantifs{\forall x\in\intervee{a-\delta}{a+\delta}}f\paren{a}\leq f\paren{x}\).

On a donc \(\begin{dcases}\quantifs{\forall x\in\intervee{a}{a+\delta}}\dfrac{f\paren{x}-f\paren{a}}{x-a}\geq0 \\ \quantifs{\forall x\in\intervee{a-\delta}{a}}\dfrac{f\paren{x}-f\paren{a}}{x-a}\leq0\end{dcases}\)

D'où, par passage à la limite quand \(x\to a\) : \(\begin{dcases}f\prim\paren{a}\geq0 \\ f\prim\paren{a}\leq0\end{dcases}\)

Donc \(f\prim\paren{a}=0\).
\end{dem}

\begin{rem}
Ainsi, si \(I\) est un intervalle ouvert et si la fonction \(f:I\to\R\) est dérivable, on a, pour tout \(a\in I\) : \[f\text{ admet un extremum local en }a\imp f\prim\paren{a}=0.\]

L'implication réciproque est généralement fausse, comme le montre l'exemple de la fonction \guillemets{cube} \(f:x\mapsto x^3\). En effet, on a \(f\prim\paren{0}=0\) mais \(f\) n'admet ni minimum local, ni maximum local en \(0\) (car \(\quantifs{\forall x\in\Rms}f\paren{x}<f\paren{0}\) et \(\quantifs{\forall x\in\Rps}f\paren{x}>f\paren{0}\)).
\end{rem}

\section{Étude globale}

\subsection{Égalité des accroissements finis}

\begin{theo}[Théorème de Rolle]
Soient \(a,b\in\R\) tels que \(a<b\) et \(f:\intervii{a}{b}\to\R\).

On suppose que \(\begin{dcases}f\text{ est continue sur }\intervii{a}{b} \\ f\text{ est dérivable sur }\intervee{a}{b} \\ f\paren{a}=f\paren{b}\end{dcases}\)

Alors on a \[\quantifs{\exists c\in\intervee{a}{b}}f\prim\paren{c}=0.\]
\end{theo}

\begin{dem}
Comme \(f\) est continue sur un segment, elle admet un minimum \(m\in\R\) et un maximum \(M\in\R\).

Si \(M=m\) alors \(f\) est constante donc \(f\prim=0\) sur \(\intervee{a}{b}\) donc \(c=\dfrac{a+b}{2}\) convient.

Supposons \(M\not=m\).

Si \(f\paren{a}=m\) alors \(f\paren{a}\not=M\).

Soit \(c\in\intervii{a}{b}\) tel que \(f\paren{c}=M\).

On a \(c\not=a\) et \(c\not=b\).

Donc \(\begin{dcases}f\text{ est dérivable en }c \\ f\text{ admet un extremum (local) en }c \\ \intervee{a}{b}\text{ est un voisinage de }c\end{dcases}\)

Donc \(f\prim\paren{c}=0\), donc \(c\) convient.

Si \(f\paren{a}\not=m\) : idem en considérant \(c\in\intervii{a}{b}\) tel que \(f\paren{c}=m\).
\end{dem}

\begin{cor}[Égalité des accroissements finis]
Soient \(a,b\in\R\) tels que \(a<b\) et \(f:\intervii{a}{b}\to\R\).

On suppose que \(\begin{dcases}f\text{ est continue sur }\intervii{a}{b} \\ f\text{ est dérivable sur }\intervee{a}{b}\end{dcases}\)

Alors on a \[\quantifs{\exists c\in\intervee{a}{b}}f\prim\paren{c}=\dfrac{f\paren{b}-f\paren{a}}{b-a}.\]
\end{cor}

\begin{dem}
Posons \(\fonction{g}{\intervii{a}{b}}{\R}{x}{f\paren{x}-\dfrac{f\paren{b}-f\paren{a}}{b-a}x}\)

On remarque que \(\begin{dcases}g\text{ est continue sur }\intervii{a}{b} \\ g\text{ est dérivable sur }\intervee{a}{b} \\ g\paren{b}=g\paren{a}\end{dcases}\)

Donc selon le théorème de Rolle, il existe \(c\in\intervee{a}{b}\) tel que \(g\prim\paren{c}=0\).

Donc \(f\prim\paren{c}-\dfrac{f\paren{b}-f\paren{a}}{b-a}=0\).

D'où \(f\paren{c}=\dfrac{f\paren{b}-f\paren{a}}{b-a}\).
\end{dem}

\subsection{Inégalité des accroissements finis}

\begin{theo}[Inégalité des accroissements finis]
Soient \(a,b\in\R\) tels que \(a<b\), \(f:\intervii{a}{b}\to\R\) et \(m,M\in\R\).

On suppose que \(\begin{dcases}f\text{ est continue sur }\intervii{a}{b} \\ f\text{ est dérivable sur }\intervee{a}{b} \\ \quantifs{\forall x\in\intervee{a}{b}}m\leq f\prim\paren{x}\leq M\end{dcases}\)

Alors on a \[m\paren{b-a}\leq f\paren{b}-f\paren{a}\leq M\paren{b-a}.\]
\end{theo}

\begin{dem}
Selon l'égalité des accroissements finis, il existe \(c\in\intervee{a}{b}\) tel que \(f\prim\paren{c}=\dfrac{f\paren{b}-f\paren{a}}{b-a}\).

Or \(m\leq f\prim\paren{c}\leq M\) donc \(m\leq\dfrac{f\paren{b}-f\paren{a}}{b-a}\leq M\).

D'où \(m\paren{b-a}\leq f\paren{b}-f\paren{a}\leq M\paren{b-a}\).
\end{dem}

\begin{rem}
En supposant seulement \(\quantifs{\forall x\in\intervee{a}{b}}f\prim\paren{x}\leq M\), on obtient seulement \[f\paren{b}-f\paren{a}\leq M\paren{b-a}\].

De même, en supposant seulement \(\quantifs{\forall x\in\intervee{a}{b}}m\leq f\prim\paren{x}\), on obtient seulement \[m\paren{b-a}\leq f\paren{b}-f\paren{a}\].
\end{rem}

\begin{cor}
Soient \(f:I\to\R\) dérivable et \(k\in\Rp\).

On a \[f\text{ \(k\)-lipschitzienne}\ssi\quantifs{\forall x\in I}\abs{f\prim\paren{x}}\leq k.\]
\end{cor}

\begin{dem}
\impdir

Supposons \(f\) \(k\)-lipschitzienne.

Montrons que \(\quantifs{\forall x\in I}\abs{f\prim\paren{x}}\leq k\).

Soit \(x\in I\).

On a \(\quantifs{\forall y\in I}\abs{f\paren{y}-f\paren{x}}\leq k\abs{y-x}\).

Donc \(\quantifs{\forall y\in I\excluant\accol{x}}\abs{\dfrac{f\paren{y}-f\paren{x}}{y-x}}\leq k\).

D'où, par passage à la limite quand \(y\to x\) : \(\abs{f\prim\paren{x}}\leq k\).

\imprec

Supposons \(\quantifs{\forall x\in I}\abs{f\prim\paren{x}}\leq k\).

Montrons que \(\quantifs{\forall a,b\in I}\abs{f\paren{b}-f\paren{a}}\leq k\abs{b-a}\).

Soient \(a,b\in I\).

Si \(a=b\), on a le résultat.

Si \(a<b\), on a \(\begin{dcases}f\text{ continue sur }\intervii{a}{b} \\ f\text{ dérivable sur }\intervee{a}{b} \\ \quantifs{\forall x\in\intervee{a}{b}}-k\leq f\prim\paren{x}\leq k\end{dcases}\)

Donc selon l'inégalité des accroissements finis, on a \[-k\paren{b-a}\leq f\paren{b}-f\paren{a}\leq k\paren{b-a}.\]

Donc \(\abs{f\paren{b}-f\paren{a}}\leq k\abs{b-a}\).

Si \(a>b\) : idem en échangeant \(a\) et \(b\).
\end{dem}

\begin{cor}
Soit \(f:I\to\R\) dérivable.

Alors \[f\text{ lipschitzienne}\ssi f\prim\text{ bornée}.\]
\end{cor}

\begin{ex}
La fonction \(\sin:\R\to\R\) est \(1\)-lipschitzienne.

En effet, on a \(\begin{dcases}\R\text{ est un intervalle} \\ \sin\text{ est dérivable sur }\R \\ \quantifs{\forall x\in\R}\abs{\sin\prim x}=\abs{\cos x}\leq1\end{dcases}\)

Donc \(\sin\) est \(1\)-lipschitzienne.
\end{ex}

\begin{ex}
Les fonctions \(\exp\), \(\ln\) et \(\fonctionlambda{\Rps}{\R}{x}{\sqrt{x}}\) en sont pas lipschitziennes car leurs dérivées ne sont pas bornées.
\end{ex}

\begin{ex}
On retrouve \(\quantifs{\forall x\in\R}\abs{\sin x}\leq\abs{x}\).
\end{ex}

\begin{ex}
La fonction \(\fonction{f}{\Rm}{\R}{x}{\e{x}}\) est \(1\)-lipschitzienne car \(\R\) est un intervalle et \[\quantifs{\forall x\in\Rm}\abs{f\prim\paren{x}}=\abs{\e{x}}\leq1.\]
\end{ex}

\begin{ex}
La fonction \(f=\restr{\sg}{\Rs}\), \cad \(\fonction{f}{\Rs}{\R}{x}{\begin{dcases}-1 &\text{si }x<0 \\ 1 &\text{si }x>0\end{dcases}}\) est dérivable, de dérivée nulle.

Cependant, on ne peut pas en déduire qu'elle est lipschitzienne car \(\Rs\) n'est pas un intervalle.

Montrons que \(f\) n'est pas lipschitzienne.

Par l'absurde, supposons que \(f\) est lipschitzienne.

Soit \(k\in\Rp\) tel que \(\quantifs{\forall x,y\in\Rs}\abs{\sg x-\sg y}\leq k\abs{x-y}\).

On a \(\quantifs{\forall n\in\Ns}\abs{\sg\dfrac{1}{n}-\sg\dfrac{-1}{n}}\leq k\abs{\dfrac{1}{n}-\dfrac{-1}{n}}\).

Donc \(\quantifs{\forall n\in\Ns}2\leq\dfrac{2k}{n}\).

D'où \(\quantifs{\forall n\in\Ns}n\leq k\).

Donc \(k\) majore \(\Ns\) : contradiction.

Donc \(f\) n'est pas lipschitzienne.
\end{ex}

\subsection{Fonctions croissantes}

\begin{theo}
Soit \(f:I\to\R\) dérivable.

On a :

\begin{enumerate}
\item \(f\text{ constante}\ssi\quantifs{\forall x\in I}f\prim\paren{x}=0\) \\

\item \(f\text{ croissante}\ssi\quantifs{\forall x\in I}f\prim\paren{x}\geq0\) \\

\item \(f\text{ décroissante}\ssi\quantifs{\forall x\in I}f\prim\paren{x}\leq0\) \\

\item \(f\text{ strictement croissante}\ssi\begin{dcases}\quantifs{\forall x\in I}f\prim\paren{x}\geq0 \\ \quantifs{\forall a,b\in I}a<b\imp\croch{\quantifs{\exists x\in\intervee{a}{b}}f\prim\paren{x}>0}\end{dcases}\) \\

\item \(f\text{ strictement croissante}\impr\begin{dcases}\quantifs{\forall x\in I}f\prim\paren{x}\geq0 \\ f\prim\text{ ne s'annule qu'en un nombre fini de points}\end{dcases}\) \\

\item \(f\text{ strictement décroissante}\ssi\begin{dcases}\quantifs{\forall x\in I}f\prim\paren{x}\leq0 \\ \quantifs{\forall a,b\in I}a<b\imp\croch{\quantifs{\exists x\in\intervee{a}{b}}f\prim\paren{x}<0}\end{dcases}\) \\

\item \(f\text{ strictement décroissante}\impr\begin{dcases}\quantifs{\forall x\in I}f\prim\paren{x}\leq0 \\ f\prim\text{ ne s'annule qu'en un nombre fini de points}\end{dcases}\)
\end{enumerate}
\end{theo}

\begin{dem}[2]
\impdir

Supposons \(f\) croissante.

Soit \(x\in I\).

Montrons que \(f\prim\paren{x}\geq0\).

On remarque \(\quantifs{\forall y\in I\excluant\accol{x}}\dfrac{f\paren{y}-f\paren{x}}{y-x}\geq0\).

D'où, par passage à la limite quand \(y\to x\) : \(f\prim\paren{x}\geq0\).

\imprec

Supposons \(\quantifs{\forall x\in I}f\prim\paren{x}\geq0\).

Montrons que \(f\) est croissante, \cad \(\quantifs{\forall a,b\in I}a\leq b\imp f\paren{a}\leq f\paren{b}\).

Soient \(a,b\in I\) tels que \(a\leq b\).

Si \(a=b\) : on a le résultat.

Si \(a<b\) :

On a \(\begin{dcases}f\text{ continue sur }\intervii{a}{b} \\ f\text{ dérivable sur }\intervee{a}{b} \\ \quantifs{\forall x\in\intervee{a}{b}}0\leq f\prim\paren{x}\end{dcases}\)

D'où, selon l'inégalité des accroissements finis : \(0\paren{b-a}\leq f\paren{b}-f\paren{a}\).

Donc \(f\paren{a}\leq f\paren{b}\).
\end{dem}

\begin{dem}[3]
Idem.
\end{dem}

\begin{dem}[1]
Découle de (2) et (3).
\end{dem}

\begin{dem}[4]
\impdir

Supposons \(f\) strictement croissante.

On a \(f\) croissante donc \(f\prim\geq0\) selon (2).

Montrons que \(\quantifs{\forall a,b\in I}a<b\imp\croch{\quantifs{\exists c\in\intervee{a}{b}}f\prim\paren{c}>0}\).

Par l'absurde, soient \(a,b\in I\) tels que \(\begin{dcases}a<b \\ \quantifs{\forall c\in\intervee{a}{b}}f\prim\paren{c}\leq0\end{dcases}\)

On a \(\begin{dcases}a<b \\ f\text{ continue sur }\intervii{a}{b} \\ f\text{ dérivable sur }\intervee{a}{b} \\ f\prim=0\end{dcases}\)

Donc \(f\paren{a}=f\paren{b}\) : contradiction.

\imprec

Supposons \(\begin{dcases}f\prim\geq0 \\ \quantifs{\forall a,b\in I}a<b\imp\croch{\quantifs{\exists c\in\intervee{a}{b}}f\prim\paren{c}>0}\end{dcases}\)

Montrons que \(f\) est strictement croissante.

On a \(f\) croissante car \(f\prim\geq0\) selon (2).

Soient \(a,b\in I\) tels que \(a<b\).

Montrons que \(f\paren{a}<f\paren{b}\).

Si \(f\paren{a}>f\paren{b}\) : impossible car \(f\) est croissante.

Si \(f\paren{a}=f\paren{b}\) alors on a \(\quantifs{\forall c\in\intervii{a}{b}}f\paren{a}\leq f\paren{c}\leq f\paren{b}=f\paren{a}\).

Donc \(f\) est constante sur \(\intervii{a}{b}\).

Donc \(\quantifs{\forall c\in\intervii{a}{b}}f\prim\paren{c}=0\) : impossible.

Donc \(f\paren{a}<f\paren{b}\).

Donc \(f\) est strictement croissante.
\end{dem}

\begin{dem}[5]
Découle de (4).
\end{dem}

\begin{dem}[6 et 7]
Idem que (4) et (5).
\end{dem}

\subsection{Théorème de la limite de la dérivée}

\begin{theo}[Théorème de la limite de la dérivée]
Soient \(f:I\to\R\) et \(a\in I\).

On suppose \(\begin{dcases}f\text{ continue sur }I \\ f\text{ dérivable sur }I\excluant\accol{a} \\ l=\lim_{\substack{x\to a \\ x\not=a}}f\prim\paren{x}\text{ existe}\end{dcases}\)

Si \(l\) est finie alors \(f\) est dérivable en \(a\), de dérivée \(f\prim\paren{a}=l\).

Si \(l\) est infinie alors \(f\) n'est pas dérivable en \(a\). Son graphe admet une tangente verticale.
\end{theo}

\begin{rem}
On a \[f\text{ dérivable en }a\ssi\lim_{\substack{x\to a \\ x\not=a}}\dfrac{f\paren{x}-f\paren{a}}{x-a}\text{ existe et est finie}.\]
\end{rem}

\begin{dem}
Supposons \(l\) finie.

Montrons que \(\lim_{\substack{x\to a \\ x\not=a}}\dfrac{f\paren{x}-f\paren{a}}{x-a}=l\), \cad \[\quantifs{\forall\epsilon\in\Rps;\exists\delta\in\Rps;\forall x\in I\excluant\accol{a}}\abs{x-a}\leq\delta\imp\abs{\dfrac{f\paren{x}-f\paren{a}}{x-a}-l}\leq\epsilon.\]

Soit \(\epsilon\in\Rps\).

Soit \(\delta\in\Rps\) tel que \(\quantifs{\forall c\in I\excluant\accol{a}}\abs{c-a}\leq\delta\imp\abs{f\prim\paren{c}-l}\leq\epsilon\).

Un tel \(\delta\) existe car on a \(\lim_{\substack{x\to a \\ x\not=a}}f\prim\paren{x}=l\).

Montrons que \(\delta\) convient, \cad \[\quantifs{\forall x\in I\excluant\accol{a}}\abs{x-a}\leq\delta\imp\abs{\dfrac{f\paren{x}-f\paren{a}}{x-a}-l}\leq\epsilon.\]

Soit \(x\in I\excluant\accol{a}\) tel que \(\abs{x-a}\leq\delta\).

Si \(a<x\) :

On a \(\begin{dcases}f\text{ continue sur }\intervii{a}{x}\text{ (car \(I\) est un intervalle)} \\ f\text{ dérivable sur }\intervee{a}{x}\end{dcases}\)

Donc selon l'égalité des accroissements finis, il existe \(c_x\in\intervee{a}{x}\) tel que \(\dfrac{f\paren{x}-f\paren{a}}{x-a}=f\prim\paren{c_x}\).

On a \(\abs{c_x-a}\leq\delta\) car \(a<c_x<x<a+\delta\).

Donc \(\abs{f\prim\paren{c_x}-l}\leq\epsilon\) (par définition de \(\delta\)).

Donc \(\abs{\dfrac{f\paren{x}-f\paren{a}}{x-a}-l}\leq\epsilon\).

Si \(x<a\) : idem en appliquant l'égalité des accroissements finis sur \(\intervii{x}{a}\).

Donc \(\delta\) convient.

Donc \(\lim_{\substack{x\to a \\ x\not=a}}\dfrac{f\paren{x}-f\paren{a}}{x-a}=l=\lim_{\substack{x\to a \\ x\not=a}}f\prim\paren{x}\).

Si \(l=\pm\infty\), on montre de la même façon que \(\lim_{\substack{x\to a \\ x\not=a}}\dfrac{f\paren{x}-f\paren{a}}{x-a}=l\).
\end{dem}

\begin{exo}
Soit \(\alpha\in\Rps\).

On pose \[\fonction{f}{\Rp}{\R}{x}{x^\alpha=\begin{dcases}\e{\alpha\ln x} &\text{si }x>0 \\ 0 &\text{sinon}\end{dcases}}\]

\begin{enumerate}
\item Montrer que \(f\) est continue en \(0\). \\

\item Donner une CNS sur \(\alpha\) pour que \(f\) soit dérivable en \(0\).
\end{enumerate}
\end{exo}

\begin{corr}[1]
On a \(\lim_{x\to0^+}f\paren{x}=\lim_{x\to0^+}\e{\alpha\ln x}=0\).

Donc \(f\) est continue en \(0\).
\end{corr}

\begin{corr}[2]
On a \(f\) continue sur \(\Rp\) et \(f\) dérivable sur \(\Rps\).

On a \(\quantifs{\forall x\in\Rps}f\prim\paren{x}=\dfrac{\alpha}{x}\e{\alpha\ln x}=\dfrac{\alpha}{x}x^\alpha=\alpha x^{\alpha-1}\).

Donc \(\lim_{x\to0^+}f\prim\paren{x}=\begin{dcases}0 &\text{si }\alpha-1>0 \\ \pinf &\text{si }\alpha-1<0 \\ 1 &\text{sinon}\end{dcases}\)

Donc d'après le théorème de la limite de la dérivée : \[\begin{aligned}
f\text{ dérivable en }0&\ssi\lim_{x\to0^+}f\prim\paren{x}\text{ finie} \\
&\ssi\alpha-1\geq0 \\
&\ssi\alpha\geq1.
\end{aligned}\]
\end{corr}

\subsection{Fonctions usuelles}

Les fonctions suivantes sont dérivables sur leur ensemble de définition :

\begin{center}
\large
\begin{tabular}{|c|c|c|}
\hline
La fonction : & définie sur : & est dérivable, de dérivée : \\
\hline
\(\exp\) & \(\R\) & \(\exp\) \\[1em]
\(\ln\) & \(\Rps\) & \(x\mapsto\dfrac{1}{x}\) \\[1em]
\(\sh\) & \(\R\) & \(\ch\) \\[1em]
\(\ch\) & \(\R\) & \(\sh\) \\[1em]
\(\sin\) & \(\R\) & \(\cos\) \\[1em]
\(\cos\) & \(\R\) & \(-\sin\) \\[1em]
\(\tan\) & \(\R\excluant\paren{\dfrac{\pi}{2}+\pi\Z}\) & \(1+\tan^2=\dfrac{1}{\cos^2}\) \\[1em]
\(\Arctan\) & \(\R\) & \(x\mapsto\dfrac{1}{1+x^2}\) \\[1em]
\(x\mapsto x^n\) avec \(n\in\Ns\) & \(\R\) & \(x\mapsto nx^{n-1}\) \\[1em]
\(x\mapsto x^n\) avec \(n\in-\Ns\) & \(\Rs\) & \(x\mapsto nx^{n-1}\) \\[1em]
\(x\mapsto x^\alpha\) avec \(\alpha\in\intervee{1}{\pinf}\excluant\Z\) & \(\Rp\) & \(x\mapsto\alpha x^{\alpha-1}\) \\[1em]
\(x\mapsto x^\alpha\) avec \(\alpha\in\intervee{\minf}{0}\excluant\Z\) & \(\Rps\) & \(x\mapsto\alpha x^{\alpha-1}\) \\[1em]
\(x\mapsto a^x\) avec \(a\in\Rps\) & \(\Rps\) & \(x\mapsto \paren{\ln a}a^x\) \\[1em]
\hline
\end{tabular}
\end{center}

Les fonctions suivantes sont continues sur leur ensemble de définition, mais ne sont pas dérivables en tout point :

\begin{center}
\large
\begin{tabular}{|c|c|c|}
\hline
La fonction : & est dérivable sur : & de dérivée : \\
\hline
\(\fonctionlambda{\R}{\R}{x}{\abs{x}}\) & \(\Rs\) & \(x\mapsto\begin{dcases}1 &\text{si }x>0 \\ -1 &\text{si }x<0\end{dcases}\) \\[1em]
\(\Arcsin\) (définie sur \(\intervii{-1}{1}\)) & \(\intervee{-1}{1}\) & \(x\mapsto\dfrac{1}{\sqrt{1-x^2}}\) \\[1em]
\(\Arccos\) (définie sur \(\intervii{-1}{1}\)) & \(\intervee{-1}{1}\) & \(x\mapsto\dfrac{-1}{\sqrt{1-x^2}}\) \\[1em]
\(\fonctionlambda{\Rp}{\R}{x}{x^\alpha}\) avec \(\alpha\in\intervee{0}{1}\) & \(\Rps\) & \(x\mapsto\alpha x^{\alpha-1}\) \\
\hline
\end{tabular}
\end{center}

\subsubsection{\(\exp\)}

La fonction \(\exp\) est dérivable, de dérivée \(\exp\) (admis) et de classe \(\classe{\infty}\).

\subsubsection{\(\ln\)}

La fonction \(\exp\) est continue et strictement croissante sur l'intervalle \(\R\).

Elle est donc bijective de \(\R\) vers \(\intervee{\lim_{\minf}\exp}{\lim_{\pinf}\exp}=\intervee{0}{\pinf}=\Rps\).

On appelle \(\ln:\Rps\to\R\) sa bijection réciproque :

\begin{center}
\begin{tkz}[scale=1.4]
\begin{axis}[axis lines=middle,
xmin=-5,xmax=5,
ymin=-5,ymax=5,
xtick={1,4},
ytick={1,4},
xticklabels={\(1\),\(y\)},
yticklabels={\(1\),\(y\)},
legend entries={\(\exp\),\(\ln\)},
legend pos=north west,
legend style={font=\footnotesize},
clip=false]
\addplot[domain=-5:1.7,samples=1000,smooth,thick,blue] {exp(x)};
\addplot[domain=0:5,samples=1000,smooth,thick,orange] {ln(x)};
\addplot[domain=-5:5,samples=1000,smooth,gray] {x};
\draw[<->,green] (axis cs:1-0.7071,-0.7071) -- (axis cs:1+0.7071,0.7071);
\draw[<->,green] (axis cs:-0.7071,1-0.7071) -- (axis cs:0.7071,1+0.7071);
\draw[dashed,gray] (axis cs:4,0) -- (axis cs:4,1.38629);
\draw[dashed,gray] (axis cs:0,4) -- (axis cs:1.38629,4);
\end{axis}
\end{tkz}
\end{center}

La bijection \(\exp\) est de classe \(\classe{\infty}\) et sa dérivée ne s'annule jamais.

Donc \(\ln\) est de classe \(\classe{\infty}\) et, selon le théorème de dérivation de la bijection réciproque, on a : \[\begin{aligned}
\quantifs{\forall y\in\Rps}\ln\prim y&=\dfrac{1}{\exp\prim\paren{\ln y}} \\
&=\dfrac{1}{\exp\paren{\ln y}} \\
&=\dfrac{1}{y}.
\end{aligned}\]

\subsubsection{\(\sh\) \& \(\ch\)}

On pose : \[\quantifs{\forall x\in\R}\begin{dcases}\sh x=\dfrac{\e{x}-\e{-x}}{2} &\text{(sinus hyperbolique)} \\ \ch x=\dfrac{\e{x}+\e{-x}}{2} &\text{(cosinus hyperbolique)}\end{dcases}\]

On a \(\quantifs{\forall x\in\R}\ch^2x-\sh^2x=1\).

Graphes :

\begin{center}
\begin{tkz}[scale=1.2]
\begin{axis}[axis lines=middle,
xmin=-4,xmax=4,
ymin=0,ymax=5,
xtick={0},
ytick={1},
yticklabels={\(1\)},
legend entries={\(\ch\)},
legend pos=south west,
legend style={font=\footnotesize},
clip=false]
\addplot[domain=-2.2:2.2,samples=1000,smooth,thick,blue] {cosh(x)};
\end{axis}
\end{tkz}
\end{center}

\begin{center}
\begin{tkz}[scale=1.2]
\begin{axis}[axis lines=middle,
xmin=-3,xmax=3,
ymin=-4,ymax=4,
xtick={0},
ytick={0},
legend entries={\(\sh\)},
legend pos=north west,
legend style={font=\footnotesize},
clip=false]
\addplot[domain=-2:2,samples=1000,smooth,thick,blue] {sinh(x)};
\end{axis}
\end{tkz}
\end{center}

La fonction \(\ch\) est paire et on a \(\ch\prim=\sh\).

La fonction \(\sh\) est impaire et on a \(\sh\prim=\ch\).

\subsubsection{\(\sin\) \& \(\cos\)}

Les fonctions \(\sin\) et \(\cos\) sont de classe \(\classe{\infty}\) et on a \(\sin\prim=\cos\) et \(\cos\prim=-\sin\) (admis).

\subsubsection{\(\tan\)}

La fonction \(\tan\) est de classe \(\classe{\infty}\) et on a \(\tan\prim=1+\tan^2=\dfrac{1}{\cos^2}\) (admis).

\subsubsection{\(\Arctan\)}

On sait que la fonction \(\Arctan:\R\to\intervee{\dfrac{-\pi}{2}}{\dfrac{\pi}{2}}\) est la bijection réciproque de \(f=\restr{\tan}{\intervee{\nicefrac{-\pi}{2}}{\nicefrac{\pi}{2}}}\).

Comme \(f\) est de classe \(\classe{\infty}\) et sa dérivée ne s'annule pas, on sait que \(f\inv=\Arctan\) est de classe \(\classe{\infty}\) et qu'on a : \[\begin{aligned}
\quantifs{\forall y\in\R}\Arctan\prim y&=\dfrac{1}{f\prim\paren{\Arctan y}} \\
&=\dfrac{1}{\tan\prim\paren{\Arctan y}} \\
&=\dfrac{1}{1+\tan^2\paren{\Arctan y}} \\
&=\dfrac{1}{1+y^2}.
\end{aligned}\]

\subsubsection{Fonctions puissance}

La fonction \(x\mapsto x^{\alpha}\), où \(\alpha\in\R\excluant\Z\), est définie sur \(\begin{dcases}\Rps &\text{si }\alpha<0 \\ \Rp &\text{si }\alpha\geq0\end{dcases}\)

La fonction \(x\mapsto x^{n}\), où \(n\in\Z\), est définie sur \(\begin{dcases}\Rs &\text{si }n<0 \\ \R &\text{si }n\geq0\end{dcases}\)

Les fonctions sont de classe \(\classe{\infty}\) sur \(\Rps\).

\subsubsection{\(x\mapsto a^x\)}

Soit \(a\in\Rps\).

On remarque que \(a^x=\e{x\ln a}\) donc sa dérivée est \(x\mapsto\paren{\ln a}\e{x\ln a}=\paren{\ln a}a^x\).

\subsubsection{\(\Arcsin\)}

La fonction \[\fonction{f}{\intervii{\dfrac{-\pi}{2}}{\dfrac{\pi}{2}}}{\intervii{-1}{1}}{\theta}{\sin\theta}\] est dérivable et bijective.

Sa bijection réciproque est \(\Arcsin\).

On a \[\begin{aligned}
\quantifs{\forall t\in\intervii{-1}{1}}\Arcsin\text{ est dérivable en }t&\ssi f\prim\paren{\Arcsin t}\not=0 \\
&\ssi\cos\paren{\Arcsin t}\not=0 \\
&\ssi\sqrt{1-t^2}\not=0 \\
&\ssi t\not=\pm1.
\end{aligned}\]

On en déduit : \[\begin{aligned}
\quantifs{\forall t\in\intervee{-1}{1}}\Arcsin\prim t&=\dfrac{1}{f\prim\paren{\Arcsin t}} \\
&=\dfrac{1}{\sqrt{1-t^2}}
\end{aligned}\]

En \(t=\pm1\), la fonction \(\Arcsin\) n'est pas dérivable et son graphe admet une tangente verticale.

\subsubsection{\(\Arccos\)}

On sait qu'on a \(\quantifs{\forall t\in\intervii{-1}{1}}\Arccos t=\dfrac{\pi}{2}-\Arcsin t\).

Donc \(\Arccos\) est dérivable sur \(\intervee{-1}{1}\), de dérivée \(t\mapsto\dfrac{-1}{\sqrt{1-t^2}}\) et son graphe admet une tangente verticale en \(t=\pm1\).

\section{Fonctions convexes}

\subsection{Préliminaires}

\begin{rem}[Paramétrage d'un segment de \(\R\)]
Soient \(a,b\in\R\).

L'ensemble image de la fonction \[\fonction{f}{\intervii{0}{1}}{\R}{t}{\paren{1-t}a+tb}\] est le segment de \(\R\) d'extrémités \(a\) et \(b\), \cad \(\begin{dcases}\intervii{a}{b} &\text{si }a\leq b \\ \intervii{b}{a} &\text{sinon}\end{dcases}\)
\end{rem}

\begin{rem}[Paramétrage d'un segment de droite dans \(\R^2\)]
Soient \(A=\paren{x_A,y_A},B=\paren{x_B,y_B}\in\R^2\).

L'ensemble image de la fonction \[\fonction{F}{\intervii{0}{1}}{\R^2}{t}{\paren{1-t}A+tB=\paren{\paren{1-t}x_A+tx_B,\paren{1-t}y_A+ty_B}}\] est le segment de droite \(\intervii{A}{B}\).
\end{rem}

\begin{defi}[Vocabulaire : graphe, corde, sécante]
Soit \(f:I\to\R\).

On rappelle que le graphe de \(f\) est l'ensemble : \[\graphe{f}=\accol{\paren{x,y}\in I\times\R\tq f\paren{x}=y}\subset\R^2.\]

Soient \(x_1,x_2\in I\) tels que \(x_1\not=x_2\).

Notons \(M_1\) et \(M_2\) les points du graphe de \(f\) d'abscisses respectives \(x_1\) et \(x_2\) : \[M_1=\paren{x_1,f\paren{x_1}}\qquad\text{et}\qquad M_2=\paren{x_2,f\paren{x_2}}.\]

Le segment de droite \(\intervii{M_1}{M_2}\) est appelé la corde de \(\graphe{f}\) en \(x_1\) et \(x_2\).

La droite \(\paren{M_1M_2}\) est appelée la sécante à \(\graphe{f}\) en \(x_1\) et \(x_2\).
\end{defi}

\subsection{Définition}

\begin{defi}[Fonction convexe, fonction concave]
Soit \(f:I\to\R\).

On dit que \(f\) est une fonction convexe si elle vérifie : \[\quantifs{\forall x,y\in I;\forall t\in\intervii{0}{1}}f\paren{\paren{1-t}x+ty}\leq\paren{1-t}f\paren{x}+tf\paren{y}.\]

On dit que \(f\) est une fonction concave si elle vérifie : \[\quantifs{\forall x,y\in I;\forall t\in\intervii{0}{1}}f\paren{\paren{1-t}x+ty}\geq\paren{1-t}f\paren{x}+tf\paren{y}.\]
\end{defi}

\begin{rem}
Pour qu'une fonction soit convexe ou concave, il faut que son ensemble de définition soit un intervalle.
\end{rem}

\begin{ex}
Les fonctions \(\exp\) et \(x\mapsto x^2\) sont convexes.

La fonction \(\ln\) est concave.

La fonction \(\sin\) n'est ni convexe ni concave.
\end{ex}

\begin{rem}
Soit \(f:I\to\R\).

On a : \[f\text{ concave}\ssi-f\text{ convexe}.\]

Dans la suite de ce cours, on se concentrera sur les fonctions convexes.
\end{rem}

\begin{prop}[Inégalité de Jensen]
Soient \(f:I\to\R\), \(n\in\Ns\), \(\lambda_1,\dots,\lambda_n\in\Rp\) et \(x_1,\dots,x_n\in I\).

On suppose que \(f\) est convexe et qu'on a \(\lambda_1+\dots+\lambda_n=1\).

Alors : \[f\paren{\lambda_1x_1+\dots+\lambda_nx_n}\leq\lambda_1f\paren{x_1}+\dots+\lambda_nf\paren{x_n}.\]
\end{prop}

\begin{dem}
Pour tout \(n\in\Ns\), on note \(\P{n}\) la proposition : \[\quantifs{\forall\lambda_1,\dots,\lambda_n\in\Rp;\forall x_1,\dots,x_n\in I}\lambda_1+\dots+\lambda_n=1\imp f\paren{\lambda_1x_1+\dots+\lambda_nx_n}\leq\lambda_1f\paren{x_1}+\dots+\lambda_nf\paren{x_n}.\]

On a \(\P{1}\).

Soit \(n\in\Ns\) tel que \(\P{n}\). Montrons \(\P{n+1}\).

Soient \(\lambda_1,\dots,\lambda_{n+1}\in\Rp\) et \(x_1,\dots,x_{n+1}\in I\) tels que \(\lambda_1+\dots+\lambda_{n+1}=1\).

Si \(\lambda_1+\dots+\lambda_n=0\) et \(\lambda_{n+1}=1\) alors on a \(f\paren{\sum_{i=1}^{n+1}\lambda_ix_i}\leq\sum_{i=1}^{n+1}\lambda_if\paren{x_i}\). Donc on a \(\P{n+1}\).

Supposons \(\lambda_1+\dots+\lambda_n\not=0\).

On a : \[\begin{WithArrows}
f\paren{\sum_{i=1}^{n+1}\lambda_ix_i}&=f\paren{\paren{\lambda_1+\dots+\lambda_n}\sum_{i=1}^n\dfrac{\lambda_i}{\lambda_1+\dots+\lambda_n}x_i+\lambda_{n+1}x_{n+1}} \Arrow{car \(f\) est convexe} \\
&\leq\paren{\lambda_1+\dots+\lambda_n}f\paren{\sum_{i=1}^n\dfrac{\lambda_i}{\lambda_1+\dots+\lambda_n}x_i}+\lambda_{n+1}f\paren{x_{n+1}} \Arrow{selon \(\P{n}\)} \\
&\leq\paren{\lambda_1+\dots+\lambda_n}\sum_{i=1}^n\dfrac{\lambda_i}{\lambda_1+\dots+\lambda_n}f\paren{x_i}+\lambda_{n+1}f\paren{x_{n+1}} \\
&=\sum_{i=1}^{n+1}\lambda_if\paren{x_i}.
\end{WithArrows}\]

D'où \(\P{n+1}\).

Donc par récurrence sur \(n\in\Ns\), on a \(\quantifs{\forall n\in\Ns}\P{n}\).
\end{dem}

\subsection{Propriétés}

Par définition, une fonction convexe est une fonction qui est définie sur un intervalle et dont le graphe est situé en dessous de ses cordes. La proposition suivante précise la position du graphe par rapport à ses sécantes.

\begin{prop}[Position du graphe par rapport à ses sécantes]
Soient \(f:I\to\R\) et \(a,b\in I\) tels que \(a<b\).

On considère la sécante \(\Delta\) au graphe \(\graphe{f}\) en \(a\) et \(b\).

Le graphe de \(f\) est situé :

\begin{itemize}
\item en dessous de sa sécante \(\Delta\) sur \(\intervii{a}{b}\) ; \\

\item au dessus de sa sécante \(\Delta\) sur \(I\inter\paren{\intervei{\minf}{a}\union\intervie{b}{\pinf}}\).
\end{itemize}
\end{prop}

\begin{dem}
Soit \(s\in\intervee{\minf}{0}\union\intervee{1}{\pinf}\).

Supposons \(\paren{1-s}a+sb\in I\).

Il s'agit de montrer \(\paren{1-s}f\paren{a}+sf\paren{b}\leq f\paren{\paren{1-s}a+sb}\).

Posons \(c=\paren{1-s}a+sb=a+s\paren{b-a}\).

Supposons \(s>1\) (\cad \(a<b<c\)).

On a \(\paren{1-\dfrac{1}{s}}a+\dfrac{1}{s}c=b\) avec \(\dfrac{1}{s}\in\intervii{0}{1}\).

D'où, comme \(f\) est convexe : \(f\paren{b}\leq\paren{1-\dfrac{1}{s}}f\paren{a}+\dfrac{1}{s}f\paren{c}\).

D'où, en multipliant par \(s\) : \(\paren{1-s}f\paren{a}+sf\paren{b}\leq f\paren{c}\).

Supposons \(s<0\) (\cad \(c<a<b\)).

On a \(\dfrac{1}{1-s}c-\dfrac{s}{1-s}b=a\), \cad \(\paren{1-\dfrac{1}{1-s}}b+\dfrac{1}{1-s}c=a\).

D'où, comme \(f\) est convexe : \(f\paren{a}\leq\paren{1-\dfrac{1}{1-s}}f\paren{b}+\dfrac{1}{1-s}f\paren{c}\).

D'où, en multipliant par \(s\) : \(f\paren{c}\geq\paren{1-s}f\paren{a}+sf\paren{b}\).
\end{dem}

\begin{prop}[Inégalité des pentes]\thlabel{prop:inégalitéDesPentes}
Soit \(f:I\to\R\).

Pour tous \(a,b\in I\) tels que \(a\not=b\), on note \(\tau\paren{a,b}\) le taux d'accroissement de \(f\) entre \(a\) et \(b\) : \[\tau\paren{a,b}=\dfrac{f\paren{b}-f\paren{a}}{b-a}.\]

NB : cette notation n'est pas officielle.

Les propositions suivantes sont équivalentes :

\begin{enumerate}
\item La fonction \(f\) est convexe \\

\item \(\quantifs{\forall a,b,c\in I}a<b<c\imp\tau\paren{a,b}\leq\tau\paren{a,c}\leq\tau\paren{b,c}\) \\

\item \(\quantifs{\forall a,b,c\in I}a<b<c\imp\tau\paren{a,b}\leq\tau\paren{b,c}\)
\end{enumerate}
\end{prop}

\begin{dem}[(1) \(\imp\) (2)]
Soient \(a,b,c\in I\) tels que \(a<b<c\).

On suppose \(f\) convexe.

Posons \(t=\dfrac{b-a}{c-a}\).

On a \(0\leq t\leq1\) et \(t\paren{c-a}=b-a\) donc \(\paren{1-t}a+tc=b\).

Donc, comme \(f\) est convexe, on a \(f\paren{b}\leq\paren{1-t}f\paren{a}+tf\paren{c}\).

Donc \(f\paren{b}-f\paren{a}\leq t\paren{f\paren{c}-f\paren{a}}\).

Donc \(\dfrac{f\paren{b}-f\paren{a}}{b-a}\leq\dfrac{t\paren{f\paren{c}-f\paren{a}}}{b-a}\).

Donc \(\tau\paren{a,b}\leq\tau\paren{a,c}\).

De même, comme \(t\paren{c-a}=b-a\), on a \(\paren{1-t}\paren{c-a}=\paren{c-a}-\paren{b-a}=c-b\).

Et comme \(t\paren{f\paren{c}-f\paren{a}}\geq f\paren{b}-f\paren{a}\), on a : \[\paren{1-t}\paren{f\paren{c}-f\paren{a}}\leq f\paren{a}-f\paren{b}+f\paren{c}-f\paren{a}=f\paren{c}-f\paren{b}.\]

D'où \(\dfrac{\paren{1-t}\paren{f\paren{c}-f\paren{a}}}{c-b}\leq\dfrac{f\paren{c}-f\paren{b}}{c-b}\).

Donc \(\tau\paren{a,c}\leq\tau\paren{b,c}\).
\end{dem}

\begin{dem}[(2) \(\imp\) (3)]
Clair.
\end{dem}

\begin{dem}[(3) \(\imp\) (1)]
Supposons (3).

Montrons que \(f\) est convexe.

Soient \(a,c\in I\) et \(t\in\intervii{0}{1}\).

Montrons qu'on a \[f\paren{\paren{1-t}a+tc}\leq\paren{1-t}f\paren{a}+tf\paren{c}\quad(*)\]

Quitte à remplacer \(a,c,t\) par \(c,a,1-t\), on peut supposer \(a\leq c\).

Posons \(b=\paren{1-t}a+tc\).

On a \(a\leq b\leq c\).

Si \(a=c\), \(t=0\) ou \(t=1\), alors \((*)\) est claire.

On suppose \(a\not=c\), \(t\not=0\) et \(t\not=1\).

D'où \(a<b<c\).

Il s'agit de montrer \(f\paren{b}\leq\paren{1-t}f\paren{a}+tf\paren{c}\).

On a \(\dfrac{f\paren{b}-f\paren{a}}{b-a}\leq\dfrac{f\paren{c}-f\paren{b}}{c-b}\).

Or \(b-a=t\paren{c-a}\) et \(c-b=\paren{1-t}\paren{c-a}\).

D'où \(\dfrac{f\paren{b}-f\paren{a}}{t\paren{c-a}}\leq\dfrac{f\paren{c}-f\paren{b}}{\paren{1-t}\paren{c-a}}\).

D'où, en multipliant par \(t\paren{1-t}\paren{c-a}>0\) : \(\paren{1-t}\paren{f\paren{b}-f\paren{a}}\leq t\paren{f\paren{c}-f\paren{b}}\).

Donc \(f\paren{b}\leq\paren{1-t}f\paren{a}+tf\paren{c}\).

Donc \(f\) est convexe.
\end{dem}

La proposition suivante est une simple reformulation de la proposition précédente :

\begin{prop}[Croissance des pentes]
Soit \(f:I\to\R\).

On pose \[\quantifs{\forall a\in I}\fonction{g_a}{I\excluant\accol{a}}{\R}{x}{\dfrac{f\paren{x}-f\paren{a}}{x-a}}\]

On a alors : \[f\text{ est convexe}\ssi\quantifs{\forall a\in I}g_a\text{ est croissante}.\]
\end{prop}

\begin{dem}
\impdir

Supposons \(f\) convexe.

Soit \(a\in I\).

Montrons que \(g_a\) est croissante.

Soient \(x,y\in I\) tels que \(x<y\).

Montrons que \(g_a\paren{x}\leq g_a\paren{y}\).

Selon la \thref{prop:inégalitéDesPentes}, on a :

\begin{itemize}
\item Si \(a<x<y\) : on a \(\tau\paren{a,x}\leq\tau\paren{a,y}\). Donc \(g_a\paren{x}\leq g_a\paren{y}\). \\

\item Si \(x<a<y\) : on a \(\tau\paren{x,a}\leq\tau\paren{a,y}\). Donc \(g_a\paren{x}\leq g_a\paren{y}\). \\

\item Si \(x<y<a\) : on a \(\tau\paren{x,a}\leq\tau\paren{y,a}\). Donc \(g_a\paren{x}\leq g_a\paren{y}\).
\end{itemize}

\imprec

Supposons \(\quantifs{\forall a\in I}g_a\text{ est croissante}\).

Montrons que \(f\) est convexe.

Soient \(a,b,c\in I\) tels que \(a<b<c\).

Comme \(g_a\) est croissante, on a \(g_a\paren{a}\leq g_a\paren{b}\).

Donc \(\tau\paren{a,b}\leq\tau\paren{b,c}\).

Donc d'après la \thref{prop:inégalitéDesPentes}, \(f\) est convexe.
\end{dem}

\subsection{Fonctions convexes dérivables}

\begin{prop}
Soit \(f:I\to\R\) dérivable.

On a \[f\text{ est convexe}\ssi f\prim\text{ est croissante}.\]
\end{prop}

\begin{dem}
\impdir

Supposons \(f\) convexe.

Montrons que \(f\prim\) est croissante.

Soient \(x,y\in I\) tels que \(x<y\).

Montrons que \(f\prim\paren{x}\leq f\prim\paren{y}\).

On a, selon la \thref{prop:inégalitéDesPentes} :

\begin{itemize}
\item \(\quantifs{\forall z\in\intervee{x}{y}}\tau\paren{x,z}\leq\tau\paren{x,y}\) \\

\item \(\quantifs{\forall z\in\intervee{x}{y}}\tau\paren{x,y}\leq\tau\paren{z,y}\).
\end{itemize}

D'où, par passage à la limite quand \(z\to x^+\) : \(f\prim\paren{x}\leq\tau\paren{x,y}\).

Et, par passage à la limite quand \(z\to y^-\) : \(\tau\paren{x,y}\leq f\prim\paren{y}\).

Finalement, on a \(f\prim\paren{x}\leq f\prim\paren{y}\).

Donc \(f\prim\) est croissante.

\imprec

Supposons \(f\prim\) croissante.

Montrons que \(f\) est convexe.

Soient \(a,b,c\in I\) tels que \(a<b<c\).

Montrons que \(\tau\paren{a,b}\leq\tau\paren{b,c}\).

La fonction \(f\) est \(\begin{dcases}\text{continue sur }\intervii{a}{b} \\ \text{dérivable sur }\intervee{a}{b}\end{dcases}\) et \(\begin{dcases}\text{continue sur }\intervii{b}{c} \\ \text{dérivable sur }\intervee{b}{c}\end{dcases}\)

Selon l'égalité des accroissements finis, il existe \(x,y\in I\) tels que \(a<x<b<y<c\) et \[\tau\paren{a,b}=f\prim\paren{x}\qquad\text{et}\qquad\tau\paren{b,c}=f\prim\paren{y}.\]

Or \(f\prim\) est croissante donc \(f\prim\paren{x}\leq f\prim\paren{y}\).

Donc \(\tau\paren{a,b}\leq\tau\paren{b,c}\).

Donc \(f\) est convexe.
\end{dem}

\begin{cor}
Soit \(f:I\to\R\).

On a : \[f\text{ est convexe}\ssi f\seconde\geq0.\]
\end{cor}

\begin{ex}
On en déduit facilement la convexité (\ie le caractère convexe ou concave) des fonctions usuelles :

\begin{itemize}
\item \(\exp\) est convexe car \(\exp\seconde=\exp\geq0\).

\item \(\ln\) est concave car \(\quantifs{\forall x\in\Rps}\ln\seconde x=\dfrac{-1}{x^2}\leq0\).

\item \(\cos\) n'est ni convexe ni concave sur \(\R\) (car \(\cos\seconde0=-1<0\) et \(\cos\seconde\pi=1>0\)).

\item \(\Arccos\) est concave sur \(\intervii{0}{1}\) et convexe sur \(\intervii{-1}{0}\) car \(t\mapsto\dfrac{-1}{\sqrt{1-t^2}}\) est décroissante sur \(\intervii{0}{1}\) et croissante sur \(\intervii{-1}{0}\).

\item \(\Arctan\) est concave sur \(\Rp\) et convexe sur \(\Rm\) car \(t\mapsto\dfrac{1}{1+t^2}\) est décroissante sur \(\Rp\) et croissante sur \(\Rm\).

\item \(\tan\) est convexe sur \(\intervii{0}{\dfrac{\pi}{2}}\) et concave sur \(\intervii{\dfrac{-\pi}{2}}{0}\) car \(1+\tan^2\) est croissante sur \(\intervii{0}{\dfrac{\pi}{2}}\) et décroissante sur \(\intervii{\dfrac{-\pi}{2}}{0}\).
\end{itemize}
\end{ex}

\begin{exo}
Étudier la convexité de la fonction \(\fonction{f}{\R}{\R}{x}{\ln\paren{1+x^2}}\).
\end{exo}

\begin{corr}
On a \(\quantifs{\forall x\in\R}f\prim\paren{x}=2x\times\dfrac{1}{1+x^2}=\dfrac{2x}{1+x^2}\).

Donc \(\quantifs{\forall x\in\R}f\seconde\paren{x}=\dfrac{2\paren{1+x^2}-2x\times2x}{\paren{1+x^2}^2}=\dfrac{2\paren{1-x^2}}{\paren{1+x^2}^2}\).

Donc \[\begin{aligned}
f\seconde\paren{x}\geq0&\ssi x^2\leq1 \\
&\ssi\abs{x}\leq1 \\
&\ssi x\in\intervii{-1}{1}.
\end{aligned}\]

Donc \(f\) est convexe sur \(\intervii{-1}{1}\) et concave sur \(\intervei{\minf}{-1}\) et sur \(\intervie{1}{\pinf}\).
\end{corr}

\begin{exo}[Inégalité arithmético-géométrique]
Soient \(n\in\Ns\) et \(x_1,\dots,x_n\in\Rp\).

Montrer : \[\sqrt[n]{x_1\dots x_n}\leq\dfrac{x_1+\dots+x_n}{n}.\]
\end{exo}

\begin{corr}
Si l'un des réels \(x_1,\dots,x_n\) est nul, l'inégalité est claire.

Supposons \(x_1,\dots,x_n\in\Rps\).

On a \(\exp\) convexe sur \(\R\).

On considère les points \(\ln x_1,\dots,\ln x_n\in\R\).

D'après l'inégalité de Jensen, on a : \[\exp\paren{\dfrac{\ln x_1+\dots+\ln x_n}{n}}\leq\dfrac{\exp\paren{\ln x_1}+\dots+\exp\paren{\ln x_n}}{n}.\]

Donc \(\paren{\exp\paren{\ln x_1+\dots+\ln x_n}}^{\nicefrac{1}{n}}\leq\dfrac{x_1+\dots+x_n}{n}\).

D'où \(\paren{x_1\dots x_n}^{\nicefrac{1}{n}}\leq\dfrac{x_1+\dots+x_n}{n}\).
\end{corr}

\begin{prop}
Soit \(f:I\to\R\) convexe et dérivable.

Le graphe de \(f\) est situé au dessus de ses tangentes.
\end{prop}

\begin{dem}
Soit \(a\in I\).

On note \(\graphe{f}\) le graphe de \(f\) et \(T\) sa tangente en \(a\).

On a \[\graphe{f}:y=f\paren{x}\qquad\text{et}\qquad T:y=f\prim\paren{a}\paren{x-a}+f\paren{a}.\]

Posons \(\fonction{\delta}{I}{\R}{x}{f\paren{x}-\paren{f\prim\paren{a}\paren{x-a}+f\paren{a}}}\)

Il s'agit de montrer que \(\quantifs{\forall x\in I}\delta\paren{x}\geq0\).

On a \(\quantifs{\forall x\in I}\delta\prim\paren{x}=f\prim\paren{x}-f\prim\paren{a}\).

Or \(f\) est convexe donc \(f\prim\) est croissante donc \(\quantifs{\forall x\in I}\begin{dcases}\delta\prim\paren{x}\geq0 &\text{si }x\geq a \\ \delta\prim\paren{x}\leq0 &\text{si }x\leq a\end{dcases}\)

On a donc le tableau de variations suivant :

\begin{center}
\begin{tkz}
\tableauinit{\(x\)/1,\(\delta\)/2}{,\(a\),}
\tableauvariations{+/,-/\(0\),+/}
\end{tkz}
\end{center}

D'où \(\quantifs{\forall x\in I}\delta\paren{x}\geq0\).

Donc \(\graphe{f}\) est situé au dessus de \(T\).
\end{dem}

\begin{exo}
Montrer les encadrements suivants :

\begin{enumerate}
\item \(\quantifs{\forall\theta\in\intervii{0}{\dfrac{\pi}{4}}}\theta\leq\tan\theta\leq\dfrac{4}{\pi}\theta\) \\

\item \(\quantifs{\forall\theta\in\intervii{0}{\dfrac{\pi}{2}}}\dfrac{2}{\pi}\theta\leq\sin\theta\leq\theta\)
\end{enumerate}
\end{exo}

\begin{corr}[1]
La fonction \(\tan\) est convexe sur \(\intervii{0}{\dfrac{\pi}{4}}\) car sa dérivée \(\tan\prim=1+\tan^2\) est croissante sur \(\intervii{0}{\dfrac{\pi}{4}}\).

On en déduit d'une part que son graphe est situé au dessus de sa tangente en \(\theta=0\), \cad \(y=\tan\prim\paren{0}\paren{\theta-0}+\tan\paren{0}=\theta\).

Donc \(\quantifs{\forall\theta\in\intervii{0}{\dfrac{\pi}{4}}}\theta\leq\tan\theta\).

On en déduit d'autre part que son graphe est situé en dessous de sa corde entre \(0\) et \(\dfrac{\pi}{4}\), \cad \(\quantifs{\forall\theta\in\intervii{0}{\dfrac{\pi}{4}}}\tan\theta\leq\dfrac{4}{\pi}\theta\).

D'où l'encadrement.
\end{corr}

\begin{corr}[2]
La fonction \(\sin\) est concave sur \(\intervii{0}{\dfrac{\pi}{2}}\) car sa dérivée \(\sin\prim=\cos\) est décroissante sur \(\intervii{0}{\dfrac{\pi}{2}}\).

On en déduit d'une part que son graphe est situé en dessous de sa tangente en \(0\), \cad \(y=\sin\prim\paren{0}\paren{\theta-0}+\sin\paren{0}=\theta\).

Donc \(\quantifs{\forall\theta\in\intervii{0}{\dfrac{\pi}{2}}}\sin\theta\leq\theta\).

On en déduit d'autre part que son graphe est situé au dessus de sa corde entre \(0\) et \(\dfrac{\pi}{2}\), \cad \(\quantifs{\forall\theta\in\intervii{0}{\dfrac{\pi}{2}}}\dfrac{2}{\pi}\theta\leq\sin\theta\).

D'où l'encadrement.
\end{corr}

\chapter{Polynômes, fractions rationnelles}

\minitoc

Dans tout ce chapitre, on considère un corps \(\corps{\K}\).

En pratique, aux concours, on se limite à \(\K=\R\) ou \(\C\), voire parfois \(\Q\).

\section{Polynômes}

\subsection{Anneau des polynômes}

Il n'est pas difficile de donner une construction rigoureuse de l'anneau des polynômes, mais c'est hors programme (et cela n'apporte rien en pratique). On se contente donc de la \guillemets{définition} suivante, qui décrit ce qu'il faut savoir.

\begin{defi}
L'anneau des polynômes en l'indéterminée \(X\) à coefficients dans \(\K\) est un anneau commutatif noté \(\anneau{\poly}\) tel que :

\begin{enumerate}
\item Le corps \(\corps{\K}\) est un sous-anneau de \(\anneau{\poly}\). \\

\item Il existe un élément \(X\in\poly\) appelé l'indéterminée. \\

\item Tout élément \(P\in\poly\) s'écrit de façon unique sous la forme : \[P=\sum_{k=0}^{n}a_kX^k\quad\text{où : }\begin{dcases}n\in\N \\ \quantifs{\forall k\in\interventierii{0}{n}}a_k\in\K\end{dcases}\] à des termes nuls près.

Cette écriture est appelée l'écriture canonique du polynôme.

Le coefficient \(a_k\) est appelé le coefficient de \(P\) de degré \(k\). \\
\end{enumerate}
\end{defi}

\begin{rem}
\begin{itemize}
\item Le point (1) signifie que \(\K\) est inclus dans \(\poly\) et que, pour tout \(\lambda,\mu\in\K\), la somme \(\lambda+\mu\) et le produit \(\lambda\mu\) ont même valeur dans \(\K\) et dans \(\poly\). \\

\item Dans le point (3), l'unicité de l'écriture à des termes nuls près signifie : \[\quantifs{\forall m,n\in\N;\forall a_0,\dots,a_n,b_0,\dots,b_m\in\K}\begin{dcases}n\leq m \\ \sum_{k=0}^na_kX^k=\sum_{k=0}^mb_kX^k\end{dcases}\imp\begin{dcases}\quantifs{\forall k\in\interventierii{0}{n}}a_k=b_k \\ \quantifs{\forall k\in\interventierii{n+1}{m}}b_k=0\end{dcases}\] \\

\item L'indéterminée est parfois aussi notée \(Y\), \(Z\) ou \(T\). L'anneau des polynômes est alors noté \(\poly[\K][Y]\), \(\poly[\K][Z]\) ou \(\poly[\K][T]\). \\

\item Les éléments \(\lambda\in\K\) sont appelés polynômes constants. Leur écriture canonique est \(\lambda X^0\) ou simplement \(\lambda\). On appelle polynôme nul le polynôme \(0=0X^0\). \\

\item On pourrait tout aussi bien considérer l'anneau \(\poly[\A]\) des polynômes à coefficients dans un anneau commutatif \(\A\), mais dans ce chapitre, on ne s'intéresse qu'au cas où les polynômes sont à coefficients dans un corps. \\
\end{itemize}
\end{rem}

\begin{rem}[Structure de l'anneau \(\poly\)]~\\
Soient deux polynômes \(A=\sum_{k=0}^na_kX^k\) et \(B=\sum_{k=0}^mb_kX^k\) avec \(m,n\in\N\) et \(a_0,\dots,a_n,b_0,\dots,b_m\in\K\).

\begin{itemize}
\item Quitte à ajouter des termes nuls, on peut supposer \(m=n\).

La somme de \(A\) et \(B\) est le polynôme \(C=\sum_{k=0}^{n}c_kX^k\) défini par : \[\quantifs{\forall k\in\interventierii{0}{n}}c_k=a_k+b_k.\] \\

\item On pose : \(\quantifs{\forall k\in\interventierie{n+1}{\pinf}}a_k=0\) et \(\quantifs{\forall k\in\interventierie{m+1}{\pinf}}b_k=0\).

Le produit de \(A\) et \(B\) est le polynôme \(D=\sum_{k=0}^{m+n}d_kX^k\) défini par : \[\quantifs{\forall k\in\interventierii{0}{m+n}}d_k=\sum_{j=0}^ka_jb_{k-j}.\] \\

\item Les éléments neutres de l'anneau \(\poly\) sont les polynômes constants \(0=0X^0\) et \(1=1X^0\). \\
\end{itemize}
\end{rem}

\begin{nota}\thlabel{nota:évaluationD'unPolynômeEnUnElément}
Soit \(A\) un anneau. On suppose que \(\K\) est un sous-anneau de \(A\).

Soient \(P\in\poly\) et \(x\in A\).

On considère l'écriture canonique de \(P\) : \[P=\sum_{k=0}^{n}a_kX^k\quad\text{où : }\begin{dcases}n\in\N \\ \quantifs{\forall k\in\interventierii{0}{n}}a_k\in\K\end{dcases}\]

On note \(P\paren{x}\) l'élément de \(A\) défini par : \[P\paren{x}=\sum_{k=0}^na_kx^k.\]

On dit que \(P\paren{x}\) est l'élément obtenu en évaluant \(P\) en \(x\).

Cela ne signifie pas que \(P\) est considéré comme une fonction : il n'y a pas d'ensemble de définition bien défini, on peut \guillemets{évaluer} \(P\) en n'importe quel élément d'un anneau contenant \(\K\) comme sous-anneau.
\end{nota}

\begin{ex}[mêmes notations]
\begin{itemize}
\item Si \(A=\K\) alors la notation \(P\paren{\lambda}\) est valide pour tout \(\lambda\in\K\). \\

\item En particulier, si \(A=\K=\R\) et \(P=X^2+1\) alors la notation \(P\paren{\lambda}\) est valide pour tout \(\lambda\in\R\) mais aussi pour tout \(\lambda\in\C\). On a : \[P\paren{0}=1\qquad P\paren{1}=2\qquad P\paren{\i}=0.\] \\

\item Si \(A=\poly\) alors la notation \(P\paren{Q}\) est valide pour tout \(Q\in\poly\). On dit que \(P\paren{Q}\) est la composition des polynômes \(P\) et \(Q\).

Par exemple, si \(P=X^2+1\) alors \(P\paren{Q}=Q^2+1\in\poly\).

La composition des polynômes \(P\) et \(Q\) est parfois aussi notée \(P\rond Q\). \\

\item Dans les exemples qui précèdent, on a considéré des anneaux commutatifs, mais ce n'est pas nécessaire. En deuxième année, on appliquera très souvent des polynômes \(P\in\poly\) à des matrices carrées à coefficients dans \(\K\), et l'anneau formé par ces matrices n'est pas commutatif. \\
\end{itemize}
\end{ex}

\begin{rem}[Rédaction]
\begin{itemize}
\item Quand on évalue le polynôme \(P\) en \(\i\) ou en \(X^2\), on ne doit surtout pas écrire \guillemets{on pose \(X=\i\)} ou \guillemets{on pose \(X=X^2\)}. \\

\item La notation \(P\paren{X}\) désigne la composition des polynômes \(P\) et \(X\), \cad \(P\). On peut donc écrire au choix \guillemets{le polynôme \(P\paren{X}=X^2+1\)} ou \guillemets{le polynôme \(P=X^2+1\)}, il n'y a aucune différence, alors que si \(f\) est une fonction, il ne faut surtout pas confondre \(f\) et \(f\paren{x}\). \\
\end{itemize}
\end{rem}

\begin{prop}
Soit \(A\) un anneau et \(x\in A\). On suppose que \(\K\) est un sous-anneau de \(A\).

Alors il existe un unique morphisme d'anneaux \(\phi:\poly\to A\) tel que : \[\phi\paren{X}=x\quad\text{et}\quad\quantifs{\forall\lambda\in\K}\phi\paren{\lambda}=\lambda.\]
\end{prop}

\begin{dem}
\analyse

Soit \(\phi:\poly\to A\) tel que \(\phi\paren{X}=x\) et \(\restr{\phi}{\K}=\id{\K}\).

On montre par une récurrence immédiate : \(\quantifs{\forall n\in\N}\phi\paren{X^n}=x^n\).

Soit \(P=\sum_{k=0}^{n}a_kX^k\) avec \(\begin{dcases}n\in\N \\ a_0,\dots,a_n\in\K\end{dcases}\).

On a \[\begin{WithArrows}
\phi\paren{P}&=\sum_{k=0}^{n}\phi\paren{a_kX^k}\Arrow{car \(\phi\) morphisme d'anneaux} \\
&=\sum_{k=0}^{n}\phi\paren{a_k}\phi\paren{X^k} \\
&=\sum_{k=0}^{n}a_kx^k \\
&=P\paren{x}.
\end{WithArrows}\]

Ainsi, on a : \(\quantifs{\forall P\in\poly}\phi\paren{P}=P\paren{x}\).

\synthese

Posons \(\fonction{\phi}{\poly}{A}{P}{P\paren{x}}\).

On a : \begin{itemize}
\item \(\phi\paren{X}=x\) \\

\item \(\quantifs{\forall\lambda\in\K}\phi\paren{\lambda}=\lambda\) \\

\item \(\phi\) est un morphisme d'anneaux : \[\phi\paren{1}=1\quad\text{et}\quad\quantifs{\forall P,Q\in\poly}\begin{dcases}\phi\paren{P+Q}=\paren{P+Q}\paren{x}=P\paren{x}+Q\paren{x}=\phi\paren{P}+\phi\paren{Q} \\ \phi\paren{PQ}=PQ\paren{x}=P\paren{x}Q\paren{x}=\phi\paren{P}\phi\paren{Q}\end{dcases}\] \\
\end{itemize}

\conclusion

Le morphisme existe et est unique.
\end{dem}

\begin{defi}[Conjugaison]
Soit un polynôme à coefficients complexes \(P=\sum_{k=0}^{n}a_kX^k\in\poly[\C]\) (avec \(n\in\N\) et \(a_0,\dots,a_n\in\C\)).

On appelle conjugué de \(P\) le polynôme : \[\conj{P}=\sum_{k=0}^{n}\conj{a_k}X^k.\]
\end{defi}

\begin{rem}
On a : \(\quantifs{\forall P\in\poly[\C]}P=\conj{P}\ssi P\in\poly[\R]\).
\end{rem}

\begin{prop}
La conjugaison \(\fonctionlambda{\poly[\C]}{\poly[\C]}{P}{\conj{P}}\) est un automorphisme d'anneaux.
\end{prop}

\begin{dem}
\note{EXERCICE}
\end{dem}

\subsection{Degré}

\begin{defi}[Degré d'un polynôme]~\\
Soit un polynôme \(P=\sum_{k=0}^{n}a_kX^k\in\poly\) (avec \(n\in\N\) et \(a_0,\dots,a_n\in\K\)).

Le degré de \(P\) est défini par : \[\deg P=\begin{dcases}\minf&\text{si }P=0 \\ \max\accol{k\in\interventierii{0}{n}\tq a_k\not=0}&\text{sinon}\end{dcases}\]
\end{defi}

\begin{prop}
Soient \(A,B\in\poly\). On a :

\begin{enumerate}
\item \(\deg\paren{A+B}\leq\max\accol{\deg A;\deg B}\) avec égalité si \(\deg A\not=\deg B\). \\

\item \(\deg\paren{AB}=\deg A+\deg B\). \\

\item Si \(B\) non-constant alors \(\deg \paren{A\paren{B}}=\paren{\deg A}\times\paren{\deg B}\). \\
\end{enumerate}
\end{prop}

\begin{dem}[notations pour les démonstrations suivantes]~\\\renewcommand{\cqfd}{}
Soient \(m,n\in\N\) et \(a_0,\dots,a_n,b_0,\dots,b_m\in\K\) tels que \(A=\sum_{k=0}^{n}a_kX^k\) et \(B=\sum_{k=0}^{m}b_kX^k\).

On pose \(\begin{dcases}\quantifs{\forall k\in\interventierie{n+1}{\pinf}}a_k=0 \\ \quantifs{\forall k\in\interventierie{m+1}{\pinf}}b_k=0\end{dcases}\)

On pose \(\begin{dcases}C=\sum_{k=0}^{p}c_kX^k=A+B&\text{avec }\begin{dcases}p=\max\accol{m;n} \\ c_0,\dots,c_p\in\K\end{dcases} \\ D=\sum_{k=0}^{m+n}d_kX^k=AB&\text{avec }d_0,\dots,d_{m+n}\in\K\end{dcases}\)
\end{dem}

\begin{dem}[1]~\\
On a : \(\begin{dcases}\quantifs{\forall k>\deg A}a_k=0 \\ \quantifs{\forall k>\deg B}b_k=0\end{dcases}\)

Donc \(\quantifs{\forall k\in\interventierii{\max\accol{\deg A;\deg B}+1}{p}}c_k=a_k+b_k=0\).

Donc \(\deg C\leq\max\accol{\deg A;\deg B}\).

Supposons \(\deg A\not=\deg B\), par exemple \(\deg A<\deg B\).

On a : \(\max\accol{\deg A;\deg B}=\deg B\).

Et : \(c_{\deg B}=\underbrace{a_{\deg B}}_{=0}+\underbrace{b_{\deg B}}_{\not=0}\not=0\).

Donc \(\deg C\leq\deg B\) donc \(\deg C=\deg B\).
\end{dem}

\begin{dem}[2]
Si \(A\) ou \(B\) est nul, alors \(\deg\paren{AB}=\minf=\deg A+\deg B\).

Supposons \(A\) et \(B\) non-nuls.

On a : \(\begin{dcases}a_{\deg A}\not=0\quad\text{et}\quad\quantifs{\forall k>\deg A}a_k=0 \\ b_{\deg B}\not=0\quad\text{et}\quad\quantifs{\forall k>\deg B}b_k=0\end{dcases}\)

Montrons que \(\deg\paren{AB}=\deg A+\deg B\).

Si \(k>\deg A+\deg B\) alors \(d_k=\sum_{j=0}^{k}a_jb_{k-j}=0\) car \(\quantifs{\forall j\in\interventierii{0}{k}}j>\deg A\quad\text{ou}\quad k-j>\deg B\) (car \(j+k-j>\deg A+\deg B\)).

Si \(k=\deg A+\deg B\) alors \(d_k=\sum_{j=0}^{k}a_jb_{k-j}=a_{\deg A}+b_{\deg B}\not=0\) car si \(j>\deg A\) alors \(a_j=0\) et si \(j<\deg A\) alors \(k-j>\deg B\) donc \(b_{k-j}=0\).
\end{dem}

\begin{dem}[3]~\\
On a \(A\paren{B}=\sum_{k=0}^{n}a_kB^k\).

On suppose \(B\) non-constant.

On a \(\paren{\deg B}^{\deg A}=\paren{\deg A}\paren{\deg B}\).

Donc, comme \(a_{\deg A}\not=0\) : \[\deg\paren{a_{\deg A}B^{\deg B}}=\paren{\deg A}\paren{\deg B}\quad\text{et}\quad\quantifs{\forall k\in\interventierii{0}{\deg A-1}}\deg\paren{a_kB^k}<\paren{\deg A}\paren{\deg B}.\]

Donc \(\deg\paren{\sum_{k=0}^{\deg A-1}a_kB^k}<\paren{\deg A}\paren{\deg B}\).

Donc selon (1) : \(\deg\paren{A\rond B}=\paren{\deg A}\paren{\deg B}\).
\end{dem}

\begin{nota}
Soit \(n\in\N\).

On pose : \[\poly[\K_n]=\accol{P\in\poly\tq\deg P\leq n}.\]

Autrement dit : \[\poly[\K_n]=\accol{P\in\poly\tq\quantifs{\exists a_0,\dots,a_n\in\K}P=\sum_{k=0}^{n}a_kX^k}.\]
\end{nota}

\begin{rem}
\begin{itemize}
\item Si \(n=0\) alors \(\poly[\K_0]=\K\) est l'ensemble des polynômes constants. \\

\item Si \(n\in\Ns\) alors \(\poly[\K_n]\) est un sous-groupe de \(\groupe{\poly}\) (mais pas un sous-anneau de \(\anneau{\poly}\) car il n'est pas stable par produit).

Le groupe \(\groupe{\poly[\K_n]}\) est isomorphe au groupe \(\K^{n+1}\). \\
\end{itemize}
\end{rem}

\begin{prop}[Propriétés de l'anneau \(\poly\)]
\begin{enumerate}
\item L'anneau \(\anneau{\poly}\) est intègre. \\

\item Les éléments inversibles de \(\poly\) sont les polynômes constants non-nuls : \[\poly\croix=\K\excluant\accol{0}.\]
\end{enumerate}
\end{prop}

\begin{dem}[1]
On sait que \(\poly\) est non-nul et commutatif.

Soient \(P,Q\in\poly\) tels que \(PQ=0\).

On a \(\deg\paren{PQ}=\minf\).

Donc \(\underbrace{\deg P}_{\in\N\union\accol{\minf}}+\underbrace{\deg Q}_{\in\N\union\accol{\minf}}=\minf\).

Donc \(\deg P=\minf\quad\text{ou}\quad\deg Q=\minf\).

Donc \(P=0\quad\text{ou}\quad Q=0\). Donc \(\poly\) est intègre.
\end{dem}

\begin{dem}[2]
\analyse

Soit \(P\in\poly\croix\).

On a \(PP^{-1}=1\).

Donc \(\deg\paren{PP^{-1}}=\deg1\).

Donc \(\underbrace{\deg P}_{\in\N\union\accol{\minf}}+\underbrace{\deg P^{-1}}_{\in\N\union\accol{\minf}}=0\).

Donc \(\deg P=\deg P^{-1}=0\).

Donc \(P\) constant non-nul.

\synthese

Soit \(P\in\poly\) tel que \(\deg P=0\).

On a \(P=\lambda\in\K\excluant\accol{0}\).

Donc \(P\) inversible : \(\lambda\times\dfrac{1}{\lambda}=1\).

\conclusion

Les polynômes inversibles sont les polynômes constants non-nuls.
\end{dem}

\begin{defi}[Coefficient dominant]
Le coefficient dominant d'un polynôme non-nul \(P\) est le coefficient de \(P\) de degré \(\deg P\).

Un polynôme unitaire est un polynôme non-nul dont le coefficient dominant vaut \(1\).
\end{defi}

\begin{ex}
\begin{itemize}
\item Le coefficient dominant du polynôme \(P=7X^5-3X^2+5\) vaut \(7\). \\

\item Le coefficient dominant du polynôme \(Q=X^3+3X^2+3X+1\) vaut \(1\) donc \(Q\) est unitaire. \\
\end{itemize}
\end{ex}

\subsection{Division euclidienne}

\begin{defprop}[Division euclidienne dans \(\poly\)]
Soient \(A,B\in\poly\). On suppose \(B\not=0\).

Il existe un unique couple \(\paren{Q,R}\in\poly^2\) tel que : \[\begin{dcases}A=QB+R \\ \deg R<\deg B\end{dcases}\]

Le polynôme \(Q\) est appelé le quotient de la division euclidienne de \(A\) par \(B\).

Le polynôme \(R\) est appelé le reste de la division euclidienne de \(A\) par \(B\).
\end{defprop}

\begin{dem}
\unicite

Soient \(Q_1,R_1,Q_2,R_2\in\poly\) tels que \(\begin{dcases}A=Q_1B+R_1=Q_2B+R_2 \\ \deg R_1<\deg B \\ \deg R_2<\deg B\end{dcases}\)

On a \(\begin{dcases}\paren{Q_1-Q_2}B=R_2-R_1 \\ \deg\paren{R_2-R_1}<\deg B\end{dcases}\) donc \(\deg\paren{\paren{Q_1-Q_2}B}<\deg B\) donc \(\deg\paren{Q_1-Q_2}+\deg B<\deg B\).

Donc \(\deg\paren{Q_1-Q_2}=\minf\) donc \(Q_1=Q_2\) donc \(R_1-R_2=0\times B=0\).

Donc \(\paren{Q_1,R_1}=\paren{Q_2,R_2}\).

\existence

On fixe le polynôme \(B\in\poly\excluant\accol{0}\).

Si \(B\) est constant, la proposition est vraie car \(\quantifs{\forall A\in\poly}A=QB\) avec \(Q=\dfrac{A}{B}\).

Supposons \(B\) non-constant. Montrons que \(\underbrace{\quantifs{\forall n\in\N;\forall A\in\poly[\K_n];\exists Q,R\in\poly\excluant\accol{0}}\begin{dcases}A=QB+R \\ \deg R<\deg B\end{dcases}}_{\P{n}}\)

Par récurrence sur \(n\in\N\) :

Soit \(A\in\poly[\K_0]\). On a \(\begin{dcases}A=0\times B+A \\ \deg A=0<\deg B\end{dcases}\) donc \(Q=0\) et \(R=A\) conviennent. D'où \(\P{0}\).

Soit \(n\in\N\) tel que \(\P{n}\). Montrons \(\P{n+1}\).

Soit \(A\in\poly[\K_{n+1}]\).

Il existe \(a_0,\dots,a_{n+1}\in\K\) tels que \(A=a_{n+1}X^{n+1}+\dots+a_0X^0\).

Si \(n+1<\deg B\) alors le couple \(\paren{Q,R}=\paren{0,A}\) convient.

Supposons \(n+1\geq\deg B\).

Posons \(m=\deg B\) et \(B=b_mX^m+\dots+b_0X^0\) (avec \(b_0,\dots,b_m\in\K\) et \(b_m\not=0\)).

Posons \(A_1=A-\dfrac{a_{n+1}}{b_m}X^{n+1-m}B\).

On a \(\begin{dcases}\deg A\leq n+1 \\ \deg\paren{\dfrac{a_{n+1}}{b_m}X^{n+1-m}B}\leq n+1\end{dcases}\)

Donc \(\deg A_1\leq n+1\).

De plus, le coefficient de degré \(n+1\) de \(A_1\) est : \(a_{n+1}-\dfrac{a_{n+1}}{b_m}b_m=0\).

Donc \(\deg A_1<n+1\).

Donc il existe \(Q_1,R_1\in\poly\) tels que \(\begin{dcases}A_1=Q_1B+R_1 \\ \deg R_1<\deg B\end{dcases}\)

Finalement, on a \(\begin{dcases}A=\paren{Q_1+\dfrac{a_{n+1}}{b_m}X^{n+1-m}}B+R_1 \\ \deg R_1<\deg B\end{dcases}\)

D'où \(\P{n+1}\).

D'où \(\quantifs{\forall n\in\N}\P{n}\).

D'où l'existence car \(\poly=\bigunion_{n\in\N}\poly[\K_n]\).
\end{dem}

\begin{ex}
Calculons la division euclidienne de \(2X^3+3X^2+1\) par \(X^2+1\) : \[\polylongdiv[style=D]{2X^3+3X^2+1}{X^2+1}\]

Donc \(2X^3+3X^2+1=\underbrace{\paren{2X+3}}_{\text{quotient}}\paren{X^2+1}\underbrace{-2X-2}_{\text{reste}}\) et \(\deg\paren{-2X-2}<\deg\paren{X^2+1}\).

De même, on a : \[\polylongdiv[style=D]{3X^4+X+1}{X+2}\]

Donc \(3X^4+X+1=\underbrace{\paren{3X^3-6X^2+12X-23}}_{\text{quotient}}\paren{X+2}+\underbrace{47}_{\text{reste}}\) et \(\deg47<\deg\paren{X+2}\).
\end{ex}

\subsection{Divisibilité}

\begin{defi}[Divisibilité dans \(\poly\)]
Soient \(A,B\in\poly\).

Si on a : \[\quantifs{\exists C\in\poly}AC=B\] alors on dit que \begin{itemize}
\item \(A\) divise \(B\) ;
\item \(A\) est un diviseur de \(B\) ;
\item \(B\) est un multiple de \(A\) ;
\item \(B\) est divisible par \(A\).
\end{itemize}
\end{defi}

\begin{nota}
Soient \(A,B\in\poly\).

La notation \(A\divise B\) signifie \guillemets{\(A\) divise \(B\)}.

L'ensemble \(\accol{AC}_{C\in\poly}\) des multiples de \(A\) est noté \(A\poly\) : \[A\poly=\accol{AC}_{C\in\poly}=\accol{P\in\poly\tq A\divise P}.\]

Dans ce cours, on notera \(\ensdiv B\) (notation non-officielle) l'ensemble des polynômes nuls ou unitaires qui divisent \(B\) : \[\ensdiv B=\accol{P\in\poly\tq P\divise B\quad\text{et}\quad\paren{P=0\quad\text{ou}\quad P\text{ unitaire}}}.\]
\end{nota}

\begin{rem}[Lien avec la division euclidienne]
Soient \(A,B\in\poly\). On suppose \(B\not=0\).

On note \(R\) le reste de la division euclidienne de \(A\) par \(B\).

Alors : \[B\divise A\ssi R=0.\]
\end{rem}

\begin{prop}[Divisibilité dans \(\poly\)]\thlabel{prop:divisibilitéDansPoly}
\begin{enumerate}
\item La relation binaire \(\divise\) sur \(\poly\) est réflexive et transitive (mais pas antisymétrique, donc ce n'est pas une relation d'ordre sur \(\poly\)). \\

\item On a : \[\quantifs{\forall A,B,C\in\poly}A\divise B\imp CA\divise CB\] et : \[\quantifs{\forall A,B\in\poly;\forall C\in\poly\excluant\accol{0}}A\divise B\ssi CA\divise CB.\] \\

\item On a : \[\quantifs{\forall A,B\in\poly}A\divise B\imp\paren{\deg A\leq\deg B\quad\text{ou}\quad B=0}.\]
\end{enumerate}
\end{prop}

\begin{defprop}
Soient \(A,B\in\poly\).

On a : \[\paren{A\divise B\quad\text{et}\quad B\divise A}\ssi\quantifs{\exists\lambda\in\K\excluant\accol{0}}A=\lambda B.\]

Lorsque ces propositions sont vérifiées, on dit que \(A\) et \(B\) sont associés.
\end{defprop}

\begin{dem}
\impdir

Supposons \(A\divise B\quad\text{et}\quad B\divise A\).

Soient \(C,D\in\poly\) tels que \(AC=B\) et \(BD=A\). On a \(A=BD=ACD\).

Si \(A\not=0\) alors \(CD=1\) donc \(D\) inversible (d'inverse \(C\)). Donc \(D=\lambda\) où \(\lambda\in\K\excluant\accol{0}\) et \(A=\lambda B\).

Si \(A=0\) alors \(B=0\) car \(A\divise B\) et \(\lambda=1\) convient.

\imprec

Claire : on a \(B=\dfrac{1}{\lambda}A\).
\end{dem}

\begin{prop}[Divisibilité entre polynômes unitaires]\thlabel{prop:divisibilitéEntrePolynômesUnitaires}
\renewcommand{\U}{\mathscr{U}}
Notons \(\U\) l'ensemble des polynômes de \(\poly\) nuls ou unitaires : \[\begin{aligned}
\U&=\ensdiv0 \\
&=\accol{P\in\poly\tq\quantifs{\exists n\in\N;\exists R\in\poly[\K_n]}P=X^{n+1}+R}\union\accol{0;1}.
\end{aligned}\]

\begin{enumerate}
\item La relation binaire \(\divise\) sur \(\U\) est une relation d'ordre sur \(\U\). \\

\item Pour cette relation d'ordre, le polynôme nul est le plus grand élément de \(\U\) et le polynôme constant \(1\) est le plus petit : \[\quantifs{\forall P\in\U}P\divise0\quad\text{et}\quad1\divise P.\]
\end{enumerate}
\end{prop}

\subsection{Racines}

\begin{defi}[Racine]
Soit \(P\in\poly\).

On appelle racine de \(P\) (dans \(\K\)) tout élément \(\lambda\in\K\) tel que \(P\paren{\lambda}=0\).
\end{defi}

\begin{ex}
Le polynôme \(X^2+1\) admet \(\i\) et \(-\i\) comme racines dans \(\C\). Il n'admet aucune racine dans \(\R\).

Tout polynôme de degré \(1\) admet exactement une racine.
\end{ex}

\begin{theo}[Théorème de d'Alembert-Gauss]
Tout polynôme non-constant à coefficients complexes admet une racine : \[\quantifs{\forall P\in\poly[\C]}\deg P\geq1\imp\quantifs{\exists\lambda\in\C}P\paren{\lambda}=0.\]
\end{theo}

\begin{dem}
\note{ADMIS} (hors programme).
\end{dem}

\begin{prop}[Racines complexes des polynômes réels]\thlabel{prop:lambdaBarreEstRacine}
Soit \(P\in\poly[\R]\). Soit \(\lambda\in\C\) une racine complexe de \(P\).

Alors \(\conj{\lambda}\) est racine de \(P\).
\end{prop}

\begin{dem}
On a \(P\paren{\lambda}=0\) donc \(\conj{P\paren{\lambda}}=0\) donc \(\conj{P}\paren{\conj{\lambda}}=0\).

Or \(P=\conj{P}\) car \(P\in\poly[\R]\).

Donc \(P\paren{\conj{\lambda}}=0\).

Donc \(\conj{\lambda}\) est racine de \(P\).
\end{dem}

\begin{prop}\thlabel{prop:XMoinsLambdaDiviseP}
Soit \(P\in\poly\). Soit \(\lambda\in\K\).

On a : \[\lambda\text{ racine de }P\ssi\paren{X-\lambda}\divise P.\]
\end{prop}

\begin{dem}
Notons \(Q\) et \(R\) le quotient et le reste de la division euclidienne de \(P\) par \(X-\lambda\) : \[P=\paren{X-\lambda}Q+R\quad\text{et}\quad\deg R<\deg\paren{X-\lambda}.\]

On a donc \(R=\mu\) avec \(\mu\in\K\).

On a : \[\begin{aligned}
\lambda\text{ racine de }P&\ssi P\paren{\lambda}=0 \\
&\ssi\paren{\lambda-\lambda}Q\paren{\lambda}+\mu=0 \\
&\ssi\mu=0 \\
&\ssi R=0 \\
&\ssi\paren{X-\lambda}\divise P.
\end{aligned}\]
\end{dem}

\subsection{Fonctions polynomiales}

\begin{defi}[Fonction polynomiale associée à un polynôme]
Soit un polynôme \(P\in\poly\).

Soient \(n\in\N\) et \(a_0,\dots,a_n\in\K\) tels que \(P=a_nX^n+\dots+a_0X^0\).

On appelle fonction polynomiale associée à \(P\) la fonction : \[\fonction{\tilde{P}}{\K}{\K}{x}{P\paren{x}=a_nx^n+\dots+a_0x^0}\]
\end{defi}

\begin{prop}[Anneau des fonctions polynomiales]\thlabel{prop:anneauDesFonctionsPolynomiales}
L'application \[\fonction{\phi}{\poly}{\F{\K}{\K}}{P}{\tilde{P}}\] est un morphisme d'anneaux.

En particulier, l'ensemble \(\Im\phi\) des fonctions polynomiales de \(\K\) dans \(\K\) est un sous-anneau de \(\F{\K}{\K}\).
\end{prop}

\begin{dem}
\note{EXERCICE}
\end{dem}

\begin{rem}
On a aussi : \[\quantifs{\forall P,Q\in\poly}\widetilde{P\rond Q}=\tilde{P}\rond\tilde{Q}.\]
\end{rem}

\begin{rem}
On verra \hyperref[subsec:nbRacinesPoly]{plus loin} que si \(\K\) est infini, alors \(\phi\) est un isomorphisme d'anneaux. Ainsi, lorsque \(\K\) est infini, deux polynômes à coefficients dans \(\K\) sont égaux si, et seulement si, leur fonction polynomiale associée sont égales.

Cela est faux lorsque le corps \(\K\) est fini puisque dans ce cas, il y a une infinité de polynômes à coefficients dans \(\K\) mais seulement un nombre fini de fonctions polynomiales. Le \hyperref[theo:petitThéorèmeDeFermat]{petit théorème de Fermat} donne un exemple de polynôme non-nul dont la fonction polynomiale associée est nulle : \(X^p-X\).
\end{rem}

\subsection{Dérivation}

\begin{defi}[Polynôme dérivé d'un polynôme]
Soit \(P\in\poly\) d'écriture canonique : \[P=\sum_{k=0}^{n}a_kX^k\quad\text{où : }\begin{dcases}n\in\N \\ \quantifs{\forall k\in\interventierii{0}{n}}a_k\in\K\end{dcases}\]

On appelle polynôme dérivé de \(P\) le polynôme : \[P\prim=\sum_{k=1}^{n}ka_kX^{k-1}.\]
\end{defi}

\begin{rem}[Lien entre dérivation des polynômes et dérivation des fonctions]
Soit \(P\in\poly[\R]\).

La fonction polynomiale associée à \(P\prim\) est la dérivée de la fonction polynomiale associée à \(P\) : \[\widetilde{P\prim}=\paren{\tilde{P}}\prim.\]

En particulier, toute fonction polynomiale est de classe \(\classe{\infty}\).
\end{rem}

\begin{prop}[Opérations algébriques sur les dérivés]
Soient \(P,Q\in\poly\) et \(\lambda,\mu\in\K\).

Somme : on a \[\paren{P+Q}\prim=P\prim+Q\prim.\]

Combinaison linéaire : on a, plus généralement \[\paren{\lambda P+\mu Q}\prim=\lambda P\prim+\mu Q\prim.\]

Produit : on a \[\paren{PQ}\prim=P\prim Q+PQ\prim.\]

Composition : on a \[\paren{P\rond Q}\prim=Q\prim\times\paren{P\prim\rond Q}.\]
\end{prop}

\begin{dem}
\note{EXERCICE}
\end{dem}

\begin{prop}[Formule de Leibniz pour les polynômes]
Soit \(n\in\N\). Soient \(P,Q\in\poly\).

On a : \[\paren{PQ}\deriv{n}=\sum_{k=0}^{n}\binom{k}{n}P\deriv{k}Q\deriv{n-k}.\]
\end{prop}

\begin{dem}
\note{EXERCICE} (par récurrence, exactement comme on a démontré la formule de Leibniz pour les fonctions).
\end{dem}

\begin{rem}
Soient \(n\in\N\) et \(P_1,\dots,P_n\in\poly\).

On a : \[\paren{P_1\dots P_n}\prim=\sum_{k=0}^{n}P_1\dots P_{k-1}P_k\prim P_{k+1}\dots P_n.\]
\end{rem}

\begin{prop}[Degré du polynôme dérivé]
On suppose ici que \(\K\) est un sous-corps de \(\C\).

Soit \(P\in\poly\).

On a : \[\deg P\prim=\begin{dcases}\deg P-1&\text{si }P\text{ non-constant, \cad si }\deg P\geq1 \\ \minf&\text{si }P\text{ constant, \cad si }\deg P\leq0\end{dcases}\]

En particulier, on a : \(P\text{ constant}\ssi P\prim=0\).
\end{prop}

\begin{prop}[Formule de Taylor pour les polynômes]\label{prop:TaylorPoly}
On suppose ici que \(\K\) est un sous-corps de \(\C\).

Soit \(P\in\poly\excluant\accol{0}\) un polynôme de degré \(n\in\N\). Soit \(\lambda\in\K\).

On a : \[P=\sum_{k=0}^{n}\dfrac{P\deriv{k}\paren{\lambda}}{k!}\paren{X-\lambda}^k.\]
\end{prop}

\begin{dem}
Montrons que \(\quantifs{\forall n\in\N}\underbrace{\quantifs{\forall P\in\poly[\K_n]}P=\sum_{k=0}^{n}\dfrac{P\deriv{k}\paren{\lambda}}{k!}\paren{X-\lambda}^k}_{\P{n}}\) par récurrence sur \(n\in\N\).

Soit \(P=\mu\in\poly[\K_0]\). On a \(\begin{dcases}P\deriv{0}\paren{\lambda}=\mu \\ \quantifs{\forall k\in\Ns}P\deriv{k}\paren{\lambda}=0\end{dcases}\) donc la somme vaut \(\dfrac{\mu}{0!}\paren{X-\lambda}^0=\mu\). D'où \(\P{0}\).

Soit \(n\in\N\) tel que \(\P{n}\). Soit \(P\in\poly[\K_{n+1}]\). On a \(P\prim\in\poly[\K_n]\) donc selon \(\P{n}\) : \[\begin{aligned}
P\prim&=\sum_{k=0}^{n}\dfrac{\paren{P\prim}\deriv{k}}{k!}\paren{X-\lambda}^k \\
&=\sum_{k=0}^{n}\dfrac{P\deriv{k+1}}{k!}\paren{X-\lambda}^k
\end{aligned}\]

Donc \(P\prim-\sum_{k=0}^{n}\dfrac{P\deriv{k+1}}{k!}\paren{X-\lambda}^k=0\).

Donc \(\paren{P-\sum_{k=0}^{n}\dfrac{P\deriv{k+1}}{\paren{k+1}!}\paren{X-\lambda}^{k+1}}\prim=0\).

Donc le polynôme \(Q=P-\sum_{l=1}^{n+1}\dfrac{P\deriv{l}}{l!}\paren{X-\lambda}^l\) est constant.

Or \(Q\paren{\lambda}=P\paren{\lambda}-0\) donc \(P-\sum_{l=1}^{n+1}\dfrac{P\deriv{l}}{l!}\paren{X-\lambda}^l=P\paren{\lambda}=\dfrac{P\deriv{0}}{0!}\paren{X-\lambda}^0\).

D'où \(\P{n+1}\).

D'où \(\quantifs{\forall n\in\N}\P{n}\).
\end{dem}

\section{Arithmétique des polynômes}

\subsection{Idéal d'un anneau commutatif}

\begin{defi}[Idéal d'un anneau commutatif]
Soit \(\anneau{A}\) un anneau commutatif.

On appelle idéal de \(A\) toute partie \(I\subset A\) telle que :

\begin{enumerate}
\item \(I\) est un sous-groupe de \(\groupe{A}\). \\

\item \(\quantifs{\forall a\in A;\forall x\in I}ax\in I\). \\
\end{enumerate}
\end{defi}

\begin{prop}
Soit \(\anneau{A}\) un anneau commutatif. Soit \(I\subset A\).

Alors la partie \(I\) est un idéal de \(A\) si, et seulement si, elle vérifie :

\begin{enumerate}\setcounter{enumi}{2}
\item \(0_A\in I\). \\

\item \(\quantifs{\forall x,y\in I}x+y\in I\). \\

\item \(\quantifs{\forall a\in A;\forall x\in I}ax\in I\). \\
\end{enumerate}
\end{prop}

\begin{dem}
\impdir

Supposons (1) et (2).

Alors on a (3) et (4) car \(I\) est un sous-groupe de \(A\) et on a (5) car on a (2).

\imprec

Supposons (3), (4) et (5).

Alors (1) est vraie selon (3) et (4) et (2) est vraie selon (5).
\end{dem}

\begin{ex}
Soit \(\anneau{A}\) un anneau commutatif dont on note \(0_A\) l'élément neutre pour la loi \(+\).

Le singleton \(\accol{0}\) et l'ensemble en \(A\) tout-entier sont des idéaux de l'anneau commutatif \(A\).

On appelle ces idéaux les idéaux triviaux de \(A\).
\end{ex}

\begin{ex}\thlabel{ex:ensembleDesMultiplesD'unElementEstUnIdeal}
Soient \(\anneau{A}\) et \(a\in A\).

L'ensemble des multiples de \(a\) : \[aA=\accol{b\in A\tq\quantifs{\exists c\in A}b=ca}\] est un idéal de \(A\).
\end{ex}

\begin{dem}
On a \(0\in aA\) car \(0=0\times a\).

Si \(x,y\in A\), il existe \(b,c\in A\) tels que \(x=ba\) et \(y=ca\).

Donc \(x+y=\paren{b+c}a\in aA\).

Si \(x\in aA\) et \(b\in A\), il existe \(c\in A\) tel que \(x=ac\) et on a \(bx=\paren{bc}a\in aA\).
\end{dem}

\begin{ex}\thlabel{ex:idéauxDeZ}
Les idéaux de l'anneau \(\anneau{\Z}\) sont les sous-groupes du groupe \(\groupe{\Z}\).
\end{ex}

\begin{dem}
Tout idéal de \(\Z\) est un sous-groupe de \(\Z\).

Réciproquement, si \(H\) est un sous-groupe de \(\Z\) :

Soit \(h\in H\).

On a \(-h\in H\) et \(\quantifs{\forall n\in\N}\begin{dcases}nh\in H \\ -nh\in H\end{dcases}\) donc \(\quantifs{\forall n\in\Z}nh\in H\).

Donc \(H\) est un idéal de \(\Z\).
\end{dem}

\begin{ex}
Soient \(\anneau{A}\) et \(\anneau{B}\) deux anneaux et \(\phi:A\to B\) un morphisme d'anneaux.

On suppose que l'anneau \(A\) est commutatif.

On note \(0_B\) l'élément neutre de \(B\) pour la loi \(+\).

On appelle noyau du morphisme d'anneaux \(\phi\) le noyau de \(\phi\) vu comme un morphisme de groupes de \(\groupe{A}\) vers \(\groupe{B}\), \cad : \[\ker\phi=\phi\inv\paren{\accol{0_B}}=\accol{x\in A\tq\phi\paren{x}=0_B}.\]

Alors \(\ker\phi\) est un idéal de \(A\).
\end{ex}

\begin{dem}
On sait que \(\ker\phi\) est un sous-groupe de \(\groupe{A}\).

De plus, si \(a\in A\) et \(x\in\ker\phi\), alors \(\phi\paren{ax}=\phi\paren{a}\phi\paren{x}=\phi\paren{a}\times0=0\).

Donc \(ax\in\ker\phi\).

Donc \(\ker\phi\) est un idéal de \(A\).
\end{dem}

\begin{prop}[Intersection d'idéaux]
Soient \(\anneau{A}\) un anneau commutatif et \(\paren{I_j}_{j\in J}\) une famille d'idéaux de \(A\).

Alors l'intersection \(\biginter_{j\in J}I_j\) est un idéal de \(A\).
\end{prop}

\begin{dem}
Pour tout \(j\in J\), \(I_j\) est un sous-groupe de \(\groupe{A}\) donc \(\biginter_{j\in J}I_j\) est un sous-groupe de \(\groupe{A}\).

Soient \(a\in A\) et \(x\in\biginter_{j\in J}I_j\).

On a \(\quantifs{\forall j\in J}ax\in I_j\) car \(x\in I_j\) et \(I_j\) est un idéal de \(A\).

Donc \(ax\in\biginter_{j\in J}I_j\).

Donc \(\biginter_{j\in J}\) est un idéal de \(A\).
\end{dem}

\begin{defprop}[Somme d'idéaux]
Soient \(\anneau{A}\) un anneau commutatif et \(I_1,I_2\) deux idéaux de \(A\).

On appelle somme de \(I_1\) et \(I_2\) et on note \(I_1+I_2\) l'ensemble : \[\begin{aligned}
I_1+I_2&=\accol{x\in A\tq\quantifs{\exists x_1\in I_1;\exists x_2\in I_2}x=x_1+x_2} \\
&=\accol{x_1+x_2}_{\paren{x_1,x_2}\in I_1\times I_2}
\end{aligned}\]

On a :

\begin{enumerate}
\item L'ensemble \(I_1+I_2\) est un idéal de \(A\). \\

\item La loi \(+\) est une loi de composition interne sur l'ensemble des idéaux de \(A\). \\

\item Cette loi \(+\) est associative et commutative.
\end{enumerate}
\end{defprop}

\begin{dem}[1]
On a \(I_1+I_2\subset A\) et \(\underbrace{0}_{\in A}=\underbrace{0}_{\in I_1}+\underbrace{0}_{\in I_2}\in I_1+I_2\).

Enfin, si \(x\in I_1+I_2\) et \(a\in A\), il existe \(x_1\in I_1\) et \(x_2\in I_2\) tels que \(x=x_1+x_2\).

D'où \(ax=\underbrace{ax_1}_{\in I_1}+\underbrace{ax_2}_{\in I_2}\in I_1+I_2\).

Donc \(I_1+I_2\) est un idéal de \(A\).
\end{dem}

\begin{dem}[2 et 3]
Clair car la loi \(+\) de \(A\) est associative et commutative.
\end{dem}

\begin{theo}[Idéaux de \(\Z\)]\thlabel{theo:idéauxDeZ}
Les idéaux de l'anneau commutatif \(\anneau{\Z}\) sont les ensembles de la forme \(n\Z\) où \(n\in\N\).
\end{theo}

\begin{dem}
On a vu à l'\thref{ex:idéauxDeZ} que les idéaux de \(\Z\) sont ses sous-groupes donc le théorème n'est qu'un reformulation de la description des sous-groupes de \(\Z\) (\cf \thref{theo:nZSontLesSousGroupesDeZ}).
\end{dem}

\begin{theo}[Idéaux de \(\poly\)]\thlabel{theo:idéauxDePoly}
Les idéaux de l'anneau commutatif \(\anneau{\poly}\) sont les ensembles de la forme \(\poly P\) où \(P\in\poly\), \cad les idéaux formés des multiples d'un polynôme donné.
\end{theo}

\begin{dem}
\increc

On a déjà vu que pour tout \(P\in\poly\), \(\poly P\) est un idéal de \(\poly\) (\thref{ex:ensembleDesMultiplesD'unElementEstUnIdeal}).

\incdir

Soit \(I\) un idéal de \(\poly\).

Si \(I=\accol{0}\) alors \(I=\poly P\) en prenant \(P=0\).

On suppose \(I\not=\accol{0}\).

On a \(I\excluant\accol{0}\not=\ensvide\) donc \(\accol{\deg P}_{P\in I\excluant\accol{0}}\) est une partie non-vide de \(\N\).

Soit \(P\in I\excluant\accol{0}\) tel que \(\deg P\) soit minimal (\cad tel que \(\quantifs{\forall Q\in I\excluant\accol{0}}\deg P\leq\deg Q\)).

Montrons que \(I=\poly P\).

\increc On a \(\quantifs{\forall Q\in\poly}QP\in I\) donc \(\poly P\subset I\).

\incdir

Soit \(A\in I\).

On note \(Q\) et \(R\) le quotient et le reste de la division euclidienne de \(A\) par \(P\).

On a \(\begin{dcases}A=QP+R \\ \deg R<\deg P\end{dcases}\)

On a \(R\in I\) car \(R=A-QP\) et \(A,P\in I\).

Donc \(R=0\) car \(\deg R<\deg P\).

Donc \(A=QP\in\poly P\).
\end{dem}

\begin{rem}[Hors-programme]
Soit \(\anneau{A}\) un anneau commutatif.

On appelle idéal principal de \(A\) tout idéal de la forme \(aA\) où \(a\in A\), \cad tout idéal formé des multiples d'un élément de \(A\) (\cf \thref{ex:ensembleDesMultiplesD'unElementEstUnIdeal}).

On dit que \(A\) est un anneau principal si \(A\) est un anneau intègre et si tout idéal de \(A\) est principal.

On peut résumer le \thref{theo:idéauxDeZ} et le \thref{theo:idéauxDePoly} en disant que les anneaux \(\Z\) et \(\poly\) sont des anneaux principaux.
\end{rem}

\subsection{PGCD}

L'étude du PGCD de deux polynômes dans \(\poly\) se fait exactement de la même façon que celle du PGCD de deux entiers dans \(\Z\) (et, plus généralement, on pourrait traiter de la même façon le PGCD de deux éléments dans un anneau principal).

Afin d'alléger l'exposé, on se contente de la preuve algébrique de l'existence du PGCD et de la relation de Bézout, mais on aurait aussi pu en donner, comme pour les entiers, une preuve algorithmique basée sur l'algorithme d'Euclide.

\subsubsection{PGCD de deux polynômes}

\begin{deftheo}
Soient \(A,B\in\poly\).

On rappelle qu'on note \(\ensdiv{A}\) l'ensemble des diviseurs de \(A\) nuls ou unitaires.

Les diviseurs nuls ou unitaires communs à \(A\) et \(B\) sont les éléments de l'ensemble \(\ensdiv{A}\inter\ensdiv{B}\).

On munit cet ensemble de la relation d'ordre \(\divise\) (la divisibilité est bien une relation d'ordre sur cet ensemble selon la \thref{prop:divisibilitéEntrePolynômesUnitaires}).

L'ensemble \(\ensdiv{A}\inter\ensdiv{B}\) possède un plus grand élément appelé le plus grand diviseur commun à \(A\) et \(B\) et noté \(A\et B\).
\end{deftheo}

\begin{dem}
L'ensemble \(\poly A+\poly B\) est un idéal de \(\poly\) donc il existe \(D_1\in\poly\) tel que \(\poly A+\poly B=\poly D_1\).

Posons \(D=\begin{dcases}D_1 &\text{si }D_1=0 \\ \dfrac{1}{\lambda}D_1 &\text{sinon, en notant \(\lambda\) le coefficient dominant de \(D_1\)}\end{dcases}\)

Ainsi, on a \(\begin{dcases}D\text{ nul ou unitaire} \\ \poly A+\poly B=\poly D\end{dcases}\)

On a \(A=1A+0B\in\poly A+\poly B\) donc \(A\in\poly D\) donc \(D\divise A\).

On montre de même \(D\divise B\).

Soit \(C\in\poly\) divisant \(A\) et \(B\).

Soient \(C_1,C_2\in\poly\) tels que \(CC_1=A\) et \(CC_2=B\).

On a \(D\in\poly D\) donc \(D\in\poly A+\poly B\).

Soient \(U,V\in\poly\) tels que \(UA+VB=D\).

On a \(D=UCC_1+VCC_2=C\paren{UC_1+VC_2}\).

Donc \(C\divise D\).

Donc \(D\) est le plus grand élément de \(\ensdiv{A}\inter\ensdiv{B}\) pour \(\divise\).
\end{dem}

\begin{ex}
Posons \(A=X^2-1\) et \(B=X^2-X\).

Alors \(A\et B=X-1\).
\end{ex}

\begin{dem}
On a \(A=\paren{X-1}\paren{X+1}\) et \(B=\paren{X-1}X\).

Donc \(X-1\) divise \(A\) et \(B\).

De plus, \(X-1\) est unitaire.

Quels sont les diviseurs unitaires de \(A\) ?

\analyse

Soient \(A_1\in\poly\) un diviseur de \(A\) et \(A_2\in\poly\) tel que \(A_1A_2=A\).

On a \(\deg A_1+\deg A_2=\deg A\) et \(\begin{dcases}A_1\not=0 \\ A_2\not=0\end{dcases}\)

Donc \(\deg A_1\in\accol{0;1;2}\).

Si \(\deg A_1=0\) alors \(A_1\) est constant et non-nul.

Si \(\deg A_1=1\) alors \(A_1\) admet une unique racine qui doit être racine de \(A\), \ie \(1\) ou \(-1\).

Si \(\deg A_1=2\) alors \(A_2\) est constant et non-nul, donc \(A_1\) est associé à \(A\).

Si, de plus, \(A_1\) est unitaire, alors \(A_1=1\), \(A_1=X-1\), \(A_1=X+1\) ou \(A_1=X^2-1\).

\synthese

Ces quatre polynômes divisent \(A\) et sont unitaires.

\conclusion

On a \(\ensdiv{A}=\accol{1;X-1;X+1;X^2-1}\).

De même, on a \(\ensdiv{B}=\accol{1;X-1;X;X^2-X}\).

Donc \(\ensdiv{A}\inter\ensdiv{B}=\accol{1;X-1}\).

Donc \(A\et B=X-1\).
\end{dem}

\begin{rem}
Soit \(A\in\poly\).

On a \(A\et1=1\).

Si \(A=0\) alors \(A\et0=0\) ; sinon \(A\et0\) est l'unique polynôme unitaire associé à \(A\).
\end{rem}

\begin{rem}
Soient \(A,B\in\poly\).

\begin{enumerate}
\item Le polynôme \(A\et B\) est le diviseur commun à \(A\) et \(B\) nul ou unitaire et qui est divisible par tous les autres diviseurs communs à \(A\) et \(B\). \\

\item Il est donc caractérisé par : \[A\et B\text{ nul ou unitaire}\qquad\text{et}\qquad\ensdiv{A}\inter\ensdiv{B}=\ensdiv{A\et B}\] ou par : \[A\et B\text{ nul ou unitaire}\qquad\text{et}\qquad\quantifs{\forall P\in\poly}P\divise A\et B\ssi\croch{P\divise A\text{ et }P\divise B}.\]

\item Si \(A=B=0\) alors \(0\et0=0\).

Sinon, \(A\et B\) est le polynôme unitaire du plus haut degré qui divise \(A\) et \(B\) (selon la \thref{prop:divisibilitéDansPoly}).
\end{enumerate}
\end{rem}

\begin{dem}[2]
Soit \(P\in\poly\).

Montrons qu'on a \(P\divise A\et B\ssi\croch{P\divise A\text{ et }P\divise B}\).

\impdir Si \(P\divise A\et B\) alors \(P\divise A\) car \(A\et B\divise A\) et \(P\divise B\) car \(A\et B\divise B\).

\imprec

Supposons \(P\divise A\) et \(P\divise B\).

On pose \(P_1\) le polynôme nul ou unitaire associé à \(P\) :

\(P_1=\begin{dcases}P_1=0 &\text{si }P=0 \\ P_1=\dfrac{1}{\lambda}P &\text{en notant }\lambda\text{ le coefficient dominant de }P\end{dcases}\)

On a \(P_1\divise P\), \(P\divise A\) et \(P\divise B\).

Donc \(P_1\divise A\) et \(P_1\divise B\).

Donc \(P_1\divise A\et B\).

Donc \(P\divise A\et B\).
\end{dem}

\begin{rem}
Soient \(A,B\in\poly\).

Le polynôme \(A\et B\) est nul si, et seulement si, on a \(A=B=0\).

Si \(A\) ou \(B\) est non-nul, on impose systématiquement au polynôme \(A\et B\) d'être unitaire.

En revanche, on s'autorise parfois à appeler \guillemets{plus grand commun diviseur de \(A\) et \(B\)} tout polynôme \(P\) associé à \(A\) et \(B\), même s'il n'est pas unitaire.

Ainsi, on dit que \(2X-2\) est un PGCD de \(X^2-1\) et \(X^2-X\).
\end{rem}

\subsubsection{Propriétés}

\begin{prop}\thlabel{prop:PGCDFoisPolynomesAssocies}
Soient \(A,B,P\in\poly\).

Les polynômes \(PA\et PB\) et \(P\times\paren{A\et B}\) sont associés.
\end{prop}

\begin{dem}
Montrons que \(P\times\paren{A\et B}\) divise \(PA\et PB\).

On a \(A\et B\) divise \(A\) et \(B\).

Donc \(P\times\paren{A\et B}\) divise \(PA\) et \(PB\).

Donc \(P\times\paren{A\et B}\) divise \(PA\et PB\).

Montrons que \(PA\et PB\) divise \(P\times\paren{A\et B}\).

On a \(P\) divise \(PA\) et \(PB\).

Donc \(P\) divise \(PA\et PB\).

Soit \(Q\in\poly\) tel que \(PA\et PB=PQ\).

On remarque que \(PQ\) divise \(PA\) et \(PB\).

Donc, en supposant que \(P\not=0\), on a \(Q\) divise \(A\) et \(B\).

Donc \(Q\) divise \(A\et B\).

Donc \(PA\et PB=PQ\) divise \(P\times\paren{A\et B}\).

Ceci est également vrai si \(P=0\).

Finalement, comme les deux polynômes se divisent mutuellement, ils sont associés.
\end{dem}

\begin{defprop}[Relation de Bézout]\thlabel{defprop:relationDeBezoutPolys}
Soient \(A,B\in\poly\).

Il existe \(U,V\in\poly\) tels que \[UA+VB=A\et B.\]

Une telle écriture s'appelle une relation de Bézout.

Elle n'est pas unique.
\end{defprop}

\begin{dem}
On a vu que \(\poly A+\poly B=\poly\paren{A\et B}\).

On a \(A\et B\in\poly\paren{A\et B}\) donc \(A\et B\in\poly A+\poly B\).

Donc il existe \(U,V\in\poly\) tels que \(A\et B=UA+VB\).

Ces polynômes ne sont pas unique car si \(\paren{U,V}\) convient, \(\paren{U+B,V+A}\) convient aussi.
\end{dem}

\subsubsection{Algorithme d'Euclide}

Les deux lemmes suivants servent à justifier l'algorithme d'Euclide.

\begin{lem}
Soient \(A,B\in\poly\) tels que \(B\not=0\).

On note \(R\) le reste de la division euclidienne de \(A\) par \(B\).

Alors \[A\et B=R\et B.\]
\end{lem}

\begin{dem}
C'est clair car les polynômes qui divisent \(A\) et \(B\) sont ceux qui divisent \(R\) et \(B\).
\end{dem}

\begin{lem}
Soient \(A,B\in\poly\).

On définit le polynôme \(A_1\) en posant : \begin{itemize}
\item si \(A=0\) alors \(A_1=0\) ;

\item sinon \(A_1\) est l'unique polynôme unitaire associé à \(A\). \\
\end{itemize}

On a : \[A\et B=A_1\ssi A\divise B.\]
\end{lem}

\begin{dem}
\impdir

Comme \(A\) et \(A_1\) sont associés, ils se divisent mutuellement.

On a \(A\divise A_1\) et \(A_1=A\et B\divise B\).

D'où \(A\divise B\).

\imprec

Supposons \(A\divise B\).

Alors les diviseurs communs à \(A\) et \(B\) sont les diviseurs de \(A\), \cad les diviseurs de \(A_1\).

Donc \(A\et B=A_1\).
\end{dem}

\begin{algo}[Algorithme d'Euclide]
\Cf \thref{algo:EuclideEntiers}.
\end{algo}

\begin{algo}[Algorithme d'Euclide étendu]
\Cf \thref{algo:EuclideEtenduEntiers} (même principe).
\end{algo}

\subsubsection{PGCD de plusieurs polynômes}

\begin{prop}
La loi \(\et\) est une loi de composition interne sur \(\poly\).

Elle est associative et commutative.
\end{prop}

\begin{dem}
\note{Exercice} (\cf \thref{dem:pgcdLCIsurZ}).
\end{dem}

\begin{defprop}
Soient \(r\in\Ns\) et \(A_1,\dots,A_r\in\poly\).

Les diviseurs communs à \(A_1,\dots,A_r\) sont les diviseurs du polynôme \(A_1\et\dots\et A_r\).

Ce polynôme est appelé le plus grand commun diviseur des polynômes \(A_1,\dots,A_r\).
\end{defprop}

\begin{dem}
\note{Exercice}
\end{dem}

\begin{prop}
Soient \(r\in\Ns\) et \(A_1,\dots,A_r,P\in\poly\).

Les polynômes \(PA_1\et\dots\et PA_r\) et \(P\times\paren{A_1\et\dots\et A_r}\) sont associés.
\end{prop}

\begin{dem}
Découle de la \thref{prop:PGCDFoisPolynomesAssocies} par récurrence sur \(r\in\Ns\).
\end{dem}

\begin{defprop}[Relation de Bézout pour \(r\in\Ns\) polynômes]
Soient \(r\in\Ns\) et \(A_1,\dots,A_r\in\poly\).

Alors il existe \(U_1,\dots,U_r\in\poly\) tels que \[U_1A_1+\dots+U_rA_r=A_1\et\dots\et A_r.\]

Une telle écriture est appelée une relation de Bézout. Elle n'est pas unique.
\end{defprop}

\begin{dem}
On raisonne par récurrence sur \(r\in\interventierie{2}{\pinf}\).

Pour tout \(r\geq2\), on note \(\P{r}\) la proposition \[\quantifs{\forall A_1,\dots,A_r\in\poly;\exists U_1,\dots,U_r\in\poly}U_1A_1+\dots+U_rA_r=A_1\et\dots\et A_r.\]

D'après la \thref{defprop:relationDeBezoutPolys}, on a \(\P{2}\).

Soit \(r\in\interventierie{2}{\pinf}\) tel que \(\P{r}\).

Soient \(A_1,\dots,A_{r+1}\in\poly\).

Selon \(\P{r}\), il existe \(U_1,\dots,U_r\in\poly\) tels que \(U_1A_1+\dots+U_rA_r=A_1\et\dots\et A_r\).

Selon \(\P{2}\), il existe \(U,V\in\poly\) tels que \(U\times\paren{A_1\et\dots\et A_r}+VA_{r+1}=\paren{A_1\et\dots\et A_r}\et A_{r+1}\).

Finalement : \[UU_1A_1+\dots+UU_rA_r+VA_{r+1}=A_1\et\dots\et A_{r+1}.\]

D'où \(\P{r+1}\).

Donc on a \(\quantifs{\forall r\in\interventierie{2}{\pinf}}\P{r}\).
\end{dem}

\subsection{Polynômes premiers entre eux}

Les démonstrations sont laissées en exercice (\cf \ref{sec:entiersPremiersEntreEux}).

\subsubsection{Cas de deux polynômes}

\begin{defi}[Polynômes premiers entre eux]
Deux polynômes \(A,B\in\poly\) sont dits premiers entre eux s'ils vérifient \[A\et B=1.\]

Cela signifie que leurs seuls diviseurs communs sont les polynômes inversibles.
\end{defi}

\begin{theo}[Théorème de Bézout]
Soient \(A,B\in\poly\).

On a \[A\text{ et }B\text{ sont premiers entre eux}\ssi\quantifs{\exists U,V\in\poly}UA+VB=1.\]
\end{theo}

\begin{lem}[Lemme de Gauss]
Soient \(A,B,P\in\poly\).

On suppose \[P\divise AB\qquad\text{et}\qquad P\et B=1.\]

Alors \[P\divise A.\]
\end{lem}

\begin{prop}
Soient \(A,B,P\in\poly\).

On suppose \[A\et P=1\qquad\text{et}\qquad B\et P=1.\]

Alors \[P\et AB=1.\]
\end{prop}

\begin{cor}\thlabel{cor:polysTousPremiersImpliqueProduitPremier}
Soient \(r\in\Ns\) et \(A_1,\dots,A_r\in\poly\).

On suppose \[\quantifs{\forall k\in\interventierii{1}{r}}A_k\et P=1.\]

Alors \[P\et A_1\dots A_r=1.\]
\end{cor}

\begin{prop}\thlabel{prop:polysPremiersEntreEuxQuiDivisentDoncProduitDivise}
Soient \(A,B,P\in\poly\).

On suppose \[A\divise P\qquad\text{et}\qquad B\divise P\qquad\text{et}\qquad A\et B=1.\]

Alors \[AB\divise P.\]
\end{prop}

\subsubsection{Cas de plusieurs polynômes}

\begin{defi}
Soient \(r\in\Ns\) et \(A_1,\dots,A_r\in\poly\).

On dit que les polynômes \(A_1,\dots,A_r\) sont premiers entre eux deux à deux si on a \[\quantifs{\forall i,j\in\interventierii{1}{r}}i\not=j\imp A_i\et A_j=1.\]

On dit que les polynômes \(A_1,\dots,A_r\) sont premiers entre eux dans leur ensemble si \[A_1\et\dots\et A_r=1.\]
\end{defi}

\begin{rem}
Soient \(r\in\interventierie{2}{\pinf}\) et \(A_1,\dots,A_r\in\poly\).

La proposition \guillemets{\(A_1,\dots,A_r\) sont premiers entre eux deux à deux} implique la proposition \guillemets{\(A_1,\dots,A_r\) sont premiers entre eux dans leur ensemble}.

L'implication réciproque est fausse.
\end{rem}

\subsection{PPCM}

\subsubsection{PPCM de deux polynômes}

\begin{defprop}
\renewcommand{\U}{\mathscr{U}}
Soient \(A,B\in\poly\).

Notons \(\U\) l'ensemble des polynômes de \(\poly\) nuls ou unitaires.

On a vu que la relation \(\divise\) est une relation d'ordre sur cet ensemble (\cf \thref{prop:divisibilitéEntrePolynômesUnitaires}).

Dans cet ensemble ordonné \(\groupe{\U}[\divise]\), il existe un plus petit multiple commun à \(A\) et \(B\), \cad un polynôme \(A\ou B\in\poly\) tel que \[A\ou B\in\U\qquad\text{et}\qquad\begin{dcases}A\divise A\ou B \\ B\divise A\ou B\end{dcases}\qquad\text{et}\qquad\quantifs{\forall M\in\poly}\begin{dcases}A\divise M \\ B\divise M\end{dcases}\imp A\ou B\divise M.\]

Ce polynôme \(A\ou B\) est appelé le plus petit commun multiple de \(A\) et \(B\).
\end{defprop}

\begin{dem}
\renewcommand{\U}{\mathscr{U}}
Posons \(I=\poly A\inter\poly B\).

\(I\) est un idéal de \(\poly\) car c'est l'intersection de deux idéaux.

Soit \(P\in\poly\) tel que \(I=\poly P\).

Quitte à multiplier \(P\) par un élément non-nul de \(\K\), on peut supposer \(P\in\U\).

On a \(P\in\poly A\) donc \(A\divise P\) et \(P\in\poly B\) donc \(B\divise P\).

Soit \(M\in\poly\) tel que \(\begin{dcases}A\divise M \\ B\divise M\end{dcases}\)

On a \(M\in\poly A\) et \(M\in\poly B\) donc \(M\in I\).

Donc \(P\divise M\).
\end{dem}

\begin{ex}
On pose \(A=X^2-1\) et \(B=X^2-X\).

Alors \(A\ou B=X^3-X\).
\end{ex}

\begin{dem}
On a \(X^3-X\) unitaire.

\(X^3-X\) est un multiple commun à \(A\) et \(B\) car \(X^3-X=X\paren{X^2-1}=\paren{X+1}\paren{X^2-X}\).

Soit \(M\in\poly\) tel que \(A\divise M\) et \(B\divise M\).

On a \(\begin{dcases}\paren{X+1}\paren{X-1}\divise M \\ X\paren{X-1}\divise M\end{dcases}\) donc \(\begin{dcases}X+1\divise M \\ X\paren{X-1}\divise M\end{dcases}\)

De plus, on a \(\paren{X+1}\et\paren{X^2-X}=1\) car \(\paren{X-2}\paren{X+1}-\paren{X^2-X}=2\).

Donc \(\paren{X+1}X\paren{X-1}\divise M\).

Donc \(X\paren{X-1}\paren{X+1}=X^3-X=A\ou B\).
\end{dem}

\begin{rem}
Soit \(A\in\poly\).

On a \(A\ou0=0\).

Si \(A=0\) alors \(A\ou1=0\). Sinon, \(A\ou1\) est l'unique polynôme unitaire associé à \(A\).
\end{rem}

\begin{rem}
Soient \(A,B\in\poly\).

Le PPCM de \(A\) et \(B\) est le multiple commun à \(A\) et \(B\) nul ou unitaire et qui divise tous les autres multiples communs à \(A\) et \(B\).

Il est donc caractérisé par : \[A\ou B\text{ nul ou unitaire}\qquad\text{et}\qquad\poly A\inter\poly B=\poly\paren{A\ou B}\] ou par \[A\ou B\text{ nul ou unitaire}\qquad\text{et}\qquad\quantifs{\forall P\in\poly}A\ou B\divise P\ssi\croch{A\divise P\text{ et }B\divise P}.\]

Si \(A=0\) ou \(B=0\) alors \(A\ou B=0\).

Sinon, \(A\ou B\) est le polynôme unitaire de plus bas degré qui est multiple de \(A\) et \(B\) (selon la \thref{prop:divisibilitéDansPoly}).
\end{rem}

\begin{rem}
Soient \(A,B\in\poly\).

Le polynôme \(A\ou B\) est nul si, et seulement si, on a \(A=B=0\).

Si \(A\) ou \(B\) est non-nul, on impose systématiquement au polynôme noté \(A\ou B\) d'être unitaire.

En revanche, on s'autorise parfois à appeler \guillemets{plus petit commun multiple de \(A\) et \(B\)} tout polynôme \(P\) associé à \(A\ou B\), même s'il n'est pas unitaire.

Ainsi, on dit que \(2X^3-2X\) est un PPCM de \(X^2-1\) et \(X^2-X\).
\end{rem}

\begin{rem}
Soient \(A,B\in\poly\).

On définit le polynôme \(A_1\) en posant : \begin{itemize}
\item si \(A=0\) alors \(A_1=0\) ;

\item si \(A\not=0\) alors \(A_1\) est l'unique polynôme unitaire associé à \(A\). \\
\end{itemize}

On a \[A\ou B=A_1\ssi B\divise A.\]
\end{rem}

\begin{prop}\thlabel{prop:PPCMFoisPolyAssociés}
Soient \(A,B,P\in\poly\).

Les polynômes \(PA\ou PB\) et \(P\times\paren{A\ou B}\) sont associés.
\end{prop}

\begin{dem}
Si \(P=0\), la proposition est vraie.

Supposons \(P\not=0\).

Montrons que \(PA\ou PB\) divise \(P\times\paren{A\ou B}\).

On a \(A\) et \(B\) divisent \(A\ou B\).

Donc \(PA\) et \(PB\) divisent \(P\times\paren{A\ou B}\).

Donc \(PA\ou PB\) divise \(P\times\paren{A\ou B}\).

Montrons que \(P\times\paren{A\ou B}\) divise \(PA\ou PB\).

On remarque que \(P\) divise \(PA\ou PB\) (car \(P\divise PA\divise PA\ou PB\)).

Soit \(M\in\poly\) tel que \(MP=PA\ou PB\).

On a \(PA\) et \(PB\) divisent \(MP\).

Comme \(P\not=0\), on en déduit \(A\) et \(B\) divisent \(M\).

Donc \(A\ou B\) divise \(M\).

Donc \(P\times\paren{A\ou B}\) divise \(PM\).

Donc \(P\times\paren{A\ou B}\) divise \(PA\ou PB\).

Conclusion : les polynômes \(PA\ou PB\) et \(P\times\paren{A\ou B}\) sont associés.
\end{dem}

\begin{prop}
Soient \(A,B\in\poly\).

Les polynômes \(\paren{A\ou B}\paren{A\et B}\) et \(AB\) sont associés.
\end{prop}

\begin{dem}
\note{Exercice}
\end{dem}

\subsubsection{PPCM de plusieurs polynômes}

\begin{prop}
La loi \(\ou\) est une loi de composition interne sur \(\poly\).

Elle est associative et commutative.
\end{prop}

\begin{dem}
\note{Exercice}
\end{dem}

\begin{defprop}
Soient \(r\in\Ns\) et \(A_1,\dots,A_r\in\poly\).

Les diviseurs communs à \(A_1,\dots,A_r\) sont les diviseurs du polynôme \(A_1\ou\dots\ou A_r\).

Ce polynôme et ses polynômes associés sont appelés les plus petits communs multiples des polynômes \(A_1,\dots,A_r\).
\end{defprop}

\begin{dem}
\note{Exercice}
\end{dem}

\begin{prop}
Soient \(r\in\Ns\) et \(A_1,\dots,A_r,P\in\poly\).

Les polynômes \(PA_1\ou\dots\ou PA_r\) et \(P\times\paren{A_1\ou\dots\ou A_r}\) sont associés.
\end{prop}

\begin{dem}
Découle de la \thref{prop:PPCMFoisPolyAssociés} par récurrence sur \(r\in\Ns\).
\end{dem}

\subsection{Polynômes irréductibles}

\subsubsection{Définition}

\begin{defi}[Polynôme irréductible]
Un polynôme \(P\in\poly\) est dit irréductible sur \(\K\) ou dans \(\poly\) s'il est non-constant et s'il n'est pas le produit de deux polynômes non-constants : \[\begin{dcases}P\text{ non-constant} \\ \quantifs{\forall Q_1,Q_2\in\poly}P=Q_1Q_2\imp\croch{Q_1\text{ constant ou }Q_2\text{ constant}}\end{dcases}\]

On a donc \[\quantifs{\forall P\in\poly}P\text{ irréductible}\ssi\croch{P\not\in\K\text{ et }\ensdiv{P}=\accol{1;P_1}}\] avec \(P_1\) l'unique polynôme unitaire associé à \(P\).
\end{defi}

\begin{dem}
\impdir

Supposons \(P\) irréductible.

Alors \(P\not\in\K\) (en particulier, \(P\not=0\)).

Déterminons \(\ensdiv{P}\) :

\analyse

Soit \(D\in\ensdiv{P}\).

On a \(D\not=0\) car \(P\not=0\) donc \(D\) est unitaire.

Soit \(Q\in\poly\) tel que \(P=QD\).

Si \(D\) est constant alors \(D=1\) car \(D\) est unitaire.

Si \(Q\) est constant alors \(Q\) est constant et non-nul (car \(P\not=0\)) donc \(D\) est unitaire et associé à \(P\) donc \(D=P_1\).

\synthese \(1\) et \(P_1\) divisent \(P\) et sont irréductibles.

\conclusion On a \(\ensdiv{P}=\accol{1;P_1}\).

\imprec

Supposons \(P\not\in\K\) et \(\ensdiv{P}=\accol{1;P_1}\).

Montrons que \(P\) est irréductible.

On a bien \(P\) non-constant.

Soient \(Q_1,Q_2\in\poly\) tels que \(P=Q_1Q_2\).

Comme \(P\not=0\), on a \(Q_1\not=0\) et \(Q_2\not=0\).

On note \(\lambda_1\) le coefficient dominant de \(Q_1\).

On a \(P=\paren{\dfrac{1}{\lambda_1}Q_1}\lambda_1Q_2\) donc \(\dfrac{1}{\lambda_1}Q_1\in\ensdiv{P}=\accol{1;P_1}\).

Si \(\dfrac{1}{\lambda_1}Q_1=1\) alors \(Q_1\) est constant.

Si \(\dfrac{1}{\lambda_1}Q_1=P_1\) alors \(Q_1\) est associé à \(P\).

Donc \(\deg Q_1=\deg P\).

Or \(P=Q_1Q_2\) donc \(\deg P=\deg Q_1+\deg Q_2\).

Donc \(\deg Q_2=0\) donc \(Q_2\) est constant.
\end{dem}

\begin{rem}
On n'impose pas aux polynômes irréductibles d'être unitaires.
\end{rem}

\begin{ex}
\begin{enumerate}
\item Tout polynôme de degré \(1\) est irréductible. \\

\item Le polynôme \(P=X^2+1\) est irréductible sur \(\R\) mais pas sur \(\C\).
\end{enumerate}
\end{ex}

\begin{dem}[1]
Soit \(P\in\poly\) tel que \(\deg P=1\).

Montrons que \(P\) est irréductible.

On a \(P\) non-constant.

Soient \(Q_1,Q_2\in\poly\) tels que \(P=Q_1Q_2\).

On a \(P\not=0\) donc \(Q_1\not=0\) et \(Q_2\not=0\).

On a \(1=\deg P=\deg Q_1+\deg Q_2\).

Donc \(\deg Q_1=0\) ou \(\deg Q_2=0\).

Donc \(Q_1\) constant ou \(Q_2\) constant.

Donc \(P\) est irréductible.
\end{dem}

\begin{dem}[2]
On a \(X^2+1=\paren{X-\i}\paren{X+\i}\) donc \(P\) n'est pas irréductible sur \(\C\).

Montrons que \(P\) est irréductible sur \(\R\).

On a \(P\) non-constant.

Soient \(Q_1,Q_2\in\poly[\R]\) tels que \(P=Q_1Q_2\).

On a \(2=\deg P=\deg Q_1+\deg Q_2\).

Donc \(\paren{\deg Q_1,\deg Q_2}\in\accol{\paren{2,0};\paren{1,1};\paren{0,2}}\).

Supposons par l'absurde \(\paren{\deg Q_1,\deg Q_2}=\paren{1,1}\).

Alors \(Q_1\) et \(Q_2\) admettent chacun une racine dans \(\R\).

Donc \(P\) admet une racine dans \(\R\) : contradiction.

Donc \(\paren{\deg Q_1,\deg Q_2}\in\accol{\paren{2,0};\paren{0,2}}\).

Donc \(Q_1\) ou \(Q_2\) est constant.

Donc \(P\) est irréductible.
\end{dem}

\begin{rem}\thlabel{rem:polynômeIrréductibleEstDeDegré1S'ilAdmetUneRacine}
Soit \(P\in\poly\) irréductible.

\begin{enumerate}
\item Si \(P\) admet une racine alors \(\deg P=1\). \\

\item D'où, par contraposée : si \(\deg P\geq2\) alors \(P\) n'admet aucune racine.
\end{enumerate}
\end{rem}

\begin{dem}
Montrons (1).

Supposons que \(P\) admet une racine \(\lambda\in\K\).

Alors \(X-\lambda\divise P\).

Soit \(Q\in\poly\) tel que \(\paren{X-\lambda}Q=P\).

Comme \(P\) est irréductible, on a \(X-\lambda\) constant ou \(Q\) constant.

Donc \(Q\) est constant.

De plus, \(Q\not=0\) car \(P\not=0\).

Donc \(\deg Q=0\) et \(\deg P=\deg\paren{X-\lambda}+\deg Q=1+0=1\).
\end{dem}

\begin{rem}
Soient \(P,Q\in\poly\).

On suppose que \(P\) est irréductible.

Alors \[P\not\divise Q\ssi P\et Q=1.\]
\end{rem}

\subsubsection{Polynômes irréductibles sur \(\C\) et \(\R\)}

\begin{theo}
Les polynômes irréductibles dans \(\poly[\C]\) sont ceux de degré \(1\).
\end{theo}

\begin{dem}
\increc On a déjà vu que tout polynôme de degré \(1\) est irréductible.

\incdir

Soit \(P\in\poly[\C]\) irréductible.

Selon le théorème de D'Alembert-Gauss, \(P\) admet une racine car \(P\) n'est pas constant.

Donc \(\deg P=1\) (selon la \thref{rem:polynômeIrréductibleEstDeDegré1S'ilAdmetUneRacine}).
\end{dem}

\begin{theo}
Les polynômes irréductibles dans \(\poly[\R]\) sont ceux de degré \(1\) et ceux de la forme \[aX^2+bX+c\] où \(\begin{dcases}a,b,c\in\R \\ a\not=0 \\ \Delta=b^2-4ac<0\end{dcases}\)
\end{theo}

\begin{dem}
\increc

Tout polynôme de degré \(1\) est irréductible.

Soient \(a,b,c\in\R\) tels que \(\begin{dcases}a\not=0 \\ \Delta=b^2-4ac<0\end{dcases}\)

Le polynôme \(P=aX^2+bX+c\) n'admet aucune racine dans \(\R\).

On a \(P\) non-constant.

Soient \(Q_1,Q_2\in\poly[\R]\) tels que \(P=Q_1Q_2\).

On a \(2=\deg P=\deg Q_1+\deg Q_2\).

De plus, \(\deg Q_1\not=1\) car sinon \(Q_1\) admettrait une racine dans \(\R\) et \(P\) aussi.

Donc \(\paren{Q_1,Q_2}\in\accol{\paren{0,2};\paren{2,0}}\).

Donc \(Q_1\) ou \(Q_2\) constant.

Donc \(P\) est irréductible.

\incdir

Soit \(P\in\poly[\R]\) irréductible.

On a \(P\) non-constant.

D'après le théorème de D'Alembert-Gauss, il existe une racine complexe de \(P\).

Soit \(\lambda\in\C\) tel que \(P\paren{\lambda}=0\).

Si \(\lambda\in\R\) alors \(X-\lambda\divise P\) (dans \(\poly[\R]\)).

Soit \(Q\in\poly[\R]\) tel que \(\paren{X-\lambda}Q=P\).

Comme \(P\) est irréductible, \(X-\lambda\) ou \(Q\) est constant donc \(Q\) est constant.

Donc \(\deg P=1\) car \(Q\not=0\).

Si \(\lambda\not\in\R\) alors \(P\paren{\conj{\lambda}}=0\).

Donc \(X-\lambda\) et \(X-\conj{\lambda}\) divisent \(P\).

Or \(\paren{X-\lambda}\et\paren{X-\conj{\lambda}}=1\) car \(\lambda\not=\conj{\lambda}\).

Donc \(\paren{X-\lambda}\paren{X-\conj{\lambda}}\divise P\) selon la \thref{prop:polysPremiersEntreEuxQuiDivisentDoncProduitDivise}.

Donc \(X^2-\paren{\lambda+\conj{\lambda}}X+\lambda\conj{\lambda}=X^2-2\Re\paren{\lambda}X+\abs{\lambda}^2\in\poly[\R]\).

Ainsi \(X^2-2\Re\paren{\lambda}X+\abs{\lambda}^2\divise P\).

Donc il existe \(Q\in\poly[\R]\) tel que \(\paren{X^2-2\Re\paren{\lambda}X+\abs{\lambda}^2}Q=P\) avec nécessairement \(Q\) constant car \(P\) est irréductible et \(Q\not=0\) car \(P\not=0\).

Finalement, on a \[\quantifs{\exists a\in\Rs}ax^2-2a\Re\paren{\lambda}X+\abs{\lambda}^2a=P\] et \(\Delta<0\) car \(P\) n'a pas de racine réelle.
\end{dem}

\subsection{Théorème fondamental de l'arithmétique des polynômes}

\subsubsection{Cas général}

Le théorème suivant affirme que tout polynôme non-constant s'écrit de façon unique comme le produit de polynômes irréductibles, à l'ordre des facteurs près et à des facteurs inversibles près :

\begin{theo}
Soit \(P\in\poly\excluant\K\).

Alors il existe un entier \(r\in\Ns\) et des polynômes irréductibles \(A_1,\dots,A_r\in\poly\) tels que \[P=A_1\dots A_r.\]

De plus, si l'on a une autre décomposition \[P=B_1\dots B_s\] avec \(s\in\Ns\) et \(B_1,\dots,B_s\in\poly\) des polynômes irréductibles, alors il existe une bijection \(\sigma:\interventierii{1}{r}\to\interventierii{1}{s}\) telle que \[\quantifs{\forall k\in\interventierii{1}{r}}A_k\text{ et }B_{\sigma\paren{k}}\text{ sont associés}.\]
\end{theo}

\begin{dem}
Pour tout \(n\in\Ns\), on note \(\P{n}\) la proposition : \guillemets{tout polynôme \(P\) de degré \(n\) s'écrit comme le produit de polynômes irréductibles, de façon unique à l'ordre des facteurs près et à des constantes multiplicatives non-nulles près}.

Soit \(P\in\poly\) tel que \(\deg P=1\).

\existence On a \(P=P\) et \(P\) est irréductible.

\unicite

Soient \(B_1,\dots,B_s\in\poly\) irréductibles et tels que \(P=B_1\dots B_s\).

On a \(\quantifs{\forall j\in\interventierii{1}{s}}\deg B_j\geq1\) (car \(B_j\) est non-constant).

Donc \(1=\deg P=\sum_{j=1}^s\deg B_j\geq\sum_{j=1}^s1=s\).

Donc \(s\leq1\).

De plus, \(s\not=0\) (sinon \(P=1\) est constant).

Donc \(s=1\).

Donc \(P=B_1\).

D'où \(\P{1}\).

Soit \(n\in\Ns\) tel que \(\quantifs{\forall k\in\interventierii{1}{n}}\P{k}\).

\existence

Soit \(P\in\poly\) tel que \(\deg P=n+1\).

Si \(P\) est irréductible, \(P=P\) convient.

Supposons \(P\) non-irréductible.

Il existe \(Q_1,Q_2\in\poly\) non-constants tels que \(P=Q_1Q_2\).

On a \(n+1=\deg P=\underbrace{\deg Q_1}_{\geq1}+\underbrace{\deg Q_2}_{\geq1}\).

Donc \(\deg Q_1\leq n\) et \(\deg Q_2\leq n\).

Selon l'hypothèse de récurrence, \(Q_1\) et \(Q_2\) sont produits de polynômes irréductibles donc leur produit \(P\) l'est aussi.

\unicite

Soient \(r,s\in\Ns\) et \(A_1,\dots,A_r,B_1,\dots,B_s\in\poly\) irréductibles.

Supposons \(P=A_1\dots A_r=B_1\dots B_s\).

Comme \(A_1\) est irréductible, on a \[\quantifs{\forall j\in\interventierii{1}{s}}\orenv{A_1\divise B_j \\ A_1\et B_j=1}\]

Si \(\quantifs{\forall j\in\interventierii{1}{s}}A_1\et B_j=1\) alors \(A_1\et\paren{B_1\dots B_s}=1\) donc \(A_1\et P=1\) : contradiction car \(A_1\divise P\).

Donc il existe \( j\in\interventierii{1}{s}\) tel que \(A_1\divise B_j\)  donc \(A_1\) est associé à \(B_j\) car \(A_1\) est non-constant et \(B_j\) est irréductible.

Soit \(\lambda\in\K\excluant\accol{0}\) tel que \(B_j=\lambda A_1\).

Quitte à renuméroter \(B_1,\dots,B_s\) et à remplacer \(B_1\) par \(\dfrac{1}{\lambda}B_1\) et \(B_2\) par \(\lambda B_2\), on a \(A_1=B_1\) donc \[A_2\dots A_r=B_2\dots B_s\] car \(\poly\) est intègre.

Posons \(Q=A_2\dots A_r\).

On a \(\deg Q\leq n\).

Selon l'hypothèse de récurrence appliquée à \(Q\), il existe \(\sigma:\interventierii{2}{r}\to\interventierii{1}{s}\) une bijection telle que \[\quantifs{\forall i\in\interventierii{1}{r}}A_i\text{ et }B_{\sigma\paren{i}}\text{ sont associés}.\]

Alors \(\fonction{\tilde{\sigma}}{\interventierii{1}{r}}{\interventierii{1}{s}}{i}{\begin{dcases}1 &\text{si }i=1 \\ \sigma\paren{i} &\text{si }i\geq2\end{dcases}}\) vérifie \[\quantifs{\forall i\in\interventierii{1}{r}}A_i\text{ et }B_{\tilde{\sigma}\paren{i}}\text{ sont associés}.\]

D'où l'unicité.

Donc, par récurrence forte, \(\quantifs{\forall n\in\Ns}\P{n}\).
\end{dem}

\begin{cor}[Reformulation]
Soit \(P\in\poly\excluant\accol{0}\).

On note \(\lambda\in\K\excluant\accol{0}\) le coefficient dominant de \(P\).

Alors il existe \(r\in\N\), \(A_1,\dots,A_r\in\poly\) irréductibles, unitaires et deux à deux distincts et \(\alpha_1,\dots,\alpha_2\in\Ns\) tels que \[P=\lambda A_1^{\alpha_1}\dots A_r^{\alpha_r}.\]

De plus, si l'on a une autre décomposition \[P=\mu B_1^{\beta_1}\dots B_s^{\beta_s}\] avec \(\mu\in\K\), \(s\in\N\), \(B_1,\dots,B_s\in\poly\) irréductibles, unitaires et deux à deux distincts, et \(\beta_1,\dots,\beta_s\in\Ns\), alors \(\lambda=\mu\) et il existe une bijection \(\sigma:\interventierii{1}{r}\to\interventierii{1}{s}\) telle que \[\quantifs{\forall k\in\interventierii{1}{r}}A_k=B_{\sigma\paren{k}}\qquad\text{et}\qquad\alpha_k=\beta_{\sigma\paren{k}}.\]
\end{cor}

\subsubsection{Polynômes à coefficients complexes}

\begin{theo}
Soit \(P\in\poly[\C]\excluant\accol{0}\).

Alors \(P\) s'écrit de façon unique, à permutation de facteurs près : \[P=\mu\paren{X-\lambda_1}^{\alpha_1}\dots\paren{X-\lambda_r}^{\alpha_r}\text{ où }\begin{dcases}\mu\in\Cs \\ r\in\N \\ \lambda_1,\dots,\lambda_r\in\C\text{ deux à deux distincts} \\ \alpha_1,\dots,\alpha_r\in\Ns\end{dcases}\]

Le coefficient dominant de \(P\) est \(\mu\).

Les racines de \(P\) sont \(\lambda_1,\dots,\lambda_r\).

Pour tout \(j\in\interventierii{1}{r}\), l'entier \(\alpha_j\) est appelé multiplicité de la racine \(\lambda_j\).

Le polynôme \(P\) admet \(r\) racines comptées sans multiplicité et \(\alpha_1+\dots+\alpha_r\) racines comptées avec multiplicité.
\end{theo}

\begin{exo}[À retenir]
Soit \(n\in\Ns\).

Décomposer le polynôme \(X^n-1\) en produit de polynômes irréductibles sur \(\C\).
\end{exo}

\begin{corr}
Le coefficient dominant de \(X^n-1\) est \(1\).

Les racines de \(X^n-1\) sont les éléments de \(\U_n\) (il y en a \(n\)).

Notons \(\alpha_\omega\in\Ns\) la multiplicité de \(\omega\in\U_n\).

On a \(X^n-1=1\times\prod_{\omega\in\U_n}\paren{X-\omega}^{\alpha_\omega}\).

Donc \(\deg\paren{X^n-1}=\sum_{\omega\in\U_n}\deg\paren{X-\omega}^{\alpha_\omega}\).

Donc \(n=\sum_{\omega\in\U_n}\alpha_\omega\).

Donc \(\quantifs{\forall\omega\in\U_n}\alpha_\omega=1\).

Finalement : \[X^n-1=\prod_{\omega\in\U_n}\paren{X-\omega}.\]
\end{corr}

\begin{prop}
Soient \(P,Q\in\poly[\C]\).

Alors \(P\) et \(Q\) sont premiers entre eux si, et seulement si, \(P\) et \(Q\) n'ont aucune racine commune.

Autrement dit, par contraposée : \[P\et Q\not=1\ssi\quantifs{\exists\lambda\in\C}P\paren{\lambda}=Q\paren{\lambda}=0.\]
\end{prop}

\begin{dem}
S'il existe une racine \(\lambda\in\C\) commune à \(P\) et \(Q\), alors \(X-\lambda\) divise \(P\) et \(Q\).

Donc \(X-\lambda\divise P\et Q\).

Donc \(P\et Q\not=1\).

S'il n'existe aucune racine commune à \(P\) et \(Q\) :

Décomposons \(P\) et \(Q\) en produits de polynômes irréductibles : \(\begin{dcases}P=\lambda\paren{X-\lambda_1}\dots\paren{X-\lambda_\alpha} \\ Q=\mu\paren{X-\mu_1}\dots\paren{X-\mu_\beta}\end{dcases}\) où \(\begin{dcases}\alpha,\beta\in\N \\ \lambda,\mu\in\Cs \\ \lambda_1,\dots,\lambda_\alpha,\mu_1,\dots,\mu_\beta\in\C\end{dcases}\)

On a \(\accol{\lambda_1;\dots;\lambda_\alpha}\inter\accol{\mu_1;\dots;\mu_\beta}=\ensvide\).

Soit \(j\in\interventierii{1}{\beta}\).

On a \(\quantifs{\forall i\in\interventierii{1}{\alpha}}\paren{X-\lambda_i}\et\paren{X-\mu_j}=1\) car \(\lambda_i\not=\mu_j\).

Donc, selon le \thref{cor:polysTousPremiersImpliqueProduitPremier}, on a \[\lambda\prod_{i=1}^{\alpha}\paren{X-\lambda_i}\et\paren{X-\mu_j}=1.\]

D'où, selon le \thref{cor:polysTousPremiersImpliqueProduitPremier}, on a \[\lambda\prod_{i=1}^{\alpha}\paren{X-\lambda_i}\et\mu\prod_{j=1}^{\beta}\paren{X-\mu_j}=1.\]

Donc \(P\et Q=1\).
\end{dem}

\begin{prop}
Soient \(P,Q\in\poly[\C]\excluant\accol{0}\).

Alors \(P\) divise \(Q\) si, et seulement si, toute racine \(\lambda\) de \(P\) est racine de \(Q\) avec une multiplicité comme racine de \(Q\) au moins égale à sa multiplicité comme racine de \(P\).
\end{prop}

\begin{ex}
On a \(2\paren{X-1}^{\alpha}\paren{X+\i}^\beta\divise\paren{X-1}^2\paren{X+\i}^3\ssi\begin{dcases}\alpha\leq2 \\ \beta\leq3\end{dcases}\)
\end{ex}

\subsubsection{Polynômes à coefficients réels}

\begin{theo}
Soit \(P\in\poly[\R]\excluant\accol{0}\).

Alors \(P\) s'écrit de façon unique, à permutation des facteurs près : \[P=\mu\times\paren{X-\lambda_1}^{\alpha_1}\dots\paren{X-\lambda_r}^{\alpha_r}\times\paren{X^2+b_1X+c_1}^{\beta_1}\dots\paren{X^2+b_sX+c_s}^{\beta_s}\] où \(\begin{dcases}\mu\in\Rs \\ r,s\in\N \\ \lambda_1,\dots,\lambda_r\in\R\text{ deux à deux distincts} \\ \paren{b_1,c_1},\dots,\paren{b_s,c_s}\in\R^2\text{ deux à deux distincts} \\ \text{les facteurs de degré \(2\) sont irréductibles, \cad }\quantifs{\forall i\in\interventierii{1}{s}}b_i^2-4c_i<0 \\ \alpha_1,\dots,\alpha_r,\beta_1,\dots,\beta_s\in\Ns\end{dcases}\)

Le coefficient dominant de \(P\) est \(\mu\).

Les racines réelles de \(P\) sont \(\lambda_1,\dots,\lambda_r\).

Pour tout \(j\in\interventierii{1}{r}\), l'entier \(\alpha_j\) est appelé multiplicité de la racine \(\lambda_j\).

Le polynôme \(P\) admet \(r\) racines réelles comptées sans multiplicité et \(\alpha_1+\dots+\alpha_r\) racines comptées avec multiplicité.
\end{theo}

\begin{exo}[À retenir]
Soit \(n\in\Ns\).

Décomposer le polynôme \(X^n-1\) en produit de polynômes irréductibles sur \(\R\).
\end{exo}

\begin{corr}
On a vu \[X^n-1=\prod_{\omega\in\U_n}\paren{X-\omega}=\prod_{k=0}^{n-1}\paren{X-\e{\frac{2\i k\pi}{n}}}.\]

Si \(n\) est pair, on a \(n=2m\) où \(m=\dfrac{n}{2}\).

Donc \(\U_n=\accol{-1;1}\union\bigunion_{k=1}^{m-1}\accol{\e{\frac{2\i k\pi}{n}};\e{\frac{-2\i k\pi}{n}}}\).

D'où \[\begin{aligned}
X^n-1&=\paren{X-1}\paren{X+1}\prod_{k=1}^{m-1}\paren{X-\e{\frac{2\i k\pi}{n}}}\paren{X-\e{\frac{-2\i k\pi}{n}}} \\
&=\paren{X-1}\paren{X+1}\prod_{k=1}^{\frac{n}{2}-1}\paren{X^2-2\cos\paren{\dfrac{2k\pi}{n}}X+1}
\end{aligned}\]

Si \(n\) est impair, on a \(n=2m+1\) avec \(m=\dfrac{n-1}{2}\).

On obtient de même \(\U_n=\accol{1}\union\bigunion_{k=1}^m\accol{\e{\frac{2\i k\pi}{n}};\e{\frac{-2\i k\pi}{n}}}\).

Donc \[X^n-1=\paren{X-1}\prod_{k=1}^{\frac{n-1}{2}}\paren{X^2-2\cos\paren{\dfrac{2k\pi}{n}}X+1}.\]
\end{corr}

\begin{ex}
On a \[\begin{aligned}
X^4-1&=\paren{X^2-1}\paren{X^2+1} \\
&=\paren{X-1}\paren{X+1}\paren{X^2+1}
\end{aligned}\] et \[X^5-1=\paren{X-1}\paren{X^2-2\cos\paren{\dfrac{2\pi}{5}}X+1}\paren{X^2-2\cos\paren{\dfrac{4\pi}{5}}X+1}.\]
\end{ex}

\begin{prop}[Racines complexes de polynômes réels]
Soient \(P\in\poly[\R]\) et \(\lambda\in\C\) une racine de \(P\).

On sait, d'après la \thref{prop:lambdaBarreEstRacine}, que \(\conj{\lambda}\) est racine de \(P\).

Les racines \(\lambda\) et \(\conj{\lambda}\) ont même multiplicité.
\end{prop}

\begin{dem}
On considère la décomposition de \(P\) en produit d'irréductibles sur \(\C\) : \[P=\mu\paren{X-\lambda_1}^{\alpha_1}\dots\paren{X-\lambda_r}^{\alpha_r}\] où \(\begin{dcases}\mu\in\Cs \\ \lambda_1,\dots,\lambda_r\in\C\text{ deux à deux distincts} \\ \alpha_1,\dots,\alpha_r\in\Ns\end{dcases}\)

On pose \(\lambda=\lambda_1\).

En conjuguant, on a \[\conj{P}=\conj{\mu}\paren{X-\conj{\lambda_1}}^{\alpha_1}\dots\paren{X-\conj{\lambda_r}}^{\alpha_r}=P.\]

Selon l'unicité de l'écriture en produit d'irréductibles, il existe \(k\in\interventierii{1}{r}\) tel que \(\paren{X-\lambda_1}^{\alpha_1}=\paren{X-\lambda_k}^{\alpha_k}\).

Donc \(\lambda_k=\conj{\lambda}\) et \(\alpha_k=\alpha_1\).

Or \(\alpha_k\) est la multiplicité de la racine \(\conj{\lambda}\) de \(P\) et \(\alpha_1\) est la multiplicité de la racine \(\lambda\) de \(P\).
\end{dem}

\section{Racines des polynômes}

\subsection{Multiplicités}

\begin{defi}[Multiplicité d'une racine]
Soient \(P\in\poly\) non-nul et \(\lambda\in\K\) une racine de \(P\).

On a vu, à la \thref{prop:XMoinsLambdaDiviseP}, que \(X-\lambda\) divise \(P\).

On appelle multiplicité de \(\lambda\) comme racine de \(P\) l'exposant de \(X-\lambda\) dans la décomposition de \(P\) en produit de polynômes irréductibles.

On dit que \(\lambda\) est racine simple de \(P\) si sa multiplicité vaut \(1\) ; sinon on dit que \(\lambda\) est racine multiple de \(P\).
\end{defi}

\begin{ex}
Pour le polynôme \(P=X^2\paren{X-4}^3\paren{X-7}\in\poly[\R]\) :

\begin{itemize}
\item \(0\) est racine de multiplicité \(2\) (ou \guillemets{racine double}) \\

\item \(4\) est racine de multiplicité \(3\) (ou \guillemets{racine triple}) \\

\item et \(7\) est racine simple.
\end{itemize}

Ainsi, \(P\) admet :

\begin{itemize}
\item \(3\) racines comptées sans multiplicité \\

\item \(6\) racines comptées avec multiplicité.
\end{itemize}
\end{ex}

\begin{rem}
Soient \(P\in\poly\) non-nul et \(\lambda\in\K\) une racine de \(P\).

La multiplicité de \(\lambda\) est l'entier \(\alpha\in\Ns\) caractérisé par : \[\begin{dcases}\paren{X-\lambda}^{\alpha}\divise P \\ \paren{X-\lambda}^{\alpha+1}\not\divise P\end{dcases}\]

En particulier, \(\lambda\) est racine multiple de \(P\) si, et seulement si, on a : \[\paren{X-\lambda}^2\divise P.\]
\end{rem}

\begin{prop}[Caractérisation à l'aide des dérivés]
Soient \(P\in\poly\) non-nul et \(\lambda\in\K\) une racine de \(P\).

La multiplicité de \(\lambda\) est l'entier \(\alpha\in\Ns\) caractérisé par : \[\begin{dcases}\quantifs{\forall j\in\interventierii{0}{\alpha-1}}P\deriv{j}\paren{\lambda}=0 \\ P\deriv{\alpha}\paren{\lambda}\not=0\end{dcases}\]

En particulier, \(\lambda\) est racine multiple de \(P\) si, et seulement si, on a : \[P\paren{\lambda}=P\prim\paren{\lambda}=0.\]
\end{prop}

\begin{dem}
On note \(n\) le degré de \(P\) et \(\lambda\) le coefficient dominant de \(P\).

On a \(P\deriv{n}=\lambda n!\) donc \(P\deriv{n}\paren{\lambda}=\lambda n!\not=0\).

On note \(\gamma\in\N\) le plus petit entier tel que \(P\deriv{\gamma}\paren{\lambda}\not=0\) (toute partie non-vide de \(\N\) admet un minimum).

On a \(\begin{dcases}\quantifs{\forall k\in\interventierii{0}{\gamma-1}}P\deriv{k}\paren{\lambda}=0 \\ P\deriv{\gamma}\paren{\lambda}\not=0\end{dcases}\)

Donc selon la formule de Taylor pour les polynômes : \[\begin{aligned}
P&=\sum_{k=\gamma}^n\dfrac{P\deriv{k}\paren{\lambda}}{k!}\paren{X-\lambda}^k \\
&=\paren{X-\lambda}^{\gamma}Q\text{ où }Q=\sum_{k=\gamma}^n\dfrac{P\deriv{k}\paren{\lambda}}{k!}\paren{X-\lambda}^{k-\gamma}
\end{aligned}\] et \(Q\paren{\lambda}=\dfrac{P\deriv{\gamma}\paren{\lambda}}{\gamma!}\not=0\).

Donc \(X-\lambda\not\divise Q\) donc \[\begin{dcases}\paren{X-\lambda}^{\gamma}\divise P \\ \paren{X-\lambda}^{\gamma+1}\not\divise P\end{dcases}\]

Donc la multiplicité de \(\lambda\) comme racine de \(P\) est \(\gamma\).
\end{dem}

\subsection{Nombre de racines d'un polynôme}\label{subsec:nbRacinesPoly}

\begin{prop}
Soit \(P\in\poly\) de degré \(n\in\N\).

Alors \(P\) admet au plus \(n\) racines comptées avec multiplicité.

En particulier, \(P\) admet au plus \(n\) racines comptées sans multiplicité.
\end{prop}

\begin{dem}
Notons \(\lambda_1,\dots,\lambda_r\) les racines de \(P\) deux à deux distinctes et \(\alpha_1,\dots,\alpha_r\) leurs multiplicités respectives, avec \(r\in\N\).

On a \(\paren{X-\lambda_1}^{\alpha_1}\dots\paren{X-\lambda_r}^{\alpha_r}\divise P\).

Or, comme \(P\not=0\), on a \[\deg\paren{\paren{X-\lambda_1}^{\alpha_1}\dots\paren{X-\lambda_r}^{\alpha_r}}\leq\deg P.\]

Donc (avec multiplicité) : \[\alpha_1+\dots+\alpha_r\leq n.\]

Donc (sans multiplicité) : \[r\leq n.\]
\end{dem}

\begin{cor}
Soient \(n\in\N\) et \(P\in\polydeg{n}\).

Si \(P\) admet \(n+1\) racines alors \(P=0\).
\end{cor}

\begin{cor}
Soit \(P\in\poly\).

Si \(P\) admet une infinité de racines alors \(P=0\).
\end{cor}

\begin{cor}
On note \(\Pol{\K}{\K}\) l'ensemble des fonctions polynomiales de \(\K\) dans \(\K\) (c'est un sous-anneau de \(\F{\K}{\K}\) selon la \thref{prop:anneauDesFonctionsPolynomiales}).

Si \(\K\) est infini, alors l'application \[\fonction{\phi}{\poly}{\Pol{\K}{\K}}{P}{\tilde{P}}\] est un isomorphisme d'anneaux.
\end{cor}

\begin{dem}
On a déjà vu que \(\phi\) est un morphisme d'anneaux (\cf \thref{prop:anneauDesFonctionsPolynomiales}).

\(\phi\) est clairement une surjection.

Montrons que \(\phi\) est injective.

Soient \(P,Q\in\poly\) tels que \(\phi\paren{P}=\phi\paren{Q}\).

On a \(\phi\paren{P-Q}=0\), \cad \[\quantifs{\forall\lambda\in\K}\paren{P-Q}\paren{\lambda}=0.\]

Donc \(P-Q\) admet une infinité de racines (car \(\K\) est infini).

Donc \(P-Q=0\) donc \(P=Q\) donc \(\phi\) est injectif.
\end{dem}

\subsection{Polynômes interpolateurs de Lagrange}

\begin{defprop}[Polynôme interpolateur de Lagrange]\thlabel{defprop:polynômeInterpolateurDeLagrange}
Soient \(x_0,\dots,x_n\in\K\) deux à deux distincts et \(y_0,\dots,y_n\in\K\).

Les polynômes \(L_0,\dots,L_n\in\polydeg{n}\) définis par : \[\quantifs{\forall j\in\interventierii{0}{n}}L_j=\prod_{k\in\interventierii{0}{n}\excluant\accol{j}}\dfrac{X-x_k}{x_j-x_k}\] vérifient \[\quantifs{\forall i,j\in\interventierii{0}{n}}L_j\paren{x_i}=\delta_{ij}=\begin{dcases}1 &\text{si }i=j \\ 0 &\text{si }i\not=j\end{dcases}\]

Le polynôme \[y_0L_0+\dots+y_nL_n\] est l'unique polynôme \(P\in\polydeg{n}\) tel que \[\quantifs{\forall k\in\interventierii{0}{n}}P\paren{x_k}=y_k.\]
\end{defprop}

\begin{dem}
\unicite

Soient \(P,Q\in\polydeg{n}\) tel que \(\quantifs{\forall k\in\interventierii{0}{n}}P\paren{x_k}=Q\paren{x_k}=y_k\).

On a \(\quantifs{\forall k\in\interventierii{0}{n}}\paren{P-Q}\paren{x_k}=0\).

Donc \(P-Q\) admet \(n+1\) racines et \(P-Q\in\polydeg{n}\).

Donc \(P-Q=0\).

Donc \(P=Q\).
\end{dem}

\begin{exo}
Donner l'unique polynôme \(P\in\polydeg[\R]{3}\) tel que \[\begin{dcases}P\paren{-2}=-30 \\ P\paren{0}=-4 \\ P\paren{1}=3 \\ P\paren{2}=22\end{dcases}\]
\end{exo}

\begin{corr}
On pose \[\begin{aligned}
P&=-30\dfrac{\paren{X-0}\paren{X-1}\paren{X-2}}{\paren{-2-0}\paren{-2-1}\paren{-2-2}}-4\dfrac{\paren{X+2}\paren{X-1}\paren{X-2}}{\paren{0+2}\paren{0-1}\paren{0-2}}+3\dfrac{\paren{X+2}\paren{X-0}\paren{X-2}}{\paren{1+2}\paren{1-0}\paren{1-2}} \\
&\color{white}=\color{black}+22\dfrac{\paren{X+2}\paren{X-0}\paren{X-1}}{\paren{2+2}\paren{2-0}\paren{2-1}} \\
&=\dfrac{30}{24}\paren{X^2-X}\paren{X-2}-\paren{X^2+X-2}\paren{X-2}-\paren{X^2+2X}\paren{X-2}+\dfrac{22}{8}\paren{X^2+2X}\paren{X-1} \\
&=\dfrac{5}{4}\paren{X^3-3X^2+2X}-\paren{X^3-X^2-4X+4}-\paren{X^3-4X}+\dfrac{11}{4}\paren{X^3+X^2-2X} \\
&=2X^3+5X-4
\end{aligned}\]
\end{corr}

\begin{exo}
On garde les notations de la \thref{defprop:polynômeInterpolateurDeLagrange}.

Compléter les égalités suivantes :

\begin{enumerate}
\item \(\quantifs{\forall P\in\polydeg{n}}P\paren{x_0}L_0+\dots+P\paren{x_n}L_n=\) \\

\item \(L_0+\dots+L_n=\)
\end{enumerate}
\end{exo}

\begin{corr}
On a :

\begin{enumerate}
\item \(\quantifs{\forall P\in\polydeg{n}}P\paren{x_0}L_0+\dots+P\paren{x_n}L_n=P\) \\

\item \(L_0+\dots+L_n=1\)
\end{enumerate}
\end{corr}

\subsection{Polynômes scindés}

\begin{defi}[Polynôme scindé]
Soit \(P\in\poly\excluant\accol{0}\).

On dit que \(P\) est scindé (sur \(\K\)) s'il est produit de polynômes de degré \(1\), \cad s'il admet \(\deg P\) racines comptées avec multiplicité.
\end{defi}

\begin{ex}
Le polynôme \(X^2+1\) est scindé sur \(\C\) mais pas sur \(\R\).
\end{ex}

\begin{prop}
Sur \(\C\), tout polynôme est scindé.

Sur \(\R\), un polynôme est scindé si, et seulement si, dans sa décomposition en produit d'irréductibles, il n'y a aucun polynôme de degré \(2\). Cela revient à dire que toutes les racines complexes du polynômes sont réelles.
\end{prop}

\begin{defi}
Soit \(P\in\poly\excluant\accol{0}\).

On dit que \(P\) est scindé à racines simples (sur \(\K\)) s'il est scindé (sur \(\K\)) et si toutes ses racines sont simples (\cad de multiplicité \(1\)).

Cela revient à dire que \(P\) admet \(\deg P\) racines comptées sans multiplicité.
\end{defi}

\begin{bilan}
Soit \(P\in\poly\excluant\accol{0}\).

On note \(n\) le nombre de racines de \(P\) comptées sans multiplicité.

On note \(N\) le nombre de racines de \(P\) comptées avec multiplicité.

On a \[n\underset{\text{(1)}}{\leq}N\underset{\text{(2)}}{\leq}\deg P.\]

De plus :

\begin{itemize}
\item (1) est une égalité si, et seulement si, \(P\) est à racines simples. \\

\item (2) est une égalité si, et seulement si, \(P\) est scindé. \\

\item (1) et (2) sont des égalités si, et seulement si, \(P\) est scindé à racines simples.
\end{itemize}
\end{bilan}

\begin{defi}[Polynômes symétriques élémentaires]
Soit \[P=a_nX^n+\dots+a_0X^0\in\poly\] un polynôme de degré \(n\in\N\), où \(a_0,\dots,a_n\in\K\).

On suppose ce polynôme scindé : \[P=\mu\paren{X-\lambda_1}\dots\paren{X-\lambda_n},\] où \(\mu,\lambda_1,\dots,\lambda_n\in\K\).

Les polynômes symétriques élémentaires en les racines de \(P\) sont les sommes \(\sigma_1,\dots,\sigma_n\) définies par : \[\quantifs{\forall k\in\interventierii{1}{n}}\sigma_k=\sum_{1\leq i_1<\dots<i_k\leq n}\lambda_{i_1}\times\dots\times\lambda_{i_k}.\]
\end{defi}

\begin{ex}
Si \(n=3\), on a :

\begin{description}
\item \(\sigma_1=\lambda_1+\lambda_2+\lambda_3\) \\

\item \(\sigma_2=\lambda_1\lambda_2+\lambda_1\lambda_3+\lambda_2\lambda_3\) \\

\item \(\sigma_3=\lambda_1\lambda_2\lambda_3\).
\end{description}

Si \(n=4\), on a :

\begin{description}
\item \(\sigma_1=\lambda_1+\lambda_2+\lambda_3+\lambda_4\) \\

\item \(\sigma_2=\lambda_1\lambda_2+\lambda_1\lambda_3+\lambda_1\lambda_4+\lambda_2\lambda_3+\lambda_2\lambda_4+\lambda_3\lambda_4\) \\

\item \(\sigma_3=\lambda_1\lambda_2\lambda_3+\lambda_1\lambda_2\lambda_4+\lambda_1\lambda_3\lambda_4+\lambda_2\lambda_3\lambda_4\) \\

\item \(\sigma_4=\lambda_1\lambda_2\lambda_3\lambda_4\).
\end{description}
\end{ex}

\begin{prop}[Relations coefficients/racines]
On garde les notations de la définition précédente.

Les polynômes symétriques élémentaires en les racines de \(P\) s'expriment en fonction des coefficients de \(P\) : \[\quantifs{\forall k\in\interventierii{1}{n}}\sigma_k=\sum_{1\leq i_1<\dots<i_k\leq n}\lambda_{i_1}\times\dots\times\lambda_{i_k}=\paren{-1}^k\dfrac{a_{n-k}}{a_n}.\]

En particulier, la somme et le produit des racines de \(P\), comptées avec multiplicité, se lisent facilement sur les coefficients de \(P\) : \[\lambda_1+\dots+\lambda_n=\dfrac{-a_{n-1}}{a_n}\qquad\text{et}\qquad\lambda_1\dots\lambda_n=\paren{-1}^n\dfrac{a_0}{a_n}.\]
\end{prop}

\begin{dem}[Idée]
On développe : \[P=\mu\paren{X-\lambda_1}\dots\paren{X-\lambda_n}=\mu X^n-\mu\sigma_1 X^{n-1}+\mu\sigma_2 X^{n-2}+\dots+\paren{-1}^n\mu\sigma_n.\]

D'où \(P=\sum_{k=0}^n\paren{-1}^k\mu\sigma_k X^{n-k}\).

Or \(P=\sum_{k=0}^na_{n-k}X^{n-k}\).

Donc \(\quantifs{\forall k\in\interventierii{1}{n}}\paren{-1}^k\mu\sigma_k=a_{n-k}\).

Donc \(\quantifs{\forall k\in\interventierii{1}{n}}\sigma_k=\paren{-1}^k\dfrac{a_{n-k}}{a_n}\) (car \(a_n=\mu\)).

On en déduit \[\sum_{k=1}^n\lambda_k=\dfrac{-a_{n-1}}{a_n}\qquad\text{et}\qquad\prod_{k=1}^n\lambda_k=\paren{-1}^n\dfrac{a_0}{a_n}.\]
\end{dem}

\begin{exo}
On pose \(P=X^3+3X^2+1\).

\begin{enumerate}
\item Montrer que toutes les racines complexes de \(P\) sont simples. \\

\item Calculer la somme des carrés des racines complexes de \(P\).
\end{enumerate}
\end{exo}

\begin{corr}[1]
On a \(P\prim=3X^2+6X=X\paren{3X+6}\).

Donc \(P\prim\) a pour racines \(0\) et \(-2\).

Or \(P\paren{0}=1\not=0\) et \(P\paren{-2}=5\not=0\).

Donc \(P\prim\) et \(P\) n'admettent aucune racine commune.

Donc \(P\) n'admet aucune racine double.

Donc toutes les racines de \(P\) sont simples.
\end{corr}

\begin{corr}[2]
Notons \(x,y,z\in\C\) les racines de \(P\).

On a \(\paren{x+y+z}^2=x^2+y^2+z^2+2xy+2xz+2yz\).

Donc \[\begin{aligned}
x^2+y^2+z^2&=\paren{x+y+z}^2-2\paren{xy+xz+yz} \\
&=\sigma_1^2-2\sigma_2 \\
&=\paren{-3}^2-2\times0 \\
&=9.
\end{aligned}\]
\end{corr}

\section{Fractions rationnelles}

\subsection{Corps des fractions rationnelles}

Le corps \(\fracrat\) se construit à partir de l'anneau intègre \(\poly\) exactement comme le corps \(\Q\) se construit à partir de l'anneau intègre \(\Z\) (plus généralement, on peut associer à tout anneau intègre sont \guillemets{corps des fractions}).

Cette construction n'est pas difficile, mais elle est hors-programme (et n'apporte rien en pratique). On se contente donc de la définition suivante, qui décrit ce qu'il faut savoir en admettant l'existence du corps \(\fracrat\).

\begin{defi}[Fractions rationnelles]
Le corps des fractions rationnelles en l'indéterminée \(X\) à coefficients dans \(\K\) est un corps noté \(\corps{\fracrat}\) tel que :

\begin{enumerate}
\item L'anneau \(\poly\) est un sous-anneau de \(\corps{\fracrat}\). \\

\item Tout élément de \(\fracrat\) est le quotient de deux éléments de \(\poly\) : \[\quantifs{\forall F\in\fracrat;\exists A\in\poly;\exists B\in\poly\excluant\accol{0}}F=\dfrac{A}{B}.\]
\end{enumerate}

À retenir :

\begin{itemize}
\item Les fractions rationnelles sont les quotients \(\dfrac{A}{B}\) où \(A\in\poly\) et \(B\in\poly\excluant\accol{0}\). \\

\item Savoir reconnaître deux écritures de la même fraction rationnelle : \[\quantifs{\forall A,C\in\poly;\forall B,D\in\poly\excluant\accol{0}}\dfrac{A}{B}=\dfrac{C}{D}\ssi AD=BC.\]

\item Tout polynôme \(A\in\poly\) est une fraction rationnelle : \(A=\dfrac{A}{1}\). \\

\item Connaître enfin la structure de corps \(\corps{\fracrat}\) : \\

On considère des polynômes \(A,C\in\poly\) et \(B,D\in\poly\excluant\accol{0}\). \\

On a la somme et le produit : \[\dfrac{A}{B}+\dfrac{C}{D}=\dfrac{AD+BC}{BD}\qquad\text{et}\qquad\dfrac{A}{B}\times\dfrac{C}{D}=\dfrac{AC}{BD}.\]

L'élément neutre de la somme est la fraction rationnelle nulle, qui est aussi le polynôme nul : \(0=\dfrac{0}{1}\). \\

L'élément neutre du produit est le polynôme constant \(1=\dfrac{1}{1}\). \\

L'opposé de \(\dfrac{A}{B}\) est \(\dfrac{-A}{B}\). \\

Si \(A\not=0\), l'inverse de \(\dfrac{A}{B}\) est \(\dfrac{B}{A}\).
\end{itemize}
\end{defi}

\begin{defprop}[Forme irréductible d'une fraction rationnelle]
Soit \(F\in\fracrat\).

Il existe un unique couple \(\paren{A,B}\in\poly\times\paren{\poly\excluant\accol{0}}\) tel que : \[F=\dfrac{A}{B}\qquad\text{et}\qquad A\et B=1\qquad\text{et}\qquad B\text{ est unitaire}.\]

On appelle forme irréductible de \(F\) toute écriture de \(F\) sous la forme \[F=\dfrac{A}{B}\qquad\text{avec }A\et B=1.\]
\end{defprop}

\begin{dem}
\note{Exercice} (\cf \thref{dem:formeIrreductibleD'unRationnel}).
\end{dem}

\begin{nota}
\renewcommand{\L}{\mathbb{L}}
Soit \(\L\) un corps tel que \(\K\) soit un sous-anneau de \(\L\).

Soient \(F\in\fracrat\) et \(x\in\L\).

On considère une forme irréductible de \(F\) : \[F=\dfrac{A}{B}\qquad\text{avec }\begin{dcases}A\in\poly \\ B\in\poly\excluant\accol{0} \\ A\et B=1\end{dcases}\]

On a déjà défini les éléments \(A\paren{x}\in\L\) et \(B\paren{x}\in\L\) (\cf \thref{nota:évaluationD'unPolynômeEnUnElément}).

Si \(B\paren{x}\not=0\) alors on note \(F\paren{x}\) l'élément de \(\L\) : \[F\paren{x}=\dfrac{A\paren{x}}{B\paren{x}}.\]

On dit que \(F\paren{x}\) est l'élément obtenu en évaluant \(F\) en \(x\).
\end{nota}

\begin{ex}
On garde les notations précédentes.

Si \(\L=\K\) alors la notation \(F\paren{\lambda}\) est valide pour tout \(\lambda\in\K\) qui n'est pas racine de \(B\).

Si \(\L=\fracrat\) alors la notation \(F\paren{G}\) est valide pour tout \(G\in\fracrat\) sauf les polynômes constants qui sont racines de \(B\).

On dit que \(F\paren{G}\) est la composition des fractions rationnelles \(F\) et \(G\) (qu'on note parfois \(F\rond G\)).

En particulier, on a le droit d'écrire \(F=F\paren{X}\).
\end{ex}

\begin{defi}[Conjugaison]~\\
Soient \(F=\dfrac{A}{B}\in\fracrat[\C]\) (avec \(A\in\poly[\C]\) et \(B\in\poly[\C]\excluant\accol{0}\)).

On appelle conjuguée de \(F\) la fraction rationnelle \[\conj{F}=\dfrac{\conj{A}}{\conj{B}}.\]

Elle ne dépend pas du choix de \(A\) et \(B\).
\end{defi}

\begin{prop}
La conjugaison \[\fonctionlambda{\fracrat[\C]}{\fracrat[\C]}{F}{\conj{F}}\] est un automorphisme d'anneau.
\end{prop}

\begin{dem}
\note{Exercice}
\end{dem}

\subsection{Degré}

\begin{defi}[Degré d'une fraction rationnelle]
Soit \(F=\dfrac{A}{B}\in\fracrat\) (avec \(A\in\poly\) et \(B\in\poly\excluant\accol{0}\)).

Le degré de \(F\) est : \[\deg F=\deg A-\deg B.\]

Le degré de \(F\) ne dépend pas du choix de \(A\) et \(B\), et appartient à \(\Z\union\accol{\minf}\).

Si \(F\) est un polynôme, son degré comme fraction rationnelle coïncide avec son degré comme polynôme.
\end{defi}

\begin{prop}
Soient \(F,G\in\fracrat\).

On a :

\begin{enumerate}
\item \(\deg\paren{F+G}\leq\max\accol{\deg F;\deg G}\) avec égalité si \(\deg F\not=\deg G\) \\

\item \(\deg FG=\deg F+\deg G\) \\

\item \(\deg\dfrac{1}{F}=-\deg F\).
\end{enumerate}
\end{prop}

\begin{dem}[Globale]
Soient \(A,C\in\poly\) et \(B,D\in\poly\excluant\accol{0}\) tels que \(F=\dfrac{A}{B}\) et \(G=\dfrac{C}{D}\).
\end{dem}

\begin{dem}[1]
On a : \[\begin{aligned}
\deg\paren{F+G}&=\deg\paren{\dfrac{A}{B}+\dfrac{C}{D}} \\
&=\deg\dfrac{AD+BC}{BD} \\
&=\deg\paren{AD+BC}-\deg BD \\
&\leq\max\accol{\deg AD;\deg BC}-\deg BD\qquad\text{(*)} \\
&=\max\accol{\deg AD-\deg BD;\deg BC-\deg BD} \\
&=\max\accol{\deg A-\deg B;\deg C-\deg D} \\
&=\max\accol{\deg F;\deg G}.
\end{aligned}\]

On a : \[\begin{aligned}
\text{(*) est une égalité}&\impr\deg AD\not=\deg BC \\
&\ssi\deg A+\deg D\not=\deg B+\deg C \\
&\ssi\deg\dfrac{A}{B}\not=\deg\dfrac{C}{D} \\
&\ssi\deg F\not=\deg G.
\end{aligned}\]
\end{dem}

\begin{dem}[2]
On a : \[\begin{aligned}
\deg FG&=\deg\dfrac{AC}{BD} \\
&=\deg AC-\deg BD \\
&=\deg A+\deg C-\deg B-\deg D \\
&=\deg\dfrac{A}{B}+\deg\dfrac{C}{D} \\
&=\deg F+\deg G.
\end{aligned}\]
\end{dem}

\begin{dem}[3]
On a : \[\begin{aligned}
\deg\dfrac{1}{F}&=\deg\dfrac{B}{A} \\
&=\deg B-\deg A \\
&=-\deg F.
\end{aligned}\]
\end{dem}

\begin{defprop}[Partie entière d'une fraction rationnelle]
Soit \(F\in\fracrat\).

Il existe un unique couple \(\paren{E,G}\in\poly\times\fracrat\) tel que \(\begin{dcases}F=E+G \\ \deg G<0\end{dcases}\)

Le polynôme \(E\) est appelé la partie entière de \(F\).
\end{defprop}

\begin{dem}
\existence

Soient \(A\in\poly\) et \(B\in\poly\excluant\accol{0}\) tels que \(F=\dfrac{A}{B}\).

Soient \(Q,R\in\poly\) le quotient et le reste de la division euclidienne de \(A\) par \(B\) : \(\begin{dcases}A=QB+R \\ \deg R<\deg B\end{dcases}\)

On a : \[\begin{aligned}
F&=\dfrac{A}{B} \\
&=\dfrac{QB+R}{B} \\
&=\underbrace{Q}_{\in\poly}+\underbrace{\dfrac{R}{B}}_{\substack{\in\fracrat \\ \deg<0}}
\end{aligned}\]

\unicite

Soient \(E_1,E_2\in\poly\) et \(G_1,G_2\in\fracrat\) tels que \(\begin{dcases}F=E_1+G_1=E_2+G_2 \\ \deg G_1<0 \\ \deg G_2<0\end{dcases}\)

On a \(E_1-E_2=G_2-G_1\).

Donc \(\underbrace{\deg\paren{E_1-E_2}}_{\in\N\union\accol{\minf}}=\underbrace{\deg\paren{G_2-G_1}}_{<0}\).

Donc \(\deg\paren{E_1-E_2}=\minf\).

Donc \(E_1=E_2\).

Donc \(G_1=G_2\).
\end{dem}

\begin{rem}
Soient \(A\in\poly\) et \(B\in\poly\excluant\accol{0}\).

La partie entière de la fraction rationnelle \(\dfrac{A}{B}\) est :

\begin{itemize}
\item nulle si, et seulement si, \(\deg A<\deg B\) ; \\

\item de degré \(\deg A-\deg B\) sinon.
\end{itemize}
\end{rem}

\begin{exo}
Donner la partie entière de \(F=\dfrac{X+1}{X-1}\) et de \(G=\dfrac{X^8+7}{X^4+X^2-1}\).
\end{exo}

\begin{corr}
On a \[\begin{aligned}
F&=\dfrac{X+1}{X-1} \\
&=\dfrac{X-1+2}{X-1} \\
&=\underbrace{1}_{\in\poly[\R]}+\underbrace{\dfrac{2}{X-1}}_{\deg<0}
\end{aligned}\]

Donc la partie entière de \(F\) est \(1\).

On calcule la division euclidienne de \(X^8+7\) par \(X^4+X^2-1\) : \[\polylongdiv[style=D]{X^8+7}{X^4+X^2-1}\]

Donc \(X^8+7=\paren{X^4+X^2-1}\paren{X^4-X^2+2}-3X^2+9\).

Donc \(G=X^4-X^2+2+\underbrace{\dfrac{-3X^2+9}{X^4+X^2-1}}_{\deg<0}\).

Donc la partie entière de \(G\) est \(X^4-X^2+2\).
\end{corr}

\subsection{Racines, pôles}

\begin{defi}[Racines, pôles]
Soit \(F\in\fracrat\).

On considère une forme irréductible de \(F\) : \[F=\dfrac{A}{B}\qquad\text{avec }\begin{dcases}A\in\poly \\ B\in\poly\excluant\accol{0} \\ A\et B=1\end{dcases}\]

On appelle racines de \(F\) les racines de \(A\) et pôles de \(F\) les racines de \(B\).

De plus, la multiplicité d'une racine de \(F\) est sa multiplicité comme racine de \(A\) et la multiplicité d'un pôle de \(F\) est sa multiplicité comme racine de \(B\).
\end{defi}

\begin{ex}~\\
Si \(F=\dfrac{\paren{X-1}\paren{X+2}^3}{\paren{X-4}\paren{X-5}^6}\) alors \(\begin{dcases}1\text{ est racine simple de }F \\ -2\text{ est racine triple de }F \\ 4\text{ est pôle simple de }F \\ 5\text{ est pôle sextuple de }F\end{dcases}\)
\end{ex}

\begin{prop}
Soit \(F\in\fracrat\).

Alors \(F\) admet une infinité de racines si, et seulement si, \(F\) est nulle.
\end{prop}

\begin{dem}
Soient \(A,B\in\poly\) tels que \(\begin{dcases}F=\dfrac{A}{B} \\ A\et B=1\end{dcases}\)

Les racines de \(F\) sont les racines de \(A\).

Donc on a : \[\begin{aligned}
F\text{ admet une infinité de racines}&\ssi A\text{ admet une infinité de racines} \\
&\ssi A=0 \\
&\ssi F=0.
\end{aligned}\]
\end{dem}

\begin{rem}
Soit \(F\in\fracrat\).

\(F\) admet un nombre fini de pôles.
\end{rem}

\begin{defi}[Fonction rationnelle]
Soit \(F\in\fracrat\).

On note \(Z\) l'ensemble des pôles de \(F\) (c'est une partie finie de \(\K\)).

La fonction rationnelle associée à \(F\) est la fonction \[\fonction{\tilde{F}}{\K\excluant Z}{\K}{x}{F\paren{x}}\]
\end{defi}

\subsection{Dérivation}

\begin{defprop}[Dérivée d'une fraction rationnelle]~\\
Soient \(F\in\fracrat\) et \(A\in\poly\) et \(B\in\poly\excluant\accol{0}\) tels que \(F=\dfrac{A}{B}\).

La fraction rationnelle \[\dfrac{A\prim B-AB\prim}{B^2}\] ne dépend pas du choix de \(A\) et \(B\).

On l'appelle la fraction rationnelle dérivée de \(F\) et on la note \(F\prim\).
\end{defprop}

\begin{rem}[Lien entre dérivation des fractions rationnelles et des fonctions]
Soit \(F\in\fracrat[\R]\).

La fonction rationnelle associée à \(F\prim\) est la dérivée de la fonction rationnelle associée à \(F\) : \[\widetilde{F\prim}=\paren{\tilde{F}}\prim.\]

En particulier, toute fonction rationnelle est de classe \(\classe{\infty}\).
\end{rem}

\begin{prop}[Opérations algébriques sur les dérivées]
Soient \(F,G\in\fracrat\) et \(\lambda,\mu\in\K\).

On a la somme : \[\paren{F+G}\prim=F\prim+G\prim.\]

On a, plus généralement : \[\paren{\lambda F+\mu G}\prim=\lambda F\prim+\mu G\prim.\]

On a le produit : \[\paren{FG}\prim=F\prim G+FG\prim.\]

On a le quotient : \[\paren{\dfrac{F}{G}}\prim=\dfrac{F\prim G-FG\prim}{G^2}.\]

On a la composition : \[\paren{F\rond G}\prim=G\prim\times\paren{F\prim\rond G}.\]
\end{prop}

\begin{dem}
\note{Exercice}
\end{dem}

\subsection{Décomposition en éléments simples}

\subsubsection{Le théorème}

\begin{defi}[Élément simple]
On appelle élément simple (sur \(\K\)) toute fraction rationnelle \(F\) de la forme \[F=\dfrac{P}{Q^{\alpha}}\qquad\text{avec }\begin{dcases}P,Q\in\poly\excluant\accol{0} \\ \alpha\in\Ns \\ Q\text{ irréductible (sur \(\K\))} \\ \deg P<\deg Q\end{dcases}\]
\end{defi}

\begin{ex}
Éléments simples sur \(\R\) et sur \(\C\) : \[\dfrac{4}{X+1}\qquad\dfrac{5}{\paren{X+1}^3}\qquad\dfrac{1}{X^5}\]

Éléments simples sur \(\R\) mais pas sur \(\C\) : \[\dfrac{2X+1}{X^2+1}\qquad\dfrac{X}{\paren{X^2+1}^3}\]
\end{ex}

\begin{rem}
Le principal résultat de ce paragraphe affirme que toute fraction rationnelle s'écrit de façon unique comme la somme d'un polynôme et d'éléments simples.

Il faut savoir écrire le théorème sur \(\C\) et sur \(\R\).

Voici d'abord la forme générale du théorème (hors-programme, mais elle peut aider à mieux retenir les deux formes au programme) :
\end{rem}

\begin{theo}[Décomposition en éléments simples sur \(\K\)]\thlabel{theo:DESsurK}
Soit \(F\in\fracrat\).

Soient \(A\in\poly\) et \(B\in\poly\excluant\accol{0}\) tels que : \[F=\dfrac{A}{B}\qquad\text{et}\qquad A\et B=1\qquad\text{et}\qquad B\text{ est unitaire}.\]

Considérons la décomposition de \(B\) en produit de polynômes irréductibles : soient les polynômes irréductibles unitaires deux à deux distincts \(Q_1,\dots,Q_r\in\poly\) (avec \(r\in\N\)) et les entiers \(\alpha_1,\dots,\alpha_r\in\Ns\) tels que : \[B=Q_1^{\alpha_1}\dots Q_r^{\alpha_r}.\]

Alors il existe un polynôme \(E\in\poly\) et un polynôme \(P_{i\gamma}\) pour tout couple d'entiers \(\paren{i,\gamma}\) tel que \(1\leq i\leq r\) et \(1\leq\gamma\leq\alpha_i\) tels que : \[\dfrac{A}{B}=E+\sum_{i=1}^r\sum_{\gamma=1}^{\alpha_i}\dfrac{P_{i\gamma}}{Q_i^\gamma}\qquad\text{et}\qquad\quantifs{\forall i\in\interventierii{1}{r};\forall\gamma\in\interventierii{1}{\alpha_i}}\deg P_{i\gamma}<\deg Q_i.\]

Ces polynômes \(E\) et \(P_{i\gamma}\) sont uniques.

Le polynôme \(E\) est la partie entière de \(F\).
\end{theo}

\begin{dem}
\note{Admis} (hors-programme).
\end{dem}

\begin{theo}[Décomposition en éléments simples sur \(\C\)]
Soit \(F\in\fracrat[\C]\).

Soient \(A\in\poly[\C]\) et \(B\in\poly[\C]\excluant\accol{0}\) tels que \[F=\dfrac{A}{B}\qquad\text{et}\qquad A\et B=1\qquad\text{et}\qquad B\text{ est unitaire}.\]

Considérons la décomposition de \(B\) en produit de polynômes irréductibles sur \(\C\) : \[B=\paren{X-\lambda_1}^{\alpha_1}\dots\paren{X-\lambda_r}^{\alpha_r}\qquad\text{avec }\begin{dcases}r\in\N \\ \lambda_1,\dots,\lambda_r\in\C\text{ deux à deux distincts} \\ \alpha_1,\dots,\alpha_r\in\Ns\end{dcases}\]

Alors il existe un polynôme \(E\in\poly[\C]\) et un nombre complexe \(x_{i\gamma}\) pour tout couple d'entiers \(\paren{i,\gamma}\) tel que \(1\leq i\leq r\) et \(1\leq\gamma\leq\alpha_i\) tels que : \[\dfrac{A}{B}=E+\sum_{i=1}^r\sum_{\gamma=1}^{\alpha_i}\dfrac{x_{i\gamma}}{\paren{X-\lambda_i}^\gamma}.\]

Ce polynôme \(E\) et ces nombres complexes \(x_{i\gamma}\) sont uniques.

Le polynôme \(E\) est la partie entière de \(F\).
\end{theo}

\begin{dem}
\note{Admis} (hors-programme ; découle du \thref{theo:DESsurK}).
\end{dem}

\begin{theo}[Décomposition en éléments simples sur \(\R\)]
Soit \(F\in\fracrat[\R]\).

Soient \(A\in\poly[\R]\) et \(B\in\poly[\R]\excluant\accol{0}\) tels que \[F=\dfrac{A}{B}\qquad\text{et}\qquad A\et B=1\qquad\text{et}\qquad B\text{ est unitaire}.\]

Considérons la décomposition de \(B\) en produit de polynômes irréductibles sur \(\R\) : \[B=\paren{X-\lambda_1}^{\alpha_1}\dots\paren{X-\lambda_r}^{\alpha_r}\times\paren{X^2+b_1X+c_1}^{\beta_1}\dots\paren{X^2+b_sX+c_s}^{\beta_s}\] avec \(\begin{dcases}r,s\in\N \\ \lambda_1,\dots,\lambda_r\in\R\text{ deux à deux distincts} \\ \paren{b_1,c_1},\dots,\paren{b_s,c_s}\in\R^2\text{ deux à deux distincts} \\ \text{les facteurs de degré 2 sont irréductibles, \cad}\quantifs{\forall i\in\interventierii{1}{s}}b_i^2-4c_i<0 \\ \alpha_1,\dots,\alpha_r,\beta_1,\dots,\beta_s\in\Ns\end{dcases}\)

Alors il existe :

\begin{itemize}
\item un polynôme \(E\in\poly[\R]\) ; \\

\item un réel \(x_{i\gamma}\) pour tout couple d'entiers \(\paren{i,\gamma}\) tel que \(1\leq i\leq r\) et \(1\leq\gamma\leq\alpha_i\) ; \\

\item deux réels \(y_{i\gamma},z_{i\gamma}\) pour tout couple d'entiers \(\paren{i,\gamma}\) tel que \(1\leq i\leq s\) et \(1\leq\gamma\leq\beta_i\)
\end{itemize}

tels que : \[\dfrac{A}{B}=E+\sum_{i=1}^r\sum_{\gamma=1}^{\alpha_i}\dfrac{x_{i\gamma}}{\paren{X-\lambda_i}^\gamma}+\sum_{i=1}^s\sum_{\gamma=1}^{\beta_i}\dfrac{y_{i\gamma}X+z_{i\gamma}}{\paren{X^2+b_iX+c_i}^\gamma}.\]

Ce polynôme \(E\) et ces réels \(x_{i\gamma},y_{i\gamma},z_{i\gamma}\) sont uniques.

Le polynôme \(E\) est la partie entière de \(F\).
\end{theo}

\begin{dem}
\note{Admis} (hors-programme ; découle du \thref{theo:DESsurK}).
\end{dem}

\begin{defi}[Partie polaire associée au pôle \(\lambda\)]
Soient \(F\in\fracrat\) et \(\lambda\in\K\) un pôle de \(F\) dont on note \(\alpha\) la multiplicité.

On appelle partie polaire de \(F\) associée au pôle \(\lambda\) la somme des éléments simples de pôle \(\lambda\) dans la décomposition en éléments simples de \(F\).

Elle est de la forme \[\sum_{\gamma=1}^{\alpha}\dfrac{x_\gamma}{\paren{X-\lambda}^\gamma}\qquad\text{avec }x_1,\dots,x_\alpha\in\K\] et \(\lambda\) n'est pas un pôle de \[F-\sum_{\gamma=1}^{\alpha}\dfrac{x_\gamma}{\paren{X-\lambda}^\gamma}.\]
\end{defi}

\begin{ex}
On a les décompositions en éléments simples suivantes :

\begin{itemize}
\item Sur \(\K=\R\) ou \(\C\) : \[\dfrac{X^8+X+1}{X^4\paren{X-1}^3}=X+3+\underbrace{\dfrac{22}{X-1}-\dfrac{3}{\paren{X-1}^2}+\dfrac{3}{\paren{X-1}^3}}_{\text{partie polaire associée à }1}\underbrace{-\dfrac{16}{X}-\dfrac{9}{X^2}-\dfrac{4}{X^3}-\dfrac{1}{X^4}}_{\text{partie polaire associée à }0}\]

\item Sur \(\K=\R\) : \[\dfrac{4}{X^4-1}=\dfrac{1}{X-1}-\dfrac{1}{X+1}-\dfrac{2}{X^2+1}\]

\item Sur \(\K=\C\) : \[\dfrac{4}{X^4-1}=\dfrac{1}{X-1}-\dfrac{1}{X+1}+\dfrac{\i}{X-\i}-\dfrac{\i}{X+\i}\]
\end{itemize}
\end{ex}

\begin{rem}
Il est simple de calculer des décompositions en éléments simples avec Python en utilisant le module \verb|sympy|.

Par exemple :

\begin{verbatim}
>>> from sympy import apart, pprint
>>> from sympy.abc import x
>>> apart(1 / (x ** 2 + x))
-1/(x + 1) + 1/x
>>> pprint(apart(1 / (x ** 2 + x)))
    1     1
- ----- + -
  x + 1   x
\end{verbatim}
\end{rem}

\begin{rem}
On verra que pour primitiver une fonction rationnelle réelle, on décompose la fraction rationnelle correspondante en éléments simples sur \(\R\).

On se ramène ainsi, via des changements de variables affines, à primitiver par rapport à \(t\) des expressions de la forme \[\dfrac{1}{t-\lambda}\qquad\dfrac{1}{\paren{t-\lambda}^{\alpha}}\qquad\dfrac{t}{1+t^2}\qquad\dfrac{t}{\paren{1+t^2}^{\alpha}}\qquad\dfrac{1}{1+t^2}\qquad\dfrac{1}{\paren{1+t^2}^{\alpha}}\] avec \(\lambda\in\R\) et \(\alpha\in\interventierie{2}{\pinf}\).
\end{rem}

\begin{rem}
La décomposition en éléments simples suivantes est à retenir. On l'énonce sous deux formes équivalentes :
\end{rem}

\begin{prop}[Décomposition en éléments simples de \(\dfrac{P\prim}{P}\) quand \(P\) est scindé]
Soit \(P\in\poly\excluant\accol{0}\) scindé sur \(\K\).

On a \[\dfrac{P\prim}{P}=\sum_{k=1}^n\dfrac{1}{X-\lambda_k}\] où \(\lambda_1,\dots,\lambda_n\in\K\) sont les racines de \(P\) comptées avec multiplicité.
\end{prop}

\begin{prop}[Décomposition en éléments simples de \(\dfrac{P\prim}{P}\) quand \(P\) est scindé]
Soit \(P\in\poly\excluant\accol{0}\) scindé sur \(\K\).

On a \[\dfrac{P\prim}{P}=\sum_{j=1}^r\dfrac{\alpha_j}{X-\mu_j}\] où \(\mu_1,\dots,\mu_r\in\K\) sont les racines de \(P\) comptées sans multiplicité et \(\alpha_1,\dots,\alpha_r\in\Ns\) leurs multiplicités respectives.
\end{prop}

\begin{dem}
On a, en notant \(\mu\) le coefficient dominant de \(P\) : \[P=\mu\paren{X-\lambda_1}\dots\paren{X-\lambda_n}.\]

Donc \[\begin{aligned}
\dfrac{P\prim}{P}&=\dfrac{\mu\ds\sum_{k=1}^n\prod_{j\in\interventierii{1}{n}\excluant\accol{k}}\paren{X-\lambda_j}}{\mu\ds\prod_{j=1}^n\paren{X-\lambda_j}} \\
&=\sum_{k=1}^n\dfrac{1}{X-\lambda_k}.
\end{aligned}\]
\end{dem}

\begin{ex}
Si \(P=5\paren{X-7}^3\paren{X-9}\) alors \[\dfrac{P\prim}{P}=\dfrac{3}{X-7}+\dfrac{1}{X-9}.\]
\end{ex}

\subsubsection{Quelques méthodes de calcul}

\begin{rem}
Les méthodes qui suivent servent à obtenir la décomposition en éléments simples d'une fraction rationnelle de degré strictement négatif.
\end{rem}

\begin{meth}[Fonctionne à tous les coups, mais à éviter car très calculatoire]
Mettre au même dénominateur puis identifier.

Cette méthode peut aussi être utilisée après les autres pour trouver les coefficients qui résistent.
\end{meth}

\begin{ex}
On pose \(F=\dfrac{X}{X^2-1}\).

On a \(X^2-1=\paren{X-1}\paren{X+1}\).

La partie entière de \(F\) est nulle car \(\deg F<0\).

On a \(F=\dfrac{a}{X-1}+\dfrac{b}{X+1}\) avec \(a,b\in\R\).

Donc \(F=\dfrac{a\paren{X+1}+b\paren{X-1}}{\paren{X-1}\paren{X+1}}=\dfrac{\paren{a+b}X+a-b}{X^2-1}\).

Donc \(\begin{dcases}a+b=1 \\ a-b=0\end{dcases}\)

Donc \(a=b=\num{0.5}\).

D'où \[F=\dfrac{\num{0.5}}{X+1}+\dfrac{\num{0.5}}{X-1}.\]
\end{ex}

\begin{meth}[Classique]
Si \(\lambda\) est un pôle de multiplicité \(\alpha\) de \(F\) : multiplier par \(\paren{X-\lambda}^{\alpha}\) puis évaluer en \(\alpha\).
\end{meth}

\begin{ex}
On pose \(F=\dfrac{X^4+1}{\paren{X-1}^2\paren{X-2}^3}\).

On a \(F=0+\dfrac{a}{X-1}+\dfrac{b}{\paren{X-1}^2}+\dfrac{c}{X-2}+\dfrac{d}{\paren{X-2}^2}+\dfrac{e}{\paren{X-2}^3}\) avec \(a,b,c,d,e\in\R\).

On multiplie par \(\paren{X-1}^2\) et on évalue en \(1\) ; on obtient \(b=-2\).

On multiplie par \(\paren{X-2}^3\) et on évalue en \(2\) ; on obtient \(e=17\).

On a donc \[F=\dfrac{a}{X-1}-\dfrac{2}{\paren{X-1}^2}+\dfrac{c}{X-2}+\dfrac{d}{\paren{X-2}^2}+\dfrac{17}{\paren{X-2}^3}.\]

De plus, on a \[\quantifs{\forall t\in\R\excluant\accol{1;2}}t\dfrac{t^4+1}{\paren{t-1}^2\paren{t-2}^3}=\dfrac{at}{t-1}-\dfrac{2t}{\paren{t-1}^2}+\dfrac{ct}{t-2}+\dfrac{dt}{\paren{t-2}^2}+\dfrac{17t}{\paren{t-2}^3}.\]

Donc quand \(t\to\pinf\) : \(1=a+c\).

\note{Fin en exercice}
\end{ex}

\begin{meth}[Pour obtenir des informations sur les coefficients]
\begin{itemize}
\item Évaluer la fraction rationnelle en un point. \\

\item Multiplier la fraction rationnelle par une puissance de \(X\) et évaluer sa limite en \(\pinf\).
\end{itemize}
\end{meth}

\begin{meth}[Cas d'une fraction rationnelle réelle]
\begin{itemize}
\item On peut faire sa décomposition en éléments simples sur \(\C\) puis regrouper les éléments simples conjugués pour en déduire la décomposition en éléments simples sur \(\R\). \\

\item Lorsqu'on fait sa décomposition en éléments simples sur \(\C\), on peut remarquer que la partie polaire de pôle \(\lambda\) et la partie polaire de pôle \(\conj{\lambda}\) sont conjuguées (d'après l'unicité de la décomposition en éléments simples).
\end{itemize}
\end{meth}

\begin{ex}
On pose \(F=\dfrac{X+1}{X^2+X+1}\in\fracrat[\R]\).

La partie entière de \(F\) est nulle car \(\deg F<0\).

On a \(X^2+X+1=\paren{X-j}\paren{X-j^2}\).

D'où la décomposition en éléments simples : \[F=\dfrac{a}{X-j}+\dfrac{b}{X-j^2}\qquad\text{avec }a,b\in\C.\]

Or \(F=\conj{F}=\dfrac{\conj{a}}{X-j^2}+\dfrac{\conj{b}}{X-j}\) (car \(j^2=\conj{j}\)).

D'où, par unicité de la décomposition en éléments simples : \(\begin{dcases}a=\conj{b} \\ \conj{a}=b\end{dcases}\)

On multiplie par \(X-j\) et on évalue en \(j\) ; on obtient \[\begin{aligned}
a&=\dfrac{j+1}{j-j^2} \\
&=\dfrac{-j^2}{\i\sqrt{3}} \\
&=\dfrac{\i j^2}{\sqrt{3}} \\
&=\dfrac{\e{\frac{-\i\pi}{6}}}{\sqrt{3}}
\end{aligned}\]

D'où \[F=\dfrac{\frac{\e{\frac{-\i\pi}{6}}}{\sqrt{3}}}{X-j}+\dfrac{\frac{\e{\frac{\i\pi}{6}}}{\sqrt{3}}}{X-j^2}.\]
\end{ex}

\begin{ex}
On pose \(F=\dfrac{1}{X\paren{X^2+1}^2}\).

Déterminons la décomposition en éléments simples de \(F\) sur \(\C\) puis sur \(\R\).

La partie entière de \(F\) est nulle car \(\deg F<0\).

On a \(X\paren{X^2+1}^2=X\paren{X+\i}^2\paren{X-\i}^2\).

D'où la décomposition en éléments simples sur \(\C\) : \[F=\dfrac{a}{X}+\dfrac{b}{X-\i}+\dfrac{c}{\paren{X-\i}^2}+\dfrac{d}{X+\i}+\dfrac{e}{\paren{X+\i}^2}\qquad\text{avec }a,b,c,d,e\in\C.\]

Or \(F=\conj{F}=\dfrac{\conj{a}}{X}+\dfrac{\conj{b}}{X+\i}+\dfrac{\conj{c}}{\paren{X+\i}^2}+\dfrac{\conj{d}}{X-\i}+\dfrac{\conj{e}}{\paren{X-\i}^2}\).

D'où, par une unicité de la décomposition en éléments simples : \(\begin{dcases}a=\conj{a} \\ b=\conj{d} \\ e=\conj{c}\end{dcases}\)

On multiplie par \(X\) et on évalue en \(0\) ; on obtient \(a=1\).

On multiplie par \(\paren{X-\i}^2\) et on évalue en \(\i\) ; on obtient \(c=\dfrac{\i}{4}\).

On multiplie par \(X\) et on prend la limite en \(\pinf\) ; on obtient \(b=\dfrac{-1}{2}\).

Finalement, \(F\) est impaire donc on a \[\begin{aligned}
F&=-F\paren{-X} \\
&=\dfrac{-a}{-X}+\dfrac{-b}{-X-\i}+\dfrac{-c}{\paren{-X-\i}^2}+\dfrac{-d}{-X+\i}+\dfrac{-e}{\paren{-X+\i}^2} \\
&=\dfrac{a}{X}+\dfrac{b}{X+\i}-\dfrac{c}{\paren{X+\i}^2}+\dfrac{d}{X-\i}-\dfrac{e}{\paren{X-\i}^2}.
\end{aligned}\]

D'où, par unicité de la décomposition en éléments simples : \(\begin{dcases}b=d \\ c=-e\end{dcases}\)

D'où \[F=\dfrac{a}{X}+\dfrac{b}{X-\i}+\dfrac{c}{\paren{X-\i}^2}+\dfrac{b}{X+\i}-\dfrac{c}{\paren{X+\i}^2}.\]

D'où la décomposition en éléments simples sur \(\C\) : \[F=\dfrac{1}{X}-\dfrac{1}{2\paren{X-\i}}+\dfrac{\i}{4\paren{X-\i}^2}-\dfrac{1}{2\paren{X+\i}}-\dfrac{\i}{4\paren{X+\i}^2}.\]

On en déduit la décomposition en éléments simples sur \(\R\) : \[\begin{aligned}
F&=\dfrac{1}{X}-\dfrac{1}{2}\paren{\dfrac{1}{X+\i}+\dfrac{1}{X-\i}}+\dfrac{\i}{4}\paren{\dfrac{1}{\paren{X-\i}^2}-\dfrac{1}{\paren{X+\i}^2}} \\
&=\dfrac{1}{X}-\dfrac{1}{2}\times\dfrac{2X}{X^2+1}+\dfrac{\i}{4}\times\dfrac{4\i X}{\paren{X^2+1}^2} \\
&=\dfrac{1}{X}-\dfrac{X}{X^2+1}-\dfrac{X}{\paren{X^2+1}^2}.
\end{aligned}\]
\end{ex}

\begin{rem}[Astuce]
On aurait aussi pu ruser : \[\begin{aligned}
\dfrac{1}{X\paren{X^2+1}^2}&=\dfrac{X^2+1-X^2}{X\paren{X^2+1}^2} \\
&=\dfrac{1}{X\paren{X^2+1}}-\dfrac{X}{\paren{X^2+1}^2} \\
&=\dfrac{X^2+1-X^2}{X\paren{X^2+1}}-\dfrac{X}{\paren{X^2+1}^2} \\
&=\dfrac{1}{X}-\dfrac{X}{X^2+1}-\dfrac{X}{\paren{X^2+1}^2}.
\end{aligned}\]
\end{rem}

\begin{meth}
La proposition suivante :
\end{meth}

\begin{prop}[Cas des pôles simples]\thlabel{prop:pôlesSimplesDES}~\\
Soient \(F=\dfrac{A}{B}\in\fracrat\) une fraction rationnelle écrite sous forme irréductible et \(\lambda\in\K\) un pôle simple de \(F\) (\cad une racine simple de \(B\)).

Alors l'élément simple de pôle \(\lambda\) dans la décomposition en éléments simples de \(F\) est : \[\dfrac{\frac{A\paren{\lambda}}{B\prim\paren{\lambda}}}{X-\lambda}.\]
\end{prop}

\begin{dem}
On sait que la décomposition en éléments simples de \(F\) s'écrit \[\dfrac{A}{B}=\underbrace{\dfrac{\mu}{X-\lambda}}_{\substack{\text{partie polaire} \\ \text{associée à }\lambda}}+\underbrace{G}_{\substack{\text{la somme des autres} \\ \text{parties polaires} \\ \text{et de la partie} \\ \text{entière de }F}}\] où \(\mu\in\K\) et \(G\in\fracrat\) telle que \(\lambda\) n'est pas un pôle de \(G\).

\(\lambda\) est racine simple de \(B\) donc \(X-\lambda\divise B\).

Soit \(Q\in\poly\) tel que \(B=\paren{X-\lambda}Q\).

On a \(B\prim=Q+\paren{X-\lambda}Q\prim\).

Donc \(B\prim\paren{\lambda}=Q\paren{\lambda}+0\).

En multipliant la décomposition en éléments simples par \(X-\lambda\), on a : \[\dfrac{A}{Q}=\mu+\paren{X-\lambda}G.\]

Puis en évaluant en \(\lambda\) : \(\dfrac{A\paren{\lambda}}{Q\paren{\lambda}}=\mu\).

D'où \[\mu=\dfrac{A\paren{\lambda}}{B\prim\paren{\lambda}}.\]
\end{dem}

\begin{exo}
Soit \(n\in\Ns\).

On pose \(F=\dfrac{1}{X^n-1}\).

\begin{enumerate}
\item Donner la décomposition en éléments simples de \(F\) sur \(\C\). \\

\item En déduire la décomposition en éléments simples de \(F\) sur \(\R\).
\end{enumerate}
\end{exo}

\begin{corr}[1]
La partie entière de \(F\) est nulle car \(\deg F<0\).

On a \(X^n-1=\prod_{\omega\in\U_n}\paren{X-\omega}\).

Donc on a la décomposition en éléments simples suivante : \[F=\sum_{\omega\in\U_n}\dfrac{\mu_\omega}{X-\omega}.\]

D'après la \thref{prop:pôlesSimplesDES}, on a \[\quantifs{\forall\omega\in\U_n}\mu_\omega=\dfrac{1}{n\omega^{n-1}}=\dfrac{\omega}{n}\text{ car }\omega^n=1.\]

D'où la décomposition en éléments simples sur \(\C\) : \[F=\sum_{\omega\in\U_n}\dfrac{\omega}{n\paren{X-\omega}}.\]
\end{corr}

\begin{corr}[2]
Si \(\omega=\e{\frac{2\i k\pi}{n}}\) avec \(k\in\interventierii{0}{n-1}\), alors \[\begin{aligned}
\dfrac{\omega}{X-\omega}+\dfrac{\conj{\omega}}{X-\conj{\omega}}&=\dfrac{\omega\paren{X-\conj{\omega}}+\conj{\omega}\paren{X-\omega}}{\paren{X-\omega}\paren{X-\conj{\omega}}} \\
&=\dfrac{2\paren{\Re\omega}X-2}{X^2-2\paren{\Re\omega}X+1} \\
&=\dfrac{2\paren{\cos\frac{2k\pi}{n}}X-2}{X^2-2\paren{\cos\frac{2k\pi}{n}}X+1}.
\end{aligned}\]

Donc, si \(n\) est pair, on a la décomposition en éléments simples sur \(\R\) suivante : \[F=\dfrac{1}{n\paren{X-1}}+\dfrac{-1}{n\paren{X+1}}+\dfrac{2}{n}\sum_{k=1}^{\frac{n}{2}-1}\dfrac{\paren{\cos\frac{2k\pi}{n}}X-1}{X^2-2\paren{\cos\frac{2k\pi}{n}}X+1}.\]

Et si \(n\) est impair, on a la décomposition en éléments simples sur \(\R\) suivante : \[F=\dfrac{1}{n\paren{X-1}}+\dfrac{2}{n}\sum_{k=1}^{\frac{n+1}{2}}\dfrac{\paren{\cos\frac{2k\pi}{n}}X-1}{X^2-2\paren{\cos\frac{2k\pi}{n}}X+1}.\]
\end{corr}

\chapter{Intégrales sur un segment}

\minitoc

Dans ce chapitre, on définit l'intégrale d'une fonction continue par morceaux sur un segment. L'étude des intégrales de fonctions sur un intervalle autre qu'un segment sera faite en deuxième année.

Dans tout le chapitre, on pose \(\K=\R\) ou \(\C\) et on considère un segment \(\intervii{a}{b}\) de \(\R\) (avec \(a,b\in\R\) tels que \(a<b\)).

\section{Définitions}

\subsection{Subdivisions d'un segment}

\begin{defi}
On appelle subdivision de \(\intervii{a}{b}\) toute suite finie strictement croissante \(\sigma=\paren{x_0,x_1,\dots,x_n}\) d'éléments de \(\intervii{a}{b}\) telle que \[a=x_0<x_1<\dots<x_{n-1}<x_n=b.\]

L'application qui à une telle subdivision associe l'ensemble fini \(\accol{x_1;x_2;\dots;x_{n-1}}\) est une bijection de l'ensemble des subdivisions de \(\intervii{a}{b}\) vers l'ensemble des parties finies de \(\intervee{a}{b}\).

En pratique, on s'autorise à voir les subdivisions comme des ensembles finis et à parler, par exemple, de la subdivision réunion de deux subdivisions.
\end{defi}

\begin{defi}
Soient \(\sigma\) et \(\sigma\prim\) deux subdivisions du segment \(\intervii{a}{b}\).

On dit que \(\sigma\prim\) est plus fine que \(\sigma\) si on a \[\sigma\subset\sigma\prim.\]

\Cad, du point de vue des suites finies : \(\sigma\) est une suite extraite de \(\sigma\prim\).
\end{defi}

\begin{rem}
Pour toutes subdivisions \(\sigma_1\) et \(\sigma_2\) de \(\intervii{a}{b}\), il existe une subdivision \(\sigma_3\) plus fine que \(\sigma_1\) et \(\sigma_2\).
\end{rem}

\begin{defi}
Soit \(n\in\Ns\).

On appelle subdivision régulière de pas \(\dfrac{1}{n}\) la subdivision \(a=x_0<x_1<\dots<x_{n-1}<x_n=b\) définie par : \[\quantifs{\forall k\in\interventierii{0}{n}}x_k=a+\dfrac{k}{n}\paren{b-a}.\]
\end{defi}

\subsection{Fonctions en escalier}

\begin{defi}
On dit qu'une fonction \(\phi:\intervii{a}{b}\to\K\) est en escalier s'il existe une subdivision \(a=x_0<x_1<\dots<x_{n-1}<x_n=b\) de \(\intervii{a}{b}\) telle que : \[\quantifs{\forall j\in\interventierii{0}{n-1}}\restr{\phi}{\intervee{x_j}{x_{j+1}}}\text{ est constante},\] \cad si : \[\quantifs{\forall j\in\interventierii{0}{n-1};\forall t\in\intervee{x_j}{x_{j+1}}}\phi\paren{t}=\phi\paren{\dfrac{x_j+x_{j+1}}{2}}.\]

Une telle subdivision est dite adaptée à \(\phi\).
\end{defi}

\begin{ex}
La restriction à tout segment de la fonction \guillemets{partie entière} est en escalier.

La restriction à tout segment de la fonction \guillemets{signum} est en escalier.
\end{ex}

\begin{nota}[Non-officielle]
On notera \(\Esc\) l'ensemble des fonctions en escalier de \(\intervii{a}{b}\) vers \(\K\).

C'est un sous-anneau de \(\anneau{\F{\intervii{a}{b}}{\K}}\).
\end{nota}

\begin{rem}
Soient \(\phi_1,\phi_2\in\F{\intervii{a}{b}}{\R}\).

On pose \(\phi=\phi_1+\i\phi_2\in\F{\intervii{a}{b}}{\C}\).

Alors la fonction \(\phi\) est en escalier si, et seulement si, \(\phi_1\) et \(\phi_2\) sont en escalier.
\end{rem}

\begin{rem}
Soit \(\phi\in\F{\intervii{a}{b}}{\R}\) une fonction en escalier.

Alors l'ensemble image \(\Im\phi\) est fini.

En particulier, la fonction \(\phi\) est bornée.
\end{rem}

\subsection{Fonctions continues par morceaux}

\begin{defi}[Fonction continue par morceaux sur un segment]
On dit qu'une fonction \(f:\intervii{a}{b}\to\K\) est continue par morceaux s'il existe une subdivision \(a=x_0<x_1<\dots<x_{n-1}<x_n=b\) de \(\intervii{a}{b}\) telle que : \[\quantifs{\forall j\in\interventierii{0}{n-1}}\restr{f}{\intervee{x_j}{x_{j+1}}}\text{ admet un prolongement continu à }\intervii{x_j}{x_{j+1}},\] \cad si \[\begin{dcases}\quantifs{\forall j\in\interventierii{0}{n-1}}\restr{f}{\intervee{x_j}{x_{j+1}}}\text{ est continue} \\ \quantifs{\forall j\in\interventierii{0}{n-1}}f\text{ admet une limite finie à droite en }x_j \\ \quantifs{\forall j\in\interventierii{1}{n}}f\text{ admet une limite finie à gauche en }x_j\end{dcases}\]

Une telle subdivision est dite adaptée à \(f\).
\end{defi}

\begin{ex}\thlabel{ex:fonctionsContinuesParMorceauxOuPas}
Toute fonction en escalier est continue par morceaux.

Toute fonction continue est continue par morceaux.

La fonction \(\fonction{f}{\intervii{0}{1}}{\R}{x}{\begin{dcases}\sin\dfrac{1}{x} &\text{si }x\not=0 \\ 0 &\text{si }x=0\end{dcases}}\) n'est pas continue par morceaux (elle est continue sur \(\intervei{0}{1}\) mais n'admet pas de limite finie en \(0^+\)).
\end{ex}

\begin{nota}[Non-officielle]
On notera \(\contm\) l'ensemble des fonctions continues par morceaux de \(\intervii{a}{b}\) vers \(\K\).

C'est un sous-anneau de l'anneau \(\anneau{\F{\intervii{a}{b}}{\K}}\).
\end{nota}

\begin{rem}
Soient \(\phi_1,\phi_2\in\F{\intervii{a}{b}}{\R}\).

On pose \(\phi=\phi_1+\i\phi_2\in\F{\intervii{a}{b}}{\C}\).

Alors la fonction \(\phi\) est continue par morceaux si, et seulement si, \(\phi_1\) et \(\phi_2\) sont continues par morceaux.
\end{rem}

\begin{rem}
Toute fonction continue par morceaux sur un segment est bornée (mais n'atteint pas forcément ses bornes).
\end{rem}

\begin{rem}
Toute fonction continue par morceaux est la somme d'une fonction continue et d'une fonction en escalier.
\end{rem}

\begin{dem}
On raisonne par récurrence sur le nombre (nécessairement fini) de points où la fonction continue par morceaux n'est pas continue.

\note{Exercice} : montrer que l'écriture comme somme d'une fonction continue et d'une fonction en escalier est unique à une constante additive près.
\end{dem}

\subsection{Intégrales}

\subsubsection{Intégrale d'une fonction en escalier}

\begin{defi}
Soient \(\phi\in\Esc\) et \(a=x_0<x_1<\dots<x_{n-1}<x_n=b\) une subdivision adaptée à \(\phi\).

La valeur de la somme \[\sum_{j=0}^{n-1}\paren{x_{j+1}-x_j}\phi\paren{\dfrac{x_j+x_{j+1}}{2}}\] ne dépend pas de la subdivision choisie.

On l'appelle intégrale de \(\phi\) sur \(\intervii{a}{b}\) et on la note : \[\int_{\intervii{a}{b}}\phi\qquad\text{ou}\qquad\int_a^b\phi\paren{t}\odif{t}\qquad\text{ou}\qquad\int_a^b\phi.\]
\end{defi}

\begin{ex}
On a : \[\begin{aligned}
\int_{-1}^2\floor{t}\odif{t}&=\sum_{j=0}^1\paren{x_{j+1}-x_j}\floor{\dfrac{x_j+x_{j+1}}{2}} \\
&=\paren{0+1}\floor{\dfrac{0-1}{2}}+\paren{1-0}\floor{\dfrac{1+0}{2}}+\paren{2-1}\floor{\dfrac{2+1}{2}} \\
&=-1+0+1 \\
&=0
\end{aligned}\]
\end{ex}

\begin{rem}
Les propriétés suivantes (laissées en exercice) sont utiles pour définir l'intégrale d'une fonction continue par morceaux.
\end{rem}

\begin{prop}[Linéarité de l'intégrale]
On a : \[\quantifs{\forall\lambda,\mu\in\K;\forall\phi,\psi\in\Esc}\int_{\intervii{a}{b}}\paren{\lambda\phi+\mu\psi}=\lambda\int_{\intervii{a}{b}}\phi+\mu\int_{\intervii{a}{b}}\psi.\]

On dit que l'application \(\fonctionlambda{\Esc}{\K}{\phi}{\int_{\intervii{a}{b}}\phi}\) est linéaire (cela signifie que l'image d'une combinaison linéaire est la combinaison linéaire des images).
\end{prop}

\begin{prop}[Positivité et croissance de l'intégrale]\thlabel{prop:positivitéEtCroissanceDeLIntégraleDUneFonctionEnEscalier}
Soient \(a,b\in\R\) tels que \(a\leq b\) et \(\phi,\psi\in\Esc\).

On a \[\phi\geq0\imp\int_a^b\phi\paren{t}\odif{t}\geq0\qquad\text{(positivité)}\] et \[\phi\leq\psi\imp\int_a^b\phi\paren{t}\odif{t}\leq\int_a^b\psi\paren{t}\odif{t}\qquad\text{(croissance).}\]
\end{prop}

\subsubsection{Intégrale d'une fonction continue par morceaux}

\begin{deftheo}[Cas d'une fonction à valeurs réelles]
\newcommand{\Em}{\mathcal{E}_-}
\newcommand{\Ep}{\mathcal{E}_+}
Soit \(f\in\contm\).

On note \(\Em\) l'ensemble des fonctions en escalier inférieurs à \(f\) : \[\Em=\accol{\phi\in\Esc\tq\phi\leq f}\] et \(\Ep\) l'ensemble des fonctions en escalier supérieures à \(f\) : \[\Ep=\accol{\psi\in\Esc\tq f\leq\psi}.\]

Les bornes supérieure et inférieure suivantes sont bien définies : \[I_-\paren{f}=\sup_{\phi\in\Em}\int_{\intervii{a}{b}}\phi\qquad\text{et}\qquad I_+\paren{f}=\inf_{\psi\in\Ep}\int_{\intervii{a}{b}}\psi.\]

De plus, elles ont la même valeur, que l'on appelle intégrale de \(f\) sur \(\intervii{a}{b}\) et qu'on note : \[\int_{\intervii{a}{b}}f\qquad\text{ou}\qquad\int_a^bf\paren{t}\odif{t}\qquad\text{ou}\qquad\int_a^bf.\]

En résumé : \[\int_{\intervii{a}{b}}f=I_-\paren{f}=I_+\paren{f}.\]
\end{deftheo}

\begin{dem}~\\
\newcommand{\Em}{\mathcal{E}_-}
\newcommand{\Ep}{\mathcal{E}_+}
L'ensemble \(\accol{\int_{\intervii{a}{b}}\phi}_{\phi\in\Em}\) est une partie non-vide de \(\R\) car \(f\) est continue par morceaux sur son segment, donc bornée, donc minorée par un réel \(m\in\R\) et donc la fonction constante égale à \(m\) appartient à \(\Em\), et majorée car \(f\) est majorée par un réel \(M\in\R\) qui vérifie : \[\quantifs{\forall\phi\in\Em}\phi\leq f\leq M\] et donc, selon la \thref{prop:positivitéEtCroissanceDeLIntégraleDUneFonctionEnEscalier} : \[\quantifs{\forall\phi\in\Em}\int_{\intervii{a}{b}}\phi\leq\int_{\intervii{a}{b}}M=\paren{b-a}M.\] Donc cet ensemble admet une borne supérieure. Donc \(I_-\paren{f}\) est bien définie.

Idem pour \(I_+\paren{f}\).

Montrons que \(I_-\paren{f}=I_+\paren{f}\).

\leqbox

On a \[\quantifs{\forall\phi\in\Em;\forall\psi\in\Ep}\phi\leq f\leq\psi.\]

D'où, selon la \thref{prop:positivitéEtCroissanceDeLIntégraleDUneFonctionEnEscalier} : \[\quantifs{\forall\phi\in\Em;\forall\psi\in\Ep}\int_{\intervii{a}{b}}\phi\leq\int_{\intervii{a}{b}}\psi.\]

Donc, par définition de \(I_+\paren{f}\) : \[\quantifs{\forall\phi\in\Em}\int_{\intervii{a}{b}}\phi\leq I_+\paren{f}.\]

Et donc, par définition de \(I_-\) : \[I_-\paren{f}\leq I_+\paren{f}.\]

\geqbox

Soit \(\epsilon\in\Rps\).

Comme \(f\) est continue par morceaux, il existe \(f_0\in\ensclasse{0}{\intervii{a}{b}}{\R}\) et \(\phi_0\in\Esc\) telles que \(f=f_0+\phi_0\).

Comme \(f_0\) est continue sur un segment, elle est uniformément continue d'après le théorème de Heine.

Il existe donc \(\delta\in\Rps\) tel que \[\quantifs{\forall x,y\in\intervii{a}{b}}\abs{x-y}\leq\delta\imp\abs{f_0\paren{x}-f_0\paren{y}}\leq\epsilon.\]

Soit \(n\in\Ns\) tel que \(\dfrac{b-a}{n}\leq\delta\) (un tel entier existe car \(\lim_n\dfrac{b-a}{n}=0\)).

On définit \(\phi\in\Em\) et \(\psi\in\Ep\) en posant \(\quantifs{\forall k\in\interventierii{0}{n}}x_k=a+k\dfrac{b-a}{n}\) et  \(\phi\paren{b}=\psi\paren{b}=f\paren{b}\) et \[\quantifs{\forall k\in\interventierii{0}{n-1};\forall t\in\intervie{x_k}{x_{k+1}}}\begin{dcases}\phi\paren{t}=\min_{\intervie{x_k}{x_{k+1}}}f_0 \\ \psi\paren{t}=\max_{\intervie{x_k}{x_{k+1}}}f_0\end{dcases}\]

De plus, on a : \[\quantifs{\forall k\in\interventierii{0}{n-1};\forall x,y\in\intervie{x_k}{x_{k+1}}}\abs{x-y}\leq\dfrac{b-a}{n}\leq\delta.\]

Donc, par définition de \(\delta\) : \[\quantifs{\forall k\in\interventierii{0}{n-1}}\abs{\min_{\intervie{x_k}{x_{k+1}}}f_0-\max_{\intervie{x_k}{x_{k+1}}}f_0}\leq\epsilon.\]

D'où \(\quantifs{\forall t\in\intervii{a}{b}}\abs{\phi\paren{t}-\psi\paren{t}}\leq\epsilon\) donc \(\quantifs{\forall t\in\intervii{a}{b}}\psi\paren{t}-\phi\paren{t}\leq\epsilon\).

D'où \(\psi\leq\phi+\epsilon\).

Finalement, on a obtenu : \[\begin{dcases}\phi\leq f_0\leq\psi \\ \psi\leq\phi+\epsilon\end{dcases}\]

D'où, en posant \(\phi_1=\phi+\phi_0\) et \(\psi_1=\psi+\psi_0\) : \[\begin{dcases}\phi_1\leq f\leq\psi_1 \\ \psi_1\leq\phi_1+\epsilon\end{dcases}\]

On a donc \(\phi_1\in\Em\) et \(\psi_1\in\Ep\) donc \[\begin{dcases}\int_a^b\phi_1\leq I_-\paren{f} \\ \int_a^b\psi_1\geq I_+\paren{f}\end{dcases}\]

Or \(\psi_1\leq\phi_1+\epsilon\) donc \[\int_a^b\psi_1\leq\int_a^b\phi_1+\paren{b-a}\epsilon.\]

D'où \[I_+\paren{f}\leq\int_a^b\psi_1\leq\int_a^b\phi_1+\paren{b-a}\epsilon\leq I_-\paren{f}+\paren{b-a}\epsilon.\]

D'où, avec \(\epsilon\to0\) : \[I_+\paren{f}\leq I_-\paren{f}.\]

Donc \(I_-\paren{f}=I_+\paren{f}\).
\end{dem}

\begin{rem}
\newcommand{\Em}{\mathcal{E}_-}
\newcommand{\Ep}{\mathcal{E}_+}
Seule l'intégration des fonctions continues par morceaux est au programme, mais la construction précédente (due au mathématicien Gaston Darboux) s'applique à toute fonction \(f\) bornée sur \(\intervii{a}{b}\) (afin que les ensembles \(\Em\) et \(\Ep\) ne soient pas vides) et telle que \(I_-\paren{f}=I_+\paren{f}\). Une telle fonction est dite Riemann-intégrable ou Darboux-intégrable.

Par exemple, la fonction \(f\) de l'\thref{ex:fonctionsContinuesParMorceauxOuPas} n'est pas continue par morceaux mais est Riemann-intégrable.

Il existe une autre théorie plus récente (1901) et plus satisfaisante pour définir l'intégrale d'une fonction réelle ou complexe, due à Henri Lebesgue (hors-programme).
\end{rem}

\begin{defi}[Cas d'une fonction à valeurs complexes]
Soit \(f\in\contm[\intervii{a}{b}][\C]\).

On note \(f_1,f_2\in\contm[\intervii{a}{b}][\R]\) les fonctions réelles telles que \(f=f_1+\i f_2\).

L'intégrale de \(f\) sur \(\intervii{a}{b}\) est définie par : \[\int_{\intervii{a}{b}}f=\int_{\intervii{a}{b}}f_1+\i\int_{\intervii{a}{b}}f_2.\]
\end{defi}

\begin{rem}[Pour se ramener à des fonctions continues]
Soient \(f\in\contm\) et \(a=x_0<x_1<\dots<x_{n-1}<x_n=b\) une subdivision adaptée à \(f\), \cad telle que pour tout \(j\in\interventierii{0}{n-1}\), la fonction \(\restr{f}{\intervee{x_j}{x_{j+1}}}\) admet un prolongement continu \(g_j:\intervii{x_j}{x_{j+1}}\to\K\).

L'intégrale de \(f\) sur \(\intervii{a}{b}\) est alors \[\int_{\intervii{a}{b}}f=\int_a^bf\paren{t}\odif{t}=\sum_{j=0}^{n-1}\int_{x_j}^{x_{j+1}}g_j\paren{t}\odif{t}.\]
\end{rem}

\begin{nota}[Généralisation]
Soit \(f\in\contm\).

On pose : \[\quantifs{\forall x,y\in\intervii{a}{b}}\int_x^yf\paren{t}\odif{t}=\begin{dcases}\int_{\intervii{x}{y}}f &\text{si }x<y \\ -\int_{\intervii{x}{y}}f &\text{si }x>y \\ 0 &\text{si }x=y\end{dcases}\]
\end{nota}

\section{Propriétés}

\subsection{Premières propriétés}

\begin{prop}[Linéarité de l'intégrale]
On a : \[\quantifs{\forall\lambda,\mu\in\K;\forall f,g\in\contm}\int_{\intervii{a}{b}}\paren{\lambda f+\mu g}=\lambda\int_{\intervii{a}{b}}f+\mu\int_{\intervii{a}{b}}g.\]

On dit que l'application \(\fonctionlambda{\contm}{\K}{f}{\int_{\intervii{a}{b}}f}\) est linéaire.
\end{prop}

\begin{rem}
Soient \(f,g\in\contm\).

Si \(f\) et \(g\) coïncident sauf en un nombre fini de points, alors \[\int_{\intervii{a}{b}}f=\int_{\intervii{a}{b}}g.\]
\end{rem}

\begin{dem}
\(g-f\) est une fonction en escalier nulle partout sauf en un nombre fini de points donc \(\int_a^b\paren{g-f}=0\).

D'où, par linéarité de l'intégrale : \(\int_a^bg-\int_a^bf=0\).
\end{dem}

\begin{prop}[Relation de Chasles]
Soit \(f\in\contm\).

On a : \[\quantifs{\forall x,y,z\in\intervii{a}{b}}\int_x^yf\paren{t}\odif{t}=\int_x^zf\paren{t}\odif{t}+\int_z^yf\paren{t}\odif{t}.\]
\end{prop}

\begin{prop}[Inégalité triangulaire intégrale\protect\footnotemark]
\footnotetext{Dans les programmes précédents (donc dans les annales de concours), cette inégalité était appelée \guillemets{inégalité de la moyenne}.}
On a : \[\quantifs{\forall f\in\contm}\abs{\int_{\intervii{a}{b}}f}\leq\int_{\intervii{a}{b}}\abs{f}.\]
\end{prop}

\begin{prop}[Positivité et croissance de l'intégrale]
Soient \(a,b\in\R\) tels que \(a\leq b\) et \(f,g\in\contm\).

On a \[f\geq0\imp\int_a^bd\paren{t}\odif{t}\geq0\qquad\text{(positivité)}\] et \[f\leq g\imp\int_a^bf\paren{t}\odif{t}\leq\int_a^bg\paren{t}\odif{t}\qquad\text{(croissance)}.\]
\end{prop}

\begin{dem}[Positivité]
\newcommand{\Em}{\mathcal{E}_-}
Supposons \(f\geq0\).

La fonction nulle \(\fonction{\phi_0}{\intervii{a}{b}}{\K}{t}{0}\) appartient à \(\Em\) donc on a \[\int_a^bf=\sup_{\phi\in\Em}\int_a^b\phi\geq\int_a^b\phi_0.\]
\end{dem}

\begin{dem}[Croissance]
Supposons \(f\leq g\).

Alors \(0\leq g-f\).

Donc selon la positivité de l'intégrale, on a \(0\leq\int_a^b\paren{g-f}\).

D'où, par linéarité de l'intégrale : \(0\leq\int_a^bg-\int_a^bf\).
\end{dem}

\begin{prop}
On suppose \(a<b\).

Soit \(f:\intervii{a}{b}\to\R\) continue.

On a \[\croch{f\geq0\quad\text{et}\quad\int_a^bf\paren{t}\odif{t}=0}\imp f=0.\]

NB : l'implication est fausse pour les fonctions continues par morceaux.
\end{prop}

\begin{dem}
On suppose \(f\) continue et positive.

Montrons que \(\int_a^bf\paren{t}\odif{t}=0\imp f=0\) par contraposée.

Supposons \(f\not=0\).

Montrons que \(\int_a^bf\paren{t}\odif{t}\not=0\).

Soit \(x_0\in\intervii{a}{b}\) tel que \(f\paren{x_0}\not=0\).

Supposons \(x_0\not=a\) et \(x_0\not=b\).

On a \(\lim_{x\to x_0}f\paren{x}=f\paren{x_0}\) car \(f\) est continue en \(x_0\) et \(f\paren{x_0}>0\) car \(f\paren{x_0}\not=0\) et \(f\geq0\).

Donc \(\dfrac{f\paren{x_0}}{2}<\lim_{x\to x_0}f\paren{x}\).

Il existe \(\delta\in\Rps\) tel que \[\begin{dcases}\intervee{x_0-\delta}{x_0+\delta}\subset\intervii{a}{b} \\ \quantifs{\forall t\in\intervee{x_0-\delta}{x_0+\delta}}\dfrac{f\paren{x_0}}{2}<f\paren{t}\end{dcases}\] car \(\intervii{a}{b}\) est un voisinage de \(x_0\) et car \(f\) est continue en \(x_0\).

Posons \(\fonction{\phi}{\intervii{a}{b}}{\R}{t}{\begin{dcases}\dfrac{f\paren{x_0}}{2} &\text{si }\abs{t-x_0}\leq\delta \\ 0 &\text{sinon}\end{dcases}}\)

On a \(\phi\leq f\) donc par croissance de l'intégrale, on a \[0<\delta f\paren{x_0}=2\delta\dfrac{f\paren{x_0}}{2}=\int_a^b\phi\leq\int_a^bf.\]

De même si \(x_0=a\) ou \(x_0=b\), on obtient \(\int_a^bf\geq\delta\dfrac{f\paren{x_0}}{2}>0\).
\end{dem}

\subsection{Sommes de Riemann}

\begin{prop}[Convergence des sommes de Riemann]
Soit \(f\in\contm\).

On a : \[\lim_{n\to\pinf}\dfrac{b-a}{n}\sum_{k=1}^{n}f\paren{a+k\dfrac{b-a}{n}}=\lim_{n\to\pinf}\dfrac{b-a}{n}\sum_{k=0}^{n-1}f\paren{a+k\dfrac{b-a}{n}}=\int_a^bf\paren{t}\odif{t}.\]
\end{prop}

\begin{dem}[Cas où \(f\) est une fonction lipschitzienne]~\\
On pose \(\quantifs{\forall k\in\interventierii{0}{n-1}}x_k=a+k\dfrac{b-a}{n}\).

On pose \(\quantifs{\forall n\in\Ns}R_n\paren{f}=\dfrac{b-a}{n}\sum_{k=0}^{n-1}f\paren{x_k}\).

On a : \[\begin{aligned}
\int_a^bf\paren{t}\odif{t}-R_n\paren{f}&=\sum_{k=0}^{n-1}\int_{x_k}^{x_{k+1}}f\paren{t}\odif{t}-\sum_{k=0}^{n-1}\int_{x_k}^{x_{k+1}}f\paren{x_k}\odif{t} \\
&=\sum_{k=0}^{n-1}\int_{x_k}^{x_{k+1}}\croch{f\paren{t}-f\paren{x_k}}\odif{t}.
\end{aligned}\]

Soit \(K\in\Rp\) tel que \(f\) est \(K\)-lipschitzienne.

On a : \[\quantifs{\forall k\in\interventierii{0}{n-1};\forall t\in\intervii{x_k}{x_{k+1}}}\abs{f\paren{t}-f\paren{x_k}}\leq K\abs{t-x_k}.\]

Donc on a : \[\begin{aligned}
\quantifs{\forall k\in\interventierii{0}{n-1}}\abs{\int_{x_k}^{x_{k+1}}\croch{f\paren{t}-f\paren{x_k}}\odif{t}}&\leq\int_{x_k}^{x_{k+1}}\abs{f\paren{t}-f\paren{x_k}}\odif{t}\qquad\text{(inégalité triangulaire intégrale)} \\
&\leq\int_{x_k}^{x_{k+1}}K\paren{t-x_k}\odif{t}\qquad\text{(croissance de l'intégrale)} \\
&=\croch{K\dfrac{\paren{t-x_k}^2}{2}}_{x_k}^{x_{k+1}} \\
&=\dfrac{K}{2}\paren{\dfrac{b-a}{n}}^2
\end{aligned}\]

D'où : \[\begin{aligned}
\abs{\int_a^bf\paren{t}\odif{t}-R_n\paren{f}}&\leq\sum_{k=0}^{n-1}\abs{\int_{x_k}^{x_{k+1}}\croch{f\paren{t}-f\paren{x_k}}\odif{t}} \\
&\leq\sum_{k=0}^{n-1}\dfrac{K}{2}\dfrac{\paren{b-a}^2}{n^2} \\
&=\dfrac{K}{2}\dfrac{\paren{b-a}^2}{n}.
\end{aligned}\]

Or \(\lim_n\dfrac{K}{2}\dfrac{\paren{b-a}^2}{n}=0\).

Donc selon le \hyperref[theo:gendarmes]{théorème des gendarmes}, \(\lim_nR_n\paren{f}=\int_a^bf\paren{t}\odif{t}\).

L'autre limite en découle car on a \[\dfrac{b-a}{n}\sum_{k=1}^{n}f\paren{x_k}=R_n\paren{f}+\dfrac{b-a}{n}\paren{f\paren{b}-f\paren{a}}\quad\text{et}\quad\begin{dcases}\lim_nR_n\paren{f}=\int_a^bf\paren{t}\odif{t} \\ \lim_n\dfrac{b-a}{n}\paren{f\paren{b}-f\paren{a}}=0\end{dcases}\].
\end{dem}

\begin{dem}[Cas où \(f\) est une fonction continue]\thlabel{dem:convergenceDesSommesDeRiemannDansLeCasContinu}
Cf. \thref{exo:convergenceDesSommesDeRiemannDansLeCasContinu}.
\end{dem}

\subsection{Théorème fondamental}

\begin{defi}
On appelle primitive d'une fonction \(f\) toute fonction dérivable \(F\) dont la dérivée est \(f\) : \[F\prim=f.\]
\end{defi}

\begin{prop}[Unicité à une constante additive près]
Soient \(I\) un intervalle de \(\R\) et \(f:I\to\K\).

On suppose que \(F_1,F_2\in\F{I}{\K}\) sont des primitives de \(f\).

Alors \[\quantifs{\exists c\in\K}F_2=F_1+c.\]
\end{prop}

\begin{dem}
On a \(\paren{F_2-F_1}\prim=f-f=0\).

Donc la fonction \(F_2-F_1\) est constante sur l'intervalle \(I\) : \(\quantifs{\exists c\in\K}F_2-F_1=c\).

Donc \(\quantifs{\exists c\in\K}F_2=F_1+c\).
\end{dem}

\begin{theo}[Théorème fondamental de l'analyse]
Soient \(I\) un intervalle de \(\R\), \(f:I\to\K\) continue et \(x_0\in I\).

Alors la fonction \[\fonction{F}{I}{\K}{x}{\int_{x_0}^xf\paren{t}\odif{t}}\] est une primitive de \(f\) sur \(I\) (c'est l'unique primitive de \(f\) qui s'annule en \(x_0\)).
\end{theo}

\begin{dem}
Soit \(x\in I\).

Montrons que \(\lim_{\substack{x\prim\to x \\ x\prim\not=x}}\dfrac{F\paren{x\prim}-F\paren{x}}{x\prim-x}=f\paren{x}\), \cad \[\quantifs{\forall\epsilon\in\Rps;\exists\delta\in\Rps;\forall x\prim\in I}\abs{x\prim-x}\leq\delta\imp\abs{\dfrac{F\paren{x\prim}-F\paren{x}}{x\prim-x}-f\paren{x}}\leq\epsilon.\]

Soit \(\epsilon\in\Rps\).

Soit \(\delta\in\Rps\) tel que \[\quantifs{\forall x\seconde\in I}\abs{x-x\seconde}\leq\delta\imp\abs{f\paren{x}-f\paren{x\seconde}}\leq\epsilon.\]

Un tel \(\delta\) existe car \(f\) est continue en \(x\).

Soit \(x\prim\in I\) tel que \(\abs{x-x\prim}\leq\delta\).

On a, selon la relation de Chasles : \[F\paren{x\prim}-F\paren{x}=\int_x^{x\prim}f\paren{t}\odif{t}.\]

D'où \[\begin{aligned}
\abs{F\paren{x\prim}-F\paren{x}-\paren{x\prim-x}f\paren{x}}&=\abs{\int_x^{x\prim}f\paren{t}\odif{t}-\int_x^{x\prim}f\paren{x}\odif{t}} \\
&=\abs{\int_x^{x\prim}\paren{f\paren{t}-f\paren{x}}\odif{t}} \\
&\leq\abs{\int_x^{x\prim}\abs{f\paren{t}-f\paren{x}}\odif{t}} \\
&\leq\abs{\int_x^{x\prim}\epsilon\odif{t}} \\
&=\epsilon\abs{x\prim-x}.
\end{aligned}\]

D'où, si \(x\not=x\prim\) : \[\abs{\dfrac{F\paren{x\prim}-F\paren{x}}{x\prim-x}}\leq\epsilon.\]

Donc \(\delta\) convient.

D'où \(F\prim\paren{x}=f\paren{x}\).
\end{dem}

\begin{cor}[Existence de primitives]
Toute fonction continue sur un intervalle admet une primitive.
\end{cor}

\begin{exo}
Donner une primitive des fonctions suivantes (à l'aide d'une intégrale) :

\begin{enumerate}
\item \(\Arctan:\R\to\R\) \\

\item \(f:x\mapsto\dfrac{1}{\sqrt{\Arcsin x}}\)
\end{enumerate}
\end{exo}

\begin{corr}[1]
On a la primitive de \(\Arctan\) suivante : \[\fonctionlambda{\R}{\R}{x}{\int_0^x\Arctan t\odif{t}}\]
\end{corr}

\begin{corr}[2]
\(f\) est bien définie sur \(\intervei{0}{1}\).

On a la primitive suivante : \[\fonctionlambda{\intervei{0}{1}}{\R}{x}{\int_1^x\dfrac{\odif{t}}{\sqrt{\Arcsin t}}}\]
\end{corr}

\begin{cor}\thlabel{cor:intégraleEgaleDifférencePrimitive}
Soient \(f:\intervii{a}{b}\to\K\) continue et \(F:\intervii{a}{b}\to\K\) une primitive de \(f\).

Alors \[\int_a^bf\paren{t}\odif{t}=F\paren{b}-F\paren{a}.\]

L'accroissement \(F\paren{b}-F\paren{a}\) est noté : \[\croch{F\paren{t}}_a^b\qquad\text{ou}\qquad\croch{F\paren{t}}_{t=a}^b.\]
\end{cor}

\begin{dem}
Posons \(\fonction{G}{\intervii{a}{b}}{\K}{x}{\int_a^xf\paren{t}\odif{t}}\)

Les fonctions \(F\) et \(G\) sont des primitives de \(f\) sur l'intervalle \(\intervii{a}{b}\).

Il existe donc \(c\in\K\) tel que \(\quantifs{\forall x\in\intervii{a}{b}}G\paren{x}=F\paren{x}+c\).

En particulier, si \(x=a\) : \(0=G\paren{a}=F\paren{a}+c\).

Donc \(c=-F\paren{a}\) et \(\quantifs{\forall x\in\intervii{a}{b}}G\paren{x}=F\paren{x}-F\paren{a}\).

D'où, en prenant \(x=b\) : \[\int_a^bf\paren{t}\odif{t}=G\paren{b}=F\paren{b}-F\paren{a}.\]
\end{dem}

\begin{cor}
Soit \(g\in\ensclasse{1}{\intervii{a}{b}}{\K}\).

On a : \[g\paren{b}-g\paren{a}=\int_a^bg\prim\paren{t}\odif{t}.\]
\end{cor}

\begin{dem}
Découle du \thref{cor:intégraleEgaleDifférencePrimitive} en prenant \(\begin{dcases}f=g\prim \\ F=g\end{dcases}\)
\end{dem}

\subsection{Intégration par parties}

\begin{prop}
Soient \(f,g\in\ensclasse{1}{\intervii{a}{b}}{\K}\).

On a \[\int_a^bf\paren{t}g\prim\paren{t}\odif{t}=\croch{f\paren{t}g\paren{t}}_a^b-\int_a^bf\prim\paren{t}g\paren{t}\odif{t}.\]
\end{prop}

\begin{dem}
La fonction \(fg\) est de classe \(\classe{1}\).

On a donc : \[\begin{aligned}
\croch{f\paren{t}g\paren{t}}_a^b&=\int_a^b\paren{fg}\prim\paren{t}\odif{t} \\
&=\int_a^bf\prim\paren{t}g\paren{t}\odif{t}+\int_a^bg\prim\paren{t}f\paren{t}\odif{t}.
\end{aligned}\]
\end{dem}

\begin{exoex}
\begin{enumerate}
\item Calculer une primitive de \(\ln\) sur \(\Rps\). \\

\item Calculer \(\int_0^1\Arctan t\odif{t}\). \\

\item Calculer une primitive de \(t\mapsto t^2\ln t\) sur \(\Rps\). \\

\item Calculer une primitive de \(t\mapsto t\e{t}\) sur \(\R\). \\

\item Calculer une primitive de \(t\mapsto t^2\cos t\) sur \(\R\).
\end{enumerate}
\end{exoex}

\begin{corr}[1]
On a la primitive \[\begin{aligned}
x\mapsto\int_1^x\ln t\odif{t}&=\croch{t\ln t}_1^x-\int_1^xt\dfrac{1}{t}\odif{t} \\
&=x\ln x-\paren{x-1}
\end{aligned}\]

Donc une autre primitive est \(x\mapsto x\ln x-x\).
\end{corr}

\begin{corr}[2]
On a : \[\begin{aligned}
\int_0^1\Arctan t\odif{t}&=\croch{t\Arctan t}_0^1-\int_0^1\dfrac{t}{1+t^2}\odif{t} \\
&=\dfrac{\pi}{4}-\dfrac{1}{2}\int_0^1\dfrac{2t}{t^2+1}\odif{t} \\
&=\dfrac{\pi}{4}-\dfrac{1}{2}\croch{\ln\paren{t^2+1}}_0^1 \\
&=\dfrac{\pi}{4}-\dfrac{1}{2}\ln2.
\end{aligned}\]
\end{corr}

\begin{corr}[3]
On a la primitive \[\begin{aligned}
x\mapsto\int_1^xt^2\ln t\odif{t}&=\croch{\dfrac{t^3}{3}\ln t}_1^x-\int_1^x\dfrac{t^3}{3}\times\dfrac{1}{t}\odif{t} \\
&=\dfrac{x^3\ln x}{3}-\croch{\dfrac{t^3}{9}}_1^x \\
&=\dfrac{x^3\ln x}{3}-\dfrac{x^3}{9}+\dfrac{1}{9} \\
&=\dfrac{x^3\paren{3\ln x-1}+1}{9}.
\end{aligned}\]
\end{corr}

\begin{corr}[4]
On a la primitive \[\begin{aligned}
x\mapsto\int_0^xt\e{t}\odif{t}&=\croch{t\e{t}}_0^x-\int_0^x\e{t}\odif{t} \\
&=x\e{x}-\e{x}+1 \\
&=\e{x}\paren{x-1}+1.
\end{aligned}\]
\end{corr}

\begin{corr}[5]
On a la primitive \[\begin{aligned}
x\mapsto\int_0^xt^2\cos t\odif{t}&=\croch{t^2\sin t}_0^x-\int_0^x2t\sin t\odif{t} \\
&=x^2\sin x-\croch{-2t\cos t}_0^x+\int_0^x-2\cos t\odif{t} \\
&=x^2\sin x+2x\cos x-2\sin x.
\end{aligned}\]
\end{corr}

\subsection{Formules de Taylor}

\begin{theo}[Formule de Taylor avec reste intégral]
Soient \(n\in\N\) et \(f\in\ensclasse{n+1}{\intervii{a}{b}}{\K}\).

On a \[f\paren{b}=\sum_{k=0}^n\dfrac{f\deriv{k}\paren{a}}{k!}\paren{b-a}^k+\int_a^b\dfrac{\paren{b-t}^n}{n!}f\deriv{n+1}\paren{t}\odif{t}.\]
\end{theo}

\begin{dem}
On raisonne par récurrence finie.

On note \[\quantifs{\forall m\in\interventierii{0}{n}}\underbrace{f\paren{b}=\sum_{k=0}^m\dfrac{f\deriv{k}\paren{a}}{k!}\paren{b-a}^k+\int_a^b\dfrac{\paren{b-t}^m}{m!}f\deriv{m+1}\paren{t}\odif{t}}_{\P{m}}.\]

On a bien \(f\paren{b}=f\paren{a}+\int_a^bf\prim\paren{t}\odif{t}\) car \(f\in\ensclasse{1}{\intervii{a}{b}}{\K}\), d'où \(\P{0}\).

Soit \(m\in\interventierii{0}{n-1}\) tel que \(\P{m}\).

Montrons \(\P{m+1}\).

On peut intégrer la fonction \(t\mapsto\dfrac{\paren{b-t}^m}{m!}f\deriv{m+1}\paren{t}\) car les fonctions \(t\mapsto\dfrac{-\paren{b-t}^{m+1}}{\paren{m+1}!}\) et \(f\deriv{m+1}\) sont de classe \(\classe{1}\).

D'où \[\begin{aligned}
\int_a^b\dfrac{\paren{b-t}^m}{m!}f\deriv{m+1}\paren{t}\odif{t}&=\croch{\dfrac{-\paren{b-t}^{m+1}}{\paren{m+1}!}f\deriv{m+1}\paren{t}}_a^b-\int_a^b\dfrac{-\paren{b-t}^{m+1}}{\paren{m+1}!}f\deriv{m+2}\paren{t}\odif{t} \\
&=0-\dfrac{-\paren{b-a}^{m+1}}{\paren{m+1}!}f\deriv{m+1}\paren{a}+\int_a^b\dfrac{\paren{b-t}^{m+1}}{\paren{m+1}!}f\deriv{m+2}\paren{t}\odif{t}.
\end{aligned}\]

D'où, selon \(\P{m}\) : \[\begin{aligned}
f\paren{b}&=\sum_{k=0}^m\dfrac{f\deriv{k}\paren{a}}{k!}\paren{b-a}^k+\dfrac{\paren{b-a}^{m+1}}{\paren{m+1}!}f\deriv{m+1}\paren{a}+\int_a^b\dfrac{\paren{b-t}^{m+1}}{\paren{m+1}!}f\deriv{m+2}\paren{t}\odif{t} \\
&=\sum_{k=0}^{m+1}\dfrac{f\deriv{k}\paren{a}}{k!}\paren{b-a}^k+\int_a^b\dfrac{\paren{b-t}^{m+1}}{\paren{m+1}!}f\deriv{m+2}\paren{t}\odif{t}.
\end{aligned}\]

D'où \(\P{m+1}\).

D'où, par récurrence : \(\P{n}\).
\end{dem}

\begin{cor}[Inégalité de Taylor-Lagrange]
Soient \(n\in\N\), \(f\in\ensclasse{n+1}{\intervii{a}{b}}{\K}\) et \(M\in\Rp\) tel que \(\quantifs{\forall x\in\intervii{a}{b}}\abs{f\deriv{n+1}\paren{x}}\leq M\).

Alors \[\abs{f\paren{b}-\sum_{k=0}^n\dfrac{f\deriv{k}\paren{a}}{k!}\paren{b-a}^k}\leq\dfrac{M}{\paren{n+1}!}\paren{b-a}^{n+1}.\]
\end{cor}

\begin{dem}
On a, selon la formule de Taylor avec reste intégral : \[\begin{WithArrows}
\abs{f\paren{b}-\sum_{k=0}^n\dfrac{f\deriv{k}\paren{a}}{k!}\paren{b-a}^k}&=\abs{\int_a^b\dfrac{\paren{b-t}^n}{n!}f\deriv{n+1}\paren{t}\odif{t}} \Arrow[tikz={text width=4cm}]{inégalité triangulaire intégrale} \\
&\leq\int_a^b\abs{\dfrac{\paren{b-t}^n}{n!}f\deriv{n+1}\paren{t}}\odif{t} \Arrow{croissance de l'intégrale} \\
&\leq\int_a^b\dfrac{\paren{b-t}^n}{n!}M\odif{t} \\
&\leq\dfrac{M}{n!}\croch{\dfrac{\paren{b-t}^{n+1}}{n+1}}_a^b \\
&=\dfrac{M}{n!}\paren{0-\dfrac{\paren{b-a}^{n+1}}{n+1}} \\
&=\dfrac{M}{\paren{n+1}!}\paren{b-a}^{n+1}.
\end{WithArrows}\]
\end{dem}

\begin{ex}
Soit \(x\in\Rp\).

Prenons \(a=0\), \(b=x\), \(f=\exp\) et \(n=3\).

On a \[\begin{aligned}
\abs{\e{x}-\paren{1+x+\dfrac{x^2}{2}+\dfrac{x^3}{6}}}&\leq\dfrac{\ds\max_{\intervii{0}{x}}\exp}{4!}x^4 \\
&=\dfrac{\e{x}x^4}{24}.
\end{aligned}\]
\end{ex}

\subsection{Changements de variable}

\begin{theo}
Soient \(I\) un intervalle de \(\R\), \(f\in\ensclasse{1}{I}{\R}\) et \(u\in\ensclasse{1}{\intervii{a}{b}}{\R}\) telle que \(\Im u\subset I\).

On a \[\int_{u\paren{a}}^{u\paren{b}}f\paren{t}\odif{t}=\int_a^bf\paren{u\paren{s}}u\prim\paren{s}\odif{s}.\]

On dit qu'on fait le changement de variable \(\begin{dcases}t=u\paren{s} \\ \odif{t}=u\prim\paren{s}\odif{s}\end{dcases}\)
\end{theo}

\begin{dem}
Posons \(\fonction{g}{\intervii{a}{b}}{\R}{x}{\int_{u\paren{a}}^{u\paren{x}}f\paren{t}\odif{t}-\int_a^xf\paren{u\paren{s}}u\prim\paren{s}\odif{s}}\)

Montrons que \(g\) est dérivable.

La fonction \(\fonctionlambda{\intervii{a}{b}}{\K}{s}{f\paren{u\paren{s}}u\prim\paren{s}}\) est continue car \(\begin{dcases}f\text{ est continue} \\ u\text{ est de classe }\classe{1}\end{dcases}\)

Donc \(x\mapsto\int_a^xf\paren{u\paren{s}}u\prim\paren{s}\odif{s}\) est dérivable de dérivée \(x\mapsto f\paren{u\paren{x}}u\prim\paren{x}\).

La fonction \(\fonctionlambda{I}{\K}{y}{\int_{u\paren{a}}^yf\paren{t}\odif{t}}\) est dérivable car \(f\) est continue, de dérivée \(y\mapsto f\paren{y}\).

De plus, \(u\) est de classe \(\classe{1}\) donc par composition de fonctions dérivables : \(\fonctionlambda{\intervii{a}{b}}{\K}{x}{\int_{u\paren{a}}^{u\paren{x}}f\paren{t}\odif{t}}\) est dérivable de dérivée \(\fonctionlambda{\intervii{a}{b}}{\K}{x}{u\prim\paren{x}f\paren{u\paren{x}}}\)

Finalement, \(g\) est dérivable de dérivée nulle.

Donc \(g\) est constante sur l'intervalle \(\intervii{a}{b}\).

Enfin, \(g\paren{a}=0-0=0\).

Donc \(g\) est nulle, d'où la formule énoncée.
\end{dem}

\begin{rem}
Si \(f\) était continue par morceaux alors \(s\mapsto f\paren{u\paren{s}}u\prim\paren{s}\) pourrait ne pas être continue par morceaux.
\end{rem}

\begin{exoex}~\\
\begin{enumerate}
\item Calculer \(I=\int_{-1}^0\dfrac{t}{t^2+2t+2}\odif{t}\). \\

\item Calculer une primitive de \(\dfrac{1}{\ch}\). \\

\item Calculer une primitive de \(f:x\mapsto\dfrac{1}{x\ln x\ln\paren{\ln x}}\).
\end{enumerate}
\end{exoex}

\begin{corr}[1]~\\
On a \(I=\int_{-1}^0\dfrac{t}{\paren{t+1}^2+1}\odif{t}\).

On fait le changement de variable \[\begin{dcases}u=t+1 \\ \odif{u}=\odif{t}\end{dcases}\ssi\begin{dcases}t=u-1 \\ \odif{t}=\odif{u}\end{dcases}\]

Donc \[\begin{aligned}
I&=\int_0^1\dfrac{u-1}{u^2+1}\odif{u} \\
&=\dfrac{1}{2}\int_0^1\dfrac{2u}{u^2+1}\odif{u}-\int_0^1\dfrac{\odif{u}}{u^2+1} \\
&=\dfrac{1}{2}\croch{\ln\paren{u^2+1}}_0^1-\croch{\Arctan u}_0^1 \\
&=\dfrac{1}{2}\ln2-\dfrac{\pi}{4}.
\end{aligned}\]
\end{corr}

\begin{corr}[2]~\\
Une primitive de \(\dfrac{1}{\ch}\) est \(F:x\mapsto\int_0^x\dfrac{1}{\ch t}\odif{t}=\int_0^x\dfrac{2}{\e{t}+\e{-t}}\odif{t}\).

On fait le changement de variable \[\begin{dcases}u=\e{t} \\ \odif{u}=\e{t}\odif{t}\end{dcases}\ssi\begin{dcases}t=\ln u \\ \odif{t}=\dfrac{\odif{u}}{u}\end{dcases}\]

On a \[\begin{aligned}
F:x\mapsto\int_0^x\dfrac{2}{\e{t}+\e{-t}}\odif{t}&=\int_1^{\e{x}}\dfrac{2}{u^2+1}\odif{u} \\
&=2\croch{\Arctan u}_1^{\e{x}} \\
&=2\Arctan\e{x}-\dfrac{\pi}{2}.
\end{aligned}\]
\end{corr}

\begin{corr}[3]
Une primitive de \(f\) est \(F:x\mapsto\int_{10}^x\dfrac{1}{t\ln t\ln\paren{\ln t}}\odif{t}\).

On fait le changement de variable \[\begin{dcases}u=\ln t \\ \odif{u}=\dfrac{\odif{t}}{t}\end{dcases}\ssi\begin{dcases}t=\e{u} \\ \odif{t}=\e{u}\odif{u}\end{dcases}\]

Donc \[\begin{aligned}
F:x\mapsto\int_{\ln10}^{\ln x}\dfrac{\e{u}}{\e{u}u\ln u}\odif{u}&=\int_{\ln10}^{\ln x}\dfrac{1}{u\ln u}\odif{u} \\
&=\int_{\ln10}^{\ln x}\dfrac{\frac{1}{u}}{\ln u}\odif{u} \\
&=\croch{\ln\abs{\ln u}}_{\ln10}^{\ln x} \\
&=\ln\paren{\ln\paren{\ln x}}-\ln\paren{\ln\paren{\ln10}}.
\end{aligned}\]

Donc une primitive de \(f\) est \(x\mapsto\ln\paren{\ln\paren{\ln x}}\).
\end{corr}

\begin{nota}[Officielle]~\\
On peut noter \(x\mapsto\int^xf\paren{t}\odif{t}\) une primitive de la fonction continue \(f\).
\end{nota}

\subsection{Symétries}

\begin{prop}
Soient \(\alpha\in\Rp\) et \(f\in\contm[\intervii{-\alpha}{\alpha}]\).

Si \(f\) est impaire alors \[\int_{-\alpha}^{\alpha}f\paren{t}\odif{t}=0.\]

Si \(f\) est paire, alors \[\int_{-\alpha}^{\alpha}f\paren{t}\odif{t}=2\int_0^{\alpha}f\paren{t}\odif{t}.\]
\end{prop}

\begin{dem}[Dans le cas continu]
On a \[\begin{aligned}
\int_{-\alpha}^0f\paren{t}\odif{t}&=\int_{\alpha}^0-f\paren{-u}\odif{u}\qquad\text{avec }\begin{dcases}u=-t \\ \odif{u}=-\odif{t}\end{dcases}\text{ car }\begin{dcases}f\text{ est de classe }\classe{0} \\ t\mapsto-t\text{ est de classe }\classe{1}\end{dcases} \\
&=\int_0^{\alpha}f\paren{-u}\odif{u} \\
&=\begin{dcases}-\int_0^{\alpha}f\paren{u}\odif{u} &\text{si }f\text{ est impaire} \\ \int_0^{\alpha}f\paren{u}\odif{u} &\text{si }f\text{  est paire}\end{dcases}
\end{aligned}\]

D'où le résultat (car \(\int_{-\alpha}^{\alpha}f\paren{t}\odif{t}=\int_{-\alpha}^0f\paren{t}\odif{t}+\int_0^{\alpha}f\paren{t}\odif{t}\)).
\end{dem}

\begin{prop}
Soient \(f\in\contm[\R]\) et \(T\in\Rps\).

On suppose que \(f\) est \(T\)-périodique.

Alors \[\quantifs{\forall x\in\R}\int_x^{x+T}f\paren{t}\odif{t}=\int_0^Tf\paren{t}\odif{t}.\]
\end{prop}

\begin{dem}[Dans le cas continu]
On pose \(\fonction{g}{\R}{\K}{x}{\int_x^{x+T}f\paren{t}\odif{t}}\)

On a \(\quantifs{\forall x\in\R}g\paren{x}=\int_0^{x+T}f\paren{t}\odif{t}-\int_0^xf\paren{t}\odif{t}\).

Donc \(g\) est dérivable et \(\quantifs{\forall x\in\R}g\prim\paren{x}=f\paren{x+T}-f\paren{x}=0\).

Donc \(g\) est constante sur l'intervalle \(\R\).

D'où le résultat.
\end{dem}

\section{Techniques de calcul}

\subsection{Utiliser les nombres complexes}

\begin{rappel}
Soit \(\alpha\in\C\).

La fonction \(\fonctionlambda{\R}{\C}{x}{\e{\alpha x}}\) est dérivable de dérivée \(\fonctionlambda{\R}{\C}{x}{\alpha\e{\alpha x}}\)
\end{rappel}

\begin{ex}
Calculons une primitive de \(f:x\mapsto\e{2x}\sin x\).

On a \(f=\Im g\) en posant \(\fonction{g}{\R}{\C}{x}{\e{\paren{2+\i}x}}\)

On a \[\begin{aligned}
\quantifs{\forall x\in\R}\int^xg\paren{t}\odif{t}&=\int^x\e{\paren{2+\i}t}\odif{t} \\
&=\croch{\dfrac{1}{2+\i}\e{\paren{2+\i}t}}^x \\
&=\dfrac{1}{2+\i}\e{\paren{2+\i}x} \\
&=\dfrac{\paren{2-\i}\e{\paren{2+\i}x}}{5} \\
&=\dfrac{2}{5}\e{2x}\cos x+\dfrac{2\i}{5}\e{2x}\sin x-\dfrac{\i}{5}\e{2x}\cos x+\dfrac{1}{5}\e{2x}\sin x.
\end{aligned}\]

Donc \(\int^xf\paren{t}\odif{t}=\dfrac{\e{2x}}{5}\paren{2\sin x-\cos x}\).
\end{ex}

\subsection{Primitives des fonctions rationnelles}

\begin{ex}
Calculons une primitive de \(f:x\mapsto\dfrac{3}{x^3-1}\) (sur \(\R\excluant\accol{1}\)).

Calculons la décomposition en éléments simples de \(\dfrac{3}{X^3-1}\).

On a \(X^3-1=\paren{X-1}\paren{X^2+X+1}\).

Donc \(\dfrac{3}{X^3-1}=\dfrac{a}{X-1}+\dfrac{bX+c}{X^2+X+1}\) avec \(a,b,c\in\R\).

Donc \(\dfrac{3}{X^3-1}=\dfrac{a\paren{X^2+X+1}+\paren{X-1}\paren{bX+c}}{X^3-1}\).

Donc \(3=aX^2+aX+a+bX^2+\paren{c-b}X-c\).

Donc \(\begin{dcases}a+b=0 \\ a-b+c=0 \\ a-c=3\end{dcases}\)

Donc \(\begin{dcases}a=1 \\ b=-1 \\ c=-2\end{dcases}\)

Donc \(\dfrac{3}{X^3-1}=\dfrac{1}{X-1}-\dfrac{X+2}{X^2+X+1}\).

D'où \[\int^x\dfrac{3}{t^3-1}\odif{t}=\int^x\dfrac{\odif{t}}{t-1}-\int^x\dfrac{t+2}{t^2+t+1}\odif{t}.\]

Or \(\int^x\dfrac{\odif{t}}{t-1}=\croch{\ln\abs{t-1}}^x=\ln\abs{x-1}\).

De plus, \(\int^x\dfrac{t+2}{t^2+t+1}\odif{t}=\int^x\dfrac{t+2}{\paren{t+\frac{1}{2}}^2+\frac{3}{4}}\odif{t}=\int^x\dfrac{t+2}{\frac{3}{4}\paren{\paren{\frac{2}{\sqrt{3}}\paren{t+\frac{1}{2}}}^2}+1}\odif{t}\).

On fait le changement de variable \[\begin{dcases}u=\dfrac{2}{\sqrt{3}}\paren{t+\dfrac{1}{2}} \\ \odif{u}=\dfrac{2}{\sqrt{3}}\odif{t}\end{dcases}\ssi\begin{dcases}t=\dfrac{\sqrt{3}}{2}u-\dfrac{1}{2} \\ \odif{t}=\dfrac{\sqrt{3}}{2}\odif{u}\end{dcases}\]

D'où \[\begin{aligned}
\int^x\dfrac{t+2}{t^2+t+1}\odif{t}&=\int^{\frac{2}{\sqrt{3}}\paren{x+\frac{1}{2}}}\dfrac{\frac{\sqrt{3}}{2}u-\frac{1}{2}+2}{\frac{3}{4}\paren{u^2+1}}\odif{u} \\
&=\dfrac{2}{\sqrt{3}}\int^{\frac{2}{\sqrt{3}}\paren{x+\frac{1}{2}}}\dfrac{\frac{\sqrt{3}}{2}+\frac{3}{2}}{u^2+1}\odif{u} \\
&=\dfrac{2}{\sqrt{3}}\croch{\dfrac{\sqrt{3}}{4}\ln\paren{u^2+1}+\dfrac{3}{2}\Arctan u}^{\frac{2}{\sqrt{3}}\paren{x+\frac{1}{2}}} \\
&=\dfrac{1}{2}\ln\paren{\dfrac{4}{3}\paren{x+\dfrac{1}{2}}^2+1}+\sqrt{3}\Arctan\paren{\dfrac{2}{\sqrt{3}}\paren{x+\dfrac{1}{2}}} \\
&=\dfrac{1}{2}\ln\paren{\dfrac{4}{3}\paren{x^2+x+1}}+\sqrt{3}\Arctan\dfrac{2x+1}{\sqrt{3}}.
\end{aligned}\]

Donc une primitive de \(f\) est \[x\mapsto\ln\abs{x-1}-\dfrac{1}{2}\ln\paren{\dfrac{4}{3}\paren{x^2+x+1}}-\sqrt{3}\Arctan\dfrac{2x+1}{\sqrt{3}}.\]
\end{ex}

\begin{prop}
On peut primitiver toute fonction rationnelle en écrivant sa décomposition en éléments simples sur \(\R\).
\end{prop}

\begin{rem}
C'est suffisant, pas nécessaire.

Par exemple : \(\int_0^1\dfrac{t^7}{t^8+1}\odif{t}=\croch{\dfrac{1}{8}\ln\paren{t^8+1}}_0^1=\dfrac{\ln2}{8}\).
\end{rem}

\begin{dem}
On peut primitiver chaque terme de la décomposition en éléments simples sur \(\R\) :

La partie entière : c'est un polynôme.

Les termes de la forme \(t\mapsto\dfrac{1}{t-\lambda}\) avec \(\lambda\in\R\) : primitive \[t\mapsto\ln\abs{t-\lambda}.\]

Les termes de la forme \(t\mapsto\dfrac{1}{\paren{t-\lambda}^{\alpha}}=\paren{t-\lambda}^{-\alpha}\) où \(\begin{dcases}\lambda\in\R \\ \alpha\in\interventierie{2}{\pinf}\end{dcases}\) : primitive \[t\mapsto\dfrac{1}{1-\alpha}\paren{t-\lambda}^{1-\alpha}.\]

Les termes de la forme \(t\mapsto\dfrac{\lambda t+\mu}{t^2+bt+c}\) avec \(\begin{dcases}\lambda,\mu\in\R \\ b,c\in\R\text{ tels que }\Delta=b^2-4c<0\end{dcases}\) :

Par un changement de variable affine, on se ramène à \[\int^y\dfrac{\lambda\prim u+\mu\prim}{u^2+1}\odif{u}=\dfrac{\lambda\prim}{2}\ln\paren{y^2+1}+\mu\prim\Arctan y\] et \[\begin{aligned}
\int^x\dfrac{\odif{t}}{t^2+bt+c}&=\int^x\dfrac{\odif{t}}{\paren{t+\frac{b}{2}}^2-\frac{b^2}{4}+c} \\
&=\int^x\dfrac{\odif{t}}{\paren{t+\frac{b}{2}}^2\underbrace{-\frac{\Delta}{4}}_{>0}} \\
&=\int^x\dfrac{\odif{t}}{-\frac{\Delta}{4}\paren{\paren{\frac{\sqrt{-\Delta}}{2}\paren{t+\frac{b}{2}}}^2+1}}.
\end{aligned}\]

D'où ce qu'on voulait par le changement de variable \(u=\dfrac{\sqrt{-\Delta}}{2}\paren{t+\dfrac{b}{2}}\).

Les termes de la forme \(t\mapsto\dfrac{\lambda t+\mu}{\paren{t^2+bt+c}^{\alpha}}\) avec \(\begin{dcases}\alpha\in\interventierie{2}{\pinf} \\ \lambda,\mu,b,c\in\R \\ b^2-4c<0\end{dcases}\) :

Quitte à faire un changement de variable affine, on peut supposer \(b=0\) et \(c=1\). Il suffit donc de savoir calculer \[\int_0^x\dfrac{\lambda t+\mu}{\paren{t^2+1}^{\alpha}}\odif{t}=\underbrace{\lambda\int_0^xt\paren{t^2+1}^{-\alpha}\odif{t}}_{\frac{\lambda}{2\paren{1-\alpha}}\paren{x^2+1}^{1-\alpha}}+\underbrace{\mu\int_0^x\dfrac{\odif{t}}{\paren{t^2+1}^{\alpha}}}_{I_\alpha\paren{x}}.\]

Reste à calculer \(I_\alpha\paren{x}\) par récurrence sur \(\alpha\), à l'aide d'une intégration par partie.

En effet : \[\begin{aligned}
\int_0^x\paren{t^2+1}^{-\alpha}\odif{t}&=\croch{t\paren{t^2+1}^{-\alpha}}_0^x-\int_0^xt\times2t\paren{-\alpha}\paren{t^2+1}^{-\alpha-1}\odif{t} \\
&=\dfrac{x}{\paren{x^2+1}^{\alpha}}+2\alpha\int_0^x\paren{t^2+1-1}\paren{t^2+1}^{-\alpha-1}\odif{t} \\
&=\dfrac{x}{\paren{x^2+1}^{\alpha}}+2\alpha I_\alpha\paren{x}-2\alpha I_{\alpha+1}\paren{x}.
\end{aligned}\]

D'où \(\paren{1-2\alpha}I_\alpha\paren{x}=\dfrac{x}{\paren{x^2+1}^{\alpha}}-2\alpha I_{\alpha+1}\paren{x}\).

On connaît \(I_1\paren{x}\) (\cf termes de la forme \(t\mapsto\dfrac{1}{\paren{t-\lambda}^{\alpha}}\)).

On en déduit \(I_2\), \(I_3\), ...
\end{dem}

\subsection{Fonctions rationnelles en \(\e{t}\)}

\begin{ex}
\[t\mapsto\dfrac{1}{\ch t}=\dfrac{2}{\e{t}+\e{-t}}=\dfrac{2\e{t}}{\e{2t}+1}\qquad\text{et}\qquad t\mapsto\dfrac{\e{2t}+1}{\e{t}-1}\]
\end{ex}

\begin{prop}
On remarque \[\quantifs{\forall F\in\fracrat}\int^xF\paren{\e{t}}\odif{t}=\int^{\e{x}}\dfrac{F\paren{u}}{u}\odif{u}\] en opérant le changement de variable \[\begin{dcases}u=\e{t} \\ \odif{u}=\e{t}\odif{t}\end{dcases}\ssi\begin{dcases}t=\ln u \\ \odif{t}=\dfrac{\odif{u}}{u}\end{dcases}\]

Or \(\dfrac{F}{X}\in\fracrat\) donc on sait primitiver.
\end{prop}

\subsection{Règle de Bioche}

\begin{meth}~\\
Pour calculer \(\int F\paren{\cos\theta,\sin\theta}\odif{\theta}\) :

\begin{itemize}
\item Si \(F\paren{\cos\theta,\sin\theta}\odif{\theta}\) est invariant quand on remplace \(\theta\) par \(-\theta\) : \[\text{faire le changement de variable }t=\cos\theta.\]

\item Si \(F\paren{\cos\theta,\sin\theta}\odif{\theta}\) est invariant quand on remplace \(\theta\) par \(\pi-\theta\) : \[\text{faire le changement de variable }t=\sin\theta.\]

\item Si \(F\paren{\cos\theta,\sin\theta}\odif{\theta}\) est invariant quand on remplace \(\theta\) par \(\pi+\theta\) : \[\text{faire le changement de variable }t=\tan\theta.\]

\item Dans tous les cas, on peut \[\text{faire le changement de variable }t=\tan\dfrac{\theta}{2}.\]
\end{itemize}
\end{meth}

\begin{exoex}
Calculer une primitive de \(x\mapsto\dfrac{\sin^3x}{1+\cos^2x}\).
\end{exoex}

\begin{corr}
On a la primitive \[x\mapsto\int^x\dfrac{\sin^3\theta}{1+\cos^2\theta}\odif{\theta}.\]

On fait le changement de variable \(\begin{dcases}t=\cos\theta \\ \odif{t}=-\sin\theta\odif{\theta}\end{dcases}\)

Donc \[\begin{aligned}
x\mapsto\int^x\dfrac{\sin^3\theta}{1+\cos^2\theta}\odif{\theta}&=\int^x\dfrac{-1+\cos^2\theta}{1+\cos^2\theta}\paren{-\sin\theta}\odif{\theta} \\
&=\int^{\cos x}\dfrac{-1+t^2}{1+t^2}\odif{t} \\
&=\int^{\cos x}\dfrac{t^2+1-2}{t^2+1}\odif{t} \\
&=\int^{\cos x}\paren{1-\dfrac{2}{t^2+1}}\odif{t} \\
&=\cos x-2\Arctan\paren{\cos x}.
\end{aligned}\]
\end{corr}

\chapter{Espaces vectoriels}

\minitoc

Dans tout le chapitre, on considère un corps \(\K\) (en pratique, \(\K=\R\) ou \(\C\)).

\begin{nota}
Soient \(p\in\N\) et \(x_1,\dots,x_p\in\K\).

On s'autorise à écrire le \(p\)-uplet \(\paren{x_1,\dots,x_p}\) verticalement (écriture \guillemets{matricielle}) : \[\paren{x_1,\dots,x_p}=\begin{pmatrix}x_1 \\ \vdots \\ x_p\end{pmatrix}\in\K^p.\]
\end{nota}

\section{Espaces vectoriels, sous-espaces vectoriels}

\subsection{Espaces vectoriels}

\begin{defi}[\(\K\)-espace vectoriel]\thlabel{def:K-EV}
On appelle \(\K\)-espace vectoriel (ou espace vectoriel sur \(\K\)) tout triplet \(\corps{E}[+][\cdot]\) où \(\groupe{E}\) est un groupe abélien et \(\cdot\) est une application (appelée parfois loi externe) : \[\fonctionlambda{\K\times E}{E}{\paren{\lambda,x}}{\lambda\cdot x=\lambda x}\] qui vérifient \[\begin{dcases}\quantifs{\forall\lambda\in\K;\forall x,y\in E}\lambda\cdot\paren{x+y}=\lambda\cdot x+\lambda\cdot y \\ \quantifs{\forall\lambda,\mu\in\K;\forall x\in E}\paren{\lambda+\mu}\cdot x=\lambda\cdot x+\mu\cdot x \\ \quantifs{\forall\lambda,\mu\in\K;\forall x\in E}\paren{\lambda\mu}\cdot x=\lambda\cdot\paren{\mu\cdot x} \\ \quantifs{\forall x\in E}1_\K\cdot x=x\end{dcases}\]

Les éléments de \(E\) sont appelés les vecteurs.

Les éléments de \(\K\) sont appelés les scalaires.

L'élément neutre du groupe abélien \(\groupe{E}\) est noté \(0\) ou \(0_E\). Il est appelé le vecteur nul de \(E\).

Par abus, on dit que \(E\) est un \(\K\)-espace vectoriel.

Un espace vectoriel est nécessairement non-vide (puisqu'il contient son vecteur nul). Si \(E\) ne contient pas d'autre vecteur que le vecteur nul (\cad si \(E=\accol{0_E}\)), on dit que \(E\) est l'espace vectoriel nul.
\end{defi}

\begin{defi}[Combinaison linéaire]\thlabel{def:combinaisonLinéaire}
Soient \(E\) un \(\K\)-espace vectoriel et \(x_1,\dots,x_n\in E\) (où \(n\in\N\)).

On appelle combinaison linéaire de \(x_1,\dots,x_n\) tout vecteur de la forme : \[\lambda_1\cdot x_1+\dots+\lambda_n\cdot x_n\] avec \(\lambda_1,\dots,\lambda_n\in\K\).
\end{defi}

\begin{ex}\thlabel{ex:EVs}
Soit \(n\in\Ns\). Alors \(\K^n\) est naturellement un \(\K\)-espace vectoriel.

\(\poly\) est naturellement un \(\K\)-espace vectoriel.

Soit \(\Omega\) un ensemble quelconque. L'ensemble \(\F{\Omega}{\K}\) des fonctions de \(\Omega\) dans \(\K\) est naturellement un \(\K\)-espace vectoriel.

L'ensemble \(\K^\N\) des suites d'éléments de \(\K\) est naturellement un \(\K\)-espace vectoriel.

Soient \(E_1,\dots,E_m\) des \(\K\)-espaces vectoriels. On rappelle que l'ensemble produit \(E_1\times\dots\times E_m\) est formé des \(m\)-uplets de la forme \(\paren{x_1,\dots,x_m}\) où \(x_1\in E_1,\dots,x_m\in E_m\). L'ensemble \(E_1\times\dots\times E_m\) est naturellement un \(\K\)-espace vectoriel.
\end{ex}

\begin{reform}[Concrète]
La reformulation suivante n'est ici que pour faire le point sur les opérations autorisées dans un \(\K\)-espace vectoriel (à lire simplement, la définition à connaître est la \thref{def:K-EV}).

Un \(\K\)-espace vectoriel est un triplet \(\corps{E}[+][\cdot]\) où \(E\) est un ensemble, \(+\) est une fonction de \(E\times E\) dans \(E\) et \(\cdot\) est une fonction de \(\K\times E\) dans \(E\) : \[\fonctionlambda{E\times E}{E}{\paren{x,y}}{x+y}\qquad\text{et}\qquad\fonctionlambda{\K\times E}{E}{\paren{\lambda,x}}{\lambda\cdot x=\lambda x}\] qui vérifient \[\begin{dcases}
\quantifs{\forall x,y,z\in E}\paren{x+y}+z=x+\paren{y+z} \\
\quantifs{\forall x,y\in E}x+y=y+x \\
\quantifs{\exists0_E\in E;\forall x\in E}x+0_E=x \\
\quantifs{\forall x\in E;\exists-x\in E}x+\paren{-x}=0_E
\end{dcases}\] et \[\begin{dcases}
\quantifs{\forall\lambda\in\K;\forall x,y\in E}\lambda\cdot\paren{x+y}=\lambda\cdot x+\lambda\cdot y \\
\quantifs{\forall\lambda,\mu\in\K;\forall x\in E}\paren{\lambda+\mu}\cdot x=\lambda\cdot x+\mu\cdot x \\
\quantifs{\forall\lambda,\mu\in\K;\forall x\in E}\paren{\lambda\mu}\cdot x=\lambda\cdot\paren{\mu\cdot x} \\
\quantifs{\forall x\in E}1_\K\cdot x=x
\end{dcases}\]

Le vecteur \(0_E\) est en fait unique. Il est appelé le vecteur nul de \(E\).

Pour tout vecteur \(x\in E\), le vecteur \(-x\) est unique et appelé l'opposé de \(x\). Pour simplifier, on note \(y-x\) pour \(y+\paren{-x}\).
\end{reform}

\begin{reform}[Abstraite]
Un \(\K\)-espace vectoriel est un groupe abélien \(\groupe{E}\) muni d'un morphisme d'anneaux \(\K\to\Hom{E}{E}\).
\end{reform}

\begin{rem}
On a vu à l'\thref{ex:EVs} que \(\K\) est naturellement un \(\K\)-espace vectoriel (en prenant \(n=1\)). La loi \(\cdot\) coïncide alors avec la loi produit du corps \(\K\) ; les éléments de \(\K\) peuvent alors être vus comme des scalaires ou comme des vecteurs.

Si \(E\) est un \(\C\)-espace vectoriel, alors \(E\) est naturellement un \(\R\)-espace vectoriel (en ne gardant de la loi \(\C\times E\to E\) que sa restriction \(\R\times E\to E\)).
\end{rem}

\begin{exo}
\begin{enumerate}
\item On considère le \(\R\)-espace vectoriel \(\poly[\R]\).

Le vecteur \(X^2-1\) est-il combinaison linéaire des vecteurs \(X-1\) et \(X+1\) ?

Le vecteur \(X^2-1\) est-il combinaison linéaire des vecteurs \(X^2+X\) et \(X+1\) ? \\

\item On considère le \(\R\)-espace vectoriel \(\R^4\).

Le vecteur \(\paren{2,2,2,2}\) est-il combinaison linéaire des vecteurs \(\paren{1,2,0,0}\) et \(\paren{0,1,-1,-1}\) ?

Le vecteur \(\paren{1,0,0,0}\) est-il combinaison linéaire des vecteurs \(\paren{1,2,0,0}\) et \(\paren{0,1,-1,-1}\) ?
\end{enumerate}
\end{exo}

\begin{corr}[1]
On a \(\quantifs{\forall\lambda,\mu\in\R}\deg\paren{\lambda\paren{X-1}+\mu\paren{X+1}}\leq1\) donc \(\quantifs{\forall\lambda,\mu\in\R}\lambda\paren{X-1}+\mu\paren{X+1}\not=X^2-1\). Donc \(X^2-1\) n'est pas une combinaison linéaire de \(X-1\) et \(X+1\).

De plus, on a \(X^2-1=1\cdot\paren{X^2+X}-1\cdot\paren{X+1}\) donc \(X^2-1\) est une combinaison linéaire de \(X^2+X\) et \(X+1\).
\end{corr}

\begin{corr}[2]
On a \(\paren{2,2,2,2}=2\cdot\paren{1,2,0,0}-2\cdot\paren{0,1,-1,-1}\) donc \(\paren{2,2,2,2}\) est combinaison linéaire de \(\paren{1,2,0,0}\) et \(\paren{0,1,-1,-1}\).

On a \(\quantifs{\forall\lambda,\mu\in\R}\lambda\paren{1,2,0,0}+\mu\paren{0,1,-1,-1}=\paren{1,0,0,0}\ssi\begin{dcases}
\lambda=1 \\
2\lambda+\mu=0 \\
-\mu=0 \\
-\mu=0
\end{dcases}\text{ : impossible}\)

Donc \(\paren{1,0,0,0}\) n'est pas combinaison linéaire de \(\paren{1,2,0,0}\) et \(\paren{0,1,-1,-1}\).
\end{corr}

\begin{prop}\thlabel{prop:lambdaFoisXNulSsiLambdaNulOuXNul}
Soient \(E\) un \(\K\)-espace vectoriel, \(\lambda\in\K\) un scalaire et \(x\in E\) un vecteur.

On a \[\lambda\cdot x=0_E\ssi\orenv{\lambda=0_\K \\ x=0_E}\]
\end{prop}

\begin{dem}
Supposons \(\lambda=0\).

Montrons que \(\lambda\cdot x=0_E\).

On a \(\lambda+1_\K=1_\K\).

Donc \(\paren{\lambda+1_\K}\cdot x=1_\K\cdot x\).

Donc \(\lambda\cdot x+1_\K\cdot x=1_\K\cdot x\).

Donc \(\lambda\cdot x=0_E\) car \(\groupe{E}\) est un groupe.

D'où l'équivalence.

Supposons \(\lambda\not=0_\K\).

Alors \(\lambda\) est inversible car \(\K\) est un corps.

On a \[\begin{aligned}
\lambda\cdot x=0_E&\imp\lambda\inv\cdot\paren{\lambda\cdot x}=\lambda\inv\cdot0_E \\
&\imp\paren{\lambda\inv\lambda}\cdot x=0_E \\
&\imp1_\K\cdot x=0_E \\
&\imp x=0_E \\
&\imp\lambda\cdot x=0_E
\end{aligned}\]

Donc \(\lambda\cdot x\ssi x=0_E\).

D'où l'équivalence.
\end{dem}

\begin{prop}
Soient \(E\) un \(\K\)-espace vectoriel, \(\lambda\in\K\) un scalaire et \(x\in E\) un vecteur.

On a \[\paren{-\lambda}\cdot x=\lambda\cdot\paren{-x}=-\paren{\lambda\cdot x}\] et \[\paren{-\lambda}\cdot\paren{-x}=\lambda\cdot x.\]
\end{prop}

\begin{dem}
On remarque \[\paren{-\lambda}\cdot x+\lambda\cdot x=\paren{-\lambda+\lambda}\cdot x=0_\K\cdot x=0_E.\]

Donc \(\paren{-\lambda}\cdot x=-\paren{\lambda\cdot x}\).

De même : \[\lambda\cdot\paren{-x}+\lambda\cdot x=\lambda\cdot\paren{x-x}=\lambda\cdot0_E=0_E.\]

Donc \(\lambda\cdot\paren{-x}=-\paren{\lambda\cdot x}\).

Enfin : \[\paren{-\lambda}\cdot\paren{-x}=-\paren{\lambda\cdot\paren{-x}}=-\paren{\paren{-\lambda}\cdot x}=\lambda x.\]
\end{dem}

\subsection{Sous-espaces vectoriels}

\begin{defi}
Soient \(E\) un \(\K\)-espace vectoriel dont on note \(0_E\) l'élément neutre et \(F\) une partie de \(E\).

On dit que \(F\) est un sous-\(\K\)-espace vectoriel de \(E\) (ou sous-espace vectoriel de \(E\)) si les conditions suivantes sont vérifiées :

\begin{enumerate}
\item \(0_E\in F\) \\

\item La partie \(F\) est \guillemets{stable par combinaison linéaire} : \[\quantifs{\forall\lambda_1,\lambda_2\in\K;\forall x_1,x_2\in F}\lambda_1\cdot x_1+\lambda_2\cdot x_2\in F.\]
\end{enumerate}
\end{defi}

\begin{prop}
Tout sous-espace vectoriel \(F\) d'un \(\K\)-espace vectoriel \(\corps{E}[+][\cdot]\) est naturellement un \(\K\)-espace vectoriel : ses lois sont celles induites par celles de \(E\).
\end{prop}

\begin{dem}
On note \(0_E\) le vecteur nul de \(E\).

Montrons que \(F\) est un sous-groupe de \(\groupe{E}\).

On a \[\begin{dcases}0_E\in F \\ \quantifs{\forall x,y\in F}x-y=1_\K\cdot x+\paren{-1_\K}\cdot y\in F\end{dcases}\]

Donc \(F\) est un sous-groupe de \(\groupe{E}\).

Donc la loi \(+\) de \(E\) induit une loi de groupe abélien \(\fonctionlambda{F\times F}{F}{\paren{x,y}}{x+y}\)

On a \[\quantifs{\forall\lambda\in\K;\forall x\in F}\lambda\cdot x\in F\] car \(\lambda\cdot x=\lambda\cdot x+0_\K\cdot0_E\in F\).

Donc la loi externe \(\cdot\) induite \(\fonctionlambda{\K\times F}{F}{\paren{\lambda,x}}{\lambda\cdot x}\) est bien définie.

Enfin, les quatre propriétés sont clairement conservées.
\end{dem}

\begin{ex}
Soient \(n\in\N\) et \(I\) un intervalle de \(\R\) contenant au moins deux points.

L'ensemble \(\ensclasse{n}{I}{\K}\) est un sous-espace vectoriel de \(\F{I}{\K}\) et est donc muni d'une structure naturelle de \(\R\)-espace vectoriel.
\end{ex}

\begin{prop}[Intersection de sous-espaces vectoriels]
Soient \(E\) un \(\K\)-espace vectoriel, \(I\) un ensemble quelconque et \(\paren{F_i}_{i\in I}\) une famille de sous-espaces vectoriels de \(E\).

Alors \(\biginter_{i\in I}F_i\) est un sous-espace vectoriel de \(E\).
\end{prop}

\begin{dem}
On a \(\quantifs{\forall i\in I}0_E\in F_i\) donc \(0_E\in\biginter_{i\in I}F_i\).

Soient \(\lambda,\mu\in\K\) et \(x,y\in\biginter_{i\in I}F_i\).

On a \(\quantifs{\forall i\in I}x,y\in F_i\).

Donc \(\quantifs{\forall i\in I}\lambda\cdot x+\mu\cdot y\in F_i\).

Donc \(\lambda\cdot x+\mu\cdot y\in\biginter_{i\in I}F_i\).
\end{dem}

\begin{ex}
Soit \(I\) un intervalle de \(\R\) contenant au moins deux points.

L'ensemble \(\ensclasse{\infty}{I}{\K}\) est un sous-espace vectoriel de \(\F{I}{\K}\) car c'est l'intersection des sous-espaces vectoriels \(\ensclasse{n}{I}{\K}\) où \(n\in\N\).
\end{ex}

\begin{defprop}
Soient \(E\) un \(\K\)-espace vectoriel et \(x_1,\dots,x_n\in E\).

On appelle sous-espace vectoriel engendré par \(x_1,\dots,x_n\) et on note \(\Vect{x_1,\dots,x_n}\) le plus petit sous-espace vectoriel de \(E\) qui contient \(x_1,\dots,x_n\).

Ses éléments sont les combinaisons linéaires en \(x_1,\dots,x_n\) : \[\Vect{x_1,\dots,x_n}=\accol{\lambda_1x_1+\dots+\lambda_nx_n}_{\lambda_1,\dots,\lambda_n\in\K}\]
\end{defprop}

\begin{dem}
Notons \(F\) l'ensemble des combinaisons linéaires en \(x_1,\dots,x_n\) : \[F=\accol{\lambda_1x_1+\dots+\lambda_nx_n}_{\lambda_1,\dots,\lambda_n\in\K}\]

Montrons que \(F\) est un sous-espace vectoriel de \(E\).

On a \(0_E=0_\K x_1+\dots+0_\K x_n\) donc \(0_E\in F\).

De plus, si \(x,x\prim\in F\) alors il existe \(\lambda_1,\dots,\lambda_n,\lambda_1\prim,\dots,\lambda_n\prim\in\K\) tels que \[x=\lambda_1x_1+\dots+\lambda_nx_n\qquad\text{et}\qquad x\prim=\lambda_1\prim x_1+\dots+\lambda_n\prim x_n.\]

Soient \(\mu,\mu\prim\in\K\).

On a \[\mu x+\mu\prim x\prim=\paren{\mu\lambda_1+\mu\prim\lambda_1\prim}x_1+\dots+\paren{\mu\lambda_n+\mu\prim\lambda_n\prim}x_n.\]

Donc \(\mu x+\mu\prim x\prim\in F\).

Donc \(F\) est un sous-espace vectoriel de \(E\).

On a \(x_1,\dots,x_n\in F\) car \(\quantifs{\forall k\in\interventierii{1}{n}}x_k=\sum_{i=1}^n\delta_{ki}x_i\) où \(\delta_{ki}=\begin{dcases}1 &\text{si }k=i \\ 0 &\text{sinon}\end{dcases}\) (c'est le symbole de Kronecker).

Enfin, si \(G\) est un sous-espace vectoriel de \(E\) contenant \(x_1,\dots,x_n\) alors on a \[\quantifs{\forall\lambda_1,\dots,\lambda_n\in\K}\lambda_1x_1+\dots+\lambda_nx_n\in G\] donc \(F\subset G\).

Donc \(F\) est le plus petit sous-espace vectoriel de \(E\) contenant \(x_1,\dots,x_n\).
\end{dem}

\subsection{Sommes, sommes directes}

\begin{defprop}[Somme de deux sous-espaces vectoriels]
Soient \(E\) un \(\K\)-espace vectoriel et \(F\) et \(G\) deux sous-espaces vectoriels de \(E\).

L'ensemble des vecteurs de la forme \(y+z\), où \(y\in F\) et \(z\in G\), est un sous-espace vectoriel de \(E\), appelé somme des sous-espaces vectoriels \(F\) et \(G\). On le note \(F+G\) : \[F+G=\accol{y+z}_{\paren{y,z}\in F\times G}=\accol{x\in E\tq\quantifs{\exists y\in F;\exists z\in G}x=y+z}.\]
\end{defprop}

\begin{defi}[Somme directe]
Soient \(E\) un \(\K\)-espace vectoriel et \(F\) et \(G\) deux sous-espaces vectoriels de \(E\).

On dit que \(F\) et \(G\) sont en somme directe si l'écriture de tout vecteur de \(F+G\) sous la forme \(y+z\) est unique, où \(y\in F\) et \(z\in G\), \cad si : \[\quantifs{\forall x\in F+G;\exists!\paren{y,z}\in F\times G}x=y+z.\]

La somme de \(F\) et \(G\) est alors appelée somme directe de \(F\) et \(G\) et est notée \[F\oplus G.\]
\end{defi}

\begin{prop}
Soient \(E\) un \(\K\)-espace vectoriel et \(F\) et \(G\) deux sous-espaces vectoriels de \(E\).

Alors \(F\) et \(G\) sont en somme directe si, et seulement si, leur intersection est le sous-espace vectoriel nul : \[F\inter G=\accol{0_E}.\]
\end{prop}

\begin{dem}
Montrons l'équivalence \[F\text{ et }G\text{ sont en somme directe}\ssi F\inter G=\accol{0_E}.\]

\impdir

\increc Claire.

\incdir

Soit \(x\in F\inter G\).

On a \[x=\underbrace{x}_{\in F}+\underbrace{0_E}_{\in G}=\underbrace{0_E}_{\in F}+\underbrace{x}_{\in G}.\]

Donc comme \(F\) et \(G\) sont en somme directe : \(\paren{x,0_E}=\paren{0_E,x}\).

Donc \(x=0_E\).

Donc \(F\inter G=\accol{0_E}\).

\imprec

Supposons \(F\inter G=\accol{0_E}\).

Montrons que \(F\) et \(G\) sont en somme directe.

Soient \(x_F,y_F\in F\) et \(x_G,y_G\in G\) tels que \(x_F+x_G=y_F+y_G\).

Donc \(\underbrace{x_F-y_F}_{\in F}=\underbrace{y_G-x_G}_{\in G}\).

Donc \(\begin{dcases}x_F-y_F\in F\inter G=\accol{0_E} \\ y_G-x_G\in F\inter G=\accol{0_E}\end{dcases}\)

Donc \(\begin{dcases}x_F-y_F=0_E \\ y_G-x_G=0_E\end{dcases}\)

Donc \(\paren{x_F,x_G}=\paren{y_F,y_G}\).

Donc l'écriture est unique.

Donc \(F\) et \(G\) sont en somme directe.
\end{dem}

\begin{defi}[Supplémentaire]
Soient \(E\) un \(\K\)-espace vectoriel et \(F\) et \(G\) deux sous-espaces vectoriels de \(E\).

On dit que \(G\) est un supplémentaire de \(F\) dans \(E\) (ou que \(F\) et \(G\) sont deux sous-espaces vectoriels supplémentaires dans \(E\)) si on a : \[E=F\oplus G,\] \cad \(\begin{dcases}F\inter G=\accol{0_E} \\ E=F+G\end{dcases}\)
\end{defi}

\begin{ex}\thlabel{ex:EVsSupplémentaires}
\begin{enumerate}
\item En voyant \(\C\) comme un \(\R\)-espace vectoriel, on a \(\C=\R\oplus\i\R\). \\

\item On considère le \(\R\)-espace vectoriel \(\F{\R}{\R}\). On note \(F\) (respectivement \(G\)) l'ensemble des fonctions paires (respectivement impaires) de \(\R\) dans \(\R\). Alors \(F\) et \(G\) sont deux sous-espaces vectoriels supplémentaires dans \(\F{\R}{\R}\). \\

\item Soit \(n\in\N\) et \(Q\in\poly\) un polynôme de degré \(n+1\). On note \(\poly Q\) l'ensemble des multiples de \(Q\) dans \(\poly\). Il admet pour supplémentaire l'espace des polynômes de degré au plus \(n\) : \[\poly=\poly Q\oplus\polydeg{n}.\]
\end{enumerate}
\end{ex}

\begin{dem}[2]
Montrons que \(F\inter G=\accol{0}\).

Soit \(f\in F\inter G\).

On a \(\quantifs{\forall x\in\R}f\paren{x}=f\paren{-x}=-f\paren{x}\) donc \(f=0\).

Donc \(F\inter G=\accol{0}\).

Donc \(F\) et \(G\) sont en somme directe.

Montrons que \(F+G=\F{\R}{\R}\).

\incdir Claire.

\increc

Soit \(f\in\F{\R}{\R}\).

On remarque \[\quantifs{\forall x\in\R}f\paren{x}=\dfrac{f\paren{x}+f\paren{-x}}{2}+\dfrac{f\paren{x}-f\paren{-x}}{2}.\]

Donc \(f=g+h\) en posant \(\fonction{g}{\R}{\R}{x}{\dfrac{f\paren{x}+f\paren{-x}}{2}}\) et \(\fonction{h}{\R}{\R}{x}{\dfrac{f\paren{x}-f\paren{-x}}{2}}\)

De plus, on a \(g\in F\) et \(h\in G\).

Donc \(f\in F+G\).

Conclusion : on a montré \(\begin{dcases}F\inter G=\accol{0} \\ F+G=\F{\R}{\R}\end{dcases}\) donc \(F\oplus G=\F{\R}{\R}\).

Donc \(F\) et \(G\) sont supplémentaires dans \(\F{\R}{\R}\).
\end{dem}

\begin{dem}[3]
Montrons que \(\poly Q\inter\polydeg{n}=\accol{0}\).

Soit \(P\in\poly Q\inter\polydeg{n}\).

Comme \(P\in\poly Q\), il existe \(A\in\poly\) tel que \(P=AQ\).

On a donc \(\deg P=\underbrace{\deg A}_{\in\N\union\accol{\minf}}+\underbrace{\deg Q}_{=n+1}\leq n\).

Donc \(\deg P=\minf\) donc \(P=0\).

Donc \(\poly Q\inter\polydeg{n}=\accol{0}\).

Montrons que \(\poly Q+\polydeg{n}=\poly\).

Soit \(P\in\poly\).

On note \(A\) et \(B\) le quotient et le reste de la division euclidienne de \(P\) par \(Q\) : on a \(P=AQ+B\) avec \(\begin{dcases}AQ\in\poly Q \\ B\in\polydeg{n}\end{dcases}\)

Donc \(P\in\poly Q+\polydeg{n}\).

Finalement : \[\poly=\poly Q\oplus\polydeg{n}.\]
\end{dem}

\begin{rem}
\begin{itemize}
\item Ne pas confondre supplémentaire et complémentaire. Le complémentaire d'un sous-espace vectoriel n'est jamais un sous-espace vectoriel car il ne contient pas le vecteur nul. \\

\item Ne pas confondre \guillemets{deux sous-espaces vectoriels sont en somme directe} et \guillemets{deux sous-espaces vectoriels sont supplémentaires}. \\

\item Il n'y a pas unicité du supplémentaire d'un sous-espace vectoriel, sauf dans le cas des deux sous-espaces vectoriels triviaux : si \(E\) est un \(\K\)-espace vectoriel, le seul supplémentaire de \(\accol{0_E}\) dans \(E\) est \(E\) ; le seul supplémentaire de \(E\) dans \(E\) est \(\accol{0_E}\).
\end{itemize}
\end{rem}

\section{Applications linéaires}

\subsection{Définitions}

\begin{defi}[Application linéaire]
Soient \(E\) et \(F\) deux \(\K\)-espaces vectoriels et \(u:E\to F\).

On dit que \(u\) est une application linéaire si on a : \[\quantifs{\forall\lambda,\mu\in\K;\forall x,y\in E}u\paren{\lambda x+\mu y}=\lambda u\paren{x}+\mu u\paren{y}.\]

L'ensemble des applications linéaires de \(E\) dans \(F\) est noté \(\L{E}{F}\).
\end{defi}

\begin{defi}[Endomorphisme]
Soit \(E\) un \(\K\)-espace vectoriel.

Un endomorphisme de \(E\) est une application linéaire de \(E\) dans \(E\).

L'ensemble des endomorphismes de \(E\) est noté \(\Lendo{E}\).
\end{defi}

\begin{defi}[Forme linéaire]
Soit \(E\) un \(\K\)-espace vectoriel.

Une forme linéaire sur \(E\) est une application linéaire de \(E\) dans \(\K\).

L'ensemble des formes linéaires sur \(E\) est noté \(E\etoile\) et est appelé le dual de \(E\).
\end{defi}

\begin{ex}
L'application \[\fonctionlambda{\contm[\intervii{a}{b}][\R]}{\R}{f}{\int_{\intervii{a}{b}}f}\] est une forme linéaire.
\end{ex}

\begin{rem}
Si \(E\) et \(F\) sont des \(\K\)-espaces vectoriels alors on a \[\Lendo{E}=\L{E}{E}\qquad\text{et}\qquad E\etoile=\L{E}{\K}.\]
\end{rem}

\begin{ex}[Homothéties]
Soient \(E\) un \(\K\)-espace vectoriel et \(\lambda\in\K\).

On appelle homothétie de rapport \(\lambda\) l'endomorphisme \[\fonctionlambda{E}{E}{x}{\lambda\cdot x}\]

Cet endomorphisme est aussi noté \(\lambda\id{E}\).

L'endomorphisme nul est une homothétie (de rapport \(0\)).

L'application \(\id{E}\) est une homothétie (de rapport \(1\)).
\end{ex}

\begin{ex}
Autres exemples d'endomorphismes : les opérateurs de dérivation \[\fonction{D}{\poly}{\poly}{P}{P\prim}\qquad\text{et}\qquad\fonction{D}{\ensclasse{\infty}{\R}{\R}}{\ensclasse{\infty}{\R}{\R}}{f}{f\prim}\]
\end{ex}

\subsection{Opérations sur les applications linéaires}

\subsubsection{Cas des applications linéaires}

\begin{prop}
Une composée d'applications linéaires est linéaire.
\end{prop}

\begin{dem}
Soient \(E,F,G\) des \(\K\)-espaces vectoriels et \(u\in\L{E}{F}\) et \(v\in\L{F}{G}\).

Montrons que \(v\rond u\in\L{E}{G}\).

Soient \(\lambda,\mu\in\K\) et \(x,y\in E\).

On a \[\begin{aligned}
v\rond u\paren{\lambda x+\mu y}&=v\paren{\lambda u\paren{x}+\mu u\paren{y}}\qquad\text{car }u\text{ est linéaire} \\
&=\lambda v\rond u\paren{x}+\mu v\rond u\paren{y}\qquad\text{car }v\text{ est linéaire}
\end{aligned}\]

Donc \(v\rond u\) est linéaire.
\end{dem}

\begin{nota}
Soient \(E,F,G\) des \(\K\)-espaces vectoriels et \(u\in\L{E}{F}\) et \(v\in\L{F}{G}\).

La composée de \(u\) et \(v\) est souvent notée \(vu\) au lieu de \(v\rond u\).
\end{nota}

\begin{prop}
Soient \(E\) et \(F\) deux \(\K\)-espaces vectoriels.

L'ensemble des applications linéaires de \(E\) dans \(F\) est naturellement muni d'une structure de \(\K\)-espace vectoriel \(\corps{\L{E}{F}}[+][\cdot]\) dont le vecteur nul (noté \(0\) ou \(0_{\L{E}{F}}\)) est l'application identiquement nulle (\ie en tout point) de \(E\) dans \(F\).

En particulier, \(\Lendo{E}\) et \(E\etoile\) sont des \(\K\)-espaces vectoriels.
\end{prop}

\begin{dem}
Montrons que \(\L{E}{F}\) est un sous-espace vectoriel de \(\corps{\F{E}{F}}[+][\cdot]\).

Montrons que \(0_{\F{E}{F}}\in\L{E}{F}\).

On a \[\begin{aligned}
\quantifs{\forall\lambda,\mu\in\K;\forall x,y\in E}0_{\F{E}{F}}\paren{\lambda x+\mu y}&=0_F \\
&=\lambda0_F+\mu0_F \\
&=\lambda0_{\F{E}{F}}\paren{x}+\mu0_{\F{E}{F}}\paren{y}.
\end{aligned}\]

Donc \(0_{\F{E}{F}}\in\L{E}{F}\).

Soient \(\lambda,\mu\in\K\) et \(u,v\in\L{E}{F}\).

Montrons que \(\lambda u+\mu v\in\L{E}{F}\).

Soient \(\omega_1,\omega_2\in\K\) et \(x_1,x_2\in E\).

On a \[\begin{aligned}
\paren{\lambda u+\mu v}\paren{\omega_1x_1+\omega_2x_2}&=\lambda u\paren{\omega_1x_1+\omega_2x_2}+\mu v\paren{\omega_1x_1+\omega_2x_2} \\
&=\lambda\paren{\omega_1u\paren{x_1}+\omega_2u\paren{x_2}}+\mu\paren{\omega_1v\paren{x_1}+\omega_2v\paren{x_2}} \\
&=\omega_1\paren{\lambda u\paren{x_1}+\mu v\paren{x_1}}+\omega_2\paren{\lambda v\paren{x_2}+\mu v\paren{x_2}} \\
&=\omega_1\paren{\lambda u+\mu v}\paren{x_1}+\omega_2\paren{\lambda u+\mu v}\paren{x_2}.
\end{aligned}\]

Donc \(\lambda u+\mu v\in\L{E}{F}\).

Donc \(\L{E}{F}\) est un sous-espace vectoriel de \(\F{E}{F}\).

Donc \(\L{E}{F}\) est un \(\K\)-espace vectoriel.
\end{dem}

\subsubsection{Cas des endomorphismes}

\begin{nota}
Soient \(E\) un espace vectoriel, \(u\in\Lendo{E}\) et \(k\in\N\).

L'itéré \(k\)-ème de \(u\) est souvent noté \(u^k\) (au lieu de \(u^{\rond k}\)).
\end{nota}

\begin{prop}
Soit \(E\) un \(\K\)-espace vectoriel.

L'ensemble des endomorphismes de \(E\) est naturellement muni d'une structure d'anneau \(\corps{\Lendo{E}}[+][\rond]\).

Ses éléments neutres sont l'application identiquement nulle \(0=0_{\Lendo{E}}\) pour \(+\) et l'application identité \(\id{E}\) pour \(\rond\).

En effet : \[\quantifs{\forall u\in\Lendo{E}}u\rond0=0\rond u=0\qquad\text{et}\qquad u\rond\id{E}=\id{E}\rond u=u.\]

L'anneau \(\Lendo{E}\) n'est pas commutatif en général.
\end{prop}

\begin{dem}
On sait que \(\anneau{\Lendo{E}}[+][\cdot]\) est un \(\K\)-espace vectoriel.

Donc \(\groupe{\Lendo{E}}\) est un groupe abélien.

La loi \(\rond\) est associative et elle admet \(\id{E}\) comme élément neutre.

Montrons que \(\rond\) est distributive par rapport à \(+\).

Soient \(u,v,w\in\Lendo{E}\).

On a \(\quantifs{\forall x\in E}\paren{u+v}w\paren{x}=uw\paren{x}+vw\paren{x}=\paren{uw+vw}\paren{x}\).

Donc \(\paren{u+v}w=uw+vw\).

De plus : \[\begin{WithArrows}
\quantifs{\forall x\in E}u\paren{v+w}\paren{x}&=u\paren{v\paren{x}+w\paren{x}} \Arrow{car \(u\) est linéaire} \\
&=uv\paren{x}+uw\paren{x} \\
&=\paren{uv+uw}\paren{x}.
\end{WithArrows}\]

Donc \(u\paren{v+w}=uv+uw\).

Donc \(\rond\) est distributive par rapport à \(+\).

Donc \(\anneau{\Lendo{E}}[+][\rond]\) est un anneau.
\end{dem}

\begin{rem}
\(\anneau{\F{E}{E}}[+][\rond]\) n'est pas un anneau.
\end{rem}

\begin{prop}
Soient \(E\) un \(\K\)-espace vectoriel et \(u,v\in\Lendo{E}\) deux endomorphismes qui commutent : \(uv=vu\).

On a la formule du binôme de Newton : \[\quantifs{\forall m\in\N}\paren{u+v}^m=\sum_{k=0}^m\binom{k}{m}u^kv^{m-k}\] et \[\quantifs{\forall m\in\N}u^m-v^m=\paren{u-v}\sum_{k=0}^{m-1}u^kv^{m-1-k}.\]
\end{prop}

\subsection{Applications linéaires et sous-espaces vectoriels}

\begin{prop}
Soient \(E\) et \(F\) deux \(\K\)-espaces vectoriels, \(u\in\L{E}{F}\) et \(x_1,\dots,x_n\in E\).

On a \[u\paren{\Vect{x_1,\dots,x_n}}=\Vect{u\paren{x_1},\dots,u\paren{x_n}}.\]
\end{prop}

\begin{dem}
On a \[\begin{WithArrows}
u\paren{\Vect{x_1,\dots,x_n}}&=u\paren{\accol{\lambda_1x_1+\dots+\lambda_nx_n}_{\lambda_1,\dots,\lambda_n\in\K}} \\
&=\accol{u\paren{\lambda_1x_1+\dots+\lambda_nx_n}}_{\lambda_1,\dots,\lambda_n\in\K} \Arrow{car \(u\) est linéaire} \\
&=\accol{\lambda_1u\paren{x_1}+\dots+\lambda_nu\paren{x_n}}_{\lambda_1,\dots,\lambda_n\in\K} \\
&=\Vect{u\paren{x_1},\dots,u\paren{x_n}}.
\end{WithArrows}\]
\end{dem}

\begin{ex}
Considérons l'endomorphisme \(\fonction{D}{\poly}{\poly}{P}{P\prim}\)

Soit \(n\in\Ns\).

Alors \[D\paren{\polydeg{n}}=\polydeg{n-1}.\]
\end{ex}

\begin{dem}
On a \[\polydeg{n}=\accol{a_nX^n+\dots+a_0X^0}_{a_0,\dots,a_n\in\K}=\Vect{X^n,\dots,X^0}.\]

En particulier, \(\polydeg{n}\) est un sous-espace vectoriel de \(\poly\).

On a \[\begin{WithArrows}
D\paren{\polydeg{n}}&=D\paren{\Vect{X^n,\dots,X^0}} \Arrow{car \(D\) est linéaire} \\
&=\Vect{D\paren{X^n},\dots,D\paren{X^0}} \\
&=\Vect{nX^{n-1},\dots,4X^3,3X^2,2X,1,0} \\
&=\Vect{X^{n-1},\dots,X^3,X^2,X,1} \\
&=\polydeg{n-1}.
\end{WithArrows}\]
\end{dem}

\begin{prop}\thlabel{prop:imageEtImageRéciproqueD'UnSousEVParUneApplicationLinéaireSontDesSousEVs}
Soient \(E\) et \(F\) deux \(\K\)-espaces vectoriels, \(u\in\L{E}{F}\), \(E_1\) un sous-espace vectoriel de \(E\) et \(F_1\) un sous-espace vectoriel de \(F\).

Alors \[u\inv\paren{F_1}\text{ est un sous-espace vectoriel de }E\] et \[u\paren{E_1}\text{ est un sous-espace vectoriel de }F.\]
\end{prop}

\begin{dem}
Montrons que \(u\inv\paren{F_1}\) est un sous-espace vectoriel de \(E\).

On a \(u\inv\paren{F_1}\subset E\).

On a \(u\paren{0_E}=0_F\) car \(u\) est linéaire donc \(u\paren{0_E}\in F_1\) donc \(0_E\in u\inv\paren{F_1}\).

Montrons que \(u\inv\paren{F_1}\) est stable par combinaison linéaire.

Soient \(\lambda,\mu\in\K\) et \(x,y\in u\inv\paren{F_1}\).

Montrons que \(\lambda x+\mu y\in u\inv\paren{F_1}\).

Comme \(u\) est linéaire, on a \[u\paren{\lambda x+\mu y}=\underbrace{\lambda u\paren{x}}_{\substack{\in F_1\text{ car} \\ x\in u\inv\paren{F_1}}}+\underbrace{\mu u\paren{y}}_{\substack{\in F_1\text{ car} \\ y\in u\inv\paren{F_1}}}.\]

Donc \(u\paren{\lambda x+\mu y}\in F_1\) car \(F_1\) est un sous-espace vectoriel de \(F\).

Donc \(\lambda u\paren{x}+\mu u\paren{y}\in u\inv\paren{F_1}\).

Donc \(u\inv\paren{F_1}\) est un sous-espace vectoriel de \(E\).

Montrons que \(u\paren{E_1}\) est un sous-espace vectoriel de \(F\).

On a \(u\paren{E_1}\subset F\).

On a \(0_F=u\paren{0_E}\) et \(0_E\in E_1\) donc \(0_F\in u\paren{E_1}\).

Montrons que \(u\paren{E_1}\) est stable par combinaison linéaire.

Soient \(\lambda_1,\lambda_2\in\K\) et \(y_1,y_2\in u\paren{E_1}\).

Montrons que \(\lambda_1y_1+\lambda_2y_2\in u\paren{E_1}\).

Soient \(x_1,x_2\in E_1\) tels que \(\begin{dcases}y_1=u\paren{x_1} \\ y_2=u\paren{x_2}\end{dcases}\)

On a \[\begin{WithArrows}
\lambda_1y_1+\lambda_2y_2&=\lambda_1u\paren{x_1}+\lambda_2u\paren{x_2} \Arrow{car \(u\) est linéaire} \\
&=u\paren{\lambda_1x_1+\lambda_2x_2}
\end{WithArrows}\] et \(\lambda_1x_1+\lambda_2x_2\in E_1\) car \(E_1\) est un sous-espace vectoriel de \(E\).

Donc \(\lambda_1y_1+\lambda_2y_2\in u\paren{E_1}\).

Donc \(u\paren{E_1}\) est un sous-espace vectoriel de \(F\).
\end{dem}

\begin{defi}[Noyau et image d'une application linéaire]
Soient \(E\) et \(F\) deux \(\K\)-espaces vectoriels et \(u\in\L{E}{F}\).

On note \(0_F\) l'élément neutre de \(F\).

L'image de \(u\) est l'ensemble image de l'application \(u\) : \[\Im u=\accol{u\paren{x}}_{x\in E}=\accol{y\in F\tq\quantifs{\exists x\in E}u\paren{x}=y}=u\paren{E}.\]

Le noyau de \(u\) est l'ensemble des vecteurs de \(E\) d'image nulle : \[\ker u=u\inv\paren{\accol{0_F}}=\accol{x\in E\tq u\paren{x}=0_F}.\]
\end{defi}

\begin{prop}
Soient \(E\) et \(F\) deux \(\K\)-espaces vectoriels et \(u\in\L{E}{F}\).

Alors \(\ker u\) est un sous-espace vectoriel de \(E\) et \(\Im u\) est un sous-espace vectoriel de \(F\).
\end{prop}

\begin{dem}
\(\ker u\) est l'image réciproque de \(\accol{0_F}\) par \(u\).

Or \(u\) est linéaire et \(\accol{0_F}\) est un sous-espace vectoriel de \(F\).

Donc \(\ker u\) est un sous-espace vectoriel de \(E\) selon la \thref{prop:imageEtImageRéciproqueD'UnSousEVParUneApplicationLinéaireSontDesSousEVs}.

De même, \(\Im u=u\paren{E}\) est l'image directe de \(E\) par \(u\).

Or \(u\) est linéaire et \(E\) est un sous-espace vectoriel de \(E\).

Donc \(\Im u\) est un sous-espace vectoriel de \(F\) selon la \thref{prop:imageEtImageRéciproqueD'UnSousEVParUneApplicationLinéaireSontDesSousEVs}.
\end{dem}

\begin{theo}
Soient \(E\) et \(F\) deux \(\K\)-espaces vectoriels et \(u\in\L{E}{F}\).

On note \(0_E\) l'élément neutre de \(E\).

Alors \(u\) est injective si, et seulement si, son noyau est \guillemets{nul}, \cad \[u\text{ est injective}\ssi\ker u=\accol{0_E}.\]
\end{theo}

\begin{dem}[Découle du théorème analogue pour les groupes (\cf \thref{theo:morphismeDeGroupeInjectifSsiNoyauNul})]
\impdir

Supposons \(u\) injective.

Alors \(0_F\) n'admet d'autre antécédent que \(0_E\) par \(u\).

Donc \(\ker u=\accol{0_E}\).

\imprec

Supposons \(\ker u=\accol{0_E}\).

Montrons que \(u\) est injective.

Soient \(x,y\in E\) tels que \(u\paren{x}=u\paren{y}\).

On a \(u\paren{x}-u\paren{y}=0_F\) donc \(u\paren{x-y}=0_F\) car \(u\) est linéaire.

Donc \(x-y\in\ker u\).

Donc \(x-y=0_E\) donc \(x=y\).

Donc \(u\) est injective.
\end{dem}

\subsection{Projecteurs}

\begin{defprop}
Soient \(E\) un \(\K\)-espace vectoriel et \(F\) et \(G\) deux sous-espaces vectoriels supplémentaires dans \(E\).

Le projecteur \(p_F\) sur \(F\) parallèlement à \(G\) et le projecteur \(p_G\) sur \(G\) parallèlement à \(F\) sont les endomorphismes de \(E\) caractérisés par : \[\quantifs{\forall x\in E}\begin{dcases}
x=p_F\paren{x}+p_G\paren{x} \\
p_F\paren{x}\in F \\
p_G\paren{x}\in G
\end{dcases}\]

On a \[\Im p_F=F\qquad\text{et}\qquad\ker p_F=G.\]
\end{defprop}

\begin{dem}[\(\Im p_F\)]
On a, par définition de \(p_F\) : \(\Im p_F\subset F\).

Montrons que \(\Im p_F\supset F\).

Soit \(x\in F\).

Montrons que \(x\in\Im p_F=F\).

On a \(x=\underbrace{x}_{\in F}+\underbrace{0_E}_{\in G}\) donc \(p_F\paren{x}=x\).

Donc \(x\in\Im p_F\).

D'où \(\Im p_F=F\).
\end{dem}

\begin{dem}[\(\ker p_F=G\)]
\increc

Soit \(x\in G\)

On a \(x=\underbrace{0_E}_{\in F}+\underbrace{x}_{\in G}\) donc \(p_F\paren{x}=0_E\).

Donc \(x\in\ker p_F\).

\incdir

Soit \(x\in\ker p_F\).

On a \(x=p_F\paren{x}+p_G\paren{x}=p_G\paren{x}\in G\).

D'où l'égalité.
\end{dem}

\begin{defi}
Soient \(E\) un \(\K\)-espace vectoriel et \(p\in\Lendo{E}\).

On dit que \(p\) est un projecteur s'il existe deux sous-espaces vectoriels \(F\) et \(G\) supplémentaires dans \(E\) tels que \(p\) soit le projecteur sur \(F\) parallèlement à \(G\).
\end{defi}

\begin{ex}[On reprend les couples de supplémentaires de l'\thref{ex:EVsSupplémentaires}]
\begin{enumerate}
\item On a vu \(\C=\R\oplus\i\R\).

Le projecteur sur \(\R\) parallèlement à \(\i\R\) est \[\fonctionlambda{\C}{\C}{z}{\Re z}\]

Le projecteur sur \(\i\R\) parallèlement à \(\R\) est \[\fonctionlambda{\C}{\C}{z}{\i\Im z}\]

\item On a vu \(\F{\R}{\R}=F\oplus G\) où \(F\) (respectivement \(G\)) est l'ensemble des fonctions paires (respectivement impaires) de \(\R\) dans \(\R\).

Les projecteurs \(p_F\) et \(p_G\) associés à cette décomposition de \(\F{\R}{\R}\) en somme directe sont : \[\fonction{p_F}{\F{\R}{\R}}{\F{\R}{\R}}{f}{\fonctionlambda{\R}{\R}{x}{\dfrac{f\paren{x}+f\paren{-x}}{2}}}\] et \[\fonction{p_G}{\F{\R}{\R}}{\F{\R}{\R}}{f}{\fonctionlambda{\R}{\R}{x}{\dfrac{f\paren{x}-f\paren{-x}}{2}}}\]

\item Soient \(n\in\N\) et \(Q\in\poly\) un polynôme de degré \(n+1\).

On a vu \(\poly=\poly Q\oplus\polydeg{n}\).

Le projecteur sur \(\polydeg{n}\) parallèlement à \(\poly Q\) est l'application qui à tout polynôme \(P\in\poly\) associe le reste de sa division euclidienne par \(Q\).
\end{enumerate}
\end{ex}

\begin{prop}
Soient \(E\) un \(\K\)-espace vectoriel et \(p\in\Lendo{E}\).

Alors \(p\) est un projecteur si, et seulement si, \(p^2=p\) (\cad \(p\rond p=p\)).

Précisément, \(p\) est alors le projecteur sur \(\Im p\) parallèlement à \(\ker p\).
\end{prop}

\begin{dem}
\impdir

Supposons que \(p\) est un projecteur.

Soient \(F\) et \(G\) deux sous-espaces vectoriels de \(E\) tels que \(\begin{dcases}E=F\oplus G \\ p\text{ est le projecteur sur }F\text{ parallèlement à }G\end{dcases}\)

Montrons que \(p^2=p\).

Soit \(x\in E\).

On a \(p\paren{x}\in F\) donc \(p\paren{x}=\underbrace{p\paren{x}}_{\in F}+\underbrace{0_E}_{\in G}\).

Donc \(p\paren{p\paren{x}}=p\paren{x}\).

Donc \(p^2=p\).

\imprec

Supposons \(p^2=p\).

Posons \(F=\Im p\) et \(G=\ker p\).

Montrons que \(F\) et \(G\) sont supplémentaires dans \(E\).

Montrons que \(F\inter G=\accol{0_E}\).

Soit \(y\in F\inter G\).

Comme \(y\in F\), il existe \(x\in E\) tel que \(y=p\paren{x}\).

Comme \(y\in G\), on a \(p\paren{y}=0_E\).

Donc \(p\paren{p\paren{x}}=p^2\paren{x}=p\paren{x}=0_E\).

Donc \(y=0_E\).

Montrons que \(F+G=E\).

Soit \(x\in E\).

On remarque \(x=\underbrace{x-p\paren{x}}_{\substack{\in\ker p \\ \text{car }p\paren{x-p\paren{x}}=p\paren{x}-p^2\paren{x}=0_E}}+\underbrace{p\paren{x}}_{\in\Im p}\).

Donc \(F+G=E\).

Finalement, \(F\oplus G=E\).

Notons \(p_F\) le projecteur sur \(F\) parallèlement à \(G\).

Montrons que \(p=p_F\).

Soit \(x\in E\).

On a \(x=\underbrace{p\paren{x}}_{\in F}+\underbrace{x-p\paren{x}}_{\in G}\).

Donc \(p_F\paren{x}=p\paren{x}\) donc \(p=p_F\).

Donc \(p\) est un projecteur.
\end{dem}

\begin{prop}
Soient \(E\) un \(\K\)-espace vectoriel et \(p\) un projecteur de \(E\).

Alors \(\Im p\) est l'ensemble des points fixes de \(p\) : \[\Im p=\accol{x\in E\tq p\paren{x}=x}=\ker\paren{p-\id{E}}.\]
\end{prop}

\begin{dem}
On remarque \[\begin{aligned}
\quantifs{\forall x\in E}x\in\ker\paren{p-\id{E}}&\ssi\paren{p-\id{E}}\paren{x}=0_E \\
&\ssi p\paren{x}-x=0_E \\
&\ssi p\paren{x}=x
\end{aligned}\] donc \(\ker\paren{p-\id{E}}\) est l'ensemble des points fixes de \(p\).

Montrons que \(\Im p=\accol{x\in E\tq p\paren{x}=x}\).

\increc Soit \(x\in E\) tel que \(p\paren{x}=x\). Alors \(x\in\Im p\).

\incdir

Soit \(y\in\Im p\).

Il existe \(x\in E\) tel que \(y=p\paren{x}\).

Donc \(p\paren{y}=p\paren{p\paren{x}}\).

Donc \(p\paren{y}=p^2\paren{x}=p\paren{x}=y\).
\end{dem}

\begin{prop}\thlabel{prop:applicationsLinéairesRestrictionSSEVsSupplémentaires}
Soient \(E\) et \(F\) deux \(\K\)-espaces vectoriels, \(E_1\) et \(E_2\) deux sous-espaces vectoriels supplémentaires dans \(E\) et \(u_1\in\L{E_1}{F}\) et \(u_2\in\L{E_2}{F}\).

Alors il existe une unique application linéaire \(u\in\L{E}{F}\) telle que \[\restr{u}{E_1}=u_1\qquad\text{et}\qquad\restr{u}{E_2}=u_2.\]

Elle est donnée par : \[\quantifs{\forall x\in E}u\paren{x}=u_1\paren{p_1\paren{x}}+u_2\paren{p_2\paren{x}}\] où \(p_1\) (respectivement \(p_2\)) est le projecteur sur \(E_1\) (respectivement \(E_2\)) parallèlement à \(E_2\) (respectivement \(E_1\)).
\end{prop}

\begin{dem}
Déterminons les applications linéaires \(u\in\L{E}{F}\) telles que \(\restr{u}{E_1}=u_1\) et \(\restr{u}{E_2}=u_2\).

\analyse

Soient \(u\in\L{E}{F}\) telle que \(\begin{dcases}\restr{u}{E_1}=u_1 \\ \restr{u}{E_2}=u_2\end{dcases}\) et \(x\in E\).

On a \(x=\underbrace{p_1\paren{x}}_{\in E_1}+\underbrace{p_2\paren{x}}_{\in E_2}\).

Comme \(u\) est linéaire, on a \[\begin{aligned}
u\paren{x}&=u\paren{p_1\paren{x}}+u\paren{p_2\paren{x}} \\
&=u_1\paren{p_1\paren{x}}+u_2\paren{p_2\paren{x}}.
\end{aligned}\]

Donc \(u=u_1p_1+u_2p_2\).

\synthese

Posons \(u=\overbrace{\underbrace{u_1}_{\in\L{E_1}{F}}\underbrace{p_1}_{\in\L{E}{E_1}}}^{\in\L{E}{F}}+\overbrace{\underbrace{u_2}_{\in\L{E_2}{F}}\underbrace{p_2}_{\in\L{E}{E_2}}}^{\in\L{E}{F}}\).

On a \(u\in\L{E}{F}\).

De plus, on a : \[\begin{aligned}
\quantifs{\forall x\in E_1}u\paren{x}&=u_1\paren{p_1\paren{x}}+u_2\paren{p_2\paren{x}} \\
&=u_1\paren{x}+u_2\paren{0_E} \\
&=u_1\paren{x}
\end{aligned}\] donc \(\restr{u}{E_1}=u_1\).

On montre de même \(\restr{u}{E_2}=u_2\).

\conclusion

\(u_1p_1+u_2p_2\) est l'unique \(u\in\L{E}{F}\) telle que \(\restr{u}{E_1}=u_1\) et \(\restr{u}{E_2}=u_2\).
\end{dem}

\begin{rem}
On peut reformuler la proposition précédente (dont on garde les notations) en écrivant que l'application \[\fonctionlambda{\L{E}{F}}{\L{E_1}{F}\times\L{E_2}{F}}{u}{\paren{\restr{u}{E_1},\restr{u}{E_2}}}\] est une bijection.
\end{rem}

\subsection{Symétries}

Dans ce paragraphe, on suppose que le corps \(\K\) considéré est un sous-corps de \(\C\).

\begin{defprop}
Soient \(E\) un \(\K\)-espace vectoriel et \(F\) et \(G\) deux sous-espaces vectoriels supplémentaires dans \(E\).

On note \(p_F\) (respectivement \(p_G\)) le projecteur sur \(F\) (respectivement \(G\)) parallèlement à \(G\) (respectivement \(F\)).

On appelle symétrie par rapport à \(F\) parallèlement à \(G\) l'endomorphisme \(s\in\Lendo{E}\) tel que : \[\begin{dcases}\quantifs{\forall x_F\in F}s\paren{x_F}=x_F \\ \quantifs{\forall x_G\in G}s\paren{x_G}=-x_G\end{dcases}\]

Selon la \thref{prop:applicationsLinéairesRestrictionSSEVsSupplémentaires}, cela définit bien l'endomorphisme \(s\).

En d'autres termes, on a : \[s=p_F-p_G.\]

D'où les relations : \[s=2p_F-\id{E}\qquad\text{et}\qquad p_F=\dfrac{1}{2}\paren{\id{E}+s}.\]

On a enfin : \[F=\ker\paren{s-\id{E}}\qquad\text{et}\qquad G=\ker\paren{s+\id{E}}.\]
\end{defprop}

\begin{dem}
Soient \(x\in E\) et \(x_F\in F\) et \(x_G\in G\) tels que \(x=x_F+x_G\).

On a \(s\paren{x}=x_F-x_G\).

On a \[\begin{aligned}
x\in\ker\paren{s-\id{E}}&\ssi s\paren{x}=x \\
&\ssi x_F-x_G=x_F+x_G \\
&\ssi-x_G=x_G \\
&\ssi x_G=0 \\
&\ssi x\in F.
\end{aligned}\]

Donc \(F=\ker\paren{s-\id{E}}\).

De même, on a \[\begin{aligned}
x\in\ker\paren{s+\id{E}}&\ssi s\paren{x}=-x \\
&\ssi x_F-x_G=-x_F-x_G \\
&\ssi x_F=-x_F \\
&\ssi x_F=0 \\
&\ssi x\in G.
\end{aligned}\]

Donc \(G=\ker\paren{s+\id{E}}\).
\end{dem}

\begin{defi}
Soient \(E\) un \(\K\)-espace vectoriel et \(s\in\Lendo{E}\).

On dit que \(s\) est une symétrie s'il existe deux sous-espaces vectoriels \(F\) et \(G\) supplémentaires dans \(E\) tels que \(s\) soit la symétrie par rapport à \(F\) parallèlement à \(G\).
\end{defi}

\begin{prop}
Soient \(E\) un \(\K\)-espace vectoriel et \(s\in\Lendo{E}\).

Alors \(s\) est un symétrie si, et seulement si, \(s^2=\id{E}\) (\cad \(s\rond s=\id{E}\)).

Précisément, \(s\) est alors la symétrie par rapport à \(\ker\paren{s-\id{E}}\) parallèlement à \(\ker\paren{s+\id{E}}\).
\end{prop}

\begin{dem}
\imprec

Supposons \(s^2=\id{E}\).

Montrons que \(E=\ker\paren{s-\id{E}}\oplus\ker\paren{s+\id{E}}\), \cad \[\quantifs{\forall x\in E;\exists!\paren{x_1,x_{-1}}\in\ker\paren{s-\id{E}}\times\ker\paren{s+\id{E}}}x=x_1+x_{-1}.\]

Soit \(x\in E\).

\analyse

Soit \(\paren{x_1,x_{-1}}\in\ker\paren{s-\id{E}}\times\ker\paren{s+\id{E}}\) tel que \(x=x_1+x_{-1}\).

On a \(x_1\in\ker\paren{s-\id{E}}\) donc \(s\paren{x_1}=x_1\).

On a \(x_{-1}\in\ker\paren{s+\id{E}}\) donc \(s\paren{x_{-1}}=-x_{-1}\).

Or \(x=x_1+x_{-1}\) donc \(s\paren{x}=s\paren{x_1}+s\paren{x_{-1}}=x_1-x_{-1}\).

Donc \[x_1=\dfrac{x+s\paren{x}}{2}\qquad\text{et}\qquad x_{-1}=\dfrac{x-s\paren{x}}{2}.\]

\synthese

Posons \(x_1=\dfrac{x+s\paren{x}}{2}\) et \(x_{-1}=\dfrac{x-s\paren{x}}{2}\).

On a \(s\paren{x_1}=\dfrac{s\paren{x}+s^2\paren{x}}{2}=\dfrac{s\paren{x}+x}{2}=x_1\).

On a \(s\paren{x_{-1}}=\dfrac{s\paren{x}-s^2\paren{x}}{2}=\dfrac{s\paren{x}-x}{2}=-x_{-1}\).

On a \(x_1+x_{-1}=\dfrac{x+s\paren{x}}{2}+\dfrac{x-s\paren{x}}{2}=\dfrac{2x}{2}=x\).

\conclusion D'où l'existence et l'unicité de \(x_1\) et \(x_{-1}\).

Montrons que \(s\) est une symétrie par rapport à \(\ker\paren{s-\id{E}}\) parallèlement à \(\ker\paren{s+\id{E}}\).

On a \[\begin{aligned}
\quantifs{\forall x_1\in\ker\paren{s-\id{E}};\forall x_{-1}\in\ker\paren{s+\id{E}}}s\paren{x_1+x_{-1}}&=s\paren{x_1}+s\paren{x_{-1}} \\
&=x_1-x_{-1}.
\end{aligned}\]

D'où le résultat.

\impdir

Supposons que \(s\) est une symétrie.

Alors il existe \(F\) et \(G\) deux sous-espaces vectoriels de \(E\) tels que \(E=F\oplus G\) et \(s\) est la symétrie par rapport à \(F\) parallèlement à \(G\) : \[\quantifs{\forall x_F\in F;\forall x_G\in G}s\paren{x_F+x_G}=x_F-x_G.\]

On remarque : \[\quantifs{\forall x_F\in F;\forall x_G\in G}s^2\paren{x_F+x_G}=s\paren{x_F-x_G}=x_F+x_G.\]

Donc \(s^2=\id{E}\).
\end{dem}

\begin{ex}
L'application suivante est une symétrie : \[\fonctionlambda{\F{\R}{\R}}{\F{\R}{\R}}{f}{\fonctionlambda{\R}{\R}{x}{f\paren{-x}}}\]
\end{ex}

\begin{rem}
Soient \(E\) un \(\K\)-espace vectoriel et \(s\) une symétrie de \(E\).

On a \(s^2=\id{E}\) donc \(s\) est surjective et injective donc \(\begin{dcases}\ker s=\accol{0_E} \\ \Im s=E\end{dcases}\)
\end{rem}

\subsection{Isomorphismes, automorphismes}

\begin{defi}[Isomorphisme]
Soient \(E\) et \(F\) deux \(\K\)-espaces vectoriels.

On appelle isomorphisme (de \(\K\)-espaces vectoriels) de \(E\) vers \(F\) toute application linéaire \(u:E\to F\) qui est une bijection de \(E\) vers \(F\).

Lorsqu'il existe un isomorphisme de \(E\) vers \(F\), on dit que \(E\) et \(F\) sont des espaces vectoriels isomorphes.
\end{defi}

\begin{prop}
Soient \(E\), \(F\) et \(G\) des \(\K\)-espaces vectoriels et \(u:E\to F\) et \(v:F\to G\) deux isomorphismes de \(\K\)-espaces vectoriels.

Alors \(vu:E\to G\) est un isomorphisme de \(\K\)-espaces vectoriels.
\end{prop}

\begin{dem}
\note{Exercice}
\end{dem}

\begin{prop}\thlabel{prop:isomorphismeD'EVsImpliqueBijectionRéciproqueIsomorphismeD'EVs}
Si \(u:E\to F\) est un isomorphisme de \(\K\)-espaces vectoriels alors sa bijection réciproque \(u\inv:F\to E\) est aussi un isomorphisme de \(\K\)-espaces vectoriels.
\end{prop}

\begin{dem}
On a \(u\inv:F\to E\).

Montrons que \(u\inv\) est linéaire.

Soient \(\lambda_1,\lambda_2\in\K\) et \(y_1,y_2\in F\).

Posons \(\begin{dcases}
x_1=u\inv\paren{y_1} \\
x_2=u\inv\paren{y_2}
\end{dcases}\)

On a \(\begin{dcases}
y_1=u\paren{x_1} \\
y_2=u\paren{x_2}
\end{dcases}\)

Donc comme \(u\) est linéaire : \(\lambda_1y_1+\lambda_2y_2=u\paren{\lambda_1x_1+\lambda_2x_2}\).

Donc \[u\inv\paren{\lambda_1y_1+\lambda_2y_2}=\lambda_1x_1+\lambda_2x_2=\lambda_1u\inv\paren{y_1}+\lambda_2u\inv\paren{y_2}.\]

Donc \(u\inv\) est linéaire.
\end{dem}

\begin{defi}[Automorphisme]
Soit \(E\) un \(\K\)-espace vectoriel.

Un automorphisme de \(\K\)-espace vectoriel de \(E\) est un isomorphisme \(u:E\to E\) de \(\K\)-espaces vectoriels de \(E\) vers \(E\), \cad un endomorphisme de \(\K\)-espace vectoriel de \(E\) bijectif.

L'ensemble des automorphismes de \(\K\)-espace vectoriel de \(E\) est appelé le groupe linéaire de \(E\) et est noté \(\GL{}[E]\).
\end{defi}

\begin{prop}
Soit \(E\) un \(\K\)-espace vectoriel.

Alors \(\groupe{\GL{}[E]}[\rond]\) est un groupe.

Son élément neutre est \(\id{E}\).

L'inverse \(u\inv\) d'un élément \(u\in\GL{}[E]\) est sa bijection réciproque.

Rappel : on a \[\quantifs{\forall u,v\in\GL{}[E]}\paren{u\rond v}\inv=v\inv\rond u\inv.\]
\end{prop}

\begin{dem}
On sait que \(\groupe{S_E}[\rond]\) est un groupe (où \(S_E\) est l'ensemble des bijections de \(E\) dans \(E\)) et que dans ce groupe, l'élément neutre est \(\id{E}\) et l'inverse d'un élément \(f\) est sa bijection réciproque \(f\inv\).

Montrons que \(\GL{}[E]\) est un sous-groupe de \(S_E\).

On a \(\GL{}[E]\subset S_E\).

On a \(\id{E}\in\GL{}[E]\).

Soient \(u,v\in\GL{}[E]\).

Alors \(u\rond v\in\GL{}[E]\) (\(u\rond v\) est une bijection car \(u\) et \(v\) sont des bijections et \(u\rond v\) est linéaire car \(u\) et \(v\) sont linéaires).

De plus, \(u\inv\in\GL{}[E]\) selon la \thref{prop:isomorphismeD'EVsImpliqueBijectionRéciproqueIsomorphismeD'EVs}.

Donc \(\GL{}[E]\) est un sous-groupe de \(S_E\).

Donc \(\GL{}[E]\) est un groupe.
\end{dem}

\section{Familles de vecteurs}

\subsection{Familles libres}

\begin{defi}
Soient \(E\) un \(\K\)-espace vectoriel, un entier \(p\in\Ns\) et une famille de vecteurs \(\fami{F}=\paren{x_1,\dots,x_p}\in E^p\).

On dit que \(\fami{F}\) est une famille libre (sur \(\K\)) ou que les vecteurs \(x_1,\dots,x_p\) sont linéairement indépendants (sur \(\K\)) si : \[\quantifs{\forall\lambda_1,\dots,\lambda_p\in\K}\lambda_1x_1+\dots+\lambda_px_p=0_E\imp\paren{\lambda_1,\dots,\lambda_p}=\paren{0,\dots,0}.\]

Si la famille \(\fami{F}\) n'est pas libre, on dit qu'elle est liée.
\end{defi}

\begin{ex}
Soient \(E\) un \(\K\)-espace vectoriel et \(x\in E\).

Alors la famille \(\paren{x}\) est liée si, et seulement si, le vecteur \(x\) est nul.
\end{ex}

\begin{dem}
Découle de la \thref{prop:lambdaFoisXNulSsiLambdaNulOuXNul}.
\end{dem}

\begin{exoex}
On pose \[E=\R^2\qquad v_1=\paren{1,0}\qquad v_2=\paren{1,1}\qquad v_3=\paren{0,1}.\]

Montrer :

\begin{enumerate}
\item que la famille \(\paren{v_1,v_2}\in E^2\) est libre. \\

\item que la famille \(\paren{v_1,v_2,v_3}\in E^3\) est liée.
\end{enumerate}
\end{exoex}

\begin{corr}[2]
On a \(v_1-v_2+v_3=\paren{0,0}\) or \(\paren{1,-1,1}\not=\paren{0,0,0}\) donc \(\paren{v_1,v_2,v_3}\) est liée.
\end{corr}

\begin{corr}[1]
Soient \(\lambda_1,\lambda_2\in\R\) tels que \(\lambda_1v_1+\lambda_2v_2=0\).

Montrons que \(\lambda_1=\lambda_2=0\).

On a \(\lambda_1\dcoords{1}{0}+\lambda_2\dcoords{1}{1}=\dcoords{0}{0}\) donc \(\dcoords{\lambda_1+\lambda_2}{\lambda_2}=\dcoords{0}{0}\).

Donc \(\begin{dcases}
\lambda_1+\lambda_2=0 \\
\lambda_2=0
\end{dcases}\) donc \(\lambda_2=0\) et \(\lambda_1=0\).

Donc \(\paren{v_1,v_2}\) est libre.
\end{corr}

\begin{rem}
Soient \(E\) un \(\K\)-espace vectoriel, \(p\in\Ns\) et \(\fami{F}=\paren{x_1,\dots,x_p}\in E^p\).

\begin{enumerate}
\item Le fait que la famille \(\fami{F}\) soit libre ne dépend pas de l'ordre de ses vecteurs : \[\quantifs{\forall\sigma\in S_p}\paren{x_{\sigma\paren{1}},\dots,x_{\sigma\paren{p}}}\text{ est libre}\ssi\paren{x_1,\dots,x_p}\text{ est libre}.\]

\item Si la famille \(\fami{F}\) contient le vecteur nul, alors elle est liée. \\

\item Pour que \(\fami{F}\) soit libre, il faut que ses vecteurs soient deux à deux distincts. \\

\item Soit \(\fami{G}=\paren{x_1,\dots,x_p,\dots,x_m}\in E^m\) une famille contenant \(\fami{F}\) (avec \(m\geq p\)).

On a l'implication \[\fami{F}\text{ est liée}\imp\fami{G}\text{ est liée}.\]

On a donc aussi, en contraposant : \[\fami{G}\text{ est libre}\imp\fami{F}\text{ est libre}.\]
\end{enumerate}
\end{rem}

\begin{dem}[2]
Soit \(k\in\interventierii{1}{p}\) tel que \(x_k=0_E\).

On a \[0_\K\cdot x_1+\dots+0_\K\cdot x_{k-1}+1_\K\cdot x_k+0_\K\cdot x_{k+1}+\dots+0_\K\cdot x_p=0_E.\]

Donc \(\paren{x_1,\dots,x_p}\) est liée.
\end{dem}

\begin{dem}[3]
Soient \(k,l\in\interventierii{1}{p}\) tels que \(k<l\) et \(x_k=x_l\).

On a \[0x_1+\dots+0x_{k-1}+x_k+0x_{k+1}+\dots+0x_{l-1}-x_l+0x_{l+1}+\dots+0x_p=0.\]

Donc \(\paren{x_1,\dots,x_p}\) est liée.
\end{dem}

\begin{dem}[4]
Supposons que \(\fami{F}\) est liée.

Soient \(\lambda_1,\dots,\lambda_p\in\K\) tels que \(\begin{dcases}
\lambda_1x_1+\dots+\lambda_px_p=0 \\
\paren{\lambda_1,\dots,\lambda_p}\not=\paren{0,\dots,0}
\end{dcases}\)

Posons \(\quantifs{\forall k\in\interventierii{p+1}{m}}\lambda_k=0\).

On a \(\begin{dcases}
\lambda_1x_1+\dots+\lambda_mx_m=0 \\
\paren{\lambda_1,\dots,\lambda_m}\not=\paren{0,\dots,0}
\end{dcases}\)

Donc \(\fami{G}\) est liée.
\end{dem}

\begin{prop}\thlabel{prop:familleLiéeSsiUnVecteurEstCombinaisonLinéaireDesAutres}
Une famille est liée si, et seulement si, l'un de ses vecteurs est combinaison linéaire de ses autres vecteurs.
\end{prop}

\begin{dem}
Soient \(E\) un \(\K\)-espace vectoriel et \(\paren{x_1,\dots,x_p}\in E^p\).

\impdir

Supposons que \(\paren{x_1,\dots,x_p}\) est liée.

Soient \(\lambda_1,\dots,\lambda_p\in\K\) tels que \(\begin{dcases}
\lambda_1x_1+\dots+\lambda_px_p=0_E \\
\paren{\lambda_1,\dots,\lambda_p}\not=\paren{0,\dots,0}
\end{dcases}\)

Soit \(k\in\interventierii{1}{p}\) tel que \(\lambda_k\not=0\).

On remarque : \[x_k=\sum_{j\in\interventierii{1}{p}\excluant\accol{k}}\dfrac{-\lambda_j}{\lambda_k}x_j.\]

Donc \(x_k\in\Vect{x_1,\dots,x_{k-1},x_{k+1},\dots,x_p}\).

\imprec

Supposons que l'un des vecteurs de \(\paren{x_1,\dots,x_p}\) est combinaison linéaire des autres.

Montrons que \(\paren{x_1,\dots,x_p}\) est liée.

Quitte à permuter les vecteurs de la famille, on peut supposer \(x_1\in\Vect{x_2,\dots,x_p}\).

Soient \(\mu_2,\dots,\mu_p\in\K\) tels que \(x_1=\mu_2x_2+\dots+\mu_px_p\).

On remarque \(\begin{dcases}
x_1-\mu_2x_2-\dots-\mu_px_p=0_E \\
\paren{1,-\mu_2,\dots,-\mu_p}\not=\paren{0,\dots,0}
\end{dcases}\)

Donc \(\paren{x_1,\dots,x_p}\) est liée.
\end{dem}

\begin{defi}
Soient \(E\) un \(\K\)-espace vectoriel et \(x,y\in E\).

On dit que les vecteurs \(x\) et \(y\) sont colinéaires si la famille \(\paren{x,y}\) est liée, \cad si \[\quantifs{\exists\lambda\in\K}y=\lambda x\qquad\text{ou}\qquad\quantifs{\exists\lambda\in\K}x=\lambda y.\]
\end{defi}

\begin{ex}\thlabel{ex:familleDePolynômesADegrésDeuxADeuxDistinctsEstLibre}
Soit une famille de polynômes non-nuls \(\fami{F}=\paren{P_1,\dots,P_r}\in\poly^r\) de degrés deux à deux distincts, \cad telle que : \[\quantifs{\forall i,j\in\interventierii{1}{r}}i\not=j\imp\deg P_i\not=\deg P_j.\]

Alors \(\fami{F}\) est libre.
\end{ex}

\begin{dem}~\\
Posons \(\quantifs{\forall i\in\interventierii{1}{r}}\begin{dcases}
\mu_i\text{ le coefficient dominant de }P_i \\
d_i=\deg P_i\in\N
\end{dcases}\)

Quitte à permuter les vecteurs de la famille, on peut supposer \(d_1>d_2>\dots>d_r\).

Montrons que \(\paren{P_1,\dots,P_r}\) est libre.

Soient \(\lambda_1,\dots,\lambda_r\in\K\) tels que \(\lambda_1P_1+\dots+\lambda_rP_r=0\).

Montrons que \(\lambda_1=\dots=\lambda_r=0\).

Le coefficient de degré \(d_1\) de \(\lambda_1P_1+\dots+\lambda_rP_r\) est \(\lambda_1\mu_1\).

Donc \(\lambda_1\mu_1=0\) or \(\mu_1\not=0\) donc \(\lambda_1=0\).

Donc \(\lambda_2P_2+\dots+\lambda_rP_r=0\).

On continue de même (récurrence finie) : \(\lambda_2=0\) puis \(\lambda_3=0\), etc...

Donc \(\paren{P_1,\dots,P_r}\) est libre.
\end{dem}

\begin{ex}
\(\paren{X^5-X^3+1,X^2,X^4-X,5}\) est une famille libre de \(\poly\) car c'est une famille de polynômes non-nuls de degrés deux à deux distincts.
\end{ex}

\begin{prop}
Soient \(E\) un \(\K\)-espace vectoriel, \(\paren{x_1,\dots,x_n}\in E^n\) une famille libre de \(E\) et \(x_{n+1}\in E\).

Alors \(\paren{x_1,\dots,x_{n+1}}\) est une famille libre si, et seulement si, \(x_{n+1}\not\in\Vect{x_1,\dots,x_n}\).
\end{prop}

\begin{dem}
Par contraposée, montrons que \(\paren{x_1,\dots,x_{n+1}}\text{ est liée}\ssi x_{n+1}\in\Vect{x_1,\dots,x_n}\).

\imprec Vraie selon la \thref{prop:familleLiéeSsiUnVecteurEstCombinaisonLinéaireDesAutres}.

\impdir

Supposons que \(\paren{x_1,\dots,x_{n+1}}\) est liée.

Soient \(\lambda_1,\dots,\lambda_{n+1}\in\K\) tels que \(\begin{dcases}
\lambda_1x_1+\dots+\lambda_{n+1}x_{n+1}=0_E \\
\paren{\lambda_1,\dots,\lambda_{n+1}}\not=\paren{0,\dots,0}
\end{dcases}\)

Montrons que \(\lambda_{n+1}\not=0\).

Par l'absurde, supposons \(\lambda_{n+1}=0\).

On a \(\begin{dcases}
\lambda_1x_1+\dots+\lambda_nx_n=0 \\
\paren{\lambda_1,\dots,\lambda_n}\not=\paren{0,\dots,0}
\end{dcases}\)

Donc \(\paren{x_1,\dots,x_n}\) est liée : contradiction.

Donc \(\lambda_{n+1}\not=0\).

Donc \(x_{n+1}=\dfrac{-\lambda_1}{\lambda_{n+1}}x_1+\dots+\dfrac{-\lambda_n}{\lambda_{n+1}}x_n\).

Donc \(x_{n+1}\in\Vect{x_1,\dots,x_n}\).
\end{dem}

\subsection{Familles génératrices}

\begin{defi}
Soient \(E\) un \(\K\)-espace vectoriel et \(\paren{x_1,\dots,x_p}\in E^p\) une famille d'élément de \(E\).

On dit que \(\paren{x_1,\dots,x_p}\) est une famille génératrice de \(E\) ou que la famille \(\paren{x_1,\dots,x_p}\) engendre \(E\) si tout vecteur de \(E\) est combinaison linéaire de ses vecteurs, \cad si \[E=\Vect{x_1,\dots,x_p},\] \cad : \[\quantifs{\forall x\in E;\exists\lambda_1,\dots,\lambda_p\in\K}x=\lambda_1x_1+\dots+\lambda_px_p.\]
\end{defi}

\begin{rem}
Soient \(E\) un \(\K\)-espace vectoriel, \(p\in\Ns\) et \(\fami{F}=\paren{x_1,\dots,x_p}\in E^p\).

\begin{enumerate}
\item Le fait que \(\fami{F}\) soit une famille génératrice de \(E\) ne dépend pas de l'ordre de ses vecteurs : \[\quantifs{\forall\sigma\in S_p}\paren{x_{\sigma\paren{1}},\dots,x_{\sigma\paren{p}}}\text{ engendre }E\ssi\paren{x_1,\dots,x_p}\text{ engendre }E.\]

\item Soit \(\fami{G}=\paren{x_1,\dots,x_p,\dots,x_m}\in E^m\) une famille contenant \(\fami{F}\) (avec \(m\geq p\)).

On a l'implication \[\fami{F}\text{ est génératrice}\imp\fami{G}\text{ est génératrice}.\]
\end{enumerate}
\end{rem}

\begin{prop}
Soient \(E\) un \(\K\)-espace vectoriel, \(m,n\in\Ns\), \(\paren{x_1,\dots,x_n}\in E^n\) une famille quelconque de vecteurs de \(E\) et \(\paren{y_1,\dots,y_m}\in E^m\) une famille génératrice de \(E\).

Alors \(\paren{x_1,\dots,x_n}\) est une famille génératrice de \(E\) si, et seulement si : \[\quantifs{\forall j\in\interventierii{1}{m}}y_j\in\Vect{x_1,\dots,x_n}.\]
\end{prop}

\begin{dem}
\impdir Claire.

\imprec

Supposons \(\quantifs{\forall j\in\interventierii{1}{m}}y_j\in\Vect{x_1,\dots,x_n}\).

Soit \(\paren{a_{ij}}_{\paren{i,j}\in\interventierii{1}{n}\times\interventierii{1}{m}}\in\K^{\interventierii{1}{n}\times\interventierii{1}{m}}\) tels que \[\quantifs{\forall j\in\interventierii{1}{m}}y_j=\sum_{i=1}^na_{ij}x_i.\]

Montrons que \(\paren{x_1,\dots,x_n}\) est génératrice de \(E\).

Soit \(x\in E\).

Montrons que \(x\in\Vect{x_1,\dots,x_n}\).

Comme \(\paren{y_1,\dots,y_m}\) est génératrice de \(E\), il existe \(b_1,\dots,b_m\in\K\) tels que \[x=\sum_{j=1}^mb_jy_j.\]

On a donc \[\begin{aligned}
x&=\sum_{j=1}^mb_j\sum_{i=1}^na_{ij}x_i \\
&=\sum_{i=1}^n\underbrace{\sum_{j=1}^ma_{ij}b_j}_{\in\K}x_i \\
&\in\Vect{x_1,\dots,x_n}.
\end{aligned}\]
\end{dem}

\subsection{Bases}

\begin{defi}
Soient \(E\) un \(\K\)-espace vectoriel et \(\paren{e_1,\dots,e_p}\in E^p\) une famille d'éléments de \(E\).

On dit que \(\paren{e_1,\dots,e_p}\) est une base de \(E\) si c'est une famille libre et génératrice de \(E\).

Il est équivalent de dire que tout vecteur de \(E\) s'écrit de façon unique comme combinaison linéaire des vecteurs de la famille : \[\quantifs{\forall x\in E;\exists!\paren{\lambda_1,\dots,\lambda_p}\in\K^p}x=\lambda_1e_1+\dots+\lambda_pe_p.\]

On appelle alors \(\paren{\lambda_1,\dots,\lambda_p}\) les coordonnées du vecteur \(x\) dans la base \(\paren{e_1,\dots,e_p}\).
\end{defi}

\begin{ex}[Base canonique de \(\K^n\)]
Posons : \[\quantifs{\forall k\in\interventierii{1}{n}}e_k=\paren{\delta_{1k},\delta_{2k},\dots,\delta_{nk}}\in\K^n\] (autrement dit, le \(n\)-uplet \(e_k\) a tous ses coefficients nuls sauf le \(k\)-ème qui vaut \(1\)).

La famille \(\paren{e_1,\dots,e_n}\in\paren{\K^n}^n\) est une base de \(\K^n\) appelée la base canonique de \(\K^n\).
\end{ex}

\begin{ex}[Base canonique de \(\polydeg{n}\)]
La famille \[\paren{1,X,X^2,\dots,X^n}\] est une base de \(\polydeg{n}\) appelée la base canonique de \(\polydeg{n}\).
\end{ex}

\begin{exoex}
On pose \[E=\R^2\qquad v_1=\paren{1,0}\qquad v_2=\paren{1,1}.\]

Montrer que la famille \(\paren{v_1,v_2}\in E^2\) est une base de \(E\) et donner les coordonnées d'un vecteur quelconque \(\paren{x,y}\in\R^2\).
\end{exoex}

\begin{corr}~\\
Soient \(\dcoords{x}{y}\in\R^2\) et \(\lambda,\mu\in\R\).

On a \[\begin{aligned}
\lambda v_1+\mu v_2=\dcoords{x}{y}&\ssi\lambda\dcoords{1}{0}+\mu\dcoords{1}{1}=\dcoords{x}{y} \\
&\ssi\begin{dcases}
\lambda+\mu=x \\
\mu=y
\end{dcases} \\
&\ssi\begin{dcases}
\lambda=x-y \\
\mu=y
\end{dcases}
\end{aligned}\]

Donc tout vecteur \(\dcoords{x}{y}\in\R^2\) s'écrit de façon unique comme combinaison linéaire de \(v_1\) et \(v_2\).

Donc \(\paren{v_1,v_2}\) est une base de \(\R^2\).

De plus, on a obtenu les coordonnées de \(\dcoords{x}{y}\) dans la base \(\paren{v_1,v_2}\) : \[\paren{x-y,y}.\]
\end{corr}

\begin{prop}
Soient \(E\) un \(\K\)-espace vectoriel et \(\paren{x_1,\dots,x_p}\in E^p\) une famille d'éléments de \(E\).

Considérons l'application linéaire \[\fonction{\phi}{\K^p}{E}{\paren{\lambda_1,\dots,\lambda_p}}{\lambda_1x_1+\dots+\lambda_px_p}\]

On a les équivalences suivantes : \[\begin{aligned}
\paren{x_1,\dots,x_p}\text{ est une famille libre}&\ssi\phi\text{ est injective} \\
\paren{x_1,\dots,x_p}\text{ est une famille génératrice de }E&\ssi\phi\text{ est surjective} \\
\paren{x_1,\dots,x_p}\text{ est une base de }E&\ssi\phi\text{ est un isomorphisme}
\end{aligned}\]
\end{prop}

\begin{dem}
On a : \[\begin{aligned}
\paren{x_1,\dots,x_p}\text{ est libre}&\ssi\croch{\quantifs{\forall\paren{\lambda_1,\dots,\lambda_p}\in\K^p}\lambda_1x_1+\dots+\lambda_px_p=0_E\imp\paren{\lambda_1,\dots,\lambda_p}=\paren{0,\dots,0}} \\
&\ssi\croch{\quantifs{\forall\paren{\lambda_1,\dots,\lambda_p}\in\K^p}\paren{\lambda_1,\dots,\lambda_p}\in\ker\phi\imp\paren{\lambda_1,\dots,\lambda_p}=\paren{0,\dots,0}} \\
&\ssi\ker\phi=\accol{0_E} \\
&\ssi\phi\text{ est injective}.
\end{aligned}\]

De plus : \[\begin{aligned}
\paren{x_1,\dots,x_p}\text{ est génératrice de }E&\ssi\quantifs{\forall x\in E;\exists\paren{\lambda_1,\dots,\lambda_p}\in\K^p}x=\phi\paren{\paren{\lambda_1,\dots,\lambda_p}} \\
&\ssi\phi\text{ est une surjection de }\K^p\text{ vers }E.
\end{aligned}\]

Enfin : \[\begin{aligned}
\paren{x_1,\dots,x_p}\text{ est une base de }E&\ssi\paren{x_1,\dots,x_p}\text{ est libre et génératrice de }E \\
&\ssi\phi\text{ est injective et surjective} \\
&\ssi\phi\text{ est bijective}.
\end{aligned}\]
\end{dem}

\begin{prop}
Soit \(n\in\N\).

On considère une famille de polynômes \(\fami{B}=\paren{P_0,\dots,P_n}\in\polydeg{n}^{n+1}\) \guillemets{à degrés échelonnés}, \cad telle que \[\quantifs{\forall k\in\interventierii{0}{n}}\deg P_k=k.\]

Alors \(\fami{B}\) est une base de \(\polydeg{n}\).
\end{prop}

\begin{dem}
On sait que \(\fami{B}\) est une famille libre (selon l'\thref{ex:familleDePolynômesADegrésDeuxADeuxDistinctsEstLibre}).

Montrons que \(\fami{B}\) est génératrice de \(\polydeg{n}\).

Montrons que \(\quantifs{\forall k\in\interventierii{0}{n}}\underbrace{\quantifs{\forall P\in\polydeg{k}}P\in\Vect{P_0,\dots,P_k}}_{\P{k}}\) par récurrence sur \(k\).

Soit \(P\in\polydeg{0}\).

On a \(P_0\) constant et non-nul et \(P\) constant donc \(P=\dfrac{P}{P_0}P_0\) avec \(\dfrac{P}{P_0}\in\K\).

Donc \(P\in\Vect{P_0}\).

D'où \(\P{0}\).

Soit \(k\in\interventierii{0}{n-1}\) tel que \(\P{k}\).

Montrons \(\P{k+1}\).

Soient \(P\in\polydeg{k+1}\) et \(a_0,\dots,a_{k+1}\in\K\) tels que \[P=a_{k+1}X^{k+1}+\dots+a_0X^0.\]

On note \(\mu\) le coefficient dominant de \(P_{k+1}\).

On a \[\begin{dcases}
\deg\paren{P-\dfrac{a_{k+1}}{\mu}P_{k+1}}\leq k+1\text{ car }\begin{dcases}
\deg P\leq k+1 \\
\deg P_{k+1}\leq k+1
\end{dcases} \\
\text{le coefficient de degré }k+1\text{ de }P-\dfrac{a_{k+1}}{\mu}P_{k+1}\text{ vaut }a_{k+1}-\dfrac{a_{k+1}}{\mu}\mu=0
\end{dcases}\]

Donc \(P-\dfrac{a_{k+1}}{\mu}P_{k+1}\in\polydeg{k}\).

Selon \(\P{k}\), il existe \(\omega_0,\dots,\omega_k\in\K\) tels que \[P-\dfrac{a_{k+1}}{\mu}P_{k+1}=\omega_0P_0+\dots+\omega_kP_k.\]

Finalement, on a : \[P=\omega_0P_0+\dots+\omega_kP_k+\dfrac{a_{k+1}}{\mu}P_{k+1}.\]

Donc \(P\in\Vect{P_0,\dots,P_{k+1}}\).

D'où \(\P{k+1}\).

On a \(\P{n}\) donc \(\polydeg{n}=\Vect{\fami{B}}\).

Donc \(\fami{B}\) est une famille génératrice de \(\polydeg{n}\).

Finalement, \(\fami{B}\) est une base de \(\polydeg{n}\).
\end{dem}

\subsection{Pivot de Gauss pour les systèmes linéaires}

Soient \(n,p\in\Ns\) et deux familles de scalaires : \[\paren{a_{ij}}_{\paren{i,j}\in\interventierii{1}{n}\times\interventierii{1}{p}}\in\K^{\interventierii{1}{n}\times\interventierii{1}{p}}\qquad\text{et}\qquad\paren{b_1,\dots,b_n}\in\K^n.\]

Considérons le système linéaire de \(n\) équations à \(p\) inconnues suivant : \[\paren{S}\begin{dcases}
a_{11}x_1+\dots+a_{1p}x_p=b_1 \\
\vdots \\
a_{n1}x_1+\dots+a_{np}x_p=b_n
\end{dcases}\]

On peut dire au choix :

\begin{itemize}
\item que ses inconnues sont les scalaires \(x_1,\dots,x_p\in\K\). \\

\item que son inconnue est le \(p\)-uplet \(\paren{x_1,\dots,x_p}\in\K^p\).
\end{itemize}

L'ensemble solution \(\fami{S}\subset\K^p\) de \(\paren{S}\) est l'ensemble des \(p\)-uplets qui vérifient \(\paren{S}\).

On dit que \(\paren{S}\) est un système linéaire homogène si \(b_1=\dots=b_n=0\).

On a \[\fami{S}\text{ est un sous-espace vectoriel de }\K^p\ssi\paren{S}\text{ est un système linéaire homogène}.\]

L'algorithme du \guillemets{pivot de Gauss} permet de résoudre le système \(\paren{S}\) en raisonnant par équivalences, en appliquant au système les transformations suivantes :

\begin{itemize}
\item \(L_i\echange L_j\) : échange des lignes \(L_i\) et \(L_j\). \\

\item \(L_i\gets\lambda L_i\) : multiplication de la ligne \(L_i\) par \(\lambda\in\K\excluant\accol{0}\). \\

\item \(L_i\gets L_i+\lambda L_j\) : ajout à \(L_i\) de \(\lambda L_j\) où \(\lambda\in\K\excluant\accol{0}\).
\end{itemize}

De plus, si le système \(\paren{S}\) est linéaire homogène, il permet d'obtenir une base de l'espace vectoriel \(\fami{S}\).

\begin{exoex}
Soient \(\alpha,\beta,\gamma\in\R\).

Résoudre les systèmes linéaires suivants en appliquant l'algorithme du pivot de Gauss.

Donner une base de l'ensemble solution de chaque système linéaire homogène.

\[\paren{S_1}\begin{dcases}
2x+3y=-1 \\
x+2y=-3
\end{dcases}\quad\paren{S_2}\begin{dcases}
x+2y-z=\alpha \\
2x+5y+3z=\beta \\
3x+7y+2z=\gamma
\end{dcases}\quad\paren{S_3}\begin{dcases}
a+2b+c+3d=0 \\
3a+7b+3c+6d=0 \\
a+3b+c=0
\end{dcases}\quad\paren{S_4}~8a+4b-2c+d=0\]
\end{exoex}

\begin{corr}[1]
On a \[\begin{aligned}
\paren{S_1}&\ssi\begin{dcases}
2x+3y=-1 \\
x+2y=-3
\end{dcases} \\
&\ssi\begin{dcases}
x+2y=-3 &L_1\echange L_2 \\
2x+3y=-1
\end{dcases} \\
&\ssi\begin{dcases}
x+2y=-3 \\
-y=5 &L_2\gets L_2-2L_1
\end{dcases} \\
&\ssi\begin{dcases}
x=7 \\
y=-5
\end{dcases}
\end{aligned}\]

Donc \(\fami{S}_1=\accol{\paren{7,-5}}\).
\end{corr}

\begin{corr}[2]
On a \[\begin{aligned}
\paren{S_2}&\ssi\begin{dcases}
x+2y-z=\alpha \\
2x+5y+3z=\beta \\
3x+7y+2z=\gamma
\end{dcases} \\
&\ssi\begin{dcases}
x+2y-z=\alpha \\
y+5z=\beta-2\alpha &L_2\gets L_2-2L_1 \\
y+5z=\gamma-3\alpha &L_3\gets L_3-3L_1
\end{dcases} \\
&\ssi\begin{dcases}
x+2y-z=\alpha \\
y+5z=\beta-2\alpha \\
0=\gamma-\beta-\alpha
\end{dcases} \\
&\ssi\begin{dcases}
x-11z=5\alpha-2\beta &L_1\gets L_1-2L_2 \\
y+5z=\beta-2\alpha \\
0=\gamma-\beta-\alpha
\end{dcases}
\end{aligned}\]

Si \(\gamma-\beta-\alpha\not=0\) alors \(\fami{S}_2=\ensvide\).

Supposons \(\gamma-\beta-\alpha=0\).

Alors \[\fami{S}_2=\accol{\tcoords{5\alpha-2\beta}{\beta-2\alpha}{0}+\lambda\tcoords{11}{-5}{1}}_{\lambda\in\R}\]

De plus, si \(\alpha=\beta=\gamma=0\) alors \(\paren{S_2}\) est homogène et \(\fami{S}_2\) est un \(\R\)-espace vectoriel de base \(\paren{\tcoords{11}{-5}{1}}\).
\end{corr}

\begin{corr}[3]
On a \[\begin{aligned}
\paren{S_3}&\ssi\begin{dcases}
a+2b+c+3d=0 \\
3a+7b+3c+6d=0 \\
a+3b+c=0
\end{dcases} \\
&\ssi\begin{dcases}
a+2b+c+3d=0 \\
b-3d=0 &L_2\gets L_2-3L_1 \\
b-3d=0 &L_3\gets L_3-L_1
\end{dcases} \\
&\ssi\begin{dcases}
a+c+9d=0 &L_1\gets L_1-2L_2 \\
b-3d=0
\end{dcases}
\end{aligned}\]

Donc \[\fami{S}_3=\accol{\lambda\begin{pmatrix}-1 \\ 0 \\ 1 \\ 0\end{pmatrix}+\mu\begin{pmatrix}-9 \\ 3 \\ 0 \\ 1\end{pmatrix}}_{\paren{\lambda,\mu}\in\R^2}\]

Donc \(\fami{S}_3\) est un \(\R\)-espace vectoriel de base \(\paren{\begin{pmatrix}-1 \\ 0 \\ 1 \\ 0\end{pmatrix},\begin{pmatrix}-9 \\ 3 \\ 0 \\ 1\end{pmatrix}}\).
\end{corr}

\begin{corr}[4]
On a \[\begin{aligned}
\paren{S_4}&\ssi8a+4b-2c+d=0 \\
&\ssi d+8a+4b-2c=0
\end{aligned}\]

On a \[\begin{aligned}
\quantifs{\forall\begin{pmatrix}a\\b\\c\\d\end{pmatrix}\in\R^4}\begin{pmatrix}a\\b\\c\\d\end{pmatrix}\in\fami{S}_4&\ssi d=-8a-4b+2c \\
&\ssi\begin{pmatrix}a\\b\\c\\d\end{pmatrix}=\begin{pmatrix}a\\b\\c\\-8a-4b+2c\end{pmatrix} \\
&\ssi\begin{pmatrix}a\\b\\c\\d\end{pmatrix}=a\begin{pmatrix}1\\0\\0\\-8\end{pmatrix}+b\begin{pmatrix}0\\1\\0\\-4\end{pmatrix}+c\begin{pmatrix}0\\0\\1\\2\end{pmatrix}
\end{aligned}\]

Donc \[\fami{S}_4=\Vect{\begin{pmatrix}1\\0\\0\\-8\end{pmatrix},\begin{pmatrix}0\\1\\0\\-4\end{pmatrix},\begin{pmatrix}0\\0\\1\\2\end{pmatrix}}\] et \(\paren{\begin{pmatrix}1\\0\\0\\-8\end{pmatrix},\begin{pmatrix}0\\1\\0\\-4\end{pmatrix},\begin{pmatrix}0\\0\\1\\2\end{pmatrix}}\) est une base de \(\fami{S}_4\).
\end{corr}

\subsection{Familles de vecteurs et applications linéaires}

\subsubsection{Image d'une famille par une application linéaire}

\begin{prop}
Soient \(E\) et \(F\) deux \(\K\)-espaces vectoriels, \(u\in\L{E}{F}\) et \(\paren{x_1,\dots,x_p}\in E^p\) une famille de vecteurs de \(E\).

On a :

\begin{enumerate}
\item Si \(\paren{x_1,\dots,x_p}\) est libre et \(u\) injective, alors \(\paren{u\paren{x_1},\dots,u\paren{x_p}}\) est libre. \\

\item Si \(\paren{x_1,\dots,x_p}\) est génératrice de \(E\) et \(u\) surjective, alors \(\paren{u\paren{x_1},\dots,u\paren{x_p}}\) est génératrice de \(F\). \\

\item Si \(\paren{x_1,\dots,x_p}\) est une base de \(E\) et \(u\) un isomorphisme, alors \(\paren{u\paren{x_1},\dots,u\paren{x_p}}\) est une base de \(F\).
\end{enumerate}
\end{prop}

\begin{dem}[1]
On suppose que \(\paren{x_1,\dots,x_p}\) est libre et \(u\) injective.

Soient \(\lambda_1,\dots,\lambda_p\in\K\) tels que \(\lambda_1u\paren{x_1}+\dots+\lambda_pu\paren{x_p}=0_F\).

Montrons que \(\lambda_1=\dots=\lambda_p=0\).

Comme \(u\) est linéaire, on a \(u\paren{\lambda_1x_1+\dots+\lambda_px_p}=0_F\).

Donc, comme \(u\) est injective, on a \(\lambda_1x_1+\dots+\lambda_px_p=0_E\).

Donc, comme \(\paren{x_1,\dots,x_p}\) est libre, on a \(\lambda_1=\dots=\lambda_p=0\).

Donc \(\paren{u\paren{x_1},\dots,u\paren{x_p}}\) est libre.
\end{dem}

\begin{dem}[2]
Supposons que \(\paren{x_1,\dots,x_p}\) engendre \(E\) et que \(u\) est surjective.

On a \[\begin{WithArrows}
\Vect{u\paren{x_1},\dots,u\paren{x_p}}&=u\paren{\Vect{x_1,\dots,x_p}} \Arrow{car \(\paren{x_1,\dots,x_p}\) engendre \(E\)} \\
&=u\paren{E} \Arrow{car \(u\) est surjective} \\
&=F
\end{WithArrows}\]

Donc \(\paren{u\paren{x_1},\dots,u\paren{x_p}}\) engendre \(F\).
\end{dem}

\begin{dem}[3]
Découle de (1) et (2).
\end{dem}

\subsubsection{Application linéaire définie par l'image d'une base}

\begin{prop}
Soient \(E\) et \(F\) deux \(\K\)-espaces vectoriels, \(\paren{e_1,\dots,e_p}\) une base de \(E\) et \(\paren{y_1,\dots,y_p}\) une famille d'éléments de \(F\).

Alors \[\quantifs{\exists!u\in\L{E}{F};\forall j\in\interventierii{1}{p}}u\paren{e_j}=y_j.\]
\end{prop}

\begin{dem}
Pour tout vecteur \(x\in E\), on note \(\paren{e_1\etoile\paren{x},\dots,e_p\etoile\paren{x}}\) les coordonnées de \(x\) dans la base \(\paren{e_1,\dots,e_p}\). Autrement dit, on a \[\begin{dcases}
e_1\etoile\paren{x},\dots,e_p\etoile\paren{x}\in\K \\
x=e_1\etoile\paren{x}e_1+\dots+e_p\etoile\paren{x}e_p
\end{dcases}\]

\analyse

Soit \(u\in\L{E}{F}\) tel que \(\quantifs{\forall j\in\interventierii{1}{p}}u\paren{e_j}=y_j\).

On a \(\quantifs{\forall x\in E}x=e_1\etoile\paren{x}e_1+\dots+e_p\etoile\paren{x}e_p\).

Donc comme \(u\) est linéaire, on a : \[\begin{aligned}
\quantifs{\forall x\in E}u\paren{x}&=e_1\etoile\paren{x}u\paren{e_1}+\dots+e_p\etoile\paren{x}u\paren{e_p} \\
&=e_1\etoile\paren{x}y_1+\dots+e_p\etoile\paren{x}y_p.
\end{aligned}\]

D'où \(\fonction{u}{E}{F}{x}{e_1\etoile\paren{x}y_1+\dots+e_p\etoile\paren{x}y_p}\)

\synthese

Posons \(\fonction{u}{E}{F}{x}{e_1\etoile\paren{x}y_1+\dots+e_p\etoile\paren{x}y_p}\)

\(u\) est clairement linéaire car \(e_1\etoile,\dots,e_p\etoile\) sont linéaires.

On a \[\quantifs{\forall j\in\interventierii{1}{p}}u\paren{e_j}=e_1\etoile\paren{x}y_1+\dots+e_p\etoile\paren{x}y_p=\delta_{1j}y_1+\dots+\delta_{pj}y_p=y_j.\]
\end{dem}

\subsection{Extension aux familles quelconques}

Dans ce paragraphe, on étend ce qu'on a vu précédemment à des familles de vecteurs indicées par un ensemble quelconque (éventuellement infini).

\subsubsection{Combinaisons linéaires}

\begin{defi}[Famille de scalaires \guillemets{presque tous nuls}]
Soient \(I\) un ensemble et \(\paren{\lambda_i}_{i\in I}\in\K^I\) une famille de scalaires indicée par \(I\).

On appelle support de la famille \(\paren{\lambda_i}_{i\in I}\in\K^I\) l'ensemble \[\Supp\paren{\lambda_i}_{i\in I}=\accol{i\in I\tq\lambda_i\not=0}.\]

Cette famille est dite à support fini si son support est fini. Si l'ensemble \(I\) est infini, on dit alors aussi que c'est une famille de scalaires presque tous nuls.

L'ensemble des familles de scalaires indicées par \(I\) à support fini est souvent noté \(\K\deriv{I}\) : \[\K\deriv{I}=\accol{\paren{\lambda_i}_{i\in I}\in\K^I\tq\Card\Supp\paren{\lambda_i}_{i\in I}<\pinf}.\]

Si \(I\) est fini, on a \(\K\deriv{I}=\K^I\).
\end{defi}

\begin{nota}
Soient \(I\) un ensemble, \(E\) un \(\K\)-espace vectoriel et \(\fami{F}=\paren{x_i}_{i\in I}\in E^I\) une famille de vecteurs de \(E\).

Si \(\paren{\lambda_i}_{i\in I}\in\K\deriv{I}\) est une famille de scalaires à support fini, on pose : \[\sum_{i\in I}\lambda_ix_i=\sum_{i\in\Supp\fami{F}}\lambda_ix_i.\]

Ainsi, on s'autorise à écrire une somme éventuellement infinie \(\sum_{i\in I}\lambda_ix_i\) car on la voit comme une somme finie.
\end{nota}

\begin{defi}[Combinaison linéaire, généralisation de la \thref{def:combinaisonLinéaire}]
Soient \(I\) un ensemble, \(E\) un \(\K\)-espace vectoriel et \(\fami{F}=\paren{x_i}_{i\in I}\in E^I\) une famille de vecteurs de \(E\).

On appelle combinaison linéaire de \(\fami{F}\) tout vecteur de la forme : \[\sum_{i\in I}\lambda_ix_i\] où \(\paren{\lambda_i}_{i\in I}\in\K\deriv{I}\) est une famille de scalaires à support fini.
\end{defi}

\begin{defprop}
Soient \(I\) un ensemble, \(E\) un \(\K\)-espace vectoriel et \(\fami{F}=\paren{x_i}_{i\in I}\in E^I\) une famille de vecteurs de \(E\).

On appelle sous-espace vectoriel engendré par \(\fami{F}\) et on note \(\Vect{\fami{F}}\) le plus petit sous-espace vectoriel de \(E\) qui contient les vecteurs de \(\fami{F}\).

Ses éléments sont les combinaisons linéaires de \(\fami{F}\).
\end{defprop}

\begin{ex}
Considérons le \(\K\)-espace vectoriel \(\poly\) et la famille \(\fami{F}=\paren{X^n}_{n\in\N}\).

Alors \[\Vect{\fami{F}}=\poly.\]
\end{ex}

\subsubsection{Familles génératrices}

\begin{defi}
Soient \(I\) un ensemble, \(E\) un \(\K\)-espace vectoriel et \(\fami{F}=\paren{x_i}_{i\in I}\in E^I\) une famille de vecteurs de \(E\).

On dit que \(\fami{F}\) est une famille génératrice de \(E\) ou que la famille \(\fami{F}\) engendre \(E\) si tout vecteur de \(E\) est combinaison linéaire des vecteurs de \(\fami{F}\), \cad si : \[E=\Vect{\fami{F}},\] \cad : \[\quantifs{\forall x\in E;\exists\paren{\lambda_i}_{i\in I}\in\K\deriv{I}}x=\sum_{i\in I}\lambda_ix_i.\]
\end{defi}

\subsubsection{Familles libres}

\begin{defi}
Soient \(I\) un ensemble, \(E\) un \(\K\)-espace vectoriel et \(\fami{F}=\paren{x_i}_{i\in I}\in E^I\) une famille de vecteurs de \(E\).

On dit que \(\fami{F}\) est une famille libre si : \[\quantifs{\forall\paren{\lambda_i}_{i\in I}\in\K\deriv{I}}\sum_{i\in I}\lambda_ix_i=0_E\imp\croch{\quantifs{\forall i\in I}\lambda_i=0}.\]

Si la famille \(\fami{F}\) n'est pas libre, on dit qu'elle est liée.
\end{defi}

\begin{rem}
Soient \(I\) un ensemble, \(E\) un \(\K\)-espace vectoriel et \(\fami{F}=\paren{x_i}_{i\in I}\in E^I\) une famille de vecteurs de \(E\).

Alors la famille \(\fami{F}\) est libre si, et seulement si, pour tous \(p\in\Ns\) et tous \(i_1,\dots,i_p\in I\) deux à deux distincts, la famille \(\paren{x_{i_1},\dots,x_{i_p}}\) est libre.
\end{rem}

\begin{ex}
Soient \(I\) un ensemble et \(\fami{F}=\paren{P_i}_{i\in I}\in\poly^I\) une famille de polynômes non-nuls de degrés deux à deux distincts, \cad telle que : \[\quantifs{\forall i,j\in I}i\not=j\imp\deg P_i\not=\deg P_j.\]

Alors \(\fami{F}\) est libre.
\end{ex}

\subsubsection{Bases}

\begin{defi}
Soient \(I\) un ensemble, \(E\) un \(\K\)-espace vectoriel et \(\fami{F}=\paren{e_i}_{i\in I}\in E^I\) une famille de vecteurs de \(E\).

On dit que \(\fami{F}\) est une base de \(E\) si c'est une famille libre et génératrice de \(E\).

Il est équivalent de dire que tout vecteur de \(E\) s'écrit de façon unique comme combinaison linéaire des vecteurs de la famille : \[\quantifs{\forall x\in E;\exists!\paren{\lambda_i}_{i\in I}\in\K\deriv{I}}x=\sum_{i\in I}\lambda_ie_i.\]

On appelle alors \(\paren{\lambda_i}_{i\in I}\) les coordonnées du vecteur \(x\) dans la base \(\fami{F}\).
\end{defi}

\begin{ex}[Base canonique de \(\poly\)]
La famille \[\paren{X^n}_{n\in\N}\] est une base de \(\poly\) appelée la base canonique de \(\poly\).
\end{ex}

\subsubsection{Cas des ensembles de vecteurs}

\begin{defi}[Parties libres, génératrices]
Soit \(E\) un \(\K\)-espace vectoriel.

On dit qu'une partie \(A\subset E\) est :

\begin{itemize}
\item libre si la famille \(\paren{x}_{x\in A}\) est libre. \\

\item génératrice si la famille \(\paren{x}_{x\in A}\) est génératrice de \(E\). \\

\item une base de \(E\) si la famille \(\paren{x}_{x\in A}\) est une base de \(E\).
\end{itemize}
\end{defi}

\section{Géométrie affine}

\subsection{Translations}

\begin{defi}[Translation]
Soient \(E\) un \(\K\)-espace vectoriel et \(v\in E\).

On appelle translation de vecteur \(v\) la fonction : \[\fonctionlambda{E}{E}{x}{x+v}\]
\end{defi}

\begin{rem}
Soient \(E\) un \(\K\)-espace vectoriel et \(v\in E\).

L'image du vecteur nul par la translation de vecteur \(v\) est le vecteur \(v\).

En particulier, une translation n'est jamais un endomorphisme, sauf si \(v=0\) (dans ce cas, la translation de vecteur \(v\) est l'application identité \(\id{E}\)).
\end{rem}

\subsection{Sous-espaces affines}

\begin{defprop}[Sous-espace affine d'un espace vectoriel]
Soit \(E\) un \(\K\)-espace vectoriel.

On appelle sous-espace affine de \(E\) toute partie \(A\subset E\) de la forme : \[A=v_1+A_0=\accol{v_1+v_0}_{v_0\in A_0}\] où \(v_1\in E\) et \(A_0\) est un sous-espace vectoriel de \(E\).

On dit que \(v_1+A_0\) est le sous-espace affine de direction \(A_0\) passant par \(v_1\).

\begin{enumerate}
\item On a alors nécessairement : \[A_0=\accol{v-v\prim}_{v,v\prim\in A}.\]

En particulier, l'espace vectoriel \(A_0\) est unique. Il est appelé la direction du sous-espace affine \(A\). \\

\item En revanche, le vecteur \(v_1\) n'est pas unique en général car tout vecteur de \(A\) convient : \[\quantifs{\forall v_2\in A}A=v_2+A_0.\]
\end{enumerate}
\end{defprop}

\begin{dem}[1]
Montrons que \(A_0=\accol{v-v\prim}_{v,v\prim\in A}\).

\increc

Soient \(v,v\prim\in A\).

Montrons que \(v-v\prim\in A_0\).

Il existe \(v_0,v_0\prim\in A_0\) tels que \(\begin{dcases}
v=v_1+v_0 \\
v\prim=v_1+v_0\prim
\end{dcases}\)

On a \[\begin{WithArrows}
v-v\prim&=v_1+v_0-\paren{v_1+v_0}\prim \\
&=v_0-v_0\prim \Arrow{car \(A_0\) est un sous-espace vectoriel} \\
&\in A_0
\end{WithArrows}\]

\incdir

Soit \(v_0\in A_0\).

On a \(\begin{dcases}
v_1+v_0\in A \\
v_1+0_E\in A
\end{dcases}\)

D'où \[v_0=v_1+v_0-\paren{v_1+0_E}\in\accol{v-v\prim}_{v,v\prim\in A}.\]
\end{dem}

\begin{dem}[2]
Soit \(v_2\in A\).

Montrons que \(A=v_2+A_0\), \cad \(v_1+A_0=v_2+A_0\).

On a \(v_2=v_1+v_0\) où \(v_0\in A_0\).

Donc \[v_2+A_0=v_1+v_0+A_0=v_1+A_0\text{ car }v_0\in A_0.\]
\end{dem}

\begin{rem}
Les sous-espaces affines de \(E\) sont les images des sous-espaces vectoriels de \(E\) par les translations de \(E\).
\end{rem}

\begin{ex}
Soit \(E\) un espace vectoriel.

Alors \begin{itemize}
\item L'ensemble \(E\) est un sous-espace affine de \(E\), de direction \(E\). \\

\item Tout singleton de \(E\) est un sous-espace affine de \(E\), de direction \(\accol{0_E}\). \\

\item Tout sous-espace vectoriel \(F\) de \(E\) est un sous-espace affine de \(E\), de direction \(F\). \\

\item L'ensemble vide n'est pas un sous-espace affine de \(E\).
\end{itemize}
\end{ex}

\begin{ex}
Prenons \(E=\R^3\) et considérons le système suivant, d'inconnue \(\paren{x,y,z}\in E\) : \[\paren{S}\begin{dcases}
x+y-4z=1 \\
x-y-2z=3
\end{dcases}\]

Son ensemble solution \(A\) est un sous-espace affine de \(E\).
\end{ex}

\begin{dem}
On a \[\begin{aligned}
\paren{S}&\ssi\begin{dcases}
x+y-4z=1 \\
x-y-2z=3
\end{dcases} \\
&\ssi\begin{dcases}
x+y-4z=1 \\
-2y+2z=2 &L_2\gets L_2-L_1
\end{dcases} \\
&\ssi\begin{dcases}
x-3z=2 &L_1\gets L_1+\dfrac{1}{2}L_2 \\
y-z=-1 &L_2\gets\dfrac{-1}{2}L_2
\end{dcases}
\end{aligned}\]

Donc \[\begin{aligned}
A&=\accol{\tcoords{2}{-1}{0}+\lambda\tcoords{3}{1}{1}}_{\lambda\in\R} \\
&=\Vect{\tcoords{3}{1}{1}}+\tcoords{2}{-1}{0}.
\end{aligned}\]

Donc \(A\) est le sous-espace affine de \(E\) passant par \(\tcoords{2}{-1}{0}\) et dirigé par \(\Vect{\tcoords{3}{1}{1}}\).
\end{dem}

\begin{prop}[Intersection de sous-espaces affines]
Soient \(E\) un \(\K\)-espace vectoriel et \(\paren{A_i}_{i\in I}\) une famille de sous-espaces affines de \(E\).

Pour tout \(i\in I\), on note \(A_i\prim\) la direction de \(A_i\).

Alors l'intersection \(\biginter_{i\in I}A_i\) est soit l'ensemble vide, soit un sous-espace affine de \(E\) de direction \(\biginter_{i\in I}A_i\prim\).
\end{prop}

\begin{dem}
Supposons \(\biginter_{i\in I}A_i\not=\ensvide\).

Soit \(x\in\biginter_{i\in I}A_i\).

Montrons que \(\biginter_{i\in I}A_i=x+\biginter_{i\in I}A_i\prim\).

\incdir

Soit \(y\in\biginter_{i\in I}A_i\).

On a \(\quantifs{\forall i\in I}y\in A_i=x+A_i\prim\).

Donc pour tout \(i\in I\), il existe \(z_i\in A_i\prim\) tel que \(y=x+z_i\).

D'où \(\quantifs{\forall i\in I}y-x=z_i\in A_i\prim\).

Donc \(y-x\in\biginter_{i\in I}a_i\prim\).

D'où \(y=x+y-x\in x+\biginter_{i\in I}A_i\prim\).

\increc

Soit \(y\in x+\biginter_{i\in I}A_i\prim\).

Alors \(\quantifs{\forall j\in I}y\in x+A_j\prim\) car \(A_j\prim\subset\biginter_{i\in I}A_i\prim\).

Donc \(y\in\biginter_{j\in I}\paren{x+A_j\prim}=\biginter_{j\in I}A_j\).
\end{dem}

\subsection{Équations linéaires}

\begin{defi}[Équation linéaire]
Soient \(E\) et \(F\) deux \(\K\)-espaces vectoriels, \(u\in\L{E}{F}\) et \(y\in F\).

L'équation suivante, d'inconnue \(x\in E\), est appelée équation linéaire : \[u\paren{x}=y.\]
\end{defi}

\begin{ex}[Système linéaire]
Le système suivant : \[\paren{S}\begin{dcases}
x+y-4z=1 \\
x-y-2z=3
\end{dcases}\] peut être vu comme une équation linéaire.

Posons \(\fonction{u}{\R^3}{\R^2}{\tcoords{x}{y}{z}}{\dcoords{x+y-4z}{x-y-2z}}\in\L{\R^3}{\R^2}\) et \(Y=\dcoords{1}{3}\).

Alors \(\paren{S}\ssi u\paren{X}=Y\) : équation linéaire d'inconnue \(X\in\R^3\).
\end{ex}

\begin{ex}[Équation différentielle linéaire]
L'équation différentielle \[\paren{E}~y\seconde+y=\cos t\] peut être vue comme une équation linéaire.

Posons \(\fonction{u}{\ensclasse{2}{\R}{\R}}{\ensclasse{0}{\R}{\R}}{y}{y\seconde+y}\) et \(z=\cos\in\ensclasse{0}{\R}{\R}\).

Alors \(\paren{E}\ssi u\paren{y}=z\) : équation linéaire d'inconnue \(y\in\ensclasse{2}{\R}{\R}\).
\end{ex}

\begin{ex}[Polynômes interpolateurs de Lagrange]
Soient \(x_0,\dots,x_n\in\R\) deux à deux distincts et \(y_0,\dots,y_n\in\R\).

Le système suivant, d'inconnue \(P\in\poly[\R]\) : \[\paren{S}\begin{dcases}
P\paren{x_0}=y_0 \\
\vdots \\
P\paren{x_n}=y_n
\end{dcases}\] peut être vu comme une équation linéaire.

Posons \(\fonction{u}{\poly[\R]}{\R^{n+1}}{P}{\paren{P\paren{x_0},\dots,P\paren{x_n}}}\in\L{\poly[\R]}{\R^{n+1}}\) et \(Y=\paren{y_0,\dots,y_n}\in\R^{n+1}\).

Alors \(\paren{S}\ssi u\paren{P}=Y\) : équation linéaire d'inconnue \(P\in\poly[\R]\).
\end{ex}

\begin{prop}[Ensemble solution d'une équation linéaire]
Soient \(E\) et \(F\) deux \(\K\)-espaces vectoriels, \(u\in\L{E}{F}\) et \(y\in F\).

L'ensemble solution \(\fami{S}\) de l'équation linéaire suivante, d'inconnue \(x\in E\) : \[u\paren{x}=y\] est soit l'ensemble vide, soit un sous-espace affine de \(E\) de direction \(\ker u\).
\end{prop}

\begin{dem}
Supposons \(\fami{S}\not=\ensvide\).

Soit \(x_1\in\fami{S}\).

On a : \[\begin{aligned}
\quantifs{\forall x\in E}u\paren{x}=y&\ssi u\paren{x}=u\paren{x_1} \\
&\ssi u\paren{x-x_1}=0 \\
&\ssi x-x_1\in\ker u \\
&\ssi\quantifs{\exists x_0\in\ker u}x-x_1=x_0 \\
&\ssi\quantifs{\exists x_0\in\ker u}x=x_0+x_1 \\
&\ssi x\in\ker u+x_1.
\end{aligned}\]

Donc \(\fami{S}=x_1+\ker u\).
\end{dem}

\chapter{Équations différentielles}\label{chap:équationsDifférentielles}

\minitoc

On pose \(\K=\R\) ou \(\C\).

\section{Équations différentielles linéaires du premier ordre}

\subsection{Cadre}

On considère :

\begin{itemize}
\item \(I\) un intervalle de \(\R\) ; \\

\item \(a,b\in\ensclasse{0}{I}{\K}\) ; \\

\item éventuellement \(t_0\in I\) et \(v_0\in\K\) ; \\

\item l'équation différentielle linéaire du premier ordre \(\paren{E}~y\prim+a\paren{t}y=b\paren{t}\).
\end{itemize}

Résoudre\footnote{On dit aussi \guillemets{intégrer l'équation différentielle \(\paren{E}\)}.} l'équation différentielle \(\paren{E}\), c'est déterminer quelles sont les fonctions \(y\in\ensclasse{1}{I}{\K}\) qui sont solutions de \(\paren{E}\), \cad qui vérifient : \[\quantifs{\forall t\in I}y\prim\paren{t}+a\paren{t}y\paren{t}=b\paren{t}.\]

L'ensemble solution de l'équation différentielle \(E\) est l'ensemble des solutions de \(\paren{E}\) : \[\fami{S}=\accol{y\in\ensclasse{1}{I}{\K}\tq\quantifs{\forall t\in I}y\prim\paren{t}+a\paren{t}y\paren{t}=b\paren{t}}.\]

L'équation différentielle \(\paren{E}\) est dite homogène si \(b\) est la fonction identiquement nulle.

On appelle équation homogène associée à \(\paren{E}\) l'équation différentielle linéaire homogène du premier ordre : \(\paren{E_0}~y\prim+a\paren{t}y=0\).

Lorsque l'on recherche la\footnote{On verra que cette solution existe et est unique.} solution de \(\paren{E}\) qui vérifie de plus une condition initiale, on dit qu'on résout le problème de Cauchy : \[\begin{dcases}
y\prim+a\paren{t}y=b\paren{t} \\
y\paren{t_0}=v_0
\end{dcases}\]

\begin{ex}
\begin{itemize}
\item Équation différentielle linéaire homogène du premier ordre : \[y\prim+\paren{3\ln t+1}y=0.\]

\item Équation différentielle linéaire du premier ordre : \[y\prim-\dfrac{\ln t}{t}y=t.\]

\item Problème de Cauchy : \[\begin{dcases}
y\prim+ty=t \\
y\paren{0}=1
\end{dcases}\]
\end{itemize}
\end{ex}

\subsection{Cas homogène}

\begin{prop}\thlabel{prop:solutionsD'UneÉquationDifférentielleLinéaireHomogène}
Soient \(I\) un intervalle de \(\R\), \(a\in\ensclasse{0}{I}{\K}\) et \(A:I\to\K\) une primitive de \(a\).

Les solutions de l'équation différentielle linéaire homogène du premier ordre \[\paren{E_0}~y\prim+a\paren{t}y=0\] sont les fonctions de la forme \[\fonction{y_0}{I}{\K}{t}{\lambda\e{-A\paren{t}}}\qquad\text{où }\lambda\in\K.\]
\end{prop}

\begin{dem}
Soit \(y\in\ensclasse{1}{I}{\K}\).

On a \[\begin{WithArrows}
y\text{ est solution de }\paren{E_0}&\ssi\quantifs{\forall t\in I}y\prim\paren{t}+a\paren{t}y\paren{t}=0 \\
&\ssi\quantifs{\forall t\in I}y\prim\paren{t}\e{A\paren{t}}+a\paren{t}y\paren{t}\e{A\paren{t}}=0 \\
&\ssi\quantifs{\forall t\in I}\paren{y\e{A}}\prim\paren{t}=0 \Arrow[tikz={text width=3cm}]{car \(I\) est un intervalle} \\
&\ssi\quantifs{\exists\lambda\in\K;\forall t\in I}y\paren{t}\e{A\paren{t}}=\lambda \\
&\ssi\quantifs{\exists\lambda\in\K;\forall t\in I}y\paren{t}=\lambda\e{-A\paren{t}}.
\end{WithArrows}\]
\end{dem}

\begin{exoex}
Résoudre sur l'intervalle \(\Rps\) l'équation différentielle homogène \[\paren{E_0}~y\prim+\dfrac{1}{t}y=0.\]
\end{exoex}

\begin{corr}
Une primitive de \(a:t\mapsto\dfrac{1}{t}\) est \(A:t\mapsto\ln t\).

Donc les solutions de \(\paren{E_0}\) sont les fonctions de la forme \[\fonctionlambda{\Rps}{\R}{t}{\lambda\e{-\ln t}=\dfrac{\lambda}{t}}\] où \(\lambda\in\R\).
\end{corr}

\begin{exoex}
Résoudre sur l'intervalle \(\Rps\) l'équation différentielle homogène \[\paren{E_0}~y\prim+\paren{\ln t}y=0.\]
\end{exoex}

\begin{corr}
Une primitive de \(t\mapsto\ln t\) est \(t\mapsto t\ln t-t\).

Donc les solutions de \(\paren{E_0}\) sont les fonctions de la forme \[\fonctionlambda{\Rps}{\R}{t}{\lambda\e{-t\ln t+t}=\lambda t^{-t}\e{t}=\lambda\paren{\dfrac{\e{}}{t}}^t}\] où \(\lambda\in\R\).
\end{corr}

\subsection{Méthode de variation de la constante}

\begin{rem}
Il existe beaucoup de méthodes dites de \guillemets{variation de la constante}. Elles permettent de trouver une/des/les solutions d'une équation différentielle linéaire à partir d'une solution non-nulle de l'équation différentielle linéaire homogène associée.

On montre dans ce paragraphe comment obtenir une solution (particulière) d'une équation différentielle linéaire à partir d'une solution de l'équation différentielle linéaire homogène associée.
\end{rem}

\begin{meth}[Méthode de variation de la constante]\thlabel{meth:variationDeLaConstante}
Soient \(I\) un intervalle de \(\R\) et \(a,b\in\ensclasse{0}{I}{\K}\).

On considère l'équation différentielle linéaire \[\paren{E}~y\prim+a\paren{t}y=b\paren{t}\] et son équation homogène associée \[\paren{E_0}~y\prim+a\paren{t}y=0.\]

On suppose que \(y_0\in\ensclasse{1}{I}{\K}\) est une solution de \(\paren{E_0}\) qui ne s'annule jamais.

On peut en déduire une solution de \(\paren{E}\) sous la forme \[t\mapsto\lambda\paren{t}y_0\paren{t}\] où \(\lambda\in\ensclasse{1}{I}{\K}\).

En particulier, l'ensemble solution de \(\paren{E}\) est non-vide.
\end{meth}

\begin{dem}
Soit \(\lambda\in\ensclasse{1}{I}{\K}\).

On pose \(y_1=\lambda y_0\).

On a \[\begin{aligned}
y_1\text{ est solution de }\paren{E}&\ssi\quantifs{\forall t\in I}\paren{\lambda y_0}\prim\paren{t}+a\paren{t}\paren{\lambda y_0}\paren{t}=b\paren{t} \\
&\ssi\quantifs{\forall t\in I}\lambda\prim\paren{t}y_0\paren{t}+\underbrace{\lambda\paren{t}y_0\prim\paren{t}+a\paren{t}\lambda\paren{t}y_0\paren{t}}_{=0\text{ car }y_0\text{ est solution de }\paren{E_0}}=b\paren{t} \\
&\ssi\quantifs{\forall t\in I}\lambda\prim\paren{t}=\dfrac{b\paren{t}}{y_0\paren{t}}.
\end{aligned}\]

La fonction \(\dfrac{b}{y_0}\) est continue sur \(I\) donc elle admet une primitive \(\lambda\in\ensclasse{1}{I}{\K}\).

Alors \(y_1\) est solution de \(\paren{E}\).
\end{dem}

\begin{exoex}
Trouver une solution sur l'intervalle \(\Rps\) de l'équation différentielle \[\paren{E}~y\prim+\dfrac{1}{t}y=1.\]
\end{exoex}

\begin{corr}
On a l'équation homogène associée à \(\paren{E}\) : \(\paren{E_0}~y\prim+\dfrac{1}{t}y=0\).

Les solutions de \(\paren{E_0}\) sont les fonctions de la forme \(\fonctionlambda{\Rps}{\R}{t}{\dfrac{\lambda}{t}}\) où \(\lambda\in\R\).

Soit \(\lambda\in\ensclasse{1}{\Rps}{\R}\).

On pose \(y_1:t\mapsto\dfrac{\lambda\paren{t}}{t}\).

On a \[\begin{aligned}
y_1\text{ est solution de }\paren{E}&\ssi\quantifs{\forall t\in\Rps}y_1\prim\paren{t}+\dfrac{1}{t}y_1\paren{t}=1 \\
&\ssi\quantifs{\forall t\in\Rps}\dfrac{\lambda\prim\paren{t}t-\lambda\paren{t}}{t^2}+\dfrac{\lambda\paren{t}}{t^2}=1 \\
&\ssi\quantifs{\forall t\in\Rps}\dfrac{\lambda\prim\paren{t}t}{t^2}=1 \\
&\ssi\quantifs{\forall t\in\Rps}\dfrac{\lambda\prim\paren{t}}{t}=1 \\
&\ssi\quantifs{\forall t\in\Rps}\lambda\prim\paren{t}=t \\
&\impr\quantifs{\forall t\in\Rps}\lambda\paren{t}=\dfrac{t^2}{2}
\end{aligned}\]

Donc \(t\mapsto\dfrac{t}{2}\) est solution de \(\paren{E}\).
\end{corr}

\subsection{Conséquences de la linéarité}

\begin{prop}[Structure de l'ensemble solution]
Soient \(I\) un intervalle de \(\R\) et \(a,b\in\ensclasse{0}{I}{\K}\).

On considère l'équation différentielle linéaire \[\paren{E}~y\prim+a\paren{t}y=b\paren{t}\] et son équation homogène associée \[\paren{E_0}~y\prim+a\paren{t}y=0.\]

\begin{enumerate}
\item L'ensemble solution \(\fami{S}_0\) de \(\paren{E_0}\) est un espace vectoriel. \\

\item L'ensemble solution \(\fami{S}\) de \(\paren{E}\) est un espace affine de direction \(\fami{S}_0\).
\end{enumerate}
\end{prop}

\begin{dem}\thlabel{dem:structureDeL'ensembleSolutionD'uneEquaDiffDuPremierOrdre}
Posons \[\fonction{u}{\ensclasse{1}{I}{\K}}{\ensclasse{0}{I}{\K}}{y}{y\prim+ay}\]

On a \(u\in\L{\ensclasse{1}{I}{\K}}{\ensclasse{0}{I}{\K}}\).

On a \(\fami{S}_0=\ker u\) donc \(\fami{S}_0\) est un sous-espace vectoriel de \(\ensclasse{1}{I}{\K}\).

\(\fami{S}\) est l'ensemble solution de l'équation linéaire \(u\paren{y}=b\) d'inconnue \(y\in\ensclasse{1}{I}{\K}\).

Selon la \thref{meth:variationDeLaConstante}, on a \(\fami{S}\not=\ensvide\) donc \(\fami{S}\) est un sous-espace affine de \(\ensclasse{1}{I}{\K}\) de direction \(\ker u=\fami{S}_0\).
\end{dem}

\begin{prop}[Principe de superposition]\thlabel{prop:principeDeSuperposition}
Soient \(I\) un intervalle de \(\R\), \(\lambda_1,\lambda_2\in\K\) et \(a,b_1,b_2\in\ensclasse{0}{I}{\K}\).

Si \(y_1\in\ensclasse{1}{I}{\K}\) est solution de \[\paren{E_1}~y\prim+a\paren{t}y=b_1\paren{t}\] et \(y_2\in\ensclasse{1}{I}{\K}\) est solution de \[\paren{E_2}~y\prim+a\paren{t}y=b_2\paren{t}\] alors \(\lambda_1y_1+\lambda_2y_2\) est solution de \[\paren{E}~y\prim+a\paren{t}y=\lambda_1b_1\paren{t}+\lambda_2b_2\paren{t}.\]
\end{prop}

\begin{dem}
\note{Exercice}
\end{dem}

\begin{prop}\thlabel{prop:partiesRéelleEtImaginaireDesSolutionsD'UneÉquationDifférentielle}
Soient \(I\) un intervalle de \(\R\) et \(a\in\ensclasse{0}{I}{\R}\) et \(b\in\ensclasse{0}{I}{\C}\).

Si \(z\in\ensclasse{1}{I}{\C}\) est solution de \[y\prim+a\paren{t}y=b\paren{t}\] alors \(\Re z\) est solution de \[y\prim+a\paren{t}y=\Re b\paren{t}\] et \(\Im z\) est solution de \[y\prim+a\paren{t}y=\Im b\paren{t}.\]
\end{prop}

\begin{dem}
\note{Exercice}
\end{dem}

\subsection{Méthode de résolution}

Soient \(I\) un intervalle de \(\R\) et \(a,b\in\ensclasse{0}{I}{\K}\).

On considère l'équation différentielle linéaire \[\paren{E}~y\prim+a\paren{t}y=b\paren{t}.\]

\subsubsection{Résolution de l'équation homogène associée}

Soit \(A:I\to\K\) une primitive de \(a\) sur \(I\).

Selon la \thref{prop:solutionsD'UneÉquationDifférentielleLinéaireHomogène}, les solutions de l'équation homogène associée à \(\paren{E}\) : \[\paren{E_0}~y\prim+a\paren{t}y=0\] sont les fonctions de la forme \[\fonction{y_0}{I}{\K}{t}{\lambda\e{-A\paren{t}}}\qquad\text{où }\lambda\in\K.\]

\subsubsection{Recherche d'une solution particulière de \(\paren{E}\)}

On obtient une solution particulière de \(\paren{E}\), que l'on note \(y_1\) :

\begin{itemize}
\item soit en remarquant qu'il existe une solution \guillemets{évidente} ; \\

\item soit en utilisant la méthode de variation de la constante (en utilisant la \thref{prop:principeDeSuperposition} et la \thref{prop:partiesRéelleEtImaginaireDesSolutionsD'UneÉquationDifférentielle} si cela allège les calculs).
\end{itemize}

\subsubsection{Conclusion}

Les solutions de \(\paren{E}\) sont les fonctions de la forme \[\fonction{y}{I}{\K}{t}{\lambda\e{-A\paren{t}}+y_1\paren{t}}\qquad\text{où }\lambda\in\K.\]

S'il existe une condition initiale, on trouve la valeur de la constante \(\lambda\) pour que la condition initiale soit vérifiée (cette valeur existe et est unique).

\begin{exoex}
Résoudre l'équation différentielle \[\paren{E}~y\prim-\dfrac{1}{t}y=1+\dfrac{2}{t}+t^2\sin t\] sur l'intervalle \(\Rps\).
\end{exoex}

\begin{corr}
On a l'équation homogène associée \(\paren{E_0}~y\prim-\dfrac{1}{t}y=0\).

Les solutions de \(\paren{E_0}\) sont les fonctions de la forme \(\fonctionlambda{\Rps}{\R}{t}{\lambda t}\) où \(\lambda\in\R\).

Déterminons une solution particulière de \(\paren{E_1}~y\prim-\dfrac{1}{t}y=1\).

Soit \(\lambda\in\ensclasse{1}{\Rps}{\R}\).

Posons \(y_1:t\mapsto\lambda\paren{t}t\).

On a \[\begin{aligned}
y_1\text{ est solution de }\paren{E_1}&\ssi\quantifs{\forall t\in\Rps}\lambda\prim\paren{t}t+\lambda\paren{t}-\lambda\paren{t}=1 \\
&\ssi\quantifs{\forall t\in\Rps}\lambda\prim\paren{t}t=1 \\
&\ssi\quantifs{\forall t\in\Rps}\lambda\prim\paren{t}=\dfrac{1}{t} \\
&\impr\lambda=\ln
\end{aligned}\]

Donc \(y_1:t\mapsto t\ln t\) est solution de \(\paren{E_1}\).

On remarque que \(y_2:t\mapsto-1\) est solution de \(\paren{E_2}~y\prim-\dfrac{1}{t}y=\dfrac{1}{t}\).

Déterminons une solution particulière de \(\paren{E_3}~y\prim-\dfrac{1}{t}y=t^2\e{\i t}\).

Soit \(\lambda\in\ensclasse{1}{\Rps}{\C}\).

Posons \(y_3:t\mapsto\lambda\paren{t}t\).

On a \[\begin{aligned}
y_3\text{ est solution de }\paren{E_3}&\ssi\quantifs{\forall t\in\Rps}\lambda\prim\paren{t}t+\lambda\paren{t}-\lambda\paren{t}=t^2\e{\i t} \\
&\ssi\quantifs{\forall t\in\Rps}\lambda\prim\paren{t}t=t^2\e{\i t} \\
&\ssi\quantifs{\forall t\in\Rps}\lambda\prim\paren{t}=t\e{\i t} \\
&\impr\quantifs{\forall t\in\Rps}\lambda\paren{t}=\int^tx\e{\i x}\odif{x} \\
&\ssi\quantifs{\forall t\in\Rps}\lambda\paren{t}=\croch{-\i\e{\i x}x}^t-\int^t-\i\e{\i x}\odif{x} \\
&\ssi\quantifs{\forall t\in\Rps}\lambda\paren{t}=-\i t\e{\i t}+\e{\i t} \\
&\ssi\quantifs{\forall t\in\Rps}\lambda\paren{t}=\e{\i t}\paren{1-\i t}
\end{aligned}\]

Donc \(y_3:t\mapsto t\e{\i t}\paren{1-\i t}\) est solution de \(\paren{E_3}\).

On a \[\begin{aligned}
\quantifs{\forall t\in\Rps}\Im y_3\paren{t}&=\Im\paren{t\e{\i t}\paren{1-\i t}} \\
&=\Im\paren{t\paren{\cos t+\i\sin t}\paren{1-\i t}} \\
&=t\sin t-t^2\cos t \\
&=t\paren{\sin t-t\cos t}.
\end{aligned}\]

On en déduit, par le principe de superposition, que \[y_4:t\mapsto t\ln t-2+t\paren{\sin t-t\cos t}\] est solution de \(\paren{E}\).

Finalement, les solutions de \(\paren{E}\) sont les fonctions de la forme \[\fonctionlambda{\Rps}{\R}{t}{\lambda t+t\mapsto t\ln t-2+t\paren{\sin t-t\cos t}}\] où \(\lambda\in\R\).
\end{corr}

\section{Équations différentielles linéaires du second ordre à coefficients constants}

\subsection{Cadre}

On considère :

\begin{itemize}
\item un intervalle \(I\) de \(\R\) ; \\

\item des scalaires \(a,b,c\in\K\) avec \(a\not=0\) ; \\

\item une fonction continue \(f\in\ensclasse{0}{I}{\K}\) ; \\

\item éventuellement trois éléments \(t_0\in I\) et \(v_0,w_0\in\K\) ; \\

\item l'équation différentielle linéaire du second ordre à coefficients constants \[\paren{E}~ay\seconde+by\prim+cy=f\paren{t}.\]
\end{itemize}

Résoudre\footnote{On dit aussi \guillemets{intégrer l'équation différentielle \(\paren{E}\)}.} l'équation différentielle \(\paren{E}\), c'est déterminer quelles sont les fonctions \(y\in\ensclasse{2}{I}{\K}\) qui sont solutions de \(\paren{E}\), \cad qui vérifient \[\quantifs{\forall t\in I}ay\seconde\paren{t}+by\prim\paren{t}+cy\paren{t}=f\paren{t}.\]

L'ensemble solution de l'équation différentielle \(\paren{E}\) est l'ensemble des solutions de \(\paren{E}\) : \[\fami{S}=\accol{y\in\ensclasse{2}{I}{\K}\tq\quantifs{\forall t\in I}ay\seconde\paren{t}+by\prim\paren{t}+cy\paren{t}=f\paren{t}}.\]

L'équation différentielle \(\paren{E}\) est dite homogène si \(f\) est la fonction identiquement nulle.

On appelle équation homogène associée à \(\paren{E}\) l'équation différentielle linéaire homogène du second ordre \[\paren{E_0}~ay\seconde+by\prim+cy=0.\]

Lorsque l'on rechercher la\footnote{Cette solution existe et est unique.} solution de \(\paren{E}\) qui vérifie de plus une condition initiale, on dit qu'on résout le problème de Cauchy : \[\begin{dcases}
ay\seconde+by\prim+cy=f\paren{t} \\
y\paren{t_0}=v_0 \\
y\prim\paren{t_0}=w_0
\end{dcases}\]

\begin{ex}
\begin{itemize}
\item Équation différentielle linéaire homogène du second ordre à coefficients constants : \[y\seconde+2y\prim+y=0.\]

\item Équation différentielle linéaire du second ordre à coefficients constants : \[y\seconde+2y\prim+y=\cos t.\]

\item Problème de Cauchy : \[\begin{dcases}
y\seconde+2y\prim+y=\cos t \\
y\paren{0}=0 \\
y\prim\paren{0}=1
\end{dcases}\]
\end{itemize}
\end{ex}

\subsection{Conséquences de la linéarité}

\begin{prop}[Structure de l'ensemble solution]
Soient \(I\) un intervalle de \(\R\), \(a,b,c\in\K\) tels que \(a\not=0\) et \(f\in\ensclasse{0}{I}{\K}\).

On considère l'équation différentielle linéaire \[\paren{E}~ay\seconde+by\prim+cy=f\paren{t}\] et son équation homogène associée \[\paren{E_0}~ay\seconde+by\prim+cy=0.\]

\begin{enumerate}
\item L'ensemble solution \(\fami{S}_0\) de \(\paren{E_0}\) est un espace vectoriel. \\

\item L'ensemble solution \(\fami{S}\) de \(\paren{E}\) est un espace affine de direction \(\fami{S}_0\).
\end{enumerate}
\end{prop}

\begin{dem}
\Cf \thref{dem:structureDeL'ensembleSolutionD'uneEquaDiffDuPremierOrdre} en admettant que \(\fami{S}\) est non-vide.
\end{dem}

\begin{prop}[Principe de superposition]
Soient \(I\) un intervalle de \(\R\), \(\lambda_1,\lambda_2,a,b,c\in\K\) tels que \(a\not=0\) et \(f_1,f_2\in\ensclasse{0}{I}{\K}\).

Si \(y_1\in\ensclasse{2}{I}{\K}\) est solution de \[\paren{E_1}~ay\seconde+by\prim+cy=f_1\paren{t}\] et \(y_2\in\ensclasse{2}{I}{\K}\) est solution de \[\paren{E_2}~ay\seconde+by\prim+cy=f_2\paren{t}\] alors \(\lambda_1y_1+\lambda_2y_2\) est solution de \[\paren{E}~ay\seconde+by\prim+cy=\lambda_1f_1\paren{t}+\lambda_2f_2\paren{t}.\]
\end{prop}

\begin{dem}
\note{Exercice}
\end{dem}

\begin{prop}
Soient \(I\) un intervalle de \(\R\), \(a,b,c\in\R\) avec \(a\not=0\) et \(f\in\ensclasse{0}{I}{\C}\).

Si \(z\in\ensclasse{2}{I}{\C}\) est solution de \[ay\seconde+by\prim+cy=f\paren{t}\] alors \(\Re z\) est solution de \[ay\seconde+by\prim+cy=\Re f\paren{t}\] et \(\Im z\) est solution de \[ay\seconde+by\prim+cy=\Im f\paren{t}.\]
\end{prop}

\begin{dem}
\note{Exercice}
\end{dem}

\subsection{Cas homogène}

\begin{rem}
Attention à ne jamais appliquer les résultats de ce paragraphe à des équations différentielles linéaires homogènes du second ordre à coefficients non-constants.
\end{rem}

\subsubsection{Cas complexe}

\begin{prop}
Soient \(I\) un intervalle de \(\R\) et \(a,b,c\in\C\) tels que \(a\not=0\).

On considère l'équation différentielle linéaire homogène du second ordre \[\paren{E_0}~ay\seconde+by\prim+cy=0.\]

On lui associe son équation caractéristique : \[ax^2+bx+c=0\] d'inconnue \(x\in\C\) et dont on note \(\Delta\) le discriminant.

Si \(\Delta\not=0\) alors l'équation caractéristique admet deux solutions distinctes \(\alpha,\beta\in\C\) et les solutions de \(\paren{E_0}\) sur \(I\) sont les fonctions de la forme \[\fonctionlambda{I}{\C}{t}{\lambda\e{\alpha t}+\mu\e{\beta t}}\qquad\text{où }\lambda,\mu\in\C.\]

Si \(\Delta=0\) alors l'équation caractéristique admet une unique solution (double) \(\alpha\in\C\) et les solutions de \(\paren{E_0}\) sur \(I\) sont les fonctions de la forme \[\fonctionlambda{I}{\C}{t}{\paren{\lambda t+\mu}\e{\alpha t}}\qquad\text{où }\lambda,\mu\in\C.\]
\end{prop}

\begin{dem}
Soient \(\alpha\in\C\) tel que \(a\alpha^2+b\alpha+c=0\) et \(y\in\ensclasse{2}{I}{\C}\).

On pose \(\fonction{\lambda}{I}{\C}{t}{\e{-\alpha t}y\paren{t}}\) de sorte qu'on a \(\begin{dcases}
\quantifs{\forall t\in I}y\paren{t}=\lambda\paren{t}\e{\alpha t} \\
\lambda\in\ensclasse{2}{I}{\C}
\end{dcases}\)

On a, selon la formule de Leibniz : \[\begin{aligned}
y\text{ est solution de }\paren{E_0}&\ssi\quantifs{\forall t\in I}a\paren{\lambda\seconde\paren{t}\e{\alpha t}+2\alpha\lambda\prim\paren{t}\e{\alpha t}+\underbrace{\alpha^2\lambda\paren{t}\e{\alpha t}}_{\star}}+b\paren{\lambda\prim\paren{t}\e{\alpha t}+\underbrace{\alpha\lambda\paren{t}\e{\alpha t}}_{\star}} \\
&\color{white}\ssi\quantifs{\forall t\in I}\color{black}+\underbrace{c\lambda\paren{t}}_{\star}=0 \\
&\ssi\quantifs{\forall t\in I}a\lambda\seconde\paren{t}+2a\alpha\lambda\prim\paren{t}+b\lambda\prim\paren{t}=0 \\
&\ssi\lambda\prim\text{ est solution de }\paren{E_0\prim}~z\prim+\paren{2\alpha+\dfrac{b}{a}}z=0.
\end{aligned}\]

\(\star\) : s'annulent car \(a\alpha^2+b\alpha+c=0\).

On note \(\alpha,\beta\) les racines de \(aX^2+bX+c\) (\cad \(\beta=\alpha\) si \(\Delta=0\)).

On sait que \(\alpha+\beta=\dfrac{-b}{a}\) donc \[2\alpha+\dfrac{b}{a}=2\alpha-\alpha-\beta=\alpha-\beta.\]

Si \(\Delta\not=0\) :

Résolvons \(\paren{E_0\prim}\).

Une primitive de \(t\mapsto\alpha-\beta\) est \(t\mapsto\paren{\alpha-\beta}t\) donc les solutions de \(\paren{E_0\prim}\) sont les fonctions de la forme \[t\mapsto\mu\e{-\paren{\alpha-\beta}t}=\mu\e{\paren{\beta-\alpha}t}\qquad\text{où }\mu\in\C.\]

Résolvons maintenant \(\paren{E_0}\).

On a : \[\begin{aligned}
y\text{ est solution de }\paren{E_0}&\ssi\lambda\prim\text{ est solution de }\paren{E_0\prim} \\
&\ssi\quantifs{\exists\mu\in\C;\forall t\in I}\lambda\prim\paren{t}=\mu\e{\paren{\beta-\alpha}t} \\
&\ssi\quantifs{\exists\mu,\nu\in\C;\forall t\in I}\lambda\paren{t}=\dfrac{\mu}{\beta-\alpha}\e{\paren{\beta-\alpha}t}+\nu \\
&\ssi\quantifs{\exists\mu\prim,\nu\in\C;\forall t\in I}\lambda\paren{t}=\mu\prim\e{\paren{\beta-\alpha}t}+\nu \\
&\ssi\quantifs{\exists\mu\prim,\nu\in\C;\forall t\in I}y\paren{t}=\mu\prim\e{\beta t}+\nu\e{\alpha t}
\end{aligned}\]

Les solutions de \(\paren{E_0}\) sont donc les fonctions de la forme \[t\mapsto\mu\prim\e{\beta t}+\nu\e{\alpha t}\] où \(\mu\prim,\nu\in\C\).

Si \(\Delta=0\) :

On a : \[\begin{aligned}
y\text{ est solution de }\paren{E_0}&\ssi\lambda\prim\text{ est solution de }z\prim=0 \\
&\ssi\lambda\seconde=0 \\
&\ssi\quantifs{\exists\mu\in\C;\forall t\in I}\lambda\prim\paren{t}=\mu \\
&\ssi\quantifs{\exists\mu,\nu\in\C;\forall t\in I}\lambda\paren{t}=\mu t+\nu \\
&\ssi\quantifs{\exists\mu,\nu\in\C;\forall t\in I}y\paren{t}=\paren{\mu t+\nu}\e{\alpha t}
\end{aligned}\]

Les solutions de \(\paren{E_0}\) sont donc les fonctions de la forme \[t\mapsto\paren{\mu t+\nu}\e{\alpha t}\qquad\text{où }\mu,\nu\in\C.\]
\end{dem}

\begin{exoex}
Donner les solutions complexes sur l'intervalle \(\R\) de l'équation différentielle \[\paren{E_0}~y\seconde+2y\prim+y=0.\]
\end{exoex}

\begin{corr}
L'unique solution de l'équation caractéristique \(x^2+2x+1=0\) est \(-1\).

Donc les solutions de \(\paren{E_0}\) sont les fonctions de la forme \[t\mapsto\paren{\lambda t+\mu}\e{-t}\] où \(\lambda,\mu\in\C\).
\end{corr}

\begin{exoex}
Donner les solutions complexes sur l'intervalle \(\R\) de l'équation différentielle \[\paren{E_0}~y\seconde+y\prim+y=0.\]
\end{exoex}

\begin{corr}
Les solutions de l'équation caractéristique \(x^2+x+1=0\) sont \(j\) et \(j^2\).

Donc les solutions de \(\paren{E_0}\) sont les fonctions de la forme \[t\mapsto\lambda\e{jt}+\mu\e{j^2t}\] où \(\lambda,\mu\in\C\).
\end{corr}

\subsubsection{Cas réel}

\begin{prop}
Soient \(I\) un intervalle de \(\R\) et \(a,b,c\in\R\) tels que \(a\not=0\).

On considère l'équation différentielle linéaire homogène du second ordre \[\paren{E_0}~ay\seconde+by\prim+cy=0.\]

On lui associe son équation caractéristique : \[ax^2+bx+c=0\] d'inconnue \(x\in\R\) et dont on note \(\Delta\) le discriminant.

Si \(\Delta>0\) alors l'équation caractéristique admet deux solutions distinctes \(\alpha,\beta\in\R\) et les solutions de \(\paren{E_0}\) sur \(I\) sont les fonctions de la forme \[\fonctionlambda{I}{\R}{t}{\lambda\e{\alpha t}+\mu\e{\beta t}}\qquad\text{où }\lambda,\mu\in\R.\]

Si \(\Delta=0\) alors l'équation caractéristique admet une unique solution (double) \(\alpha\in\R\) et les solutions de \(\paren{E_0}\) sur \(I\) sont les fonctions de la forme \[\fonctionlambda{I}{\R}{t}{\paren{\lambda t+\mu}\e{\alpha t}}\qquad\text{où }\lambda,\mu\in\R.\]

Si \(\Delta<0\) alors l'équation caractéristique admet deux solutions complexes conjuguées \(\alpha+\i\beta\) et \(\alpha-\i\beta\) avec \(\alpha,\beta\in\R\) et les solutions de \(\paren{E_0}\) sont les fonctions de la forme \[\fonctionlambda{I}{\R}{t}{\paren{\lambda\cos\paren{\beta t}+\mu\sin\paren{\beta t}}\e{\alpha t}}\qquad\text{où }\lambda,\mu\in\R.\]
\end{prop}

\begin{dem}
\note{Exercice} (déduire le cas réel du cas complexe).
\end{dem}

\begin{exoex}
Donner les solutions réelles sur l'intervalle \(\R\) de l'équation différentielle \[\paren{E_0}~y\seconde+y\prim+y=0.\]
\end{exoex}

\begin{corr}
Les solutions de l'équation caractéristique \(x^2+x+1=0\) sont \(\dfrac{-1}{2}\pm\i\dfrac{\sqrt{3}}{2}\).

Donc les solutions de \(\paren{E_0}\) sont les fonctions de la forme \[t\mapsto\paren{\lambda\cos\paren{\dfrac{\sqrt{3}}{2}t}+\mu\sin\paren{\dfrac{\sqrt{3}}{2}t}}\e{\frac{-t}{2}}\] avec \(\lambda,\mu\in\R\).
\end{corr}

\subsection{Second membre de la forme \guillemets{polynôme \(\times\) exponentielle}}

\begin{meth}
Soient \(a,b,c,\gamma\in\K\) tels que \(a\not=0\) et \(P\in\poly\).

Pour trouver une solution de \[\paren{E}~ay\seconde+by\prim+cy=P\paren{t}\e{\gamma t},\] on la cherche sous la forme \[Q\paren{t}\e{\gamma t}\] avec \(Q\in\poly\) de degré \[\begin{dcases}
\deg P &\text{si }\gamma\text{ n'est pas racine de }aX^2+bX+c \\
\deg P+1 &\text{si }\gamma\text{ est racine simple de }aX^2+bX+c \\
\deg P+2 &\text{si }\gamma\text{ est racine double de }aX^2+bX+c \\
\end{dcases}\]
\end{meth}

\begin{exoex}
Résoudre les équations différentielles suivantes :

\begin{enumerate}
\item \(\paren{E}~y\seconde-3y\prim+2y=\ch t\) \\

\item \(\paren{E}~y\seconde-y=t\sh t\) \\

\item \(\paren{E}~y\seconde+y=\sin t\)
\end{enumerate}
\end{exoex}

\begin{corr}[1]
Résolvons l'équation homogène associée à \(\paren{E}\) : \[\paren{E_0}~y\seconde-3y\prim+2y=0.\]

L'équation caractéristique \(x^2-3x+2=0\) admet pour solutions \(1\) et \(2\).

Les solutions de \(\paren{E_0}\) sont donc les fonctions de la forme \[\fonctionlambda{\R}{\R}{t}{\lambda\e{t}+\mu\e{2t}}\qquad\text{où }\lambda,\mu\in\R.\]

Cherchons une solutions particulière de \(\paren{E}\).

On pose \[\paren{E_1}~y\seconde-3y\prim+2y=\e{t}\qquad\text{et}\qquad\paren{E_2}~y\seconde-3y\prim+2y=\e{-t}.\]

Déterminons une solution particulière de \(\paren{E_1}\).

\begin{brouill}
On a \(P\paren{t}\e{\gamma t}\) avec \(\begin{dcases}
P=1\text{ donc }\deg P=0 \\
\gamma=1\text{ racine simple de }X^2-3X+2
\end{dcases}\)

Donc on cherche une solution de la forme \[Q\paren{t}\e{\gamma t}\] avec \(\deg Q=0+1=1\), \cad de la forme \[\paren{at+\cancel{b}}\e{t}.\]

\textit{Remarque :} on peut ici considérer que \(b=0\) car \(b\e{t}\) est solution de \(\paren{E_0}\).
\end{brouill}

Soit \(a\in\R\).

On pose \(y_1:t\mapsto at\e{t}\).

On a \[\quantifs{\forall t\in\R}\begin{dcases}
y_1\prim\paren{t}=a\e{t}\paren{1+t} \\
y_1\seconde\paren{t}=a\e{t}\paren{t+2}
\end{dcases}\]

D'où : \[\begin{aligned}
y_1\text{ est solution de }\paren{E_1}&\ssi\quantifs{\forall t\in\R}a\e{t}\paren{t+2}-3a\e{t}\paren{t+1}+2at\e{t}=\e{t} \\
&\ssi\quantifs{\forall t\in\R}at\e{t}+2a\e{t}-3at\e{t}-3a\e{t}+2at\e{t}=\e{t} \\
&\ssi\quantifs{\forall t\in\R}-a\e{t}=\e{t} \\
&\ssi a=-1
\end{aligned}\]

Donc \(y_1:t\mapsto-t\e{t}\) est solution de \(\paren{E_1}\).

Déterminons une solution particulière de \(\paren{E_2}\).

\begin{brouill}
On a \(P\paren{t}\e{\gamma t}\) avec \(\begin{dcases}
P=1\text{ donc }\deg P=0 \\
\gamma=-1\text{ pas racine de }X^2-3X+2
\end{dcases}\)

Donc on cherche une solution de la forme \[Q\paren{t}\e{\gamma t}\] avec \(\deg Q=0\), \cad de la forme \[a\e{-t}.\]
\end{brouill}

Soit \(a\in\R\).

On pose \(y_2:t\mapsto a\e{-t}\).

On a \[\quantifs{\forall t\in\R}\begin{dcases}
y_2\prim\paren{t}=-a\e{-t} \\
y_2\seconde\paren{t}=a\e{-t}
\end{dcases}\]

D'où : \[\begin{aligned}
y_2\text{ est solution de }\paren{E_2}&\ssi\quantifs{\forall t\in\R}a\e{-t}+3a\e{-t}+2a\e{-t}=\e{-t} \\
&\ssi\quantifs{\forall t\in\R}6a\e{-t}=\e{-t} \\
&\ssi a=\dfrac{1}{6}
\end{aligned}\]

Donc \(y_2:t\mapsto\dfrac{\e{-t}}{6}\) est solution de \(\paren{E_2}\).

Ainsi, comme on a \(\quantifs{\forall t\in\R}\ch t=\dfrac{\e{t}+\e{-t}}{2}\), d'après le principe de superposition : \[y_3:t\mapsto-\dfrac{t\e{t}}{2}+\dfrac{\e{-t}}{12}\] est solution de \(\paren{E}\).

Finalement, les solutions de \(\paren{E}\) sont les fonctions de la forme \[t\mapsto\lambda\e{t}+\mu\e{2t}-\dfrac{t\e{t}}{2}+\dfrac{\e{-t}}{12}\qquad\text{où }\lambda,\mu\in\R.\]
\end{corr}

\begin{corr}[2]
Résolvons l'équation homogène associée à \(\paren{E}\) : \[\paren{E_0}~y\seconde-y=0.\]

Les solutions de l'équation caractéristique \(x^2-1=0\) sont \(1\) et \(-1\).

Les solutions de \(\paren{E_0}\) sont donc les fonctions de la forme \[t\mapsto\lambda\e{t}+\mu\e{-t}\qquad\text{où }\lambda,\mu\in\R.\]

Cherchons une solution particulière de \(\paren{E}\).

On pose \[\paren{E_1}~y\seconde-y=t\e{t}\qquad\text{et}\qquad\paren{E_2}~y\seconde-y=t\e{-t}\]

Déterminons une solution particulière de \(\paren{E_1}\).

\begin{brouill}
On a \(P\paren{t}\e{\gamma t}\) avec \(\begin{dcases}
P=X\text{ donc }\deg P=1 \\
\gamma=1\text{ racine simple de }X^2-1
\end{dcases}\)

Donc on cherche une solution de la forme \[Q\paren{t}\e{\gamma t}\] avec \(\deg Q=1+1=2\), \cad de la forme \[\paren{at^2+bt}\e{t}.\]
\end{brouill}

Soient \(a,b\in\R\).

On pose \(y_1:t\mapsto\paren{at^2+bt}\e{t}\).

On a \[\quantifs{\forall t\in\R}\begin{dcases}
y_1\prim\paren{t}=\e{t}\paren{2at+b+at^2+bt}=\e{t}\paren{at^2+\paren{2a+b}t+b} \\
y_1\seconde\paren{t}=\e{t}\paren{at^2+\paren{2a+b}t+b+2at+2a+b}=\e{t}\paren{at^2+\paren{4a+b}t+2\paren{a+b}}
\end{dcases}\]

D'où : \[\begin{aligned}
y_1\text{ est solution de }\paren{E_1}&\ssi\quantifs{\forall t\in\R}\e{t}\paren{at^2+\paren{4a+b}t+2\paren{a+b}}-\e{t}\paren{at^2+bt}=t\e{t} \\
&\ssi\quantifs{\forall t\in\R}4at+2a+2b=t \\
&\ssi\quantifs{\forall t\in\R}\paren{4a-1}t+2a+2b=0 \\
&\impr\begin{dcases}
2a+2b=0 \\
4a-1=0
\end{dcases} \\
&\ssi\begin{dcases}
a=\dfrac{1}{4} \\
b=\dfrac{-1}{4}
\end{dcases}
\end{aligned}\]

Donc \(y_1:t\mapsto\dfrac{1}{4}\paren{t^2-t}\e{t}\) est solution de \(\paren{E_1}\).

Déterminons une solution particulière de \(\paren{E_2}\).

\begin{brouill}
On a \(P\paren{t}\e{\gamma t}\) avec \(\begin{dcases}
P=X\text{ donc }\deg P=1 \\
\gamma=-1\text{ racine simple de }X^2-1
\end{dcases}\)

Donc on cherche une solution de la forme \[Q\paren{t}\e{\gamma t}\] avec \(\deg Q=2\) donc de la forme \[\paren{at^2+bt}\e{-t}.\]
\end{brouill}

Soient \(a,b\in\R\).

On pose \(y_2:t\mapsto\paren{at^2+bt}\e{-t}\).

On a \[\quantifs{\forall t\in\R}\begin{dcases}
y_2\prim\paren{t}=\e{-t}\paren{-at^2-bt+2at+b} \\
y_2\seconde\paren{t}=\e{-t}\paren{at^2+bt-2at-b-2at-b+2a}=\e{-t}\paren{at^2+\paren{b-4a}t+2a-2b}
\end{dcases}\]

D'où : \[\begin{aligned}
y_2\text{ est solution de }\paren{E_2}&\ssi\quantifs{\forall t\in\R}\e{-t}\paren{at^2+\paren{b-4a}t+2a-2b}-\paren{at^2+bt}\e{-t}=t\e{-t} \\
&\ssi\quantifs{\forall t\in\R}-\paren{4a+1}t+2a-2b=0 \\
&\impr\begin{dcases}
-4a-1=0 \\
2a-2b=0
\end{dcases} \\
&\ssi\begin{dcases}
a=\dfrac{-1}{4} \\
b=\dfrac{-1}{4}
\end{dcases}
\end{aligned}\]

Donc \(y_2:t\mapsto\dfrac{-1}{4}\paren{t^2+t}\e{-t}\) est solution de \(\paren{E_2}\).

Finalement, les solutions de \(\paren{E}\) sont les fonctions de la forme \[t\mapsto\lambda\e{t}+\mu\e{-t}+\dfrac{1}{8}\paren{t^2-t}\e{t}+\dfrac{1}{8}\paren{t^2+t}\e{-t}\qquad\text{où }\lambda,\mu\in\R.\]
\end{corr}

\begin{corr}[3]
Résolvons l'équation homogène associée à \(\paren{E}\) : \[\paren{E_0}~y\seconde+y=0.\]

Les solutions de l'équation caractéristique \(x^2+x=0\) sont \(\i\) et \(-\i\).

Les solutions de \(\paren{E_0}\) sont donc les fonctions de la forme \[t\mapsto\lambda\cos t+\mu\sin t\qquad\text{où }\lambda,\mu\in\R.\]

Déterminons une solution particulière de \(\paren{E_1}~y\seconde+y=\e{\i t}\).

\begin{brouill}
On a \(P\paren{t}\e{\gamma t}\) avec \(\begin{dcases}
P=1\text{ donc }\deg P=0 \\
\gamma=\i\text{ racine simple de }X^2+X
\end{dcases}\)

Donc on cherche une solution de la forme \[Q\paren{t}\e{\gamma t}\] avec \(\deg Q=1\), \cad de la forme \[at\e{\i t}.\]
\end{brouill}

Soit \(a\in\R\).

On pose \(y_1:t\mapsto at\e{\i t}\).

On a \[\quantifs{\forall t\in\R}\begin{dcases}
y_1\prim\paren{t}=a\paren{\e{\i t}+\i t\e{\i t}}=a\e{\i t}\paren{\i t+1} \\
y_1\seconde\paren{t}=a\paren{\i\e{\i t}\paren{\i t+1}+\i\e{\i t}}=a\i\e{\i t}\paren{\i t+2}
\end{dcases}\]

D'où : \[\begin{aligned}
y_1\text{ est solution de }\paren{E_1}&\ssi\quantifs{\forall t\in\R}a\i\e{\i t}\paren{\i t+2}+at\e{\i t}=\e{\i t} \\
&\ssi2a\i=1 \\
&\impr a=\dfrac{-\i}{2}
\end{aligned}\]

Donc \(y_1:t\mapsto\dfrac{-\i}{2}t\e{\i t}\) est solution de \(\paren{E_1}\).

De plus, on a : \[\begin{aligned}
\quantifs{\forall t\in\R}\Im y_1\paren{t}&=\Im\paren{\dfrac{-\i}{2}t\paren{\cos t+\i\sin t}} \\
&=\Im\paren{\dfrac{-\i}{2}t\cos t+\dfrac{1}{2}t\sin t} \\
&=\dfrac{-1}{2}t\cos t
\end{aligned}\]

Finalement, les solutions de \(\paren{E}\) sont les fonctions de la forme \[t\mapsto\lambda\cos t+\mu\sin t-\dfrac{1}{2}t\cos t\qquad\text{où }\lambda,\mu\in\R.\]
\end{corr}

\section{Changements de variable}

\Cf \thref{exo:exempleChangementDeVariableCours}.

\section{Problèmes de raccord}

\Cf \thref{exo:exempleProblèmesDeRaccordCours}.

\chapter{Espaces vectoriels de dimension finie}

\minitoc

On considère un corps \(\K\) (en pratique, \(\K=\R\) ou \(\C\), voire \(\Q\)).

\section{Familles de vecteurs}

\subsection{Quelques rappels}

\begin{rappel}
Soient \(E\) et \(F\) deux \(\K\)-espaces vectoriels, \(u\in\L{E}{F}\) et \(x_1,\dots,x_n\in E\).

On a \[u\paren{\Vect{x_1,\dots,x_n}}=\Vect{u\paren{x_1},\dots,u\paren{x_n}}.\]
\end{rappel}

\begin{rappel}\thlabel{rappel:familleLibreSiVecteurAjoutéPasDansLeVectDesAutres}
Soient \(E\) un \(\K\)-espace vectoriel, \(\paren{x_1,\dots,x_n}\in E^n\) une famille libre de \(E\) et \(x_{n+1}\in E\).

Alors \(\paren{x_1,\dots,x_{n+1}}\) est libre si, et seulement si, \(x_{n+1}\not\in\Vect{x_1,\dots,x_n}\).
\end{rappel}

\begin{rappel}\thlabel{rappel:familleGénératriceSsiTousLesVecteursSontDansLeVectD'UneFamilleQuelconque}
Soient \(E\) un \(\K\)-espace vectoriel, \(n,m\in\Ns\), \(\paren{x_1,\dots,x_n}\in E^n\) une famille quelconque de vecteurs de \(E\) et \(\paren{y_1,\dots,y_m}\in E^m\) une famille génératrice de \(E\).

Alors \(\paren{x_1,\dots,x_n}\) est une famille génératrice de \(E\) si, et seulement si : \[\quantifs{\forall j\in\interventierii{1}{m}}y_j\in\Vect{x_1,\dots,x_n}.\]
\end{rappel}

\begin{rappel}\thlabel{rappel:imageD'UneBaseParUneApplicationLinéaire}
Soient \(E\) et \(F\) deux \(\K\)-espaces vectoriels, \(u\in\L{E}{F}\) et \(\paren{x_1,\dots,x_p}\in E^p\) une famille de vecteurs de \(E\).

On a :

\begin{enumerate}
\item Si \(\paren{x_1,\dots,x_p}\) est libre et \(u\) injective, alors \(\paren{u\paren{x_1},\dots,u\paren{x_p}}\) est libre. \\

\item Si \(\paren{x_1,\dots,x_p}\) engendre \(E\) et \(u\) est surjective, alors \(\paren{u\paren{x_1},\dots,u\paren{x_p}}\) engendre \(F\). \\

\item Si \(\paren{x_1,\dots,x_p}\) est une base de \(E\) et \(u\) un isomorphisme, alors \(\paren{u\paren{x_1},\dots,u\paren{x_p}}\) est une base de \(F\).
\end{enumerate}
\end{rappel}

\subsection{Cardinaux des familles libres / génératrices}

\begin{lem}\thlabel{lem:cardFamilleLibreInférieurCardFamilleGénératrice}
Soient \(E\) un \(\K\)-espace vectoriel, \(n,m\in\N\) et \(x_1,\dots,x_n,y_1,\dots,y_m\in E\).

On suppose que \(\paren{x_1,\dots,x_n}\in E^n\) est libre et que \(\paren{y_1,\dots,y_m}\) est génératrice de \(E\).

Alors on a : \[n\leq m.\]
\end{lem}

\begin{dem}
On raisonne par l'absurde : supposons \(n>m\).

\textit{Idée :} On va montrer qu'on peut remplacer un par un chaque vecteur de la famille génératrice par un vecteur de la famille libre, et cela en conservant une famille génératrice de \(E\). Les vecteurs de la famille libre qui restent seront des combinaisons linéaires des autres vecteurs de la famille libre : contradiction.

\textit{Démonstration formelle :}

Pour tout \(k\in\interventierii{0}{m}\), on note \(\P{k}\) la propriété : \[\quantifs{\exists i_{k+1},\dots,i_m\in\interventierii{1}{m}}\paren{x_1,\dots,x_k,y_{i_{k+1}},\dots,y_{i_m}}\text{ est génératrice de }E.\]

Montrons \(\P{k}\) pour tout \(k\in\interventierii{0}{m}\) par une récurrence finie sur \(k\).

La propriété \(\P{0}\) est vraie : il suffit de poser \(i_1=1,\dots,i_m=m\) et la famille \(\paren{y_{i_1},\dots,y_{i_m}}\) est bien une famille génératrice de \(E\).

Soit \(k\in\interventierii{0}{m-1}\) tel que \(\P{k}\).

Considérons \(i_{k+1},\dots,i_m\in\interventierii{1}{m}\) tels que \[\fami{G}=\paren{x_1,\dots,x_k,y_{i_{k+1}},\dots,y_{i_m}}\] soit une famille génératrice de \(E\).

Soient \(\lambda_1,\dots,\lambda_m\in\K\) tels que \[\paren{R}~x_{k+1}=\lambda_1x_1+\dots+\lambda_kx_k+\lambda_{k+1}y_{i_{k+1}}+\dots+\lambda_my_{i_m}\] (de tels scalaires existent par hypothèse).

Les coefficients \(\lambda_{k+1},\dots,\lambda_m\) ne sont pas tous nuls, car sinon on aurait \(x_{k+1}\in\Vect{x_1,\dots,x_k}\) alors que la famille \(\paren{x_1,\dots,x_n}\) est libre.

Soit \(p\in\interventierii{k+1}{m}\) tel que \(\lambda_p\not=0\).

Quitte à renuméroter les entiers \(i_{k+1},\dots,i_m\), on peut supposer \(p=k+1\).

Montrons que \[\fami{F}=\paren{x_1,\dots,x_k,x_{k+1},y_{i_{k+2}},\dots,y_{i_m}}\] est une famille génératrice de \(E\).

Pour cela, selon le \thref{rappel:familleGénératriceSsiTousLesVecteursSontDansLeVectD'UneFamilleQuelconque}, il suffit de montrer que chaque vecteur de la famille \(\fami{G}\) appartient à \(\Vect{\fami{F}}\).

On a, selon la relation \(\paren{R}\) : \[y_{i_{k+1}}=\dfrac{-\lambda_1}{\lambda_{k+1}}x_1+\dots+\dfrac{-\lambda_k}{\lambda_{k+1}}x_k+\dfrac{1}{\lambda_{k+1}}x_{k+1}+\dfrac{-\lambda_{k+2}}{\lambda_{k+1}}y_{i_{k+2}}+\dots+\dfrac{-\lambda_m}{\lambda_{k+1}}y_{i_m}\] donc \(y_{i_{k+1}}\in\Vect{\fami{F}}\).

C'est clair pour les autres vecteurs de \(\fami{G}\) (ils appartiennent à \(\Vect{\fami{F}}\) car ils appartiennent à \(\fami{F}\)).

Ainsi, la famille \(\fami{F}\) est génératrice de \(E\) et la propriété \(\P{k+1}\) est donc vraie.

On a donc, par récurrence (finie) : \(\quantifs{\forall k\in\interventierii{0}{m}}\P{k}\text{ est vraie}\).

En particulier (en prenant \(k=m\)) : la famille \(\paren{x_1,\dots,x_m}\) est génératrice de \(E\) donc \[x_{m+1}\in\Vect{x_1,\dots,x_m}.\]

On a donc bien une contradiction car la famille \(\paren{x_1,\dots,x_n}\) est libre.

Donc \(n\leq m\).
\end{dem}

\subsection{Théorème de la base incomplète}

\begin{theo}\thlabel{theo:baseÀPartirD'uneFamilleGénératriceDontUneSousFamilleEstLibre}
Soient \(E\) un \(\K\)-espace vectoriel, \(n,m\in\N\) et \(x_1,\dots,x_n,y_1,\dots,y_m\in E\).

On suppose que \(\paren{x_1,\dots,x_n}\in E^n\) est libre et que \(\paren{x_1,\dots,x_n,y_1,\dots,y_m}\) est génératrice de \(E\).

Alors il existe des éléments \(i_1,\dots,i_r\in\interventierii{1}{m}\) tels que la famille \[\paren{x_1,\dots,x_n,y_{i_1},\dots,y_{i_r}}\] soit une base de \(E\).
\end{theo}

\begin{dem}
\textit{Idée :} si \(\paren{x_1,\dots,x_n}\) est déjà une base de \(E\), le théorème est vrai (en prenant \(r=0\)). Sinon, la famille libre \(\paren{x_1,\dots,x_n}\) n'est pas une famille génératrice de \(E\). Selon le \thref{rappel:familleGénératriceSsiTousLesVecteursSontDansLeVectD'UneFamilleQuelconque}, il existe un élément de la famille génératrice \(\paren{x_1,\dots,x_n,y_1,\dots,y_m}\) qui n'est pas combinaison linéaire de \(x_1,\dots,x_n\) : considérons un tel élément \(y_{i_1}\) (où \(i_1\in\interventierii{1}{m}\)). Selon le \thref{rappel:familleLibreSiVecteurAjoutéPasDansLeVectDesAutres}, la famille \(\paren{x_1,\dots,x_n,y_{i_1}}\) est libre. On continue ensuite à rajouter des éléments parmi \(y_1,\dots,y_m\) à la famille libre jusqu'à ce qu'on obtienne une famille génératrice (à chaque étape, la famille reste libre), et on finit par obtenir une base.

\textit{Démonstration formelle :}

Posons : \[J=\accol{p\in\interventierii{0}{m}\tq\quantifs{\exists i_1,\dots,i_p\in\interventierii{1}{m}}\paren{x_1,\dots,x_n,y_{i_1},\dots,y_{i_p}}\text{ est libre}}.\]

L'ensemble \(J\) est une partie de \(\N\) non-vide (car elle contient \(0\)) et majorée (par \(m\)). Elle admet donc un plus grand élément ; notons le \(r\).

Soient \(i_1,\dots,i_r\in\interventierii{1}{m}\) tels que la famille \(\paren{x_1,\dots,x_n,y_{i_1},\dots,y_{i_r}}\) soit libre.

Montrons que la famille \(\paren{x_1,\dots,x_n,y_{i_1},\dots,y_{i_r}}\) est une base de \(E\) : il s'agit de montrer qu'elle est génératrice de \(E\).

Raisonnons par l'absurde : on suppose qu'elle ne l'est pas.

D'après le \thref{rappel:familleGénératriceSsiTousLesVecteursSontDansLeVectD'UneFamilleQuelconque}, la famille \(\paren{x_1,\dots,x_n,y_1,\dots,y_m}\) étant génératrice, elle contient un élément qui n'est pas combinaison linéaire de \(x_1,\dots,x_n,y_{i_1},\dots,y_{i_r}\).

Considérons un tel élément \(y_{i_{r+1}}\) (où \(i_{r+1}\in\interventierii{1}{m}\)).

Finalement, la famille \(\paren{x_1,\dots,x_n,y_{i_1},\dots,y_{i_r}}\) est libre, on lui ajoute un vecteur qui n'est pas combinaison linéaire de ses éléments, donc d'après le \thref{rappel:familleLibreSiVecteurAjoutéPasDansLeVectDesAutres}, la famille qu'on obtient est libre : \[\paren{x_1,\dots,x_n,y_{i_1},\dots,y_{i_{r+1}}}\text{ est libre}.\]

Donc \(r+1\) appartient à \(J\), ce qui contredit le fait que \(r\) soit le plus grand élément de \(J\).

Donc la famille libre \(\paren{x_1,\dots,x_n,y_{i_1},\dots,y_{i_r}}\) est génératrice de \(E\) : c'est une base.
\end{dem}

\begin{theo}[Reformulation]
Soient \(E\) un \(\K\)-espace vectoriel, \(\paren{x_k}_{k\in K}\in E^K\) une famille de vecteurs de \(E\) indicée par un ensemble fini \(K\) et \(I\subset K\).

On suppose que \(\paren{x_k}_{k\in I}\) est libre et que \(\paren{x_k}_{k\in K}\) est génératrice de \(E\).

Alors il existe un ensemble \(J\) tel que \[I\subset J\subset K\qquad\text{et}\qquad\paren{x_k}_{k\in J}\text{ est une base de }E.\]
\end{theo}

\begin{cor}[Théorème de la base extraite]\thlabel{cor:théorèmeDeLaBaseExtraite}
De toute famille génératrice finie d'un \(\K\)-espace vectoriel, on peut extraire une base.
\end{cor}

\begin{dem}
Découle du \thref{theo:baseÀPartirD'uneFamilleGénératriceDontUneSousFamilleEstLibre} (en prenant \(n=0\)).
\end{dem}

\begin{cor}[Théorème de la base incomplète]
Soit \(E\) un \(\K\)-espace vectoriel.

Si \(E\) admet une famille génératrice finie, alors toute famille libre de \(E\) peut être complétée en une base de \(E\) (finie).
\end{cor}

\begin{dem}
Découle du \thref{theo:baseÀPartirD'uneFamilleGénératriceDontUneSousFamilleEstLibre}.
\end{dem}

\section{Dimension}

\subsection{Définition}

\begin{deftheo}
Soit \(E\) un \(\K\)-espace vectoriel.

Si \(E\) admet une base finie, alors toutes les bases de \(E\) sont finies et on le même nombre d'éléments, appelé dimension de \(E\) et noté \(\dim E\).

Si \(E\) n'admet aucune base finie, alors on dit que \(E\) est de dimension infinie et on pose \(\dim E=\pinf\).
\end{deftheo}

\begin{dem}
Supposons que \(\fami{B}=\paren{e_1,\dots,e_n}\) est une base finie de \(E\).

Soit \(\fami{B}\prim=\paren{x_i}_{i\in I}\) une base de \(E\) (où \(I\) est un ensemble quelconque).

Montrons que \(\Card I\leq n\).

Par l'absurde, si \(\Card I\geq n+1\) alors il existe \(i_1,\dots,i_{n+1}\in I\) deux à deux distincts.

On a \(\begin{dcases}
\paren{x_{i_1},\dots,x_{i_{n+1}}}\text{ est une famille libre de vecteurs de }E \\
\paren{e_1,\dots,e_n}\text{ est une famille génératrice de }E
\end{dcases}\) : contradiction selon le \thref{lem:cardFamilleLibreInférieurCardFamilleGénératrice}.

En particulier, on a montré que \(I\) est fini.

Montrons que \(\Card I\geq n\).

On a \[\begin{dcases}
\paren{e_1,\dots,e_n}\text{ est une famille libre de vecteurs de }E \\
\paren{x_i}_{i\in I}\text{ est une famille génératrice de }E
\end{dcases}\]

Donc \(n\leq\Card I\) selon le \thref{lem:cardFamilleLibreInférieurCardFamilleGénératrice}.

Donc \(\Card I=n\).

D'où le résultat.
\end{dem}

\begin{prop}[Espaces vectoriels de dimension infinie]
Soit \(E\) un \(\K\)-espace vectoriel.

Les propositions suivantes sont équivalentes :

\begin{enumerate}
\item \(E\) est de dimension infinie (\ie \(E\) n'admet aucune base finie) \\

\item \(E\) n'admet aucune famille génératrice finie \\

\item \(\quantifs{\forall n\in\N;\exists x_1,\dots,x_n\in E}\paren{x_1,\dots,x_n}\text{ est une famille libre}\) \\

\item Il existe une famille libre infinie de vecteurs de \(E\)
\end{enumerate}
\end{prop}

\begin{dem}[(2) \(\imp\) (1)]
Claire.
\end{dem}

\begin{dem}[(1) \(\imp\) (2)]
Découle du théorème de la base extraite (\thref{cor:théorèmeDeLaBaseExtraite}).
\end{dem}

\begin{dem}[(4) \(\imp\) (3)]
Claire car toute sous-famille d'une famille libre est libre.
\end{dem}

\begin{dem}[(3) \(\imp\) (2)]
Supposons (3).

Par l'absurde, soit \(\paren{x_1,\dots,x_m}\) une famille génératrice finie de \(E\) (où \(m\in\N\)).

Selon (3) et le \thref{lem:cardFamilleLibreInférieurCardFamilleGénératrice}, on a \[\quantifs{\forall n\in\N}n\leq m\text{ : contradiction}.\]
\end{dem}

\begin{dem}[(2) \(\imp\) (4)]
Supposons (2).

On construit par récurrence une suite \(\paren{x_n}_{n\in\N}\in E^\N\) de la façon suivante :

On considère \(x_0\in E\excluant\accol{0}\).

Soient \(n\in\N\) et \(x_0,\dots,x_n\in E\) tels que \(\paren{x_0,\dots,x_n}\) est libre.

Selon (2), \(\paren{x_0,\dots,x_n}\) n'est pas une famille génératrice de \(E\).

Donc il existe \(x_{n+1}\in E\) tel que \(x_{n+1}\not\in\Vect{x_0,\dots,x_n}\).

Donc \(\paren{x_0,\dots,x_{n+1}}\) est libre.

On construit ainsi par récurrence une suite \(\paren{x_n}_{n\in\N}\in E^\N\) telle que \[\quantifs{\forall n\in\N}\paren{x_0,\dots,x_n}\text{ est libre}.\]

Déduisons-en que \(\paren{x_n}_{n\in\N}\) est libre.

Soit \(I\subset\N\) tel que \(I\) est fini.

Montrons que \(\paren{x_n}_{n\in I}\) est libre.

Soit \(N\in\N\) un majorant de \(I\).

\(\paren{x_n}_{n\in I}\) est une sous-famille de la famille libre \(\paren{x_n}_{n\in\interventierii{0}{N}}\).

Donc \(\paren{x_n}_{n\in I}\) est libre.

Donc \(\paren{x_n}_{n\in\N}\) est libre.
\end{dem}

\begin{defi}
Soit \(E\) un \(\K\)-espace vectoriel.

On a \(\dim E=0\) si, et seulement si, \(E\) est l'espace vectoriel nul : \(E=\accol{0_E}\).

Si \(\dim E=1\), on dit que \(E\) est une droite vectorielle (cela revient à dire qu'on a \(E=\Vect{x}\) où \(x\) est un vecteur non-nul).

Si \(\dim E=2\), on dit que \(E\) est un plan vectoriel (cela revient à dire qu'on a \(E=\Vect{x,y}\) où \(x\) et \(y\) sont deux vecteurs non-colinéaires).
\end{defi}

\begin{defi}
On appelle dimension d'un sous-espace affine la dimension de sa direction.

Une droite affine est un sous-espace affine de dimension \(1\).

Un plan affine est un sous-espace affine de dimension \(2\).
\end{defi}

\subsection{Exemples}

\begin{exoex}
Donner la dimension des espaces vectoriels suivants :

\begin{enumerate}
\item \(\K^n\) (où \(n\in\Ns\)) \\

\item \(\polydeg{n}\) (où \(n\in\N\)) \\

\item \(\poly\) \\

\item \(\ensclasse{\infty}{\R}{\R}\) \\

\item On prend \(\K=\R\) ou \(\C\).

L'ensemble solution \(\fami{S}_0\) de l'équation différentielle linéaire homogène du premier ordre : \[\paren{E_0}~y\prim+a\paren{t}y=0\] où \(I\) est un intervalle de \(\R\) tel que \(\Card I\geq2\) et \(a\in\ensclasse{0}{I}{\K}\). \\

\item On prend \(\K=\R\) ou \(\C\).

L'ensemble solution \(\fami{S}_0\) de l'équation différentielle linéaire homogène du second ordre : \[\paren{E_0}~ay\seconde+by\prim+cy=0\] où \(a,b,c\in\K\) tels que \(a\not=0\).
\end{enumerate}
\end{exoex}

\begin{corr}
\begin{enumerate}
\item On a \(\dim\K^n=n\) car la base canonique de \(\K^n\) possède \(n\) éléments. \\

\item On a \(\dim\polydeg{n}=n+1\) car la base canonique de \(\polydeg{n}\) possède \(n+1\) éléments. \\

\item On a \(\dim\poly=\pinf\) car \(\paren{X^n}_{n\in\N}\) est une famille libre infinie de vecteurs de \(\poly\). \\

\item On a \(\dim\ensclasse{\infty}{\R}{\R}=\pinf\) car \(\paren{t\mapsto\e{\lambda t}}_{\lambda\in\R}\) est une famille libre infinie de vecteurs de \(\ensclasse{\infty}{\R}{\R}\). \\

\item On a vu que \(\fami{S}_0\) admet pour base la famille \(\paren{f_0}\) où \(\fonction{f_0}{I}{\K}{t}{\e{-A\paren{t}}}\) où \(A\) est une primitive de \(a\). Donc \(\dim\fami{S}_0=1\). \\

\item De même, on a \(\dim\fami{S}_0=2\) (\cf chapitre \ref{chap:équationsDifférentielles}).
\end{enumerate}
\end{corr}

\begin{prop}
Soit \(E\) un \(\C\)-espace vectoriel de dimension finie.

On sait que \(E\) est naturellement un \(\R\)-espace vectoriel.

Notons \(\dim_\C E\) la dimension de \(E\) comme \(\C\)-espace vectoriel et \(\dim_\R E\) la dimension de \(E\) comme \(\R\)-espace vectoriel.

Alors on a : \[\dim_\R E=2\dim_\C E.\]
\end{prop}

\begin{dem}
Soit \(\fami{B}=\paren{e_1,\dots,e_n}\) une base de \(E\) en tant que \(\C\)-espace vectoriel.

On a \(n=\dim_\C E\).

Posons \(\fami{B}\prim=\paren{e_1,\i e_1,\dots,e_n,\i e_n}\).

Montrons que \(\fami{B}\prim\) est une base de \(E\) en tant que \(\R\)-espace vectoriel.

Montrons que \(\fami{B}\prim\) est génératrice de \(E\) sur \(\R\).

Comme \(\fami{B}\) est une base de \(E\) sur \(\C\), on a : \[\quantifs{\forall x\in E;\exists z_1,\dots,z_n\in\C}x=z_1e_1+\dots+z_ne_n.\]

Donc \(\quantifs{\forall x\in E;\exists a_1,b_1,\dots,a_n,b_n\in\R}x=a_1e_1+\i b_1e_1+\dots+a_ne_n+\i b_ne_n\).

Donc \(\quantifs{\forall x\in E}x\in\Vect[\R]{\fami{B}\prim}\).

Donc \(\fami{B}\prim\) est génératrice de \(E\) sur \(\R\).

Montrons que \(\fami{B}\prim\) est libre sur \(\R\).

Soient \(a_1,b_1,\dots,a_n,b_n\in\R\) tels que \(a_1e_1+\i b_1e_1+\dots+a_ne_n+\i b_ne_n=0\).

On a \(\paren{a_1+\i b_1}e_1+\dots+\paren{a_n+\i b_n}e_n=0\).

Or \(\fami{B}\) est libre sur \(\C\).

Donc \(\quantifs{\forall k\in\interventierii{1}{n}}a_k+\i b_k=0\).

Donc \(\quantifs{\forall k\in\interventierii{1}{n}}a_k=b_k=0\).

Donc \(\fami{B}\prim\) est libre sur \(\R\).

Donc \(\fami{B}\prim\) est une base de \(E\) sur \(\R\) possédant \(2n\) vecteurs.

Donc \(\dim_\R E=2n\).
\end{dem}

\section{Familles de vecteurs en dimension finie}

\begin{theo}\thlabel{theo:nombreDeVecteursDansLesFamillesLibresEtGénératricesD'UnEVDeDimensionFinie}
Soit \(E\) un \(\K\)-espace vectoriel de dimension finie.

On pose \(n=\dim E\).

Alors :

\begin{enumerate}
\item Toute famille libre de \(E\) possède au plus \(n\) vecteurs. \\

\item Toute famille génératrice de \(E\) possède au moins \(n\) vecteurs.
\end{enumerate}
\end{theo}

\begin{dem}
Soit \(\fami{B}\) une base de \(E\) (qui possède donc \(n\) vecteurs).

\begin{enumerate}
\item Comme \(\fami{B}\) est une famille génératrice de \(E\) possédant \(n\) vecteurs, toute famille libre de \(E\) possède au plus \(n\) vecteurs. \\

\item Comme \(\fami{B}\) est une famille libre de \(E\) possédant \(n\) vecteurs, toute famille génératrice de \(E\) possède au moins \(n\) vecteurs.
\end{enumerate}
\end{dem}

\begin{prop}
Soient \(E\) et \(F\) deux espaces vectoriels.

\begin{enumerate}
\item S'il existe une application linéaire injective de \(E\) vers \(F\) alors \[\dim E\leq\dim F.\]

\item S'il existe une application linéaire surjective de \(E\) vers \(F\) alors \[\dim E\geq\dim F.\]

\item Si \(E\) et \(F\) sont isomorphes alors \[\dim E=\dim F.\]
\end{enumerate}
\end{prop}

\begin{dem}[1]
Soit \(u\in\L{E}{F}\) injective.

Si \(\dim E<\pinf\) :

Soit \(\fami{B}=\paren{e_1,\dots,e_n}\) une base de \(E\).

Comme \(u\) est injective, \(\paren{u\paren{e_1},\dots,u\paren{e_n}}\) est une famille libre de \(F\) selon le \thref{rappel:imageD'UneBaseParUneApplicationLinéaire}.

Donc \(n\leq\dim F\) car \(F\) possède une famille libre de \(n\) vecteurs.

Si \(\dim E=\pinf\) alors il existe une famille libre infinie \(\paren{x_i}_{i\in I}\in E^I\) de vecteurs de \(E\) et \(\paren{u\paren{x_i}}_{i\in I}\) est une famille libre infinie de vecteurs de \(F\) (selon le \thref{rappel:imageD'UneBaseParUneApplicationLinéaire}).

Donc \(\dim E\leq\dim F\).
\end{dem}

\begin{dem}[2]
Soit \(u\in\L{E}{F}\) surjective.

Si \(\dim E=\pinf\) alors \(\dim E\geq\dim F\).

Si \(\dim E<\pinf\) :

Soit \(\paren{e_1,\dots,e_n}\) une base de \(E\).

Selon le \thref{rappel:imageD'UneBaseParUneApplicationLinéaire}, \(\paren{u\paren{e_1},\dots,u\paren{e_n}}\) est une famille génératrice de \(F\).

Donc \(\dim F\leq n\).
\end{dem}

\begin{dem}[3]
Découle de (1) et (2).
\end{dem}

\begin{theo}\thlabel{theo:conditionsPourqQu'UneFamilleD'UnEVDeDimensionFinieSoitUneBase}
Soient \(E\) un \(\K\)-espace vectoriel de dimension finie \(n\) et \(\paren{e_1,\dots,e_p}\) une famille d'éléments de \(E\).

Alors \(\paren{e_1,\dots,e_p}\) est une base de \(E\) si elle vérifie deux des trois propositions suivantes :

\begin{enumerate}
\item \(p=n\) \\

\item \(\paren{e_1,\dots,e_p}\) est une famille libre \\

\item \(\paren{e_1,\dots,e_p}\) est une famille génératrice de \(E\).
\end{enumerate}
\end{theo}

\begin{dem}
Si on a (2) et (3) alors \(\paren{e_1,\dots,e_p}\) est une base de \(E\).

Si on a (1) et (2) alors on peut compléter \(\paren{e_1,\dots,e_p}\) en une base de \(E\) selon le théorème de la base incomplète. Donc \(\paren{e_1,\dots,e_p}\) est une base de \(E\).

Si on a (3) alors on peut extraire de \(\paren{e_1,\dots,e_p}\) une base de \(E\) selon le théorème de la base extraite. Si, de plus, \(n=p\) alors \(\paren{e_1,\dots,e_p}\) est une base de \(E\).
\end{dem}

\begin{exo}
Montrer que la famille \(\fami{B}=\paren{\tcoords{1}{2}{0},\tcoords{0}{1}{2},\tcoords{2}{1}{0}}\) est une base de \(\R^3\).
\end{exo}

\begin{corr}
Montrons que \(\fami{B}\) est libre.

Soient \(a,b,c\in\R\) tels que \(a\tcoords{1}{2}{0}+b\tcoords{0}{1}{2}+c\tcoords{2}{1}{0}=0\).

Alors \(\begin{dcases}
a+2c=0 \\
2a+b+c=0 \\
2b=0
\end{dcases}\) donc \(\begin{dcases}
-3c=0 &L_1\gets L_2-2L_1 \\
-3a=0 &L_2\gets L_1-2L_2 \\
b=0
\end{dcases}\)

Donc \(\fami{B}\) est libre.

De plus, elle possède \(3\) éléments et \(\dim\R^3=3\) donc c'est une base de \(\R^3\).
\end{corr}

\section{Sous-espaces vectoriels en dimension finie}

\subsection{Dimension d'un sous-espace vectoriel}

\begin{theo}\thlabel{theo:dimensionD'UnSousEVEstPlusPetite}
Soient \(E\) un \(\K\)-espace vectoriel de dimension finie et \(F\) un sous-espace vectoriel de \(E\).

Alors : \[\dim F\leq\dim E,\] avec égalité si, et seulement si, \(E=F\).
\end{theo}

\begin{dem}
On pose \(n=\dim E\).

Toute famille libre de \(F\) est une famille libre de \(E\) et possède donc au plus \(n\) éléments.

Donc \(F\) est de dimension finie.

Soit \(\paren{e_1,\dots,e_p}\) une base de \(F\) où \(p\in\N\).

Comme \(\paren{e_1,\dots,e_p}\) est une famille libre de vecteurs de \(E\), on a \(p\leq n\).

Donc \[\dim F=p\leq n=\dim E.\]

Montrons que \(\dim E=\dim F\ssi E=F\).

\imprec Claire.

\impdir

Si \(p=n\) alors \(\paren{e_1,\dots,e_p}\) est une famille libre de vecteurs de \(E\) possédant \(p=\dim E\) vecteurs de \(E\).

C'est donc une base de \(E\).

Donc \(E=\Vect{e_1,\dots,e_p}=F\).
\end{dem}

\begin{defi}[Base adaptée à un sous-espace vectoriel]
Soient \(E\) un espace vectoriel de dimension finie et \(F\) un sous-espace vectoriel de \(E\).

On dit qu'une base \(\fami{B}\) de \(E\) est adaptée au sous-espace vectoriel \(F\) si ses premiers vecteurs forment une base de \(F\), \cad si elle est de la forme \[\fami{B}=\paren{e_1,\dots,e_p,\dots,e_n}\in E^n\] avec \(p=\dim F\), \(n=\dim E\) et \(\paren{e_1,\dots,e_p}\) est une base de \(F\).

Il est facile de voir si un vecteur de \(E\) appartient à \(F\) à partir de ses coordonnées dans \(\fami{B}\) : \[\quantifs{\forall x_1,\dots,x_n\in\K}x_1e_1+\dots+x_ne_n\in F\ssi x_{p+1}=\dots=x_n=0.\]
\end{defi}

\begin{exoex}
Donner une base de \(E\) adaptée au sous-espace vectoriel \(F\) dans les situations suivantes :

\begin{enumerate}
\item \(E=\polydeg{N}\) et \(F=\polydeg{n}\) où les entiers \(n,N\in\N\) vérifient \(n\leq N\). \\

\item \(E=\polydeg{4}\) et \(F=\accol{P\in\polydeg{4}\tq3\text{ est racine multiple de }P}\).
\end{enumerate}
\end{exoex}

\begin{corr}[1]
La base canonique de \(\polydeg{N}\) \(\paren{\underbrace{1,\dots,X^n}_{\text{base de }\polydeg{n}},\dots,X^N}\) est adaptée à \(\polydeg{n}\).
\end{corr}

\begin{corr}[2]
On a \(\dim E=5\).

On a \[\begin{aligned}
\quantifs{\forall P\in E}P\in F&\ssi\paren{X-3}^2\divise P \\
&\ssi\quantifs{\exists a,b,c\in\K}P=\paren{X-3}^2\paren{aX^2+bX+c} \\
&\ssi\quantifs{\exists a,b,c\in\K}P=aX^2\paren{X-3}^2+bX\paren{X-3}^2+c\paren{X-3}^2 \\
&\ssi P\in\Vect{\paren{X-3}^2,X\paren{X-3}^2,X^2\paren{X-3}^2}
\end{aligned}\]

D'où une base de \(E\) adaptée à \(F\) : \(\paren{\paren{X-3}^2,X\paren{X-3}^2,X^2\paren{X-3}^2,1,X}\).

C'est bien une base de \(E\) car c'est une famille de polynômes à degrés échelonnés.
\end{corr}

\begin{prop}
Soient \(E\) un espace vectoriel de dimension finie et \(F\) un sous-espace vectoriel de \(E\).

Il existe une base de \(E\) adaptée à \(F\).
\end{prop}

\begin{dem}
Comme \(F\) est un sous-espace vectoriel de \(E\), il est de dimension finie.

Soit \(\paren{e_1,\dots,e_p}\) une base de \(F\) (avec \(p\in\N\)).

Comme \(\paren{e_1,\dots,e_p}\) est une famille libre de vecteurs de \(F\), c'est une famille libre de vecteurs de \(E\).

Selon le théorème de la base incomplète, on peut la compléter en une base de \(E\) : \[\paren{\underbrace{e_1,\dots,e_p}_{\text{base de }F},\dots,e_n}.\]
\end{dem}

\subsection{Rang d'une famille de vecteurs}

\begin{defi}
Soient \(E\) un \(\K\)-espace vectoriel et \(\fami{F}=\paren{x_1,\dots,x_p}\in E^p\) une famille de vecteurs de \(E\) (où \(p\in\N\)).

Le rang de la famille \(\fami{F}\) est la dimension du sous-espace vectoriel qu'elle engendre : \[\rg\fami{F}=\dim\Vect{\fami{F}}.\]
\end{defi}

\begin{rem}
On ne modifie pas le rang d'une famille de vecteurs :

\begin{itemize}
\item en permutant ses vecteurs ; \\

\item en multipliant l'un de ses vecteurs par un scalaire non-nul ; \\

\item en ajoutant à l'un de ses vecteurs une combinaison linéaire de ses autres vecteurs ; \\

\item en lui ôtant le vecteur nul.
\end{itemize}
\end{rem}

\begin{dem}
C'est clair puisque ces transformations ne modifient pas l'espace vectoriel engendré par la famille.
\end{dem}

\begin{exoex}
Donner le rang des familles de vecteurs suivantes :

\begin{enumerate}
\item La famille \(\paren{\tcoords{2}{4}{0},\tcoords{1}{2}{0},\tcoords{0}{1}{1},\tcoords{1}{0}{-2}}\) (famille de vecteurs de \(\R^3\)) ; \\

\item La famille \(\paren{X-1,X^3-X^2,X^3-3X^2+2,X^2-1}\) (famille de vecteurs de \(\poly[\R]\)).
\end{enumerate}
\end{exoex}

\begin{corr}[1]~\\
On a, comme \(\tcoords{2}{4}{0}=2\tcoords{1}{2}{0}\) : \[\begin{WithArrows}
\rg\paren{\tcoords{2}{4}{0},\tcoords{1}{2}{0},\tcoords{0}{1}{1},\tcoords{1}{0}{-2}}&=\rg\paren{\tcoords{1}{2}{0},\tcoords{0}{1}{1},\tcoords{1}{0}{-2}} \Arrow{car \(\tcoords{1}{0}{-2}=\tcoords{1}{2}{0}-2\tcoords{0}{1}{1}\)} \\
&=\rg\paren{\tcoords{1}{2}{0},\tcoords{0}{1}{1}} \Arrow[tikz={text width=5cm}]{car \(\tcoords{1}{2}{0}\) et \(\tcoords{0}{1}{1}\) ne sont pas colinéaires} \\
&=2
\end{WithArrows}\]
\end{corr}

\begin{corr}[2]
On remarque \(X^3-3X^2+2=\paren{X^3-X^2}-2\paren{X^2-1}\).

Donc \(\rg\paren{X-1,X^3-X^2,X^3-3X^2+2,X^2-1}=\rg\paren{X-1,X^3-X^2,X^2-1}=3\).

En effet, la famille \(\paren{X-1,X^3-X^2,X^2-1}\) est libre car c'est une famille de polynômes non-nuls de degrés deux à deux distincts.

Donc c'est une base de \(\Vect{X-1,X^3-X^2,X^2-1}\).

Donc on a \(\dim\Vect{X-1,X^3-X^2,X^2-1}=3\).
\end{corr}

\begin{rem}\thlabel{rem:rangD'UneFamilleLibreInférieurÀSonCardEtRangD'UneFamilleGénératriceInférieurÀLaDimension}
Soient \(E\) un \(\K\)-espace vectoriel et \(\fami{F}=\paren{x_1,\dots,x_p}\in E^p\) une famille de vecteurs de \(E\) (où \(p\in\N\)).

On a d'une part (selon le \thref{theo:nombreDeVecteursDansLesFamillesLibresEtGénératricesD'UnEVDeDimensionFinie}) : \[\rg\fami{F}\leq p,\] avec égalité si, et seulement si, la famille \(\fami{F}\) est libre (selon le \thref{theo:conditionsPourqQu'UneFamilleD'UnEVDeDimensionFinieSoitUneBase}).

On a d'autre part (selon le \thref{theo:dimensionD'UnSousEVEstPlusPetite}) : \[\rg\fami{F}\leq\dim E,\] avec égalité si, et seulement si, la famille \(\fami{F}\) est une famille génératrice de \(E\).
\end{rem}

\subsection{Sommes directes en dimension finie}

\begin{defi}
Soit \(E\) un \(\K\)-espace vectoriel.

Si \(\fami{F}_1=\paren{v_1\deriv{1},\dots,v_{d_1}\deriv{1}},\dots,\fami{F}_m=\paren{v_1\deriv{m},\dots,v_{d_m}\deriv{m}}\) sont des familles de vecteurs de \(E\), on appelle famille obtenue en juxtaposant \(\fami{F}_1,\dots,\fami{F}_m\) la famille : \[\paren{v_1\deriv{1},\dots,v_{d_1}\deriv{1},\dots,v_1\deriv{m},\dots,v_{d_m}\deriv{m}}.\]
\end{defi}

\begin{prop}\thlabel{prop:familleObtenueEnJuxtaposantLesBasesDeSousEVSupplémentairesEstUneBase}
Soient \(E\) un espace vectoriel de dimension finie, \(F\) et \(G\) deux sous-espaces vectoriels supplémentaires dans \(E\), \(\fami{B}_F=\paren{v_1,\dots,v_n}\) une base de \(F\) et \(\fami{B}_G=\paren{w_1,\dots,w_m}\) une base de \(G\).

Alors la famille \[\fami{B}=\paren{v_1,\dots,v_n,w_1,\dots,w_m}\] obtenue en juxtaposant \(\fami{B}_F\) et \(\fami{B}_G\) est une base de \(E\).
\end{prop}

\begin{dem}
Montrons que \(\fami{B}\) est libre.

Soient \(\lambda_1,\dots,\lambda_n,\mu_1,\dots,\mu_m\in\K\) tels que \[\underbrace{\lambda_1v_1+\dots+\lambda_nv_n}_{\in F}+\underbrace{\mu_1w_1+\dots+\mu_mw_m}_{\in G}=0_E.\]

Donc comme \(F\) et \(G\) sont en somme directe : \(\begin{dcases}
\lambda_1v_1+\dots+\lambda_nv_n=0_E \\
\mu_1w_1+\dots+\mu_mw_m=0_E
\end{dcases}\)

Donc comme \(\fami{B}_F\) et \(\fami{B}_G\) sont libres : \(\begin{dcases}
\lambda_1=\dots=\lambda_n=0 \\
\mu_1=\dots=\mu_m=0
\end{dcases}\)

Donc \(\fami{B}\) est libre.

Montrons que \(\fami{B}\) est génératrice de \(E\).

Soit \(x\in E\).

Comme \(E=F+G\), il existe \(x_F\in F\) et \(x_G\in G\) tels que \(x=x_F+x_G\).

Comme \(\fami{B}_F\) est génératrice de \(F\), il existe \(\lambda_1,\dots,\lambda_n\in\K\) tels que \[x_F=\lambda_1v_1+\dots+\lambda_nv_n.\]

Comme \(\fami{B}_G\) est génératrice de \(G\), il existe \(\mu_1,\dots,\mu_m\in\K\) tels que \[x_G=\mu_1w_1+\dots+\mu_mw_m.\]

Finalement, on a \(x=\lambda_1v_1+\dots+\lambda_nv_n+\mu_1w_1+\dots+\mu_mw_m\).

Donc \(x\in\Vect{\fami{B}}\).

Donc \(\fami{B}\) est génératrice de \(E\).

Finalement, \(\fami{B}\) est une base de \(E\).
\end{dem}

\begin{cor}
Soient \(E\) un espace vectoriel et \(F\) et \(G\) deux sous-espaces vectoriels de \(E\) en somme directe : \[F\inter G=\accol{0_E}.\]

Alors on a : \[\dim F\oplus G=\dim F+\dim G.\]
\end{cor}

\begin{dem}
Posons \(E\prim=F\oplus G\).

Si \(F\) ou \(G\) est de dimension infinie, alors \(E\prim\) aussi donc on a bien \(\dim E\prim=\dim F+\dim G\).

Supposons \(F\) et \(G\) de dimension finie.

Soient \(\fami{B}_F\) une base de \(F\) et \(\fami{B}_G\) une base de \(G\).

On note \(\fami{B}\) la famille obtenue en juxtaposant \(\fami{B}_F\) et \(\fami{B}_G\).

Selon la \thref{prop:familleObtenueEnJuxtaposantLesBasesDeSousEVSupplémentairesEstUneBase}, \(\fami{B}\) est une base de \(E\prim\).

Donc \[\dim E\prim=\Card\fami{B}=\Card\fami{B}_F+\Card\fami{B}_G=\dim F+\dim G.\]
\end{dem}

\begin{defi}[Base adaptée à une somme directe]
Soient \(E\) un espace vectoriel de dimension finie et \(F\) et \(G\) deux sous-espaces vectoriels supplémentaires dans \(E\) : \[E=F\oplus G.\]

Une base de \(E\) obtenue en juxtaposant une base de \(F\) et une base de \(G\) est dite adaptée à la décomposition de \(E\) en somme directe.

On a vu à la \thref{prop:familleObtenueEnJuxtaposantLesBasesDeSousEVSupplémentairesEstUneBase} qu'une telle base existe.
\end{defi}

\begin{exo}
Soient \(F\) et \(G\) deux plans vectoriels de \(\K^3\) tels que \(F\not=G\).

Quelle est la dimension de leur intersection ?
\end{exo}

\begin{corr}~\\
On a \(\begin{dcases}
F\inter G\text{ est un sous-espace vectoriel de }F \\
\dim F=2
\end{dcases}\) donc \(\dim F\inter G\leq2\).

Si \(\dim F\inter G=2\) :

On a \(\begin{dcases}
F\inter G\subset F \\
\dim F\inter G=\dim F<\pinf
\end{dcases}\) donc \(F\inter G=F\) donc \(F\subset G\).

Ainsi \(\begin{dcases}
F\subset G \\
\dim F=\dim G<\pinf
\end{dcases}\) donc \(F=G\) : contradiction.

Si \(\dim F\inter G=0\) alors on a \(F\inter G=\accol{0}\) donc \(F\oplus G\) est un sous-espace vectoriel de \(\K^3\) tel que \(\dim F\oplus G=\dim F+\dim G=4\) : contradiction car \(\dim\K^3=3\).

Donc \(\dim F\inter G=1\).
\end{corr}

\subsection{Supplémentaire en dimension finie}

\begin{theo}[Existence d'un supplémentaire en dimension finie]
Soient \(E\) un \(\K\)-espace vectoriel de dimension finie et \(F\) un sous-espace vectoriel de \(E\).

Alors \(F\) admet un supplémentaire dans \(E\).
\end{theo}

\begin{dem}
Soit \(\paren{e_1,\dots,e_p}\) une base de \(F\) (avec \(p\in\N\)).

La famille \(\paren{e_1,\dots,e_p}\) est une famille libre de vecteurs de \(E\).

Donc selon le théorème de la base incomplète, on peut la compléter en une base \(\paren{e_1,\dots,e_n}\) de \(E\) (avec \(n\in\N\) tel que \(n\geq p\)).

Posons \(S=\Vect{e_{p+1},\dots,e_n}\).

Montrons que \(S\) est un supplémentaire de \(F\) dans \(E\), \cad \[F\inter S=\accol{0_E}.\]

Soit \(x\in F\inter S\).

On a \(x\in F=\Vect{e_1,\dots,e_p}\) donc il existe \(x_1,\dots,x_p\in\K\) tels que \[x=x_1e_1+\dots+x_pe_p\] et \(x\in S=\Vect{e_{p+1},\dots,e_n}\) donc il existe \(x_{p+1},\dots,x_n\in\K\) tels que \[x=x_{p+1}e_{p+1}+\dots+x_ne_n.\]

On a \(x_1e_1+\dots+x_pe_p-x_{p+1}e_{p+1}-\dots-x_ne_n=x-x=0_E\).

Or \(\paren{e_1,\dots,e_n}\) est libre.

Donc \(\quantifs{\forall k\in\interventierii{1}{n}}x_k=0\).

Donc \(x=0_E\) donc \(F\inter S=\accol{0_E}\).

Montrons que \(F+S=E\).

Soit \(x\in E\).

Comme \(\paren{e_1,\dots,e_n}\) est une base de \(E\), il existe \(x_1,\dots,x_n\in\K\) tels que \[x=\underbrace{x_1e_1+\dots+x_pe_p}_{\in F}\underbrace{+x_{p+1}e_{p+1}+\dots+x_ne_n}_{\in S}.\]

Donc \(x\in F+S\).

Finalement, \(E=F\oplus S\).
\end{dem}

\section{Autres exemples de dimensions}

\subsection{Dimension d'un produit d'espaces vectoriels}

\begin{prop}
Soient \(F_1,\dots,F_m\) des espaces vectoriels de dimension finie.

On pose : \[\begin{dcases}
F_1\prim=F_1\times\accol{0_{F_2}}\times\dots\times\accol{0_{F_m}} \\
\vdots \\
F_m\prim=\accol{0_{F_1}}\times\dots\accol{0_{F_{m-1}}}\times F_m
\end{dcases}\]

On a \(\quantifs{\forall k\in\interventierii{1}{m}}F_k\text{ est isomorphe à }F_k\prim\) donc \(\dim F_k=\dim F_k\prim\).

Soient \(\fami{B}_1\) une base de \(F_1\prim\), ..., \(\fami{B}_m\) une base de \(F_m\prim\).

On note \(\fami{B}\) la famille obtenue en juxtaposant \(\fami{B}_1,\dots,\fami{B}_m\).

Alors \(\fami{B}\) est une base de l'espace vectoriel produit \(F_1\times\dots\times F_m\).
\end{prop}

\begin{dem}
\note{Exercice}
\end{dem}

\begin{cor}\thlabel{cor:dimensionDuProduitD'EVsEstLaSommeDesDimensionsDesEVs}
Soient \(F_1,\dots,F_m\) des espaces vectoriels.

Alors : \[\dim\paren{F_1\times\dots\times F_m}=\dim F_1+\dots+\dim F_m.\]
\end{cor}

\begin{dem}
On distingue deux cas :

\begin{itemize}
\item Si l'un des espaces vectoriels \(F_1,\dots,F_m\) est de dimension infinie, alors \(F_1\times\dots\times F_m\) est aussi de dimension infinie. \\

\item Si les espaces vectoriels \(F_1,\dots,F_m\) sont de dimension finie alors la formule découle de la proposition précédente.
\end{itemize}
\end{dem}

\subsection{Dimension de \(\L{E}{F}\)}

\begin{theo}
Soient \(E\) et \(F\) deux espaces vectoriels de dimension finie.

Alors on a : \[\dim\L{E}{F}=\paren{\dim E}\paren{\dim F}.\]

En particulier, l'espace vectoriel \(\L{E}{F}\) est de dimension finie.
\end{theo}

\begin{dem}
Soit \(\paren{e_1,\dots,e_n}\) une base de \(E\).

On sait que \[\fonctionlambda{\L{E}{F}}{F^n}{u}{\paren{u\paren{e_1},\dots,u\paren{e_n}}}\] est un isomorphisme d'espaces vectoriels.

Donc \[\begin{WithArrows}
\dim\L{E}{F}&=\dim F^n \Arrow{selon le \thref{cor:dimensionDuProduitD'EVsEstLaSommeDesDimensionsDesEVs}} \\
&=n\dim F \\
&=\paren{\dim E}\paren{\dim F}.
\end{WithArrows}\]
\end{dem}

\begin{cor}
Soit \(E\) un espace vectoriel de dimension finie.

Alors on a : \[\dim\Lendo{E}=\paren{\dim E}^2\] et : \[\dim E\etoile=\dim E.\]

En particulier, les espaces vectoriels \(\Lendo{E}\) et \(E\etoile\) sont de dimension finie.
\end{cor}

\begin{dem}
Posons \(n=\dim E\).

On a \(\dim\Lendo{E}=\dim\L{E}{E}=n^2\).

On a \(\dim E\etoile=\dim\L{E}{\K}=n\times1=n\).
\end{dem}

\begin{rem}
Soit \(E\) un espace vectoriel de dimension finie.

On a vu en TD (\cf \thref{exo:baseDuale}) que si \(\fami{B}\) est une base de \(E\) alors la base duale \(\fami{B}\prim\) possède le même nombre d'éléments que \(\fami{B}\), ce qui redémontre que \(\dim E\etoile=\dim E\).
\end{rem}

\section{Rang d'une application linéaire}

\subsection{Définition}

\begin{defi}
Soient \(E\) et \(F\) deux \(\K\)-espaces vectoriels et \(u\in\L{E}{F}\).

On appelle rang de l'application linéaire \(u\) la dimension de son image : \[\rg u=\dim\Im u.\]
\end{defi}

\begin{rem}\thlabel{rem:rangD'UneApplicationLinéaireÉgalAuRangD'UneFamilleGénératriceAppliquéeÀL'ApplicationLinéaire}
Soient \(E\) et \(F\) deux \(\K\)-espaces vectoriels et \(u\in\L{E}{F}\).

On suppose que \(\paren{e_1,\dots,e_p}\in E^p\) est une famille génératrice de \(E\).

Alors \[\rg u=\rg\paren{u\paren{e_1},\dots,u\paren{e_p}}.\]
\end{rem}

\begin{dem}
On a : \[\begin{aligned}
\rg u&=\dim\Im u \\
&=\dim u\paren{E} \\
&=\dim u\paren{\Vect{e_1,\dots,e_p}} \\
&=\dim\Vect{u\paren{e_1},\dots,u\paren{e_p}} \\
&=\rg\paren{u\paren{e_1},\dots,u\paren{e_p}}.
\end{aligned}\]
\end{dem}

\begin{prop}
Soient \(E\) et \(F\) deux \(\K\)-espaces vectoriels et \(u\in\L{E}{F}\).

On a : \[\rg u\leq\min\accol{\dim E;\dim F}.\]
\end{prop}

\begin{dem}
Montrons que \(\rg u\leq\dim E\).

C'est vrai si \(\dim E<\pinf\).

Sinon, on considère une base \(\paren{e_1,\dots,e_n}\) de \(E\) (où \(n\in\N\)) et on a : \[\begin{WithArrows}
\rg u&=\rg\paren{u\paren{e_1},\dots,u\paren{e_n}} \Arrow{selon la \thref{rem:rangD'UneFamilleLibreInférieurÀSonCardEtRangD'UneFamilleGénératriceInférieurÀLaDimension}} \\
&\leq n.
\end{WithArrows}\]

Montrons que \(\rg u\leq\dim F\).

On a \(\Im u\subset F\) donc \(\rg u=\dim\Im u\leq\dim F\).
\end{dem}

\begin{exoex}
Soit \(n\in\N\).

Quel est le rang de l'endomorphisme \[\fonction{D}{\polydeg[\R]{n}}{\polydeg[\R]{n}}{P}{P\prim}\text{ ?}\]
\end{exoex}

\begin{corr}
On a \(\polydeg[\R]{n}=\Vect{1,X,\dots,X^n}\).

Donc selon la \thref{rem:rangD'UneApplicationLinéaireÉgalAuRangD'UneFamilleGénératriceAppliquéeÀL'ApplicationLinéaire} : \[\begin{WithArrows}
\rg D&=\rg\paren{D\paren{1},\dots,D\paren{X^n}} \\
&=\rg\paren{0,1,2X,3X^2,\dots,nX^{n-1}} \\
&=\rg\paren{1,2X,3X^2,\dots,nX^{n-1}} \Arrow[tikz={text width=5cm}]{car \(\Vect{1,2X,\dots,nX^{n-1}}=\polydeg[\R]{n-1}\)} \\
&=n.
\end{WithArrows}\]
\end{corr}

\begin{prop}[Invariance du rang par composition par un isomorphisme]
Soient \(E\), \(F\) et \(G\) des \(\K\)-espaces vectoriels et \(u\in\L{E}{F}\) et \(v\in\L{F}{G}\).

\begin{enumerate}
\item Si \(u\) est un isomorphisme, alors : \(\rg vu=\rg v\). \\

\item Si \(v\) est un isomorphisme, alors : \(\rg vu=\rg u\).
\end{enumerate}
\end{prop}

\begin{dem}[1]
On a : \[\begin{WithArrows}
\rg vu&=\dim vu\paren{E} \\
&=\dim v\paren{u\paren{E}} \Arrow{car \(u\paren{E}=F\) car \(u\) est surjectif} \\
&=\dim v\paren{F} \\
&=\rg v.
\end{WithArrows}\]
\end{dem}

\begin{dem}[2]
On a \[\Im vu=vu\paren{E}=v\paren{\Im u}.\]

Comme \(v\) est un isomorphisme, les espaces vectoriels \(\Im vu\) et \(\Im u\) sont isomorphes.

Donc \(\dim\Im vu=\dim\Im u\).

Donc \(\rg vu=\rg u\).
\end{dem}

\subsection{Théorème du rang}

\begin{theo}[Forme géométrique du théorème du rang]
Soient \(E\) et \(F\) deux \(\K\)-espace vectoriel, \(u\in\L{E}{F}\) et \(S\) un supplémentaire de \(\ker u\) dans \(E\) : \[E=\ker u\oplus S.\]

Alors \(u\) induit un isomorphisme de \(S\) vers \(\Im u\).

Cela signifie que l'application \[\fonction{v}{S}{\Im u}{x}{u\paren{x}}\] est un isomorphisme.
\end{theo}

\begin{dem}
On a \(u\in\L{E}{F}\) et \(S\) est un sous-espace vectoriel de \(E\) donc \(\restr{u}{S}\in\L{S}{F}\) et \(\Im\restr{u}{S}\subset\Im u\) donc \(v\in\L{S}{\Im u}\).

Montrons que \(\ker v=\accol{0_E}\).

Soit \(x\in S\) tel que \(v\paren{x}=0_F\).

On a \(u\paren{x}=0_F\) car \(v\paren{x}=u\paren{x}\).

Donc \(x\in\ker v\inter S=\accol{0_E}\).

Donc \(x=0_E\).

Donc \(v\) est injectif.

Montrons que \(v\) est surjectif de \(S\) vers \(\Im u\).

Soient \(y\in\Im u\) et \(x\in E\) tel que \(u\paren{x}=y\).

Comme \(\ker u+S=E\), il existe \(x_0\in\ker u\) et \(x_1\in S\) tels que \(x=x_0+x_1\).

On a \[y=u\paren{x}=u\paren{x_0}+u\paren{x_1}=0_E+v\paren{x_1}.\]

Donc \(y\in\Im v\).

Donc \(v\) est surjectif.

Donc \(v\) est un isomorphisme de \(S\) vers \(\Im u\).
\end{dem}

\begin{theo}[Théorème du rang]
Soient \(E\) et \(F\) deux \(\K\)-espaces vectoriels et \(u\in\L{E}{F}\).

On suppose que \(E\) est de dimension finie.

On a : \[\dim E=\rg u+\dim\ker u.\]
\end{theo}

\begin{dem}
Soit \(S\) un supplémentaire de \(\ker u\) dans \(E\).

Selon le théorème précédent, \(S\) et \(\Im u\) sont isomorphes donc \(\dim S=\dim\Im u\), \cad \(\dim E-\dim\ker u=\rg u\).

Donc \[\dim E=\rg u+\dim\ker u.\]
\end{dem}

\begin{rem}
Le théorème est également vrai en dimension infinie.
\end{rem}

\subsection{Applications}

\subsubsection{Isomorphismes en dimension finie}

\begin{theo}
Soient \(E\) et \(F\) deux \(\K\)-espaces vectoriels de même dimension finie et \(u\in\L{E}{F}\).

Les propositions suivantes sont équivalentes :

\begin{enumerate}
\item \(u\) est injective (\cad \(\ker u=\accol{0_E}\)) \\

\item \(u\) est surjective (\cad \(\rg u=\dim F\)) \\

\item \(u\) est bijective.
\end{enumerate}
\end{theo}

\begin{dem}
On a : \[\begin{WithArrows}
u\text{ est surjectif}&\ssi\Im u=F \Arrow[tikz={text width=5cm}]{car \(\Im u\) est un sous-espace vectoriel de \(F\) et \(\dim F<\pinf\)} \\
&\ssi\dim\Im u=\dim F \\
&\ssi\rg u=\dim F.
\end{WithArrows}\]

Les implications (3) \(\imp\) (1) et (3) \(\imp\) (2) sont claires.

Selon le théorème du rang appliqué à \(u\), on a \[\dim E=\rg u+\dim\ker u.\]

D'où \[\begin{aligned}
u\text{ est injective}&\ssi\ker u=\accol{0_E} \\
&\ssi\dim\ker u=0 \\
&\ssi\rg u=\dim E \\
&\ssi\rg u=\dim F \\
&\ssi u\text{ est surjective}.
\end{aligned}\]

D'où les équivalences.
\end{dem}

\begin{cor}
Soient \(E\) un \(\K\)-espace vectoriel de dimension finie et \(u\in\Lendo{E}\).

On rappelle que l'anneau \(\anneau{\Lendo{E}}[+][\rond]\) n'est pas commutatif (sauf si \(\dim E\leq1\)).

Les propositions suivantes sont équivalentes :

\begin{enumerate}
\item \(u\) est inversible : \(u\in\GL{}[E]\). \\

\item \(u\) est inversible à droite : \(\quantifs{\exists v\in\Lendo{E}}uv=\id{E}\). \\

\item \(u\) est inversible à gauche : \(\quantifs{\exists v\in\Lendo{E}}vu=\id{E}\).
\end{enumerate}
\end{cor}

\begin{dem}
Supposons \(u\) inversible à droite.

Alors \(u\) est une surjection.

Donc \(u\) est un isomorphisme car \(\dim E<\pinf\).

D'où (1).

Supposons \(u\) inversible à gauche.

Alors \(u\) est une injection.

Donc \(u\) est un isomorphisme car \(\dim E<\pinf\).

D'où (1).

Finalement, les implications (1) \(\imp\) (2) et (1) \(\imp\) (3) sont claires.
\end{dem}

\subsubsection{Formule de Grassmann}

\begin{prop}[Formule de Grassmann]
Soient \(E\) un \(\K\)-espace vectoriel et \(F\) et \(G\) deux sous-espaces vectoriels de \(E\) de dimension finie.

On a : \[\dim\paren{F+G}=\dim F+\dim G-\dim F\inter G.\]
\end{prop}

\begin{dem}
On pose : \[\fonction{u}{F\times G}{E}{\paren{x,y}}{x+y}\]

On a \(\begin{dcases}
u\in\L{F\times G}{E} \\
\Im u=F+G
\end{dcases}\)

Selon le théorème du rang appliqué à \(u\), on a : \[\dim\paren{F\times G}=\dim\ker u+\rg u,\] avec \(\begin{dcases}
\dim\paren{F\times G}=\dim F+\dim G \\
\rg u=\dim\paren{F+G} \\
\ker u=\accol{\paren{x,y}\in F\times G\tq x+y=0_E}=\accol{\paren{z,-z}}_{z\in F\inter G}
\end{dcases}\)

Or \(\accol{\paren{z,-z}}_{z\in F\inter G}\) est isomorphe à \(F\inter G\) (en effet, \(\fonctionlambda{F\inter G}{\accol{\paren{z,-z}}_{z\in F\inter G}}{z}{\paren{z,-z}}\) est clairement linéaire et bijective donc c'est un isomorphisme).

Donc \(\dim\ker u=\dim F\inter G\).

Finalement, on a \(\dim F+\dim G=\dim F\inter G+\dim\paren{F+G}\), \cad : \[\dim\paren{F+G}=\dim F+\dim G-\dim F\inter G.\]
\end{dem}

\begin{cor}
Soient \(E\) un \(\K\)-espace vectoriel et \(F\) et \(G\) deux sous-espaces vectoriels de \(E\) de dimension finie.

On a : \[\dim\paren{F+G}\leq\dim F+\dim G\] avec égalité si, et seulement si, \(F\) et \(G\) sont en somme directe.
\end{cor}

\begin{dem}
Selon la formule de Grassmann, on a \[\begin{aligned}
\dim\paren{F+G}&=\dim F+\dim G-\dim F\inter G \\
&\leq\dim F+\dim G
\end{aligned}\] avec égalité ssi \(\dim F\inter G=0\) ssi \(F\inter G=\accol{0_E}\) ssi \(F\) et \(G\) sont en somme directe.
\end{dem}

\subsubsection{Supplémentaires}

\begin{prop}[Caractérisation des supplémentaires]
Soient \(E\) un \(\K\)-espace vectoriel de dimension finie et \(F\) et \(G\) deux sous-espaces vectoriels de \(E\).

On a \[E=F\oplus G\] si deux des trois conditions suivantes sont satisfaites :

\begin{enumerate}
\item Les sous-espaces vectoriels \(F\) et \(G\) sont en somme directe \\

\item \(E=F+G\) \\

\item \(\dim E=\dim F+\dim G\).
\end{enumerate}
\end{prop}

\begin{dem}[((1) et (3))\(\imp\)(2)]
Supposons (1) et (3).

Selon la formule de Grassmann et (1), on a : \[\begin{WithArrows}
\dim\paren{F+G}&=\dim F+\dim G-0 \Arrow{selon (3)} \\
&=\dim E
\end{WithArrows}\]

Ainsi, on a \(\begin{dcases}
F+G\subset E \\
\dim\paren{F+G}=\dim E<\pinf
\end{dcases}\)

Donc \(F+G=E\).
\end{dem}

\begin{dem}[((2) et (3))\(\imp\)(1)]
Supposons (2) et (3).

Selon la formule de Grassmann, on a : \[\begin{aligned}
\dim F\inter G&=\underbrace{\dim F+\dim G}_{=\dim E\text{ selon (3)}}-\underbrace{\dim\paren{F+G}}_{=\dim E\text{ selon (2)}} \\
&=0
\end{aligned}\]

Donc \(F\inter G=\accol{0_E}\).
\end{dem}

\section{Hyperplans}

\subsection{Hyperplans en dimension quelconque}

\begin{defi}
Soient \(E\) un espace vectoriel et \(H\) un sous-espace vectoriel de \(E\).

On dit que \(H\) est un hyperplan (de \(E\)) si \(H\) est le noyau d'une forme linéaire non-nulle : \[\quantifs{\exists l\in E\etoile\excluant\accol{0}}H=\ker l.\]
\end{defi}

\begin{prop}\thlabel{prop:droiteVectorielleNonIncluseDansUnHyperplanEstUnSupplémentaireDeCetHyperplan}
Soient \(E\) un espace vectoriel, \(D\) une droite vectorielle de \(E\) et \(H\) un hyperplan de \(E\) ne contenant pas \(D\) : \[D\not\subset H.\]

Alors \(H\) et \(D\) sont supplémentaires dans \(E\) : \[H\oplus D=E.\]
\end{prop}

\begin{dem}
Montrons que \(H\inter D=\accol{0_E}\).

On a \(H\inter D\subset D\) et \(\dim D=1\) donc \(\dim H\inter D=0\) ou \(\dim H\inter D=1\).

Si \(\dim H\inter D=1\) :

Alors \(\begin{dcases}
H\inter D\subset D \\
\dim H\inter D=\dim D<\pinf
\end{dcases}\)

Donc \(H\inter D=D\).

Donc \(D\subset H\) : contradiction.

Donc \(\dim H\inter D=0\) donc \(H\inter D=\accol{0_E}\).

Montrons que \(H+D=E\).

Comme \(H\) est un hyperplan de \(E\), il existe \(l\in E\etoile\excluant\accol{0}\) telle que \(H=\ker l\).

Soit \(x\in E\).

Montrons que \(x\in H+D\).

On a \(D\not\subset H\) donc \(D\not\subset\ker l\).

Donc \(\quantifs{\exists x\in D}l\paren{x}\not=0\).

Donc \(\restr{l}{D}\not=0\) donc \(\restr{l}{D}\) est surjective.

Donc \(\quantifs{\exists y\in D}l\paren{y}=l\paren{x}\).

Donc \[\begin{aligned}
\quantifs{\exists y\in D}&y-x\in\ker l \\
&\quantifs{\exists z\in\ker l}y-x=z \\
&\quantifs{\exists z\in H}y-z=x.
\end{aligned}\]

Donc \(H+D=E\).
\end{dem}

\begin{prop}\thlabel{prop:hyperplansD'UnEVSontLesSousEVsQuiAdmettentCommeSupplémentaireUneDroite}
Les hyperplans de \(E\) sont les sous-espaces vectoriels de \(E\) qui admettent comme supplémentaire une droite vectorielle.
\end{prop}

\begin{dem}
Soit \(H\subset E\).

Montrons que \[H\text{ est un hyperplan de }E\ssi H\text{ admet comme supplémentaire une droite}.\]

\impdir

Supposons que \(H\) est un hyperplan de \(E\).

Soit \(l\in E\etoile\excluant\accol{0}\) telle que \(H=\ker l\).

Comme \(l\not=0\), on a \(H\not=E\).

Soit \(x\in E\excluant H\).

Alors \(D=\Vect{x}\) est une droite (car \(x\not=0_E\) car \(x\not\in H\)) non-incluse dans \(H\) (car \(x\not\in H\)).

Donc \(D\) est une droite supplémentaire de \(H\) dans \(E\) selon la \thref{prop:droiteVectorielleNonIncluseDansUnHyperplanEstUnSupplémentaireDeCetHyperplan}.

\imprec

Supposons \(E=H\oplus D\) où \(D\) est une droite de \(E\).

On a \(\dim D=\dim\K=1\).

Donc il existe un isomorphisme \(l_1\in D\etoile\).

Notons \(l_0\in H\etoile\) la forme linéaire nulle sur \(H\).

Soit \(l\in E\etoile\) l'unique forme linéaire sur \(E\) telle que \[\restr{l}{H}=l_0\qquad\text{et}\qquad\restr{l}{D}=l_1.\]

On a : \[\begin{WithArrows}
\quantifs{\forall h\in H;\forall d\in D}h+d\in\ker l&\ssi l\paren{h+d}=0_\K \\
&\ssi l_0\paren{h}+l_1\paren{d}=0_\K \\
&\ssi l_1\paren{d}=0_\K \Arrow{car \(l_1\) est injective} \\
&\ssi d=0 \\
&\ssi h+d=h \\
&\ssi h+d\in H
\end{WithArrows}\]

Donc \(\ker l=H\) donc \(H\) est un hyperplan car \(l\in E\etoile\excluant\accol{0}\).
\end{dem}

\begin{prop}
Soient \(E\) un \(\K\)-espace vectoriel, \(H\) un hyperplan de \(E\) et \(l\in E\etoile\excluant\accol{0}\) une forme linéaire non-nulle de \(E\) telle que \(\ker l=H\).

Les formes linéaires nulles sur \(H\) sont celles qui sont colinéaires à \(l\) : \[\quantifs{\forall l\prim\in E\etoile}H\subset\ker l\prim\ssi\croch{\quantifs{\exists\lambda\in\K}l\prim=\lambda l}.\]

Les formes linéaires dont \(H\) est le noyau sont donc celles de la forme \(\lambda l\) où \(\lambda\in\K\excluant\accol{0}\).
\end{prop}

\begin{dem}
Soient \(D\) une droite telle que \(H\oplus D=E\) et \(x\in E\) tel que \(D=\Vect{x}\).

Soit \(l\prim\in E\etoile\).

Montrons que \[\quantifs{\forall l\prim\in E\etoile}H\subset\ker l\prim\ssi\croch{\quantifs{\exists\lambda\in\K}l\prim=\lambda l}.\]

\imprec S'il existe \(\lambda\in\K\) tel que \(l\prim=\lambda l\) alors \(\quantifs{\forall h\in H}l\prim\paren{h}=\lambda l\paren{h}=0\) donc \(H\subset\ker l\prim\).

\impdir

Supposons \(H\subset\ker l\prim\).

On a \(l\paren{x}\not=0\).

Posons \(\lambda=\dfrac{l\prim\paren{x}}{l\paren{x}}\).

On a \(l\prim\paren{x}=\lambda l\paren{x}\) donc comme \(l\prim\) est linéaire : \[\begin{aligned}
\quantifs{\forall h\in H;\forall\mu\in\K}l\prim\paren{h+\mu x}&=l\prim\paren{h}+\mu l\prim\paren{x} \\
&=0+\mu\lambda l\paren{x} \\
&=\lambda l\paren{h}+\mu\lambda l\paren{x} \\
&=\lambda l\paren{h+\mu x}.
\end{aligned}\]

Ainsi, \(\quantifs{\forall y\in H+\Vect{x}}l\prim\paren{y}=\lambda l\paren{y}\).

Donc \(l\prim=\lambda l\) car \(E=H+\Vect{x}\).

Déterminons les formes linéaires de noyau \(H\).

\analyse

Soit \(l\prim\in E\etoile\) telle que \(H=\ker l\prim\).

Alors \(H\subset\ker l\prim\).

Donc selon ce qui précède, il existe \(\lambda\in\K\) tel que \(l\prim=\lambda l\).

\synthese

Soit \(\lambda\in\K\).

Posons \(l\prim=\lambda l\).

Si \(\lambda=0\), on a \(\ker l\prim=E\not=H\).

Sinon, on a \(\ker l\prim=\ker\lambda l=\ker l=H\).

\conclusion Les formes linéaires de noyau \(H\) sont celles de la forme \(\lambda l\) où \(\lambda\in\K\excluant\accol{0}\).
\end{dem}

\subsection{Hyperplans en dimension finie}

\begin{prop}
Soit \(E\) un espace vectoriel de dimension finie \(n\in\Ns\).

Les hyperplans de \(E\) sont ses sous-espaces vectoriels de dimension \(n-1\).
\end{prop}

\begin{dem}
Rappel : si \(l\in E\etoile=\L{E}{\K}\) alors \(\rg l\leq\dim\K=1\) donc toute forme linéaire non-nulle est surjective.

\incdir

Soient \(H\) un hyperplan de \(E\) et \(l\in E\etoile\excluant\accol{0}\) telle que \(H=\ker l\).

D'après le théorème du rang appliqué à \(l\), on a : \[\dim E=\dim\ker l+\rg l.\]

Donc \(n=\dim H+1\) donc \(\dim H=n-1\).

\increc

Soient \(H\) un sous-espace vectoriel de \(E\) de dimension \(n-1\) et \(D\) un supplémentaire de \(H\) dans \(E\) : \[E=H\oplus D.\]

On a \(\dim E=\dim H+\dim D\) donc \(\dim D=1\).

Donc \(H\) est un hyperplan selon la \thref{prop:hyperplansD'UnEVSontLesSousEVsQuiAdmettentCommeSupplémentaireUneDroite}.
\end{dem}

\begin{dem}[Autre méthode]
Soient \(H\) un sous-espace vectoriel de \(E\) de dimension \(n-1\) et \(\paren{\underbrace{e_1,\dots,e_{n-1}}_{\text{base de }H},e_n}\) une base de \(E\) adaptée à \(H\).

On note \(\paren{e_1\etoile,\dots,e_n\etoile}\) la base duale de \(\paren{e_1,\dots,e_n}\).

On a \(H=\ker e_n\etoile\) avec \(e_n\etoile\) une forme linéaire non-nulle.

Donc \(H\) est un hyperplan de \(E\).
\end{dem}

\begin{prop}[Équation cartésienne d'un hyperplan dans une base]
Soient \(E\) un espace vectoriel de dimension finie \(n\in\Ns\), \(H\) un hyperplan de \(E\) et \(\fami{B}=\paren{e_1,\dots,e_n}\) une base de \(E\).

Alors \(H\) admet dans \(\fami{B}\) une équation cartésienne de la forme : \[a_1x_1+\dots+a_nx_n=0,\] où \(a_1,\dots,a_n\) sont des scalaires non-tous nuls.

En d'autres termes, on a : \[\quantifs{\exists\paren{a_1,\dots,a_n}\in\K^n\excluant\accol{0};\forall x_1,\dots,x_n\in\K}x_1e_1+\dots+x_ne_n\in H\ssi a_1x_1+\dots+a_nx_n=0.\]

De plus, l'équation cartésienne d'un hyperplan de \(E\) est unique à un facteur multiplicatif non-nul près.

Autrement dit, si \(a_1x_1+\dots+a_nx_n=0\) est une équation cartésienne de \(H\) dans \(\fami{B}\), alors les autres équations cartésiennes de \(H\) dans \(\fami{B}\) sont celles de la forme : \[\lambda a_1x_1+\dots+\lambda a_nx_n=0\] où \(\lambda\in\K\excluant\accol{0}\).
\end{prop}

\begin{rem}
Soient \(E\) un \(\K\)-espace vectoriel et \(\fami{B}=\paren{e_1,\dots,e_n}\) une base de \(E\).

Déterminons les formes linéaires sur \(E\).

\analyse

Soit \(l\in E\etoile\).

On note \(\paren{x_1,\dots,x_n}\in\K^n\) les coordonnées de \(x\) dans \(\fami{B}\) : \(x=x_1e_1+\dots+x_ne_n\).

On a \[\begin{aligned}
l\paren{x}&=l\paren{x_1e_1+\dots+x_ne_n} \\
&=x_1l\paren{e_1}+\dots+x_nl\paren{e_n} \\
&=a_1x_1+\dots+a_nx_n
\end{aligned}\] en posant \(a_1=l\paren{e_1},\dots,a_n=l\paren{e_n}\).

\synthese

Si \(a_1,\dots,a_n\in\K\) alors la fonction \(\fonctionlambda{E}{\K}{x_1e_1+\dots+x_ne_n}{a_1x_1+\dots+a_nx_n}\) est clairement une forme linéaire sur \(E\).

\conclusion

Les formes linéaires sur \(E\) sont les fonctions de la forme \[\fonction{l}{E}{\K}{x_1e_1+\dots+x_ne_n}{a_1x_1+\dots+a_nx_n}\] où \(a_1,\dots,a_n\in\K\).

De plus, comme \(\quantifs{\forall k\in\interventierii{1}{n}}a_k=l\paren{e_k}\), on remarque : \[l=0\ssi a_1=\dots=a_n=0\]
\end{rem}

\begin{dem}[De la proposition précédente]
Soient \(l\in E\etoile\excluant\accol{0}\) telle que \(H=\ker l\) et \(\paren{a_1,\dots,a_n}\in\K^n\excluant\accol{0}\) tel que \[\quantifs{\forall x_1,\dots,x_n\in\K}l\paren{x_1e_1+\dots+x_ne_n}=a_1x_1+\dots+a_nx_n.\]

On a \[\begin{aligned}
\quantifs{\forall x_1,\dots,x_n\in\K}x_1e_1+\dots+x_ne_n\in H&\ssi l\paren{x_1e_1+\dots+x_ne_n}=0 \\
&\ssi a_1x_1+\dots+a_nx_n=0.
\end{aligned}\]
\end{dem}

\begin{ex}
On pose \[F=\accol{\paren{x,y,z}\in\K^3\tq2x+z=0}\].

Alors \(F\) est un hyperplan de \(\K^3\) car les hyperplans de \(\K^3\) sont les noyaux de ses formes linéaires non-nulles.
\end{ex}

\subsection{Intersections d'hyperplans en dimension finie}

\begin{prop}
Soient \(E\) un espace vectoriel de dimension finie \(n\in\Ns\) et \(H_1,\dots,H_m\) des hyperplans de \(E\) (où \(m\in\Ns\)).

Alors : \[\dim\biginter_{i=1}^mH_i\geq n-m.\]
\end{prop}

\begin{dem}
Soient \(l_1,\dots,l_m\in E\etoile\excluant\accol{0}\) telles que \(\quantifs{\forall i\in\interventierii{1}{m}}H_i=\ker l_i\).

On pose \[\fonction{u}{E}{\K^m}{x}{\paren{l_1\paren{x},\dots,l_m\paren{x}}}\] de sorte que \[\ker u=\biginter_{i=1}^m\ker l_i=\biginter_{i=1}^mH_i.\]

Appliquons le théorème du rang à \(u\) : \[\dim E=\dim\ker u+\rg u\qquad\text{avec }\begin{dcases}
\dim E=n \\
\dim\ker u=\dim\biginter_{i=1}^mH_i \\
\rg u\leq m
\end{dcases}\]

Donc \(\dim\biginter_{i=1}^mH_i\geq n-m\).
\end{dem}

\begin{prop}
Soient \(E\) un espace vectoriel de dimension finie \(n\in\Ns\) et \(F\) un sous-espace vectoriel de \(E\) de dimension \(m\in\interventierii{0}{n-1}\).

Alors \(F\) est l'intersection de \(n-m\) hyperplans de \(E\).
\end{prop}

\begin{dem}~\\
Soit \(\paren{\underbrace{e_1,\dots,e_m}_{\text{base de }F},\dots,e_n}\) une base de \(E\) adaptée à \(F\).

On a \[F=\underbrace{\biginter_{i=m+1}^n\ker e_i\etoile}_{n-m\text{ hyperplans}}.\]
\end{dem}

\begin{cor}
Soient \(E\) un espace vectoriel de dimension finie \(n\in\Ns\) et \(F\) un sous-espace vectoriel de \(E\) tel que \(E\not=F\).

Alors il existe un hyperplan \(H\) de \(E\) tel que \(F\subset H\).
\end{cor}

\begin{dem}~\\
On a \(F=\biginter_{i=1}^mH_i\) avec \(m\geq1\) donc \(F\subset H_1\).
\end{dem}

\begin{ex}
Les droites vectorielles du plan \(\R^2\) sont les parties de \(\R^2\) définies par une équation cartésienne de la forme \[ax+by=0\] où \(\paren{a,b}\in\R^2\excluant\accol{\paren{0,0}}\).
\end{ex}

\begin{ex}
Les droites affines du plan \(\R^2\) sont les parties de \(\R^2\) définies par une équation cartésienne de la forme \[ax+by=c\] où \(\paren{a,b}\in\R^2\excluant\accol{\paren{0,0}}\) et \(c\in\R\).
\end{ex}

\begin{ex}
Les plans vectoriels de \(\R^3\) sont les parties de \(\R^3\) définies par une équation cartésienne de la forme \[ax+by+cz=0\] où \(\paren{a,b,c}\in\R^3\excluant\accol{\paren{0,0,0}}\).
\end{ex}

\begin{ex}
Les plans affines de \(\R^3\) sont les parties de \(\R^3\) définies par une équation cartésienne de la forme \[ax+by+cz=d\] où \(\paren{a,b,c}\in\R^3\excluant\accol{\paren{0,0,0}}\) et \(d\in\R\).
\end{ex}

\begin{ex}
Les droites vectorielles de \(\R^3\) sont les parties de \(\R^3\) définies par un système cartésien de la forme \[\begin{dcases}
ax+by+cz=0 \\
a\prim x+b\prim y+c\prim z=0
\end{dcases}\] où les vecteurs \(\tcoords{a}{b}{c}\) et \(\tcoords{a\prim}{b\prim}{c\prim}\) sont non-colinéaires.
\end{ex}

\begin{ex}
Les droites affines de \(\R^3\) sont les parties de \(\R^3\) définies par un système cartésien de la forme \[\begin{dcases}
ax+by+cz=d \\
a\prim x+b\prim y+c\prim z=d\prim
\end{dcases}\] où les vecteurs \(\tcoords{a}{b}{c}\) et \(\tcoords{a\prim}{b\prim}{c\prim}\) sont non-colinéaires et \(d,d\prim\in\R\).
\end{ex}

\chapter{Matrices I}

\minitoc

On considère un corps \(\K\) (en pratique, \(\K=\R\) ou \(\C\)).

\section{Matrices}

\begin{defi}[Matrice]
On appelle matrice toute famille de scalaires de la forme \[A=\paren{a_{ij}}_{\paren{i,j}\in\interventierii{1}{n}\times\interventierii{1}{p}}\in\K^{\interventierii{1}{n}\times\interventierii{1}{p}}\] avec \(n,p\in\Ns\).

On représente aussi une telle matrice sous la forme d'un tableau : \[A=\begin{pmatrix}
a_{11} & a_{12} & \dots & a_{1p} \\
a_{21} &  &  & \vdots \\
\vdots &  &  & \vdots \\
a_{n1} & \dots & \dots & a_{np}
\end{pmatrix}\] et on dit que \(A\) est une matrice à \(n\) lignes et \(p\) colonnes ou que \(A\) est de taille \(\paren{n,p}\) ou encore que \(A\) est une matrice \(n\times p\).

Son coefficient \(a_{ij}\) est appelé coefficient de \(A\) en position \(\paren{i,j}\).

Si \(n=1\), on dit que \(A\) est une matrice-ligne.

Si \(p=1\), on dit que \(A\) est une matrice-colonne.

L'ensemble des matrices de taille \(\paren{n,p}\) à coefficients dans \(\K\) est noté \[\M{np}=\K^{\interventierii{1}{n}\times\interventierii{1}{p}}.\]

On pose enfin, si \(n=p\) : \[\M{n}=\M{nn}=\K^{\interventierii{1}{n}^2}.\]

Une matrice de taille \(\paren{n,n}\) est appelée matrice carrée de taille \(n\).
\end{defi}

\begin{ex}
On a : \[\paren{100i+j}_{\paren{i,j}\in\interventierii{1}{2}\times\interventierii{1}{4}}=\begin{pmatrix}
101 & 102 & 103 & 104 \\
201 & 202 & 203 & 204
\end{pmatrix}\in\M{24}[\R].\]
\end{ex}

\begin{nota}
Soit \(n\in\Ns\).

On appelle matrice identité de taille \(n\) la matrice carrée \[I_n=\paren{\delta_{ij}}_{\paren{i,j}\in\interventierii{1}{n}^2}=\begin{pmatrix}
1 & 0 & \dots & 0 \\
0 & \ddots & \ddots & \vdots \\
\vdots & \ddots & \ddots & 0 \\
0 & \dots & 0 & 1
\end{pmatrix}\in\M{n}.\]
\end{nota}

\begin{nota}[Non-officielle]
Soient \(n,p\in\Ns\).

On appelle matrice nulle de taille \(\paren{n,p}\) la matrice de taille \(\paren{n,p}\) dont tous les coefficients sont nuls et on la note parfois \(0_{np}\) : \[0_{np}=\paren{0}_{\paren{i,j}\in\interventierii{1}{n}\times\interventierii{1}{p}}=\begin{pmatrix}
0 & \dots & 0 \\
\vdots & & \vdots \\
0 & \dots & 0
\end{pmatrix}\in\M{np}.\]

Enfin, on pose : \(0_n=0_{nn}\in\M{n}\).
\end{nota}

\section{Combinaisons linéaires de matrices}

\begin{prop}
Soient \(n,p\in\Ns\).

L'ensemble \(\M{np}\) des matrices à \(n\) lignes et \(p\) colonnes est naturellement un espace vectoriel.
\end{prop}

\begin{nota}[Matrices élémentaires]
Soient \(n,p\in\Ns\).

On définit pour tout \(\paren{i_0,j_0}\in\interventierii{1}{n}\times\interventierii{1}{p}\) la matrice élémentaire \[E_{i_0j_0}=\paren{\delta_{ii_0}\delta_{jj_0}}_{\paren{i,j}\in\interventierii{1}{n}\times\interventierii{1}{p}}\in\M{np}.\]

Tous ses coefficients sont nuls, sauf celui en position \(\paren{i_0,j_0}\) qui vaut \(1\).
\end{nota}

\begin{ex}
Les matrices élémentaires de \(\M{2}\) sont : \[E_{11}=\begin{pmatrix}
1 & 0 \\
0 & 0
\end{pmatrix}\qquad E_{12}=\begin{pmatrix}
0 & 1 \\
0 & 0
\end{pmatrix}\qquad E_{21}=\begin{pmatrix}
0 & 0 \\
1 & 0
\end{pmatrix}\qquad E_{22}=\begin{pmatrix}
0 & 0 \\
0 & 1
\end{pmatrix}\]
\end{ex}

\begin{prop}
Soient \(n,p\in\Ns\).

La famille des matrices élémentaires de \(\M{np}\) \[\paren{E_{ij}}_{\paren{i,j}\in\interventierii{1}{n}\times\interventierii{1}{p}}\in\M{np}^{\interventierii{1}{n}\times\interventierii{1}{p}}\] est une base de \(\M{np}\) appelée base canonique de \(\M{np}\).

Les coordonnées d'une matrice dans cette base sont ses coefficients.
\end{prop}

\begin{dem}
On note \(\fami{B}\) la famille des matrices élémentaires de \(\M{np}\).

Montrons que \(\fami{B}\) est libre.

Soit \(\paren{\lambda_{ij}}_{\paren{i,j}}\in\K^{\interventierii{1}{n}\times\interventierii{1}{p}}\) telle que \[\sum_{i=1}^n\sum_{j=1}^p\lambda_{ij}E_{ij}=0_{np}.\]

On a \(\begin{pmatrix}
\lambda_{11} & \dots & \lambda_{1p} \\
\vdots &  & \vdots \\
\lambda_{n1} & \dots & \lambda_{np}
\end{pmatrix}=0_{np}\).

Donc \[\quantifs{\forall i\in\interventierii{1}{n};\forall j\in\interventierii{1}{p}}\lambda_{ij}=0.\]

Donc \(\fami{B}\) est libre.

Montrons que \(\fami{B}\) est génératrice de \(\M{np}\).

Soit \(A=\begin{pmatrix}
a_{11} & \dots & a_{1p} \\
\vdots &  & \vdots \\
a_{n1} & \dots & a_{np}
\end{pmatrix}\in\M{np}\).

On remarque \[A=\sum_{i=1}^n\sum_{j=1}^pa_{ij}E_{ij}.\]

Donc \(\fami{B}\) est génératrice de \(\M{np}\).

Donc \(\fami{B}\) est une base de \(\M{np}\).
\end{dem}

\begin{ex}
On a : \[\quantifs{\forall a,b,c,d\in\K}\begin{pmatrix}
a & b \\
c & d
\end{pmatrix}=a E_{11}+b E_{12}+c E_{21}+d E_{22}.\]
\end{ex}

\begin{cor}
On a : \[\quantifs{\forall n,p\in\Ns}\dim\M{np}=np.\]
\end{cor}

\section{Produit matriciel}

\subsection{Produits de matrices}

\begin{defi}[Produit matriciel]
Soient \(m,n,p\in\Ns\), \(A=\paren{a_{ij}}_{\paren{i,j}\in\interventierii{1}{m}\times\interventierii{1}{n}}\in\M{mn}\) et \(B=\paren{b_{jk}}_{\paren{j,k}\in\interventierii{1}{n}\times\interventierii{1}{p}}\in\M{np}\).

On appelle produit des matrices \(A\) et \(B\) la matrice \(C\) de taille \(\paren{m,p}\) : \[C=\paren{c_{ik}}_{\paren{i,k}\in\interventierii{1}{m}\times\interventierii{1}{p}}=A\times B=AB\in\M{mp}\] définie par : \[\quantifs{\forall\paren{i,k}\in\interventierii{1}{m}\times\interventierii{1}{p}}c_{ik}=\sum_{j=1}^na_{ij}b_{jk}.\]

L'application ainsi définie \[\M{mn}\times\M{np}\to\M{mp}\] est appelée le produit matriciel.
\end{defi}

\begin{prop}[Bilinéarité du produit matriciel]\thlabel{prop:bilinéaritéDuProduitMatriciel}
Soient \(m,n,p\in\Ns\).

Le produit matriciel \(\fonctionlambda{\M{mn}\times\M{np}}{\M{mp}}{\paren{A,B}}{AB}\) est bilinéaire, \cad qu'on a : \[\begin{dcases}
\quantifs{\forall\lambda_1,\lambda_2\in\K;\forall A_1,A_2\in\M{mn};\forall B\in\M{np}}\paren{\lambda_1A_1+\lambda_2A_2}B=\lambda_1A_1B+\lambda_2A_2B \\
\quantifs{\forall\mu_1,\mu_2\in\K;\forall A\in\M{mn};\forall B_1,B_2\in\M{np}}A\paren{\mu_1B_1+\mu_2B_2}=\mu_1AB_1+\mu_2AB_2
\end{dcases}\]

Cela équivaut à : \[\begin{aligned}
\quantifs{\forall\lambda_1,\lambda_2,\mu_1,\mu_2\in\K;\forall A_1,A_2\in\M{mn};\forall B_1,B_2\in\M{np}}\paren{\lambda_1A_1+\lambda_2A_2}\paren{\mu_1B_1+\mu_2B_2}&=\lambda_1\mu_1A_1B_1 \\
&\color{white}=\color{black}+\lambda_1\mu_2A_1B_2 \\
&\color{white}=\color{black}+\lambda_2\mu_1A_2B_1 \\
&\color{white}=\color{black}+\lambda_2\mu_2A_2B_2
\end{aligned}\]
\end{prop}

\begin{prop}[Matrices nulles, matrices identités]\thlabel{prop:produitMatricielMatricesNullesEtMatricesIdentités}
Soient \(n,p\in\Ns\).

On a : \[0_n\times A=A\times0_p=0_{np}\qquad\text{et}\qquad I_n\times A=A\times I_p=A.\]
\end{prop}

\begin{dem}
Les égalités sont claires pour les produits avec les matrices nulles.

On pose \(C=\paren{c_{ik}}_{\paren{i,k}}=I_nA\).

Montrons que \(C=A\).

On a \(C\in\M{np}\) et \(\quantifs{\forall i\in\interventierii{1}{n};\forall k\in\interventierii{1}{p}}c_{ik}=\sum_{j=1}^n\delta_{ij}a_{jk}=a_{ik}\).

Donc \(C=A\).

On montre de même \(AI_p=A\).
\end{dem}

\begin{rem}[Matrices \guillemets{scalaires}]
Soient \(m,n,p\in\Ns\).

On a, selon la \thref{prop:bilinéaritéDuProduitMatriciel} : \[\quantifs{\forall\lambda\in\K;\forall A\in\M{mn};\forall B\in\M{np}}\paren{\lambda A}B=A\paren{\lambda B}=\lambda AB.\]

D'où, selon la \thref{prop:produitMatricielMatricesNullesEtMatricesIdentités} : \[\quantifs{\forall\lambda\in\K;\forall A\in\M{np}}\lambda A=\paren{\lambda I_n}A=A\paren{\lambda I_p}.\]

En particulier, les matrices de la forme \(\lambda I_n\) où \(\lambda\in\K\) (qu'on appelle parfois \guillemets{matrices scalaires}) commutent avec toute matrice carrée de taille \(n\) : \[\quantifs{\forall\lambda\in\K;\forall A\in\M{n}}\lambda A=\paren{\lambda I_n}A=A\paren{\lambda I_n}.\]
\end{rem}

\begin{exoex}
Compléter la table de multiplication (ne rien mettre dans la case quand le produit n'est pas défini) :

\begin{center}
\large
\begin{tabular}{|c|c|c|c|c|c|c|c|}
\hline
\(\times\) & \(\begin{pmatrix}1\\2\\3\end{pmatrix}\) & \(\begin{pmatrix}0&0&7\end{pmatrix}\) & \(\begin{pmatrix}-1&2\\-1&3\\-1&0\end{pmatrix}\) & \(\begin{pmatrix}2&4\end{pmatrix}\) & \(\begin{pmatrix}9\end{pmatrix}\) & \(\begin{pmatrix}1&0\\0&1\end{pmatrix}\) & \(\begin{pmatrix}2\\0\end{pmatrix}\)\\
\hline
\(\begin{pmatrix}1&0&0\\0&1&0\\0&0&1\end{pmatrix}\) &&&&&&&\\
\hline
\(\begin{pmatrix}1&2&3&4\end{pmatrix}\) &&&&&&&\\
\hline
\(\begin{pmatrix}1\\2\\3\\4\end{pmatrix}\) &&&&&&&\\
\hline
\(\begin{pmatrix}7\end{pmatrix}\) &&&&&&&\\
\hline
\(\begin{pmatrix}1&1\\0&0\\1&1\end{pmatrix}\) &&&&&&&\\
\hline
\(\begin{pmatrix}3&-3\end{pmatrix}\) &&&&&&&\\
\hline
\(\begin{pmatrix}1\\0\end{pmatrix}\)&&&&&&&\\
\hline
\end{tabular}
\end{center}
\end{exoex}

\begin{corr}~\\
\begin{center}
\large
\begin{tabular}{|c|c|c|c|c|c|c|c|}
\hline
\(\times\) & \(\begin{pmatrix}1\\2\\3\end{pmatrix}\) & \(\begin{pmatrix}0&0&7\end{pmatrix}\) & \(\begin{pmatrix}-1&2\\-1&3\\-1&0\end{pmatrix}\) & \(\begin{pmatrix}2&4\end{pmatrix}\) & \(\begin{pmatrix}9\end{pmatrix}\) & \(\begin{pmatrix}1&0\\0&1\end{pmatrix}\) & \(\begin{pmatrix}2\\0\end{pmatrix}\)\\
\hline
\(\begin{pmatrix}1&0&0\\0&1&0\\0&0&1\end{pmatrix}\) & \(\begin{pmatrix}1\\2\\3\end{pmatrix}\) && \(\begin{pmatrix}-1&2\\-1&3\\-1&0\end{pmatrix}\) &&&&\\
\hline
\(\begin{pmatrix}1&2&3&4\end{pmatrix}\) &&&&&&&\\
\hline
\(\begin{pmatrix}1\\2\\3\\4\end{pmatrix}\) && \(\begin{pmatrix}0&0&7\\0&0&14\\0&0&21\\0&0&28\end{pmatrix}\) && \(\begin{pmatrix}2&4\\4&8\\6&12\\8&16\end{pmatrix}\) & \(\begin{pmatrix}9\\18\\27\\36\end{pmatrix}\) &&\\
\hline
\(\begin{pmatrix}7\end{pmatrix}\) && \(\begin{pmatrix}0&0&49\end{pmatrix}\) && \(\begin{pmatrix}14&28\end{pmatrix}\) & \(\begin{pmatrix}63\end{pmatrix}\) &&\\
\hline
\(\begin{pmatrix}1&1\\0&0\\1&1\end{pmatrix}\) &&&&&& \(\begin{pmatrix}1&1\\0&0\\1&1\end{pmatrix}\) & \(\begin{pmatrix}2\\0\\2\end{pmatrix}\)\\
\hline
\(\begin{pmatrix}3&-3\end{pmatrix}\) &&&&&& \(\begin{pmatrix}3&-3\end{pmatrix}\) & \(\begin{pmatrix}6\end{pmatrix}\)\\
\hline
\(\begin{pmatrix}1\\0\end{pmatrix}\) && \(\begin{pmatrix}0&0&7\\0&0&0\end{pmatrix}\) && \(\begin{pmatrix}2&4\\0&0\end{pmatrix}\) & \(\begin{pmatrix}9\\0\end{pmatrix}\) &&\\
\hline
\end{tabular}
\end{center}
\end{corr}

\begin{prop}[Associativité du produit matriciel]\thlabel{prop:associativitéDuProduitMatriciel}
Soient \(m,n,p,q\in\Ns\).

On a : \[\quantifs{\forall A\in\M{mn};\forall B\in\M{np};\forall C\in\M{pq}}\paren{AB}C=A\paren{BC}.\]

En pratique, on n'écrit pas les parenthèses.
\end{prop}

\begin{dem}
On a \[\begin{dcases}
AB\in\M{mp}\text{ donc }\paren{AB}C\in\M{mq} \\
BC\in\M{nq}\text{ donc }A\paren{BC}\in\M{mq}
\end{dcases}\]

Posons \(D=AB\) ; \(E=BC\) ; \(F=DC\) et \(G=AE\).

Montrons que \(F=G\).

On a vu que \(F,G\in\M{mq}\).

Par définition du produit matriciel, on a \(\quantifs{\forall i\in\interventierii{1}{m};\forall k\in\interventierii{1}{p}}d_{ik}=\sum_{j=1}^na_{ij}b_{jk}\) donc \[\quantifs{\forall i\in\interventierii{1}{m};\forall l\in\interventierii{1}{q}}f_{il}=\sum_{k=1}^pd_{ik}c_{kl}=\sum_{j=1}^n\sum_{k=1}^pa_{ij}b_{jk}c_{kl}.\]

De même, on a \(\quantifs{\forall j\in\interventierii{1}{n};\forall l\in\interventierii{1}{q}}e_{jl}=\sum_{k=1}^pb_{jk}c_{kl}\) donc \[\quantifs{\forall i\in\interventierii{1}{m};\forall l\in\interventierii{1}{q}}g_{il}=\sum_{j=1}^na_{ij}e_{jl}=\sum_{j=1}^n\sum_{k=1}^pa_{ij}b_{jk}c_{kl}.\]

D'où \(F=G\).
\end{dem}

\begin{prop}[Produit de matrices élémentaires]
Soient \(m,n,p\in\Ns\).

Considérons deux matrices élémentaires \[E_{ij}\in\M{mn}\qquad\text{et}\qquad E_{kl}\in\M{np}\] avec \(i\in\interventierii{1}{m}\), \(j,k\in\interventierii{1}{n}\) et \(l\in\interventierii{1}{p}\).

Alors on a : \[E_{ij}E_{kl}=\delta_{jk}E_{il}=\begin{dcases}
E_{il} &\text{si }j=k \\
0_{mp} &\text{sinon}
\end{dcases}\]
\end{prop}

\begin{dem}
On a \(E_{ij}\in\M{mn}\) et \(E_{kl}\in\M{np}\) donc \[E_{ij}E_{kl}\in\M{mp}.\]

Posons \(A=\paren{a_{xz}}_{\paren{x,z}}=E_{ij}\times E_{kl}\).

On a : \[\begin{aligned}
\quantifs{\forall x\in\interventierii{1}{m};\forall z\in\interventierii{1}{p}}a_{xz}&=\sum_{y=1}^n\underbrace{\delta_{xi}\delta_{yj}}_{\substack{\text{coefficient de}\\E_{ij}\text{ en }\paren{x,y}}}\times\underbrace{\delta_{yk}\delta_{zl}}_{\substack{\text{coefficient de}\\E_{kl}\text{ en }\paren{y,z}}} \\
&=\begin{dcases}
1 &\text{si }\paren{x,z}=\paren{i,l}\text{ et }j=k \\
0 &\text{sinon}
\end{dcases}
\end{aligned}\]

D'où le résultat.
\end{dem}

\subsection{Produits de matrices carrées}

\begin{prop}[Anneau des matrices carrées]
Soit \(n\in\Ns\).

Alors \(\anneau{\M{n}}\) est un anneau.

Son élément neutre pour \(+\) est la matrice nulle \(0_n\).

Son élément neutre pour \(\times\) est la matrice identité \(I_n\).

C'est un anneau non-commutatif si \(n\geq2\).
\end{prop}

\begin{dem}
\(\groupe{\M{n}}\) est clairement un groupe abélien car \(\anneau{\M{n}}[+][\cdot]\) est un \(\K\)-espace vectoriel.

\(\times\) est une loi de composition interne sur \(\M{n}\) : on a \[\quantifs{\forall A,B\in\M{n}}AB\in\M{n},\] c'est une loi associative selon la \thref{prop:associativitéDuProduitMatriciel} et elle admet \(I_n\) comme élément neutre selon la \thref{prop:produitMatricielMatricesNullesEtMatricesIdentités}.

De plus, elle est distributive par rapport à \(+\) car le produit matriciel est bilinéaire.

Donc \(\anneau{\M{n}}\) est un anneau.

Enfin, si \(n\geq2\), l'anneau est non-commutatif : \[\begin{dcases}
E_{12}E_{21}=E_{11} \\
E_{21}E_{12}=E_{22}\not=E_{11}
\end{dcases}\]
\end{dem}

\begin{rem}
Tout ce qu'on a vu sur les anneaux s'applique donc à l'anneau des matrices carrées.

Par exemple, on note \(A^k\) la puissance k-ème d'une matrice si \(k\in\N\), voire si \(k\in\Z\) dans le cas où \(A\) est inversible.

La proposition suivante s'applique aussi :
\end{rem}

\begin{prop}
Soit \(n\in\Ns\).

Si \(A\) et \(B\) sont deux éléments de \(\M{n}\) qui commutent (\cad tels que \(AB=BA\)), alors on a la formule du binôme de Newton : \[\quantifs{\forall m\in\N}\paren{A+B}^m=\sum_{k=0}^m\binom{k}{m}A^kB^{m-k}\] et la formule : \[\quantifs{\forall m\in\N}A^m-B^m=\paren{A-B}\sum_{k=0}^{m-1}A^kB^{m-1-k}=\paren{\sum_{k=0}^{m-1}A^kB^{m-1-k}}\paren{A-B}.\]
\end{prop}

\begin{rem}[\guillemets{Diviseurs de zéro}]
Soit \(n\in\interventierie{2}{\pinf}\).

On a : \[\quantifs{\exists A,B\in\M{n}}\begin{dcases}
A\not=0 \\
B\not=0 \\
AB=0
\end{dcases}\]
\end{rem}

\begin{dem}
On a : \[\begin{dcases}
E_{11}E_{22}=0 \\
E_{11}\not=0 \\
E_{22}\not=0
\end{dcases}\]
\end{dem}

\begin{rem}[Matrices nilpotentes]
Soit \(n\in\interventierie{2}{\pinf}\).

On a : \[\quantifs{\exists A\in\M{n};\exists k\in\Ns}\begin{dcases}
A\not=0 \\
A^k=0
\end{dcases}\]

Une telle matrice \(A\) est appelée matrice nilpotente.
\end{rem}

\begin{dem}
On a : \[\begin{dcases}
E_{12}^2=0 \\
E_{12}\not=0
\end{dcases}\]
\end{dem}

\subsection{Matrices inversibles, groupe linéaire}

\begin{defi}[Matrice inversible]
Soit \(n\in\Ns\).

On dit qu'une matrice carrée \(A\in\M{n}\) est inversible si elle est inversible dans l'anneau \(\anneau{\M{n}}\).

On rappelle que cela signifie qu'elle admet un inverse à gauche et à droite pour le produit matriciel : \[\quantifs{\exists B\in\M{n}}AB=BA=I_n.\]

On sait que la matrice \(B\) est alors unique, on l'appelle l'inverse de \(A\) et on la note \(A\inv\).
\end{defi}

\begin{ex}
On a : \[\begin{pmatrix}
1 & 1 \\
0 & 1
\end{pmatrix}\inv=\begin{pmatrix}
1 & -1 \\
0 & 1
\end{pmatrix}.\]

En effet, on a bien : \[\begin{pmatrix}
1 & 1 \\
0 & 1
\end{pmatrix}\begin{pmatrix}
1 & -1 \\
0 & 1
\end{pmatrix}=I_2=\begin{pmatrix}
1 & -1 \\
0 & 1
\end{pmatrix}\begin{pmatrix}
1 & 1 \\
0 & 1
\end{pmatrix}.\]
\end{ex}

\begin{rem}
On verra (\cf \thref{bilan:matriceInversible}) que pour qu'un matrice \(A\in\M{n}\) soit inversible, il suffit qu'elle soit \guillemets{inversible à gauche} ou \guillemets{inversible à droite} : \[\croch{\quantifs{\exists B\in\M{n}}BA=I_n}\qquad\text{ou}\qquad\croch{\quantifs{\exists B\in\M{n}}AB=I_n}.\]

La matrice \(B\) est alors automatiquement l'inverse de \(A\) (à gauche et à droite).
\end{rem}

\begin{defprop}[Groupe linéaire]
Soit \(n\in\Ns\).

L'ensemble des matrices carrées de taille \(n\) qui sont inversibles est appelé le groupe linéaire d'ordre \(n\) et est noté \(\GL{n}\).

On a donc : \[\GL{n}=\accol{A\in\M{n}\tq\quantifs{\exists B\in\M{n}}AB=BA=I_n}.\]

Le groupe linéaire est un groupe pour le produit matriciel.

Son élément neutre est la matrice identité \(I_n\).

Le produit de deux matrices inversibles est une matrice inversible et on a : \[\quantifs{\forall P,Q\in\GL{n}}\paren{PQ}\inv=Q\inv P\inv.\]
\end{defprop}

\begin{dem}
On a vu que \(\anneau{\M{n}}\) est un anneau d'élément neutre \(I_n\) pour \(\times\).

On sait que l'ensemble de ses éléments inversibles est un groupe pour \(\times\), d'élément neutre \(I_n\).
\end{dem}

\section{Matrices diagonales, matrices triangulaires}

\begin{defi}[Matrice diagonale]
Soit \(n\in\Ns\).

Une matrice carrée \(A=\paren{a_{ij}}_{\paren{i,j}\in\interventierii{1}{n}^2}\in\M{n}\) est dite diagonale si : \[\quantifs{\forall i,j\in\interventierii{1}{n}}\croch{i\not=j\imp a_{ij}=0},\] \cad si \(A\) s'écrit : \[A=\begin{pmatrix}
a_{11} & 0 & \dots & 0 \\
0 & \ddots & \ddots & \vdots \\
\vdots & \ddots & \ddots & 0 \\
0 & \dots & 0 & a_{nn}
\end{pmatrix}.\]
\end{defi}

\begin{nota}
Soient \(n\in\Ns\) et \(\lambda_1,\dots,\lambda_n\in\K\).

La matrice diagonale de taille \(n\) et de coefficients diagonaux \(\lambda_1,\dots,\lambda_n\) est parfois notée \[\diag{\lambda_1,\dots,\lambda_n}.\]
\end{nota}

\begin{prop}
Soit \(n\in\Ns\).

L'ensemble des matrices diagonales de \(\M{n}\) est :

\begin{itemize}
\item un \(\K\)-espace vectoriel de dimension \(n\) (sous-espace vectoriel de \(\M{n}\)) ; \\

\item un anneau commutatif (sous-anneau de \(\anneau{\M{n}}\)).
\end{itemize}
\end{prop}

\begin{dem}
Notons \(F\) l'ensemble des matrices diagonales de \(\M{n}\).

On remarque \(F=\Vect{E_{11},E_{22}\dots,E_{nn}}\) donc \(F\) est un sous-espace vectoriel de \(\M{n}\).

De plus, \(\paren{E_{11},E_{22},\dots,E_{nn}}\) est génératrice de \(F\) et libre (sous-famille de famille libre (base canonique de \(\M{n}\))) donc c'est une base de \(F\) et on a \(\dim F=n\).

Montrons que \(F\) est un sous-anneau de \(\M{n}\).

On a \(I_n\in F\), on sait que \(F\) est un sous-espace vectoriel de \(\M{n}\) et on a : \[\begin{aligned}
\quantifs{\forall\lambda_1,\dots,\lambda_n,\mu_1,\dots,\mu_n\in\K}\diag{\lambda_1,\dots,\lambda_n}\diag{\mu_1,\dots,\mu_n}&=\paren{\lambda_1E_{11}+\dots+\lambda_nE_{nn}}\paren{\mu_1E_{11}+\dots+\mu_nE_{nn}} \\
&=\lambda_1\mu_1E_{11}+\dots+\lambda_n\mu_nE_{nn} \\
&=\diag{\lambda_1\mu_1,\dots,\lambda_n\mu_n}
\end{aligned}\]

Donc \(F\) est stable par produit et deux matrices diagonales commutent toujours.

Donc \(F\) est un sous-anneau de \(\M{n}\).
\end{dem}

\begin{defi}[Matrice triangulaire]
Soit \(n\in\Ns\).

Une matrice carrée \(A=\paren{a_{ij}}_{\paren{i,j}\in\interventierii{1}{n}^2}\in\M{n}\) est dite triangulaire supérieure si : \[\quantifs{\forall i,j\in\interventierii{1}{n}}\croch{i>j\imp a_{ij}=0},\] \cad si \(A\) s'écrit : \[A=\begin{pmatrix}
a_{11} & \dots & \dots & a_{1n} \\
0 & \ddots & & \vdots \\
\vdots & \ddots & \ddots & \vdots \\
0 & \dots & 0 & a_{nn}
\end{pmatrix}.\]

La matrice \(A\) est dite triangulaire inférieure si : \[\quantifs{\forall i,j\in\interventierii{1}{n}}\croch{i<j\imp a_{ij}=0},\] \cad si \(A\) s'écrit : \[A=\begin{pmatrix}
a_{11} & 0 & \dots & 0 \\
\vdots & \ddots & \ddots & \vdots \\
\vdots & & \ddots & 0 \\
a_{n1} & \dots & \dots & a_{nn}
\end{pmatrix}.\]

Cela revient à dire que sa transposée \(\trans{A}\) est triangulaire supérieure.

On appelle matrice triangulaire toute matrice triangulaire supérieure ou triangulaire inférieure.
\end{defi}

\begin{prop}
Soit \(n\in\Ns\).

L'ensemble des matrices triangulaires supérieures de \(\M{n}\) est :

\begin{itemize}
\item un \(\K\)-espace vectoriel de dimension \(\dfrac{n\paren{n+1}}{2}\) (sous-espace vectoriel de \(\M{n}\)) ; \\

\item un sous-anneau de \(\anneau{\M{n}}\).
\end{itemize}

Même chose pour l'ensemble des matrices triangulaires inférieures.
\end{prop}

\begin{dem}
Notons \(F\) l'ensemble des matrices triangulaires supérieures.

On pose \(I=\accol{\paren{i,j}\in\interventierii{1}{n}^2\tq i\leq j}\).

Alors \(F=\Vect{\paren{E_{ij}}_{\paren{i,j}\in I}}\).

Donc \(F\) est un sous-espace vectoriel de \(\M{n}\).

De plus, \(\paren{E_{ij}}_{\paren{i,j}\in I}\) est une famille génératrice de \(F\) et libre (car c'est une sous-famille de la base canonique de \(\M{n}\)) donc c'est une base de \(F\) et \(\dim F=\Card I=\sum_{k=1}^nk=\dfrac{n\paren{n+1}}{2}\).

Montrons que \(F\) est un sous-anneau de \(\anneau{\M{n}}\).

On a \(I_n\in F\) et \(\quantifs{\forall A,B\in F}A-B\in F\).

Soient \(A=\paren{a_{ij}}_{\paren{i,j}},B=\paren{b_{jk}}_{\paren{j,k}}\in F\).

Posons \(C=\paren{c_{ik}}_{\paren{i,k}}=AB\).

Montrons que \(C\in F\), \cad \[\quantifs{\forall\paren{i,k}\in\interventierii{1}{n}^2}\croch{i>k\imp c_{ik}=0}.\]

Soit \(\paren{i,k}\in\interventierii{1}{n}^2\) tel que \(i>k\).

On a \[\begin{aligned}
c_{ik}&=\sum_{j=1}^n\underbrace{a_{ij}}_{\substack{=0\text{ si} \\ i>j}}\underbrace{b_{jk}}_{\substack{=0\text{ si} \\ j>k}} \\
&=\sum_{j=1}^{i-1}\underbrace{a_{ij}}_{\substack{=0 \\ \text{car }j<i}}b_{jk}+\sum_{j=i}^{n}a_{ij}\underbrace{b_{jk}}_{\substack{=0 \\ \text{car }j\geq i>k}}
\end{aligned}\]

Donc \(F\) est un sous-anneau de \(\M{n}\).
\end{dem}

\section{Transposition}

\begin{defi}[Transposée d'une matrice]
Soient \(n,p\in\Ns\) et \(A=\paren{a_{ij}}_{\paren{i,j}\in\interventierii{1}{n}\times\interventierii{1}{p}}\in\M{np}\) une matrice de taille \(\paren{n,p}\).

On appelle transposée de \(A\) et on note \(\trans{A}\) ou \(A^T\) la matrice de taille \(\paren{p,n}\) suivante : \[\trans{A}=\paren{a_{ij}}_{\paren{j,i}\in\interventierii{1}{p}\times\interventierii{1}{n}}\in\M{pn}.\]

On a donc : \[\trans{\begin{pmatrix}
a_{11} & a_{12} & \dots & a_{1p} \\
a_{21} &  &  & \vdots \\
\vdots &  &  & \vdots \\
a_{n1} & \dots & \dots & a_{np}
\end{pmatrix}}=\begin{pmatrix}
a_{11} & a_{21} & \dots & a_{n1} \\
a_{12} &  &  & \vdots \\
\vdots &  &  & \vdots \\
a_{1p} & \dots & \dots & a_{np}
\end{pmatrix}\in\M{pn}.\]
\end{defi}

\begin{ex}
On a : \[\trans{\begin{pmatrix}
1 & 2 & 3 \\
4 & 5 & 6
\end{pmatrix}}=\begin{pmatrix}
1 & 4 \\
2 & 5 \\
3 & 6
\end{pmatrix}.\]
\end{ex}

\begin{prop}[Linéarité de la transposition]
Soient \(n,p\in\Ns\).

L'application \[\fonctionlambda{\M{np}}{\M{pn}}{A}{\trans{A}}\] est linéaire.
\end{prop}

\begin{dem}
\note{Exercice}
\end{dem}

\begin{prop}[Transposée d'un produit]
Soient \(m,n,p\in\Ns\).

On a : \[\quantifs{\forall A\in\M{mn};\forall B\in\M{np}}\trans{\paren{AB}}=\trans{B}\trans{A}\in\M{pm}.\]
\end{prop}

\begin{dem}
On a \(AB\in\M{mp}\) donc \(\trans{\paren{AB}}\in\M{pm}\) et \(\begin{dcases}
\trans{B}\in\M{pn} \\
\trans{A}\in\M{nm}
\end{dcases}\) donc \(\trans{B}\trans{A}\in\M{pm}\).

On pose \(C=AB\) ; \(D=\trans{C}\) et \(F=\trans{B}\trans{A}\).

On a : \[\quantifs{\forall i\in\interventierii{1}{m};\forall k\in\interventierii{1}{p}}\begin{dcases}
c_{ik}=\sum_{j=1}^na_{ij}b_{jk} \\
d_{ki}=c_{ik}=\sum_{j=1}^na_{ij}b_{jk} \\
f_{ki}=\sum_{j=1}^nb_{jk}a_{ij}=d_{ki}
\end{dcases}\]

D'où l'égalité.
\end{dem}

\begin{cor}
Soit \(n\in\Ns\).

On a : \[\quantifs{\forall A\in\M{n};\forall k\in\N}\trans{\paren{A^k}}=\paren{\trans{A}}^k.\]

On écrit donc simplement \guillemets{\(\trans{A}^k\)}.
\end{cor}

\begin{prop}[Transposition et inversibilité]
Soit \(n\in\Ns\).

On a : \[\quantifs{\forall A\in\M{n}}A\in\GL{n}\ssi\trans{A}\in\GL{n}.\]

Les matrices \(\trans{\paren{A\inv}}\) et \(\paren{\trans{A}}\inv\) sont donc égales et notées \(\trans{A}\inv\).
\end{prop}

\begin{dem}
Soit \(A\in\M{n}\).

\impdir

Supposons \(A\in\GL{n}\).

On a \(\begin{dcases}
AA\inv=I_n \\
A\inv A=I_n
\end{dcases}\) donc \(\begin{dcases}
\trans{\paren{A\inv}}\trans{A}=\trans{I_n} \\
\trans{A}\trans{\paren{A\inv}}=\trans{I_n}
\end{dcases}\)

Or \(\trans{I_n}=I_n\) donc \(\trans{A}\) est inversible, d'inverse \(\paren{\trans{A}}\inv=\trans{\paren{A\inv}}\).

\imprec

Supposons \(\trans{A}\in\GL{n}\).

Alors \(\trans{\paren{\trans{A}}}\in\GL{n}\) selon \impdir

Donc \(A\in\GL{n}\).
\end{dem}

\section{Matrices symétriques, matrices antisymétriques}

\begin{defi}
Soit \(n\in\Ns\).

Une matrice \(A\in\M{n}\) est dite \(\begin{dcases}
\text{symétrique} &\text{si }\trans{A}=A \\
\text{antisymétrique} &\text{si }\trans{A}=-A
\end{dcases}\)
\end{defi}

\begin{prop}
On suppose ici que \(\K\) est un sous-corps de \(\C\).

On note \(\sym{n}\) (respectivement \(\antisym{n}\)) l'ensemble des matrices symétriques (respectivement antisymétriques) de \(\M{n}\).

Alors \(\sym{n}\) et \(\antisym{n}\) sont deux sous-espaces vectoriels supplémentaires dans \(\M{n}\) : \[\M{n}=\sym{n}\oplus\antisym{n}.\]

On a : \[\dim\sym{n}=\dfrac{n\paren{n+1}}{2}\qquad\text{et}\qquad\dim\antisym{n}=\dfrac{n\paren{n-1}}{2}.\]
\end{prop}

\begin{dem}
Posons \(I=\accol{\paren{i,j}\in\interventierii{1}{n}^2\tq i<j}\).

On a \(\antisym{n}=\Vect{\paren{E_{ij}-E_{ji}}_{\paren{i,j}\in I}}\) donc \(\antisym{n}\) est un sous-espace vectoriel de \(\M{n}\) de dimension \[\begin{aligned}
\dim\antisym{n}&=\Card I \\
&=\sum_{k=1}^{n-1}k \\
&=\dfrac{n\paren{n-1}}{2}.
\end{aligned}\]

Notons \(\fami{B}\) la famille obtenue en juxtaposant les familles \(\paren{E_{ii}}_{i\in\interventierii{1}{n}}\) et \(\paren{E_{ij}+E_{ji}}_{\paren{i,j}\in I}\).

\(\fami{B}\) est clairement libre et on a \(\sym{n}=\Vect{\fami{B}}\).

Donc \(\sym{n}\) est un sous-espace vectoriel de \(\M{n}\) de dimension \[\begin{aligned}
\dim\sym{n}&=n+\Card I \\
&=n+\sum_{k=1}^{n-1}k \\
&=\dfrac{n\paren{n+1}}{2}.
\end{aligned}\]

Montrons que \(\M{n}=\sym{n}\oplus\antisym{n}\).

On note \(u\) la transposition : \(\fonction{u}{\M{n}}{\M{n}}{M}{\trans{M}}\).

On a \(\begin{dcases}
u\in\Lendo{\M{n}} \\
u^2=\id{\M{n}}
\end{dcases}\) donc \(u\) est une symétrie de \(\M{n}\).

Donc \[\M{n}=\underbrace{\ker\paren{u-\id{\M{n}}}}_{=\sym{n}}\oplus\underbrace{\ker\paren{u+\id{\M{n}}}}_{=\antisym{n}}.\]
\end{dem}

\section{Lignes et colonnes d'une matrice}

\subsection{Abus autorisés}

\begin{abus}[Matrice-colonne \(=\) vecteur]
Soit \(n\in\Ns\).

L'application \[\fonctionlambda{\K^n}{\M{n1}}{\paren{a_1,\dots,a_n}}{\begin{pmatrix}
a_1 \\
\vdots \\
a_n
\end{pmatrix}}\] est un isomorphisme d'espaces vectoriels.

On s'autorise (abusivement) à identifier les espaces vectoriels \(\K^n\) et \(\M{n1}\) : \[\K^n=\M{n1}.\]

On identifie donc les \(n\)-uplets de scalaires et les matrices-colonne : \[\quantifs{\forall a_1,\dots,a_n}\paren{a_1,\dots,a_n}=\begin{pmatrix}
a_1 \\
\vdots \\
a_n
\end{pmatrix}.\]

En particulier, si \(n=1\), on identifie matrice de taille \(\paren{1,1}\) et scalaire : \[\K=\M{1}\qquad\text{et}\qquad\quantifs{\forall\lambda\in\K}\lambda=\begin{pmatrix}
\lambda
\end{pmatrix}.\]
\end{abus}

\begin{defi}
Soient \(n,p\in\Ns\) et \(A=\paren{a_{ij}}_{\paren{i,j}}\in\M{np}\).

On appelle famille des vecteurs-colonne de \(A\) la famille \(\paren{C_1,\dots,C_p}\in\paren{\K^n}^p\) définie par : \[\quantifs{\forall j\in\interventierii{1}{p}}C_j=\begin{pmatrix}
a_{1j} \\
\vdots \\
a_{nj}
\end{pmatrix}\in\K^n.\]

On s'autorise à noter la matrice \(A\) \guillemets{par colonnes} : \[A=\begin{pmatrix}
a_{11} & \dots & a_{1p} \\
\vdots &  & \vdots \\
a_{n1} & \dots & a_{np}
\end{pmatrix}=\begin{pmatrix}
C_1 & \dots & C_p
\end{pmatrix}.\]
\end{defi}

\begin{rappel}
Soit \(n\in\Ns\).

Les formes linéaires sur \(\K^n\) sont les fonctions de la forme \[\fonction{l_{b_1,\dots,b_n}}{\K^n}{\K}{\paren{a_1,\dots,a_n}}{b_1a_1+\dots+b_na_n}\] où \(b_1,\dots,b_n\in\K\).

De plus, la fonction \[\fonctionlambda{\K^n}{\paren{\K^n}\etoile}{\paren{b_1,\dots,b_n}}{l_{b_1,\dots,b_n}}\] est un isomorphisme.
\end{rappel}

\begin{abus}[Matrice-ligne \(=\) forme linéaire]
Soit \(n\in\Ns\).

L'application \[\fonctionlambda{\M{1n}}{\paren{\K^n}\etoile}{L}{\croch{C\mapsto LC}}\] \cad \[\fonctionlambda{\M{1n}}{\L{\K^n}{\K}}{\begin{pmatrix}
b_1 & \dots & b_n
\end{pmatrix}}{\croch{\begin{pmatrix}
a_1 \\
\vdots \\
a_n
\end{pmatrix}\mapsto\begin{pmatrix}
b_1 & \dots & b_n
\end{pmatrix}\begin{pmatrix}
a_1 \\
\vdots \\
a_n
\end{pmatrix}=b_1a_1+\dots+b_na_n}}\] est un isomorphisme d'espaces vectoriels.

On s'autorise (abusivement) à identifier les espaces vectoriels \(\paren{\K^n}\etoile\) et \(\M{1n}\) : \[\paren{\K^n}\etoile=\M{1n}.\]

On identifie donc les formes linéaires sur \(\K^n\) et les matrices-ligne : \[\quantifs{\forall b_1,\dots,b_n\in\K}l_{b_1,\dots,b_n}=\begin{pmatrix}
b_1 & \dots & b_n
\end{pmatrix}.\]
\end{abus}

\begin{defi}
Soient \(n,p\in\Ns\) et \(A=\paren{a_{ij}}_{\paren{i,j}}\in\M{np}\).

On appelle famille des lignes de \(A\) la famille \(\paren{L_1,\dots,L_n}\in\M{1p}^n\) définie par : \[\quantifs{\forall i\in\interventierii{1}{n}}L_i=\begin{pmatrix}
a_{i1} & \dots & a_{ip}
\end{pmatrix}\in\M{1p}.\]

On s'autorise à noter la matrice \(A\) \guillemets{par lignes} : \[A=\begin{pmatrix}
a_{11} & \dots & a_{1p} \\
\vdots &  & \vdots \\
a_{n1} & \dots & a_{np}
\end{pmatrix}=\begin{pmatrix}
L_1 \\
\vdots \\
L_n
\end{pmatrix}.\]
\end{defi}

\subsection{Produits matriciels}

\begin{rem}[Produit avec une matrice-colonne ou une matrice-ligne]\thlabel{rem:produitAvecUneMatriceColonneOuUneMatriceLigne}
Soient \(n,p\in\Ns\) et \(M=\paren{m_{ij}}_{\paren{i,j}}\in\M{np}\).

On note \(\paren{L_1,\dots,L_n}\in\M{1p}^n\) la famille des lignes de \(M\) et \(\paren{C_1,\dots,C_p}\in\paren{\K^n}^p\) la famille des colonnes de \(M\).

On peut donc écrire : \[M=\begin{pmatrix}
m_{11} & \dots & m_{1p} \\
\vdots &  & \vdots \\
m_{n1} & \dots & m_{np}
\end{pmatrix}=\begin{pmatrix}
C_1 & \dots & C_p
\end{pmatrix}=\begin{pmatrix}
L_1 \\
\vdots \\
L_n
\end{pmatrix}.\]

Soient enfin une matrice-ligne \(L=\begin{pmatrix}
b_1 & \dots & b_n
\end{pmatrix}\) et une matrice-colonne \(C=\begin{pmatrix}
a_1 \\
\vdots \\
a_p
\end{pmatrix}\).

Alors on a : \[LM=\begin{pmatrix}
b_1 & \dots & b_n
\end{pmatrix}\begin{pmatrix}
L_1 \\
\vdots \\
L_n
\end{pmatrix}=b_1L_1+\dots+b_nL_n\] et : \[MC=\begin{pmatrix}
C_1 & \dots & C_p
\end{pmatrix}\begin{pmatrix}
a_1 \\
\vdots \\
a_n
\end{pmatrix}=a_1C_1+\dots+a_pC_p.\]
\end{rem}

\begin{ex}
On a : \[\begin{pmatrix}
1 & 2 & 3 & 4 \\
5 & 6 & 7 & 8 \\
9 & 10 & 11 & 12
\end{pmatrix}\begin{pmatrix}
1 \\
0 \\
-1 \\
0
\end{pmatrix}=\begin{pmatrix}
-2 \\
-2 \\
-2
\end{pmatrix}.\]
\end{ex}

\begin{cor}
Soient \(n,p\in\Ns\) et \(M=\paren{m_{ij}}_{\paren{i,j}}\in\M{np}\).

On note \(\paren{L_1,\dots,L_n}\in\M{1p}^n\) la famille des lignes de \(M\) et \(\paren{C_1,\dots,C_p}\in\paren{\K^n}^p\) la famille des colonnes de \(M\) : \[M=\begin{pmatrix}
m_{11} & \dots & m_{1p} \\
\vdots &  & \vdots \\
m_{n1} & \dots & m_{np}
\end{pmatrix}=\begin{pmatrix}
C_1 & \dots & C_p
\end{pmatrix}=\begin{pmatrix}
L_1 \\
\vdots \\
L_n
\end{pmatrix}.\]

De plus, on note \(\paren{e_1,\dots,e_p}\) la base canonique de \(\K^p\) et \(\paren{e_1\prim,\dots,e_n\prim}\) la base canonique de \(\K^n\).

Alors on a : \[\quantifs{\forall j\in\interventierii{1}{p}}M\times e_j=C_j\] et : \[\quantifs{\forall i\in\interventierii{1}{n}}\trans{e_i\prim}\times M=L_i.\]
\end{cor}

\begin{rem}\thlabel{rem:produitMatricielLignesADroiteOuColonnesAGauche}
Soient \(m,n,p\in\Ns\), \(B=\paren{b_{ij}}_{\paren{i,j}}\in\M{mn}\) et \(A=\paren{a_{jk}}_{\paren{j,k}}\in\M{np}\).

On note \(\paren{L_1,\dots,L_m}\in\M{1n}^m\) la famille des lignes de \(B\) et \(\paren{C_1,\dots,C_p}\in\paren{\K^n}^p\) la famille des colonnes de \(A\) : \[B=\begin{pmatrix}
b_{11} & \dots & b_{1n} \\
\vdots &  & \vdots \\
b_{m1} & \dots & b_{mn}
\end{pmatrix}=\begin{pmatrix}
L_1 \\
\vdots \\
L_m
\end{pmatrix}\qquad\text{et}\qquad A=\begin{pmatrix}
a_{11} & \dots & a_{1p} \\
\vdots &  & \vdots \\
a_{n1} & \dots & a_{np}
\end{pmatrix}=\begin{pmatrix}
C_1 & \dots & C_p
\end{pmatrix}.\]

Alors on a, par définition du produit matriciel : \[BA=\paren{L_iC_k}_{\paren{i,k}}\in\M{mp}.\]

On a aussi : \[BA=\begin{pmatrix}
L_1 \\
\vdots \\
L_m
\end{pmatrix}A=\begin{pmatrix}
L_1A \\
\vdots \\
L_mA
\end{pmatrix}\] et : \[BA=B\begin{pmatrix}
C_1 & \dots & C_p
\end{pmatrix}=\begin{pmatrix}
BC_1 & \dots & BC_p
\end{pmatrix}.\]

Ainsi, calculer le produit matriciel \(BA\) revient à multiplier chaque ligne de \(B\) par \(A\) (à droite) ou chaque colonne de \(A\) par \(B\) (à gauche).
\end{rem}

\begin{ex}
En utilisant la \thref{rem:produitAvecUneMatriceColonneOuUneMatriceLigne} et la \thref{rem:produitMatricielLignesADroiteOuColonnesAGauche}, on calcule les produits matriciels suivants : \[\begin{pmatrix}
1 & 2 & 3 \\
4 & 5 & 6 \\
7 & 8 & 9
\end{pmatrix}\begin{pmatrix}
a & 0 & 0 \\
0 & b & 0 \\
0 & 0 & c
\end{pmatrix}=\begin{pmatrix}
a & 2b & 3c \\
4a & 5b & 6c \\
7a & 8b & 9c
\end{pmatrix}\] et \[\begin{pmatrix}
a & 0 & 0 \\
0 & b & 0 \\
0 & 0 & c
\end{pmatrix}\begin{pmatrix}
1 & 2 & 3 \\
4 & 5 & 6 \\
7 & 8 & 9
\end{pmatrix}=\begin{pmatrix}
a & 2a & 3a \\
4b & 5b & 6b \\
7c & 8c & 9c
\end{pmatrix}.\]
\end{ex}

\subsection{Opérations élémentaires sur les matrices}

\begin{defi}[Opérations élémentaires sur une matrice]
On appelle opérations élémentaires sur une matrice les opérations suivantes :

\begin{itemize}
\item Opérations élémentaires sur les lignes : \[\begin{aligned}
L_i\echange L_j &\qquad\text{échange des lignes }L_i\text{ et }L_j \\
L_i\gets\lambda L_i &\qquad\text{multiplication de la ligne }L_i\text{ par }\lambda\in\K\excluant\accol{0} \\
L_i\gets L_i+\lambda L_j &\qquad\text{ajout à }L_i\text{ de }\lambda L_j\text{ où }\lambda\in\K\excluant\accol{0}
\end{aligned}\]

\item Opérations élémentaires sur les colonnes : \[\begin{aligned}
C_i\echange C_j &\qquad\text{échange des colonnes }C_i\text{ et }C_j \\
C_i\gets\lambda C_i &\qquad\text{multiplication de la colonne }C_i\text{ par }\lambda\in\K\excluant\accol{0} \\
C_i\gets C_i+\lambda C_j &\qquad\text{ajout à }C_i\text{ de }\lambda C_j\text{ où }\lambda\in\K\excluant\accol{0}
\end{aligned}\]
\end{itemize}
\end{defi}

\begin{exoex}
Soit \(M\in\M{32}\).

Donner les produits matriciels ayant même effet que les opérations élémentaires suivantes : \[L_1\echange L_3\qquad L_2\gets\lambda L_2\qquad L_1\gets L_1+\lambda L_3\qquad C_1\echange C_2\qquad C_2\gets\lambda C_2\qquad C_2\gets C_2+\lambda C_1.\]
\end{exoex}

\begin{corr}~\\
\(L_1\echange L_3\) : \(\begin{pmatrix}
0 & 0 & 1 \\
0 & 1 & 0 \\
1 & 0 & 0
\end{pmatrix}M\).

\(L_2\gets\lambda L_2\) : \(\begin{pmatrix}
1 & 0 & 0 \\
0 & \lambda & 0 \\
0 & 0 & 1
\end{pmatrix}M\).

\(L_1\gets L_1+\lambda L_3\) : \(\begin{pmatrix}
1 & 0 & \lambda \\
0 & 1 & 0 \\
0 & 0 & 1
\end{pmatrix}M\).

\(C_1\echange C_2\) : \(M\begin{pmatrix}
0 & 1 \\
1 & 0
\end{pmatrix}\).

\(C_2\gets\lambda C_2\) : \(M\begin{pmatrix}
1 & 0 \\
0 & \lambda
\end{pmatrix}\).

\(C_2\gets C_2+\lambda C_1\) : \(M\begin{pmatrix}
1 & \lambda \\
0 & 1
\end{pmatrix}\).
\end{corr}

\begin{bilan}\thlabel{bilan:opérationsÉlémentairesÉquivalentesÀDesMultiplications}
Appliquer une opération élémentaire aux lignes d'une matrice \(M\) revient à multiplier \(M\) à gauche par une matrice \(G\) qui ne dépend que de l'opération élémentaire et du nombre de lignes de \(M\).

Appliquer une opération élémentaire aux colonnes d'une matrice \(M\) revient à multiplier \(M\) à droite par une matrice \(D\) qui ne dépend que de l'opération élémentaire et du nombre de colonnes de \(M\).

Les matrices \(G\) et \(D\) sont faciles à déterminer : il suffit de prendre comme matrice \(M\) une matrice identité.
\end{bilan}

\section{Rang d'une matrice}

\begin{defi}
Soient \(n,p\in\Ns\).

Le rang d'une matrice \(A\in\M{np}\) est le rang de la famille de ses vecteurs-colonne (famille de \(p\) vecteurs de \(\K^n\)).
\end{defi}

\begin{rem}
Soient \(n,p\in\Ns\) et \(A\in\M{np}\).

On note \(\paren{C_1,\dots,C_p}\) la famille des vecteurs-colonne de \(A\).

On a : \[0\leq\rg A\leq\min\accol{n;p}.\]

Cas d'égalité :

\begin{enumerate}
\item \(\rg A=0\ssi A=0_{np}\) ; \\

\item \(\rg A=n\ssi\paren{C_1,\dots,C_p}\) est génératrice de \(\K^n\) ; \\

\item \(\rg A=p\ssi\paren{C_1,\dots,C_p}\) est libre.
\end{enumerate}
\end{rem}

\begin{dem}[1]
On a : \[\begin{aligned}
A=0&\ssi\quantifs{\forall j\in\interventierii{1}{p}}C_j=0 \\
&\ssi\rg\paren{C_1,\dots,C_p}=0.
\end{aligned}\]
\end{dem}

\begin{dem}[2]
On a : \[\begin{aligned}
\rg A=n&\ssi\rg\paren{C_1,\dots,C_p}=n \\
&\ssi\dim\Vect{C_1,\dots,C_p}=n \\
&\ssi\Vect{C_1,\dots,C_p}=\K^n \\
&\ssi\paren{C_1,\dots,C_p}\text{ est génératrice de }\K^n.
\end{aligned}\]
\end{dem}

\begin{dem}[3]
On a : \[\begin{aligned}
\rg A=p&\ssi\rg\paren{C_1,\dots,C_p}=p \\
&\ssi\paren{C_1,\dots,C_p}\text{ est libre}.
\end{aligned}\]
\end{dem}

\section{Application linéaire canoniquement associée à une matrice}

\subsection{Définition}

\begin{defi}[Application linéaire canoniquement associée à une matrice]
Soient \(n,p\in\Ns\) et \(A\in\M{np}\).

On appelle application linéaire canoniquement associée à \(A\) le \guillemets{produit à gauche par \(A\)} : \[\fonction{u_A}{\K^p}{\K^n}{X}{AX}\]

C'est l'unique application linéaire \(u_A\in\L{\K^p}{\K^n}\) telle que : \[\quantifs{\forall i\in\interventierii{1}{p}}u_A\paren{e_i}=C_i\] en notant \(e_1,\dots,e_p\) les vecteurs de la base canonique de \(\K^p\) et \(\paren{C_1,\dots,C_p}\) la famille des vecteurs-colonne de \(A\).
\end{defi}

\begin{defi}[Endomorphisme canoniquement associé à une matrice]
Soient \(n\in\Ns\) et \(A\in\M{n}\).

On appelle endomorphisme de \(\K^n\) canoniquement associé à \(A\) le \guillemets{produit à gauche par \(A\)} : \[\fonction{u_A}{\K^n}{\K^n}{X}{AX}\]

C'est l'unique endomorphisme \(u_A\in\Lendo{\K^n}\) tel que : \[\quantifs{\forall i\in\interventierii{1}{n}}u_A\paren{e_i}=C_i\] en notant \(e_1,\dots,e_n\) les vecteurs de la base canonique de \(\K^n\) et \(\paren{C_1,\dots,C_n}\) la famille des vecteurs-colonne de \(A\).
\end{defi}

\begin{exoex}
Que dire des endomorphismes canoniquement associés aux matrices suivantes (savoir lire ces matrices colonne par colonne et ligne par ligne) : \[M_1=\begin{pmatrix}
1 & 0 & 0 \\
0 & 1 & 0 \\
0 & 0 & 0
\end{pmatrix}\qquad\text{et}\qquad M_2=\begin{pmatrix}
0 & -1 \\
1 & 0
\end{pmatrix}.\]
\end{exoex}

\begin{corr}
On a : \[\quantifs{\forall\tcoords{x}{y}{z}\in\R^3}u_{M_1}\paren{\tcoords{x}{y}{z}}=\begin{pmatrix}
1 & 0 & 0 \\
0 & 1 & 0 \\
0 & 0 & 0
\end{pmatrix}\tcoords{x}{y}{z}=\tcoords{x}{y}{0}\] et \[\begin{dcases}
u_{M_1}\paren{\tcoords{1}{0}{0}}=\tcoords{1}{0}{0} \\
u_{M_1}\paren{\tcoords{0}{1}{0}}=\tcoords{0}{1}{0} \\
u_{M_1}\paren{\tcoords{0}{0}{1}}=\tcoords{0}{0}{0}
\end{dcases}\]

On reconnaît le projecteur sur \(\Vect{\tcoords{1}{0}{0},\tcoords{0}{1}{0}}\) parallèlement à \(\Vect{\tcoords{0}{0}{1}}\).

De même, on a : \[\quantifs{\forall\dcoords{x}{y}\in\R^2}u_{M_2}\paren{\dcoords{x}{y}}=\begin{pmatrix}
0 & -1 \\
1 & 0
\end{pmatrix}\dcoords{x}{y}=\dcoords{-y}{x}\] et \[\begin{dcases}
u_{M_2}\paren{\dcoords{1}{0}}=\dcoords{0}{1} \\
u_{M_2}\paren{\dcoords{0}{1}}=\dcoords{-1}{0}
\end{dcases}\]
\end{corr}

\begin{rem}\thlabel{rem:liensEntreMatriceEtApplicationLinéaireCanoniquementAssociée}
Soient \(n,p\in\Ns\) et \(A\in\M{np}\).

On note \(u_A\in\L{\K^p}{\K^n}\) l'application linéaire canoniquement associée à \(A\), \(\paren{L_1,\dots,L_n}\in\M{1p}^n\) la famille des lignes de \(A\) et \(\paren{C_1,\dots,C_p}\in\paren{\K^n}^p\) la famille des colonnes de \(A\) : \[A=\begin{pmatrix}
C_1 & \dots & C_p
\end{pmatrix}=\tcoords{L_1}{\vdots}{L_n}.\]

Alors :

\begin{enumerate}
\item L'ensemble image de \(u_A\) est le sous-espace vectoriel engendré par les colonnes de \(A\), vues comme des vecteurs de \(\K^n\) : \[\Im u_A=\Vect{C_1,\dots,C_p}.\]

\item Le noyau de \(u_A\) est l'intersection des noyaux des lignes de \(A\), vues comme des formes linéaires : \[\ker u_A=\accol{X\in\K^p\tq L_1X=\dots=L_nX=0_{n1}}.\]

\item On a : \[\begin{aligned}
u_A\text{ est injective}&\ssi\paren{C_1,\dots,C_p}\text{ est libre} \\
&\ssi\rg A=p.
\end{aligned}\]

\item On a : \[\begin{aligned}
u_A\text{ est surjective}&\ssi\paren{C_1,\dots,C_p}\text{ est génératrice de }\K^n \\
&\ssi\rg A=n.
\end{aligned}\]

\item On a : \[\begin{aligned}
u_A\text{ est bijective}&\ssi\begin{dcases}
n=p \\
\paren{C_1,\dots,C_p}\text{ est une base de }\K^n
\end{dcases} \\
&\ssi\rg A=n=p.
\end{aligned}\]

\item On a : \[\rg A=\rg u_A.\]
\end{enumerate}
\end{rem}

\begin{dem}[1]
\(\Im u_A\) est l'ensemble des vecteurs de \(\K^n\) de la forme \(u_A\paren{\tcoords{x_1}{\vdots}{x_p}}=A\tcoords{x_1}{\vdots}{x_p}=x_1C_1+\dots+x_pC_p\) où \(x_1,\dots,x_p\in\K\).

D'où \(\Im u_A=\Vect{C_1,\dots,C_p}\).
\end{dem}

\begin{dem}[2]~\\
Soit \(X=\tcoords{x_1}{\vdots}{x_p}\in\K^p\).

On a : \[\begin{aligned}
X\in\ker u_A&\ssi u_A\paren{X}=0_{\K^n} \\
&\ssi AX=0_{n1} \\
&\ssi\quantifs{\forall i\in\interventierii{1}{n}}L_iX=0.
\end{aligned}\]
\end{dem}

\begin{dem}
On a : \[\begin{aligned}
u_A\text{ est injective}&\ssi\ker u_A=\accol{0} \\
&\ssi\croch{\quantifs{\forall X\in\K^p}AX=0\ssi X=0} \\
&\ssi\croch{\quantifs{\forall x_1,\dots,x_p\in\K}x_1C_1+\dots+x_pC_p=0\imp x_1=\dots=x_p=0} \\
&\ssi\paren{C_1,\dots,C_p}\text{ est libre}.
\end{aligned}\]
\end{dem}

\begin{dem}[4]
On a : \[\begin{aligned}
u_A\text{ est surjective}&\ssi\Im u_A=\K^n \\
&\ssi\Vect{C_1,\dots,C_p}=\K^n \\
&\ssi\paren{C_1,\dots,C_p}\text{ est génératrice de }\K^n.
\end{aligned}\]
\end{dem}

\begin{dem}[5]
Découle de (3) et (4).
\end{dem}

\begin{dem}[6]
On a : \[\begin{aligned}
\rg u_A&=\dim\Im u_A \\
&=\dim\Vect{C_1,\dots,C_p} \\
&=\rg\paren{C_1,\dots,C_p} \\
&=\rg A.
\end{aligned}\]
\end{dem}

\subsection{Propriétés}

Dans ce paragraphe, on note \(u_A\in\L{\K^p}{\K^n}\) l'application linéaire canoniquement associée à la matrice \(A\in\M{np}\).

\begin{prop}[Matrices quelconques]\thlabel{prop:applicationQuiAssocieAUneMatriceSonApplicationLinéaireCanoniquementAssociéeEstUnIsomorphismeD'EspacesVectoriels}
Soient \(n,p\in\Ns\).

L'application \[\fonction{\phi}{\M{np}}{\L{\K^p}{\K^n}}{A}{u_A}\] est un isomorphisme d'espaces vectoriels.
\end{prop}

\begin{dem}
Montrons que \(\phi\) est linéaire.

Soient \(\lambda,\mu\in\K\) et \(A,B\in\M{np}\).

Montrons que \(u_{\lambda A+\mu B}=\lambda u_A+\mu u_B\).

On a : \[\begin{WithArrows}
\quantifs{\forall X\in\K^p}u_{\lambda A+\mu B}\paren{X}&=\paren{\lambda A+\mu B}X \Arrow[tikz={text width=4cm}]{par bilinéarité du produit matriciel} \\
&=\lambda AX+\mu BX \\
&=\lambda u_A\paren{X}+\mu u_B\paren{X}.
\end{WithArrows}\]

Donc \(\phi\) est linéaire.

Montrons que \(\phi\) est injective.

Soit \(A\in\ker\phi\).

On a \(\phi\paren{A}=u_A=0\).

Donc les colonnes \(C_1,\dots,C_p\) de \(A\) sont nulles car \(\quantifs{\forall j\in\interventierii{1}{p}}C_j=u_A\paren{e_j}\) en notant \(\paren{e_1,\dots,e_p}\) la base canonique de \(\K^p\).

Donc \(A=0\).

Donc \(\ker\phi=\accol{0_{np}}\).

Donc \(\phi\) est injective.

Finalement, on a \(\dim\M{np}=\dim\L{\K^p}{\K^n}=np<\pinf\).

Donc \(\phi\) est un isomorphisme d'espaces vectoriels.
\end{dem}

\begin{prop}\thlabel{prop:applicationLinéaireCanoniquementAssociéeAUnProduitDeMatriceEgaleALaComposéeDesApplicationsLinéairesCanoniquementAssociéesAuxMatrices}
Soient \(m,n,p\in\Ns\), \(A\in\M{mn}\) et \(B\in\M{np}\).

On a : \[u_{AB}=u_A\rond u_B.\]
\end{prop}

\begin{dem}
On a \(u_A\rond u_B,u_{AB}\in\L{\K^p}{\K^m}\) et \[\begin{aligned}
\quantifs{\forall X\in\K^p}u_A\rond u_B\paren{X}&=u_A\paren{u_B\paren{X}} \\
&=A\paren{BX} \\
&=\paren{AB}X \\
&=u_{AB}\paren{X}.
\end{aligned}\]
\end{dem}

\begin{prop}[Matrices carrées]
Soit \(n\in\Ns\).

L'application \[\fonction{\phi}{\M{n}}{\Lendo{\K^n}}{A}{u_A}\] est un isomorphisme d'anneaux de \(\anneau{\M{n}}\) vers \(\anneau{\Lendo{\K^n}}[+][\rond]\)
\end{prop}

\begin{dem}
Montrons que \(\phi\) est un morphisme d'anneaux.

On a \(\phi\paren{I_n}=\id{\K^n}\). En effet : \(\quantifs{\forall X\in\K^n}u_{I_n}\paren{X}=I_nX=X\).

De plus, selon la \thref{prop:applicationQuiAssocieAUneMatriceSonApplicationLinéaireCanoniquementAssociéeEstUnIsomorphismeD'EspacesVectoriels}, on a : \[\quantifs{\forall A,B\in\M{n}}\begin{dcases}
\phi\paren{A+B}=\phi\paren{A}+\phi\paren{B} \\
\phi\paren{AB}=\phi\paren{A}\rond\phi\paren{B}
\end{dcases}\]

Donc \(\phi\) est un morphisme d'anneaux.

De plus, d'après la même proposition, \(\phi\) est une bijection.

Donc \(\phi\) est un isomorphisme d'anneaux.
\end{dem}

\begin{prop}
Soient \(n\in\Ns\) et \(A\in\M{n}\).

La matrice \(A\) est inversible dans l'anneau \(\anneau{\M{n}}\) si, et seulement si, l'endomorphisme \(u_A\) est inversible dans l'anneau \(\anneau{\Lendo{\K^n}}[+][\rond]\) : \[A\in\GL{n}\ssi u_A\in\GL{}[\K^n].\]
\end{prop}

\begin{dem}
Cela découle de la proposition précédente car un isomorphisme d'anneaux conserve les éléments inversibles.
\end{dem}

\begin{prop}\thlabel{prop:rangInchangéQuandOnMultiplieLaMatriceParUneMatriceInversible}
Soient \(n,p\in\Ns\) et \(A\in\M{np}\).

On ne change pas le rang de \(A\) en la multipliant à gauche ou à droite par une matrice inversible : \[\quantifs{\forall P\in\GL{n};\forall Q\in\GL{p}}\rg PAQ=\rg A.\]
\end{prop}

\begin{dem}
On sait qu'on ne change pas le rang d'une application linéaire quand on la compose à gauche ou à droite par un isomorphisme.

Donc si \(P\in\GL{n}\) et \(Q\in\GL{p}\) alors \[\rg u_P\rond u_A\rond u_Q=\rg u_A\] car \(u_P\) et \(u_Q\) sont des isomorphismes selon la proposition précédente.

Donc selon la \thref{prop:applicationLinéaireCanoniquementAssociéeAUnProduitDeMatriceEgaleALaComposéeDesApplicationsLinéairesCanoniquementAssociéesAuxMatrices} : \(\rg u_{PAQ}=\rg u_A\).

Donc selon la \thref{rem:liensEntreMatriceEtApplicationLinéaireCanoniquementAssociée} : \[\rg PAQ=\rg A.\]
\end{dem}

\subsection{Conséquences sur l'inversibilité des matrices}

\begin{bilan}\thlabel{bilan:matriceInversible}
Soient \(n\in\Ns\) et \(A\in\M{n}\).

On note \(u_A\in\Lendo{\K^n}\) l'endomorphisme canoniquement associé à \(A\), \(\paren{C_1,\dots,C_n}\) la famille des vecteurs-colonne de \(A\) et \(\paren{L_1,\dots,L_n}\) la famille des lignes de \(A\).

Les propositions suivantes sont équivalentes :

\begin{enumerate}
\item \(A\) est une matrice inversible \\

\item \(A\) est \guillemets{inversible à droite} dans l'anneau \(\anneau{\M{n}}\) : \[\quantifs{\exists B\in\M{n}}AB=I_n\]

\item \(A\) est \guillemets{inversible à gauche} dans l'anneau \(\anneau{\M{n}}\) : \[\quantifs{\exists B\in\M{n}}BA=I_n\]

\item \(u_A:\K^n\to\K^n\) est un automorphisme de \(\K^n\) \\

\item \(u_A:\K^n\to\K^n\) est une injection \\

\item \(u_A:\K^n\to\K^n\) est une surjection \\

\item \(\rg u_A=n\) \\

\item \(\rg A=n\) \\

\item \(\paren{C_1,\dots,C_n}\) est une famille génératrice de \(\K^n\) \\

\item \(\paren{C_1,\dots,C_n}\) est une famille libre \\

\item \(\paren{C_1,\dots,C_n}\) est une base de \(\K^n\) \\

\item \(\trans{A}\) est une matrice inversible \\

\item \(\paren{L_1,\dots,L_n}\) est une famille génératrice de \(\M{1n}\) \\

\item \(\paren{L_1,\dots,L_n}\) est une famille libre \\

\item \(\paren{L_1,\dots,L_n}\) est une base de \(\M{1n}\).
\end{enumerate}
\end{bilan}

\subsection{Remarque}

\begin{abus}
Soient \(n,p\in\Ns\).

Le programme de MP2I autorise à confondre une matrice \(A\in\M{np}\) et l'application linéaire qui lui est canoniquement associée.

On peut donc écrire, par exemple : \guillemets{\(\ker A\)} ou \guillemets{\(\Im A\)}.

Nous ne ferons jamais cet abus (sauf parfois, dans le cas où \(A\) est une matrice-ligne) car il rend les choses un peu confuses.
\end{abus}

\section{Matrices équivalentes}

\begin{defi}
Soient \(n,p\in\Ns\) et \(A,B\in\M{np}\).

On dit que la matrice \(B\) est équivalente à la matrice \(A\) si on a : \[\quantifs{\exists P\in\GL{n};\exists Q\in\GL{p}}B=PAQ.\]
\end{defi}

\begin{prop}
Soit \(n,p\in\Ns\).

La relation \guillemets{être équivalente à} est une relation d'équivalence sur \(\M{np}\).
\end{prop}

\begin{dem}
Notons \(\sim\) la relation \guillemets{être équivalente à} : \[\quantifs{\forall A,B\in\M{np}}B\sim A\ssi\croch{\quantifs{\exists P\in\GL{n};\exists Q\in\GL{p}}B=PAQ}.\]

On a \(\quantifs{\forall A\in\M{np}}A=I_nAI_p\) donc \(\quantifs{\forall A\in\M{np}}A\sim A\) donc \(\sim\) est réflexive.

Soient \(A,B\in\M{np}\) telles que \(B\sim A\).

Il existe \(P\in\GL{n}\) et \(Q\in\GL{p}\) telles que \(B=PAQ\).

D'où \(P\inv BQ\inv=P\inv PAQQ\inv\) donc \(P\inv BQ\inv=A\) avec \(\begin{dcases}
P\inv\in\GL{n} \\
Q\inv\in\GL{p}
\end{dcases}\)

D'où \(A\sim B\) donc \(\sim\) est symétrique.

Soient \(A,B,C\in\M{np}\) telles que \(C\sim B\) et \(B\sim A\).

Il existe \(P_1,P_2\in\GL{n}\) et \(Q_1,Q_2\in\GL{p}\) telles que \(C=P_1BQ_1\) et \(B=P_2AQ_2\).

Donc \(C=P_1P_2AQ_2Q_1\) avec \(\begin{dcases}
P_1P_2\in\GL{n} \\
Q_2Q_1\in\GL{p}
\end{dcases}\)

Donc \(C\sim A\) donc \(\sim\) est transitive.

Finalement, \(\sim\) est une relation d'équivalence.
\end{dem}

\begin{exoex}~\\
Montrer que les matrices \(\begin{pmatrix}
1 & 1 \\
0 & 1
\end{pmatrix}\) et \(\begin{pmatrix}
1 & 0 \\
0 & 2
\end{pmatrix}\) sont équivalentes.
\end{exoex}

\begin{corr}~\\
On remarque \(\begin{pmatrix}
1 & -1 \\
0 & 1
\end{pmatrix}\begin{pmatrix}
1 & 1 \\
0 & 1
\end{pmatrix}=\begin{pmatrix}
1 & 0 \\
0 & 1
\end{pmatrix}\) donc : \[\underbrace{\begin{pmatrix}
1 & -1 \\
0 & 1
\end{pmatrix}}_{\in\GL{2}}\begin{pmatrix}
1 & 1 \\
0 & 1
\end{pmatrix}\underbrace{\begin{pmatrix}
1 & 0 \\
0 & 2
\end{pmatrix}}_{\in\GL{2}}=\begin{pmatrix}
1 & 0 \\
0 & 2
\end{pmatrix}.\]
\end{corr}

\begin{rem}
Soient \(n\in\Ns\) et \(A,B\in\M{n}\).

On suppose que \(A\) et \(B\) sont équivalentes.

Alors \(A\) est inversible si, et seulement si, \(B\) est inversible.

De plus, \(\rg A=\rg B\).
\end{rem}

\begin{prop}\thlabel{prop:existenceDePetQPourQueTouteMatriceSoitEquivalenteAJr}
Soient \(n,p\in\Ns\), \(A\in\M{np}\) et \(r\in\N\).

Alors : \[\rg A=r\ssi\quantifs{\exists P\in\GL{n};\exists Q\in\GL{p}}A=PJ_rQ,\] en posant \[J_r=\begin{pNiceMatrix}[first-col,last-row]
1 & 1 & 0 & \dots & 0 & \Block{4-4}<\Huge>{0} \\
& 0 & \ddots & \ddots & \vdots & & & & \\
& \vdots & \ddots & \ddots & 0 & & & & \\
r & 0 & \dots & 0 & 1 & & & & \\
& \Block{4-4}<\Huge>{0} & & & & \Block{4-4}<\Huge>{0} \\
&&&&&&&&& \\
&&&&&&&&& \\
&&&&&&&&& \\
n &&&&&&&&& \\
& 1 &  &  & r & & & & p
\end{pNiceMatrix}.\]
\end{prop}

\begin{dem}\thlabel{dem:matriceDeRangrÉquivalenteÀJrAMoitiéAdmis}
\impdir Admise provisoirement. On verra dans le chapitre suivant que cette proposition découle naturellement de la forme géométrique du théorème du rang.

\imprec Découle de la \thref{prop:rangInchangéQuandOnMultiplieLaMatriceParUneMatriceInversible}.
\end{dem}

\begin{cor}
Soient \(n,p\in\Ns\) et \(A,B\in\M{np}\).

Alors : \[A\text{ et }B\text{ sont équivalentes}\ssi\rg A=\rg B.\]
\end{cor}

\begin{dem}
\impdir Claire car on ne change pas le rang d'une matrice en la multipliant à gauche et à droite par des matrices inversibles.

\imprec

Supposons \(\rg A=\rg B\).

Selon la proposition précédente, \(A\) et \(B\) sont équivalentes à \(J_r\) en posant \(r=\rg A\).

Donc \(A\) est équivalente à \(B\).
\end{dem}

\section{Matrices semblables}

\begin{defi}
Soient \(n\in\Ns\) et \(A,B\in\M{n}\).

On dit que la matrice \(B\) est semblable à la matrice \(A\) si : \[\quantifs{\exists P\in\GL{n}}B=PAP\inv.\]
\end{defi}

\begin{prop}
Soit \(n\in\Ns\).

La relation \guillemets{être semblable à} est une relation d'équivalence sur \(\M{n}\).
\end{prop}

\begin{dem}
On pose la relation binaire \(\rel\) sur \(\M{n}\) telle que : \[\quantifs{\forall A,B\in\M{n}}A\rel B\ssi\croch{\quantifs{\exists P\in\GL{n}}B=PAP\inv}.\]

Soient \(A,B,C\in\M{n}\).

Montrons que \(\rel\) est réflexive.

On a \(A=I_nAI_n\inv\) donc \(A\rel A\).

Montrons que \(\rel\) est symétrique.

Supposons \(A\rel B\).

Il existe \(P\in\GL{n}\) telle que \(B=PAP\inv\).

On a \(P\inv BP=P\inv PAP\inv P\) donc \(P\inv B\paren{P\inv}\inv=A\).

Donc \(B\rel A\).

Montrons que \(\rel\) est transitive.

Supposons \(A\rel B\) et \(B\rel C\).

Il existe \(P,Q\in\GL{n}\) telles que \(\begin{dcases}
B=PAP\inv \\
C=QBQ\inv
\end{dcases}\)

Donc \(C=QPAP\inv Q\inv=QPA\paren{QP}\inv\).

Donc \(A\rel C\).

Donc \(\rel\) est une relation d'équivalence sur \(\M{n}\).
\end{dem}

\begin{exoex}~\\
Montrer que les matrices \(\begin{pmatrix}
1 & 1 \\
0 & 1
\end{pmatrix}\) et \(\begin{pmatrix}
1 & 2 \\
0 & 1
\end{pmatrix}\) sont semblables.
\end{exoex}

\begin{corr}
L'idée est qu'on a : \[\begin{pmatrix}
\lambda & 0 \\
0 & 1
\end{pmatrix}\begin{pmatrix}
a & b \\
c & d
\end{pmatrix}=\begin{pmatrix}
\lambda a & \lambda b \\
c & d
\end{pmatrix}\qquad\text{et}\qquad\begin{pmatrix}
a & b \\
c & d
\end{pmatrix}\begin{pmatrix}
\lambda\inv & 0 \\
0 & 1
\end{pmatrix}=\begin{pmatrix}
a \lambda\inv & b \\
c \lambda\inv & d
\end{pmatrix}.\]

On remarque : \[\begin{pmatrix}
1 & 2 \\
0 & 1
\end{pmatrix}=\begin{pmatrix}
2 & 0 \\
0 & 1
\end{pmatrix}\begin{pmatrix}
1 & 1 \\
1 & 0
\end{pmatrix}\begin{pmatrix}
2 & 0 \\
0 & 1
\end{pmatrix}\inv.\]
\end{corr}

\begin{rem}
Soient \(n\in\Ns\) et \(P\in\GL{n}\).

L'application \[\fonctionlambda{\M{n}}{\M{n}}{M}{PMP\inv}\] est un automorphisme de \(\M{n}\) en tant qu'espace vectoriel et en tant qu'anneau (elle conserve les combinaisons linéaires et les produits).
\end{rem}

\begin{rem}
Soient \(n\in\Ns\), \(A,B\in\M{n}\) et \(P\in\GL{n}\) tels que \(B=PAP\inv\).

Alors on a : \[\quantifs{\forall k\in\N}B^k=PA^kP\inv.\]

De plus, si \(A\) est inversible, alors \(B\) aussi et on a : \[\quantifs{\forall k\in\Z}B^k=PA^kP\inv.\]
\end{rem}

\begin{rem}
Si deux matrices sont semblables, elles ont même rang.

On retrouve en particulier que l'une est inversible si, et seulement si, l'autre est inversible.
\end{rem}

\begin{dem}
On ne change pas le rang d'une matrice en la multipliant à gauche ou à droite par une matrice inversible (\cf \thref{prop:rangInchangéQuandOnMultiplieLaMatriceParUneMatriceInversible}).
\end{dem}

\section{Rang d'une matrice (suite)}

\begin{prop}
Soient \(n,p\in\Ns\) et \(A\in\M{np}\).

On note \(u_A\in\L{\K^p}{\K^n}\) l'application linéaire canoniquement associée à \(A\), \(\paren{C_1,\dots,C_p}\) la famille des vecteurs-colonne de \(A\) et \(\paren{L_1,\dots,L_n}\) la famille des lignes de \(A\).

Alors : \[\rg A=\rg u_A=\rg\paren{C_1,\dots,C_p}=\rg\paren{L_1,\dots,L_n}=\rg\paren{\trans{A}}.\]
\end{prop}

\begin{dem}
Il s'agit de montrer \(\rg A=\rg\paren{\trans{A}}\).

\underline{Méthode 1 :}

Posons \(r=\rg A\).

Selon la \thref{prop:existenceDePetQPourQueTouteMatriceSoitEquivalenteAJr}, il existe \(P\in\GL{n}\) et \(Q\in\GL{p}\) telles que \(A=PJ_rQ\) où \(J_r\in\M{np}\).

On a \(\trans{A}=\trans{Q}\trans{J_r}\trans{P}\).

Donc, comme \(\trans{Q}\) et \(\trans{P}\) sont inversibles, on a : \[\rg\paren{\trans{A}}=\rg\paren{\trans{J_r}}=r.\]

\underline{Méthode 2 :}

On a vu que \(\fonction{u_A}{\K^p}{\K^n}{x}{\paren{l_1\paren{x},\dots,l_n\paren{x}}}\) est de rang \(\rg u_A=\rg\paren{l_1,\dots,l_n}\) (en notant \(l_i\) la forme linéaire canoniquement associée à \(L_i\) pour tout \(i\in\interventierii{1}{n}\)).

Donc \[\rg A=\rg\paren{L_1,\dots,L_n}=\rg\paren{\trans{A}}.\]
\end{dem}

\begin{lem}
On ne change pas le rang d'une matrice en appliquant des opérations élémentaires à ses lignes ou à ses colonnes.
\end{lem}

\begin{dem}
Découle de la \thref{prop:rangInchangéQuandOnMultiplieLaMatriceParUneMatriceInversible} et du \thref{bilan:opérationsÉlémentairesÉquivalentesÀDesMultiplications}.
\end{dem}

\begin{algo}[Calcul du rang d'une matrice]
Pour calculer le rang d'une matrice, il suffit d'appliquer des opérations élémentaires à ses lignes et ses colonnes jusqu'à obtenir une matrice de rang évident.

NB : on peut opérer tantôt sur les lignes, tantôt sur les colonnes, ce qui rend le calcul facile et rapide.
\end{algo}

\begin{ex}
On a : \[\begin{aligned}
\rg\begin{pmatrix}
1 & 1 & 0 & 1 \\
1 & 2 & 1 & 0 \\
1 & 3 & 2 & 5
\end{pmatrix}&=\rg\begin{pNiceMatrix}[last-col]
1 & 1 & 0 & 1 &  \\
0 & 1 & 1 & -1 & L_2\gets L_2-L_1 \\
0 & 2 & 2 & 4 & L_3\gets L_3-L_1
\end{pNiceMatrix} \\
&=\rg\begin{pNiceMatrix}[last-col]
1 & 0 & 0 & 1 & C_2\gets C_2-C_1-C_3 \\
0 & 0 & 1 & -1 & \\
0 & 0 & 2 & 4 &
\CodeAfter
\begin{tikzpicture}
\draw[ultra thick,blue] (1-2.north) -- (3-2.south);
\end{tikzpicture}
\end{pNiceMatrix} \\
&=\rg\begin{pNiceMatrix}[last-col]
1 & 0 & 0 & C_3\gets C_3-C_1-2C_2 \\
0 & 1 & -3 & \\
0 & 2 & 0 &
\end{pNiceMatrix} \\
&=\rg\begin{pNiceMatrix}[last-col]
1 & 0 & 0 & \\
0 & 0 & -3 & L_2\gets L_2-\frac{1}{2}L_3 \\
0 & 2 & 0
\end{pNiceMatrix} \\
&=3
\end{aligned}\] et : \[\begin{WithArrows}
\rg\begin{pmatrix}
1 & 3 & 2 \\
2 & 2 & 2 \\
3 & 1 & 2
\end{pmatrix}&=\rg\begin{pNiceMatrix}[last-col]
1 & 3 & 2 & \\
0 & -4 & -2 & L_2\gets L_2-2L_1 \\
0 & -8 & -4 & L_3\gets L_3-3L_1
\end{pNiceMatrix} \\
&=\rg\begin{pNiceMatrix}[last-col]
1 & 3 & 2 & \\
0 & -4 & -2 & \\
0 & 0 & 0 & L_3\gets L_3-2L_2
\CodeAfter
\begin{tikzpicture}
\draw[ultra thick,blue] (3-1.west) -- (3-3.east);
\end{tikzpicture}
\end{pNiceMatrix} \\
&=\rg\begin{pNiceMatrix}[last-col]
1 & 3 & 2 & \\
0 & 2 & 1 & L_2\gets\frac{-1}{2}L_2
\end{pNiceMatrix}\Arrow[tikz={text width=4cm}]{car \(\paren{1,3,2}\) et \(\paren{0,2,1}\) ne sont pas colinéaires} \\
&=2.
\end{WithArrows}\]
\end{ex}

\begin{defi}[Matrice extraite]~\\
Soient \(n,p\in\Ns\) et \(A=\begin{pmatrix}
a_{11} & \dots & a_{1p} \\
\vdots &  & \vdots \\
a_{n1} & \dots & a_{np}
\end{pmatrix}\in\M{np}\).

On appelle matrice extraite de \(A\) toute matrice de la forme \[M=\begin{pmatrix}
a_{i_1j_1} & \dots & a_{i_1j_s} \\
\vdots &  & \vdots \\
a_{i_rj_1} & \dots & a_{i_rj_s}
\end{pmatrix}\in\M{rs}\qquad\text{où }\begin{dcases}
r\in\interventierii{1}{n} \\
s\in\interventierii{1}{p} \\
1\leq i_1<i_2<\dots<i_r\leq n \\
1\leq j_1<j_2<\dots<j_s\leq p
\end{dcases}\]

Concrètement, cela signifie qu'on obtient \(M\) à partir de \(A\) en supprimant quelques lignes et quelques colonnes.
\end{defi}

\begin{prop}
Le rang d'une matrice \(A\) est la taille de la plus grande matrice inversible extraite de \(A\).
\end{prop}

\begin{dem}
On note \(\paren{C_1,\dots,C_p}\) la famille des vecteurs-colonne de \(A\) et on pose \(r=\rg A\).

On a \(\rg A=\rg\paren{C_1,\dots,C_p}=\dim\Vect{C_1,\dots,C_p}\).

Selon le théorème de la base extraite, il existe \(j_1,\dots,j_r\in\N\) tels que \(1\leq j_1<\dots<j_r\leq p\) et \(\paren{C_{j_1},\dots,C_{j_r}}\) est une base de \(\Vect{C_1,\dots,C_p}\).

Posons \(B=\begin{pmatrix}
C_{j_1} & \dots & C_{j_r}
\end{pmatrix}\in\M{nr}\) (matrice extraite de \(A\)).

On note \(\paren{L_1,\dots,L_n}\) la famille des lignes de \(B\).

On a \(r=\rg B=\rg\paren{L_1,\dots,L_n}=\dim\Vect{L_1,\dots,L_n}\).

Selon le théorème de la base extraite, il existe \(i_1,\dots,i_r\in\N\) tels que \(1\leq i_1<\dots<i_r\leq n\) et \(\paren{L_{i_1},\dots,L_{i_r}}\) est une base de \(\Vect{L_1,\dots,L_n}\).

Posons \(C=\tcoords{L_{i_1}}{\vdots}{L_{i_r}}\in\M{r}\).

\(C\) est extraite de \(A\) et on a \(\rg C=r\) donc \(C\in\GL{r}\).

Ainsi, il existe une matrice inversible de taille \(r\) extraite de \(A\).

Enfin, toute matrice extraite de \(A\) est de rang inférieur à \(r\).

Donc il n'existe aucune matrice inversible de taille supérieure à \(r+1\) extraite de \(A\).
\end{dem}

\section{Trace d'une matrice carrée}

\begin{defi}
On appelle trace d'une matrice carrée la somme de ses coefficients diagonaux.

Ainsi, si \(n\in\Ns\) et \(A=\paren{a_{ij}}_{\paren{i,j}}\in\M{n}\), on appelle trace de \(A\) le scalaire : \[\tr A=\sum_{i=1}^na_{ii}.\]
\end{defi}

\begin{ex}
\begin{enumerate}
\item On a : \[\quantifs{\forall a,b,c,d\in\K}\tr\begin{pmatrix}
a & b \\
c & d
\end{pmatrix}=a+d.\]

\item La trace d'une matrice antisymétrique est toujours nulle.
\end{enumerate}
\end{ex}

\begin{rem}
Soit \(n\in\Ns\).

On a clairement : \[\quantifs{\forall A\in\M{n}}\tr\paren{\trans{A}}=\tr A.\]
\end{rem}

\begin{prop}
Soit \(n\in\Ns\).

L'application \(\tr:\M{n}\to\K\) est une forme linéaire sur \(\M{n}\).
\end{prop}

\begin{prop}
Soit \(n\in\Ns\).

On a : \[\quantifs{\forall A,B\in\M{n}}\tr\paren{AB}=\tr\paren{BA}.\]
\end{prop}

\begin{dem}
On pose \(C=\paren{c_{ik}}_{\paren{i,k}}=AB\) et \(D=\paren{d_{ik}}_{\paren{i,k}}=BA\).

Montrons que \(\tr C=\tr D\).

On a : \[\quantifs{\forall\paren{i,k}\in\interventierii{1}{n}^2}\begin{dcases}
c_{ik}=\sum_{j=1}^na_{ij}b_{jk} \\
d_{ik}=\sum_{j=1}^nb_{ij}a_{jk}
\end{dcases}\]

D'où : \[\begin{dcases}
\tr C=\sum_{i=1}^nc_{ii}=\sum_{i=1}^n\sum_{j=1}^na_{ij}b_{ji} \\
\tr D=\sum_{i=1}^nd_{ii}=\sum_{i=1}^n\sum_{j=1}^nb_{ij}a_{ji}
\end{dcases}\]

Donc \(\tr C=\tr D\) (changement d'indice \(\paren{i\prim,j}=\paren{j,i}\)).
\end{dem}

\begin{ex}~\\
On pose \(A=\begin{pmatrix}
a & b \\
c & d
\end{pmatrix}\) et \(B=\begin{pmatrix}
a\prim & b\prim \\
c\prim & d\prim
\end{pmatrix}\).

On a : \[\tr\paren{AB}=aa\prim+bc\prim+cb\prim+dd\prim\qquad\text{et}\qquad\tr\paren{\trans{A}A}=a^2+b^2+c^2+d^2.\]
\end{ex}

\begin{cor}
Soit \(n\in\Ns\).

Deux matrices semblables ont même trace : \[\quantifs{\forall A\in\M{n};\forall P\in\GL{n}}\tr\paren{PAP\inv}=\tr A.\]
\end{cor}

\begin{dem}
Soient \(A\in\M{n}\) et \(P\in\GL{n}\).

On a : \[\tr\paren{PAP\inv}=\tr\paren{APP\inv}=\tr A.\]
\end{dem}

\section{Matrices et systèmes linéaires}

Dans cette section, on utilise les notations suivantes :

Soient \(n,p\in\Ns\) et deux matrices : \[A=\begin{pmatrix}
a_{11} & \dots & a_{1p} \\
\vdots &  & \vdots \\
a_{n1} & \dots & a_{np}
\end{pmatrix}=\begin{pmatrix}
C_1 & \dots & C_p
\end{pmatrix}\in\M{np}\qquad\text{et}\qquad B=\tcoords{b_1}{\vdots}{b_n}\in\M{n1}\] (en notant \(\paren{C_1,\dots,C_p}\) la famille des vecteurs-colonne de \(A\)).

Le système linéaire de \(n\) équations à \(p\) inconnues \[\paren{S}~\begin{dcases}
a_{11}x_1+\dots+a_{1p}x_p=b_1 \\
\vdots \\
a_{n1}x_1+\dots+a_{np}x_p=b_n
\end{dcases}\] d'inconnue \(\paren{x_1,\dots,x_p}\in\K^p\) peut se réécrire \guillemets{matriciellement} ou \guillemets{vectoriellement}.

Écriture matricielle de \(\paren{S}\) : \[\paren{S}~AX=B\qquad\text{équation matricielle d'inconnue }X=\tcoords{x_1}{\vdots}{x_p}\in\M{p1}.\]

Écriture vectorielle de \(\paren{S}\) : \[\paren{S}~x_1C_1+\dots+x_pC_p=B\qquad\text{équation vectorielle d'inconnue }X=\tcoords{x_1}{\vdots}{x_p}\in\K^p.\]

\begin{rem}
\begin{enumerate}
\item Notons \(\paren{S_0}\) le système homogène associé à \(\paren{S}\) (\cad celui obtenu en \guillemets{remplaçant \(B\) par \(0_{n1}\)}). L'écriture matricielle de \(\paren{S_0}\) montre que son espace vectoriel solution \(\fami{S}_0\) est le noyau de l'application linéaire canoniquement associée à \(A\) : \[\ker u_A=\fami{S}_0\subset\K^p.\]

\item L'écriture vectorielle de \(\paren{S}\) montre que \(\paren{S}\) possède au moins une solution\footnote{Si le système \(\paren{S}\) possède au moins une solution, on dit que \(\paren{S}\) est compatible.} si, et seulement si, le second membre \(B\) est combinaison linéaire des vecteurs-colonne de \(A\) : \[\fami{S}\not=\ensvide\ssi B\in\Vect{C_1,\dots,C_p}.\]

\item Cette solution est alors unique si, et seulement si, la famille \(\paren{C_1,\dots,C_p}\) des vecteurs-colonne de \(A\) est libre.
\end{enumerate}
\end{rem}

\begin{exoex}
Écrire matriciellement et vectoriellement le système linéaire suivant : \[\paren{S}~\begin{dcases}
a+2b+c+3d=0 \\
3a+7b+3c+6d=0 \\
a+3b+c=7
\end{dcases}\]
\end{exoex}

\begin{corr}
On a : \[\paren{S}~\begin{pmatrix}
1 & 2 & 1 & 3 \\
3 & 7 & 3 & 6 \\
1 & 3 & 1 & 0
\end{pmatrix}\begin{pmatrix}
a \\
b \\
c \\
d
\end{pmatrix}=\tcoords{0}{0}{7}\qquad\text{ou}\qquad\paren{S}~a\tcoords{1}{3}{1}+b\tcoords{2}{7}{3}+c\tcoords{1}{3}{1}+d\tcoords{3}{6}{0}=\tcoords{0}{0}{7}.\]
\end{corr}

\begin{defi}
On appelle rang du système linéaire \(\paren{S}\) le rang de la matrice \(A\) (dans l'écriture matricielle), \cad le rang de la famille de vecteurs \(\paren{C_1,\dots,C_p}\) (dans l'écriture vectorielle).
\end{defi}

\begin{prop}
Notons \(r\) le rang du système \(\paren{S}\).

L'ensemble solution \(\fami{S}\) du système \(\paren{S}\) est soit l'ensemble vide, soit un sous-espace affine de \(\K^p\) de dimension \(p-r\).
\end{prop}

\begin{dem}
Supposons \(\fami{S}\not=\ensvide\).

On a vu que \(\fami{S}\) est un sous-espace affine de \(\K^p\) dirigé par \(\fami{S}_0=\ker u_A\).

Selon le théorème du rang appliqué à \(u_A\in\L{\K^p}{\K^n}\), on a : \[\begin{aligned}
\dim\K^p&=\rg u_A+\dim\ker u_A \\
p&=r+\dim\fami{S}_0.
\end{aligned}\]

Donc \(\fami{S}_0\) est un espace vectoriel de dimension \(p-r\) et \(\fami{S}\) est un sous-espace affine de \(\K^p\) dirigé par \(\fami{S}_0\).
\end{dem}

\begin{defi}[Système de Cramer]
On dit que \(\paren{S}\) est un système de Cramer si \(n=p\) et si la matrice \(A\) est inversible.
\end{defi}

\begin{prop}
Tout système de Cramer admet une unique solution.
\end{prop}

\begin{dem}
C'est clair avec l'écriture matricielle : \[\quantifs{\forall X\in\K^p}AX=B\ssi X=A\inv B.\]
\end{dem}

\section{Calculer l'inverse d'une matrice}

\begin{algo}
Soient \(n\in\Ns\) et \(A\in\M{n}\).

En appliquant l'algorithme du pivot de Gauss aux lignes de \(A\), on arrive à \(I_n\) si \(A\) est inversible ou à une matrice simple de même rang que \(A\) si \(A\) n'est pas inversible.
\end{algo}

\begin{exoex}~\\
Décider si \(\begin{pmatrix}
2 & 1 \\
1 & 1
\end{pmatrix}\) est inversible et calculer son inverse.
\end{exoex}

\begin{corr}
On applique l'algorithme : \[\begin{aligned}
&\begin{pNiceArray}{cc|cc}
2 & 1 & 1 & 0 \\
1 & 1 & 0 & 1
\end{pNiceArray} \\
&\begin{pNiceArray}{cc|cc}[last-col]
1 & 1 & 0 & 1 & L_1\echange L_2 \\
2 & 1 & 1 & 0 &
\end{pNiceArray} \\
&\begin{pNiceArray}{cc|cc}[last-col]
1 & 1 & 0 & 1 & \\
0 & -1 & 1 & -2 & L_2\gets L_2-2L_1
\end{pNiceArray} \\
&\begin{pNiceArray}{cc|cc}[last-col]
1 & 1 & 0 & 1 & \\
0 & 1 & -1 & 2 & L_2\gets-L_2
\end{pNiceArray} \\
&\begin{pNiceArray}{cc|cc}[last-col]
1 & 0 & 1 & -1 & L_1\gets L_1-L_2 \\
0 & 1 & -1 & 2
\end{pNiceArray}
\end{aligned}\]

Donc \(\begin{pmatrix}
2 & 1 \\
1 & 1
\end{pmatrix}\) est inversible, d'inverse \(\begin{pmatrix}
1 & -1 \\
-1 & 2
\end{pmatrix}\).

Vérification : on a bien \(\begin{pmatrix}
2 & 1 \\
1 & 1
\end{pmatrix}\begin{pmatrix}
1 & -1 \\
-1 & 2
\end{pmatrix}=I_2\).
\end{corr}

\chapter{Matrices II}

\minitoc

On considère un corps \(\K\) (en pratique \(\K=\R\) ou \(\C\)).

\section{Matrice d'une famille de vecteurs}

\subsection{Définition}

\begin{defi}[Matrice d'une famille de vecteurs dans une base]
Soient \(E\) un \(\K\)-espace vectoriel de dimension \(n\in\Ns\), \(\fami{B}=\paren{e_1,\dots,e_n}\) une base de \(E\) et \(\paren{x_1,\dots,x_p}\in E^p\) une famille d'éléments de \(E\).

La matrice de la famille \(\paren{x_1,\dots,x_p}\) dans la base \(\fami{B}\) est la matrice : \[\Mat{x_1,\dots,x_p}=\begin{pmatrix}
a_{11} & \dots & a_{1p} \\
\vdots &  & \vdots \\
a_{n1} & \dots & a_{np}
\end{pmatrix}\in\M{np}\] dont les colonnes sont les coordonnées respectives des vecteurs de la famille \(\paren{x_1,\dots,x_p}\) : \[\quantifs{\forall j\in\interventierii{1}{n}}x_j=\sum_{i=1}^{n}a_{ij}e_i.\]
\end{defi}

\begin{rem}
On garde les notations de la définition précédente.

Si \(p=1\) (\cad si la famille de vecteurs ne possède qu'un seul vecteur), alors la matrice \(\Mat{x_1}\) de la famille \(\paren{x_1}\) dans la base \(\fami{B}\) est la matrice-colonne (\cad le \(n\)-uplet) des coordonnées de \(x_1\) dans la base \(\fami{B}\).
\end{rem}

\begin{exoex}
On note \(T_0\), \(T_1\) et \(T_2\) les trois premiers polynômes de Tchebychev.

Écrire la matrice de la famille \(\paren{T_0,T_1,T_2}\) dans la base canonique de \(\polydeg[\R]{2}\) puis dans la base canonique de \(\polydeg[\R]{3}\).
\end{exoex}

\begin{corr}
On a : \[\Mat[\paren{1,X,X^2}]{1,X,2X^2-1}=\begin{pmatrix}
1 & 0 & -1 \\
0 & 1 & 0 \\
0 & 0 & 2
\end{pmatrix}\qquad\text{et}\qquad\Mat[\paren{1,X,X^2,X^3}]{1,X,2X^2-1}=\begin{pmatrix}
1 & 0 & -1 \\
0 & 1 & 0 \\
0 & 0 & 2 \\
0 & 0 & 0
\end{pmatrix}.\]
\end{corr}

\begin{defi}[Matrice de passage]
Soient \(E\) un \(\K\)-espace vectoriel de dimension \(n\in\Ns\) et \(\fami{B}=\paren{e_1,\dots,e_n}\) et \(\fami{B}\prim=\paren{x_1,\dots,x_n}\) deux bases de \(E\).

La matrice \(\Mat{x_1,\dots,x_n}\) est appelée matrice de passage de la base \(\fami{B}\) à la base \(\fami{B}\prim\) et est notée \(\pass{\fami{B}}{\fami{B}\prim}\) : \[\pass{\fami{B}}{\fami{B}\prim}=\Mat{\fami{B}\prim}=\Mat{x_1,\dots,x_n}.\]
\end{defi}

\begin{rem}
Soient \(E\) un \(\K\)-espace vectoriel de dimension \(n\in\Ns\) et \(\fami{B}\) une base de \(E\).

On a : \[\pass{\fami{B}}{\fami{B}}=I_n.\]
\end{rem}

\subsection{Formules de changement de base}

\begin{prop}[Formule de changement de base pour les coordonnées d'un vecteur]
Soient \(E\) un \(\K\)-espace vectoriel de dimension \(n\in\Ns\), \(\fami{B}\) et \(\fami{B}\prim\) deux bases de \(E\) et \(x\) un vecteur de \(E\).

On considère les coordonnées \(X\in\K^n\) et \(X\prim\in\K^n\) de \(x\) dans les bases \(\fami{B}\) et \(\fami{B}\prim\) : \[X=\tcoords{x_1}{\vdots}{x_n}=\Mat{x}\qquad\text{et}\qquad X\prim=\tcoords{x_1\prim}{\vdots}{x_n\prim}=\Mat[\fami{B}\prim]{x}.\]

On a : \[X=\pass{\fami{B}}{\fami{B}\prim}X\prim,\] \cad : \[\Mat{x}=\pass{\fami{B}}{\fami{B}\prim}\Mat[\fami{B}\prim]{x}.\]
\end{prop}

\begin{dem}
On note \(\fami{B}=\paren{e_1,\dots,e_n}\), \(\fami{B}\prim=\paren{e_1\prim,\dots,e_n\prim}\) et \(\pass{\fami{B}}{\fami{B}\prim}=\paren{a_{ij}}_{\paren{i,j}}\).

On a : \[x=\sum_{k=1}^{n}x_ke_k=\sum_{k=1}^{n}x_k\prim e_k\prim\qquad\text{et}\qquad\quantifs{\forall j\in\interventierii{1}{n}}e_j\prim=\sum_{i=1}^{n}a_{ij}e_i.\]

Donc : \[x=\sum_{j=1}^{n}x_j\prim\sum_{i=1}^{n}a_{ij}e_i=\sum_{i=1}^{n}e_i\sum_{j=1}^{n}a_{ij}x_j\prim.\]

D'où \(\quantifs{\forall i\in\interventierii{1}{n}}x_i=\sum_{j=1}^{n}a_{ij}x_j\prim\), \cad : \[X=\pass{\fami{B}}{\fami{B}\prim}X\prim.\]
\end{dem}

\begin{exoex}
On note \(\fami{C}\) le cercle unité de \(\R^2\), d'équation cartésienne (dans la base canonique \(\fami{B}\)) : \[\fami{C}:x^2+y^2=1.\]

\begin{enumerate}
    \item Justifier que la famille \(\fami{B}\prim=\paren{\dcoords{2}{0},\dcoords{1}{1}}\) est une base de \(\R^2\). \\
    \item Donner une équation cartésienne de \(\fami{C}\) dans la base \(\fami{B}\prim\), \cad, pour tout point \(M\in\R^2\), une CNS en fonction des coordonnées \(\paren{x\prim,y\prim}\) de \(M\) dans \(\fami{B}\prim\) pour que \(M\) appartienne à \(\fami{C}\).
\end{enumerate}
\end{exoex}

\begin{corr}[1]
\(\fami{B}\prim\) est clairement libre et possède deux éléments donc c'est une base de \(\R^2\).
\end{corr}

\begin{corr}[2]
Soit \(M\in\R^2\) dont on note \(\paren{x,y}\) les coordonnées dans \(\fami{B}\) et \(\paren{x\prim,y\prim}\) les coordonnées dans \(\fami{B}\prim\).

On a : \[\dcoords{x}{y}=\pass{\fami{B}}{\fami{B}\prim}\dcoords{x\prim}{y\prim}=\begin{pmatrix}
2 & 1 \\
0 & 1
\end{pmatrix}\dcoords{x\prim}{y\prim}=\dcoords{2x\prim+y\prim}{y\prim}.\]

Donc \(\begin{dcases}
x=2x\prim+y\prim \\
y=y\prim
\end{dcases}\)

D'où : \[\begin{aligned}
M\in\fami{C}&\ssi x^2+y^2=1 \\
&\ssi\paren{2x\prim+y\prim}^2+{y\prim}^2=1 \\
&\ssi4{x\prim}^2+4x\prim y\prim+2{y\prim}^2=1.
\end{aligned}\]
\end{corr}

\begin{prop}\thlabel{prop:formuleDeChangementDeBasePourLaMatriceD'UneFamilleDeVecteurs}
{\normalfont\bfseries(Formule de changement de base pour la matrice d'une famille de vecteurs)}

Soient \(E\) un \(\K\)-espace vectoriel de dimension \(n\in\Ns\), \(\fami{B}\) et \(\fami{B}\prim\) deux bases de \(E\) et \(\fami{F}\) une famille de vecteurs.

On a : \[\Mat{\fami{F}}=\pass{\fami{B}}{\fami{B}\prim}\Mat[\fami{B}\prim]{\fami{F}}.\]
\end{prop}

\begin{dem}
On note \(\fami{F}=\paren{x_1,\dots,x_p}\in E^p\), \(\fami{B}=\paren{e_1,\dots,e_n}\), \(\fami{B}\prim=\paren{e_1\prim,\dots,e_n\prim}\), \(\Mat{\fami{F}}=\begin{pmatrix}C_1 & \dots & C_p\end{pmatrix}\) et \(\Mat[\fami{B}\prim]{\fami{F}}=\begin{pmatrix}C_1\prim & \dots & C_p\prim\end{pmatrix}\).

On a \(\quantifs{\forall j\in\interventierii{1}{p}}C_j=\pass{\fami{B}}{\fami{B}\prim}C_j\prim\) donc : \[\begin{aligned}
\Mat{\fami{F}}&=\begin{pmatrix}\pass{\fami{B}}{\fami{B}\prim}C_1\prim & \dots & \pass{\fami{B}}{\fami{B}\prim}C_p\prim\end{pmatrix} \\
&=\pass{\fami{B}}{\fami{B}\prim}\begin{pmatrix}C_1\prim & \dots & C_p\prim\end{pmatrix} \\
&=\pass{\fami{B}}{\fami{B}\prim}\Mat[\fami{B}\prim]{\fami{F}}.
\end{aligned}\]
\end{dem}

\begin{cor}
Soient \(E\) un \(\K\)-espace vectoriel de dimension \(n\in\Ns\) et \(\fami{B}\), \(\fami{B}\prim\) et \(\fami{B}\seconde\) trois bases de \(E\).

On a : \[\pass{\fami{B}}{\fami{B}\seconde}=\pass{\fami{B}}{\fami{B}\prim}\pass{\fami{B}\prim}{\fami{B}\seconde}.\]
\end{cor}

\begin{dem}
On a, selon la \thref{prop:formuleDeChangementDeBasePourLaMatriceD'UneFamilleDeVecteurs} : \[\Mat{\fami{B}\seconde}=\pass{\fami{B}}{\fami{B}\prim}\Mat[\fami{B}\prim]{\fami{B}\seconde}.\]

Comme \(\fami{B}\seconde\) est aussi une base, cette formule s'écrit : \[\pass{\fami{B}}{\fami{B}\seconde}=\pass{\fami{B}}{\fami{B}\prim}\pass{\fami{B}\prim}{\fami{B}\seconde}.\]
\end{dem}

\begin{prop}[Inversibilité des matrices de passage]
Soient \(E\) un \(\K\)-espace vectoriel de dimension \(n\in\Ns\) et \(\fami{B}\) et \(\fami{B}\prim\) deux bases de \(E\).

La matrice de passage \(\pass{\fami{B}}{\fami{B}\prim}\) est inversible, d'inverse : \[\pass{\fami{B}}{\fami{B}\prim}\inv=\pass{\fami{B}\prim}{\fami{B}}.\]
\end{prop}

\begin{dem}
On a : \[I_n=\pass{\fami{B}}{\fami{B}}=\pass{\fami{B}}{\fami{B}\prim}\pass{\fami{B}\prim}{\fami{B}}.\]

Donc \(\pass{\fami{B}}{\fami{B}\prim}\) est inversible, d'inverse \(\pass{\fami{B}\prim}{\fami{B}}\).
\end{dem}

\subsection{Isomorphisme entre familles de vecteurs et matrices}

\begin{prop}
Soient \(n,p\in\Ns\), \(E\) un \(\K\)-espace vectoriel de dimension \(n\) et \(\fami{B}\) une base de \(E\).

L'application \[\fonction{\phi}{E^p}{\M{np}}{\paren{x_1,\dots,x_p}}{\Mat{x_1,\dots,x_p}}\] est un isomorphisme d'espaces vectoriels.

De plus, cet isomorphisme conserve le rang : \[\quantifs{\forall\fami{F}\in E^p}\rg\fami{F}=\rg\Mat{\fami{F}}.\]

Ainsi, étant donnée une famille de vecteurs en dimension finie, on peut calculer son rang et décider si elle est libre ou génératrice en écrivant sa matrice dans une base de l'espace et en calculant le rang de cette matrice.
\end{prop}

\begin{exoex}
On considère la famille de polynômes de \(\polydeg[\R]{3}\) : \[\fami{F}=\paren{X^3+X^2+X+1,X^2-1,\paren{X+1}^3,X^3-X}.\]

\begin{enumerate}
    \item Écrire la matrice de \(\fami{F}\) dans la base canonique de \(\polydeg[\R]{3}\). \\
    \item En déduire le rang de \(\fami{F}\). \\
    \item Est-ce une famille libre ? une famille génératrice de \(\polydeg[\R]{3}\) ?
\end{enumerate}
\end{exoex}

\begin{corr}[1]
On pose : \[A=\Mat[\paren{1,X,X^2,X^3}]{\fami{F}}=\begin{pmatrix}
1 & -1 & 1 & 0 \\
1 & 0 & 3 & -1 \\
1 & 1 & 3 & 0 \\
1 & 0 & 1 & 1
\end{pmatrix}.\]
\end{corr}

\begin{corr}[2]
On a : \[\begin{aligned}
\rg\fami{F}&=\rg A \\
&=\rg\begin{pmatrix}
1 & -1 & 1 & 0 \\
1 & 0 & 3 & -1 \\
1 & 1 & 3 & 0 \\
1 & 0 & 1 & 1
\end{pmatrix} \\
&=\rg\begin{pNiceMatrix}[last-col]
0 & -1 & 0 & -1 & L_1\gets L_1-L_4 \\
1 & 0 & 3 & -1 & \\
0 & 1 & 0 & 1 & L_3\gets L_3-L_2 \\
1 & 0 & 1 & 1
\end{pNiceMatrix} \\
&=\rg\begin{pNiceMatrix}
1 & 0 & 3 & -1 \\
0 & 1 & 0 & 1 \\
1 & 0 & 1 & 1
\end{pNiceMatrix} \\
&=\rg\begin{pNiceMatrix}[last-col]
1 & 0 & 3 & -1 & \\
0 & 1 & 0 & 1 & \\
0 & 0 & -2 & 2 & L_3\gets L_3-L_1
\end{pNiceMatrix} \\
&=\rg\begin{pNiceMatrix}[last-col]
1 & 0 & 2 & -1 & C_3\gets C_3+C_4-C_2 \\
0 & 1 & 0 & 1 & \\
0 & 0 & 0 & 2 &
\end{pNiceMatrix} \\
&=\rg\begin{pNiceMatrix}[last-col]
1 & 0 & 0 & 0 & C_3\gets C_3-2C_1 \\
0 & 1 & 0 & 0 & C_4\gets C_4+C_1-C_2 \\
0 & 0 & 0 & 2 &
\end{pNiceMatrix} \\
&=\rg\begin{pmatrix}
1 & 0 & 0 \\
0 & 1 & 0 \\
0 & 0 & 2
\end{pmatrix} \\
&=3.
\end{aligned}\]
\end{corr}

\begin{corr}[3]
\(\fami{F}\) n'est pas libre car son rang diffère de son nombre de vecteurs et n'est pas génératrice de \(\polydeg[\R]{3}\) car son rang diffère de la dimension de \(\polydeg[\R]{3}\).
\end{corr}

\section{Matrice d'une application linéaire}

\subsection{Définition}

\begin{defi}[Matrice d'une application linéaire dans deux bases]
Soient \(E\) un \(\K\)-espace vectoriel de dimension \(p\in\Ns\), \(F\) un \(\K\)-espace vectoriel de dimension \(n\in\Ns\), \(\fami{B}=\paren{e_1,\dots,e_p}\) une base de \(E\), \(\fami{C}=\paren{\epsilon_1,\dots,\epsilon_n}\) une base de \(F\) et \(u\in\L{E}{F}\).

On appelle matrice de l'application linéaire \(u\) dans les bases \(\fami{B}\) et \(\fami{C}\) la matrice de la famille \(\paren{u\paren{e_1},\dots,u\paren{e_p}}\) de vecteurs de \(F\) dans la base \(\fami{C}\) : \[\Mat[\fami{B},\fami{C}]{u}=\Mat[\fami{C}]{u\paren{e_1},\dots,u\paren{e_n}}.\]

Dans le cas où \(E=F\), \cad le cas où \(u\) est un endomorphisme, on choisit la même base \guillemets{au départ} et \guillemets{à l'arrivée}. On définit alors la matrice de l'endomorphisme \(u\) dans la base \(\fami{B}\) : \[\Mat{u}=\Mat[\fami{B},\fami{B}]{u}=\Mat{u\paren{e_1},\dots,u\paren{e_p}}.\]
\end{defi}

\begin{exoex}\thlabel{exoex:matriceD'ApplicationsLinéaires}
On pose \(E=\polydeg[\R]{3}\).

On note \(\fami{B}\) la base canonique de \(E\) et \(\fami{C}=\paren{1}\) la base canonique de \(\R\).

\begin{enumerate}
    \item Écrire la matrice de \(\fonction{D}{E}{E}{P}{P\prim}\) dans la base \(\fami{B}\). \\
    \item Écrire la matrice de \(\fonction{u}{E}{E}{P}{P\paren{X+1}}\) dans la base \(\fami{B}\). \\
    \item Écrire la matrice de \(\fonction{l}{E}{\R}{P}{P\paren{2}}\) dans les bases \(\fami{B}\) et \(\fami{C}\).
\end{enumerate}
\end{exoex}

\begin{corr}[1]
On a : \[\Mat{D}=\begin{pmatrix}
0 & 1 & 0 & 0 \\
0 & 0 & 2 & 0 \\
0 & 0 & 0 & 3 \\
0 & 0 & 0 & 0
\end{pmatrix}.\]
\end{corr}

\begin{corr}[2]
On a : \[\Mat{u}=\begin{pmatrix}
1 & 1 & 1 & 1 \\
0 & 1 & 2 & 3 \\
0 & 0 & 1 & 3 \\
0 & 0 & 0 & 1
\end{pmatrix}.\]
\end{corr}

\begin{corr}[3]
On a : \[\Mat[\fami{B},\fami{C}]{l}=\begin{pmatrix}1 & 2 & 4 & 8\end{pmatrix}.\]
\end{corr}

\begin{rem}
Soient \(n,p\in\Ns\).

Si \(A\in\M{np}\), l'application linéaire de \(\K^p\) dans \(\K^n\) canoniquement associée à \(A\) est l'application linéaire \(u_A\in\L{\K^p}{\K^n}\) dont la matrice dans les bases canoniques respectives \(\fami{B}_0\) et \(\fami{C}_0\) de \(\K^p\) et \(\K^n\) est \(A\) : \[\Mat[\fami{B}_0,\fami{C}_0]{u_A}=A.\]

Si \(A\in\M{n}\), l'endomorphisme de \(\K^n\) canoniquement associé à \(A\) est l'endomorphisme \(u_A\in\Lendo{\K^n}\) dont la matrice dans la base canonique \(\fami{B}_0\) de \(\K^n\) est \(A\) : \[\Mat[\fami{B}_0]{u_A}=A.\]
\end{rem}

\begin{prop}[Calcul de l'image d'un vecteur]
Soient \(E\) un \(\K\)-espace vectoriel de dimension \(p\in\Ns\), \(F\) un \(\K\)-espace vectoriel de dimension \(n\in\Ns\), \(\fami{B}=\paren{e_1,\dots,e_p}\) une base de \(E\), \(\fami{C}=\paren{\epsilon_1,\dots,\epsilon_n}\) une base de \(F\) et \(u\in\L{E}{F}\).

La matrice \(\Mat[\fami{B},\fami{C}]{u}\) permet de \guillemets{calculer} l'image par \(u\) d'un vecteur de \(E\) :

Posons \(A=\Mat[\fami{B},\fami{C}]{u}\).

Si \(x\) est un vecteur de \(E\) de coordonnées \(X=\tcoords{x_1}{\vdots}{x_p}\) dans \(\fami{B}\), alors son image \(y=u\paren{x}\) est le vecteur de \(F\) dont les coordonnées \(Y=\tcoords{y_1}{\vdots}{y_n}\) dans \(\fami{C}\) sont données par la formule : \[Y=AX\qquad\text{\cad : }\Mat[\fami{C}]{u\paren{x}}=\Mat[\fami{B},\fami{C}]{u}\Mat{x}.\]

Cette propriété caractérise la matrice \(A\).
\end{prop}

\begin{dem}
On note \(\Mat[\fami{B},\fami{C}]{u}=\begin{pmatrix}C_1 & \dots & C_p\end{pmatrix}\), \cad : \[\quantifs{\forall j\in\interventierii{1}{p}}C_j=\Mat[\fami{C}]{u\paren{e_j}}.\]

On a \(x=x_1e_1+\dots+x_pe_p\) donc \(u\paren{x}=x_1u\paren{e_1}+\dots+x_pu\paren{e_p}\in F\).

Donc en prenant les coordonnées dans \(\fami{C}\) : \[\begin{aligned}
Y&=x_1C_1+\dots+x_pC_p \\
&=\begin{pmatrix}C_1 & \dots & C_p\end{pmatrix}\tcoords{x_1}{\vdots}{x_p} \\
&=AX.
\end{aligned}\]

Si \(B\) est une autre matrice vérifiant la même propriété, on a : \[\quantifs{\forall Z\in\K^p}AZ=BZ.\]

Donc en prenant comme vecteur \(Z\) le \(j\)-ème vecteur de la base canonique de \(\K^p\), \(A\) et \(B\) ont même \(j\)-ème colonne.

Donc \(A=B\).
\end{dem}

\begin{ex}
On reprend les exemples (1) et (3) de l'\thref{exoex:matriceD'ApplicationsLinéaires}.

On a \(\begin{pmatrix}
0 & 1 & 0 & 0 \\
0 & 0 & 2 & 0 \\
0 & 0 & 0 & 3 \\
0 & 0 & 0 & 0
\end{pmatrix}\begin{pmatrix}d \\ c \\ b \\ a\end{pmatrix}=\begin{pmatrix}c \\ 2b \\ 3a \\ 0\end{pmatrix}\) donc : \[D\paren{aX^3+bX^2+cX+d}=3aX^2+2bX+c.\]

De même, on a \(\begin{pmatrix}1 & 2 & 4 & 8\end{pmatrix}\begin{pmatrix}d \\ c \\ b \\ a\end{pmatrix}=d+2c+4b+8a\) donc \(l\paren{aX^3+bX^2+cX+d}=8a+4b+2c+d\) donc : \[\paren{aX^3+bX^2+cX+d}\paren{2}=8a+4b+2c+d.\]
\end{ex}

\subsection{Formules de changement de base}

\begin{prop}
{\normalfont\bfseries (Formule de changement de base pour la matrice d'une application linéaire)}

Soient \(E\) un \(\K\)-espace vectoriel de dimension \(p\in\Ns\), \(F\) un \(\K\)-espace vectoriel de dimension \(n\in\Ns\), \(\fami{B}\) et \(\fami{B}\prim\) deux bases de \(E\), \(\fami{C}\) et \(\fami{C}\prim\) deux bases de \(F\) et \(u\in\L{E}{F}\).

On a : \[\Mat[\fami{B},\fami{C}]{u}=\pass{\fami{C}}{\fami{C}\prim}\Mat[\fami{B}\prim,\fami{C}\prim]{u}\pass{\fami{B}\prim}{\fami{B}}.\]
\end{prop}

\begin{dem}
Soit \(x\in E\).

On note \(y=u\paren{x}\), \(X,X\prim\in\K^p\) les coordonnées de \(x\) dans les bases \(\fami{B}\) et \(\fami{B}\prim\) respectivement, \(Y,Y\prim\in\K^n\) les coordonnées de \(y\) dans les bases \(\fami{C}\) et \(\fami{C}\prim\) respectivement, \(A=\Mat[\fami{B},\fami{C}]{u}\), \(A\prim=\Mat[\fami{B}\prim,\fami{C}\prim]{u}\), \(P=\pass{\fami{B}}{\fami{B}\prim}\) et \(Q=\pass{\fami{C}}{\fami{C}\prim}\).

On a vu \(\begin{dcases}
X=PX\prim \\
Y=QY\prim
\end{dcases}\) et \(\begin{dcases}
Y=AX \\
Y\prim=A\prim X\prim
\end{dcases}\)

Donc \(QY\prim=APX\prim\).

Donc \(Y\prim=Q\inv APX\prim\).

Donc \(Q\inv AP=A\prim\), \cad : \[\pass{\fami{C}}{\fami{C}\prim}\inv\Mat[\fami{B},\fami{C}]{u}\pass{\fami{B}}{\fami{B}\prim}=\Mat[\fami{B}\prim,\fami{C}\prim]{u}.\]

D'où la formule souhaitée car \(\pass{\fami{C}}{\fami{C}\prim}\inv=\pass{\fami{C}\prim}{\fami{C}}\).
\end{dem}

\begin{cor}
{\normalfont\bfseries (Formule de changement de base pour la matrice d'un endomorphisme)}

Soient \(E\) un \(\K\)-espace vectoriel de dimension \(n\in\Ns\), \(\fami{B}\) et \(\fami{B}\prim\) deux bases de \(E\) et \(u\in\Lendo{E}\).

On a : \[\Mat{u}=\pass{\fami{B}}{\fami{B}\prim}\Mat[\fami{B}\prim]{u}\pass{\fami{B}\prim}{\fami{B}}.\]
\end{cor}

\begin{rem}
Deux matrices carrées sont semblables si, et seulement si, ce sont les matrices d'un même endomorphisme dans deux bases différentes.
\end{rem}

\begin{rem}
On peut maintenant démontrer que toute matrice de rang \(r\) est équivalente à \(J_r\) (on l'avait admis dans la \thref{dem:matriceDeRangrÉquivalenteÀJrAMoitiéAdmis}).
\end{rem}

\begin{appl}
Soient \(n,p\in\Ns\), \(A\in\M{np}\) et \(r\in\N\).

Alors : \[\rg A=r\ssi\quantifs{\exists P\in\GL{n};\exists Q\in\GL{p}}A=PJ_rQ,\] en posant \[J_r=\begin{pNiceMatrix}[first-col,last-row]
1 & 1 & 0 & \dots & 0 & \Block{4-4}<\Huge>{0} \\
& 0 & \ddots & \ddots & \vdots & & & & \\
& \vdots & \ddots & \ddots & 0 & & & & \\
r & 0 & \dots & 0 & 1 & & & & \\
& \Block{4-4}<\Huge>{0} & & & & \Block{4-4}<\Huge>{0} \\
&&&&&&&&& \\
&&&&&&&&& \\
&&&&&&&&& \\
n &&&&&&&&& \\
& 1 &  &  & r & & & & p
\end{pNiceMatrix}.\]
\end{appl}

\begin{dem}
Il reste à montrer \(\impdir\).

Supposons \(\rg A=r\).

On considère l'application linéaire canoniquement associée à \(A\) : \(u_A\in\L{\K^p}{\K^n}\).

On a \(\rg u_A=\rg A=r\).

Soit \(S\) un supplémentaire de \(\ker u_A\) dans \(\K^p\) : \[\K^p=S\oplus\ker u_A.\]

Selon le théorème du rang, \(u_A\) induit un isomorphisme de \(S\) vers \(\Im u_A\).

On a donc : \[\dim S=\dim\Im u_A=\rg u_A=r.\]

Soit \(\paren{e_1,\dots,e_r}\) une base de \(S\).

Comme les isomorphismes conservent les bases, \(\paren{u_A\paren{e_1},\dots,u_A\paren{e_r}}\) est une base de \(\Im u_A\).

Soit \(\paren{e_{r+1},\dots,e_p}\) une base de \(\ker u_A\).

La famille \(\fami{B}=\paren{e_1,\dots,e_p}\) est une base de \(\K^p\) car \(\K^p=S\oplus\ker u_A\).

Enfin, la famille \(\paren{u_A\paren{e_1},\dots,u_A\paren{e_r}}\) est une famille libre de \(\K^n\) donc selon le théorème de la base incomplète, on peut la compléter en une base \(\fami{C}=\paren{u_A\paren{e_1},\dots,u_A\paren{e_r},\epsilon_{r+1},\dots,\epsilon_n}\) de \(\K^n\).

On note \(\fami{B}_0\) et \(\fami{C}_0\) les bases respectives de \(\K^p\) et \(\K^n\).

On a, d'une part, \(\Mat[\fami{B}_0,\fami{C}_0]{u_A}=A\).

D'autre part : \[\Mat[\fami{B},\fami{C}]{u_A}=J_r.\] % TODO : expliciter

On a donc \[A=\Mat[\fami{B}_0,\fami{C}_0]{u_A}=\pass{\fami{C_0}}{\fami{C}}\Mat[\fami{B},\fami{C}]{u_A}\pass{\fami{B}}{\fami{B}_0}=PJ_rQ\] en posant \(\begin{dcases}
P=\pass{\fami{C_0}}{\fami{C}}\in\GL{n} \\
Q=\pass{\fami{B}}{\fami{B}_0}\in\GL{p}
\end{dcases}\)
\end{dem}

\subsection{Isomorphismes entre applications linéaires et matrices}

\begin{prop}
Soient \(E\) un \(\K\)-espace vectoriel de dimension \(p\in\Ns\), \(F\) un \(\K\)-espace vectoriel de dimension \(n\in\Ns\), \(\fami{B}\) une base de \(E\) et \(\fami{C}\) une base de \(F\).

L'application \[\fonctionlambda{\L{E}{F}}{\M{np}}{u}{\Mat[\fami{B},\fami{C}]{u}}\] est un isomorphisme d'espaces vectoriels.

De plus, elle conserve le rang : \[\quantifs{\forall u\in\L{E}{F}}\rg u=\rg\Mat[\fami{B},\fami{C}]{u}.\]
\end{prop}

\begin{prop}[Matrice d'une composée]
Soient \(E\), \(F\) et \(G\) des espaces vectoriels de dimensions finies non-nulles, \(\fami{B}_E\) une base de \(E\), \(\fami{B}_F\) une base de \(F\) et \(\fami{B}_G\) une base de \(G\), \(u\in\L{E}{F}\) et \(v\in\L{F}{G}\).

On a : \[\Mat[\fami{B}_E,\fami{B}_G]{v\rond u}=\Mat[\fami{B}_F,\fami{B}_G]{v}\Mat[\fami{B}_E,\fami{B}_F]{u}.\]

Cas des endomorphismes : si \(E=F=G\) et \(\fami{B}_E=\fami{B}_F=\fami{B}_G=\fami{B}\), la formule s'écrit : \[\Mat{v\rond u}=\Mat{v}\Mat{u}.\]
\end{prop}

\begin{dem}
Soit \(x\in E\).

On note \(y=u\paren{x}\), \(z=v\paren{y}=v\rond u\paren{x}\), \(X=\Mat[\fami{B}_E]{x}\), \(Y=\Mat[\fami{B}_F]{y}\), \(Z=\Mat[\fami{B}_G]{z}\), \(A=\Mat[\fami{B}_E,\fami{B}_F]{u}\) et \(B=\Mat[\fami{B}_F,\fami{B}_G]{v}\).

On a \(Y=AX\) et \(Z=BY\) donc \(Z=BAX\).

D'où \(\Mat[\fami{B}_E,\fami{B}_G]{v\rond u}=BA\).
\end{dem}

\begin{prop}[Matrice d'une bijection réciproque]
Soient \(E\) et \(F\) deux espaces vectoriels de dimension \(n\in\Ns\), \(\fami{B}_E\) une base de \(E\), \(\fami{B}_F\) une base de \(F\) et \(u\in\L{E}{F}\).

On a : \[u\text{ est un isomorphisme de }E\text{ vers }F\ssi\Mat[\fami{B}_E,\fami{B}_F]{u}\in\GL{n}.\]

On a alors : \[\Mat[\fami{B}_F,\fami{B}_E]{u\inv}=\Mat[\fami{B}_E,\fami{B}_F]{u}\inv.\]
\end{prop}

\begin{dem}
Posons \(A=\Mat[\fami{B}_E,\fami{B}_F]{u}\).

On a \(\rg A=\rg u\).

Donc, comme \(\dim E=\dim F\) : \[\begin{aligned}
u\text{ est un isomorphisme de }E\text{ vers }F&\ssi u\text{ est une surjection de }E\text{ vers }F \\
&\ssi\rg u=\dim F \\
&\ssi\rg A=n \\
&\ssi A\in\GL{n}.
\end{aligned}\]

Soient \(x\in E\) et \(y\in F\).

On pose \(X=\Mat[\fami{B}_E]{x}\) et \(Y=\Mat[\fami{B}_F]{y}\).

On a : \[\begin{aligned}
x=u\inv\paren{y}&\ssi u\paren{x}=y \\
&\ssi AX=Y \\
&\ssi X=A\inv Y.
\end{aligned}\]

Donc \(A\inv=\Mat[\fami{B}_F,\fami{B}_E]{u\inv}\).
\end{dem}

\begin{prop}
Soient \(E\) un espace vectoriel de dimension \(n\in\Ns\) et \(\fami{B}\) une base de \(E\).

L'application \[\fonctionlambda{\Lendo{E}}{\M{n}}{u}{\Mat{u}}\] est un isomorphisme d'anneaux de \(\anneau{\Lendo{E}}[+][\rond]\) vers \(\anneau{\M{n}}\).

En particulier, on a : \[\quantifs{\forall u\in\Lendo{E}}u\in\GL{}[E]\ssi\Mat{u}\in\GL{n},\] et on a alors : \[\Mat{u\inv}=\Mat{u}\inv.\]

De plus, on a : \[\quantifs{\forall u\in\Lendo{E};\forall k\in\N}\Mat{u^k}=\Mat{u}^k.\]
\end{prop}

\subsection{Trace d'un endomorphisme}

\begin{defprop}
Soient \(E\) un \(\K\)-espace vectoriel de dimension finie et \(u\in\Lendo{E}\).

Les matrices de l'endomorphisme \(u\) (en prenant les mêmes bases \guillemets{au départ} et \guillemets{à l'arrivée}) ont toutes la même trace (car elles sont toutes semblables).

Leur trace commune est appelée trace de l'endomorphisme \(u\).

On a ainsi, pour toute base \(\fami{B}\) de \(E\) : \[\tr u=\tr\Mat{u}.\]
\end{defprop}

\begin{ex}
Soit \(E\) un espace vectoriel de dimension \(n\in\Ns\).

On a \(\tr\id{E}=n\).
\end{ex}

\begin{dem}
Soit \(\fami{B}\) une base de \(E\).

On a \(\Mat{\id{E}}=I_n\) donc on a : \[\begin{aligned}
\tr\id{E}&=\tr\Mat{\id{E}} \\
&=\tr I_n \\
&=n.
\end{aligned}\]
\end{dem}

\begin{prop}
Soit \(E\) un espace vectoriel de dimension \(n\in\Ns\).

L'application \(\tr:\Lendo{E}\to\K\) est une forme linéaire sur \(\Lendo{E}\).
\end{prop}

\begin{prop}
Soit \(E\) un espace vectoriel de dimension \(n\in\Ns\).

On a : \[\quantifs{\forall u,v\in\Lendo{E}}\tr uv=\tr vu.\]
\end{prop}

\begin{prop}
En dimension finie non-nulle, la trace d'un projecteur est égale à son rang.
\end{prop}

\begin{dem}
Soient \(E\) un espace vectoriel de dimension \(n\in\Ns\) et \(p\in\Lendo{E}\) un projecteur.

On sait que \(p\) est le projecteur sur \(\Im p\), parallèlement à \(\ker p\) et qu'on a \(E=\Im p\oplus\ker p\).

Posons \(r=\rg p\).

Soit \(\fami{B}\) une base de \(E\) adaptée à la décomposition de \(E\) en somme directe : \[\fami{B}=\paren{\underbrace{e_1,\dots,e_r}_{\text{base de }\Im p},\underbrace{e_{r+1},\dots,e_n}_{\text{base de }\ker p}}.\]

On a : \(\begin{dcases}
\quantifs{\forall j\in\interventierii{1}{r}}p\paren{e_j}=e_j\text{ car }e_j\in\Im p \\
\quantifs{\forall j\in\interventierii{r+1}{n}}p\paren{e_j}=0_E\text{ car }e_j\in\ker p
\end{dcases}\)

D'où : \[\Mat{p}=J_r\in\M{n}.\]

Donc \(\tr p=\tr J_r=r=\rg p\).
\end{dem}

\begin{exoex}
Soient \(E\) un espace vectoriel de dimension finie non-nulle et \(F\) et \(G\) deux sous-espaces vectoriels supplémentaires dans \(E\).

On note \(s\) la symétrie par rapport à \(F\) parallèlement à \(G\).

Exprimer la trace de \(s\) en fonction des dimensions de \(F\) et \(G\).
\end{exoex}

\begin{corr}
On pose \(a=\dim F\) et \(b=\dim G\).

Soit \(\fami{B}\) une base adaptée à la décomposition de \(E\) en somme directe : \[\fami{B}=\paren{\underbrace{e_1,\dots,e_a}_{\text{base de }F},\underbrace{e_1\prim,\dots,e_b\prim}_{\text{base de }G}}.\]

On a : \(\begin{dcases}
\quantifs{\forall j\in\interventierii{1}{a}}s\paren{e_j}=e_j \\
\quantifs{\forall j\in\interventierii{1}{b}}s\paren{e_j\prim}=-e_j\prim
\end{dcases}\)

D'où : \[\Mat{s}=\begin{pmatrix}
I_a & 0 \\
0 & -I_b
\end{pmatrix}\in\M{a+b}.\]

Donc \(\tr s=\tr\Mat{s}=a-b=\dim F-\dim G\).
\end{corr}

\chapter{Relations de comparaison, développements limités}

\minitoc

Dans tout le chapitre, on pose : \(\K=\R\) ou \(\C\).

\section{Relations de comparaison : cas des suites}

\subsection{Relation de domination \(\mathscr{O}\)}

\begin{defprop}
Soient \(\paren{u_n}_n,\paren{v_n}_n\in\K^\N\).

Les propositions suivantes sont équivalentes :

\begin{enumerate}
    \item \(\quantifs{\exists M\in\Rp;\exists N\in\N;\forall n\in\interventierie{N}{\pinf}}\abs{u_n}\leq M\abs{v_n}\) \\
    \item Il existe une suite \(\paren{\lambda_n}_n\in\K^\N\) telle que : \[\paren{\lambda_n}_n\text{ est bornée}\qquad\text{et}\qquad\quantifs{\exists N\in\N;\forall n\in\interventierie{N}{\pinf}}u_n=\lambda v_n.\]
\end{enumerate}

Lorsqu'elles sont vérifiées, on dit que \[\paren{v_n}_n\text{ domine }\paren{u_n}_n\] ou que \[v_n\text{ domine }u_n\text{ quand }n\text{ tend vers }\pinf\] et on note : \[u_n=\O{v_n}\text{ quand }n\text{ tend vers }\pinf\] ou : \[u_n\egqd{n\to\pinf}\O{v_n}.\]
\end{defprop}

\begin{prop}
Soient \(\paren{u_n}_n,\paren{v_n}_n\in\K^\N\).

Si les termes de \(\paren{v_n}_n\) sont non-nuls à partir d'un certain rang \(N\in\N\) : \[\quantifs{\forall n\in\interventierie{N}{\pinf}}v_n\not=0,\] alors : \[u_n\egqd{n\to\pinf}\O{v_n}\ssi\text{la suite }\paren{\dfrac{u_n}{v_n}}_{n\geq N}\text{ est bornée}.\]

Si les termes de \(\paren{v_n}_n\) sont tous non-nuls : \[\quantifs{\forall n\in\N}v_n\not=0,\] alors : \[u_n\egqd{n\to\pinf}\O{v_n}\ssi\text{la suite }\paren{\dfrac{u_n}{v_n}}_{n\in\N}\text{ est bornée}.\]
\end{prop}

\begin{ex}
Soient \(\alpha,\beta\in\R\). On a : \[n^{\alpha}\egqd{n\to\pinf}\O{n^\beta}\ssi\alpha\leq\beta.\]

Soient \(a,b\in\Rps\). On a : \[a^n\egqd{n\to\pinf}\O{b^n}\ssi a\leq b.\]

Soit \(\paren{u_n}_n\in\K^\N\). On a : \[u_n\egqd{n\to\pinf}\O{0}\ssi\paren{u_n}_n\text{ est nulle à partir d'un certain rang}\] et : \[u_n\egqd{n\to\pinf}\O{1}\ssi\paren{u_n}_n\text{ est bornée}.\]
\end{ex}

\subsection{Relation de négligeabilité \(o\)}

\begin{defprop}
Soient \(\paren{u_n}_n,\paren{v_n}_n\in\K^\N\).

Les propositions suivantes sont équivalentes :

\begin{enumerate}
    \item \(\quantifs{\forall\epsilon\in\Rps;\exists N\in\N;\forall n\in\interventierie{N}{\pinf}}\abs{u_n}\leq\epsilon\abs{v_n}\) \\
    \item Il existe une suite \(\paren{\epsilon_n}_n\in\K^\N\) telle que : \[\lim_{n\to\pinf}\epsilon_n=0\qquad\text{et}\qquad\quantifs{\exists N\in\N;\forall n\in\interventierie{N}{\pinf}}u_n=\epsilon_nv_n.\]
\end{enumerate}

Lorsqu'elles sont vérifiées, on dit que \[\paren{u_n}_n\text{ est négligeable devant }\paren{v_n}_n\] ou que \[u_n\text{ est négligeable devant }v_n\text{ quand }n\text{ tend vers }\pinf\] et on note : \[u_n=\o{v_n}\text{ quand }n\text{ tend vers }\pinf\] ou : \[u_n\egqd{n\to\pinf}\o{v_n}.\]
\end{defprop}

\begin{prop}
Soient \(\paren{u_n}_n,\paren{v_n}_n\in\K^\N\).

Si les termes de \(\paren{v_n}_n\) sont non-nuls à partir d'un certain rang : \[\quantifs{\exists N\in\N;\forall n\in\interventierie{N}{\pinf}}v_n\not=0,\] alors : \[u_n\egqd{n\to\pinf}\o{v_n}\ssi\lim_{n\to\pinf}\dfrac{u_n}{v_n}=0.\]
\end{prop}

\begin{ex}
Soient \(\alpha,\beta\in\R\). On a : \[n^\alpha\egqd{n\to\pinf}\o{n^\beta}\ssi\alpha<\beta.\]

Soient \(a,b\in\Rps\). On a : \[a^n\egqd{n\to\pinf}\o{b^n}\ssi a<b.\]

Soit \(\paren{u_n}_n\in\K^\N\). On a : \[u_n\egqd{n\to\pinf}\o{0}\ssi\paren{u_n}_n\text{ est nulle à partir d'un certain rang}\] et : \[u_n\egqd{n\to\pinf}\o{1}\ssi\lim_{n\to\pinf}u_n=0.\]
\end{ex}

\subsection{Propriétés de \(\mathscr{O}\) et \(o\)}

\begin{prop}
Soient \(\paren{u_n}_n,\paren{v_n}_n\in\K^\N\).

On a : \[u_n\egqd{n\to\pinf}\o{v_n}\imp u_n\egqd{n\to\pinf}\O{v_n}.\]
\end{prop}

\begin{prop}
Soient \(\paren{u_n}_n,\paren{v_n}_n\in\K^\N\) et \(l,l\prim\in\intervii{0}{\pinf}\) tels que : \[\lim_{n\to\pinf}\abs{u_n}=l\qquad\text{et}\qquad\lim_{n\to\pinf}\abs{v_n}=l\prim.\]

On a, quand \(n\) tend vers \(\pinf\) :

\begin{enumerate}
    \item Si \(l=0\) et \(l\prim\not=0\) alors \(u_n=\o{v_n}\). \\
    \item Si \(l\in\intervie{0}{\pinf}\) et \(l\prim=\pinf\) alors \(u_n=\o{v_n}\). \\
    \item Si \(l,l\prim\in\intervee{0}{\pinf}\) alors \(u_n=\O{v_n}\). \\
    \item Si \(l=l\prim=0\) ou \(l=l\prim=\pinf\) alors on ne peut rien dire.
\end{enumerate}
\end{prop}

\begin{prop}[Transitivités]
Soient \(\paren{u_n}_n,\paren{v_n}_n,\paren{w_n}_n\in\K^\N\).

On a, quand \(n\) tend vers \(\pinf\) :

\begin{enumerate}
    \item Si \(\begin{dcases}
        u_n=\O{v_n} \\
        v_n=\O{w_n}
    \end{dcases}\) alors \(u_n=\O{w_n}\). \\\\ Autrement dit : la relation de domination est transitive. \\
    \item Si \(\begin{dcases}
        u_n=\O{v_n} \\
        v_n=\o{w_n}
    \end{dcases}\) ou \(\begin{dcases}
        u_n=\o{v_n} \\
        v_n=\O{w_n}
    \end{dcases}\) alors \(u_n=\o{w_n}\). \\
    \item En particulier, la relation de négligeabilité est transitive.
\end{enumerate}
\end{prop}

\begin{prop}[Sommes]
Soient \(\paren{u_n}_n,\paren{v_n}_n,\paren{w_n}_n\in\K^\N\).

On a, quand \(n\) tend vers \(\pinf\) :

\begin{enumerate}
    \item Si \(\begin{dcases}
        u_n=\O{w_n} \\
        v_n=\O{w_n}
    \end{dcases}\) alors \(u_n+v_n=\O{w_n}\). \\
    \item Si \(\begin{dcases}
        u_n=\o{w_n} \\
        v_n=\o{w_n}
    \end{dcases}\) alors \(u_n+v_n=\o{w_n}\).
\end{enumerate}
\end{prop}

\begin{prop}[Produits]
Soient \(\paren{a_n}_n,\paren{b_n}_n,\paren{c_n}_n,\paren{d_n}_n\in\K^\N\).

On a, quand \(n\) tend vers \(\pinf\) :

\begin{enumerate}
    \item Si \(\begin{dcases}
        a_n=\O{b_n} \\
        c_n=\O{d_n}
    \end{dcases}\) alors \(a_nc_n=\O{b_nd_n}\). \\
    \item En particulier : si \(a_n=\O{b_n}\) alors \(a_nc_n=\O{b_nc_n}\). \\
    \item Si \(\begin{dcases}
        a_n=\o{b_n} \\
        c_n=\O{d_n}
    \end{dcases}\) alors \(a_nc_n=\o{b_nd_n}\). \\
    \item En particulier : si \(a_n=\o{b_n}\) alors \(a_nc_n=\o{b_nc_n}\).
\end{enumerate}
\end{prop}

\begin{prop}[Puissances]
Soient \(\paren{u_n}_n,\paren{v_n}_n\in\paren{\Rps}^\N\) et \(\alpha\in\R\).

On a, quand \(n\) tend vers \(\pinf\) :

\begin{enumerate}
    \item Si \(\begin{dcases}
        u_n=\O{v_n} \\
        \alpha\geq0
    \end{dcases}\) alors \(u_n^\alpha=\O{v_n^\alpha}\). \\
    \item Si \(\begin{dcases}
        u_n=\O{v_n} \\
        \alpha\leq0
    \end{dcases}\) alors \(v_n^\alpha=\O{u_n^\alpha}\). \\
    \item Si \(\begin{dcases}
        u_n=\o{v_n} \\
        \alpha>0
    \end{dcases}\) alors \(u_n^\alpha=\o{v_n^\alpha}\). \\
    \item Si \(\begin{dcases}
        u_n=\o{v_n} \\
        \alpha<0
    \end{dcases}\) alors \(v_n^\alpha=\o{u_n^\alpha}\).
\end{enumerate}

On retient en particulier : \[u_n=\O{v_n}\imp\dfrac{1}{v_n}=\O{\dfrac{1}{u_n}}\qquad\text{et}\qquad u_n=\o{v_n}\imp\dfrac{1}{v_n}=\o{\dfrac{1}{u_n}}.\]

La même proposition est vraie en prenant \(\paren{u_n}_n,\paren{v_n}_n\in\C^\N\) et \(\alpha\in\N\) ou en prenant \(\paren{u_n}_n,\paren{v_n}_n\in\paren{\Cs}^\N\) et \(\alpha\in\Z\).
\end{prop}

\begin{prop}[Suites extraites]
Soient \(\paren{u_n}_n,\paren{v_n}_n\in\C^\N\) et \(\phi:\N\to\N\) une fonction strictement croissante.

On a, quand \(n\) tend vers \(\pinf\) : \[u_n=\O{v_n}\imp u_{\phi\paren{n}}=\O{v_{\phi\paren{n}}}\qquad\text{et}\qquad u_n=\o{v_n}\imp u_{\phi\paren{n}}=\o{v_{\phi\paren{n}}}.\]
\end{prop}

\subsection{Croissances comparées}

\begin{rem}[Culturelle]
Il existe d'autres notations que les \guillemets{notations de Landau} \(\mathscr{O}\), \(o\) et \(\sim\) qu'on utilise en mathématiques de CPGE.

Soient \(\paren{u_n}_n,\paren{v_n}_n\in\K^\N\).

\begin{itemize}
    \item \guillemets{Notations de Hardy} (peu utilisées mais très commodes pour énoncer la proposition suivante) : \[u_n\prec v_n\ssi u_n=\o{v_n}\] et : \[u_n\preccurlyeq v_n\ssi u_n=\O{v_n}.\]
    \item \guillemets{Notation de Vinogradov} : \[u_n\ll v_n\ssi u_n=\O{v_n}.\]
    \item \guillemets{Notation \(\Theta\)} : \[\begin{aligned}
        u_n=\Theta\paren{v_n}&\ssi\croch{u_n=\O{v_n}\quad\text{et}\quad v_n=\O{u_n}} \\
        &\ssi\quantifs{\exists M_1,M_2\in\Rps;\exists N\in\N;\forall n\in\interventierie{N}{\pinf}}M_1\abs{v_n}\leq\abs{u_n}\leq M_2\abs{v_n}.
    \end{aligned}\] Cette notation est surtout utilisée en informatique (pour étudier la complexité des algorithmes). On verra qu'on a : \[u_n=\Theta\paren{v_n}\imp u_n\sim v_n.\]
\end{itemize}
\end{rem}

\begin{prop}[Croissances comparées]\thlabel{prop:croissancesComparéesSuites}
On utilise exceptionnellement les notations de Hardy pour gagner en concision.

Soient \(a_1,a_2,\alpha_1,\alpha_2,\beta_1,\beta_2\in\R\) tels que : \[0<a_1<1<a_2\qquad\text{et}\qquad\alpha_1<0<\alpha_2\qquad\text{et}\qquad\beta_1<0<\beta_2.\]

Alors on a, quand \(n\) tend vers \(\pinf\) : \[\underbrace{a_1^n\quad\prec\quad n^{\alpha_1}\quad\prec\quad\ln^{\beta_1}n}_{\text{suites de limite nulle}}\quad\prec\quad1\quad\prec\quad\underbrace{\ln^{\beta_2}n\quad\prec\quad n^{\alpha_2}\quad\prec\quad a_2^n\quad\prec\quad n!}_{\text{suites de limite }\pinf}\]
\end{prop}

\begin{dem}
Montrons que \(a^n\egqd{n\to\pinf}\o{n!}\) avec \(a\in\Rps\), \cad \(\lim_{n\to\pinf}\dfrac{a^n}{n!}=0\).

On a : \[\quantifs{\forall n\in\N}a\times\dfrac{a}{2}\times\dots\times\dfrac{a}{n}=\dfrac{a^n}{n!}.\]

Soit \(N\in\Ns\) tel que \(\dfrac{a}{N}\leq\dfrac{1}{2}\) (un tel \(N\) existe car \(\lim_{n\to\pinf}\dfrac{a}{n}=0\)).

Posons \(M=\dfrac{a^{N-1}}{\paren{N-1}!}>0\).

On a : \[\quantifs{\forall n\in\interventierie{N}{\pinf}}\dfrac{a^n}{n!}=\underbrace{a\times\dots\times\dfrac{a}{N-1}}_{=M}\times\underbrace{\dfrac{a}{N}}_{\leq\frac{1}{2}}\times\dots\times\underbrace{\dfrac{a}{n}}_{\leq\frac{1}{2}}.\]

Donc : \[\quantifs{\forall n\in\interventierie{N}{\pinf}}0\leq\dfrac{a^n}{n!}\leq M\dfrac{1}{2^{n-N+1}}.\]

Or \(\lim_{n\to\pinf}\dfrac{M}{2^{n-N+1}}=0\).

Donc selon le théorème des gendarmes, on a \(\lim_{n\to\pinf}\dfrac{a^n}{n!}=0\) donc : \[a^n\egqd{n\to\pinf}\o{n!}.\]
\end{dem}

\begin{dem}
Soient \(a\in\intervee{1}{\pinf}\) et \(\alpha\in\R\).

Montrons que \(n^\alpha\egqd{n\to\pinf}\o{a^n}\), \cad \(\lim_{n\to\pinf}\dfrac{n^\alpha}{a^n}=0\).

Posons \(\fonction{f}{\Rps}{\R}{x}{\dfrac{x^\alpha}{a^x}=x^\alpha\e{-x\ln a}}\)

On a : \[\begin{aligned}
\quantifs{\forall x\in\Rps}f\prim\paren{x}&=\alpha x^{\alpha-1}a^{-x}-x^\alpha a^{\alpha x}\ln a \\
&=x^{\alpha-1}a^{-x}\paren{\alpha-x\ln a}.
\end{aligned}\]

On a \(\quantifs{\forall x\in\Rps}f\prim\paren{x}\geq0\ssi x\leq\dfrac{\alpha}{\ln a}\) donc \(f\) est décroissante sur \(\intervie{\dfrac{\alpha}{\ln a}}{\pinf}\).

Donc \(f\) admet une limite en \(\pinf\).

De plus, \(f\) est minorée par \(0\) donc \(l=\lim_{\pinf}f\in\Rps\).

De plus : \[\quantifs{\forall x\in\Rps}f\paren{2x}=\dfrac{2^\alpha}{a^x}f\paren{x}.\]

D'où, en passant à la limite quand \(x\to\pinf\) : \[l=0\times l=0\text{ car }l\text{ est finie}.\]

Donc \(n^\alpha\egqd{n\to\pinf}\o{a^n}\).
\end{dem}

\begin{dem}
Soient \(\alpha\in\Rps\) et \(\beta\in\R\).

Montrons que \(\ln^\beta n\egqd{n\to\pinf}\o{n^\alpha}\), \cad \(\lim_{n\to\pinf}\dfrac{\ln^\beta n}{n^\alpha}=0\).

Posons \(\fonction{f}{\intervee{1}{\pinf}}{\R}{x}{\dfrac{\ln^\beta x}{x^\alpha}=x^{-\alpha}\ln^\beta x}\)

On a : \[\begin{aligned}
\quantifs{\forall x\in\intervee{1}{\pinf}}f\prim\paren{x}&=\dfrac{\beta}{x}x^{-\alpha}\ln^{\beta-1}x-\alpha x^{-\alpha-1}\ln^\beta x \\
&=x^{-\alpha-1}\ln^{\beta-1}x\paren{\beta-\alpha\ln x}.
\end{aligned}\]

On a : \[\begin{aligned}
\quantifs{\forall x\in\intervee{1}{\pinf}}f\prim\paren{x}\geq0&\ssi\alpha\ln x\leq\beta \\
&\ssi x\leq\e{\nicefrac{\beta}{\alpha}}.
\end{aligned}\]

Donc \(f\) est décroissante sur \(\intervie{\e{\nicefrac{\beta}{\alpha}}}{\pinf}\).

Donc \(f\) admet une limite en \(\pinf\).

De plus, \(f\) est minorée par \(0\) donc \(l=\lim_{\pinf}f\in\Rp\).

De plus, on a \(\quantifs{\forall x\in\intervee{1}{\pinf}}f\paren{x^2}=\dfrac{2^\beta}{x^\alpha}f\paren{x}\).

Donc en passant à la limite quand \(x\to\pinf\) : \[l=0\times l=0\text{ car }l\text{ est finie}.\]

D'où \(\ln^\beta n\egqd{n\to\pinf}\o{n^{\alpha}}\).
\end{dem}

\begin{dem}
Soit \(\beta\in\Rps\).

On a \(\lim_n\ln^\beta n=\pinf\) donc \(1\egqd{n\to\pinf}\o{\ln^\beta n}\) car \(\lim_n\dfrac{1}{\ln^\beta n}=0\).
\end{dem}

\begin{dem}
Soit \(\beta\in\Rms\).

On a \(-\beta>0\) donc \(1\egqd{n\to\pinf}\o{\ln^{-\beta}n}\).

Donc \(\ln^\beta n\egqd{n\to\pinf}\o{1}\).
\end{dem}

\begin{dem}
Soient \(\alpha,\beta\in\Rms\).

On a \(-\alpha>0\) et \(-\beta>0\).

Donc selon ce qui précède, on a, quand \(n\to\pinf\) : \(\ln^{-\beta}n=\o{n^{-\alpha}}\).

D'où : \[n^\alpha\egqd{n\to\pinf}\o{\ln^\beta n}.\]
\end{dem}

\begin{dem}
Soient \(a\in\intervee{0}{1}\) et \(\alpha\in\Rms\).

De même, \(a^n\egqd{n\to\pinf}\o{n^\alpha}\) découle de ce qui précède.
\end{dem}

\subsection{Relation d'équivalence \(\sim\)}

\begin{defprop}
Soient \(\paren{u_n}_n,\paren{v_n}_n\in\K^\N\).

Les propositions suivantes sont équivalentes :

\begin{enumerate}
    \item \(v_n\egqd{n\to\pinf}u_n+\o{u_n}\) \\
    \item \(u_n\egqd{n\to\pinf}v_n+\o{v_n}\) \\
    \item Il existe une suite \(\paren{\lambda_n}_n\in\K^\N\) telle que : \[\lim_{n\to\pinf}\lambda_n=1\qquad\text{et}\qquad\quantifs{\exists N\in\N;\forall n\in\interventierie{N}{\pinf}}u_n=\lambda_nv_n\]
    \item Il existe une suite \(\paren{\mu_n}_n\in\K^\N\) telle que : \[\lim_{n\to\pinf}\mu_n=1\qquad\text{et}\qquad\quantifs{\exists N\in\N;\forall n\in\interventierie{N}{\pinf}}v_n=\mu_nu_n.\]
\end{enumerate}

Lorsqu'elles sont vérifiées, on dit que \[\paren{u_n}_n\text{ est équivalente à }\paren{v_n}_n\] ou que \[\begin{aligned}
u_n\text{ est équivalent à }v_n\text{ quand }n\text{ tend vers }\pinf \\
u_n\text{ est un équivalent de }v_n\text{ quand }n\text{ tend vers }\pinf
\end{aligned}\] et on note : \[u_n\sim v_n\text{ quand }n\text{ tend vers }\pinf\] ou : \[u_n\simqd{n\to\pinf}v_n.\]
\end{defprop}

\begin{dem}[(2) \(\ssi\) (3)]
On a : \[\begin{aligned}
u_n\egqd{n\to\pinf}v_n+\o{v_n}&\ssi\quantifs{\exists\paren{\epsilon_n}_n\in\K^\N}\begin{dcases}
\quantifs{\exists N\in\N;\forall n\in\interventierie{N}{\pinf}}u_n=v_n+\epsilon_nv_n \\
\lim_n\epsilon_n=0
\end{dcases} \\
&\ssi\quantifs{\exists\paren{\epsilon_n}_n\in\K^\N}\begin{dcases}
\quantifs{\exists N\in\N;\forall n\in\interventierie{N}{\pinf}}u_n=\paren{1+\epsilon_n}v_n \\
\lim_n\epsilon_n=0
\end{dcases} \\
&\ssi\quantifs{\exists\paren{\lambda_n}_n\in\K^\N}\begin{dcases}
\quantifs{\exists N\in\N;\forall n\in\interventierie{N}{\pinf}}u_n=\lambda_nv_n \\
\lim_n\lambda_n=1
\end{dcases}
\end{aligned}\]
\end{dem}

\begin{dem}[(1) \(\ssi\) (4)]
Idem.
\end{dem}

\begin{dem}[(3) \(\ssi\) (4)]~\\
Claire en prenant \(\mu_n=\dfrac{1}{\lambda_n}\) à partir d'un certain rang.
\end{dem}

\begin{prop}
Soient \(\paren{u_n}_n,\paren{v_n}_n\in\K^\N\).

Si les termes de \(\paren{v_n}_n\) sont non-nuls à partir d'un certain rang : \[\quantifs{\exists N\in\N;\forall n\in\interventierie{N}{\pinf}}v_n\not=0,\] alors : \[u_n\simqd{n\to\pinf}v_n\ssi\lim_{n\to\pinf}\dfrac{u_n}{v_n}=1.\]
\end{prop}

\begin{ex}
Soient \(\alpha,\beta\in\R\). On a : \[n^\alpha\simqd{n\to\pinf}n^\beta\ssi\alpha=\beta.\]

Soient \(a,b\in\Rps\). On a : \[a^n\simqd{n\to\pinf}b^n\ssi a=b.\]

Soient \(d\in\N\), \(a_0,\dots,a_d\in\C\) tels que \(a_d\not=0\) et \(P=a_dX^d+\dots+a_0X^0\in\poly[\C]\). On a : \[P\paren{n}\simqd{n\to\pinf}a_dn^d.\]

Soit \(\paren{u_n}_n\in\K^\N\). On a : \[u_n\simqd{n\to\pinf}0\ssi\paren{u_n}_n\text{ est nulle à partir d'un certain rang}\] et : \[\quantifs{\forall l\in\K\excluant\accol{0}}u_n\simqd{n\to\pinf}l\ssi\lim_{n\to\pinf}u_n=l.\]

NB : l'équivalence précédente est valable pour toute limite finie non-nulle. En particulier, deux suites admettant une même limite finie non-nulle sont équivalentes.
\end{ex}

\subsection{Propriétés de \(\sim\)}

\begin{prop}
La relation \(\sim\) est une relation d'équivalence sur \(\K^\N\).
\end{prop}

\begin{prop}
Soient \(\paren{u_n}_n,\paren{v_n}_n\in\K^\N\).

On a : \[u_n\simqd{n\to\pinf}v_n\imp\begin{dcases}
u_n\egqd{n\to\pinf}\O{v_n} \\
v_n\egqd{n\to\pinf}\O{u_n}
\end{dcases}\]
\end{prop}

\begin{prop}[Cas réel : limite et signe]
Soient \(\paren{u_n}_n,\paren{v_n}_n\in\R^\N\) et \(l\in\Rb\).

Si \(\begin{dcases}
\lim_{n\to\pinf}u_n=l \\
u_n\simqd{n\to\pinf}v_n
\end{dcases}\) alors \(\lim_{n\to\pinf}v_n=l\).

Si \(u_n\simqd{n\to\pinf}v_n\) alors \(\paren{u_n}_n\) et \(\paren{v_n}_n\) sont de même signe au sens strict à partir d'un certain rang : \[\quantifs{\exists N\in\N;\forall n\in\interventierie{N}{\pinf}}\sg u_n=\sg v_n.\]
\end{prop}

\begin{dem}
On suppose \(u_n\simqd{n\to\pinf}v_n\).

Soit \(\paren{\lambda_n}_n\in\R^\N\) telle que \(\begin{dcases}
\quantifs{\exists N\in\N;\forall n\in\interventierie{N}{\pinf}}v_n=\lambda_nu_n \\
\lim_{n\to\pinf}\lambda_n=1
\end{dcases}\)

Supposons \(\lim_nu_n=l\). Alors \(\lim_n\lambda_nu_n=l\) donc \(\lim_nv_n=l\).

Comme \(\lim_n\lambda_n=1\), il existe \(N\in\N\) tel que \(\quantifs{\forall n\in\interventierie{N}{\pinf}}\begin{dcases}
\lambda_n\geq\dfrac{1}{2} \\
v_n=\lambda_nu_n
\end{dcases}\) donc \[\quantifs{\forall n\in\interventierie{N}{\pinf}}\sg v_n=\sg u_n.\]
\end{dem}

\begin{prop}[Cas complexe : limite]
Soient \(\paren{u_n}_n,\paren{v_n}_n\in\C^\N\) et \(l\in\C\).

Si \(\begin{dcases}
\lim_{n\to\pinf}u_n=l \\
u_n\simqd{n\to\pinf}v_n
\end{dcases}\) alors \(\lim_{n\to\pinf}v_n=l\).
\end{prop}

\begin{prop}[Produits]
Soient \(\paren{a_n}_n,\paren{b_n}_n,\paren{c_n}_n,\paren{d_n}_n\in\K^\N\).

On a, quand \(n\) tend vers \(\pinf\) : \[\begin{dcases}
a_n\sim b_n \\
c_n\sim d_n
\end{dcases}\imp a_nc_n\sim b_nd_n.\]
\end{prop}

\begin{prop}[Puissances]
Soient \(\paren{u_n}_n,\paren{v_n}_n\in\paren{\Rps}^\N\) et \(\alpha\in\R\).

On a, quand \(n\) tend vers \(\pinf\) : \[u_n\sim v_n\imp u_n^\alpha\sim v_n^\alpha.\]

On retient en particulier : \[u_n\sim v_n\imp\dfrac{1}{u_n}\sim\dfrac{1}{v_n}.\]

La même proposition est vraie en prenant \(\paren{u_n}_n,\paren{v_n}_n\in\C^\N\) et \(\alpha\in\N\) ou en prenant \(\paren{u_n}_n,\paren{v_n}_n\in\paren{\Cs}^\N\) et \(\alpha\in\Z\).
\end{prop}

\begin{prop}[Suites extraites]
Soient \(\paren{u_n}_n,\paren{v_n}_n\in\C^\N\) et \(\phi:\N\to\N\) une fonction strictement croissante.

On a : \[u_n\simqd{n\to\pinf}\imp u_{\phi\paren{n}}\simqd{n\to\pinf}v_{\phi\paren{n}}.\]
\end{prop}

\begin{prop}[Sommes]
On ne fait pas de sommes d'équivalents.
\end{prop}

\begin{dem}
Quand \(n\) tend vers \(\pinf\) :

On a \(\begin{dcases}
1+\dfrac{1}{n}\sim1 \\
-1+\dfrac{1}{n}\sim-1
\end{dcases}\) mais on n'a pas \(\dfrac{2}{n}\sim0\).
\end{dem}

\begin{prop}
Soient \(\paren{a_n}_n,\paren{b_n}_n,\paren{c_n}_n,\paren{d_n}_n\in\R^\N\).

On suppose \(\begin{dcases}
\quantifs{\forall n\in\N}a_n\leq b_n\leq c_n \\
\text{quand }n\to\pinf\text{, }a_n\sim c_n\sim d_n
\end{dcases}\)

Alors : \[b_n\simqd{n\to\pinf}d_n.\]
\end{prop}

\begin{dem}
Soient \(\paren{\lambda_n}_n,\paren{\mu_n}_n\in\R^\N\) telles que \(\begin{dcases}
\lim_n\lambda_n=\lim_n\mu_n=1 \\
\quantifs{\exists N\in\N;\forall n\in\interventierie{N}{\pinf}}\begin{dcases}
    a_n=\lambda_nd_n \\
    c_n=\mu_nd_n
\end{dcases}
\end{dcases}\)

Soit un tel \(N\in\N\).

On a : \[\quantifs{\forall n\in\interventierie{N}{\pinf}}\begin{dcases}
\lambda_n=\dfrac{a_n}{d_n}\leq\dfrac{b_n}{d_n}\leq\dfrac{c_n}{d_n}=\mu_n &\text{si }d_n>0 \\
\mu_n\leq\dfrac{c_n}{d_n}\leq\dfrac{b_n}{d_n}\leq\dfrac{a_n}{d_n}=\lambda_n &\text{si }d_n<0 \\
b_n=0 &\text{si }d_n=0
\end{dcases}\]

Posons \(\quantifs{\forall n\in\N}\nu_n=\begin{dcases}
1 &\text{si }d_n=0 \\
\dfrac{b_n}{d_n} &\text{si }d_n\not=0
\end{dcases}\)

On a : \[\quantifs{\forall n\in\interventierie{N}{\pinf}}b_n=\nu_nd_n\] et : \[\quantifs{\forall n\in\interventierie{N}{\pinf}}\min\accol{\lambda_n;\mu_n}\leq\nu_n\leq\max\accol{\lambda_n;\mu_n}\qquad\text{ou}\qquad\nu_n=1.\]

Donc \(\lim_n\nu_n=1\) donc \(b_n\simqd{n\to\pinf}d_n\).
\end{dem}

\begin{ex}
Soit \(\paren{u_n}_n\in\K^\N\) telle que \(\quantifs{\forall n\in\N}n-1\leq u_n\leq n+2\).

On a, quand \(n\to\pinf\) : \(n-1\sim n+2\sim n\).

Donc \(u_n\simqd{n\to\pinf} n\).
\end{ex}

\subsection{Formule de Stirling}

\begin{theo}[Formule de Stirling]
On a : \[n!\simqd{n\to\pinf}\sqrt{2\pi n}\paren{\dfrac{n}{\e{}}}^n.\]
\end{theo}

\begin{dem}
\note{Admise}
\end{dem}

\section{Relations de comparaison : cas général}

On va maintenant étendre ce qui précède au cas des fonctions définies sur un ensemble \(A\subset\R\), à valeurs dans \(\K=\R\) ou \(\C\) et dont on étudie le comportement au voisinage d'un point \(a\in\Rb\).

Pour cela, on suppose qu'il existe des points où la fonction \(f\) est définie et qui sont arbitrairement proches de \(a\) (autrement, étudier le comportement de \(f\) au voisinage de \(a\) n'a pas de sens).

Précisément, on supposera que tout voisinage de \(a\) dans \(\R\) rencontre \(A\) : \[\quantifs{\forall V\in\V{a}}V\inter A\not=\ensvide.\]

\begin{itemize}
    \item Si \(a=\pinf\), cela signifie que \(A\) est une partie de \(\R\) non-majorée. \\
    \item Si \(a=\minf\), cela signifie que \(A\) est une partie de \(\R\) non-minorée. \\
    \item Si \(a\in\R\), cela signifie : \(\quantifs{\forall\epsilon\in\Rps;\exists a\prim\in A}\abs{a\prim-a}\leq\epsilon\).
\end{itemize}

On fixe dans la suite une telle partie \(A\) et un tel point \(a\in\Rb\).

\subsection{Relation de domination \(\mathscr{O}\)}

\begin{defprop}
Soient \(f,g\in\F{A}{\K}\).

Les propositions suivantes sont équivalentes :

\begin{enumerate}
    \item \(\quantifs{\exists M\in\Rp;\exists V\in\V{a};\forall t\in V}\abs{f\paren{t}}\leq M\abs{g\paren{t}}\) \\
    \item Il existe une fonction \(\lambda\in\F{A}{\K}\) telle que : \[\lambda\text{ est bornée}\qquad\text{et}\qquad\quantifs{\exists V\in\V{a};\forall t\in V}f\paren{t}=\lambda\paren{t}g\paren{t}.\]
\end{enumerate}

Lorsqu'elles sont vérifiées, on dit que \[g\text{ domine }f\] ou que \[g\paren{t}\text{ domine }f\paren{t}\text{ quand }t\text{ tend vers }a\] et on note : \[f\paren{t}=\O{g\paren{t}}\text{ quand }t\text{ tend vers }a\] ou : \[f\paren{t}\egqd{t\to a}\O{g\paren{t}}.\]
\end{defprop}

\begin{prop}
Soient \(f,g\in\F{A}{\K}\).

Si \(g\) ne s'annule pas sur un voisinage \(V\) de \(a\) : \[\quantifs{\forall t\in V}g\paren{t}\not=0,\] alors : \[f\paren{t}\egqd{t\to a}\O{g\paren{t}}\ssi\dfrac{f}{g}\text{ est bornée sur }V.\]

Si \(g\) ne s'annule pas : \[\quantifs{\forall t\in A}g\paren{t}\not=0,\] alors : \[f\paren{t}\egqd{t\to a}\O{g\paren{t}}\ssi\dfrac{f}{g}\text{ est bornée sur un voisinage de }a.\]
\end{prop}

\begin{ex}
Soient \(\alpha,\beta\in\R\). On a, quand \(t\) tend vers \(\pinf\) : \[t^\alpha=\O{t^\beta}\ssi\alpha\leq\beta.\]

Soit \(f\in\F{A}{\K}\). On a, quand \(t\) tend vers \(a\) : \[\begin{aligned}
f\paren{t}=\O{0}&\ssi f\text{ est nulle sur un voisinage de }a \\
&\ssi\quantifs{\exists V\in\V{a};\forall t\in V\inter A}f\paren{t}=0
\end{aligned}\] et : \[f\paren{t}=\O{1}\ssi f\text{ est bornée sur un voisinage de }a,\] \cad : \[f\paren{t}=\O{1}\ssi\quantifs{\exists V\in\V{a};\exists M\in\Rp;\forall t\in V\inter A}\abs{f\paren{t}}\leq M.\]
\end{ex}

\subsection{Relation de négligeabilité \(o\)}

\begin{defprop}
Soient \(f,g\in\F{A}{\K}\).

Les propositions suivantes sont équivalentes :

\begin{enumerate}
    \item \(\quantifs{\forall\epsilon\in\Rps;\exists V\in\V{a};\forall t\in V}\abs{f\paren{t}}\leq\epsilon\abs{g\paren{t}}\) \\
    \item Il existe une fonction \(\epsilon\in\F{A}{\K}\) telle que : \[\lim_{t\to a}\epsilon\paren{t}=0\qquad\text{et}\qquad\quantifs{\exists V\in\V{a};\forall t\in V}f\paren{t}=\epsilon\paren{t}g\paren{t}.\]
\end{enumerate}

Lorsqu'elles sont vérifiées, on dit que \[f\text{ est négligeable devant }g\] ou que \[f\paren{t}\text{ est négligeable devant }g\paren{t}\text{ quand }t\text{ tend vers }a\] et on note : \[f\paren{t}=\o{g\paren{t}}\text{ quand }t\text{ tend vers }a\] ou : \[f\paren{t}\egqd{t\to a}\o{g\paren{t}}.\]
\end{defprop}

\begin{prop}
Soient \(f,g\in\F{A}{\K}\).

Si \(g\) ne s'annule pas au voisinage de \(a\) : \[\quantifs{\exists V\in\V{a};\forall t\in V}g\paren{t}\not=0,\] alors : \[f\paren{t}\egqd{t\to a}\o{g\paren{t}}\ssi\lim_{t\to a}\dfrac{f\paren{t}}{g\paren{t}}=0.\]
\end{prop}

\begin{ex}
Soient \(\alpha,\beta\in\R\). On a, quand \(t\) tend vers \(a\) : \[t^\alpha=\o{t^\beta}\ssi\alpha<\beta.\]

Soit \(f\in\F{A}{\K}\). On a : \[f\paren{t}\egqd{t\to a}\o{0}\ssi f\text{ est nulle sur un voisinage de }a\] et : \[f\paren{t}\egqd{t\to a}\o{1}\ssi\lim_{t\to a}f\paren{t}=0.\]
\end{ex}

\subsection{Propriétés de \(\mathscr{O}\) et \(o\)}

\begin{prop}
Soient \(f,g\in\F{A}{\K}\).

On a : \[f\paren{t}\egqd{t\to a}\o{g\paren{t}}\imp f\paren{t}\egqd{t\to a}\O{g\paren{t}}.\]
\end{prop}

\begin{prop}
Soient \(f,g\in\F{A}{\K}\) et \(l,l\prim\in\intervii{0}{\pinf}\) tels que : \[\lim_{t\to a}\abs{f\paren{t}}=l\qquad\text{et}\qquad\lim_{t\to a}\abs{g\paren{t}}=l\prim.\]

On a, quand \(t\) tend vers \(a\) :

\begin{enumerate}
    \item Si \(l=0\) et \(l\prim\not=0\) alors \(f\paren{t}=\o{g\paren{t}}\) \\
    \item Si \(l\in\intervie{0}{\pinf}\) et \(l\prim=\pinf\) alors \(f\paren{t}=\o{g\paren{t}}\) \\
    \item Si \(l,l\prim\in\intervee{0}{\pinf}\) alors \(f\paren{t}=\O{g\paren{t}}\) \\
    \item Si \(l=l\prim=0\) ou \(l=l\prim=\pinf\), on ne peut rien dire.
\end{enumerate}
\end{prop}

\begin{prop}[Transitivités]
Soient \(f,g,h\in\F{A}{\K}\).

On a, quand \(t\) tend vers \(a\) :

\begin{enumerate}
    \item Si \(\begin{dcases}
        f\paren{t}=\O{g\paren{t}} \\
        g\paren{t}=\O{h\paren{t}}
    \end{dcases}\) alors \(f\paren{t}=\O{h\paren{t}}\). \\\\ Autrement dit : la relation de domination est transitive. \\
    \item Si \(\begin{dcases}
        f\paren{t}=\O{g\paren{t}} \\
        g\paren{t}=\o{h\paren{t}}
    \end{dcases}\) ou \(\begin{dcases}
        f\paren{t}=\o{g\paren{t}} \\
        g\paren{t}=\O{h\paren{t}}
    \end{dcases}\) alors \(f\paren{t}=\o{h\paren{t}}\). \\
    \item En particulier, la relation de négligeabilité est transitive.
\end{enumerate}
\end{prop}

\begin{prop}[Sommes]
Soient \(f,g,h\in\F{A}{\K}\).

On a, quand \(t\) tend vers \(a\) :

\begin{enumerate}
    \item Si \(\begin{dcases}
        f\paren{t}=\O{h\paren{t}} \\
        g\paren{t}=\O{h\paren{t}}
    \end{dcases}\) alors \(f\paren{t}+g\paren{t}=\O{h\paren{t}}\). \\
    \item Si \(\begin{dcases}
        f\paren{t}=\o{h\paren{t}} \\
        g\paren{t}=\o{h\paren{t}}
    \end{dcases}\) alors \(f\paren{t}+g\paren{t}=\o{h\paren{t}}\).
\end{enumerate}
\end{prop}

\begin{prop}[Produits]
Soient \(f,g,h,k\in\F{A}{\K}\).

On a, quand \(t\) tend vers \(a\) :

\begin{enumerate}
    \item Si \(\begin{dcases}
        f\paren{t}=\O{g\paren{t}} \\
        h\paren{t}=\O{k\paren{t}}
    \end{dcases}\) alors \(f\paren{t}h\paren{t}=\O{g\paren{t}k\paren{t}}\). \\
    \item En particulier : si \(f\paren{t}=\O{g\paren{t}}\) alors \(f\paren{t}h\paren{t}=\O{g\paren{t}h\paren{t}}\). \\
    \item Si \(\begin{dcases}
        f\paren{t}=\o{g\paren{t}} \\
        h\paren{t}=\O{k\paren{t}}
    \end{dcases}\) alors \(f\paren{t}h\paren{t}=\o{g\paren{t}k\paren{t}}\). \\
    \item En particulier : si \(f\paren{t}=\o{g\paren{t}}\) alors \(f\paren{t}h\paren{t}=\o{g\paren{t}h\paren{t}}\).
\end{enumerate}
\end{prop}

\begin{prop}[Puissances]
Soient \(f,g\in\F{A}{\Rps}\) et \(\alpha\in\R\).

On a, quand \(t\) tend vers \(a\) :

\begin{enumerate}
    \item Si \(\begin{dcases}
        f\paren{t}=\O{g\paren{t}} \\
        \alpha\geq0
    \end{dcases}\) alors \(f\paren{t}^\alpha=\O{g\paren{t}^\alpha}\). \\
    \item Si \(\begin{dcases}
        f\paren{t}=\O{g\paren{t}} \\
        \alpha\leq0
    \end{dcases}\) alors \(g\paren{t}^\alpha=\O{f\paren{t}^\alpha}\). \\
    \item Si \(\begin{dcases}
        f\paren{t}=\o{g\paren{t}} \\
        \alpha>0
    \end{dcases}\) alors \(f\paren{t}^\alpha=\o{g\paren{t}^\alpha}\). \\
    \item Si \(\begin{dcases}
        f\paren{t}=\o{g\paren{t}} \\
        \alpha<0
    \end{dcases}\) alors \(g\paren{t}^\alpha=\o{f\paren{t}^\alpha}\).
\end{enumerate}

On retient en particulier : \[f\paren{t}=\O{g\paren{t}}\imp\dfrac{1}{g\paren{t}}=\O{\dfrac{1}{f\paren{t}}}\qquad\text{et}\qquad f\paren{t}=\o{g\paren{t}}\imp\dfrac{1}{g\paren{t}}=\o{\dfrac{1}{f\paren{t}}}.\]

La même proposition est vraie en prenant \(f,g\in\F{A}{\C}\) et \(\alpha\in\N\) ou en prenant \(f,g\in\F{A}{\Cs}\) et \(\alpha\in\Z\).
\end{prop}

\subsection{Relation d'équivalence \(\sim\)}

\begin{defprop}
Soient \(f,g\in\F{A}{\K}\).

Les propositions suivantes sont équivalentes :

\begin{enumerate}
    \item \(g\paren{t}\egqd{t\to a}f\paren{t}+\o{f\paren{t}}\) \\
    \item \(f\paren{t}\egqd{t\to a}g\paren{t}+\o{g\paren{t}}\) \\
    \item Il existe une fonction \(\lambda\in\F{A}{\K}\) telle que : \[\lim_{t\to a}\lambda\paren{t}=1\qquad\text{et}\qquad\quantifs{\exists V\in\V{a};\forall t\in V}f\paren{t}=\lambda\paren{t}g\paren{t}\]
    \item Il existe une fonction \(\mu\in\F{A}{\K}\) telle que : \[\lim_{t\to a}\mu\paren{t}=1\qquad\text{et}\qquad\quantifs{\exists V\in\V{a};\forall t\in V}g\paren{t}=\mu\paren{t}f\paren{t}.\]
\end{enumerate}

Lorsqu'elles sont vérifiées, on dit que \[f\text{ est équivalente à }g\] ou \[\begin{aligned}
f\paren{t}\text{ est équivalent à }g\paren{t}\text{ quand }t\text{ tend vers }a \\
f\paren{t}\text{ est un équivalent de }g\paren{t}\text{ quand }t\text{ tend vers }a
\end{aligned}\] et on note : \[f\paren{t}\sim g\paren{t}\text{ quand }t\text{ tend vers }a\] ou : \[f\paren{t}\simqd{t\to a}g\paren{t}.\]
\end{defprop}

\begin{prop}
Soient \(f,g\in\F{A}{\K}\).

Si \(g\) ne s'annule pas au voisinage de \(a\) : \[\quantifs{\exists V\in\V{a};\forall t\in V}g\paren{t}\not=0,\] alors : \[f\paren{t}\simqd{t\to a}g\paren{t}\ssi\lim_{t\to a}\dfrac{f\paren{t}}{g\paren{t}}=1.\]
\end{prop}

\begin{ex}
Soient \(\alpha,\beta\in\R\). On a : \[t^\alpha\simqd{t\to a}t^\beta\ssi\alpha=\beta.\]

Soient \(d\in\N\), \(a_0,\dots,a_d\in\C\) tels que \(a_d\not=0\) et \(P=a_dX^d+\dots+a_0X^0\in\poly[\C]\). On a : \[P\paren{t}\simqd{t\to\pinf}a_dt^d.\]

Soit \(f\in\F{A}{\K}\). On a : \[f\paren{t}\simqd{t\to a}0\ssi f\text{ est nulle sur un voisinage de }a\] et : \[\quantifs{\forall l\in\K\excluant\accol{0}}f\paren{t}\simqd{t\to a}l\ssi\lim_{t\to a}f\paren{t}=l.\]

NB : l'équivalence précédente est valable pour toute limite finie non-nulle. En particulier, deux fonctions admettant un même limite finie non-nulle sont équivalentes.
\end{ex}

\subsection{Propriétés de \(\sim\)}

\begin{prop}
La relation \guillemets{être équivalentes au voisinage de \(a\)} est une relation d'équivalence sur \(\F{A}{\K}\).
\end{prop}

\begin{prop}
Soient \(f,g\in\F{A}{\K}\).

On a : \[f\paren{t}\simqd{t\to a}g\paren{t}\imp\begin{dcases}
    f\paren{t}\egqd{t\to a}\O{g\paren{t}} \\
    g\paren{t}\egqd{t\to a}\O{f\paren{t}}
\end{dcases}\]
\end{prop}

\begin{prop}[Cas réel : limite et signe]
Soient \(f,g\in\F{A}{\R}\) et \(l\in\Rb\).

Si \(\begin{dcases}
    \lim_{t\to a}f\paren{t}=l \\
    f\paren{t}\simqd{t\to a}g\paren{t}
\end{dcases}\) alors \(\lim_{t\to a}g\paren{t}=l\).

Si \(f\paren{t}\simqd{t\to a}g\paren{t}\) alors \(f\) et \(g\) sont de même signe au sens strict sur un voisinage de \(a\) : \[\quantifs{\exists V\in\V{a};\forall t\in V}\sg f\paren{t}=\sg g\paren{t}.\]
\end{prop}

\begin{prop}[Cas complexe : limite]
Soient \(f,g\in\F{A}{\C}\) et \(l\in\C\).

Si \(\begin{dcases}
\lim_{t\to a}f\paren{t}=l \\
f\paren{t}\simqd{t\to a}g\paren{t}
\end{dcases}\) alors \(\lim_{t\to a}g\paren{t}=l\).
\end{prop}

\begin{prop}[Produits]
Soient \(f,g,h,k\in\F{A}{\K}\).

On a, quand \(t\) tend vers \(a\) : \[\begin{dcases}
f\paren{t}\sim g\paren{t} \\
h\paren{t}\sim k\paren{t}
\end{dcases}\imp f\paren{t}h\paren{t}\sim g\paren{t}k\paren{t}.\]
\end{prop}

\begin{prop}[Puissances]
Soient \(f,g\in\F{A}{\Rps}\) et \(\alpha\in\R\).

On a, quand \(t\) tend vers \(a\) : \[f\paren{t}\sim g\paren{t}\imp f\paren{t}^\alpha\sim g\paren{t}^\alpha.\]

On retient en particulier : \[f\paren{t}\sim g\paren{t}\imp\dfrac{1}{f\paren{t}}\sim\dfrac{1}{g\paren{t}}.\]

La même proposition est vraie en prenant \(f,g\in\F{A}{\C}\) et \(\alpha\in\N\) ou en prenant \(f,g\in\F{A}{\Cs}\) et \(\alpha\in\Z\).
\end{prop}

\begin{prop}[Sommes]
On ne fait pas de sommes d'équivalents.
\end{prop}

\begin{prop}
Soient \(f,g,h,k\in\F{A}{\R}\).

Si \(\begin{dcases}
f\leq g\leq h \\
f\sim h\sim k
\end{dcases}\) alors \(g\sim k\).
\end{prop}

\subsection{Changements de variable}

\begin{prop}
Soient \(B\subset\R\) et \(b\in\Rb\) tels que tout voisinage de \(b\) dans \(\R\) rencontre \(B\) : \[\quantifs{\forall V\in\V{b}}V\inter B\not=\ensvide.\]

Soient \(\phi:B\to A\) telle que \(\lim_{s\to b}\phi\paren{s}=a\) et \(f,g\in\F{A}{\K}\).

On a :

\begin{enumerate}
    \item Si \(f\paren{t}\egqd{t\to a}\O{g\paren{t}}\) alors \(f\paren{\phi\paren{s}}\egqd{s\to b}\O{g\paren{\phi\paren{s}}}\). \\
    \item Si \(f\paren{t}\egqd{t\to a}\o{g\paren{t}}\) alors \(f\paren{\phi\paren{s}}\egqd{s\to b}\o{g\paren{\phi\paren{s}}}\). \\
    \item Si \(f\paren{t}\simqd{t\to a}g\paren{t}\) alors \(f\paren{\phi\paren{s}}\simqd{s\to b}g\paren{\phi\paren{s}}\).
\end{enumerate}
\end{prop}

\subsection{Croissances comparées}

\begin{prop}[Croissances comparées en \(\pinf\)]\thlabel{prop:croissancesComparéesFonctionsPinf}
Soient \(\alpha_1,\alpha_2,\beta_1,\beta_2,\gamma_1,\gamma_2\in\R\) tels que : \[\alpha_1<0<\alpha_2\qquad\text{et}\qquad\beta_1<0<\beta_2\qquad\text{et}\qquad\gamma_1<0<\gamma_2.\]

On a, quand \(t\) tend vers \(\pinf\) : \[\underbrace{\e{\gamma_1t}\quad\prec\quad t^{\alpha_1}\quad\prec\quad\ln^{\beta_1}t}_{\text{fonctions de }t\text{ de limite nulle}}\quad\prec\quad1\quad\prec\quad\underbrace{\ln^{\beta_2}t\quad\prec\quad t^{\alpha_2}\quad\prec\quad\e{\gamma_2t}}_{\text{fonctions de }t\text{ de limite }\pinf}.\]
\end{prop}

\begin{dem}
\Cf \thref{prop:croissancesComparéesSuites}.
\end{dem}

\begin{prop}[Croissances comparées en \(0^+\)]
Soient \(\alpha_1,\alpha_2,\beta_1,\beta_2\in\R\) tels que : \[\alpha_1<0<\alpha_2\qquad\text{et}\qquad\beta_1<0<\beta_2.\]

On a, quand \(t\) tend vers \(0^+\) : \[\underbrace{t^{\alpha_2}\quad\prec\quad\abs{\ln t}^{\beta_1}}_{\text{fonctions de }t\text{ de limite nulle}}\quad\prec\quad1\quad\prec\quad\underbrace{\abs{\ln t}^{\beta_2}\quad\prec\quad t^{\alpha_1}}_{\text{fonctions de }t\text{ de limite }\pinf}.\]
\end{prop}

\begin{dem}
Découle de la \thref{prop:croissancesComparéesFonctionsPinf} par un changement de variable.

Par exemple, on a vu \(\ln^{\beta_2}s\egqd{s\to\pinf}\o{s^{-\alpha_1}}\) car \(-\alpha_1>0\).

Or \(\lim_{t\to0^+}\dfrac{1}{t}=\pinf\) donc \(\ln^{\beta_2}\dfrac{1}{t}\egqd{t\to0^+}\o{\paren{\dfrac{1}{t}}^{-\alpha_1}}\).

D'où \(\abs{\ln t}^{\beta_2}\egqd{t\to0^+}\o{t^{\alpha_1}}\).
\end{dem}

\begin{exoex}
Calculer : \[\lim_{t\to0^+}t\ln t.\]
\end{exoex}

\begin{corr}
On a \(\ln t\egqd{t\to0^+}\o{\dfrac{1}{t}}\) donc \(\lim_{t\to0^+}\dfrac{\ln t}{\nicefrac{1}{t}}=0\).

Donc \[\lim_{t\to0^+}t\ln t=0.\]
\end{corr}

\subsection{Erreurs à ne pas faire}

\subsubsection{Sommes d'équivalents}

Quand \(t\to0^+\) :

On a \(\dfrac{1}{t}+\ln t\sim\dfrac{1}{t}+1\) car \(\ln t=\o{\dfrac{1}{t}}\).

De plus, on a \(1=\o{\dfrac{1}{t}}\) donc \[\dfrac{1}{t}+\ln t=\dfrac{1}{t}+\o{\dfrac{1}{t}}\sim\dfrac{1}{t}\qquad\text{et}\qquad\dfrac{1}{t}+1=\dfrac{1}{t}+\o{\dfrac{1}{t}}\sim\dfrac{1}{t}.\]

D'autre part, \(\dfrac{-1}{t}\sim\dfrac{-1}{t}\) mais on n'a pas \(\ln t\sim1\).

\subsubsection{Limite nulle donc équivalent nul}

L'affirmation \guillemets{\(\lim_{t\to\pinf}\dfrac{1}{t}=0\) donc \(\dfrac{1}{t}\simqd{t\to\pinf}0\)} est fausse car \(\lim_{t\to\pinf}\dfrac{0}{\nicefrac{1}{t}}\not=1\).

\subsubsection{Appliquer une fonction à un équivalent}

On a, quand \(n\to\pinf\) : \(\dfrac{1}{n}+1\sim\dfrac{2}{n}+1\) mais on n'a pas \(\ln\paren{\dfrac{1}{n}+1}\sim\ln\paren{\dfrac{2}{n}+1}\).

En effet, on a \(\lim_{\substack{t\to0 \\ t\not=0}}\dfrac{\ln\paren{1+t}-\ln1}{t-0}=\ln\prim1=1\) donc \(\ln\paren{1+t}\simqd{t\to0}t\).

Or \(\lim_n\dfrac{1}{n}=\lim_n\dfrac{2}{n}=0\).

Donc \(\begin{dcases}
\ln\paren{1+\dfrac{1}{n}}\sim\dfrac{1}{n} \\
\ln\paren{1+\dfrac{2}{n}}\sim\dfrac{2}{n}
\end{dcases}\)

Autre exemple : on a \(n\simqd{n\to\pinf}n+1\) mais on n'a pas \(\e{n}\simqd{n\to\pinf}\e{n+1}\) car \(\lim_n\dfrac{\e{n+1}}{\e{n}}=\e{}\not=1\).

\subsubsection{Changement de variable inadéquat}

On a \(\ln\paren{1+t}\simqd{t\to0}t\).

L'affirmation \guillemets{donc \(\ln\paren{1+\e{x}}\simqd{x\to\pinf}\e{x}\)} est fausse car \(\e{x}\xrightarrow[x\to\pinf]{}\pinf\not=0\).

\section{Développements limités}

\subsection*{Développements limités usuels}\label{subsec:développementsLimitésUsuels}

Soit \(\alpha\in\R\).

On a, quand \(x\) tend vers \(0\) :

\[\begin{array}{ccccc}
\dfrac{1}{1+x}&=&\sum_{k=0}^n\paren{-x}^k+\o{x^n}&=&1-x+x^2-x^3+\dots+\paren{-1}^nx^n+\o{x^n} \\\\
\ln\paren{1+x}&=&\sum_{k=1}^n\paren{-1}^{k-1}\dfrac{x^k}{k}+\o{x^n}&=&x-\dfrac{x^2}{2}+\dfrac{x^3}{3}+\dots+\paren{-1}^{n-1}\dfrac{x^n}{n}+\o{x^n} \\\\
\e{x}&=&\sum_{k=0}^n\dfrac{x^k}{k!}+\o{x^n}&=&1+x+\dfrac{x^2}{2}+\dfrac{x^3}{6}+\dots+\dfrac{x^n}{n!}+\o{x^n} \\\\
\cos x&=&\sum_{k=0}^n\paren{-1}^k\dfrac{x^{2k}}{\paren{2k}!}+\o{x^{2n+1}}&=&1-\dfrac{x^2}{2}+\dfrac{x^4}{24}+\dots+\paren{-1}^n\dfrac{x^{2n}}{\paren{2n}!}+\o{x^{2n+1}} \\\\
\sin x&=&\sum_{k=0}^n\paren{-1}^k\dfrac{x^{2k+1}}{\paren{2k+1}!}+\o{x^{2n+2}}&=&x-\dfrac{x^3}{6}+\dfrac{x^5}{120}+\dots+\paren{-1}^n\dfrac{x^{2n+1}}{\paren{2n+1}!}+\o{x^{2n+2}} \\\\
\ch x&=&\sum_{k=0}^n\dfrac{x^{2k}}{\paren{2k}!}+\o{x^{2n+1}}&=&1+\dfrac{x^2}{2}+\dfrac{x^4}{24}+\dots+\dfrac{x^{2n}}{\paren{2n}!}+\o{x^{2n+1}} \\\\
\sh x&=&\sum_{k=0}^n\dfrac{x^{2k+1}}{\paren{2k+1}!}+\o{x^{2n+2}}&=&x+\dfrac{x^3}{6}+\dfrac{x^5}{120}+\dots+\dfrac{x^{2n+1}}{\paren{2n+1}!}+\o{x^{2n+2}} \\\\
\Arctan x&=&\sum_{k=0}^n\paren{-1}^k\dfrac{x^{2k+1}}{2k+1}+\o{x^{2n+2}}&=&x-\dfrac{x^3}{3}+\dfrac{x^5}{5}+\dots+\paren{-1}^n\dfrac{x^{2n+1}}{2n+1}+\o{x^{2n+2}} \\\\
\tan x&=&&=&x+\dfrac{x^3}{3}+\dfrac{2x^5}{15}+\dfrac{17x^7}{315}+\o{x^8} \\\\
\paren{1+x}^\alpha&=&\sum_{k=0}^n\binom{k}{\alpha}x^k+\o{x^n}&=&1+\alpha x+\dfrac{\alpha\paren{\alpha-1}}{2}x^2+\dots+\binom{n}{\alpha}x^n+\o{x^n}
\end{array}\] en posant : \[\quantifs{\forall k\in\N}\binom{k}{\alpha}=\dfrac{\alpha\paren{\alpha-1}\dots\paren{\alpha-k+1}}{k!}\] (même si \(\alpha\) n'est pas un entier).

\subsection{Définition}

\begin{defi}[Développement limité]
Soient \(I\) un intervalle de \(\R\), \(f:I\to\K\), \(a\in I\) et \(n\in\N\).

On dit que \(f\) admet un développement limité à l'ordre \(n\) en \(a\) s'il existe \(a_0,\dots,a_n\in\K\) tels que : \[f\paren{a+h}\egqd{h\to0}a_0h^0+\dots+a_nh^n+\o{h^n},\] \cad : \[f\paren{x}\egqd{x\to a}a_0\paren{x-a}^0+\dots+a_n\paren{x-a}^n+\o{\paren{x-a}^n}.\]

Autrement dit : il existe \(P\in\poly\) tel que : \[f\paren{a+h}\egqd{h\to0}P\paren{h}+\o{h^n}.\]

NB : les termes de \(P\) de degrés strictement supérieurs à \(n\) vont \guillemets{dans le \(o\)}.
\end{defi}

\begin{ex}
Soit \(n\in\N\).

On a : \[\dfrac{1}{1-x}\egqd{x\to0}1+x+x^2+\dots+x^n+\o{x^n}.\]
\end{ex}

\begin{dem}
On a \(\quantifs{\forall x\in\R\excluant\accol{1}}1+x+\dots+x^n=\dfrac{1-x^{n+1}}{1-x}\) donc : \[\quantifs{\forall x\in\R\excluant\accol{1}}\dfrac{1}{1-x}=x^0+\dots+x^n+\dfrac{x^{n+1}}{1-x}.\]

Or, quand \(x\to0\) : \[\dfrac{x^{n+1}}{1-x}=x^n\dfrac{x}{1-x}=x^n\o{1}=\o{x^n}.\]

Donc : \[\dfrac{1}{1-x}\egqd{x\to0}x^0+\dots+x^n+\o{x^n}.\]
\end{dem}

\begin{cor}
Soit \(n\in\N\).

On a : \[\dfrac{1}{1+x}\egqd{x\to0}\sum_{k=0}^n\paren{-1}^kx^k+\o{x^n}.\]
\end{cor}

\subsection{Propriétés des développements limités}

\begin{prop}[Unicité du développement limité]
Soient \(I\) un intervalle de \(\R\), \(f:I\to\K\), \(a\in I\), \(n\in\N\) et \(a_0,\dots,a_n,b_0,\dots,b_n\in\K\).

On suppose qu'on a, quand \(h\to0\) : \[f\paren{a+h}=a_0h^0+\dots+a_nh^n+\o{h^n}=b_0h^0+\dots+b_nh^n+\o{h^n}.\]

Alors \(\quantifs{\forall k\in\interventierii{0}{n}}a_k=b_k\).
\end{prop}

\begin{dem}
On raisonne par l'absurde.

Soit \(k\in\interventierii{0}{n}\) tel que \(\begin{dcases}
\quantifs{\forall l\in\interventierii{0}{k-1}}a_l=b_l \\
a_k\not=b_k
\end{dcases}\)

On obtient, par différence : \[\begin{aligned}
\underbrace{\paren{a_k-b_k}}_{\not=0}h^k&=\underbrace{\paren{b_{k+1}-a_{k+1}}h^{k+1}}_{=\o{h^k}}+\dots+\underbrace{\paren{b_n-a_n}h^n}_{=\o{h^k}}+\underbrace{\o{h^n}}_{=\o{h^k}} \\
&=\o{h^k}.
\end{aligned}\]

Donc \(h^k=\o{h^k}\) : contradiction.
\end{dem}

\begin{prop}
Soient \(I\) un intervalle de \(\R\), \(f:I\to\K\) et \(a\in I\).

On a :

\begin{enumerate}
    \item \(f\text{ admet un développement limité à l'ordre 0 en }a\ssi f\text{ est continue en }a\) \\
    \item \(f\text{ admet un développement limité à l'ordre 1 en }a\ssi f\text{ est dérivable en }a\).
\end{enumerate}
\end{prop}

\begin{dem}[1]
\impdir

On suppose qu'il existe \(a_0\in\K\) tel que \(f\paren{a+h}\egqd{h\to0}a_0+\o{1}\).

On a \(\lim_{h\to0}f\paren{a+h}=a_0\in\K\).

Donc \(f\) est continue en \(a\) et \(f\paren{a}=a_0\).

\imprec

Supposons \(f\) continue en \(a\).

Alors \(\lim_{h\to0}f\paren{a+h}=f\paren{a}\).

Donc \(f\paren{a+h}\egqd{h\to0}f\paren{a}+\o{1}\).

Donc \(f\) admet un développement limité à l'ordre 0 en \(a\).
\end{dem}

\begin{dem}[2]
\impdir

On suppose qu'il existe \(a_0,a_1\in\K\) tels que \(f\paren{a+h}\egqd{h\to0}a_0+a_1h+\o{h}\).

On a, en prenant \(h=0\) : \(f\paren{a}=a_0\).

D'où \(f\paren{a+h}-f\paren{a}\egqd{h\to0}a_1h+\o{h}\).

Donc \(\dfrac{f\paren{a+h}-f\paren{a}}{h}\egqd{\substack{h\to0 \\ h\not=0}}a_1+\o{1}\).

Donc \(\lim_{\substack{h\to0 \\ h\not=0}}\dfrac{f\paren{a+h}-f\paren{a}}{h}=a_1\in\K\).

Donc \(f\) est dérivable en \(a\) et \(a_1=f\prim\paren{a}\).

\imprec

Supposons \(f\) dérivable en \(a\).

On a \(\lim_{\substack{h\to0 \\ h\not=0}}\dfrac{f\paren{a+h}-f\paren{a}}{h}=f\prim\paren{a}\in\K\).

Donc \(\lim_{\substack{h\to0 \\ h\not=0}}\dfrac{f\paren{a+h}-f\paren{a}}{h}-f\prim\paren{a}=0\).

Donc \(\dfrac{f\paren{a+h}-f\paren{a}-hf\prim\paren{a}}{h}\egqd{\substack{h\to0 \\ h\not=0}}\o{1}\).

Donc \(f\paren{a+h}-f\paren{a}-hf\prim\paren{a}\egqd{h\to0}\o{h}\).

D'où le développement limité à l'ordre 1 en \(a\) : \[f\paren{a+h}\egqd{h\to0}f\paren{a}+f\prim\paren{a}h+\o{h}.\]
\end{dem}

\begin{rem}
On a aussi obtenu que le coefficient de degré 0 est \(f\paren{a}\) et que le coefficient de degré 1 est \(f\prim\paren{a}\).
\end{rem}

\begin{ex}
On a, quand \(h\to0\) : \begin{align*}
\ln\paren{1+h}&=0+h+\o{h} & \exp\paren{h}&=1+h+\o{h} & \sin h&=0+h+\o{h} \\
&=h+\o{h} & \exp\paren{h}-1&\sim h & &=h+\o{h} \\
&\sim h & & & &\sim h.
\end{align*}
\end{ex}

\begin{prop}[Tronquer un développement limité]
Soient \(I\) un intervalle de \(\R\), \(a\in I\), \(f:I\to\K\) et \(n\in\N\).

Si \(f\) admet un développement limité d'ordre \(n+1\) en \(a\) alors \(f\) admet un développement limité d'ordre \(n\) en \(a\) obtenu en \guillemets{tronquant} son développement limité à l'ordre \(n+1\) en \(a\).
\end{prop}

\begin{dem}
Supposons qu'on a : \[f\paren{a+h}\egqd{h\to0}a_0h^0+\dots+a_nh^n+\underbrace{a_{n+1}h^{n+1}}_{=\o{h^n}}+\underbrace{\o{h^{n+1}}}_{=\o{h^n}}.\]

Alors on a : \[f\paren{a+h}\egqd{h\to0}a_0h^0+\dots+a_nh^n+\o{h^n}.\]

D'où la proposition.
\end{dem}

\begin{prop}
Soient \(I\) un intervalle de \(\R\), \(a\in I\), \(n\in\N\) et \(f,g\in\F{I}{\K}\).

Si \(f\) et \(g\) admettent un développement limité d'ordre \(n\) en \(a\) alors \(f+g\) et \(fg\) aussi.
\end{prop}

\begin{dem}
Soient \(P,Q\in\polydeg{n}\) tels que, quand \(h\to0\) : \[\begin{dcases}
f\paren{a+h}=P\paren{h}+\o{h^n} \\
g\paren{a+h}=Q\paren{h}+\o{h^n}
\end{dcases}\]

On a : \[\paren{f+g}\paren{a+h}=\paren{P+Q}\paren{a+h}+\o{h^n}\] et \[\begin{aligned}
\paren{fg}\paren{a+h}&=\croch{P\paren{h}+\o{h^n}}\croch{Q\paren{h}+\o{h^n}} \\
&=P\paren{h}Q\paren{h}+\underbrace{P\paren{h}\o{h^n}}_{=\o{h^n}\O{1}=\o{h^n}}+\underbrace{\o{h^n}Q\paren{h}}_{=\o{h^n}\O{1}=\o{h^n}}+\underbrace{\o{h^n}\o{h^n}}_{=\o{h^{2n}}=\o{h^n}} \\
&=\underbrace{P\paren{h}Q\paren{h}}_{\star}+\o{h^n}.
\end{aligned}\]

\(\star\) : il peut y avoir des termes de degré appartenant à l'intervalle entier \(\interventierii{n+1}{2n}\) mais ils sont négligeables devant \(h^n\).
\end{dem}

\begin{rem}
On peut également faire des quotients et des composées de développements limités : on le montrera au cas par cas.
\end{rem}

\subsection{Applications}

\begin{prop}
Soient \(I\) et \(J\) deux intervalles de \(\R\), \(a\in I\), \(f:I\to J\) dérivable en \(a\) et \(g:J\to\K\) dérivable en \(f\paren{a}\).

Alors \(g\rond f\) est dérivable en \(a\) et on a : \[\paren{g\rond f}\prim\paren{a}=f\prim\paren{a}\times g\prim\paren{f\paren{a}}.\]
\end{prop}

\begin{dem}
Comme \(f\) et \(g\) sont dérivables en \(a\) et \(f\paren{a}\) respectivement, on a : \[\begin{dcases}
f\paren{a+h}\egqd{h\to0}f\paren{a}+f\prim\paren{a}h+\o{h} \\
g\paren{f\paren{a}+h}\egqd{h\to0}g\paren{f\paren{a}}+g\prim\paren{f\paren{a}}h+\o{h}
\end{dcases}\]

Comme \(f\) est dérivable en \(a\), \(f\) est continue en \(a\) donc on a : \[\lim_{h\to0}f\paren{a+h}-f\paren{a}=0.\]

On a donc, quand \(h\to0\) : \[\begin{WithArrows}
g\paren{f\paren{a+h}}&=g\paren{f\paren{a}+\underbrace{f\paren{a+h}-f\paren{a}}_{\to0}} \Arrow[i]{développement limité de \(f\)} \\
&=g\paren{f\paren{a}+\underbrace{f\prim\paren{a}h+\o{h}}_{\to0}} \Arrow[lr,xoffset=1cm]{développement limité de \(g\)} \\
&=g\paren{f\paren{a}}+g\prim\paren{f\paren{a}}\croch{f\prim\paren{a}h+\o{h}}+\underbrace{\o{\underbrace{f\prim\paren{a}h+\o{h}}_{=\O{h}}}}_{=\o{h}} \\
&=g\rond f\paren{a}+g\prim\paren{f\paren{a}}\times f\prim\paren{a}\times h+\o{h}.
\end{WithArrows}\]

Donc \(g\rond f\) admet un développement limité à l'ordre 1 en \(a\).

Donc \(g\rond f\) est dérivable en \(a\) et on a : \[\paren{g\rond f}\prim\paren{a}=f\prim\paren{a}\times g\prim\paren{f\paren{a}}.\]
\end{dem}

\begin{prop}
Soient \(I\) un intervalle de \(\R\), \(a\in I\), \(n\in\N\) et \(f:I\to\R\) admettant un développement limité à l'ordre \(n\) en \(a\) : \[\quantifs{\exists a_0,\dots,a_n\in\R}f\paren{a+h}\egqd{h\to0}a_0h^0+\dots+a_nh^n+\o{h^n}.\]

On suppose qu'il existe \(d\in\interventierii{1}{n}\) tel que \(\begin{dcases}
a_1=\dots=a_{d-1}=0 \\
a_d\not=0
\end{dcases}\)

Alors \(f\paren{a+h}-f\paren{a}\) est du signe de \(a_dh^d\) quand \(h\) est petit.
\end{prop}

\begin{dem}
On a, quand \(h\to0\) : \[f\paren{a+h}=f\paren{a}+a_dh^d+\underbrace{a_{d+1}h^{d+1}}_{=\o{h^d}}+\dots+\underbrace{a_nh^n}_{=\o{h^d}}+\underbrace{\o{h^n}}_{=\o{h^d}}.\]

Donc on a : \[\begin{WithArrows}
f\paren{a+h}-f\paren{a}&=a_dh^d+\o{h^d} \Arrow{car \(a_d\not=0\)} \\
&\sim a_dh^d.
\end{WithArrows}\]

D'où le résultat.
\end{dem}

\section{Formule de Taylor}

\subsection{Primitivation de développements limités}

\begin{lem}
Soient \(J\) un intervalle de \(\R\) contenant \(0\), \(n\in\N\) et \(g:J\to\K\) dérivable et telle que \[g\prim\paren{x}\egqd{x\to0}\o{x^n}.\]

Alors on a : \[g\paren{x}-g\paren{0}\egqd{x\to0}\o{x^{n+1}}.\]
\end{lem}

\begin{dem}
Soit \(\epsilon:J\to\K\) telle que \(\begin{dcases}
\lim_0\epsilon=0 \\
\quantifs{\forall x\in J}g\prim\paren{x}=x^n\epsilon\paren{x}
\end{dcases}\)

On remarque : \[\lim_{x\to0}\underbrace{\sup_{t\in\intervii{-\abs{x}}{\abs{x}}}\abs{\epsilon\paren{t}}}_{M_x}=0.\]

Soit \(x\in J\).

Selon l'inégalité des accroissements finis appliquée à \(g\prim\) entre \(0\) et \(x\), on a : \[\begin{dcases}
\quantifs{\forall t\in\intervii{0}{x}}\abs{g\prim\paren{t}}=\epsilon\paren{t}t^n\leq M_x\abs{x}^n &\text{si }x>0 \\
\quantifs{\forall t\in\intervii{x}{0}}\abs{g\prim\paren{t}}=\epsilon\paren{t}t^n\leq M_x\abs{x}^n &\text{si }x<0
\end{dcases}\]

Donc, si \(x\not=0\), on a : \[\abs{g\paren{x}-g\paren{0}}\leq M_x\abs{x}^n\abs{x-0}=M_x\abs{x}^{n+1}.\]

Cela est aussi vrai si \(x=0\).

D'où : \[g\paren{x}-g\paren{0}\egqd{x\to0}\o{x^{n+1}}.\]
\end{dem}

\begin{rem}
On ne peut pas \guillemets{dériver un développement limité}.
\end{rem}

\begin{dem}
Posons \(\fonction{f}{\R}{\R}{x}{\begin{dcases}x^2\sin\dfrac{1}{x} &\text{si }x\not=0 \\ 0 &\text{sinon}\end{dcases}}\)

On a, quand \(x\to0\) : \(\begin{dcases}
\sin\dfrac{1}{x}=\O{1} \\
x^2=\o{x}
\end{dcases}\)

D'où : \[f\paren{x}\egqd{x\to0}\o{x},\] on a un développement limité à l'ordre 1 en \(0\) donc \(f\) est dérivable en \(0\).

Or, on a : \[\quantifs{\forall x\in\Rs}f\prim\paren{x}=\underbrace{2x\sin\dfrac{1}{x}}_{\xrightarrow[x\to0]{}0}-\underbrace{\cos\dfrac{1}{x}}_{\substack{\text{pas de} \\ \text{limite} \\ \text{en }0}}.\]

Donc \(f\prim\) n'a pas de limite en \(0\).

Donc \(f\prim\) n'est pas continue en \(0\) : \(f\prim\) n'admet pas de développement limité à l'ordre 0 en \(0\).
\end{dem}

\begin{prop}[Primitivation de développement limité]
Soient \(I\) un intervalle de \(\R\), \(a\in I\), \(n\in\N\) et \(f:I\to\K\) dérivable et telle qu'il existe \(a_0,\dots,a_n\in\K\) tels que : \[f\prim\paren{a+h}\egqd{h\to0}\sum_{k=0}^na_kh^k+\o{h^n}.\]

Alors \(f\) admet le développement limité à l'ordre \(n+1\) en \(a\) suivant : \[f\paren{a+h}\egqd{h\to0}f\paren{a}+\sum_{k=0}^na_k\dfrac{h^{k+1}}{k+1}+\o{h^{n+1}}.\]
\end{prop}

\begin{dem}
La fonction \(g:h\mapsto f\paren{a+h}-\sum_{k=0}^na_k\dfrac{h^{k+1}}{k+1}\) est dérivable et on a : \[\quantifs{\forall h\in\R}g\prim\paren{h}=f\prim\paren{a+h}-\sum_{k=0}^na_kh^k=\o{h^n}.\]

Selon le lemme précédent, on en déduit, quand \(h\to0\) : \[g\paren{h}-g\paren{0}=\o{h^{n+1}},\] \cad : \[f\paren{a+h}-\sum_{k=0}^na_k\dfrac{h^{k+1}}{k+1}-f\paren{a}=\o{h^{n+1}}.\]
\end{dem}

\begin{appl}
Établissons le développement limité à l'ordre 5 en \(0\) de la fonction \(\tan\).

Comme \(\tan\) est dérivable en \(0\), on a, quand \(x\) tend vers \(0\) : \[\tan x=\tan\paren{0}+\tan\prim\paren{0}x+\o{x}=x+\o{x}.\]

D'où : \[\tan^2x=\paren{x+\o{x}}^2=x^2+2x\o{x}+\o{x^2}=x^2+\o{x^2}.\]

D'où : \[\tan\prim x=1+x^2+\o{x^2}.\]

D'où, en primitivant le développement limité : \[\tan x=0+x+\dfrac{x^3}{3}+\o{x^3}.\]

D'où : \[\begin{aligned}
\tan^2x&=x^2+\dfrac{x^6}{9}+\o{x^6}+2\paren{\dfrac{x^4}{3}+\o{x^4}+\o{x^6}} \\
&=x^2+\dfrac{x^6}{9}+\dfrac{2x^4}{3}+\o{x^4}.
\end{aligned}\]

D'où : \[\tan\prim x=1+x^2+\dfrac{2x^4}{3}+\dfrac{x^6}{9}+\o{x^4}.\]

D'où, en primitivant le développement limité : \[\tan x=x+\dfrac{x^3}{3}+\dfrac{2x^5}{15}+\o{x^5}.\]
\end{appl}

\begin{theo}[Formule de Taylor-Young]
Soient \(I\) un intervalle de \(\R\), \(n\in\N\), \(a\in I\) et \(f\in\ensclasse{n}{I}{\K}\).

Alors \(f\) admet le développement limité à l'ordre \(n\) en \(a\) suivant : \[f\paren{a+h}\egqd{h\to0}\sum_{k=0}^n\dfrac{f\deriv{k}\paren{a}}{k!}h^k+\o{h^n}.\]
\end{theo}

\begin{dem}
Raisonnons par récurrence sur \(n\in\N\).

Pour tout \(n\in\N\), on note \(\P{n}\) la proposition \guillemets{\(\quantifs{\forall f\in\ensclasse{n}{I}{\K}}f\paren{a+h}\egqd{h\to0}\sum_{k=0}^n\dfrac{f\deriv{k}\paren{a}}{k!}h^k+\o{h^n}\)}.

On a déjà vu \(\P{0}\) et \(\P{1}\).

Soit \(n\in\N\) tel que \(\P{n}\). Montrons \(\P{n+1}\).

Soit \(f\in\ensclasse{n+1}{I}{\K}\).

On a \(f\prim\in\ensclasse{n}{I}{\K}\).

D'où, selon \(\P{n}\) : \[f\prim\paren{a+h}\egqd{h\to0}\sum_{k=0}^n\dfrac{\paren{f\prim}\deriv{k}\paren{a}}{k!}h^k+\o{h^n}.\]

D'où, en primitivant le développement limité : \[\begin{aligned}
f\paren{a+h}&\egqd{h\to0}f\paren{a}+\sum_{k=0}^n\dfrac{f\deriv{k+1}\paren{a}}{k!}\times\dfrac{h^{k+1}}{k+1}+\o{h^{n+1}} \\
&\egqd{h\to0}\dfrac{f\deriv{0}\paren{a}}{0!}+\sum_{k=0}^n\dfrac{f\deriv{k+1}\paren{a}}{\paren{k+1}!}h^{k+1}+\o{h^{n+1}} \\
&\egqd{h\to0}\sum_{k=0}^{n+1}\dfrac{f\deriv{k}\paren{a}}{k!}h^k+\o{h^{n+1}}.
\end{aligned}\]

D'où \(\P{n+1}\).

D'où la formule.
\end{dem}

\begin{ex}
\Cf \hyperref[subsec:développementsLimitésUsuels]{développements limités usuels}.
\end{ex}

\begin{dem}[\(\exp\)]
Soit \(n\in\N\).

Déterminons le développement limité de \(\exp\) à l'ordre \(n\) en \(0\).

On a \(\exp\in\ensclasse{\infty}{\R}{\R}\) donc : \[\begin{aligned}
\e{x}&\egqd{x\to0}\sum_{k=0}^n\dfrac{\exp\deriv{k}\paren{0}}{k!}x^k+\o{x^n} \\
&\egqd{x\to0}\sum_{k=0}^n\dfrac{x^k}{k!}+\o{x^n}.
\end{aligned}\]
\end{dem}

\begin{dem}[\(\sh\) \& \(\ch\)]
Soit \(n\in\N\).

Les développements limités de \(\sh\) et \(\ch\) à l'ordre \(n\) en \(0\) se déduisent de celui de \(\exp\) car on a : \[\begin{dcases}
\ch:x\mapsto\dfrac{\e{x}+\e{-x}}{2} \\
\sh:x\mapsto\dfrac{\e{x}-\e{-x}}{2}
\end{dcases}\]
\end{dem}

\begin{dem}[\(\cos\) \& \(\sin\)]
Soit \(n\in\N\).

Les développements limités de \(\cos\) et \(\sin\) à l'ordre \(n\) en \(0\) découlent de la formule de Taylor-Young.
\end{dem}

\begin{dem}[\(\ln\)]
Soit \(n\in\N\).

Déterminons le développement limité de \(\ln\) à l'ordre \(n\) en \(1\).

On a, quand \(x\to0\) : \[\dfrac{1}{1+x}=\sum_{k=0}^n\paren{-1}^kx^k+\o{x^n}.\]

D'où, en primitivant le développement limité : \[\ln\paren{1+x}=\ln1+\sum_{k=0}^n\paren{-1}^k\dfrac{x^{k+1}}{k+1}+\o{x^{n+1}}.\]
\end{dem}

\begin{dem}[\(\Arctan\)]
Soit \(n\in\N\).

Déterminons le développement limité de \(\Arctan\) à l'ordre \(n\) en \(0\).

On a, quand \(x\to0\) : \[\dfrac{1}{1+x^2}=\sum_{k=0}^n\paren{-1}^kx^{2k}+\o{x^{2n}}.\]

D'où, en primitivant le développement limité : \[\Arctan x=\Arctan0+\sum_{k=0}^n\paren{-1}^k\dfrac{x^{2k+1}}{2k+1}+\o{x^{2n+1}}.\]
\end{dem}

\begin{dem}[\(x\mapsto\paren{1+x}^\alpha\)]
Soient \(n\in\N\), \(\alpha\in\R\) et \(f:x\mapsto\paren{1+x}^\alpha\).

Déterminons le développement limité de \(f\) à l'ordre \(n\) en \(0\).

On remarque : \[\quantifs{\forall k\in\N;\forall x\in\intervee{-1}{\pinf}}f\deriv{k}\paren{x}=\paren{1+x}^{\alpha-k}\prod_{l=0}^{k+1}\paren{\alpha-l}.\]

D'où : \[\quantifs{\forall k\in\N}\dfrac{f\deriv{k}\paren{0}}{k!}=\dfrac{1}{k!}\prod_{l=0}^{k+1}\paren{\alpha-l}=\binom{k}{\alpha}.\]
\end{dem}

\begin{rem}
Il est parfois bien plus facile de calculer une dérivée en un point à partir des développements limités plutôt que par les méthodes souvent rencontrées jusqu'ici (calcul \guillemets{classique} de dérivée, théorème de la limite de la dérivée).
\end{rem}

\begin{exoex}
\begin{enumerate}
    \item Calculer la dérivée en \(0\) de la fonction \[\fonction{f}{\R}{\R}{x}{\dfrac{\e{x}\paren{1+\sh x}\paren{1+\sin x}\paren{1+\tan x}\paren{1+\ln\paren{1+x}}}{1-x}}\]
    \item Calculer la dérivée en \(0\) de la fonction \[\fonction{g}{\R}{\R}{x}{\begin{dcases}
        \dfrac{\sin x-x}{\e{x}-1-x} &\text{si }x\not=0 \\
        0 &\text{sinon}
    \end{dcases}}\]
\end{enumerate}
\end{exoex}

\begin{corr}[1]
On a les développements limités à l'ordre 1 en \(0\) suivants : \[\begin{dcases}
\e{x}=1+x+\o{x} \\
1+\sh x=1+x+\o{x} \\
1+\sin x=1+x+\o{x} \\
1+\tan x=1+x+\o{x} \\
1+\ln\paren{1+x}=1+x+\o{x} \\
\dfrac{1}{1-x}=1+x+\o{x}
\end{dcases}\]

D'où le développement limité à l'ordre 1 en \(0\) de \(f\) : \[\underbrace{\paren{1+x+\o{x}}\dots\paren{1+x+\o{x}}}_{\text{6 facteurs}}=1+6x+\o{x}.\]

Donc la dérivée de \(f\) en \(0\) est \(6\).
\end{corr}

\begin{corr}[2]
On a, quand \(x\to0\) : \[\begin{dcases}
\sin x=x-\dfrac{x^3}{6}+\o{x^3} \\
\e{x}=1+x+\dfrac{x^2}{2}+\o{x^2}
\end{dcases}\]

Donc : \[\begin{dcases}
\sin x-x=-\dfrac{x^3}{6}+\o{x^3}\sim-\dfrac{x^3}{6} \\
\e{x}-1-x=\dfrac{x^2}{2}+\o{x^2}\sim\dfrac{x^2}{2}
\end{dcases}\]

Donc on a : \[\begin{aligned}
g\paren{x}&\sim\dfrac{-\frac{x^3}{6}}{\frac{x^2}{2}} \\
&=-\dfrac{x}{3}+\o{x}.
\end{aligned}\]

D'où \(g\prim\paren{0}=-\dfrac{1}{3}\).
\end{corr}

\subsection{Application}

\begin{appl}[Extrema locaux]
Soient \(I\) un intervalle de \(\R\), \(a\in I\), \(n\in\N\) et \(f\in\ensclasse{n}{I}{\R}\) tels que : \[I\in\V{a}\qquad\text{et}\qquad f\deriv{n}\paren{a}\not=0\qquad\text{et}\qquad\quantifs{\forall k\in\interventierii{1}{n-1}}f\deriv{k}\paren{a}=0.\]

On peut en déduire le comportement de \(f\) au voisinage de \(a\) :

\begin{itemize}
    \item si \(n\) est pair et \(f\deriv{n}\paren{a}>0\) alors \(f\) admet un minimum local en \(a\) ; \\
    \item si \(n\) est pair et \(f\deriv{n}\paren{a}<0\) alors \(f\) admet un maximum local en \(a\) ; \\
    \item si \(n\) est impair alors \(f\) n'admet pas d'extremum local en \(a\).
\end{itemize}
\end{appl}

\begin{dem}
Comme \(f\) est de classe \(\classe{n}\), d'après la formule de Taylor-Young, quand \(h\to0\), on a : \[f\paren{a+h}=\sum_{k=0}^n\dfrac{f\deriv{k}\paren{a}}{k!}h^k+\o{h^n}=f\paren{a}+\dfrac{f\deriv{n}\paren{a}}{n!}h^n+\o{h^n}.\]

Donc, comme \(f\deriv{n}\paren{a}\not=0\), on a : \[f\paren{a+h}-f\paren{a}\sim\dfrac{f\deriv{n}\paren{a}}{n!}h^n.\]

Donc : \[\sg\paren{f\paren{a+h}-f\paren{a}}=\sg\paren{f\deriv{n}\paren{a}h^n}.\]

D'où les conclusions.
\end{dem}

\begin{exoex}
Étudier les extrema (globaux et locaux) de \(\fonction{f}{\R}{\R}{x}{\e{x}\sin x}\)
\end{exoex}

\begin{corr}
On remarque : \[\quantifs{\forall k\in\N}f\paren{\dfrac{\pi}{2}+k\pi}=\paren{-1}^k\e{\frac{\pi}{2}+k\pi}.\]

Donc : \[\lim_{k\to\pinf}f\paren{\dfrac{\pi}{2}+2k\pi}=\pinf\qquad\text{et}\qquad\lim_{k\to\pinf}f\paren{\dfrac{\pi}{2}+\paren{2k+1}\pi}=\minf.\]

Donc \(f\) n'est ni majorée, ni minorée, et n'admet donc aucun extremum global.

De plus, on a : \[\begin{aligned}
\quantifs{\forall x\in\R}f\prim\paren{x}&=\e{x}\paren{\sin x+\cos x} \\
&=\sqrt{2}\e{x}\paren{\dfrac{1}{\sqrt{2}}\sin x+\dfrac{1}{\sqrt{2}}\cos x} \\
&=\sqrt{2}\e{x}\paren{\cos\dfrac{\pi}{4}\sin x+\sin\dfrac{\pi}{4}\cos x} \\
&=\sqrt{2}\e{x}\sin\paren{x+\dfrac{\pi}{4}}.
\end{aligned}\]

On obtient de même : \[\quantifs{\forall x\in\R}f\seconde\paren{x}=2\e{x}\sin\paren{x+\dfrac{\pi}{2}}.\]

\analyse

Soit \(x\in\R\) tel que \(f\) admette un extremum local en \(x\).

Comme \(\R\) est un voisinage de \(x\), on a \(f\prim\paren{x}=0\) donc \(\sin\paren{x+\dfrac{\pi}{4}}=0\) donc \(x\equiv-\dfrac{\pi}{4}\croch{\pi}\).

\synthese

Soient \(x\in\R\) tel que \(x\equiv-\dfrac{\pi}{4}\croch{\pi}\) et \(k\in\Z\) tel que \(x=-\dfrac{\pi}{4}+k\pi\).

On a : \[f\seconde\paren{x}=2\e{-\frac{\pi}{4}+k\pi}\sin\paren{\dfrac{\pi}{4}+k\pi}=\paren{-1}^k\sqrt{2}\e{-\frac{\pi}{4}+k\pi}.\]

Si \(k\) est pair alors \(f\seconde\paren{x}>0\) donc \(f\) admet un minimum local en \(x\).

Si \(k\) est impair alors \(f\seconde\paren{x}<0\) donc \(f\) admet un maximum local en \(x\).

\conclusion

\(f\) admet un minimum local en tout point de la forme \(-\dfrac{\pi}{4}+2k\pi\) avec \(k\in\Z\) et un maximum local en tout point de la forme \(-\dfrac{\pi}{4}+\paren{2k+1}\pi\) avec \(k\in\Z\).
\end{corr}

\begin{rem}
En pratique, on se sert peu de la proposition précédente : il est plus facile de calculer un développement limité à l'ordre \(n\) et d'en déduire un équivalent que de calculer \(n\) dérivées successives (de plus, si l'on peut déterminer le signe de la dérivée première, on peut lire dans le tableau de variations quels sont les extrema locaux de la fonction).
\end{rem}

\section{Développements asymptotiques}

\begin{defi}
Soient \(A\subset\R\), \(f:A\to\K\) et \(a\in\Rb\) tel que tout voisinage de \(a\) dans \(\R\) rencontre \(A\), de sorte qu'on peut parler du comportement de \(f\) au voisinage de \(a\).

On appelle développement asymptotique de \(f\) en \(a\) (à la précision \(f_n\)) toute écriture de \(f\) sous la forme : \[f\paren{t}\egqd{t\to a}f_0\paren{t}+\dots+f_n\paren{t}+\o{f_n\paren{t}}\] où \(n\in\N\) et \(f_0,\dots,f_n\) sont des fonctions définies sur \(A\) (du moins au voisinage de \(a\)) telles que : \[\quantifs{\forall i\in\interventierii{1}{n}}f_i\paren{t}\egqd{t\to a}\o{f_{i-1}\paren{t}}.\]

On a alors : \[f\paren{t}\simqd{t\to a}f_0\paren{t}\] et : \[f\paren{t}-f_0\paren{t}\simqd{t\to a}f_1\paren{t}\] et : \[f\paren{t}-f_0\paren{t}-f_1\paren{t}\simqd{t\to a}f_2\paren{t}\] etc.
\end{defi}

\begin{ex}
Développement asymptotique à la précision \(\ln^2n\) : \[u_n\egqd{n\to\pinf}n!+n+\sqrt{n}\ln n+\ln^2n+\o{\ln^2n}.\]

Développement asymptotique à la précision \(\dfrac{1}{x^2}\) : \[f\paren{x}\egqd{x\to\pinf}\dfrac{x}{\ln x}+\sqrt{x}+\pi+\dfrac{2}{x^2}+\o{\dfrac{1}{x^2}}.\]

Développement asymptotique à la précision \(x\) : \[f\paren{x}\egqd{x\to0}\dfrac{1}{x}+\dfrac{\ln x}{\sqrt{x}}+7+\sqrt{x}+x+\o{x}.\]
\end{ex}

\begin{exoex}
Donner un développement asymptotique de \(\ln\paren{n!}\) quand \(n\) tend vers \(\pinf\) à la précision \(1\).
\end{exoex}

\begin{corr}
On a, quand \(n\to\pinf\) : \(n!\sim\sqrt{2\pi n}\paren{\dfrac{n}{\e{}}}^n\) donc \(\lim_n\dfrac{n!}{\sqrt{2\pi n}\paren{\dfrac{n}{\e{}}}^n}=1\) donc : \[\lim_n\ln\dfrac{n!}{\sqrt{2\pi n}\paren{\dfrac{n}{\e{}}}^n}=0.\]

Donc \(\ln\dfrac{n!}{\sqrt{2\pi n}\paren{\dfrac{n}{\e{}}}^n}=\o{1}\) donc : \[\ln\paren{n!}-\dfrac{1}{2}\ln\paren{2\pi n}-n\ln n+n\ln\e{}=\o{1}.\]

Finalement, on a le développement asymptotique suivant : \[\ln\paren{n!}=n\ln n-n+\dfrac{1}{2}\ln n+\dfrac{1}{2}\ln\paren{2\pi}+\o{1}.\]
\end{corr}

\begin{exoex}
\begin{enumerate}
    \item Donner un développement asymptotique de \(\Arctan x\) quand \(x\) tend vers \(\pinf\) à la précision \(\dfrac{1}{x^5}\). \\
    \item Application : en déduire un équivalent de \(\Arctan\paren{2x}-\Arctan x\) quand \(x\) tend vers \(\pinf\).
\end{enumerate}
\end{exoex}

\begin{corr}[1]
Rappel : on a \(\quantifs{\forall x\in\Rps}\Arctan x+\Arctan\dfrac{1}{x}=\dfrac{\pi}{2}\).

On a, quand \(h\to0\) : \[\Arctan h=h-\dfrac{h^3}{3}+\dfrac{h^5}{5}+\o{h^5}.\]

Donc, quand \(x\to\pinf\) : \[\Arctan\dfrac{1}{x}=\dfrac{1}{x}-\dfrac{1}{3x^3}+\dfrac{1}{5x^5}+\o{\dfrac{1}{x^5}}.\]

Donc : \[\Arctan x\egqd{x\to\pinf}\dfrac{\pi}{2}-\dfrac{1}{x}+\dfrac{1}{3x^3}-\dfrac{1}{5x^5}+\o{\dfrac{1}{x^5}}.\]
\end{corr}

\begin{corr}[2]
On a : \[\begin{aligned}
\Arctan\paren{2x}-\Arctan x&\egqd{x\to\pinf}-\dfrac{1}{2x}+\dfrac{1}{x}+\o{\dfrac{1}{x}} \\
&\egqd{x\to\pinf}\dfrac{1}{2x}+\o{\dfrac{1}{x}}.
\end{aligned}\]

Donc \(\Arctan\paren{2x}-\Arctan x\simqd{x\to\pinf}\dfrac{1}{2x}\).
\end{corr}

\begin{exoex}
Donner un développement asymptotique de \(\dfrac{\ln\paren{1-x}}{\cos x-1}\) quand \(x\) tend vers \(0\) à la précision \(x^2\).
\end{exoex}

\begin{corr}
On a : \[\ln\paren{1-x}\egqd{x\to0}-x-\dfrac{x^2}{2}-\dfrac{x^3}{3}-\dfrac{x^4}{4}+\o{x^4}\qquad\text{et}\qquad\cos x-1\egqd{x\to0}-\dfrac{x^2}{2}+\dfrac{x^4}{24}+\o{x^5}.\]

Donc : \[\begin{aligned}
f\paren{x}&\egqd{x\to0}\dfrac{-x-\frac{x^2}{2}-\frac{x^3}{3}-\frac{x^4}{4}+\o{x^4}}{-\frac{x^2}{2}+\frac{x^4}{24}+\o{x^5}} \\
&\egqd{x\to0}\dfrac{-2}{x^2}\paren{\dfrac{-x-\frac{x^2}{2}-\frac{x^3}{3}-\frac{x^4}{4}+\o{x^4}}{1-\frac{x^2}{12}+\o{x^3}}}.
\end{aligned}\]

Or : \(\dfrac{1}{1+h}\egqd{h\to0}1+h+\O{h^2}\).

Donc : \[\dfrac{1}{1-\frac{x^2}{12}+\o{x^3}}\egqd{x\to0}1+\dfrac{x^2}{12}+\o{x^3}.\]

D'où, quand \(x\to0\) : \[\begin{aligned}
f\paren{x}&=\dfrac{-2}{x^2}\paren{1+\dfrac{x^2}{12}+\o{x^3}}\paren{-x-\dfrac{x^2}{2}-\dfrac{x^3}{3}-\dfrac{x^4}{4}+\o{x^4}} \\
&=\dfrac{-2}{x^2}\paren{-x-\dfrac{x^2}{2}-\dfrac{x^3}{3}-\dfrac{x^4}{4}-\dfrac{x^3}{12}-\dfrac{x^4}{24}+\o{x^4}} \\
&=\dfrac{2}{x}+1+\dfrac{5}{6}x+\dfrac{7}{12}x^2+\o{x^2}.
\end{aligned}\]
\end{corr}

\begin{rem}
Apprenez à obtenir des développements limités à l'aide de votre calculatrice.

On peut aussi les obtenir avec Python, par exemple avec le module \verb|sympy| (les paramètres de la méthode \verb|series| sont la variable, le point où l'on se place et le nombre de termes du développement asymptotique voulu) :

\begin{verbatim}
>>> from sympy import sin, cos, log, pprint, latex
>>> from sympy.abc import x
>>> pprint(sin(x).series(x, 0, 9))    # "pretty print"
\end{verbatim}

\(x-\dfrac{x^3}{6}+\dfrac{x^5}{120}-\dfrac{x^7}{5040}+\O{x^9}\)

\begin{verbatim}
>>> print(latex(sin(x).series(x, 0, 7)))    # code LaTeX
x - \frac{x^{3}}{6} + \frac{x^{5}}{120} + O\left(x^{7}\right)

>>> a = log(1 - x) / (cos(x) - 1)
>>> b = a.series(x, 0, 6)
>>> pprint(b)
\end{verbatim}

\(\dfrac{2}{x}+1+\dfrac{5x}{6}+\dfrac{7x^2}{12}+\dfrac{167x^3}{360}+\dfrac{91x^4}{240}+\dfrac{4871x^5}{15120}+\O{x^6}\)
\end{rem}


\chapter{Groupe symétrique}

\minitoc

\section{Permutations}

\begin{rappel}[Groupe des permutations d'un ensemble]
Soit \(E\) un ensemble.

On appelle permutation de \(E\) toute bijection\footnote{On rappelle aussi que, si l'ensemble \(E\) est fini, alors pour qu'une fonction \(\sigma:E\to E\) soit une bijection, il suffit qu'elle soit injective ou surjective.} \(\sigma:E\to E\).

On note \(S_E\) ou \(\S{E}\) l'ensemble des permutations de \(E\).

Alors \(\groupe{\S{E}}[\rond]\) est un groupe, appelé groupe des permutations de \(E\).

Son élément neutre est \(\id{E}\).

L'inverse d'une permutation \(\sigma\in\S{E}\) est sa bijection réciproque \(\sigma\inv\).

Il est commutatif si, et seulement si, \(\Card E\leq2\).
\end{rappel}

\begin{defi}
Soit \(n\in\Ns\).

On appelle groupe symétrique d'ordre \(n\) et on note \(S_n\) ou \(\S{n}\) le groupe des permutations de l'ensemble \(\interventierii{1}{n}\) : \[S_n=\S{n}=\S{\interventierii{1}{n}}.\]

Ce groupe est commutatif si, et seulement si, \(n\leq2\).
\end{defi}

\begin{rem}
Dans ce chapitre, on étudie le groupe symétrique, \cad le groupe des permutations de \(\interventierii{1}{n}\) où \(n\in\Ns\).

Ce qu'on fait s'applique en fait au groupe des permutations de n'importe quel ensemble fini non-vide \(E\), puisque si \(n\) est son cardinal, alors il existe une bijection \(f:\interventierii{1}{n}\to E\) et l'application \[\fonctionlambda{\S{n}}{\S{E}}{\sigma}{f\rond\sigma\rond f\inv}\] est alors un isomorphisme de groupes.
\end{rem}

\begin{nota}
Soient \(n\in\Ns\) et \(a_1,\dots,a_n\in\interventierii{1}{n}\) deux à deux distincts.

La permutation \(\sigma\) telle que \[\quantifs{\forall k\in\interventierii{1}{n}}\sigma\paren{k}=a_k\] est notée : \[\sigma=\permu{1;2;3;\dots;n}{a_1;a_2;a_3;\dots;a_n}.\]
\end{nota}

\begin{exoex}
Énumérer tous les éléments de \(\S{2}\) et \(\S{3}\) à l'aide de la notation précédente.
\end{exoex}

\begin{corr}
On a : \[\S{2}=\accol{\permu{1;2}{1;2};\permu{1;2}{2;1}}\] et : \[\S{3}=\accol{\permu{1;2;3}{1;2;3};\permu{1;2;3}{2;1;3};\permu{1;2;3}{2;3;1};\permu{1;2;3}{3;1;2};\permu{1;2;3}{3;2;1};\permu{1;2;3}{1;3;2}}.\]
\end{corr}

\begin{exoex}
On pose : \[\sigma_1=\permu{1;2;3;4}{3;1;4;2}\qquad\sigma_2=\permu{1;2;3;4}{4;3;2;1}\qquad\sigma_3=\permu{1;2;3;4;5;6}{3;6;5;4;1;2}.\]

\begin{enumerate}
    \item Calculer les inverses de ces trois permutations. \\
    \item \(\sigma_1\) et \(\sigma_2\) commutent-elles ? \\
    \item Calculer \(\sigma_1^2\) et \(\sigma_1^{2023}\).
\end{enumerate}
\end{exoex}

\begin{corr}[1]
On a : \[\sigma_1\inv=\permu{1;2;3;4}{2;4;1;3}\qquad\sigma_2\inv=\permu{1;2;3;4}{4;3;2;1}\qquad\sigma_3\inv=\permu{1;2;3;4;5;6}{5;6;1;4;3;2}.\]
\end{corr}

\begin{corr}[2]
On a : \[\sigma_1\rond\sigma_2=\permu{1;2;3;4}{2;4;1;3}\qquad\text{et}\qquad\sigma_2\rond\sigma_1=\permu{1;2;3;4}{2;4;1;3}\] donc \(\sigma_1\) et \(\sigma_2\) commutent.
\end{corr}

\begin{corr}[3]
On remarque : \[\sigma_1^2=\permu{1;2;3;4}{4;3;2;1}\qquad\sigma_1^3=\permu{1;2;3;4}{2;4;1;3}\qquad\sigma_1^4=\permu{1;2;3;4}{1;2;3;4}=\id{}.\]

Donc, comme \(2023\equiv3\croch{4}\), on a : \[\sigma_1^{2023}=\sigma_1^3=\permu{1;2;3;4}{2;4;1;3}.\]
\end{corr}

\chapter{Déterminants}

\minitoc

On considère un corps \(\K\) (en pratique, \(\K=\R\) ou \(\C\)).

\section{Multilinéarité}

\subsection{Formes multilinéaires}

\begin{defi}[Application multilinéaire]
Soient \(E_1,\dots,E_r,F\) des \(\K\)-espaces vectoriels où \(r\in\Ns\) et une fonction \[\fonction{f}{E_1\times\dots\times E_r}{F}{\paren{x_1,\dots,x_r}}{f\paren{x_1,\dots,x_r}}\]

On dit que \(f\) est une fonction multilinéaire (ou, plus précisément, est \(r\)-linéaire) si elle est linéaire par rapport à chacune de ses \(r\) variables, \cad si l'on a : \[\begin{aligned}
&\quantifs{\forall j\in\interventierii{1}{r};\forall\lambda,\mu\in\K;\forall\paren{x_1,\dots,x_r}\in E_1\times\dots\times E_r;\forall y_j\in E_j} \\
&f\paren{x_1,\dots,x_{j-1},\lambda x_j+\mu y_j,x_{j+1},\dots,x_r}=\lambda f\paren{x_1,\dots,x_{j-1},x_j,x_{j+1},\dots,x_r}+\mu f\paren{x_1,\dots,x_{j-1},y_j,x_{j+1},\dots,x_r}.
\end{aligned}\]

Si, de plus, \(F=\K\) alors on dit que \(f\) est une forme \(r\)-linéaire.
\end{defi}

\begin{rem}
Soient \(E_1,\dots,E_r,F\) des \(\K\)-espaces vectoriels où \(r\in\Ns\) et une fonction \(r\)-linéaire \[f:E_1\times\dots\times E_r\to F.\]

\begin{itemize}
    \item On a : \[\quantifs{\forall\paren{x_1,\dots,x_r}\in E_1\times\dots\times E_r}\paren{\quantifs{\exists i\in\interventierii{1}{r}}x_i=0_{E_i}}\imp f\paren{x_1,\dots,x_r}=0_F.\] En français : \guillemets{Si (au moins) l'un des \(x_i\) est nul alors \(f\paren{x_1,\dots,x_r}\) est nul.} \\
    \item On a : \[\quantifs{\forall\lambda\in\K;\forall\paren{x_1,\dots,x_r}\in E_1\times\dots\times E_r}f\paren{\lambda x_1,\dots,\lambda x_r}=\lambda^rf\paren{x_1,\dots,x_r}.\] En particulier, une application \(r\)-linéaire avec \(r\geq2\) n'est pas linéaire.
\end{itemize}
\end{rem}

\begin{ex}
Les applications suivantes sont multilinéaires :

\begin{itemize}
    \item Toute application linéaire est \(1\)-linéaire. \\
    \item Le produit matriciel et la composition des applications linéaires sont \(2\)-linéaires (on dit aussi bilinéaires). Par exemple, si \(n\in\Ns\) et \(E\) est un espace vectoriel : \[\fonctionlambda{\M{n}\times\M{n}}{\M{n}}{\paren{A,B}}{A\times B}\qquad\text{et}\qquad\fonctionlambda{\Lendo{E}\times\Lendo{E}}{\Lendo{E}}{\paren{u,v}}{u\rond v}\] sont bilinéaires. \\
    \item L'application \[\fonctionlambda{\M{n}^3}{\M{n}}{\paren{A,B,C}}{A\times B\times C}\] est \(3\)-linéaire (on dit aussi trilinéaire).
\end{itemize}
\end{ex}

\chapter{Séries, familles sommables}

\minitoc

On pose \(\K=\R\) ou \(\C\).

Dans ce chapitre, on définit la valeur d'une \guillemets{somme infinie} dans \(\K\) de la forme \(x_0+x_1+x_2+\dots\) de deux façons :

\begin{itemize}
    \item une première façon au \hyperref[sec:séries]{paragraphe 1} où elle est notée \(\sum_{n=0}^{\pinf}x_n\) ; \\
    \item une seconde au paragraphe \hyperref[sec:famillesSommablesDeRéelsPositifs]{paragraphe 5} où elle est notée \(\sum_{n\in\N}x_n\).
\end{itemize}

La grande différence entre les deux façons est que l'ordre dans lequel sont énumérés les termes de la somme est important dans la première ; pas dans la seconde.

\section{Séries}\label{sec:séries}

\subsection{Deux exemples}

\begin{ex}[Série géométrique]
On a la série géométrique de raison \(\dfrac{1}{2}\) : \[\dfrac{1}{2}+\dfrac{1}{4}+\dfrac{1}{8}+\dfrac{1}{16}+\dots=1.\]
\end{ex}

\begin{ex}[Série harmonique]
On a la série harmonique : \[\dfrac{1}{1}+\dfrac{1}{2}+\dfrac{1}{3}+\dfrac{1}{4}+\dfrac{1}{5}+\dots=\pinf.\]
\end{ex}

\subsection{Séries convergentes, séries divergentes}

\begin{defi}[Série convergente, série divergente]
À toute suite \(\paren{x_n}_{n\in\N}\in\K^\N\) on associe une série notée \[\sum x_n\qquad\text{ou}\qquad\sum_nx_n\qquad\text{ou}\qquad\sum_{n\geq0}x_n\] et appelée série de terme général \(x_n\).

La suite \(\paren{S_n}_n\) des sommes partielles de la série \(\sum_nx_n\) est définie par : \[\quantifs{\forall n\in\N}S_n=\sum_{k=0}^nx_k.\]

On dit que la série \(\sum_nx_n\) converge ou est convergente si la suite \(\paren{S_n}_n\) de ses sommes partielles est convergente.

On définit alors la somme de la série \(\sum_nx_n\) par : \[\sum_{n=0}^{\pinf}x_n=\lim_{N\to\pinf}\sum_{n=0}^Nx_n.\]

Sinon, on dit que la série \(\sum_nx_n\) diverge ou est divergente.

Déterminer la nature d'une série ou étudier sa convergence, c'est décider si elle est convergente ou divergente.
\end{defi}

\begin{rem}
(Mêmes notations)

La série \(\sum_nx_n\) est toujours définie ; sa somme \(\sum_{n=0}^{\pinf}x_n\) n'est définie que si la série est convergente.
\end{rem}

\begin{rem}
En pratique, la suite \(\paren{x_n}_n\) n'est pas toujours définie à partir du rang \(n=0\).

Si la suite \(\paren{x_n}_{n\in\interventierie{n_0}{\pinf}}\in\K^{\interventierie{n_0}{\pinf}}\) est définie à partir d'un certain rang \(n_0\in\N\), on lui associe la série notée \[\sum x_n\qquad\text{ou}\qquad\sum_nx_n\qquad\text{ou}\qquad\sum_{n\geq n_0}x_n\] dont la suite \(\paren{S_n}_{n\geq n_0}\) des sommes partielles est définie par : \[\quantifs{\forall n\in\interventierie{n_0}{\pinf}}S_n=\sum_{k=n_0}^nx_k.\]

On dit que la série \(\sum_{n\geq n_0}x_n\) converge ou est convergente si la suite \(\paren{S_n}_n\) de ses sommes partielles est convergente.

On définit alors la somme de la série \(\sum_{n\geq n_0}x_n\) par : \[\sum_{n=n_0}^{\pinf}x_n=\lim_{N\to\pinf}\sum_{n=n_0}^Nx_n.\]

Dans la suite du cours, on suppose que la suite \(\paren{x_n}_n\) est définie à partir de \(n=0\) pour simplifier l'exposé.
\end{rem}

\begin{rem}[Tronquer une série]\thlabel{rem:tronquerUneSérie}
Soient \(\sum_{n\geq0}x_n\) une série et \(n_0\in\N\).

Les séries \(\sum_{n\geq0}x_n\) et \(\sum_{n\geq n_0}x_n\) sont de même nature (on dit qu'\guillemets{on ne change pas la nature d'une série en la tronquant}).

Si elles convergent, leurs sommes respectives vérifient : \[\sum_{n=0}^{\pinf}x_n=\sum_{n=0}^{n_0-1}x_n+\sum_{n=n_0}^{\pinf}x_n.\]
\end{rem}

\begin{dem}
On note \(\paren{S_n}_n\) et \(\paren{S_n\prim}_n\) les suites des sommes partielles respectives des séries \(\sum_{n\geq0}x_n\) et \(\sum_{n\geq n_0}x_n\).

On a : \[\quantifs{\forall n\geq n_0}S_n=\sum_{k=0}^nx_k=\sum_{k=0}^{n_0-1}x_k+S_n\prim.\]

Donc : \[\paren{S_n}_n\text{ converge}\ssi\paren{S_n\prim}_n\text{ converge}.\]

D'où : \[\sum_{n\geq0}x_n\text{ converge}\ssi\sum_{n\geq n_0}x_n\text{ converge}\] et, en cas de convergence, \(\lim_nS_n=\sum_{k=0}^{n_0-1}x_k+\lim_nS_n\prim\), \cad : \[\sum_{n=0}^{\pinf}x_n=\sum_{n=0}^{n_0-1}x_n+\sum_{n=n_0}^{\pinf}x_n.\]
\end{dem}

\begin{defprop}[Restes d'une série convergente]
Soit \(\sum_{n\geq0}x_n\) une série convergente.

On définit la suite \(\paren{R_n}_{n\in\N}\) des restes de cette série : \[\quantifs{\forall n\in\N}R_n=\sum_{k=n+1}^{\pinf}x_k.\]

Celle-ci vérifie : \[\quantifs{\forall n\in\N}S_n+R_n=\sum_{k=0}^{\pinf}x_k\] et : \[\lim_{n\to\pinf}R_n=0.\]
\end{defprop}

\begin{dem}
Pour tout \(n\in\N\), \(R_n\) est bien défini car la série \(\sum_nx_n\) converge.

D'après la \thref{rem:tronquerUneSérie}, on a bien : \[\quantifs{\forall n\in\N}S_n+R_n=\sum_{k=0}^{\pinf}x_k.\]

D'où : \[\begin{aligned}
\quantifs{\forall n\in\N}R_n&=\sum_{k=0}^{\pinf}-S_n \\
&\tendqd{n\to\pinf}\sum_{k=0}^{\pinf}x_k-\sum_{k=0}^{\pinf}x_k \\
&=0.
\end{aligned}\]
\end{dem}

\begin{prop}[Linéarité de la somme d'une série]
Soient \(\lambda,\mu\in\K\) et \(\paren{x_n}_n,\paren{y_n}_n\in\K^\N\).

Si les séries \(\sum_{n\geq0}x_n\) et \(\sum_{n\geq0}y_n\) sont convergentes alors la série \(\sum_{n\geq0}\paren{\lambda x_n+\mu y_n}\) l'est aussi et sa somme est : \[\sum_{n=0}^{\pinf}\paren{\lambda x_n+\mu y_n}=\lambda\sum_{n=0}^{\pinf}x_n+\mu\sum_{n=0}^{\pinf}y_n.\]
\end{prop}

\begin{dem}
\note{Exercice}
\end{dem}

\begin{rem}
\begin{itemize}
    \item La somme de deux séries convergentes est une série convergente. \\
    \item La somme d'une série convergente et d'une série divergente est une série divergente. \\
    \item Dans le cas de la somme de deux séries divergentes, on ne peut pas conclure a priori.
\end{itemize}
\end{rem}

\subsection{Séries grossièrement divergentes}

\begin{prop}\thlabel{prop:sérieConvergenteImpliqueLimiteDuTermeGénéralNulle}
Soit \(\paren{x_n}_{n\in\N}\in\K^\N\).

On a : \[\sum_nx_n\text{ converge}\imp\lim_{n\to\pinf}x_n=0.\]
\end{prop}

\begin{dem}
Supposons \(\sum_nx_n\) convergente, \cad \(\lim_nS_n=\sum_{k=0}^{\pinf}x_k\in\K\).

On remarque \(\quantifs{\forall n\in\Ns}x_n=S_n-S_{n-1}\) donc : \[\lim_nx_n=\lim_nS_n-\lim_nS_{n-1}=0.\]
\end{dem}

\begin{defi}
Soit \(\paren{x_n}_{n\in\N}\in\K^\N\).

Si la suite \(\paren{x_n}_n\) ne tend pas vers \(0\), la série \(\sum_nx_n\) est dite grossièrement divergente.

Toute série grossièrement divergente est divergente.
\end{defi}

\begin{rem}
L'implication réciproque de la \thref{prop:sérieConvergenteImpliqueLimiteDuTermeGénéralNulle} est fausse.

Autrement dit : pour qu'une série soit convergente, il ne suffit pas qu'elle ne soit pas grossièrement divergente.
\end{rem}

\begin{dem}
La série harmonique \(\sum_{n\geq1}\dfrac{1}{n}\) n'est pas grossièrement divergente mais divergente.
\end{dem}

\subsection{Exemples}

\subsubsection{Séries géométriques}

\begin{defi}[Série géométrique]
Soit \(\paren{x_n}_{n\in\N}\in\K^\N\).

On dit que \(\sum_nx_n\) est une série géométrique si la suite \(\paren{x_n}_n\) est une suite géométrique : \[\quantifs{\exists q\in\C;\forall n\in\N}x_{n+1}=qx_n.\]

L'élément \(q\) est appelé raison de la série.
\end{defi}

\begin{prop}[Convergence des séries géométriques]
Soit \(q\in\C\).

On a : \[\sum_nq^n\text{ converge}\ssi\abs{q}<1.\]

Sa somme vaut alors \[\sum_{k=0}^{\pinf}q^k=\dfrac{1}{1-q}.\]

Si \(\abs{q}\geq1\), la série géométrique est en fait grossièrement divergente.
\end{prop}

\begin{dem}
Rappel : la suite géométrique \(\paren{q^n}_{n\in\N}\) \(\begin{dcases}
\text{converge vers }0 &\text{si }\abs{q}<1 \\
\text{converge vers }1 &\text{si }\abs{q}=1 \\
\text{diverge} &\text{si }\begin{dcases}
\abs{q}\geq1 \\
q\not=1
\end{dcases}
\end{dcases}\)

Si \(\abs{q}\geq1\) alors \(\sum_nq^n\) est grossièrement divergente donc divergente.

Si \(\abs{q}<1\), on a : \[\begin{aligned}
\quantifs{\forall n\in\N}S_n&=\sum_{k=0}^nq^k \\
&=\dfrac{1-q^{n+1}}{1-q} \\
&\tendqd{n\to\pinf}\dfrac{1}{1-q}.
\end{aligned}\]
\end{dem}

\begin{rem}
Soient \(q\in\C\) tel que \(\abs{q}<1\) et \(n_0\in\N\).

On a : \[\sum_{n=n_0}^{\pinf}q^n=q^{n_0}\sum_{n=n_0}^{\pinf}q^{n-n_0}=q^{n_0}\sum_{k=0}^{\pinf}q^k=\dfrac{q^{n_0}}{1-q}.\]
\end{rem}

\subsubsection{Séries télescopiques}

\begin{defi}[Série télescopique]
Une série télescopique est une série écrite sous la forme \[\sum_n\paren{u_{n+1}-u_n}\] où \(\paren{u_n}_n\in\K^\N\).
\end{defi}

\begin{prop}[Convergence des séries télescopiques]
Soit \(\paren{u_n}_n\in\K^\N\).

On a : \[\sum_n\paren{u_{n+1}-u_n}\text{ converge}\ssi\paren{u_n}_n\text{ converge}.\]

Sa somme vaut alors \[\sum_{k=0}^{\pinf}\paren{u_{k+1}-u_k}=\lim_nu_n-u_0.\]
\end{prop}

\begin{dem}
On remarque \(\quantifs{\forall n\in\N}\sum_{k=0}^n\paren{u_{k+1}-u_k}=u_{n+1}-u_0\) donc : \[\begin{aligned}
\sum_n\paren{u_{n+1}-u_n}\text{ converge}&\ssi\paren{u_{n+1}-u_0}_n\text{ converge} \\
&\ssi\paren{u_n}_n\text{ converge}.
\end{aligned}\]

On a alors : \[\begin{aligned}
\sum_{k=0}^{\pinf}\paren{u_{k+1}-u_k}&=\lim_n\paren{u_{n+1}-u_0} \\
&=\lim_nu_n-u_0.
\end{aligned}\]
\end{dem}

\begin{exoex}
Calculer : \[\sum_{n=1}^{\pinf}\dfrac{1}{n\paren{n+1}}.\]
\end{exoex}

\begin{corr}
On a : \[\begin{aligned}
\quantifs{\forall n\in\Ns}\sum_{k=1}^n\dfrac{1}{k\paren{k+1}}&=\sum_{k=1}^n\paren{\dfrac{1}{k}-\dfrac{1}{k+1}} \\
&=1-\dfrac{1}{n+1} \\
&\tendqd{n\to\pinf}1.
\end{aligned}\]

Donc \(\sum_n\dfrac{1}{n\paren{n+1}}\) converge et on a \(\sum_{n=1}^{\pinf}\dfrac{1}{n\paren{n+1}}=1\).
\end{corr}

\begin{exoex}
Calculer : \[\sum_{n=1}^{\pinf}\dfrac{n}{\paren{n+1}!}.\]
\end{exoex}

\begin{corr}
On a : \[\begin{aligned}
\quantifs{\forall n\in\Ns}\sum_{k=1}^n\dfrac{k}{\paren{k+1}!}&=\sum_{k=1}^n\paren{\dfrac{1}{k!}-\dfrac{1}{\paren{k+1}!}} \\
&=1-\dfrac{1}{\paren{n+1}!} \\
&\tendqd{n\to\pinf}1.
\end{aligned}\]

Donc \(\sum_n\dfrac{n}{\paren{n+1}!}\) converge et on a \(\sum_{n=1}^{\pinf}\dfrac{n}{\paren{n+1}!}=1\).
\end{corr}

\begin{rem}
Toute suite peut être vue comme la suite des sommes partielles d'une série.

Cela permet d'appliquer aux suites les outils dont on dispose pour les séries (\cf TD).
\end{rem}

\begin{dem}
Soit \(\paren{u_n}_n\in\K^\N\).

Posons \(u_{-1}=0\).

La suite des sommes partielles de la série \(\sum_{n\geq0}\paren{u_n-u_{n-1}}\) est \(\paren{u_n}_{n\geq0}\).

En effet : \[\quantifs{\forall n\in\N}\sum_{k=0}^n\paren{u_k-u_{k-1}}=u_n-u_{-1}=u_n.\]
\end{dem}

\subsubsection{Séries de Riemann}

\begin{defi}[Série de Riemann]
On appelle série de Riemann une série de la forme \(\sum_n\dfrac{1}{n^\alpha}\) où \(\alpha\in\R\).

Si \(\alpha=1\), la série de Riemann \(\sum_n\dfrac{1}{n}\) est appelée série harmonique.
\end{defi}

\begin{ex}
On a vu au DS 9, question 21 : \[\sum_{n=1}^{\pinf}\dfrac{1}{n^2}=\dfrac{\pi^2}{6}.\]
\end{ex}

\begin{rem}
Soit \(\alpha\in\R\).

On montrera au \hyperref[subsubsec:applicationAuxSériesDeRiemann]{paragraphe 2.3.2} que la série de Riemann \(\sum_{n}\dfrac{1}{n^\alpha}\) est convergente si, et seulement si, \(\alpha>1\).

Elle est grossièrement divergente si, et seulement si, \(\alpha\leq0\).
\end{rem}

\begin{nota}[Fonction zêta de Riemann]
On pose : \[\quantifs{\forall\alpha\in\intervee{1}{\pinf}}\zeta\paren{\alpha}=\sum_{n=1}^{\pinf}\dfrac{1}{n^\alpha}.\]
\end{nota}

\subsubsection{Autres exemples}

\begin{rappel}[Inégalité de Taylor-Lagrange (\thref{cor:inégalitéDeTaylorLagrange})]
Soient \(I\) un intervalle de \(\R\), \(n\in\N\), \(f\in\ensclasse{n+1}{I}{\K}\), \(M\in\Rp\) et \(a,b\in I\).

On suppose : \[\quantifs{\forall t\in I}\abs{f\deriv{n+1}\paren{t}}\leq M.\]

On a : \[\abs{f\paren{b}-\sum_{k=0}^n\dfrac{f\deriv{k}\paren{a}}{k!}\paren{b-a}^k}\leq\dfrac{M\abs{b-a}^{n+1}}{\paren{n+1}!}.\]
\end{rappel}

\begin{prop}
On a, pour tous \(x\in\R\) : \[\e{x}=\sum_{n=0}^{\pinf}\dfrac{x^n}{n!}\qquad\cos x=\sum_{n=0}^{\pinf}\dfrac{\paren{-1}^nx^{2n}}{\paren{2n}!}\qquad\sin x=\sum_{n=0}^{\pinf}\dfrac{\paren{-1}^nx^{2n+1}}{\paren{2n+1}!}.\]
\end{prop}

\begin{dem}
Soit \(x\in\R\).

Montrons que \(\sum_{n=0}^{\pinf}\dfrac{x^n}{n!}=\e{x}\), \cad \(\sum_n\dfrac{x^n}{n!}\) converge et sa somme vaut \(\e{x}\), \cad \(\lim_n\sum_{k=0}^n\dfrac{x^k}{k!}=\e{x}\).

Soit \(n\in\N\).

On a : \[\quantifs{\forall t\in\intervii{-\abs{x}}{\abs{x}}}\abs{\exp\deriv{n+1}\paren{t}}=\e{t}\leq\e{\abs{x}}.\]

Donc, selon l'inégalité de Taylor-Lagrange appliquée entre \(0\) et \(x\) : \[\abs{\e{x}-\sum_{k=0}^n\dfrac{1}{k!}\paren{x-0}^k}\leq\dfrac{\e{\abs{x}}\abs{x-0}^{n+1}}{\paren{n+1}!}.\]

Or \(x^{n+1}\egqd{n\to\pinf}\o{\paren{n+1}!}\) donc : \[\lim_n\dfrac{\e{\abs{x}}\abs{x-0}^{n+1}}{\paren{n+1}!}=0.\]

Donc, selon le théorème des gendarmes : \[\lim_n\sum_{k=0}^n\dfrac{x^k}{k!}=\sum_{k=0}^{\pinf}\dfrac{x^k}{k!}=\e{x}.\]

De plus, on a : \[\quantifs{\forall t\in\R}\abs{\cos\deriv{2n+1}\paren{t}}\leq1.\]

Donc, selon l'inégalité de Taylor-Lagrange appliquée entre \(0\) et \(x\) : \[\abs{\cos x-\sum_{k=0}^{2n}\dfrac{\cos\deriv{k}\paren{0}}{k!}\paren{x-0}^k}\leq\dfrac{\abs{x-0}^{2n+1}}{\paren{2n+1}!}.\]

Or \(x^{2n+1}\egqd{n\to\pinf}\o{\paren{2n+1}!}\) donc : \[\lim_n\dfrac{\abs{x-0}^{2n+1}}{\paren{2n+1}!}=0.\]

Donc, selon le théorème des gendarmes : \[\lim_n\sum_{k=0}^n\dfrac{\paren{-1}^kx^{2k}}{\paren{2k}!}=\sum_{k=0}^{\pinf}\dfrac{\paren{-1}^kx^{2k}}{\paren{2k}!}=\cos x.\]

Idem pour \(\sin\).
\end{dem}

\section{Séries à termes positifs}

\subsection{Convergence des séries à termes positifs}

\begin{defi}
Une série à termes positifs est une série de la forme \(\sum_nx_n\) où \(\paren{x_n}_n\in\paren{\Rp}^\N\) est une suite de réels positifs.
\end{defi}

\begin{rappel}[Théorème de la limite monotone (\thref{theo:théorèmeDeLaLimiteMonotone})]\thlabel{rappel:théorèmeDeLaLimiteMonotone}
Soit \(\paren{u_n}_n\in\R^\N\) une suite réelle croissante.

Alors elle admet une limite et on a : \[\lim_nu_n=\begin{dcases}
\sup_{n\in\N}u_n &\text{si la suite est majorée} \\
\pinf &\text{sinon}
\end{dcases}\]
\end{rappel}

\begin{theo}\thlabel{theo:convergenceDesSériesATermesPositifs}
Une série à termes positifs converge si, et seulement si, la suite de ses sommes partielles est majorée.
\end{theo}

\begin{dem}
Notons \(\paren{S_n}_n\) la suite des sommes partielles de la série \(\sum_nx_n\).

On remarque : \[\quantifs{\forall n\in\N}S_{n+1}=\sum_{k=0}^{n+1}x_k=S_n+x_{n+1}\geq S_n.\]

Donc \(\paren{S_n}_n\) est croissante.

D'où : \[\begin{WithArrows}
\sum_nx_n\text{ converge}&\ssi\paren{S_n}_n\text{ converge} \Arrow{\thref{rappel:théorèmeDeLaLimiteMonotone}} \\
&\ssi\paren{S_n}_n\text{ est majorée}.
\end{WithArrows}\]
\end{dem}

\begin{nota}
Soit \(\sum_nx_n\) une série à termes positifs.

Si la série est divergente, on s'autorise à écrire : \[\sum_{n=0}^{\pinf}x_n=\pinf.\]

Cette notation est relativement naturelle car la suite des sommes partielles de la série tend vers \(\pinf\) (mais elle est réservée au cas des séries à termes positifs divergentes).
\end{nota}

\subsection{Comparaison des séries à termes positifs}

\begin{theo}[Théorème de comparaison des séries à termes positifs]\thlabel{theo:théorèmeDeComparaisonDesSériesATermesPositifs}
Soient \(\sum_nx_n\) et \(\sum_ny_n\) deux séries à termes positifs.

\begin{enumerate}
    \item Si \(\begin{dcases}
        \quantifs{\forall n\in\N}x_n\leq y_n \\
        \sum_ny_n\text{ converge}
    \end{dcases}\) alors \(\sum_nx_n\) converge. \\
    \item Si \(\begin{dcases}
        x_n\egqd{n\to\pinf}\O{y_n} \\
        \sum_ny_n\text{ converge}
    \end{dcases}\) alors \(\sum_nx_n\) converge. \\
    \item Si \(x_n\simqd{n\to\pinf}y_n\) alors \(\sum_nx_n\) et \(\sum_ny_n\) sont de même nature. \\
    \item Si \(\begin{dcases}
        \quantifs{\forall n\in\N}x_n\leq y_n \\
        \sum_nx_n\text{ diverge}
    \end{dcases}\) alors \(\sum_ny_n\) diverge. \\
    \item Si \(\begin{dcases}
        x_n\egqd{n\to\pinf}\O{y_n} \\
        \sum_nx_n\text{ diverge}
    \end{dcases}\) alors \(\sum_ny_n\) diverge.
\end{enumerate}
\end{theo}

\begin{dem}[1]
Supposons \(\quantifs{\forall n\in\N}x_n\leq y_n\) et \(\sum_ny_n\) converge.

Selon le \thref{theo:convergenceDesSériesATermesPositifs}, il existe \(M\in\Rp\) tel que \[\quantifs{\forall n\in\N}\sum_{k=0}^ny_k\leq M.\]

Donc : \[\quantifs{\forall n\in\N}\sum_{k=0}^nx_k\leq M.\]

Donc \(\sum_nx_n\) converge car c'est une série à termes positifs dont la suite des sommes partielles est majorée.
\end{dem}

\begin{dem}[2]
Découle du (1) car si \(x_n\egqd{n\to\pinf}\O{y_n}\) alors il existe \(K\in\Rp\) tel que \(\quantifs{\forall n\in\N}x_n\leq Ky_n\) et si \(\sum_ny_n\) converge alors \(\sum_nKy_n\) aussi.
\end{dem}

\begin{dem}[3]
Découle de (2) car \(x_n\simqd{n\to\pinf}y_n\imp\begin{dcases}
x_n\egqd{n\to\pinf}\O{y_n} \\
y_n\egqd{n\to\pinf}\O{x_n}
\end{dcases}\)
\end{dem}

\begin{dem}[4]
Contraposée de (1).
\end{dem}

\begin{dem}[5]
Contraposée de (2).
\end{dem}

\begin{rem}
Le (1) est suffisant pour énoncer le théorème.
\end{rem}

\begin{rem}
Attention à ne surtout pas utiliser le \thref{theo:convergenceDesSériesATermesPositifs} pour comparer des séries quelconques (\cf pour un contre-exemple).

On peut comparer des séries à termes positifs, ou plus généralement, des séries dont les termes sont de signe constant à partir d'un certain rang.
\end{rem}

\begin{exoex}
Soit \(q\in\Rps\).

Étudier la nature des séries suivantes :

\begin{enumerate}
    \item \(\sum_n\sin\dfrac{1}{2^n}\) \\
    \item \(\sum_n\dfrac{q^n}{1+q^n}\)
\end{enumerate}
\end{exoex}

\begin{corr}[1]~\\
On a \(\begin{dcases}
\sin h\simqd{h\to0}h \\
\lim_n\dfrac{1}{2^n}=0
\end{dcases}\) donc \(\sin\dfrac{1}{2^n}\simqd{n\to\pinf}\dfrac{1}{2^n}\).

Or la série géométrique \(\sum_n\dfrac{1}{2^n}\) converge donc selon le théorème de comparaison des séries à termes positifs, \(\sum_n\sin\dfrac{1}{2^n}\) converge.
\end{corr}

\begin{corr}[2]
On a, quand \(n\to\pinf\) : \[1+q^n\sim\begin{dcases}
1 &\text{si }q<1 \\
2 &\text{si }q=1 \\
q^n &\text{si }q>1
\end{dcases}\] donc : \[\dfrac{q^n}{1+q^n}\sim\begin{dcases}
q^n &\text{si }q<1 \\
\dfrac{1}{2} &\text{si }q=1 \\
1 &\text{si }q>1
\end{dcases}\]

Si \(q\geq1\) : \(\sum_n\dfrac{q^n}{1+q^n}\) diverge grossièrement.

Sinon, selon le théorème de comparaison des séries à termes positifs : \[\sum_n\dfrac{q^n}{1+q^n}\text{ converge}\ssi\sum_nq^n\text{ converge}.\]

Or la série géométrique \(\sum_nq^n\) converge car \(\abs{q}<1\) donc \(\sum_n\dfrac{q^n}{1+q^n}\) converge.

Conclusion : \[\sum_n\dfrac{q^n}{1+q^n}\text{ converge}\ssi q<1.\]
\end{corr}

\chapter{Espaces préhilbertiens}

\minitoc

\section{Produit scalaire, norme associée}

\subsection{Produit scalaire}

\begin{defi}[Produit scalaire]
Soit \(E\) un \(\R\)-espace vectoriel.

On appelle produit scalaire sur \(E\) tout forme bilinéaire symétrique définie positive, \cad toute application \[\fonction{\ps{\cdot}{\cdot}}{E\times E}{\R}{\paren{x,y}}{\ps{x}{y}}\] qui vérifie : \[\begin{dcases}
\ps{\cdot}{\cdot}\text{ est bilinéaire : }\begin{dcases}
\quantifs{\forall\lambda,\mu\in\R;\forall x_1,x_2,y\in E}\ps{\lambda x_1+\mu x_2}{y}=\lambda\ps{x_1}{y}+\mu\ps{x_2}{y} \\
\quantifs{\forall\lambda,\mu\in\R;\forall x,y_1,y_2\in E}\ps{x}{\lambda y_1+\mu y_2}=\lambda\ps{x}{y_1}+\mu\ps{x}{y_2}
\end{dcases} \\
\ps{\cdot}{\cdot}\text{ est symétrique : }\quantifs{\forall x,y\in E}\ps{x}{y}=\ps{y}{x} \\
\ps{\cdot}{\cdot}\text{ est définie positive : }\quantifs{\forall x\in E}\begin{dcases}
\ps{x}{x}\geq0 \\
\ps{x}{x}=0\ssi x=0_E
\end{dcases}
\end{dcases}\]

Le produit scalaire de deux vecteurs \(x\) et \(y\) est traditionnellement noté \(\ps{x}{y}\), \(\left(x\tq y\right)\), \(\left\langle x,y\right\rangle\) ou \(x\cdot y\). On le notera \(\ps{x}{y}\) dans tout ce cours.
\end{defi}

\begin{defi}[Espace euclidien]
On appelle espace préhilbertien (réel) tout \(\R\)-espace vectoriel muni d'un produit scalaire.

On appelle espace euclidien tout \(\R\)-espace vectoriel de dimension finie muni d'un produit scalaire.
\end{defi}

\begin{ex}\thlabel{ex:produitsScalaires}
Les exemples suivants sont à connaître parfaitement :

\begin{enumerate}
\item Soit \(n\in\Ns\). L'application : \[\fonctionlambda{\R^n\times\R^n}{\R}{\paren{\tcoords{x_1}{\vdots}{x_n},\tcoords{x_1\prim}{\vdots}{x_n\prim}}}{x_1x_1\prim+\dots+x_nx_n\prim}\] est un produit scalaire sur \(\R^n\) appelé le produit scalaire canonique de \(\R^n\).

Ce produit scalaire s'écrit aussi : \[\fonctionlambda{\R^n\times\R^n}{\R}{\paren{X,X\prim}}{\trans{X}X\prim}\]

\item Soient \(a,b\in\R\) tels que \(a<b\). On pose \(E=\ensclasse{0}{\intervii{a}{b}}{\R}\). L'application : \[\fonctionlambda{E\times E}{\R}{\paren{f,g}}{\int_a^bf\paren{t}g\paren{t}\odif{t}}\] est un produit scalaire sur \(E\). \\

\item L'application : \[\fonctionlambda{\M{n}[\R]\times\M{n}[\R]}{\R}{\paren{A,B}}{\tr\paren{\trans{A}B}}\] est un produit scalaire sur \(\M{n}[\R]\) appelé le produit scalaire canonique de \(\M{n}[\R]\).

On a, pour toutes matrices \(A=\paren{a_{ij}}_{\paren{i,j}\in\interventierii{1}{n}^2}\) et \(B=\paren{b_{ij}}_{\paren{i,j}\in\interventierii{1}{n}^2}\) : \[\tr\paren{\trans{A}B}=\sum_{i=1}^n\sum_{j=1}^na_{ij}b_{ij}.\]

\item Plus généralement, l'application : \[\fonctionlambda{\M{np}[\R]\times\M{np}[\R]}{\R}{\paren{A,B}}{\tr\paren{\trans{A}B}}\] est un produit scalaire sur \(\M{np}[\R]\) appelé le produit scalaire canonique de \(\M{np}[\R]\).

On a, pour toutes matrices \(A=\paren{a_{ij}}_{\paren{i,j}\in\interventierii{1}{n}\times\interventierii{1}{p}}\) et \(B=\paren{b_{ij}}_{\paren{i,j}\in\interventierii{1}{n}\times\interventierii{1}{p}}\) : \[\tr\paren{\trans{A}B}=\sum_{i=1}^n\sum_{j=1}^pa_{ij}b_{ij}.\]
\end{enumerate}
\end{ex}

\begin{dem}[2]\thlabel{dem:produitScalaireFonctionsContinues}
On note \(\phi:E^2\to\R\) l'application.

On a : \[\quantifs{\forall f,g\in E}\phi\paren{f,g}=\phi\paren{g,f}.\]

Donc \(\phi\) est symétrique.

On a : \[\begin{aligned}
\quantifs{\forall\lambda,\mu\in\R;\forall f_1,f_2,g\in E}\phi\paren{\lambda f_1+\mu f_2,g}&=\int_a^b\paren{\lambda f_1\paren{t}+\mu f_2\paren{t}}g\paren{t}\odif{t} \\
&=\lambda\int_a^bf_1\paren{t}g\paren{t}\odif{t}+\mu\int_a^bf_2\paren{t}g\paren{t}\odif{t} \\
&=\lambda\phi\paren{f_1,g}+\mu\phi\paren{f_2,g}.
\end{aligned}\]

Donc \(\phi\) est linéaire à gauche.

Comme \(\phi\) est symétrique, \(\phi\) est aussi linéaire à droite.

Donc \(\phi\) est bilinéaire.

Soit \(f\in E\).

On a : \[\phi\paren{f,f}=\int_a^bf^2\paren{t}\odif{t}\geq0.\]

Donc \(\phi\) est définie positive.

Enfin, si \(\phi\paren{f,f}=0\) alors \(\int_a^bf^2\paren{t}\odif{t}=0\).

Or la fonction \(f^2\) est continue et positive sur \(\intervii{a}{b}\) donc \(f^2=0\) donc \(f=0\).

Finalement, \(\phi\) est un produit scalaire sur \(E\).
\end{dem}

\begin{dem}[Autres exemples]
\note{Exercice}
\end{dem}

\subsection{Norme associée à un produit scalaire}

\begin{nota}
Soit \(E\) un espace préhilbertien.

On pose : \[\quantifs{\forall x\in E}\norme{x}=\sqrt{\ps{x}{x}}.\]

On étudie dans la suite l'application : \[\fonction{\norme{\cdot}}{E}{\R}{x}{\norme{x}=\sqrt{\ps{x}{x}}}\] appelée norme associée au produit scalaire de \(E\).
\end{nota}

\begin{theo}[Inégalité de Cauchy-Schwarz]
Soient \(E\) un espace préhilbertien et \(x,y\in E\).

On a l'inégalité de Cauchy-Schwarz : \[\abs{\ps{x}{y}}\leq\norme{x}\norme{y}.\]

De plus, on a les cas d'égalités dans l'inégalité de Cauchy-Schwarz : \[\abs{\ps{x}{y}}=\norme{x}\norme{y}\ssi x\text{ et }y\text{ sont colinéaires}\] et : \[\ps{x}{y}=\norme{x}\norme{y}\ssi\quantifs{\exists\lambda\in\Rp}\orenv{x=\lambda y \\ y=\lambda x}\]
\end{theo}

\begin{dem}[Inégalité de Cauchy-Schwarz]
Si \(y=0_E\) alors \(\ps{x}{y}=\norme{x}\norme{y}=0\).

Supposons \(y\not=0_E\).

On pose \(\fonction{f}{\R}{\R}{\lambda}{\norme{x+\lambda y}^2}\)

On a : \[\begin{WithArrows}
\quantifs{\forall\lambda\in\R}f\paren{\lambda}&=\ps{x+\lambda y}{x+\lambda y} \Arrow{car \(\ps{\cdot}{\cdot}\) est bilinéaire} \\
&=\ps{x}{x}+\lambda\ps{x}{y}+\lambda^2\ps{y}{x}+\ps{y}{y} \Arrow{car \(\ps{\cdot}{\cdot}\) est symétrique} \\
&=\lambda^2\underbrace{\norme{y}^2}_{\not=0}+2\lambda\ps{x}{y}+\norme{x}^2 \Arrow{car \(\ps{\cdot}{\cdot}\) est positif} \\
&\geq0.
\end{WithArrows}\]

Donc \(f\) est une fonction polynomiale de degré \(2\) positive sur \(\R\).

Donc le discriminant de \(\norme{y}^2X^2+2\ps{x}{y}X+\norme{x}^2\) est négatif : \[\Delta=4\ps{x}{y}^2-4\norme{x}^2\norme{y}^2\leq0.\]

D'où \(\abs{\ps{x}{y}}\leq\norme{x}\norme{y}\).
\end{dem}

\begin{dem}[Cas d'égalité en valeur absolue]
Si \(y=0_E\), l'équivalence est vraie.

Supposons \(y=0_E\).

On a : \[\begin{WithArrows}
\abs{\ps{x}{y}}=\norme{x}\norme{y}&\ssi\Delta=0 \\
&\ssi\quantifs{\exists\lambda\in\R}f\paren{\lambda}=0 \\
&\ssi\quantifs{\exists\lambda\in\R}\norme{x+\lambda y}^2=0 \\
&\ssi\quantifs{\exists\lambda\in\R}x=-\lambda y \Arrow{car \(y\not=0_E\)} \\
&\ssi x\text{ et }y\text{ sont colinéaires}.
\end{WithArrows}\]
\end{dem}

\begin{dem}[Cas d'égalité]
On a : \[\begin{aligned}
\ps{x}{y}=\norme{x}\norme{y}&\ssi\begin{dcases}
x\text{ et }y\text{ sont colinéaires} \\
\ps{x}{y}=\norme{x}\norme{y}
\end{dcases} \\
&\ssi\begin{dcases}
\quantifs{\exists\lambda\in\R}\orenv{x=\lambda y \\ y=\lambda x} \\
\ps{x}{y}=\norme{x}\norme{y}
\end{dcases} \\
&\ssi\quantifs{\exists\lambda\in\Rp}\orenv{x=\lambda y \\ y=\lambda x}
\end{aligned}\]
\end{dem}

\begin{ex}
Soient \(a,b\in\R\) tels que \(a<b\) et \(f,g\in\ensclasse{0}{\intervii{a}{b}}{\R}\).

On a : \[\abs{\int_a^bf\paren{t}g\paren{t}\odif{t}}\leq\sqrt{\int_a^bf^2\paren{t}\odif{t}}\sqrt{\int_a^bg^2\paren{t}\odif{t}}.\]
\end{ex}

\begin{dem}
C'est l'inégalité de Cauchy-Schwarz appliquée au produit scalaire (2) de l'\thref{ex:produitsScalaires}.
\end{dem}

\begin{theo}[Inégalité de Minkowski ou inégalité triangulaire pour la norme]
Soient \(E\) un espace préhilbertien et \(x,y\in E\).

On a l'inégalité de Minkowski : \[\norme{x+y}\leq\norme{x}+\norme{y}.\]

De plus, on a le cas d'égalité dans l'inégalité de Minkowski : \[\norme{x+y}=\norme{x}+\norme{y}\ssi\quantifs{\exists\lambda\in\Rp}\orenv{x=\lambda y \\ y=\lambda x}\]
\end{theo}

\begin{dem}[Inégalité de Minkowski]
On a : \[\begin{aligned}
\norme{x+y}\leq\norme{x}+\norme{y}&\ssi\norme{x+y}^2\leq\paren{\norme{x}+\norme{y}}^2 \\
&\ssi\ps{x+y}{x+y}\leq\paren{\norme{x}+\norme{y}}^2 \\
&\ssi\norme{x}^2+2\ps{x}{y}+\norme{y}^2\leq\norme{x}^2+2\norme{x}\norme{y}+\norme{y}^2 \\
&\color{white}\ssi\color{black}\text{ce qui est vrai selon l'inégalité de Cauchy-Schwarz}.
\end{aligned}\]
\end{dem}

\begin{dem}[Cas d'égalité]
Découle du cas d'égalité sans valeur absolue dans l'inégalité de Cauchy-Schwarz.
\end{dem}

\begin{ex}
Soient \(a,b\in\R\) tels que \(a<b\) et \(f,g\in\ensclasse{0}{\intervii{a}{b}}{\R}\).

On a : \[\sqrt{\int_a^b\paren{f\paren{t}+g\paren{t}}^2\odif{t}}\leq\sqrt{\int_a^bf^2\paren{t}\odif{t}}+\sqrt{\int_a^bg^2\paren{t}\odif{t}}.\]
\end{ex}

\begin{dem}
C'est l'inégalité de Minkowski appliquée au produit scalaire (2) de l'\thref{ex:produitsScalaires}.
\end{dem}

\begin{rem}
Soit \(E\) un espace préhilbertien.

La norme \(\norme{\cdot}\) associée au produit scalaire de \(E\) vérifie : \[\begin{dcases}
\quantifs{\forall x\in E}\norme{x}=0\imp x=0_E \\
\quantifs{\forall\lambda\in\R;\forall x\in E}\norme{\lambda x}=\abs{\lambda}\norme{x} \\
\quantifs{\forall x,y\in E}\norme{x+y}\leq\norme{x}+\norme{y}
\end{dcases}\]

En deuxième année, vous étudierez les fonction \(E\to\Rp\) vérifiant ces trois propriétés. Une telle fonction sur un \(\R\) (ou \(\C\))-espace vectoriel est appelée une \guillemets{norme}.

Ainsi, dans ce paragraphe, on a montré que la \guillemets{norme associée à un produit scalaire} est bien ce qu'on appelle une \guillemets{norme}.
\end{rem}

\begin{dem}
Soit \(x\in E\).

On a : \[\begin{aligned}
\norme{x}=0&\ssi\sqrt{\ps{x}{x}}=0 \\
&\ssi\ps{x}{x}=0 \\
&\ssi x=0.
\end{aligned}\]

Soit \(\lambda\in\R\).

On a : \[\begin{aligned}
\norme{\lambda x}&=\sqrt{\ps{\lambda x}{\lambda x}} \\
&=\sqrt{\lambda^2\ps{x}{x}} \\
&=\abs{\lambda}\norme{x}.
\end{aligned}\]
\end{dem}

\begin{defprop}[Distance]
Soit \(E\) un espace préhilbertien.

On appelle distance associée au produit scalaire de \(E\) l'application : \[\fonction{d}{E\times E}{\Rp}{\paren{x,y}}{\norme{y-x}}\]

Elle vérifie : \[\begin{dcases}
\quantifs{\forall x,y\in E}d\paren{x,y}=0\ssi x=y \\
\quantifs{\forall x,y\in E}d\paren{x,y}=d\paren{y,x} \\
\quantifs{\forall x,y,z\in E}d\paren{x,z}\leq d\paren{x,y}+d\paren{y,z}
\end{dcases}\]
\end{defprop}

\subsection{Propriétés}

Soit \(E\) un espace préhilbertien dont on note \(\ps{\cdot}{\cdot}\) le produit scalaire et \(\norme{\cdot}\) la norme associée.

\begin{prop}\thlabel{prop:carréDeLaNormeD'UneSommeOuD'uneDifférence}
Soient \(x,y\in E\).

On a : \[\norme{x+y}^2=\norme{x}^2+2\ps{x}{y}+\norme{y}^2\qquad\text{et}\qquad\norme{x-y}^2=\norme{x}^2-2\ps{x}{y}+\norme{y}^2.\]
\end{prop}

\begin{dem}
On a : \[\norme{x+y}^2=\ps{x+y}{x+y}=\ps{x}{x}+\ps{x}{y}+\ps{y}{x}+\ps{y}{y}=\norme{x}^2+2\ps{x}{y}+\norme{y}^2\] et : \[\norme{x-y}^2=\ps{x-y}{x-y}=\ps{x}{x}-\ps{x}{y}-\ps{y}{x}+\ps{y}{y}=\norme{x}^2-2\ps{x}{y}+\norme{y}^2.\]
\end{dem}

\begin{theo}[Identité du parallélogramme]
Soient \(x,y\in E\).

On a : \[\norme{x+y}^2+\norme{x-y}^2=2\norme{x}^2+2\norme{y}^2.\]
\end{theo}

\begin{dem}
Découle de la \thref{prop:carréDeLaNormeD'UneSommeOuD'uneDifférence} par somme des deux égalités.
\end{dem}

\begin{theo}[Identités de polarisation]
Soient \(x,y\in E\).

On a : \[\begin{dcases}
\ps{x}{y}=\dfrac{1}{2}\paren{\norme{x+y}^2-\norme{x}^2-\norme{y}^2} \\
\ps{x}{y}=\dfrac{1}{4}\paren{\norme{x+y}^2-\norme{x-y}^2}
\end{dcases}\]
\end{theo}

\begin{dem}
Découle de la \thref{prop:carréDeLaNormeD'UneSommeOuD'uneDifférence}.
\end{dem}

\section{Orthogonalité, base orthonormale}

\subsection{Vocabulaire}

\begin{defi}
Soit \(E\) un espace préhilbertien.

On dit qu'un vecteur \(x\in E\) est unitaire s'il est de norme \(1\) : \[\norme{x}=1.\]

On dit que deux vecteurs \(x,y\in E\) sont orthogonaux et on note \(x\perp y\) si leur produit scalaire est nul : \[\ps{x}{y}=0.\]

On dit qu'une famille de vecteurs \(\paren{x_i}_{i\in I}\in E^I\) est orthogonale si ses vecteurs sont deux à deux orthogonaux : \[\quantifs{\forall i,j\in I}i\not=j\imp\ps{x_i}{x_j}=0.\]

On dit qu'une famille de vecteurs \(\paren{x_i}_{i\in I}\in E^I\) est orthonormale ou orthonormée si ses vecteurs sont unitaires et deux à deux orthogonaux : \[\quantifs{\forall i,j\in I}\ps{x_i}{x_j}=\delta_{ij}.\]

On dit qu'une famille de vecteurs est une base orthonormale ou orthonormée de \(E\) si c'est une famille orthonormale et une base de \(E\).
\end{defi}

\begin{ex}
La base canonique de \(\R^n\), pour le produit scalaire canonique de \(\R^n\), est orthonormale.

La base canonique de \(\M{np}[\R]\), pour le produit scalaire canonique de \(\M{np}[\R]\), est orthonormale.
\end{ex}

\begin{exo}
On note \(\fami{C}_{2\pi}\) l'ensemble des fonctions continues \(2\pi\)-périodiques de \(\R\) dans \(\R\).

\begin{enumerate}
\item Montrer que l'application : \[\fonctionlambda{\fami{C}_{2\pi}\times\fami{C}_{2\pi}}{\R}{\paren{f,g}}{\dfrac{1}{2\pi}\int_0^{2\pi}f\paren{t}g\paren{t}\odif{t}}\] est un produit scalaire. L'espace vectoriel \(\fami{C}_{2\pi}\) est-il un espace euclidien ? \\
\item On pose : \[\quantifs{\forall k\in\interventierii{0}{n}}\fonction{f_k}{\R}{\R}{t}{\cos\paren{kt}}\] La famille \(\paren{f_0,\dots,f_n}\) est-elle orthogonale ? orthonormale ? \\
\item Même question avec la famille \(\paren{f_0,\dots,f_n,g_1,\dots,g_n}\), en posant : \[\quantifs{\forall k\in\interventierii{1}{n}}\fonction{g_k}{\R}{\R}{t}{\sin\paren{kt}}\]
\end{enumerate}
\end{exo}

\begin{corr}[1]
\Cf \thref{dem:produitScalaireFonctionsContinues}.

On note tout de même :

Soit \(f\in\fami{C}_{2\pi}\).

Si \(\ps{f}{f}=0\) alors \(\dfrac{1}{2\pi}\int_0^{2\pi}f^2\paren{t}\odif{t}=0\).

Or \(f^2\) est continue et positive.

Donc \(\quantifs{\forall t\in\intervii{0}{2\pi}}f^2\paren{t}=0\).

Donc \(f=0\) car \(f\) est \(2\pi\)-périodique.
\end{corr}

\begin{corr}[2]
On a : \[\begin{aligned}
\quantifs{\forall k,l\in\interventierii{0}{n}}\ps{f_k}{f_l}&=\dfrac{1}{2\pi}\int_0^{2\pi}f_k\paren{t}f_l\paren{t}\odif{t} \\
&=\dfrac{1}{2\pi}\int_0^{2\pi}\cos\paren{kt}\cos\paren{lt}\odif{t} \\
&=\dfrac{1}{4\pi}\paren{\int_0^{2\pi}\cos\paren{\paren{k+l}t}\odif{t}+\int_0^{2\pi}\cos\paren{\paren{k-l}t}\odif{t}} \\
&=\begin{dcases}
\dfrac{1}{4\pi}\paren{\croch{\dfrac{\sin\paren{\paren{k+l}t}}{k+l}}_0^{2\pi}+\croch{\dfrac{\sin\paren{\paren{k-l}t}}{k-l}}_0^{2\pi}}=0 &\text{si }k\not=l \\
\dfrac{1}{4\pi}\paren{\croch{\dfrac{\sin\paren{2kt}}{2k}}_0^{2\pi}+2\pi}=\dfrac{1}{2}\not=1 &\text{si }k=l\not=0 \\
\dfrac{1}{4\pi}\paren{2\pi+2\pi}=1 &\text{si }k=l=0
\end{dcases}
\end{aligned}\]

Donc \(\paren{f_0,\dots,f_n}\) est orthogonale mais pas orthonormée car \(\quantifs{\forall k\in\interventierii{1}{n}}\norme{f_k}=\dfrac{1}{\sqrt{2}}\not=1\).
\end{corr}

\begin{corr}[3]
On a : \[\begin{aligned}
\quantifs{\forall k\in\interventierii{0}{n};\forall l\in\interventierii{1}{n}}\ps{f_k}{g_l}&=\dfrac{1}{2\pi}\int_0^{2\pi}f_k\paren{t}g_l\paren{t}\odif{t} \\
&=\dfrac{1}{2\pi}\int_0^{2\pi}\cos\paren{kt}\sin\paren{lt}\odif{t} \\
&=\dfrac{1}{4\pi}\paren{\int_0^{2\pi}\sin\paren{\paren{k+l}t}\odif{t}+\int_0^{2\pi}\sin\paren{\paren{l-k}t}\odif{t}} \\
&=\begin{dcases}
\dfrac{1}{4\pi}\paren{\croch{\dfrac{-\cos\paren{\paren{k+l}t}}{k+l}}_0^{2\pi}+\croch{\dfrac{-\cos\paren{\paren{l-k}t}}{l-k}}_0^{2\pi}}=0 &\text{si }k\not=l \\
\dfrac{1}{4\pi}\paren{\croch{\dfrac{-\cos\paren{2kt}}{2k}}_0^{2\pi}+0}=0 &\text{si }k=l
\end{dcases}
\end{aligned}\]

Donc \(\quantifs{\forall k\in\interventierii{0}{n};\forall l\in\interventierii{1}{n}}f_k\perp g_l\).

De plus, on a : \[\begin{aligned}
\quantifs{\forall k,l\in\interventierii{1}{n}}\ps{g_k}{g_l}&=\dfrac{1}{2\pi}\int_0^{2\pi}g_k\paren{t}g_l\paren{t}\odif{t} \\
&=\dfrac{1}{2\pi}\int_0^{2\pi}\sin\paren{kt}\sin\paren{lt}\odif{t} \\
&=\dfrac{1}{4\pi}\paren{\int_0^{2\pi}\cos\paren{\paren{k-l}t}\odif{t}-\int_0^{2\pi}\cos\paren{\paren{k+l}t}\odif{t}} \\
&=\begin{dcases}
\dfrac{1}{4\pi}\paren{\croch{\dfrac{\sin\paren{\paren{k-l}t}}{k-l}}_0^{2\pi}-\croch{\dfrac{\sin\paren{\paren{k+l}t}}{k+l}}_0^{2\pi}}=0 &\text{si }k\not=l \\
\dfrac{1}{4\pi}\paren{-\croch{\dfrac{\sin\paren{2kt}}{2k}}_0^{2\pi}+2\pi}=\dfrac{1}{2} &\text{si }k=l
\end{dcases}
\end{aligned}\]

Donc \(\quantifs{\forall k,l\in\interventierii{1}{n}}g_k\perp g_l\).

Finalement, \(\paren{f_0,\dots,f_n,g_1,\dots,g_n}\) est orthogonale mais pas orthonormée car \(\quantifs{\forall k\in\interventierii{1}{n}}\norme{g_k}=\dfrac{1}{\sqrt{2}}\not=1\).
\end{corr}

\subsection{Propriétés des familles orthogonales}

\begin{theo}[Théorème de Pythagore]
Soient \(E\) un espace préhilbertien et \(\paren{x_1,\dots,x_n}\) une famille orthogonale de vecteurs de \(E\).

On a : \[\norme{\sum_{j=1}^{n}x_j}^2=\sum_{j=1}^{n}\norme{x_j}^2.\]
\end{theo}

\begin{dem}
On a : \[\begin{aligned}
\norme{\sum_{j=1}^{n}x_j}^2&=\ps{\sum_{j=1}^{n}x_j}{\sum_{k=1}^{n}x_k} \\
&=\sum_{j=1}^{n}\sum_{k=1}^{n}\underbrace{\ps{x_j}{x_k}}_{=0\text{ si }j\not=k} \\
&=\sum_{j=1}^{n}\norme{x_j}^2
\end{aligned}\]
\end{dem}

\begin{prop}
Soient \(E\) un espace préhilbertien et \(\paren{x_1,\dots,x_n}\in E^n\) une famille orthogonale dont tous les vecteurs sont non-nuls.

Alors \(\paren{x_1,\dots,x_n}\) est une famille libre.
\end{prop}

\begin{dem}
Soient \(\lambda_1,\dots,\lambda_n\in\R\) tels que \(\sum_{j=1}^{n}\lambda_jx_j=0_E\) et \(k\in\interventierii{1}{n}\).

On a \(\ps{\sum_{j=1}^{n}\lambda_jx_j}{x_k}=0\).

Donc \(\sum_{j=1}^{n}\lambda_j\underbrace{\ps{x_j}{x_k}}_{=0\text{ si }j\not=k}=0\).

Donc \(\lambda_k\norme{x_k}^2=0\).

Donc \(\lambda_k=0\) car \(x_k\not=0\).

Donc \(\paren{x_1,\dots,x_n}\) est libre.
\end{dem}

\begin{prop}
Soient \(E\) un espace euclidien de dimension \(n\in\Ns\) et \(\paren{x_1,\dots,x_n}\in E^n\) une famille orthonormale possédant \(n\) vecteurs.

Alors \(\paren{x_1,\dots,x_n}\) est une base orthonormale de \(E\).
\end{prop}

\begin{dem}
La famille \(\paren{x_1,\dots,x_n}\) est libre car c'est une famille orthogonale de vecteurs non-nuls. De plus, elle possède \(n\) vecteurs et \(\dim E=n\) donc c'est une base de \(E\) et donc une base orthonormale de \(E\).
\end{dem}

\subsection{Calculs dans une base orthonormale}

\begin{prop}
Soient \(E\) un espace euclidien de dimension \(n\in\Ns\), \(\fami{B}=\paren{e_1,\dots,e_n}\) une base orthonormale de \(E\) et \(x\) et \(y\) deux vecteurs de \(E\) dont on note \(X=\tcoords{x_1}{\vdots}{x_n}\) et \(Y=\tcoords{y_1}{\vdots}{y_n}\) les coordonnées respectives dans \(\fami{B}\).

On a :

\begin{enumerate}
    \item \(\quantifs{\forall i\in\interventierii{1}{n}}x_i=\ps{e_i}{x}\) ; \\
    \item \(\ps{x}{y}=\sum_{i=1}^{n}x_iy_i=\trans{X}Y\) ; \\
    \item \(\norme{x}=\sqrt{\sum_{i=1}^{n}x_i^2}=\sqrt{\trans{X}X}\).
\end{enumerate}
\end{prop}

\begin{dem}
Tout découle de : \[\begin{dcases}
x=\sum_{i=1}^{n}x_ie_i \\
y=\sum_{i=1}^{n}y_ie_i \\
\quantifs{\forall i,j\in\interventierii{1}{n}}\ps{e_i}{e_j}=\delta_{ij}
\end{dcases}\]

On a : \[\begin{aligned}
\quantifs{\forall i\in\interventierii{1}{n}}\ps{e_i}{x}&=\ps{e_i}{\sum_{j=1}^{n}x_je_j} \\
&=\sum_{j=1}^{n}x_j\ps{e_i}{e_j} \\
&=x_i
\end{aligned}\] et : \[\begin{aligned}
\ps{x}{y}&=\ps{\sum_{i=1}^{n}x_ie_i}{\sum_{j=1}^{n}y_je_j} \\
&=\sum_{i=1}^{n}\sum_{j=1}^{n}x_iy_j\ps{e_i}{e_j} \\
&=\sum_{i=1}^{n}x_iy_i.
\end{aligned}\]
\end{dem}

\begin{rem}
Attention au fait que les formules précédentes ne sont valables que dans une base orthonormale.

Ainsi, si \(\paren{e_1,\dots,e_n}\) est une base orthonormale de \(E\), alors on a : \[\quantifs{\forall x\in E}x=\sum_{i=1}^{n}\ps{e_i}{x}e_i.\]
\end{rem}

\begin{cor}
Soient \(E\) un espace euclidien, \(\fami{B}=\paren{e_1,\dots,e_n}\) une base orthonormale de \(E\) et \(u\in\Lendo{E}\).

La matrice de \(u\) dans \(\fami{B}\) est : \[\Mat{u}=\begin{pmatrix}
\ps{e_1}{u\paren{e_1}} & \dots & \ps{e_1}{u\paren{e_n}} \\
\vdots &  & \vdots \\
\ps{e_n}{u\paren{e_1}} & \dots & \ps{e_n}{u\paren{e_n}}
\end{pmatrix}.\]
\end{cor}

\section{Sous-espaces vectoriels}

\subsection{Sous-espaces vectoriels orthogonaux}

\begin{defi}
Soient \(E\) un espace préhilbertien et \(F\) et \(G\) deux sous-espaces vectoriels de \(E\).

On dit que \(F\) et \(G\) sont deux sous-espaces vectoriels orthogonaux et on note \(F\perp G\) si on a : \[\quantifs{\forall x\in F;\forall y\in G}x\perp y.\]
\end{defi}

\begin{ex}
Soit \(E\) un espace préhilbertien.

Soient \(x,y\in E\) tels que \(x\perp y\). On a alors \(\Vect{x}\perp\Vect{y}\).

Le sous-espace vectoriel nul \(\accol{0_E}\) est orthogonal à tous les sous-espaces vectoriels de \(E\) (car le vecteur nul de \(E\) est orthogonal à tous les vecteurs de \(E\)).
\end{ex}

\begin{defi}[Orthogonal d'une partie]
Soient \(E\) un espace préhilbertien et \(A\subset E\).

On appelle orthogonal de \(A\) (dans \(E\)) et on note \(A\ortho\) l'ensemble des vecteurs de \(E\) qui sont orthogonaux à tous les vecteurs de \(A\) : \[A\ortho=\accol{x\in E\tq\quantifs{\forall y\in A}x\perp y}.\]

L'ensemble \(A\ortho\) est un sous-espace vectoriel de \(E\).
\end{defi}

\begin{prop}[Orthogonal d'un sous-espace vectoriel]\thlabel{prop:orthogonalD'UnSousEspaceVectoriel}
Soient \(E\) un espace préhilbertien et \(F\) un sous-espace vectoriel de \(E\).

On a :

\begin{enumerate}
    \item \(F\ortho\) est un sous-espace vectoriel de \(E\) ; \\
    \item \(F\perp F\ortho\) ; \\
    \item \(F\inter F\ortho=\accol{0_E}\) ; \\
    \item \(F\subset\paren{F\ortho}\ortho\).
\end{enumerate}
\end{prop}

\begin{dem}[1]
On a \(F\ortho\subset E\) et \(0_E\in F\ortho\) car \(\quantifs{\forall y\in F}0_E\perp y\).

Soient \(\lambda_1,\lambda_2\in\R\) et \(x_1,x_2\in F\ortho\).

On a : \[\quantifs{\forall y\in F}\ps{\lambda_1x_1+\lambda_2x_2}{y}=\lambda_1\underbrace{\ps{x_1}{y}}_{=0}+\lambda_2\underbrace{\ps{x_2}{y}}_{=0}=0.\]

Donc \(\lambda_1x_1+\lambda_2x_2\in F\ortho\).

Donc \(F\ortho\) est un sous-espace vectoriel de \(E\).
\end{dem}

\begin{dem}[2]
Clair par définition de \(F\ortho\).
\end{dem}

\begin{dem}[3]
Soit \(x\in F\inter F\ortho\).

On a \(x\perp x\) donc \(\ps{x}{x}=0\).

Donc \(x=0_E\).
\end{dem}

\begin{dem}[4]
Soit \(x\in F\).

On a \(\quantifs{\forall y\in F\ortho}x\perp y\).

Donc \(x\in\paren{F\ortho}\ortho\).
\end{dem}

\begin{rem}
Soient \(E\) un espace préhilbertien et \(F\) et \(G\) deux sous-espaces vectoriels de \(E\).

Les propositions suivantes sont équivalentes :

\begin{enumerate}
    \item \(F\perp G\) \\
    \item \(F\subset G\ortho\) \\
    \item \(G\subset F\ortho\)
\end{enumerate}
\end{rem}

\begin{dem}
On a : \[\begin{aligned}
F\perp G&\ssi\quantifs{\forall x\in F;\forall y\in G}x\perp y \\
&\ssi\quantifs{\forall x\in F}x\in G\ortho \\
&\ssi F\subset G\ortho.
\end{aligned}\]

D'où (1) \(\ssi\) (2).

Idem pour (1) \(\ssi\) (3).
\end{dem}

\subsection{Supplémentaire orthogonal}

\begin{rappel}
Soient \(E\) un espace préhilbertien et \(F\) un sous-espace vectoriel de \(E\).

Selon la \thref{prop:orthogonalD'UnSousEspaceVectoriel}, on a : \[F\inter F\ortho=\accol{0_E}.\]

Donc \(F\) et \(F\ortho\) sont en somme directe.
\end{rappel}

\begin{defi}[Supplémentaire orthogonal]
Soient \(E\) un espace préhilbertien et \(F\) un sous-espace vectoriel de \(E\).

Un supplémentaire orthogonal de \(F\) (dans \(E\)) est un supplémentaire de \(F\) (dans \(E\)) qui est orthogonal à \(F\).

En d'autres termes, c'est un sous-espace vectoriel \(G\) de \(E\) tel que : \[\begin{dcases}
F\oplus G=E \\
F\perp G
\end{dcases}\]

On résume parfois ce système avec la notation \(F\operp G=E\) ou des variantes de cette notation.
\end{defi}

\begin{prop}[Unicité du supplémentaire orthogonal]\thlabel{prop:unicitéDuSupplémentaireOrthogonal}
Soient \(E\) un espace préhilbertien et \(F\) un sous-espace vectoriel de \(E\).

Si \(G\) est un supplémentaire orthogonal de \(F\) alors \(G=F\ortho\).
\end{prop}

\begin{dem}
\incdir Claire car \(G\perp F\).

\increc

Soit \(x\in F\ortho\).

Montrons que \(x\in G\).

Soient \(x_F\in F\) et \(x_G\in G\) tels que \(x=x_F+x_G\).

On a \(\ps{x_F}{x_F+x_G}=\ps{x_F}{x}\) donc \[\norme{x_F}^2+\underbrace{\ps{x_F}{x_G}}_{=0\text{ car }F\perp G}=\underbrace{\ps{x_F}{x}}_{=0\text{ car }F\perp G}.\]

Donc \(x_F=0_E\).

Donc \(x=x_G\).

Donc \(x\in G\).
\end{dem}

\begin{rem}
Soient \(E\) un espace préhilbertien et \(F\) un sous-espace vectoriel de \(E\).

D'après la proposition précédente, \(F\) admet un supplémentaire orthogonal si, et seulement si, \(F\ortho\) est un supplémentaire de \(F\) : \(F+F\ortho=E\).
\end{rem}

\begin{exo}\thlabel{exo:orthogonalD'OrthogonalEgalALuiMême}
Soient \(E\) un espace préhilbertien et \(F\) un sous-espace vectoriel de \(E\).

On suppose que \(F\) admet un supplémentaire orthogonal dans \(E\).

Montrer : \[\paren{F\ortho}\ortho=F.\]
\end{exo}

\begin{corr}
On a \(E=F\oplus F\ortho\) donc \(F\) est un supplémentaire orthogonal de \(F\ortho\).

Donc \(F=\paren{F\ortho}\ortho\) selon la \thref{prop:unicitéDuSupplémentaireOrthogonal}.
\end{corr}

\begin{defi}
Soit \(E\) un espace préhilbertien.

Un projecteur \(p\in\Lendo{E}\) est dit orthogonal si son image et son noyau sont orthogonaux : \[\Im p\perp\ker p.\]
\end{defi}

\begin{rem}
Soient \(E\) un espace préhilbertien et \(F\) et \(G\) deux sous-espaces vectoriels supplémentaires dans \(E\).

\begin{enumerate}
    \item Notons \(p\) le projecteur sur \(F\), parallèlement à \(G\). \\ Si \(p\) est un projecteur orthogonal, alors \(G=F\ortho\). \\ On dit simplement que \(p\) est le projecteur orthogonal sur \(F\). \\
    \item Le \guillemets{projecteur orthogonal sur \(F\)} est bien défini si, et seulement si, \(F\) admet un supplémentaire orthogonal.
\end{enumerate}
\end{rem}

\begin{dem}[1]
Si \(p\) est un projecteur orthogonal alors \(F\perp G\) et \(F\oplus G=E\) donc \(G=F\ortho\).
\end{dem}

\begin{rem}
On sait que tout projecteur est le projecteur sur son image, parallèlement à son noyau.

De même, tout projecteur orthogonal \(p\) est le projecteur orthogonal sur son image, \cad le projecteur sur \(\Im p\), parallèlement à \(\paren{\Im p}\ortho\).
\end{rem}

\subsection{Projecteur orthogonal sur un sous-espace vectoriel de dimension finie}

\begin{deftheo}
Soient \(E\) un espace préhilbertien et \(F\) un sous-espace vectoriel de \(E\).

On suppose que \(F\) est de dimension finie.

Alors \(F\) et son orthogonal \(F\ortho\) sont supplémentaires dans \(E\) : \[E=F\oplus F\ortho.\]

Soit \(\paren{e_1,\dots,e_n}\) une base orthonormale de \(F\).

Le projecteur orthogonal \(p_F\) sur \(F\) est donné par : \[\quantifs{\forall x\in E}p_F\paren{x}=\sum_{k=1}^{n}\ps{e_k}{x}e_k.\]
\end{deftheo}

\begin{dem}
Posons \(\fonction{q}{E}{F}{x}{\sum_{k=1}^{n}\ps{e_k}{x}e_k}\)

On a bien \(q\in\L{E}{F}\).

On remarque : \(\quantifs{\forall x\in E}x=\underbrace{q\paren{x}}_{\in F}+\underbrace{x-q\paren{x}}_{\in F\ortho\text{ ?}}\).

Soit \(x\in E\). Il suffit de montrer que \(x-q\paren{x}\in F\ortho\).

En effet, on en déduira que \(E=F+F\ortho\) donc \(E=F\oplus F\ortho\) et que le projecteur orthogonal sur \(F\) est \(q\).

On a : \[\begin{aligned}
\quantifs{\forall k\in\interventierii{1}{n}}\ps{e_k}{x-q\paren{x}}&=\ps{e_k}{x}-\ps{e_k}{\sum_{l=1}^{n}\ps{e_l}{x}e_l} \\
&=\ps{e_k}{x}-\ps{e_k}{x} \\
&=0.
\end{aligned}\]

Donc par combinaison linéaire : \(\quantifs{\forall y\in\Vect{e_1,\dots,e_n}}\ps{y}{x-q\paren{x}}=0\).

Donc \(x-q\paren{x}\in F\ortho\).
\end{dem}

\begin{cor}
Soient \(E\) un espace euclidien et \(F\) un sous-espace vectoriel de \(E\).

On a : \[\dim F\ortho=\dim E-\dim F.\]
\end{cor}

\begin{dem}
Comme \(F\) est de dimension finie, on sait (selon le théorème précédent) que \(F\ortho\) est un supplémentaire de \(F\). La formule en découle.
\end{dem}

\begin{defi}
Soient \(E\) un espace préhilbertien, \(F\) un sous-espace vectoriel de \(E\) et \(x\) un vecteur de \(E\).

On appelle distance de \(x\) à \(F\) le réel : \[d\paren{x,F}=\inf_{z\in F}d\paren{x,z}=\inf_{z\in F}\norme{x-z}\in\Rp.\]
\end{defi}

\begin{dem}
Cette borne inférieure est bien définie car \(\accol{d\paren{x,z}}_{z\in F}\) est une partie non-vide de \(\R\) (elle contient \(d\paren{x,0}\)) et minorée (par \(0\)).
\end{dem}

\begin{prop}
Soient \(E\) un espace préhilbertien, \(F\) un sous-espace vectoriel de \(E\) de dimension finie et \(x\) un vecteur de \(E\).

On note \(p_F\) le projecteur orthogonal sur \(F\).

Alors : \[d\paren{x,F}=\norme{x-p_F\paren{x}}.\]

Ainsi, la distance (qui est une borne inférieure) est en fait atteinte.

De plus, \(p_F\paren{x}\) est l'unique élément de \(F\) où la distance de \(x\) à \(F\) est atteinte.

Enfin : \[\norme{x}^2=\norme{p_F\paren{x}}^2+\norme{x-p_F\paren{x}}^2.\]
\end{prop}

\begin{dem}
On a : \[\begin{WithArrows}
\quantifs{\forall z\in F}\norme{x-z}^2&=\norme{\underbrace{x-p_F\paren{x}}_{\in F\ortho}+\underbrace{p_F\paren{x}-z}_{\in F}}^2 \Arrow[tikz={text width=2.5cm}]{selon le théorème de Pythagore} \\
&=\norme{x-p_F\paren{x}}^2+\norme{p_F\paren{x}-z}^2 \\
&\geq\norme{x-p_F\paren{x}}^2\text{ avec égalité ssi }p_F\paren{x}=z.
\end{WithArrows}\]

D'où \(\min_{z\in F}\norme{x-z}=\norme{x-p_F\paren{x}}\).

Enfin, en prenant \(z=0_E\), on a : \(\norme{x}^2=\norme{p_F\paren{x}}^2+\norme{x-p_F\paren{x}}^2\).
\end{dem}

\begin{exoex}
Soient \(E\) un espace euclidien et \(v\in E\) unitaire.

On pose : \[D=\Vect{v}\qquad\text{et}\qquad H=\Vect{v}\ortho.\]

Soit \(x\in E\).

\begin{enumerate}
    \item Donner le projeté orthogonal de \(x\) sur \(D\). \\
    \item Donner le projeté orthogonal de \(x\) sur \(H\). \\
    \item Donner la distance de \(x\) à \(D\). \\
    \item Donner la distance de \(x\) à \(H\).
\end{enumerate}
\end{exoex}

\begin{corr}
On remarque que \(\paren{v}\) est une base orthonormale de \(D\).

On a le projeté orthogonal de \(x\) sur \(D\) : \(p_D\paren{x}=\ps{v}{x}v\).

On a le projeté orthogonal de \(x\) sur \(H\) : \(p_H\paren{x}=x-\ps{v}{x}v\).

On a la distance de \(x\) à \(D\) : \(\norme{x-p_D\paren{x}}=\norme{x-\ps{v}{x}v}\).

On a la distance de \(x\) à \(H\) : \[\begin{WithArrows}
\norme{x-p_H\paren{x}}&=\norme{x-x+\ps{v}{x}v} \\
&=\norme{\ps{v}{x}v} \Arrow{car \(v\) est unitaire} \\
&=\abs{\ps{v}{x}}
\end{WithArrows}\]
\end{corr}

\subsection{Procédé d'orthonormalisation de Gram-Schmidt}

\begin{theo}
Soient \(E\) un espace euclidien et \(\fami{B}=\paren{e_1,\dots,e_n}\) une base de \(E\).

Alors il existe une base orthonormale \(\fami{B}\prim=\paren{e_1\prim,\dots,e_n\prim}\) de \(E\) telle que : \[\quantifs{\forall j\in\interventierii{1}{n}}e_j\prim\in\Vect{e_1,\dots,e_j},\] \cad telle que la matrice de passage \(\pass{\fami{B}}{\fami{B}\prim}\) soit triangulaire supérieure.
\end{theo}

\begin{dem}
On calcule successivement : \[e_1\prim=\dfrac{e_1}{\norme{e_1}}\qquad e_2\prim=\dfrac{e_2-\ps{e_1\prim}{e_2}e_1\prim}{\norme{e_2-\ps{e_1\prim}{e_2}e_1\prim}}\qquad e_3\prim=\dfrac{e_3-\ps{e_1\prim}{e_3}e_1\prim-\ps{e_2\prim}{e_3}e_2\prim}{\norme{e_3-\ps{e_1\prim}{e_3}e_1\prim-\ps{e_2\prim}{e_3}e_2\prim}}\qquad\text{...}\]

Formule générale : \(\quantifs{\forall k\in\interventierii{1}{n}}e_k\prim=\dfrac{e_k-\ds\sum_{j=1}^{k-1}\ps{e_j\prim}{e_k}e_j\prim}{\norme{e_k-\ds\sum_{j=1}^{k-1}\ps{e_j\prim}{e_k}e_j\prim}}\).
\end{dem}

\begin{cor}
Tout espace euclidien admet une base orthonormale.
\end{cor}

\begin{dem}
Soit \(E\) un espace euclidien.

Il existe une base \(\fami{B}\) de \(E\) car \(\dim E<\pinf\).

On en déduit une base orthonormale de \(E\) en appliquant à \(\fami{B}\) l'algorithme de Gram-Schmidt.
\end{dem}

\begin{theo}[Théorème de la base orthonormée incomplète]
Soient \(E\) un espace euclidien et \(\paren{e_1,\dots,e_r}\) une famille orthonormale de vecteurs de \(E\).

Alors on peut compléter \(\paren{e_1,\dots,e_r}\) en une base orthonormale de \(E\).
\end{theo}

\begin{dem}
La famille \(\paren{e_1,\dots,e_r}\) est une famille libre de vecteurs de \(E\).

Selon le théorème de la base incomplète, on peut la compléter en une base \(\paren{e_1,\dots,e_n}\) de \(E\).

En appliquant l'algorithme de Gram-Schmidt à cette base, on obtient une base orthonormale \(\paren{e_1\prim,\dots,e_n\prim}\) de \(E\).

On remarque \(\quantifs{\forall k\in\interventierii{1}{r}}e_k\prim=e_k\) donc on a complété \(\paren{e_1,\dots,e_r}\) en la base orthonormale \[\paren{e_1,\dots,e_r,e_{r+1}\prim,\dots,e_n\prim}.\]
\end{dem}

\begin{exoex}\thlabel{exoex:orthonormalisationEtProjetéOrthogonalEtDistanceDansR4}
On munit \(\R^4\) de son produit scalaire canonique et on pose : \[e_1=\begin{pmatrix}
1 \\ 1 \\ 0 \\ 0
\end{pmatrix}\qquad e_2=\begin{pmatrix}
1 \\ 1 \\ 1 \\ 1
\end{pmatrix}\qquad e_3=\begin{pmatrix}
1 \\ 2 \\ 3 \\ 4
\end{pmatrix}\qquad v=\begin{pmatrix}
1 \\ 0 \\ 0 \\ 0
\end{pmatrix}\]

\begin{enumerate}
    \item Déterminer une base orthonormale de \(F=\Vect{e_1,e_2,e_3}\). \\
    \item Calculer le projeté orthogonal de \(v\) sur \(F\) et la distance de \(v\) à \(F\).
\end{enumerate}
\end{exoex}

\begin{corr}[1]
On sait que \(\paren{e_1,e_2,e_3}\) est une base de \(F\). Orthonormalisons cette base.

On pose : \[e_1\prim=\dfrac{e_1}{\norme{e_1}}\qquad e_2\prim=\dfrac{e_2-\ps{e_1\prim}{e_2}e_1\prim}{\norme{e_2-\ps{e_1\prim}{e_2}e_1\prim}}\qquad e_3\prim=\dfrac{e_3-\ps{e_1\prim}{e_3}e_1\prim-\ps{e_2\prim}{e_3}e_2\prim}{\norme{e_3-\ps{e_1\prim}{e_3}e_1\prim-\ps{e_2\prim}{e_3}e_2\prim}}\]

On a \(\norme{e_1}=\sqrt{2}\) donc \(e_1\prim=\dfrac{1}{\sqrt{2}}\begin{pmatrix}
1 \\ 1 \\ 0 \\ 0
\end{pmatrix}\).

De plus, on a \(\ps{e_1\prim}{e_2}=\ps{\dfrac{1}{\sqrt{2}}\begin{pmatrix}
1 \\ 1 \\ 0 \\ 0
\end{pmatrix}}{\begin{pmatrix}
1 \\ 1 \\ 1 \\ 1
\end{pmatrix}}=\dfrac{2}{\sqrt{2}}=\sqrt{2}\) et \(e_2-\sqrt{2}e_1\prim=\begin{pmatrix}
0 \\ 0 \\ 1 \\ 1
\end{pmatrix}\) et \(\norme{\begin{pmatrix}
0 \\ 0 \\ 1 \\ 1
\end{pmatrix}}=\sqrt{2}\).

Donc \(e_2\prim=\dfrac{1}{\sqrt{2}}\begin{pmatrix}
0 \\ 0 \\ 1 \\ 1
\end{pmatrix}\).

De plus, on a \(\ps{e_1\prim}{e_3}=\ps{\dfrac{1}{\sqrt{2}}\begin{pmatrix}
1 \\ 1 \\ 0 \\ 0
\end{pmatrix}}{\begin{pmatrix}
1 \\ 2 \\ 3 \\ 4
\end{pmatrix}}=\dfrac{3}{\sqrt{2}}\) et \(\ps{e_2\prim}{e_3}=\ps{\dfrac{1}{\sqrt{2}}\begin{pmatrix}
0 \\ 0 \\ 1 \\ 1
\end{pmatrix}}{\begin{pmatrix}
1 \\ 2 \\ 3 \\ 4
\end{pmatrix}}=\dfrac{7}{\sqrt{2}}\) et \(e_3-\dfrac{3}{2}\begin{pmatrix}
1 \\ 1 \\ 0 \\ 0
\end{pmatrix}-\dfrac{7}{2}\begin{pmatrix}
0 \\ 0 \\ 1 \\ 1
\end{pmatrix}=\dfrac{1}{2}\begin{pmatrix}
-1 \\ 1 \\ -1 \\ 1
\end{pmatrix}\) et \(\norme{\dfrac{1}{2}\begin{pmatrix}
-1 \\ 1 \\ -1 \\ 1
\end{pmatrix}}=\dfrac{1}{2}\times2=1\).

Donc \(e_3\prim=\dfrac{1}{2}\begin{pmatrix}
-1 \\ 1 \\ -1 \\ 1
\end{pmatrix}\).

Finalement, \(\paren{e_1\prim,e_2\prim,e_3\prim}\) est une base orthonormale de \(F\).
\end{corr}

\begin{corr}[2]
On a : \[p_F\paren{v}=\ps{e_1\prim}{v}e_1\prim+\ps{e_2\prim}{v}e_2\prim+\ps{e_3\prim}{v}e_3\prim.\]

Or, on a : \[\ps{e_1\prim}{v}=\dfrac{1}{\sqrt{2}}\qquad\ps{e_2\prim}{v}=0\qquad\ps{e_3\prim}{v}=\dfrac{-1}{2}\]

D'où : \[\begin{aligned}
p_F\paren{v}&=\dfrac{1}{\sqrt{2}}\times\dfrac{1}{\sqrt{2}}\begin{pmatrix}
1 \\ 1 \\ 0 \\ 0
\end{pmatrix}-\dfrac{1}{2}\times\dfrac{1}{2}\begin{pmatrix}
-1 \\ 1 \\ -1 \\ 1
\end{pmatrix} \\
&=\dfrac{1}{2}\begin{pmatrix}
1 \\ 1 \\ 0 \\ 0
\end{pmatrix}-\dfrac{1}{4}\begin{pmatrix}
-1 \\ 1 \\ -1 \\ 1
\end{pmatrix} \\
&=\dfrac{1}{4}\begin{pmatrix}
3 \\ 1 \\ 1 \\ -1
\end{pmatrix}
\end{aligned}\]

D'où : \[\begin{aligned}
d\paren{v,F}&=\norme{v-p_F\paren{v}} \\
&=\norme{\begin{pmatrix}
1 \\ 0 \\ 0 \\ 0
\end{pmatrix}-\dfrac{1}{4}\begin{pmatrix}
3 \\ 1 \\ 1 \\ -1
\end{pmatrix}} \\
&=\norme{\dfrac{1}{4}\begin{pmatrix}
1 \\ -1 \\ -1 \\ 1
\end{pmatrix}} \\
&=\dfrac{1}{4}\times2 \\
&=\dfrac{1}{2}.
\end{aligned}\]
\end{corr}

\subsection{Hyperplans d'un espace euclidien}

\begin{defprop}
Soient \(E\) un espace euclidien et \(H\) un hyperplan de \(E\).

On appelle vecteur normal à \(H\) tout vecteur \(v\) non-nul et orthogonal à tout vecteur de \(H\) : \[v\in H\ortho\excluant\accol{0_E}.\]

Il vérifie : \[\quantifs{\forall x\in E}x\in H\ssi v\perp x.\]
\end{defprop}

\begin{dem}
On a \(\dim E<\pinf\) donc \(\dim H\ortho=\dim E-\dim H=1\) donc il existe \(v\in H\ortho\excluant\accol{0_E}\) donc \(H\) admet un vecteur normal.

Posons \(D=\Vect{v}\).

On a \(D=H\ortho\) donc \(H=D\ortho\) selon l'\thref{exo:orthogonalD'OrthogonalEgalALuiMême}.

Donc : \[\begin{aligned}
\quantifs{\forall x\in E}x\in H&\ssi x\in D\ortho \\
&\ssi x\perp v \\
&\ssi\ps{x}{v}=0\qquad\text{(équation cartésienne de \(H\)).}
\end{aligned}\]
\end{dem}

\begin{prop}[Isomorphisme canonique entre un espace euclidien et son dual]
Soient \(E\) un espace euclidien et \(l\in E\etoile\) une forme linéaire sur \(E\).

Il existe un unique vecteur \(v\in E\) tel que : \[\quantifs{\forall x\in E}l\paren{x}=\ps{v}{x}.\]
\end{prop}

\begin{dem}
Posons \(\fonction{\phi}{E}{E\etoile}{v}{\fonctionlambda{E}{\R}{x}{\ps{v}{x}}}\)

On a \(\phi\in\L{E}{E\etoile}\).

Montrons que \(\phi\) est une injection.

Soit \(v\in\ker\phi\).

On a \(\phi\paren{v}=0\) donc \(\quantifs{\forall x\in E}\ps{v}{x}=0\).

Donc \(\ps{v}{v}=0\).

Donc \(v=0\).

Donc \(\ker\phi=\accol{0}\) donc \(\phi\) est une injection.

De plus, \(\dim E=\dim E\etoile<\pinf\) donc \(\phi\) est une surjection.

Donc \(\phi\) est un isomorphisme : \(\quantifs{\forall l\in E\etoile;\exists!v\in E}l=\phi\paren{v}\).

Donc : \[\quantifs{\forall l\in E\etoile;\exists!v\in E;\forall x\in E}l\paren{x}=\ps{v}{x}.\]
\end{dem}

\begin{exoex}
On a vu dans l'\thref{exoex:orthonormalisationEtProjetéOrthogonalEtDistanceDansR4} un hyperplan \(F\) de \(\R^4\).

Donner un vecteur normal à \(F\).
\end{exoex}

\begin{corr}
On a \(v-p_F\paren{v}\in F\ortho\) et \(v-p_F\paren{v}\not=0_{\R^4}\) donc \(v=\dfrac{1}{4}\begin{pmatrix}
1 \\ -1 \\ -1 \\ 1
\end{pmatrix}\) est un vecteur normal à \(F\) et donc \(\begin{pmatrix}
1 \\ -1 \\ -1 \\ 1
\end{pmatrix}\) aussi.

Ainsi : \[\begin{aligned}
\quantifs{\forall\begin{pmatrix}a \\ b \\ c \\ d\end{pmatrix}\in\R^4}\begin{pmatrix}a \\ b \\ c \\ d\end{pmatrix}\in F&\ssi\ps{\begin{pmatrix}a \\ b \\ c \\ d\end{pmatrix}}{\begin{pmatrix}
1 \\ -1 \\ -1 \\ 1
\end{pmatrix}}=0 \\
&\ssi a-b-c+d=0\qquad\text{(équation cartésienne de \(F\)).}
\end{aligned}\]
\end{corr}

\begin{rem}
Soit \(E\) un espace euclidien.

On retrouve que les hyperplans de \(E\) sont les noyaux des formes linéaires non-nulles de \(E\).
\end{rem}

\begin{dem}
Soient \(\fami{B}=\paren{e_1,\dots,e_n}\) une base orthonormale de \(E\), \(H\) un hyperplan de \(E\) et \(v\) un vecteur normal à \(H\) de coordonnées \(\paren{a_1,\dots,a_n}\) dans \(\fami{B}\).

On a : \[\begin{aligned}
\quantifs{\forall\tcoords{x_1}{\vdots}{x_n}\in\R^n}\sum_{k=1}^{n}x_ke_k\in H&\ssi\sum_{k=1}^{n}x_ke_k\perp\sum_{k=1}^{n}a_ke_k \\
&\ssi\sum_{k=1}^{n}a_kx_k=0 \\
&\ssi\tcoords{x_1}{\vdots}{x_n}\in\ker l
\end{aligned}\] avec \(l:\begin{dcases}
e_1\mapsto a_1 \\
\vdots \\
e_n\mapsto a_n
\end{dcases}\)
\end{dem}

\chapter{Fonctions de deux variables réelles}

\minitoc

Dans tout le chapitre, on note \(\norme{\cdot}\) la norme euclidienne canonique de \(\R^2\), \cad la norme associée au produit scalaire canonique \(\ps{\cdot}{\cdot}\) de \(\R^2\) : \[\quantifs{\forall x,y\in\R^2}\norme{\paren{x,y}}=\sqrt{x^2+y^2}.\]

\section{Ouverts de \(\R^2\)}

\begin{defi}[Boules]
Soient \(a=\paren{a_1,a_2}\in\R^2\) et \(r\in\Rp\).

La boule ouverte de centre \(a\) et de rayon \(r\) est l'ensemble des points \(x\) dont la distance à \(a\) est strictement inférieure à \(r\) : \[\bouleo{a}{r}=\accol{x\in\R^2\tq\norme{a-x}<r}.\]

La boule fermée de centre \(a\) et de rayon \(r\) est l'ensemble des points \(x\) dont la distance à \(a\) est inférieure à \(r\) : \[\boulef{a}{r}=\accol{x\in\R^2\tq\norme{a-x}\leq r}.\]

La sphère de centre \(a\) et de rayon \(r\) est l'ensemble des points \(x\) dont la distance à \(a\) est égale à \(r\) : \[\sphere{a}{r}=\accol{x\in\R^2\tq\norme{a-x}=r}.\]

Les notations \(\bouleo{a}{r}\), \(\boulef{a}{r}\) et \(\sphere{a}{r}\) ne sont pas \guillemets{officielles}.
\end{defi}

\begin{defi}
Soient \(a\in\R^2\) et \(V\subset\R^2\).

On dit que \(V\) est un voisinage de \(a\) dans \(\R^2\) s'il contient une boule centrée en \(a\) et de rayon strictement positif : \[\quantifs{\exists\epsilon\in\Rps}\bouleo{a}{\epsilon}\subset V.\]
\end{defi}

\begin{defi}[Ouvert]
Soit \(\Omega\subset\R^2\).

On dit que \(\Omega\) est un ouvert de \(\R^2\) (ou une partie ouverte de \(\R^2\)) si \(\Omega\) est un voisinage de chacun de ses points, \cad : \[\quantifs{\forall a\in\Omega;\exists\epsilon\in\Rps}\bouleo{a}{\epsilon}\subset\Omega.\]
\end{defi}

\begin{exoex}
Parmi les parties de \(\R^2\) suivantes, lesquelles sont des ouverts ?

\[\ensvide\qquad\R^2\qquad\paren{\Rps}^2\qquad\paren{\Rp}^2\qquad\R^2\excluant\accol{\paren{0,0}}\qquad\paren{\Rs}^2\qquad\intervei{0}{1}\times\R\qquad\intervee{0}{1}\times\R\]
\end{exoex}

\begin{corr}
\(\ensvide\) est un ouvert.

\(\R^2\) est un ouvert car \(\quantifs{\forall a\in\R^2}\bouleo{a}{1}\subset\R^2\).

\(\paren{\Rps}^2\) est un ouvert car \(\quantifs{\forall\paren{x,y}\in\R^2}\bouleo{\paren{x,y}}{\min\accol{\abs{x};\abs{y}}}\subset\paren{\Rps}^2\).

\(\paren{\Rp}^2\) n'est pas un ouvert.

\(\R^2\excluant\accol{\paren{0,0}}\) est un ouvert car \(\quantifs{\forall a\in\R^2\excluant\accol{\paren{0,0}}}\bouleo{a}{\norme{a}}\subset\R^2\excluant\accol{\paren{0,0}}\).

\(\paren{\Rs}^2\) est un ouvert.

\(\intervei{0}{1}\times\R\) n'est pas un ouvert.

\(\intervee{0}{1}\times\R\) est un ouvert.
\end{corr}

\section{Continuité}

Le graphe de \(f:\Omega\to\R\) où \(\Omega\) est un ouvert de \(\R^2\) est l'ensemble : \[\accol{\paren{x,y,z}\in\Omega\times\R\tq z=f\paren{x,y}}\subset\R^3.\]

Par exemple :

\begin{center}
\begin{tikzpicture}
\begin{axis}
\addplot3[surf, domain=-10:10, y domain=-10:10]{0.4*cos(x)+0.4*sin(y)-0.00006*sinh(x)-0.00005*sinh(y)+2};
\end{axis}
\end{tikzpicture}
\end{center}

\begin{defi}[Fonction continue sur un ouvert de \(\R^2\)]
Soient \(\Omega\) un ouvert de \(\R^2\), \(f:\Omega\to\R\) et \(a\in\Omega\).

On dit que \(f\) est continue en \(a\) si on a : \[\quantifs{\forall\epsilon\in\Rps;\exists\delta\in\Rps;\forall x\in\Omega}\norme{x-a}\leq\delta\imp\abs{f\paren{x}-f\paren{a}}\leq\epsilon,\] \cad : \[\quantifs{\forall\epsilon\in\Rps;\exists\delta\in\Rps}f\paren{\Omega\inter\boulef{a}{\delta}}\subset\intervii{f\paren{a}-\epsilon}{f\paren{a}+\epsilon},\] \cad : \[\quantifs{\forall\epsilon\in\Rps;\exists\delta\in\Rps}\begin{dcases}
\boulef{a}{\delta}\subset\Omega \\
f\paren{\boulef{a}{\delta}}\subset\intervii{f\paren{a}-\epsilon}{f\paren{a}+\epsilon}
\end{dcases}\]

On dit que \(f\) est continue si elle est continue en tout de \(\Omega\).
\end{defi}

\begin{prop}
Soit \(\Omega\) un ouvert de \(\R^2\).

L'ensemble \(\ensclasse{0}{\Omega}{\R}\) des fonctions continues de \(\Omega\) dans \(\R\) est un sous-anneau de \(\F{\Omega}{\R}\).

Si \(f:\Omega\to\R\) et \(g:\Im f\to\R\) sont des fonctions continues alors \(g\rond f:\Omega\to\R\) est continue.

Si \(f:\Omega\to\R\) est continue alors l'ensemble \[\Omega\prim=f\inv\paren{\Rs}=\accol{x\in\Omega\tq f\paren{x}\not=0}\] est un ouvert de \(\R^2\) et la fonction \[\fonction{\dfrac{1}{f}}{\Omega\prim}{\R}{x}{\dfrac{1}{f\paren{x}}}\] est continue.
\end{prop}

\begin{dem}
\note{Exercice}
\end{dem}

\begin{exoex}
On pose : \[f:\paren{x,y}\mapsto x\qquad\text{et}\qquad g:\paren{x,y}\mapsto\Arctan\dfrac{y}{x}.\]

Pour chacune de ces fonctions, donner son ensemble de définition, dire si c'est un ouvert de \(\R^2\) et dire si la fonction est continue.
\end{exoex}

\begin{corr}
\(f\) est définie sur \(\R^2\) qui est un ouvert de \(\R^2\).

Soit \(\paren{a,b}\in\R^2\). Montrons que \(f\) est continue en \(\paren{a,b}\).

On a : \[\begin{aligned}
\quantifs{\forall\paren{x,y}\in\R^2}\abs{f\paren{x,y}-f\paren{a,b}}&=\abs{x-a} \\
&=\sqrt{\paren{x-a}^2} \\
&\leq\sqrt{\paren{x-a}^2+\paren{y-b}^2} \\
&=\norme{\paren{x,y}-\paren{a,b}}.
\end{aligned}\]

On a donc : \[\quantifs{\forall\epsilon\in\Rps;\exists\delta\in\Rps;\forall\paren{x,y}\in\R^2}\norme{\paren{x,y}-\paren{a,b}}\leq\delta\imp\abs{f\paren{x,y}-f\paren{a,b}}\leq\epsilon\] car \(\delta=\epsilon\) convient.

Donc \(f\) est continue.

\(g\) est définie sur \(\Rs\times\R\) qui est un ouvert de \(\R^2\).

De même que précédemment, \(\paren{x,y}\mapsto y\) est continue.

Donc \(\paren{x,y}\mapsto\dfrac{y}{x}\) est continue sur \(\Rs\times\R\).

Or \(\Arctan:\R\to\R\) est continue donc \(g\) est continue par composition.
\end{corr}

\section{Fonctions de classe \(\classe{1}\)}

\subsection{Développement limité d'ordre 1}

\begin{nota}
Soient \(\Omega\) un ouvert de \(\R^2\) contenant \(\paren{0,0}\) et \(g:\Omega\to\R\).

On dit que \(g\paren{x}\) est négligeable devant \(\norme{x}\) et on note \(g\paren{x}\underset{x\to\paren{0,0}}{=}o\paren{\norme{x}}\) si on a : \[\quantifs{\forall\epsilon\in\Rps;\exists\delta\in\Rps;\forall x\in\Omega}\norme{x}\leq\delta\imp\abs{g\paren{x}}\leq\epsilon\norme{x}.\]
\end{nota}

\begin{defi}
Soient \(\Omega\) un ouvert de \(\R^2\), \(f:\Omega\to\R\) et \(a=\paren{a_1,a_2}\in\Omega\).

On dit que \(f\) admet un développement limité à l'ordre 1 en \(a\) s'il existe \(\lambda,\mu\in\R\) tels que : \[f\paren{a_1+h_1,a_2+h_2}\underset{h=\paren{h_1,h_2}\to\paren{0,0}}{=}f\paren{a_1,a_2}+\lambda h_1+\mu h_2+o\paren{\norme{h}}.\]

Les réels \(\lambda\) et \(\mu\) sont alors uniques.
\end{defi}

\subsection{Dérivées partielles}

\begin{defi}
Soient \(\Omega\) un ouvert de \(\R^2\), \(f:\Omega\to\R\) et \(a=\paren{a_1,a_2}\in\Omega\).

On dit que \(f\) admet une dérivée partielle par rapport à \(x\) en \(a\) si la fonction \(\gamma:t\mapsto f\paren{t,a_2}\) est dérivable en \(a_1\).

On pose alors : \[\pdv{f}{x}\paren{a}=\gamma\prim\paren{a_1}.\]

On définit de même \(\pdv{f}{y}\).
\end{defi}


\begin{exoex}
Calculer les dérivées partielles de \[f:\paren{x,y}\mapsto x^3y^2+x+1\qquad\text{et}\qquad g:\paren{x,y}\mapsto\Arctan\dfrac{y}{x}.\]
\end{exoex}


\begin{corr}
On a : \[\quantifs{\forall\paren{x,y}\in\R^2}\begin{dcases}
\pdv{f}{x}\paren{x,y}=3x^2y^2+1 \\
\pdv{f}{y}\paren{x,y}=2x^3y
\end{dcases}\] et : \[\quantifs{\forall\paren{x,y}\in\Rs\times\R}\begin{dcases}
\pdv{g}{x}\paren{x,y}=\dfrac{-y}{x^2}\times\dfrac{1}{1+\paren{\frac{y}{x}}^2}=\dfrac{-y}{x^2+y^2} \\
\pdv{g}{y}\paren{x,y}=\dfrac{1}{x}\times\dfrac{1}{1+\paren{\frac{y}{x}}^2}=\dfrac{x}{x^2+y^2}
\end{dcases}\]
\end{corr}


\begin{rem}
Soient \(\Omega\) un ouvert de \(\R^2\), \(f:\Omega\to\R\) et \(a\in\Omega\).

Le fait que \(f\) admette des dérivées partielles en \(a\) n'implique pas la continuité de \(f\) en \(a\).
\end{rem}

\subsection{Fonctions de classe \(\classe{1}\)}


\begin{defi}
Soit \(\Omega\) un ouvert de \(\R^2\).

On dit qu'une fonction \(f:\Omega\to\R\) est de classe \(\classe{1}\) si ses dérivées partielles \(\pdv{f}{x}\) et \(\pdv{f}{y}\) sont définies et continues.
\end{defi}


\begin{theo}
Soient \(\Omega\) un ouvert de \(\R^2\) et \(f:\Omega\to\R\) de classe \(\classe{1}\).

Alors \(f\) admet un développement limité à l'ordre 1 en tout point de \(\Omega\).

Soit \(\paren{x_0,y_0}\in\Omega\).

Alors on a : \[f\paren{x_1,y_1}\underset{\paren{x_1,y_1}\to\paren{x_0,y_0}}{=}f\paren{x_0,y_0}+\paren{x_1-x_0}\pdv{f}{x}\paren{x_0,y_0}+\paren{y_1-y_0}\pdv{f}{y}\paren{x_0,y_0}+o\paren{\norme{\paren{x_1-x_0,y_1-y_0}}}\]
\end{theo}

\begin{dem}
On a : \[\begin{aligned}
f\paren{x_1,y_1}-f\paren{x_0,y_0}&=f\paren{x_1,y_1}-f\paren{x_1,y_0}+f\paren{x_1,y_0}-f\paren{x_0,y_0} \\
&=\int_{y_0}^{y_1}\pdv{f}{y}\paren{x_1,t}\odif{t}+\int_{x_0}^{x_1}\pdv{f}{x}\paren{t,y_0}\odif{t} \\
&=\int_{y_0}^{y_1}\paren{\pdv{f}{y}\paren{x_1,t}-\pdv{f}{y}\paren{x_0,y_0}}\odif{t}+\int_{y_0}^{y_1}\pdv{f}{y}\paren{x_0,y_0}\odif{t} \\
&\color{white}=\color{black}+\int_{x_0}^{x_1}\paren{\pdv{f}{x}\paren{t,y_0}-\pdv{f}{x}\paren{x_0,y_0}}\odif{t}+\int_{x_0}^{x_1}\pdv{f}{x}\paren{x_0,y_0}\odif{t}
\end{aligned}\]

Soient \(\epsilon,\delta\in\Rps\) tels que \(\begin{dcases}
\bouleo{\paren{x_0,y_0}}{\delta}\subset\Omega \\
\quantifs{\forall a\in\bouleo{\paren{x_0,y_0}}{\delta}}\begin{dcases}
\abs{\pdv{f}{x}\paren{a}-\pdv{f}{x}\paren{x_0,y_0}}\leq\epsilon \\
\abs{\pdv{f}{y}\paren{a}-\pdv{f}{y}\paren{x_0,y_0}}\leq\epsilon
\end{dcases}
\end{dcases}\)

On a : \[\begin{aligned}
f\paren{x_1,y_1}-f\paren{x_0,y_0}&=\underbrace{\int_{y_0}^{y_1}\underbrace{\paren{\pdv{f}{y}\paren{x_1,t}-\pdv{f}{y}\paren{x_0,y_0}}}_{\abs{\cdot}\leq\epsilon}\odif{t}}_{\leq\abs{y_1-y_0}\epsilon}+\underbrace{\int_{y_0}^{y_1}\pdv{f}{y}\paren{x_0,y_0}\odif{t}}_{=\paren{y_1-y_0}\pdv{f}{y}\paren{x_0,y_0}} \\
&\color{white}=\color{black}+\underbrace{\int_{x_0}^{x_1}\underbrace{\paren{\pdv{f}{x}\paren{t,y_0}-\pdv{f}{x}\paren{x_0,y_0}}}_{\abs{\cdot}\leq\epsilon}\odif{t}}_{\leq\abs{x_1-x_0}\epsilon}+\underbrace{\int_{x_0}^{x_1}\pdv{f}{x}\paren{x_0,y_0}\odif{t}}_{=\paren{x_1-x_0}\pdv{f}{x}\paren{x_0,y_0}}
\end{aligned}\]

Donc \[\begin{aligned}
\int_{y_0}^{y_1}\paren{\pdv{f}{y}\paren{x_1,t}-\pdv{f}{y}\paren{x_0,y_0}}\odif{t}+\int_{x_0}^{x_1}\paren{\pdv{f}{x}\paren{t,y_0}-\pdv{f}{x}\paren{x_0,y_0}}\odif{t}&\leq2\epsilon\norme{\paren{x_1,y_1}-\paren{x_0,y_0}} \\
&=o\paren{\norme{\paren{x_1-x_0,y_1-y_0}}}.
\end{aligned}\]

D'où le résultat.
\end{dem}

\begin{defi}[Gradient]
Soient \(\Omega\) un ouvert de \(\R^2\), \(f\in\ensclasse{1}{\Omega}{\R}\) et \(a\in\Omega\).

On appelle gradient de \(f\) en \(a\) le vecteur \[\nabla f\paren{a}=\paren{\pdv{f}{x}\paren{a},\pdv{f}{y}\paren{a}}.\]

Selon le théorème précédent, il vérifie : \[f\paren{a+h}\underset{h\to\paren{0,0}}{=}f\paren{a}+\ps{\nabla f\paren{a}}{h}+o\paren{\norme{h}}.\]
\end{defi}

\begin{exoex}
Calculer le gradient des fonctions \[\fonction{f}{\R^2\excluant\accol{\paren{0,0}}}{\R}{\paren{x,y}}{\sqrt{x^2+y^2}}\qquad\text{et}\qquad\fonction{g}{\Rps\times\R}{\R}{\paren{x,y}}{\Arctan\dfrac{y}{x}}\]
\end{exoex}

\begin{corr}
On a : \[\quantifs{\forall\paren{x,y}\in\R^2\excluant\accol{\paren{0,0}}}\nabla f\paren{x,y}=\paren{\dfrac{x}{\sqrt{x^2+y^2}},\dfrac{y}{\sqrt{x^2+y^2}}}=\dfrac{1}{\sqrt{x^2+y^2}}\paren{x,y}\] et : \[\quantifs{\forall\paren{x,y}\in\Rps\times\R}\nabla g\paren{x,y}=\paren{\dfrac{-y}{x^2+y^2},\dfrac{x}{x^2+y^2}}=\dfrac{1}{x^2+y^2}\paren{-y,x}.\]
\end{corr}

\subsection{Équations aux dérivées partielles}

\begin{ex}
Résolvons sur \(\R^2\) l'équation aux dérivées partielles \(\paren{E}~\pdv{f}{x}\paren{x,y}=x^2y^2\).

Soit \(f\in\ensclasse{1}{\R^2}{\R}\).

On a : \[f\text{ est solution de }\paren{E}\ssi\quantifs{\exists g\in\ensclasse{1}{\R}{\R};\forall\paren{x,y}\in\R^2}f\paren{x,y}=\dfrac{1}{3}x^3y^2+g\paren{y}.\]
\end{ex}

\begin{ex}
Résolvons sur \(\R^2\) l'équation aux dérivées partielles \(\paren{E}~\pdv[order=2]{f}{x}\paren{x,y}=\cos x+\sin y+\exp\paren{x\e{y}}\).

Soit \(f\in\ensclasse{2}{\R^2}{\R}\) (\cad \(\pdv{f}{x},\pdv{f}{y}\in\ensclasse{1}{\R^2}{\R}\)).

On a : \[\begin{aligned}
f\text{ est solution de }\paren{E}&\ssi\quantifs{\exists g\in\F{\R}{\R};\forall\paren{x,y}\in\R^2}\pdv{f}{x}\paren{x,y}=\sin x+x\sin y \\
&\color{white}\ssi\quantifs{\exists g\in\F{\R}{\R};\forall\paren{x,y}\in\R^2}\pdv{f}{x}\paren{x,y}=\color{black}+\e{-y}\exp\paren{x\e{y}}+g\paren{y} \\
&\ssi\quantifs{\exists g,h\in\F{\R}{\R};\forall\paren{x,y}\in\R^2}f\paren{x,y}=-\cos x+\dfrac{1}{2}x^2\sin y \\
&\color{white}\ssi\quantifs{\exists g,h\in\F{\R}{\R};\forall\paren{x,y}\in\R^2}f\paren{x,y}=\color{black}+\e{-2y}\exp\paren{x\e{y}}+xg\paren{y}+h\paren{y}
\end{aligned}\] où \(g,h\in\ensclasse{2}{\R}{\R}\).
\end{ex}

% le cours en est là pour le moment

\chapter{Probabilités}

\minitoc

\section{Rappels et compléments sur les ensembles}

\subsection{Ensembles finis}

\begin{defi}
Un ensemble \(E\) est dit fini s'il contient un nombre fini d'éléments.

Le nombre d'éléments de \(E\) est appelé cardinal de \(E\) et est noté \(\abs{E}\), \(\Card E\) ou encore \(\#E\).
\end{defi}

\begin{theo}
Soient \(E\) un ensemble fini et \(A\subset E\) une partie de \(E\).

Alors \(A\) est un ensemble fini.

De plus, on a : \[\Card A\leq\Card E\] avec égalité si, et seulement si, \(A=E\).
\end{theo}

\begin{theo}
Soient \(E\) et \(F\) deux ensembles finis de même cardinal et une fonction \(f:E\to F\).

Les trois propositions suivantes sont équivalentes :

\begin{enumerate}
    \item \(f\) est une bijection \\
    \item \(f\) est une injection \\
    \item \(f\) est une surjection.
\end{enumerate}
\end{theo}

\subsection{Dénombrement des ensembles finis}

Soient \(E\) un ensemble fini et \(A,B\in\P{E}\).

On a :

\begin{description}
    \item[] \(\Card A\union B=\Card A+\Card B\) si \(A\) et \(B\) sont disjoints \\
    \item[] \(\Card A\union B=\Card A+\Card B-\Card A\inter B\) \\
    \item[] \(\Card E\excluant A=\Card E-\Card A\) \\
    \item[] \(\Card A\times B=\Card A\times\Card B\) \\
    \item[] \(\Card B^A=\Card\F{A}{B}=\paren{\Card B}^{\Card A}\) \\
    \item[] \(\Card\P{E}=2^{\Card E}\) \\
    \item[] \(\Card E^p=\paren{\Card E}^p\).
\end{description}

Soit \(p\in\interventierii{0}{n}\).

On appelle :

\begin{description}
    \item[] \(p\)-arrangement de \(E\) tout \(p\)-uplet d'éléments de \(E\) deux à deux distincts ; \\
    \item[] \(p\)-combinaison de \(E\) toute partie de \(E\) contenant \(p\) éléments.
\end{description}

On a alors, en notant \(n\) le cardinal de \(E\) :

\begin{description}
    \item[] le nombre de \(p\)-arrangements de \(E\) est \(\arr{p}{n}=\dfrac{n!}{\paren{n-p}!}\) \\
    \item[] le nombre de \(p\)-combinaisons de \(E\) est \(\comb{p}{n}=\binom{p}{n}=\dfrac{n!}{p!\,\paren{n-p}!}\)
\end{description}

Enfin, \(\arr{p}{n}\) est aussi le nombre d'injections d'un ensemble de cardinal \(p\) vers \(E\). En particulier, si \(p=n\) et en notant \(\S{E}\) l'ensemble des bijections de \(E\) vers \(E\) : \[\Card\S{E}=n!\]

\begin{exoex}
Soient \(k,n\in\Ns\) tels que \(k\leq n\).

Une urne contient \(n\) boules numérotées de \(1\) à \(n\). On tire \(k\) boules dans cette urne.

Donner le nombre de résultats possibles en fonction du type de tirage :

\begin{enumerate}
    \item On tire \(k\) boules simultanément. \\
    \item On tire \(k\) boules une par une, avec remise. \\
    \item On tire \(k\) boules une par une, sans remise.
\end{enumerate}
\end{exoex}

\begin{corr}~\\
\begin{enumerate}
    \item Il y a \(\binom{k}{n}\) résultats possibles. \\
    \item Il y a \(n^k\) résultats possibles. \\
    \item Il y a \(\arr{k}{n}\) résultats possibles.
\end{enumerate}
\end{corr}

\begin{exoex}[Mines-Télécom 2016]
Soit \(E\) un ensemble fini dont on note \(n\) le cardinal.

Donner le cardinal des ensembles suivants :

\begin{enumerate}
    \item \(E_1=\accol{\paren{X,Y}\in\P{E}^2\tq\accol{X;Y}\text{ est une partition de }E}\) ; \\
    \item \(E_2=\accol{\paren{X,Y}\in\P{E}^2\tq X\inter Y=\ensvide}\) ; \\
    \item \(E_3=\accol{\paren{X,Y}\in\P{E}^2\tq X\union Y=E}\) ; \\
    \item \(E_4=\accol{\paren{X,Y}\in\P{E}^2\tq X\subset Y}\) ; \\
    \item \(E_5=\accol{\paren{X,Y,Z}\in\P{E}^3\tq X\union Y\union Z=E}\).
\end{enumerate}
\end{exoex}

\begin{corr}
On a :

\begin{enumerate}
    \item \(\Card E_1=2^n-2\) \\
    \item \(\Card E_2=\sum_{k=0}^{n}\binom{k}{n}2^{n-k}=3^n\) \\
    \item \(\Card E_3=3^n\) \\
    \item \(\Card E_4=3^n\) \\
    \item \(\Card E_5=7^n\)
\end{enumerate}
\end{corr}

\section{Espaces probabilisés}

\subsection{Cadre formel}

\subsubsection{Univers}

On considère un ensemble fini non-vide \(\Omega\) appelé l'univers.

\begin{ex}
\begin{itemize}
    \item On lance un dé une fois : \(\Omega=\interventierii{1}{6}\). \\
    \item On lance un dé deux fois : \(\Omega=\interventierii{1}{6}^2\). \\
    \item On lance un dé \(n\) fois : \(\Omega=\interventierii{1}{6}^n\).
\end{itemize}
\end{ex}

\begin{rem}
En pratique, on ne précisera pas quel est l'univers (il y a beaucoup de façons de modéliser une situation et la modélisation choisie ne change pas les probabilités obtenues).
\end{rem}

\subsubsection{Événements}

Les parties de \(\Omega\) sont appelées les événements.

\begin{ex}
\begin{itemize}
    \item Expérience aléatoire : on lance un dé une fois. \\ Univers : \(\Omega=\interventierii{1}{6}\). \\ Événement \guillemets{le résultat vaut \(6\)} : \(\accol{6}\). \\ Événement \guillemets{le résultat est pair} : \(\accol{2;4;6}\). \\
    \item Expérience aléatoire : on lance un dé deux fois. \\ Univers : \(\Omega=\interventierii{1}{6}^2\). \\ Événement \guillemets{les deux lancers donnent le même résultat} : \(\accol{\paren{1,1};\paren{2,2};\paren{3,3};\paren{4,4};\paren{5,5};\paren{6,6}}\).
\end{itemize}
\end{ex}

\begin{defi}
L'ensemble vide \(\ensvide\) est un événement, appelé l'événement impossible.

L'événement \(\Omega\) est appelé l'événement certain.

Les événements qui ne contiennent qu'un seul élément (\ie les singletons) sont appelés événements élémentaires.
\end{defi}

\begin{defi}
Deux événements sont dits incompatibles (ou disjoints) si leur intersection est vide.
\end{defi}

\begin{defi}[Système complet d'événements]
Soient \(\Omega\) un univers et \(N\in\Ns\).

On appelle système complet d'événements toute famille \(\paren{A_1,\dots,A_N}\in\P{\Omega}^N\) d'événements deux à deux incompatibles telle que \[\bigunion_{n=1}^NA_n=\Omega.\]
\end{defi}

\subsubsection{Probabilité}

\begin{defi}[Probabilité]
Soit \(\Omega\) un univers.

On appelle probabilité sur \(\Omega\) toute application \(\prem:\P{\Omega}\to\intervii{0}{1}\) telle que :

\begin{enumerate}
    \item \(\proba{\Omega}=1\) \\
    \item Pour tout entier \(N\in\Ns\) et toute famille \(\paren{A_1,\dots,A_N}\in\P{\Omega}^N\) d'événements deux à deux incompatibles : \[\proba{\bigunion_{n=1}^NA_n}=\sum_{n=1}^N\proba{A_n}.\]
\end{enumerate}
\end{defi}

\begin{defi}[Espace probabilisé]
Un couple \(\groupe{\Omega}[\prem]\), où \(\Omega\) est un univers et \(\prem\) est une probabilité sur \(\Omega\), est appelé un espace probabilisé.
\end{defi}

\begin{rem}
Soit \(\groupe{\Omega}[\prem]\) un espace probabilisé.

On a : \[\proba{\ensvide}=0.\]
\end{rem}

\begin{dem}
On a \(\ensvide=\ensvide\union\ensvide\) (événements incompatibles) donc \(\proba{\ensvide}=\proba{\ensvide}+\proba{\ensvide}\).

Donc \(\proba{\ensvide}=0\).
\end{dem}

\begin{rem}
Soit \(\Omega\) un univers.

Dans ce chapitre, on note généralement \(\conj{A}\) le complémentaire dans \(\Omega\) de l'événement \(A\in\P{\Omega}\).

Le complémentaire d'une partie n'est jamais noté ainsi dans le cadre de la topologie (notation alors réservée à l'adhérence).
\end{rem}

\begin{prop}
Soient \(\groupe{\Omega}[\prem]\) un espace probabilisé et \(A,B\in\P{\Omega}\) deux événements.

\begin{enumerate}
    \item Si \(A\subset B\) alors \(\proba{B\excluant A}=\proba{B}-\proba{A}\). En particulier, on a \(\proba{A}\leq\proba{B}\). \\
    \item On a \(\proba{\conj{A}}=1-\proba{A}\). \\
    \item On a \(\proba{A\union B}=\proba{A}+\proba{B}-\proba{A\inter B}\).
\end{enumerate}
\end{prop}

\begin{dem}[1]
On a \(B=A\union\paren{B\excluant A}\) (réunion disjointe) donc \(\proba{B}=\proba{A}+\proba{B\excluant A}\).
\end{dem}

\begin{dem}[2]
Découle du (1) en prenant \(B=\Omega\).
\end{dem}

\begin{dem}[3]
On a \(A\union B=\paren{A\excluant B}\union\paren{A\inter B}\union\paren{B\excluant A}\) (réunion disjointe).

Donc \(\proba{A\union B}=\proba{A\excluant B}+\proba{A\inter B}+\proba{B\excluant A}\).

D'autre part, on a \(\begin{dcases}
A=\paren{A\excluant B}\union\paren{A\inter B} \\
B=\paren{B\excluant A}\union\paren{A\inter B}
\end{dcases}\) (réunions disjointes).

Donc on a \(\begin{dcases}
\proba{A}=\proba{A\excluant B}+\proba{A\inter B} \\
\proba{B}=\proba{B\excluant A}+\proba{A\inter B}
\end{dcases}\)

D'où la formule.
\end{dem}

\begin{prop}[Sous-additivité]
Soient \(\groupe{\Omega}[\prem]\) un espace probabilisé et \(N\in\Ns\).

Pour toute famille d'événements \(\paren{A_1,\dots,A_N}\in\P{\Omega}^N\), on a : \[\proba{\bigunion_{n=1}^NA_n}\leq\sum_{n=1}^N\proba{A_n}.\]
\end{prop}

\begin{dem}
On pose \(\quantifs{\forall k\in\interventierii{1}{N}}A_k\prim=A_k\excluant\bigunion_{i=1}^{k-1}A_i\) de sorte que \(\bigunion_{k=1}^NA_k=\bigunion_{k=1}^NA_k\prim\) (réunion disjointe).

On a alors : \[\begin{WithArrows}
\proba{\bigunion_{k=1}^NA_k}&=\sum_{k=1}^N\proba{A_k\prim} \Arrow{car \(\quantifs{\forall k\in\interventierii{1}{N}}A_k\prim\subset A_k\)} \\
&\leq\sum_{k=1}^N\proba{A_k}
\end{WithArrows}\]
\end{dem}

\subsubsection{Distribution de probabilités}

\begin{defi}
Soit \(\Omega\) un univers.

On appelle distribution de probabilités sur \(\Omega\) toute famille \(\paren{p_\omega}_{\omega\in\Omega}\) de réels positifs de somme \(1\) : \[\quantifs{\forall\omega\in\Omega}p_\omega\geq0\qquad\text{et}\qquad\sum_{\omega\in\Omega}p_\omega=1.\]
\end{defi}

\begin{prop}
Soit \(\Omega\) un univers.

La donnée d'une probabilité sur \(\Omega\) revient à la donnée d'une distribution de probabilités sur \(\Omega\).
\end{prop}

\begin{dem}
\analyse

Soient \(\prem\) une probabilité sur \(\Omega\) et \(A\subset\Omega\).

On a \(A=\bigunion_{\omega\in A}\accol{\omega}\) (réunion disjointe).

Donc, en posant \(\quantifs{\forall\omega\in\Omega}p_\omega=\proba{\accol{\omega}}\), on a : \[\proba{A}=\sum_{\omega\in A}\proba{\accol{\omega}}=p_\omega.\]

On a bien \(\begin{dcases}
\quantifs{\forall\omega\in\Omega}p_\omega\geq0 \\
\sum_{\omega\in\Omega}p_\omega=\proba{\Omega}=1
\end{dcases}\)

Donc \(\paren{p_\omega}_{\omega\in\Omega}\) est une distribution de probabilités sur \(\Omega\).

\synthese

Soit \(\paren{p_\omega}_{\omega\in\Omega}\) une distribution de probabilités sur \(\Omega\).

On pose : \[\fonction{\prem}{\P{\Omega}}{\R}{A}{\sum_{\omega\in A}p_\omega}\]

Montrons que \(\prem\) est une probabilité sur \(\Omega\).

Soient \(N\in\Ns\) et \(A_1,\dots,A_N\in\P{\Omega}\) des événements deux à deux incompatibles.

On a : \[\begin{WithArrows}
\proba{\bigunion_{i=1}^NA_i}&=\sum_{\omega\in\bigunion_{i=1}^NA_i}p_\omega \Arrow{car \(\bigunion_{i=1}^NA_i\) est une réunion disjointe} \\
&=\sum_{i=1}^N\sum_{\omega\in A_i}p_\omega \\
&=\sum_{i=1}^N\proba{A_i}.
\end{WithArrows}\]

De plus, on a \(\proba{\Omega}=\sum_{\omega\in\Omega}p_\omega=1\).

Donc \(\prem\) est une probabilité sur \(\Omega\).
\end{dem}

\subsection{Exemple : probabilité uniforme sur un ensemble fini}

\begin{defi}[Probabilité uniforme]
Soit \(\Omega\) un univers.

On appelle probabilité uniforme sur \(\Omega\) la probabilité \(\prem\) définie par : \[\quantifs{\forall A\in\P{\Omega}}\proba{A}=\dfrac{\Card A}{\Card\Omega}.\]

Pour cette probabilité, tous les événements élémentaires sont équiprobables, de probabilité \(\dfrac{1}{\Card\Omega}\).
\end{defi}

\begin{ex}[Lancer d'un dé]
On lance un dé (non-pipé, à six faces).

Le résultat peut être modélisé par l'univers \(\Omega=\interventierii{1}{6}\) muni de la probabilité uniforme.
\end{ex}

\begin{exoex}
On tire simultanément trois cartes dans un jeu de trente-deux cartes. On suppose que tous les tirages sont équiprobables.

\begin{enumerate}
    \item Quelle est la probabilité de tirer trois as ? \\
    \item Quelle est la probabilité de tirer trois cartes qui se suivent ? \\
    \item Quelle est la probabilité de tirer deux rois et une dame ?
\end{enumerate}
\end{exoex}

\begin{corr}[1]~\\
On a \(\Card\Omega=\binom{3}{32}\).

On note \(A\) l'événement \guillemets{on tire trois as}.

On a \(\Card A=\binom{3}{4}\).

Donc : \[\begin{aligned}
\proba{A}&=\dfrac{\Card A}{\Card\Omega} \\
&=\dfrac{\binom{3}{4}}{\binom{3}{32}} \\
&=\dfrac{\frac{4\times3\times2}{3\times2\times1}}{\frac{32\times31\times30}{3\times2\times1}} \\
&=\dfrac{4\times3\times2}{32\times31\times30} \\
&=\dfrac{1}{4\times31\times10} \\
&=\dfrac{1}{1240}.
\end{aligned}\]
\end{corr}

\begin{corr}[2]
On note \(B\) l'événement \guillemets{on tire trois cartes qui se suivent}.

On a \(\Card B=6\times4^3\).

Donc : \[\begin{aligned}
\proba{B}&=\dfrac{6\times4^3}{\frac{32\times31\times30}{3\times2\times1}} \\
&=\dfrac{2\times3\times6\times2^6}{30\times31\times32} \\
&=\dfrac{2^2\times3}{5\times31} \\
&=\dfrac{12}{155}.
\end{aligned}\]
\end{corr}

\begin{corr}[3]
On note \(C\) l'événement \guillemets{on tire deux rois et une dame}.

On a \(\Card C=\binom{2}{4}\times\binom{1}{4}=4\times6\).

Donc : \[\begin{aligned}
\proba{C}&=\dfrac{6\times4}{\frac{32\times31\times30}{3\times2\times1}} \\
&=\dfrac{2^4\times3^2}{2^5\times31\times2\times5\times3} \\
&=\dfrac{3}{2^2\times5\times31} \\
&=\dfrac{3}{620}.
\end{aligned}\]
\end{corr}

\section{Conditionnement}

\subsection{Définitions}

On considère un espace probabilisé \(\groupe{\Omega}[\prem]\).

\begin{defi}
Soient \(A,B\in\P{\Omega}\) deux événements tels que \(\proba{B}\not=0\).

On appelle probabilité conditionnelle de \(A\) sachant \(B\) le réel : \[\probacond{A}{B}=\proba{A\mid B}=\dfrac{\proba{A\inter B}}{\proba{B}}.\]
\end{defi}

\begin{rem}\thlabel{rem:probaIntersectionDeuxÉvénements}
On a donc \(\proba{A\inter B}=\proba{B}\proba{A\mid B}\).
\end{rem}

\begin{exoex}
On lance deux dés.

\begin{enumerate}
    \item Quelle est la probabilité d'avoir au total au moins \(10\) ? \\
    \item Quelle est la probabilité d'avoir au total au moins \(10\) sachant que l'un des deux dés donne \(2\) ? \\
    \item Quelle est la probabilité d'avoir au total au moins \(10\) sachant que l'un des deux dés donne \(6\) ?
\end{enumerate}
\end{exoex}

\begin{corr}[1]
On a l'univers \(\Omega=\interventierii{1}{6}^2\) de cardinal \(36\).

On note \(A\) l'événement \guillemets{avoir au total au moins 10}.

On a \(A=\accol{\paren{4,6};\paren{5,5};\paren{5,6};\paren{6,4};\paren{6,5};\paren{6,6}}\) donc \(\Card A=6\).

Donc \(\proba{A}=\dfrac{6}{36}=\dfrac{1}{6}\).
\end{corr}

\begin{corr}[2]
La probabilité est nulle.
\end{corr}

\begin{corr}[3]
On note \(A\) l'événement \guillemets{avoir au total au moins 10} et \(B\) l'événement \guillemets{l'un des deux dés donne 6}.

On a : \[\proba{A\mid B}=\dfrac{\proba{A\inter B}}{\proba{B}}=\dfrac{5}{11}.\]
\end{corr}

\begin{prop}
Soit \(B\) un événement de probabilité non-nulle.

L'application \(\prem_B:\P{\Omega}\to\intervii{0}{1}\) est une probabilité sur \(\Omega\).
\end{prop}

\subsection{Propriétés}

\subsubsection{Formule des probabilités composées}

\begin{prop}
Soient \(N\in\Ns\) et \(A_1,\dots,A_N\) des événements tels que \(\proba{A_1\inter\dots\inter A_{N-1}}\not=0\).

Alors \[\proba{A_1\inter\dots\inter A_N}=\proba{A_1}\probacond{A_2}{A_1}\probacond{A_3}{A_1\inter A_2}\dots\probacond{A_N}{A_1\inter\dots\inter A_{N-1}}.\]
\end{prop}

\begin{dem}
Découle de la \thref{rem:probaIntersectionDeuxÉvénements}, par récurrence sur \(N\).
\end{dem}

\subsubsection{Formule des probabilités totales}

\begin{prop}[Formule des probabilités totales]
Soient \(N\in\Ns\), \(\paren{A_1,\dots,A_N}\) un système complet fini d'événements et \(B\) un événement.

On a : \[\proba{B}=\sum_{n=1}^N\proba{B\inter A_n}=\sum_{n=1}^N\proba{B\mid A_n}\proba{A_n},\] avec la convention \(\proba{B\mid A_n}\proba{A_n}=0\) si \(\proba{A_n}=0\).
\end{prop}

\begin{dem}~\\
On a \(B=\bigunion_{n=1}^N\paren{A_n\inter B}\) (réunion disjointe).

Donc \[\begin{WithArrows}
\proba{B}&=\sum_{n=1}^N\proba{A_n\inter B} \Arrow[tikz={text width=4cm}]{selon la formule des probabilités composées} \\
&=\sum_{n=1}^N\proba{A_n}\proba{B\mid A_n}.
\end{WithArrows}\]
\end{dem}

\subsubsection{Formules de Bayes}

\begin{prop}[Formule de Bayes]
\begin{itemize}
    \item Soient \(A\) et \(B\) deux événements de probabilité non-nulle. On a : \[\proba{A\mid B}=\dfrac{\proba{A}\proba{B\mid A}}{\proba{B}}=\dfrac{\proba{A}\proba{B\mid A}}{\proba{A}\proba{B\mid A}+\proba{\conj{A}}\proba{B\mid\conj{A}}}.\]
    \item Soient \(N\in\Ns\), \(\paren{A_1,\dots,A_n}\) un système complet fini d'événements de probabilité non-nulle, \(B\) un événement de probabilité non-nulle et \(k\in\interventierii{1}{N}\). On a : \[\proba{A_k\mid B}=\dfrac{\proba{A_k}\proba{B\mid A_k}}{\ds\sum_{n=1}^{N}\proba{A_n}\proba{B\mid A_n}}.\]
\end{itemize}
\end{prop}

\begin{exoex}
Dans une population de 60 millions d'individus, 6000 personnes sont porteuses d'un virus.

Il existe un test, mais celui-ci n'est pas complètement fiable :

Si un patient sain est testé, le résultat est négatif avec une probabilité de 98\%.

Si un patient malade est testé, le résultat est positif avec une probabilité de 99\%.

Un patient a fait un test dont le résultat est positif.

Quelle est la probabilité qu'il soit malade ?
\end{exoex}

\begin{corr}
On note \(M\) l'événement \guillemets{le patient est malade} et \(P\) l'événement \guillemets{le test est positif}.

On a : \[\begin{WithArrows}
\proba{M\mid P}&=\dfrac{\proba{M}\proba{P\mid M}}{\proba{M}\proba{P\mid M}+\proba{\conj{M}}\proba{P\mid\conj{M}}} \\
&=\dfrac{\num{e-4}\times\num{0.99}}{\num{e-2}\times\num{0.99}+\paren{1-\num{e-4}}\times\num{0.02}} \Arrow{on calcule} \\
&=\dfrac{1}{203}.
\end{WithArrows}\]
\end{corr}

\begin{exoex}
Je possède 20 dés : 18 sont non-pipés (probabilité uniforme sur \(\interventierii{1}{6}\)) et 2 sont pipés (une chance sur deux d'obtenir 6 quand on les lance).

Je prends l'un des 20 dés, le lance et obtiens 6.

Quelle est la probabilité que ce dé soit truqué ? Et si j'obtiens autre chose que 6 ?
\end{exoex}

\begin{corr}
On note \(P\) l'événement \guillemets{le dé est pipé} et \(S\) l'événement \guillemets{on obtient un 6}.

On a : \[\begin{WithArrows}
\proba{P\mid S}&=\dfrac{\proba{P}\proba{S\mid P}}{\proba{P}\proba{S\mid P}+\proba{\conj{P}}\proba{S\mid\conj{P}}} \\
&=\dfrac{\frac{1}{10}\times\frac{1}{2}}{\frac{1}{10}\times\frac{1}{2}+\frac{9}{10}\times\frac{1}{6}} \Arrow{on calcule} \\
&=\dfrac{1}{4}.
\end{WithArrows}\] et \[\begin{WithArrows}
\proba{P\mid\conj{S}}&=\dfrac{\proba{P}\proba{\conj{S}\mid P}}{\proba{P}\proba{\conj{S}\mid P}+\proba{\conj{P}}\proba{\conj{S}\mid\conj{P}}} \\
&=\dfrac{\frac{1}{10}\times\frac{1}{2}}{\frac{1}{10}\times\frac{1}{2}+\frac{9}{10}\times\frac{5}{6}} \Arrow{on calcule} \\
&=\dfrac{1}{16}.
\end{WithArrows}\]
\end{corr}

\section{Événements indépendants}

On considère toujours un espace probabilisé \(\groupe{\Omega}[\prem]\).

\begin{defi}
Deux événements \(A\) et \(B\) sont dits indépendants si on a : \[\proba{A\inter B}=\proba{A}\proba{B}.\]
\end{defi}

\begin{ex}
On pioche une carte dans un jeu de 52 cartes.

Les événements \guillemets{la carte est un 7} et \guillemets{la carte est un cœur} sont indépendants.
\end{ex}

\begin{rem}
Soient deux événements \(A\) et \(B\) tels que \(\proba{B}\not=0\).

Alors \[A\text{ et }B\text{ sont indépendants}\ssi\proba{A}=\proba{A\mid B}.\]
\end{rem}

\begin{dem}
On a : \[\begin{WithArrows}
A\text{ et }B\text{ sont indépendants}&\ssi\proba{A\inter B}=\proba{A}\proba{B} \Arrow{car \(\proba{B}\not=0\)} \\
&\ssi\dfrac{\proba{A\inter B}}{\proba{B}}=\proba{A} \\
&\ssi\proba{A\mid B}=\proba{A}.
\end{WithArrows}\]
\end{dem}

\begin{defi}
Soit \(N\in\Ns\).

Des événements \(A_1,\dots,A_N\) sont dits :

\begin{itemize}
    \item deux à deux indépendants si pour tous \(i,j\in\interventierii{1}{N}\) tels que \(i\not=j\) on a : \[\proba{A_i\inter A_j}=\proba{A_i}\proba{A_j}.\]
    \item mutuellement indépendants si pour tout \(J\subset\interventierii{1}{N}\) on a : \[\proba{\biginter_{j\in J}A_j}=\prod_{j\in J}\proba{A_j}.\]
\end{itemize}
\end{defi}

\begin{rem}
On a : \[\text{indépendance mutuelle}\imp\text{indépendance deux à deux}.\]

La réciproque est fausse.
\end{rem}

\begin{exoex}
Soient \(A\) et \(B\) deux événements indépendants.

Montrer que \(A\) et \(\conj{B}\) sont indépendants, que \(\conj{A}\) et \(B\) sont indépendants et que \(\conj{A}\) et \(\conj{B}\) sont indépendants.
\end{exoex}

\begin{corr}
On a \(\proba{A\inter B}=\proba{A}\proba{B}\).

Donc, comme \(A=\paren{A\inter B}\union\paren{A\inter\conj{B}}\) : \[\begin{aligned}
\proba{A\inter\conj{B}}&=\proba{A}-\proba{A\inter B} \\
&=\proba{A}-\proba{A}\proba{B} \\
&=\proba{A}\paren{1-\proba{B}} \\
&=\proba{A}\proba{\conj{B}}.
\end{aligned}\]

Donc \(A\) et \(\conj{B}\) sont indépendants.

Idem pour les autres.
\end{corr}

\section{Variables aléatoires}

On considère toujours un espace probabilisé \(\groupe{\Omega}[\prem]\).

\subsection{Définition}

\begin{defi}[Variable aléatoire]
On appelle variable aléatoire sur \(\Omega\) toute application de l'univers \(\Omega\) vers un ensemble \(E\) : \[X:\Omega\to E.\]

Si \(E=\R\), on dit que \(X\) est une variable aléatoire réelle.
\end{defi}

\begin{nota}
Soient \(E\) et \(F\) des ensembles, \(X:\Omega\to E\) une variable aléatoire et \(f:E\to F\) une fonction quelconque.

Alors \(f\rond X\) est une variable aléatoire, notée abusivement \(f\paren{X}\).

Par exemple, la valeur absolue d'une variable aléatoire réelle \(X\) est une variable aléatoire réelle notée \(\abs{X}\).
\end{nota}

\begin{rem}
Une somme de variables aléatoires réelles est une variable aléatoire réelle.

Un produit de variables aléatoires réelles est une variable aléatoire réelle.

L'ensemble des variables aléatoires réelles (sur \(\Omega\)) est un anneau et un \(\R\)-espace vectoriel.
\end{rem}

\begin{nota}
Soit \(E\) un ensemble.

Lorsque \(X:\Omega\to E\) est une variable aléatoire, on s'autorise, pour tout \(F\subset E\), à noter \(\paren{X\in F}\) ou \(\accol{X\in F}\) l'image réciproque de \(X\) par \(F\) : \[\paren{X\in F}=\accol{X\in F}=X\inv\paren{F}.\]

Il s'agit d'un événement.
\end{nota}

\begin{defi}[Loi d'une variable aléatoire]
Soient \(E\) un ensemble et \(X:\Omega\to E\) une variable aléatoire.

Quitte à remplacer \(E\) par \(\Im X\), on peut supposer que \(E\) est fini.

On appelle loi de \(X\) la probabilité (sur l'univers \(E\)) : \[\fonction{\prem_X}{\P{E}}{\intervii{0}{1}}{F}{\proba{\accol{X\in F}}}\]

On note simplement \(\probacond{F}{X}\) ou \(\proba{X\in F}\) la probabilité \(\proba{\accol{X\in F}}\) (intuitivement : la \guillemets{probabilité que \(X\) appartienne à \(F\)}).
\end{defi}

\begin{rem}
Mêmes notations.

Le couple \(\groupe{E}[\prem_X]\) est un espace probabilisé.
\end{rem}

\begin{defi}
Soient \(X\) et \(Y\) deux variables aléatoires.

On dit que \(X\) et \(Y\) suivent la même loi si on a \(\prem_X=\prem_Y\).

On note alors : \[X\sim Y.\]
\end{defi}

\begin{exoex}
On lance deux dés à quatre faces.

Le lancer peut être modélisé en posant : \[\Omega=\interventierii{1}{4}^2\qquad\quantifs{\forall A\in\P{\Omega}}\proba{A}=\dfrac{\Card A}{\Card\Omega}\] et \[\fonction{X}{\Omega}{\interventierii{1}{4}}{\paren{x,y}}{x}\qquad\fonction{Y}{\Omega}{\interventierii{1}{4}}{\paren{x,y}}{y}\]

On pose enfin \(S=X+Y\) et \(P=XY\).

Les fonctions \(X\), \(Y\), \(S\) et \(P\) sont des variables aléatoires.

Quelles sont leurs lois ?
\end{exoex}

\begin{corr}
\(X\) est à valeurs dans \(\interventierii{1}{4}\) et on a : \[\quantifs{\forall k\in\interventierii{1}{4}}\proba{X=k}=\dfrac{1}{4}.\]

Idem pour \(Y\).

\(S\) est à valeurs dans \(\interventierii{2}{8}\) et on a : \[\begin{dcases}
\proba{S=2}=\proba{S=8}=\dfrac{1}{16} \\
\proba{S=3}=\proba{S=7}=\dfrac{1}{8} \\
\proba{S=4}=\proba{S=6}=\dfrac{3}{16} \\
\proba{S=5}=\dfrac{1}{4}
\end{dcases}\]

\(P\) est à valeurs dans \(\accol{1;2;3;4;6;8;9;12;16}\) et on a : \[\begin{dcases}
\proba{P=1}=\proba{P=9}=\proba{P=16}=\dfrac{1}{16} \\
\proba{P=2}=\proba{P=3}=\proba{P=6}=\proba{P=8}=\proba{P=12}=\dfrac{1}{8} \\
\proba{P=4}=\dfrac{3}{16}
\end{dcases}\]
\end{corr}

\subsection{Exemples}

On considère toujours un espace probabilisé \(\groupe{\Omega}[\prem]\) et on note \(X\) une variable aléatoire.

\subsubsection{Loi uniforme sur un ensemble fini}

\begin{defi}
Soit \(E\) un ensemble fini non-vide.

On dit que \(X:\Omega\to E\) suit une loi uniforme sur \(E\) si on a : \[\quantifs{\forall A\in\P{E}}\proba{X\in A}=\dfrac{\Card A}{\Card E}.\]
\end{defi}

\begin{nota}
Mêmes notations.

On note \(\loiuniforme{E}\) la loi uniforme sur \(E\) et \(X\sim\loiuniforme{E}\) si \(X\) suit une loi uniforme sur \(E\).
\end{nota}

\subsubsection{Loi de Bernoulli de paramètre \(p\in\intervii{0}{1}\)}

\begin{defi}
Soit \(p\in\intervii{0}{1}\).

On dit que \(X:\Omega\to\accol{0;1}\) suit une loi de Bernoulli de paramètre \(p\) si on a : \[\begin{dcases}
\proba{X=1}=p \\
\proba{X=0}=1-p
\end{dcases}\]
\end{defi}

\begin{nota}
Mêmes notations.

On note \(\loibernoulli{p}\) la loi de Bernoulli de paramètre \(p\) et \(X\sim\loibernoulli{p}\) si \(X\) suit une loi de Bernoulli de paramètre \(p\).
\end{nota}

\begin{ex}
\begin{itemize}
    \item Pile ou face ; \\
    \item Si \(A\) est un événement de probabilité \(p\), alors \(\ind{A}\sim\loibernoulli{p}\).
\end{itemize}
\end{ex}

\subsubsection{Loi binomiale de paramètres \(n\in\Ns\) et \(p\in\intervii{0}{1}\)}

\begin{defi}
Soient \(n\in\Ns\) et \(p\in\intervii{0}{1}\).

On dit que \(X:\Omega\to\interventierii{0}{n}\) suit une loi binomiale de paramètres \(n\) et \(p\) si on a : \[\quantifs{\forall k\in\interventierii{0}{n}}\proba{X=k}=\binom{k}{n}p^k\paren{1-p}^{n-k}.\]
\end{defi}

\begin{nota}
Mêmes notations.

On note \(\loibinomiale{n}{p}\) la loi binomiale de paramètres \(n\) et \(p\) et \(X\sim\loibinomiale{n}{p}\) si \(X\) suit une loi binomiale de paramètres \(n\) et \(p\).
\end{nota}

\begin{ex}
\begin{itemize}
    \item Si l'on fait \(n\) tirages avec remise dans une urne contenant une proportion \(p\) de boules bleues et qu'on note \(X\) le nombre de boules bleues tirées, alors \(X\sim\loibinomiale{n}{p}\) ; \\
    \item Nombre de \guillemets{piles} quand on lance \(n\) pièces en l'air.
\end{itemize}
\end{ex}

\subsection{Couples de variables aléatoires}

On considère toujours l'espace probabilisé \(\groupe{\Omega}[\prem]\).

Soient \(E\) et \(F\) deux ensembles et \(X:\Omega\to E\) et \(Y:\Omega\to F\) deux variables aléatoires.

On leur associe la variable aléatoire : \[\fonction{Z}{\Omega}{E\times F}{\omega}{\paren{X\paren{\omega},Y\paren{\omega}}}\] (que l'on note parfois \(Z=\paren{X,Y}\)).

\begin{defi}
La loi de \(Z\) est appelée loi conjointe de \(X\) et \(Y\).

Les lois de \(X\) et \(Y\) sont appelées lois marginales de \(Z\).
\end{defi}

\begin{rem}
Si on connaît la loi conjointe, on connaît les lois marginales.

Cependant, connaître les lois marginales ne suffit pas pour connaître la loi conjointe.
\end{rem}

\begin{ex}
On lance deux dés à quatre faces et on note \(X\) et \(Y\) les résultats.

Alors \(\paren{X,Y}\) et \(\paren{X,X}\) ont mêmes lois marginales, mais pas mêmes lois conjointes :

\begin{center}
\begin{tikzpicture}
\node at (2.25,1.5) {\(\paren{X,Y}\)};
\draw[->] (0,0) -- (4.5,0) node[above right] {\(Y\)};
\draw[->] (0,0) -- (0,-4.5) node[below left] {\(X\)};
\draw (0,0) grid (4,-4);
\node at (0.5,-0.5) {\(\nicefrac{1}{16}\)};
\node at (1.5,-0.5) {\(\nicefrac{1}{16}\)};
\node at (2.5,-0.5) {\(\nicefrac{1}{16}\)};
\node at (3.5,-0.5) {\(\nicefrac{1}{16}\)};
\node at (0.5,-1.5) {\(\nicefrac{1}{16}\)};
\node at (1.5,-1.5) {\(\nicefrac{1}{16}\)};
\node at (2.5,-1.5) {\(\nicefrac{1}{16}\)};
\node at (3.5,-1.5) {\(\nicefrac{1}{16}\)};
\node at (0.5,-2.5) {\(\nicefrac{1}{16}\)};
\node at (1.5,-2.5) {\(\nicefrac{1}{16}\)};
\node at (2.5,-2.5) {\(\nicefrac{1}{16}\)};
\node at (3.5,-2.5) {\(\nicefrac{1}{16}\)};
\node at (0.5,-3.5) {\(\nicefrac{1}{16}\)};
\node at (1.5,-3.5) {\(\nicefrac{1}{16}\)};
\node at (2.5,-3.5) {\(\nicefrac{1}{16}\)};
\node at (3.5,-3.5) {\(\nicefrac{1}{16}\)};
\node at (0.5,0.5) {\(\nicefrac{1}{4}\)};
\node at (1.5,0.5) {\(\nicefrac{1}{4}\)};
\node at (2.5,0.5) {\(\nicefrac{1}{4}\)};
\node at (3.5,0.5) {\(\nicefrac{1}{4}\)};
\node at (-0.5,-0.5) {\(\nicefrac{1}{4}\)};
\node at (-0.5,-1.5) {\(\nicefrac{1}{4}\)};
\node at (-0.5,-2.5) {\(\nicefrac{1}{4}\)};
\node at (-0.5,-3.5) {\(\nicefrac{1}{4}\)};

\node at (8.25,1.5) {\(\paren{X,X}\)};
\draw[->] (6,0) -- (10.5,0) node[above right] {\(X\)};
\draw[->] (6,0) -- (6,-4.5) node[below left] {\(X\)};
\draw (6,0) grid (10,-4);
\node at (6.5,-0.5) {\(\nicefrac{1}{4}\)};
\node at (7.5,-0.5) {\(0\)};
\node at (8.5,-0.5) {\(0\)};
\node at (9.5,-0.5) {\(0\)};
\node at (6.5,-1.5) {\(0\)};
\node at (7.5,-1.5) {\(\nicefrac{1}{4}\)};
\node at (8.5,-1.5) {\(0\)};
\node at (9.5,-1.5) {\(0\)};
\node at (6.5,-2.5) {\(0\)};
\node at (7.5,-2.5) {\(0\)};
\node at (8.5,-2.5) {\(\nicefrac{1}{4}\)};
\node at (9.5,-2.5) {\(0\)};
\node at (6.5,-3.5) {\(0\)};
\node at (7.5,-3.5) {\(0\)};
\node at (8.5,-3.5) {\(0\)};
\node at (9.5,-3.5) {\(\nicefrac{1}{4}\)};
\node at (6.5,0.5) {\(\nicefrac{1}{4}\)};
\node at (7.5,0.5) {\(\nicefrac{1}{4}\)};
\node at (8.5,0.5) {\(\nicefrac{1}{4}\)};
\node at (9.5,0.5) {\(\nicefrac{1}{4}\)};
\node at (5.5,-0.5) {\(\nicefrac{1}{4}\)};
\node at (5.5,-1.5) {\(\nicefrac{1}{4}\)};
\node at (5.5,-2.5) {\(\nicefrac{1}{4}\)};
\node at (5.5,-3.5) {\(\nicefrac{1}{4}\)};
\end{tikzpicture}
\end{center}
\end{ex}

\begin{defi}
Soit \(y\in F\) tel que \(\proba{Y=y}\not=0\).

On appelle loi conditionnelle de \(X\) sachant \(Y=y\) la loi de \(X\) pour la probabilité conditionnelle \(\prem_{\accol{Y=y}}\).

On a : \[\quantifs{\forall A\subset E}\proba{X\in A\mid Y=y}=\dfrac{\proba{X\in A\text{ et }Y=y}}{\proba{Y=y}}.\]
\end{defi}

\begin{rem}
La connaissance de la loi de \(X\) et de la loi de \(X\) sachant \(Y=y\) pour tout \(y\in F\) tel que \(\proba{Y=y}\not=0\) détermine complètement la loi conjointe de \(Z\).
\end{rem}

\begin{rem}[Vecteurs aléatoires]
Soit \(n\in\Ns\).

Plus généralement, si \(X_1,\dots,X_n\) sont des variables aléatoires sur \(\groupe{\Omega}[\prem]\), on peut considérer le vecteur aléatoire \(Z=\paren{X_1,\dots,X_n}\).

La loi de \(Z\) est appelée loi conjointe des variables \(X_1,\dots,X_n\) et les lois de \(X_1,\dots,X_n\) sont appelées lois marginales de \(Z\).
\end{rem}

\section{Variables aléatoires indépendantes}

\subsection{Définition}

On considère toujours l'espace probabilisé \(\groupe{\Omega}[\prem]\) ainsi que deux ensembles \(E\) et \(F\) et deux variables aléatoires \(X:\Omega\to E\) et \(Y:\Omega\to F\).

\begin{defi}\thlabel{defi:variablesAléatoiresIndépendantes}
Les variables aléatoires \(X\) et \(Y\) sont dites indépendantes si : \[\quantifs{\forall x\in E;\forall y\in F}\proba{X=x\text{ et }Y=y}=\proba{X=x}\proba{Y=y}.\]
\end{defi}

\begin{prop}\thlabel{prop:caractérisationVariablesAléatoiresIndépendantes}
Les variables aléatoires \(X\) et \(Y\) sont indépendantes si, et seulement si : \[\quantifs{\forall A\subset E;\forall B\subset F}\proba{X\in A\text{ et }Y\in B}=\proba{X\in A}\proba{Y\in B}.\]
\end{prop}

\begin{dem}
\imprec Claire en prenant \(A=\accol{x}\) et \(B=\accol{y}\) pour tous \(x\in E\) et \(y\in F\).

\impdir

Supposons \(X\) et \(Y\) indépendantes.

Soient \(A\subset E\) et \(B\subset F\).

On a \(\accol{X\in A\text{ et }Y\in B}=\bigunion_{x\in A}\bigunion_{y\in B}\accol{X=x\text{ et }Y=y}\) (réunion disjointe).

Donc : \[\begin{WithArrows}
\proba{X\in A\text{ et }Y\in B}&=\sum_{x\in A}\sum_{y\in B}\proba{X=x\text{ et }Y=y} \Arrow[tikz={text width=3cm}]{car \(X\) et \(Y\) sont indépendantes} \\
&=\sum_{x\in A}\sum_{y\in B}\proba{X=x}\proba{Y=y} \\
&=\paren{\sum_{x\in A}\proba{X=x}}\paren{\sum_{y\in B}\proba{Y=y}} \Arrow{\(\star\)} \\
&=\proba{X\in A}\proba{Y\in B}
\end{WithArrows}\]

\(\star\) : car \(\accol{X\in A}=\bigunion_{x\in A}\accol{X=x}\) et \(\accol{Y\in B}=\bigunion_{y\in B}\accol{Y=y}\) (réunions disjointes).
\end{dem}

\begin{exoex}
On modélise le lancer de deux dés par deux variables aléatoires \(X:\Omega\to\interventierii{1}{6}\) et \(Y:\Omega\to\interventierii{1}{6}\).

La loi conjointe de \(X\) et \(Y\) est la loi uniforme sur \(\interventierii{1}{6}^2\), \cad : \[\quantifs{\forall\paren{k,l}\in\interventierii{1}{6}^2}\proba{X=k\text{ et }Y=l}=\dfrac{1}{36}.\]

Posons \(S=X+Y\) et \(P=XY\).

Les variables aléatoires \(X\) et \(Y\) sont-elles indépendantes ?

Les variables aléatoires \(S\) et \(P\) sont-elles indépendantes ?
\end{exoex}

\begin{corr}
Les lois de \(X\) et \(Y\) sont les lois marginales de \(\paren{X,Y}\) : \[\quantifs{\forall k\in\interventierii{1}{6}}\proba{X=k}=\sum_{l=1}^6\proba{X=k\text{ et }Y=l}=\dfrac{6}{36}=\dfrac{1}{6}.\]

De même pour \(Y\), d'où \(X\sim Y\sim\loiuniforme{\interventierii{1}{6}}\).

On a : \[\quantifs{\forall k,l\in\interventierii{1}{6}}\proba{X=k\text{ et }Y=l}=\dfrac{1}{36}=\proba{X=k}\proba{Y=l}.\]

Donc \(X\) et \(Y\) sont indépendantes.

De plus, on remarque \(\proba{S=2\text{ et }P=36}=0\).

Or \(\begin{dcases}
\proba{S=2}=\proba{X=Y=1}=\dfrac{1}{36} \\
\proba{P=36}=\proba{X=Y=6}=\dfrac{1}{36}
\end{dcases}\)

Donc \(\proba{S=2\text{ et }P=36}\not=\proba{S=2}\proba{P=36}\).

Donc \(S\) et \(P\) sont indépendantes.
\end{corr}

\begin{prop}
Soient \(E\prim\) et \(F\prim\) deux ensembles et \(f:E\to E\prim\) et \(g:F\to F\prim\) deux fonctions.

On suppose que \(X:\Omega\to E\) et \(Y:\Omega\to F\) sont indépendantes.

Alors les variables aléatoires \(f\paren{X}\) et \(g\paren{Y}\) sont indépendantes.
\end{prop}

\begin{dem}
Soient \(x\prim\in E\prim\) et \(y\prim\in F\prim\).

On a \(\begin{dcases}
\accol{f\paren{X}=x\prim}=\accol{X\in f\inv\paren{\accol{x\prim}}} \\
\accol{g\paren{Y}=y\prim}=\accol{Y\in g\inv\paren{\accol{y\prim}}}
\end{dcases}\)

D'où : \[\begin{WithArrows}
\proba{f\paren{X}=x\prim\text{ et }g\paren{Y}=y\prim}&=\proba{X\in f\inv\paren{\accol{x\prim}}\text{ et }Y\in g\inv\paren{\accol{y\prim}}} \Arrow[tikz={text width=2cm}]{car \(X\) et \(Y\) sont indépendantes} \\
&=\proba{X\in f\inv\paren{\accol{x\prim}}}\proba{Y\in g\inv\paren{\accol{y\prim}}} \\
&=\proba{f\paren{X}=x\prim}\proba{g\paren{Y}=y\prim}.
\end{WithArrows}\]

Donc \(f\paren{X}\) et \(g\paren{Y}\) sont indépendantes.
\end{dem}

\begin{defi}
Soient \(n\in\Ns\) et \(X_1,\dots,X_n\) des variables aléatoires.

\begin{itemize}
    \item On dit que les variables aléatoires \(X_1,\dots,X_n\) sont deux à deux indépendantes si \(X_i\) et \(X_j\) sont indépendantes pour tous \(i,j\in\interventierii{1}{n}\) tels que \(i\not=j\) (\cf \thref{defi:variablesAléatoiresIndépendantes} et \thref{prop:caractérisationVariablesAléatoiresIndépendantes}). \\
    \item On dit que les variables aléatoires \(X_1,\dots,X_n\) sont mutuellement indépendantes si les conditions équivalentes suivantes sont satisfaites : \begin{enumerate}
        \item \(\quantifs{\text{Pour tout }\paren{x_1,\dots,x_n}}\proba{X_1=x_1\text{ et }\dots\text{ et }X_n=x_n}=\prod_{k=1}^n\proba{X_k=x_k}\) \\
        \item \(\quantifs{\text{Pour tout }\paren{A_1,\dots,A_n}}\proba{X_1\in A_1\text{ et }\dots\text{ et }X_n\in A_n}=\prod_{k=1}^n\proba{X_k\in A_k}\).
    \end{enumerate}
\end{itemize}

NB : si un énoncé ne précise pas \guillemets{mutuellement} ou \guillemets{deux à deux}, cela signifie \guillemets{mutuellement}.
\end{defi}

\subsection{Lemme des coalitions}

\begin{lem}[Lemme des coalitions I]
Soient \(n\in\Ns\), \(X_1,\dots,X_n\) des variables aléatoires mutuellement indépendantes, \(m\in\interventierii{1}{n-1}\) et deux fonctions \(f\) et \(g\) définies respectivement sur \(X_1\paren{\Omega}\times\dots\times X_m\paren{\Omega}\) et \(X_{m+1}\paren{\Omega}\times\dots\times X_n\paren{\Omega}\).

Les variables aléatoires \(f\paren{X_1,\dots,X_m}\) et \(g\paren{X_{m+1},\dots,X_n}\) sont indépendantes.
\end{lem}

\begin{dem}
Soient \(A\subset\Im f\) et \(B\subset\Im g\).

On a : \[\begin{WithArrows}
&\proba{f\paren{X_1,\dots,X_m}\in A\text{ et }g\paren{X_{m+1},\dots,X_n}\in B} \\
&=\proba{\paren{X_1,\dots,X_m}\in f\inv\paren{A}\text{ et }\paren{X_{m+1},\dots,X_n}\in g\inv\paren{B}} \\
&=\sum_{\paren{x_1,\dots,x_m}\in f\inv\paren{A}}\sum_{\paren{x_{m+1},\dots,x_n}\in g\inv\paren{B}}\proba{X_1=x_1\text{ et }\dots\text{ et }X_n=x_n} \Arrow[ll,tikz={text width=5cm}]{car \(X_1,\dots,X_n\) sont mutuellement indépendantes} \\
&=\sum_{\paren{x_1,\dots,x_m}\in f\inv\paren{A}}\sum_{\paren{x_{m+1},\dots,x_n}\in g\inv\paren{B}}\proba{X_1=x_1}\dots\proba{X_n=x_n} \\
&=\paren{\sum_{\paren{x_1,\dots,x_m}\in f\inv\paren{A}}\underbrace{\proba{X_1=x_1\text{ et }\dots\text{ et }X_m=x_m}}_{\substack{\text{car }X_1,\dots,X_m\text{ sont} \\ \text{mutuellement indépendantes}}}}\paren{\sum_{\paren{x_{m+1},\dots,x_n}\in g\inv\paren{B}}\underbrace{\proba{X_{m+1}=x_{m+1}\text{ et }\dots\text{ et }X_n=x_n}}_{\substack{\text{car }X_{m+1},\dots,X_n\text{ sont} \\ \text{mutuellement indépendantes}}}} \\
&=\proba{f\paren{X_1,\dots,X_m}\in A}\proba{g\paren{X_{m+1},\dots,X_n}\in B}.
\end{WithArrows}\]
\end{dem}

\begin{lem}[Lemme des coalitions II]
Soient \(n,r\in\Ns\), \(X_1,\dots,X_n\) des variables mutuellement indépendantes et \(I_1,\dots,I_r\) des parties de \(\interventierii{1}{n}\) supposées non-vides et deux à deux disjointes.

Pour tout \(k\in\interventierii{1}{r}\), on considère une fonction \(f_k\) définie sur \(\prod_{i\in I_k}X_i\paren{\Omega}\).

Alors les variables aléatoires \(f_1\paren{\paren{X_i}_{i\in I_1}},\dots,f_r\paren{\paren{X_i}_{i\in I_r}}\) sont mutuellement indépendantes.
\end{lem}

\begin{dem}
\note{Exercice}
\end{dem}

\subsection{Interprétation de la loi binomiale}

\begin{lem}\thlabel{lem:sommeLoisBinomialesEgaleLoiBinomialeSomme}
Soient \(p\in\intervii{0}{1}\), \(n,m\in\Ns\) et \(X\) et \(Y\) des variables aléatoires indépendantes telles que \[X\sim\loibinomiale{n}{p}\qquad\text{et}\qquad Y\sim\loibinomiale{m}{p}.\]

Alors on a \(X+Y\sim\loibinomiale{n+m}{p}\).
\end{lem}

\begin{dem}
\(X\) et \(Y\) sont à valeurs dans \(\interventierii{0}{n}\) et \(\interventierii{0}{m}\) respectivement.

On a : \(\begin{dcases}
\quantifs{\forall i\in\interventierii{0}{n}}\proba{X=i}=\binom{i}{n}p^i\paren{1-p}^{n-i} \\
\quantifs{\forall j\in\interventierii{0}{m}}\proba{Y=j}=\binom{j}{m}p^j\paren{1-p}^{m-j}
\end{dcases}\)

\(X+Y\) est à valeurs dans \(\interventierii{0}{n+m}\) et on a : \[\begin{WithArrows}
\quantifs{\forall k\in\interventierii{0}{n+m}}\proba{X+Y=k}&=\sum_{i=0}^n\proba{X=i\text{ et }Y=k-i} \Arrow[ll,tikz={text width=4cm}]{car \(X\) et \(Y\) sont indépendantes} \\
&=\sum_{i=0}^n\proba{X=i}\proba{Y=k-i} \\
&=\sum_{i=0}^n\binom{i}{n}p^i\paren{1-p}^{n-i}\binom{k-i}{m}p^{k-i}\paren{1-p}^{m-k+i} \\
&=p^k\paren{1-p}^{n-k}\sum_{i=0}^n\underbrace{\binom{i}{n}\binom{k-i}{m}}_{=\binom{k}{n+m}}.
\end{WithArrows}\]

Donc \(X+Y\sim\loibinomiale{n+m}{p}\).
\end{dem}

\begin{prop}
Soient \(p\in\intervii{0}{1}\), \(n\in\Ns\) et \(X_1,\dots,X_n\) des variables aléatoires mutuellement indépendantes qui suivent chacune une loi de Bernoulli \(\loibernoulli{p}\).

Alors \(X_1+\dots+X_n\sim\loibinomiale{n}{p}\).
\end{prop}

\begin{dem}
Découle du \thref{lem:sommeLoisBinomialesEgaleLoiBinomialeSomme} par récurrence car \(\loibernoulli{p}=\loibinomiale{1}{p}\).
\end{dem}

\section{Espérance, variance}

On considère toujours un espace probabilisé \(\groupe{\Omega}[\prem]\).

\subsection{Espérance}

\subsubsection{Définition}

\begin{defi}
Soit \(X:\Omega\to\R\) une variable aléatoire réelle.

On appelle espérance de \(X\) et on note \(\esp{X}\) la somme : \[\esp{X}=\sum_{x\in X\paren{\Omega}}\proba{X=x}x.\]
\end{defi}

\begin{ex}
Lancer de dé : \(X\sim\loiuniforme{\interventierii{1}{6}}\).

On a : \[\esp{X}=\sum_{k=1}^6\dfrac{k}{6}=\dfrac{7}{2}.\]
\end{ex}

\begin{ex}
Loi de Bernoulli : \(X\sim\loibernoulli{p}\).

On a : \[\esp{X}=0\times\proba{X=0}+1\times\proba{X=1}=p.\]
\end{ex}

\begin{ex}
Loi binomiale : \(X\sim\loibinomiale{n}{p}\).

On a : \(\esp{X}=\sum_{k=0}^n\binom{k}{n}p^k\paren{1-p}^{n-k}\).

Posons \(P\paren{T}=\sum_{k=0}^n\binom{k}{n}p^k\paren{1-p}^{n-k}T^k\in\poly[\R][T]\).

On a \(P\paren{T}=\paren{pT+1-p}^n\) selon la formule du binôme de Newton.

Donc : \[P\prim\paren{T}=\sum_{k=0}^nk\binom{k}{n}p^k\paren{1-p}^{n-k}T^{k-1}=np\paren{pT+1-p}^{n-1}.\]

D'où, en évaluant en \(1\) : \[P\prim\paren{1}=\esp{X}=np.\]
\end{ex}

\subsubsection{Propriétés}

\begin{prop}[Linéarité de l'espérance]
Soient \(X:\Omega\to\R\) et \(Y:\Omega\to\R\) deux variables aléatoires réelles et \(\lambda\in\R\).

On a : \[\esp{\lambda X}=\lambda\esp{X}\qquad\text{et}\qquad\esp{X+Y}=\esp{X}+\esp{Y}.\]

On parle de \guillemets{linéarité} car l'ensemble \(\F{\Omega}{\R}\) des variables aléatoires sur \(\Omega\) est un \(\R\)-espace vectoriel et \(\operatorname{E}\) est une forme linéaire sur cet espace vectoriel.
\end{prop}

\begin{dem}
Si \(\lambda\not=0\) : \[\begin{aligned}
\esp{\lambda X}&=\sum_{x\in X\paren{\Omega}}\proba{\lambda X=\lambda x}\lambda x \\
&=\sum_{x\in X\paren{\Omega}}\proba{X=x}\lambda x \\
&=\lambda\esp{X}.
\end{aligned}\]

Si \(\lambda=0\) : vrai aussi.

D'autre part, on a : \(\paren{X+Y}\paren{\Omega}\subset X\paren{\Omega}+Y\paren{\Omega}\).

Donc : \[\begin{aligned}
\esp{X+Y}&=\sum_{z\in X\paren{\Omega}+Y\paren{\Omega}}\proba{X+Y=z} \\
&=\sum_{z\in X\paren{\Omega}+Y\paren{\Omega}}\sum_{\substack{\paren{a,b}\in X\paren{\Omega}\times Y\paren{\Omega} \\ a+b=z}}\proba{X=a\text{ et }Y=b}\paren{a+b} \\
&=\sum_{a\in X\paren{\Omega}}\sum_{b\in Y\paren{\Omega}}\proba{X=a\text{ et }Y=b}\paren{a+b} \\
&=\sum_{a\in X\paren{\Omega}}a\underbrace{\sum_{b\in Y\paren{\Omega}}\proba{X=a\text{ et }Y=b}}_{\proba{X=a}}+\sum_{b\in Y\paren{\Omega}}b\underbrace{\sum_{a\in X\paren{\Omega}}\proba{X=a\text{ et }Y=b}}_{\proba{Y=b}} \\
&=\esp{X}+\esp{Y}.
\end{aligned}\]
\end{dem}

\begin{rem}
On retrouve que si \(X\sim\loibinomiale{n}{p}\) alors \(\esp{X}=np\) car \(X\sim X_1+\dots+X_n\) où \(X_1\sim\dots\sim X_n\sim\loibernoulli{p}\) donc \(\esp{X}=\esp{X_1}+\dots+\esp{X_n}=np\).
\end{rem}

\begin{ex}[Espérance de \(aX+b\)]
Soient \(a,b\in\R\) et \(X\) une variable aléatoire réelle.

On a : \[\esp{aX+b}=a\esp{X}+b.\]
\end{ex}

\begin{prop}
Soient \(X:\Omega\to\R\) et \(Y:\Omega\to\R\) deux variables aléatoires réelles.

Positivité : si \(X\geq0\) alors \(\esp{X}\geq0\).

Croissance : si \(X\leq Y\) alors \(\esp{X}\leq\esp{Y}\).

Inégalité triangulaire : on a \(\abs{\esp{X}}\leq\esp{\abs{X}}\).
\end{prop}

\begin{dem}
\note{Exercice}
\end{dem}

\begin{prop}[Espérance d'un produit]\thlabel{prop:espéranceProduit}
Soient \(X\) et \(Y\) deux variables aléatoires réelles indépendantes.

On a : \[\esp{XY}=\esp{X}\esp{Y}.\]
\end{prop}

\begin{dem}
On a \(\paren{XY}\paren{\Omega}\subset X\paren{\Omega}Y\paren{\Omega}=\accol{xy}_{\paren{x,y}\in X\paren{\Omega}\times Y\paren{\Omega}}\).

Donc : \[\begin{WithArrows}
\esp{XY}&=\sum_{z\in X\paren{\Omega}Y\paren{\Omega}}\proba{XY=z}z \\
&=\sum_{z\in X\paren{\Omega}Y\paren{\Omega}}\sum_{\substack{\paren{x,y}\in X\paren{\Omega}\times Y\paren{\Omega} \\ xy=z}}\proba{X=x\text{ et }Y=y}xy \Arrow[tikz={text width=3cm}]{car \(X\) et \(Y\) sont indépendantes} \\
&=\sum_{x\in X\paren{\Omega}}\sum_{y\in Y\paren{\Omega}}\proba{X=x}\proba{Y=y}xy \\
&=\esp{X}\esp{Y}.
\end{WithArrows}\]
\end{dem}

\begin{cor}[Espérance d'un produit]
Soient \(N\in\Ns\) et \(X_1,\dots,X_N\) des variables aléatoires réelles mutuellement indépendantes.

On a : \[\esp{X_1\dots X_N}=\esp{X_1}\dots\esp{X_N}.\]
\end{cor}

\begin{dem}
On raisonne par récurrence sur \(N\in\Ns\).

Le corollaire est clairement vrai si \(N=1\) et si \(N=2\) selon la \thref{prop:espéranceProduit}.

Soit \(N\in\Ns\). On suppose que le corollaire est vrai pour \(N\). Montrons qu'il est vrai pour \(N+1\).

Soient \(X_1,\dots,X_{N+1}\) des variables aléatoires réelles mutuellement indépendantes.

Selon le lemme des coalitions, les variables aléatoires \(X_1\) et \(X_2,\dots,X_{N+1}\) sont indépendantes.

Donc selon la \thref{prop:espéranceProduit} : \[\begin{WithArrows}
\esp{X_1\dots X_{N+1}}&=\esp{X_1}\esp{X_2\dots X_{N+1}} \Arrow[tikz={text width=4cm}]{selon l'hypothèse de récurrence} \\
&=\esp{X_1}\dots\esp{X_{N+1}}
\end{WithArrows}\]

D'où le corollaire pour \(N+1\).
\end{dem}

\begin{prop}[Formule de transfert]
Soient \(N\in\Ns\), \(X\) une variable aléatoire d'image finie \(\Im X=\accol{x_1,\dots,x_N}\) et \(f:\Im X\to\R\) une application.

Alors \(f\paren{X}\) admet pour espérance : \[\esp{f\paren{X}}=\sum_{n=1}^N\proba{X=x_n}f\paren{x_n}.\]
\end{prop}

\begin{dem}
\note{Exercice}
\end{dem}

\subsection{Variance, écart-type, covariance}

\subsubsection{Définitions}

\begin{defprop}
Soit \(X\) une variable aléatoire réelle.

La variance de \(X\) est le réel : \[\vari{X}=\esp{\paren{X-\esp{X}}^2}=\esp{X^2}-\esp{X}^2.\]

L'écart-type de \(X\) est le réel : \[\ecarttype{X}=\sqrt{\vari{X}}.\]
\end{defprop}

\begin{dem}
On a : \[\begin{aligned}
\esp{\paren{X-\esp{X}}^2}&=\esp{X^2-2X\esp{X}-\esp{X}^2} \\
&=\esp{X^2}-2\esp{X\esp{X}}+\esp{\esp{X}^2} \\
&=\esp{X^2}-2\esp{X}^2+\esp{X}^2 \\
&=\esp{X^2}-\esp{X}^2
\end{aligned}\]
\end{dem}

\begin{ex}
On s'intéresse au lancer d'un dé.

Soit \(X\) une variable aléatoire telle que \(X\sim\loiuniforme{\interventierii{1}{6}}\).

On a : \[\begin{WithArrows}
\vari{X}&=\esp{X^2}-\esp{X}^2 \\
&=\sum_{k=1}^6\proba{X=k}k^2-\paren{\dfrac{7}{2}}^2 \\
&=\dfrac{1}{6}\sum_{k=1}^6k^2-\dfrac{49}{4} \\
&=\dfrac{1}{6}\times\dfrac{6\times7\times13}{6}-\dfrac{49}{4} \Arrow{on calcule} \\
&=\dfrac{35}{12}.
\end{WithArrows}\]
\end{ex}

\begin{defprop}
Soient \(X\) et \(Y\) deux variables aléatoires réelles.

La covariance de \(X\) et \(Y\) est le réel : \[\cov{X}{Y}=\esp{\paren{X-\esp{X}}\paren{Y-\esp{Y}}}=\esp{XY}-\esp{X}\esp{Y}.\]
\end{defprop}

\begin{dem}
On a : \[\begin{aligned}
\esp{\paren{X-\esp{X}}\paren{Y-\esp{Y}}}&=\esp{XY-X\esp{Y}-\esp{X}Y+\esp{X}\esp{Y}} \\
&=\esp{XY}-\esp{Y}\esp{X}-\esp{X}\esp{Y}+\esp{X}\esp{Y} \\
&=\esp{XY}-\esp{X}\esp{Y}.
\end{aligned}\]
\end{dem}

\begin{rem}
Si \(X\) et \(Y\) sont indépendantes, alors \(\cov{X}{Y}=0\).
\end{rem}

\begin{dem}
Si \(X\) et \(Y\) sont indépendantes, alors \(\esp{XY}=\esp{X}\esp{Y}\) donc \(\cov{X}{Y}=0\).
\end{dem}

\begin{exoex}
On lance deux dés et on note \(X\) et \(Y\) les résultats obtenus.

On pose \(S=X+Y\) et \(P=XY\).

Calculer la covariance de \(S\) et \(P\).
\end{exoex}

\begin{corr}
On a : \[\begin{WithArrows}
\cov{S}{P}&=\esp{SP}-\esp{S}\esp{P} \\
&=\esp{\paren{X+Y}XY}-\esp{X+Y}\esp{XY} \Arrow[tikz={text width=4cm}]{car \(X\) et \(Y\) sont indépendantes} \\
&=\esp{X^2Y}+\esp{XY^2}-\esp{X+Y}\esp{X}\esp{Y} \Arrow{symétrie} \\
&=2\esp{X^2Y}-2\esp{X}^2\esp{Y} \Arrow[tikz={text width=4cm}]{car \(X^2\) et \(Y\) sont indépendantes} \\
&=2\esp{X^2}\esp{Y}-2\esp{X}^2\esp{Y} \\
&=2\vari{X}\esp{Y} \\
&=2\times\dfrac{35}{12}\times\dfrac{7}{2} \\
&=\dfrac{245}{12}.
\end{WithArrows}\]
\end{corr}

\subsubsection{Propriétés}

\begin{prop}
Soient \(X\) une variable aléatoire réelle et \(a,b\in\R\).

On a : \[\vari{aX+b}=a^2\vari{X}.\]
\end{prop}

\begin{dem}
On a : \[\begin{aligned}
\vari{aX+b}&=\esp{\paren{aX+b-\esp{aX+b}}^2} \\
&=\esp{\paren{aX+b-a\esp{X}-b}^2} \\
&=\esp{a^2\paren{X-\esp{X}}^2} \\
&=a^2\vari{X}.
\end{aligned}\]
\end{dem}

\begin{prop}
Soient \(n\in\Ns\) et \(X_1,\dots,X_n\) des variables aléatoires réelles.

On a : \[\vari{\sum_{i=1}^nX_i}=\sum_{i=1}^n\vari{X_i}+2\sum_{1\leq i<j\leq n}\cov{X_i}{X_j}.\]
\end{prop}

\begin{dem}
On a, en utilisant la définition de la variance et la linéarité de l'espérance : \[\begin{aligned}
\vari{\sum_{i=1}^nX_i}&=\esp{\paren{\sum_{i=1}^nX_i}^2}-\esp{\sum_{i=1}^nX_i}^2 \\
&=\esp{\sum_{i=1}^nX_i^2+2\sum_{1\leq i<j\leq n}X_iX_j}-\paren{\sum_{i=1}^n\esp{X_i}}^2 \\
&=\sum_{i=1}^n\esp{X_i^2}+2\sum_{1\leq i<j\leq n}\esp{X_iX_j}-\paren{\sum_{i=1}^n\esp{X_i}^2+2\sum_{1\leq i<j\leq n}\esp{X_i}\esp{X_j}} \\
&=\sum_{i=1}^n\paren{\esp{X_i^2}-\esp{X_i}^2}+2\sum_{1\leq i<j\leq n}\paren{\esp{X_iX_j}-\esp{X_i}\esp{X_j}} \\
&=\sum_{i=1}^n\vari{X_i}+2\sum_{1\leq i<j\leq n}\cov{X_i}{X_j}.
\end{aligned}\]
\end{dem}

\begin{cor}\thlabel{cor:varianceSommeEgaleSommeVariancesSiVariablesIndépendantes}
Soient \(n\in\Ns\) et \(X_1,\dots,X_n\) des variables aléatoires réelles deux à deux indépendantes.

On a : \[\vari{X_1+\dots+X_n}=\vari{X_1}+\dots+\vari{X_n}.\]
\end{cor}

\begin{dem}
Découle de la proposition précédente, car la covariance de deux variables aléatoires indépendantes est nulle.
\end{dem}

\begin{prop}
Soient \(n\in\Ns\), \(p\in\intervii{0}{1}\) et \(X\) une variable aléatoire.

\begin{enumerate}
    \item Si \(X\sim\loibernoulli{p}\) alors \(\vari{X}=p\paren{1-p}\). \\
    \item Si \(X\sim\loibinomiale{n}{p}\) alors \(\vari{X}=np\paren{1-p}\)
\end{enumerate}
\end{prop}

\begin{dem}[1]
On a : \[\begin{WithArrows}
\vari{X}&=\esp{X^2}-\esp{X}^2 \Arrow{car \(X^2\sim X\)} \\
&=p-p^2 \\
&=p\paren{1-p}.
\end{WithArrows}\]
\end{dem}

\begin{dem}[2]
Soient \(X_1,\dots,X_n\) des variables aléatoires mutuellement indépendantes suivant toutes une loi \(\loibernoulli{p}\).

On a \(X\sim X_1+\dots+X_n\).

D'après le \thref{cor:varianceSommeEgaleSommeVariancesSiVariablesIndépendantes}, on a donc : \[\begin{aligned}
\vari{X}&=\vari{X_1}+\dots+\vari{X_n} \\
&=np\paren{1-p}.
\end{aligned}\]
\end{dem}

\subsection{Inégalités}

\subsubsection{Inégalité de Markov}

\begin{prop}
Soient \(X\) une variable aléatoire réelle et \(a\in\Rps\).

On a : \[\proba{\abs{X}\geq a}\leq\dfrac{\esp{\abs{X}}}{a}.\]
\end{prop}

\begin{dem}
On pose : \[\fonction{Y}{\Omega}{\R}{\omega}{\begin{dcases}
a &\text{si }\abs{X\paren{\omega}}\geq a \\
0 &\text{sinon}
\end{dcases}}\] de sorte que \(Y\leq\abs{X}\) et \(\esp{Y}=a\proba{\abs{X}\geq a}\).

Par croissance de l'espérance, on a \(\esp{Y}\leq\esp{\abs{X}}\) et donc : \[a\proba{\abs{X}\geq a}\leq\esp{\abs{X}}.\]
\end{dem}

\subsubsection{Inégalité de Bienaymé-Tchebychev}

\begin{prop}
Soient \(X\) une variable aléatoire réelle et \(a\in\Rps\).

On a : \[\proba{\abs{X-\esp{X}}\geq a}\leq\dfrac{\vari{X}}{a^2}.\]
\end{prop}

\begin{dem}
On pose \(Y=\paren{X-\esp{X}}^2\).

Selon l'inégalité de Markov, on a \(\proba{\abs{Y}\geq a^2}\leq\dfrac{\esp{Y}}{a^2}\).

Donc \(\proba{\paren{X-\esp{X}}^2\geq a^2}\leq\dfrac{\esp{\paren{X-\esp{X}}^2}}{a^2}=\dfrac{\vari{X}}{a^2}\), \cad : \[\proba{\abs{X-\esp{X}}\geq a}\leq\dfrac{\vari{X}}{a^2}\] car \(a>0\).
\end{dem}

\begin{appl}
Soient \(n\in\Ns\) et \(X_1,\dots,X_n\) des variables aléatoires réelles deux à deux indépendantes et de même loi.

On pose : \(\begin{dcases}
\quantifs{\forall n\in\Ns}S_n=\sum_{k=1}^nX_k \\
m=\esp{X_1} \\
\sigma=\ecarttype{X_1}
\end{dcases}\)

Soit \(\epsilon\in\Rps\).

On a : \[\proba{\abs{\dfrac{S_n}{n}-m}\geq\epsilon}\leq\dfrac{\sigma^2}{n\epsilon^2}.\]
\end{appl}

\begin{dem}
Appliquons l'inégalité de Bienaymé-Tchebychev à \(\dfrac{S_n}{n}\).

On a : \[\begin{dcases}
\esp{\dfrac{S_n}{n}}=\dfrac{\esp{X_1}+\dots+\esp{X_n}}{n}=m \\
\vari{\dfrac{S_n}{n}}=\dfrac{1}{n^2}\paren{\vari{X_1}+\dots+\vari{X_n}}=\dfrac{\sigma^2}{n}
\end{dcases}\]

On obtient \(\proba{\abs{\dfrac{S_n}{n}-m}\geq\epsilon}\leq\dfrac{\frac{\sigma^2}{n}}{\epsilon^2}\).
\end{dem}

\subsection{Compléments}

\begin{defi}
Soit \(X\) une variable aléatoire réelle de variance non-nulle.

\begin{itemize}
    \item La variable aléatoire \(X-\esp{X}\) a une espérance nulle et est appelée variable centrée associée à \(X\). \\
    \item La variable aléatoire \(\dfrac{X}{\ecarttype{X}}\) a un écart-type égal à \(1\) et est appelée variable réduit associée à \(X\). \\
    \item La variable aléatoire \(\dfrac{X-\esp{X}}{\ecarttype{X}}\) a une espérance nulle et un écart-type égal à \(1\) et est appelée variable centrée réduite associée à \(X\).
\end{itemize}
\end{defi}

\part{TDs}

% Trick to make chapter numbering start from 0
\setcounter{chapter}{-1}

\chapter{Préliminaires}

\minitoc

\section{Logique}

\begin{exo}[Exercice 1]
Soient \(P\), \(Q\) et \(R\) des propositions.

\begin{enumerate}
\item Montrer que les propositions suivantes sont vraies à l'aide de tables de vérité :

\begin{enumerate}
\item \(\paren{P\imp Q}\ou P\) \\

\item \(\paren{P\imp Q}\ssi\paren{Q\ou\non P}\) \\

\item \(\croch{P\et\paren{P\imp Q}}\imp Q\) \\
\end{enumerate}

\item Déduire de (b) une nouvelle démonstration de l'équivalence \[\paren{P\imp Q}\ssi\paren{\non Q\imp\non P}.\]
\end{enumerate}
\end{exo}

\begin{corr}
\note{à venir}
\end{corr}

\section{Quantificateurs}

\begin{exo}[Exercice 2]
Les propositions suivantes sont-elles vraies ?

\begin{enumerate}
\item \(\quantifs{\forall x\in\Rps}\ln\paren{x}=0\) \\

\item \(\quantifs{\exists x\in\Rps}\ln\paren{x}=0\) \\

\item \(\quantifs{\exists x\in\R}\exp\paren{x}=0\) \\

\item \(\quantifs{\forall x,y\in\R}2xy\geq x^2+y^2\) (en justifiant) \\

\item \(\quantifs{\forall x\in\R;\exists y\in\R}x<y\) \\

\item \(\quantifs{\exists y\in\R;\forall x\in\R}x<y\)
\end{enumerate}
\end{exo}

\begin{corr}
\note{à venir}
\end{corr}

\begin{exo}[Exercice 3]
Soit \(f:\R\to\R\) une fonction et \(T\) un réel strictement positif.

Écrire à l'aide de quantificateurs les propositions suivantes :

\begin{enumerate}
\item La fonction \(f\) est paire \\

\item La fonction \(f\) est impaire \\

\item La fonction \(f\) est périodique, de période \(T\) \\

\item La fonction \(f\) est périodique
\end{enumerate}
\end{exo}

\begin{corr}
\note{à venir}
\end{corr}

\begin{exo}[Exercice 4]
Écrire à l'aide de quantificateurs les négations des propositions des exercices 2 et 3.
\end{exo}

\begin{corr}
\note{à venir}
\end{corr}

\begin{exo}[Exercice 5]
Soit \(f:\R\to\R\) une fonction.

Écrire à l'aide de quantificateurs les propositions suivantes :

\begin{enumerate}
\item La fonction \(f\) est constante \\

\item La fonction \(f\) est croissante \\

\item La fonction \(f\) est strictement décroissante
\end{enumerate}
\end{exo}

\begin{corr}
\note{à venir}
\end{corr}

\begin{exo}[Exercice 6]
Les propositions suivantes sont-elles vraies ?

\begin{enumerate}
\item \(\quantifs{\forall a,b\in\Rs}a\leq b\imp\dfrac{1}{a}\geq\dfrac{1}{b}\) \\

\item \(\quantifs{\forall a,b\in\Rps}a\leq b\imp\dfrac{1}{a}\geq\dfrac{1}{b}\) \\

\item \(\quantifs{\forall a,b\in\Rps}a<b\imp\dfrac{1}{a}>\dfrac{1}{b}\) \\

\item \(\quantifs{\forall a,b,c,d\in\R}\paren{a\leq b\quad\text{et}\quad c\leq d}\imp a+c\leq b+d\) \\

\item \(\quantifs{\forall a,b,c,d\in\R}\paren{a\leq b\quad\text{et}\quad c\leq d}\imp ac\leq bd\) \\

\item \(\quantifs{\forall a,b,c,d\in\Rp}\paren{a\leq b\quad\text{et}\quad c\leq d}\imp ac\leq bd\)
\end{enumerate}
\end{exo}

\begin{corr}
\note{à venir}
\end{corr}

\section{Raisonnement par analyse-synthèse}

\begin{exo}[Exercice 7]
Déterminer les suites \(\paren{u_n}_{n\in\N}\) de réels vérifiant : \[\quantifs{\forall m,n\in\N}u_{m+n}=u_m+u_n.\]
\end{exo}

\begin{corr}
\note{à venir}
\end{corr}

\section{Congruences}

\begin{exo}[Exercice 8]
Les propositions suivantes sont-elles vraies ?

\begin{enumerate}
\item \(\quantifs{\forall x,y\in\Z}x\equiv y\croch{5}\imp x^2\equiv y^2\croch{5}\) \\

\item \(\quantifs{\forall x,y\in\R}x\equiv y\croch{5}\imp x^2\equiv y^2\croch{5}\) \\

\item \(\quantifs{\exists x\in\Z}x^2\equiv-1\croch{5}\) \\

\item \(\quantifs{\exists x\in\Z}x^2\equiv-1\croch{7}\) \\

\item \(\quantifs{\exists x\in\R}x^2\equiv-1\croch{7}\)
\end{enumerate}
\end{exo}

\begin{corr}
\note{à venir}
\end{corr}

\begin{exo}[Exercice 9]
Soient \(x\) et \(y\) deux nombres réels.

Chercher quelles implications sont vraies entre les propositions suivantes :

\begin{enumerate}
\item \(x\equiv y\croch{\pi}\) \\

\item \(x\equiv y+\pi\croch{\pi}\) \\

\item \(x\equiv y\croch{2\pi}\) \\

\item \(x\equiv y+\pi\croch{2\pi}\) \\

\item \(2x\equiv 2y\croch{2\pi}\) \\

\item \(\dfrac{x}{2}\equiv\dfrac{y}{2}\croch{\dfrac{\pi}{2}}\) \\

\item \(\dfrac{x}{2}\equiv\dfrac{y}{2}+\dfrac{\pi}{2}\croch{\pi}\)
\end{enumerate}
\end{exo}

\begin{corr}
\note{à venir}
\end{corr}

\chapter{Inégalités, calculs}

\minitoc

\begin{exo}[Exercice 1]
\begin{enumerate}
\item Montrer : \[\quantifs{\forall x,y\in\Rps}\dfrac{xy}{x+y}\leq\dfrac{x+y}{4}.\] \\

\item En déduire : \[\quantifs{\forall x,y,z\in\Rps}\dfrac{xy}{x+y}+\dfrac{yz}{y+z}+\dfrac{xz}{x+z}\leq\dfrac{x+y+z}{2}.\]
\end{enumerate}
\end{exo}

\begin{corr}
\note{à venir}
\end{corr}

\begin{exo}[Exercice 2]
Résoudre les inéquations d'inconnue \(x\in\R\) :

\begin{enumerate}
\item \(\ln\paren{1+x}\leq1+\ln x\) \\

\item \(\sqrt{\ln\paren{1+x^2}}\geq2\) \\

\item \(\dfrac{1}{x}<-1\) \\

\item \(\dfrac{1}{x}\leq1\) \\

\item \(\sqrt{3-x}-\sqrt{x+1}>\dfrac{1}{2}\) \\

\item \(\floor{\sqrt{x^2+1}}\leq2\) \\

\item \(\floor{x^2-4x}=0\)
\end{enumerate}
\end{exo}

\begin{corr}
\note{à venir}
\end{corr}

\begin{exo}[Exercice 3]
Soit \(x\in\intervie{1}{\pinf}\).

Montrer : \[\dfrac{\paren{x-1}^2}{8x}\leq\dfrac{x+1}{2}-\sqrt{x}\leq\dfrac{\paren{x-1}^2}{8}.\]
\end{exo}

\begin{corr}
\note{à venir}
\end{corr}

\begin{exo}[Exercice 4]
Soit \(n\in\Ns\).

Calculer \[\sum_{k=1}^n\ln\paren{1+\dfrac{1}{k}}\quad\text{et}\quad\sum_{k=1}^{n}\ln\paren{1+\dfrac{2}{k}}.\]
\end{exo}

\begin{corr}
\note{à venir}
\end{corr}

\begin{exo}[Exercice 5]
Soit \(n\in\Ns\).

Calculer \[\sum_{k=1}^{n}\dfrac{1}{k\paren{k+1}}.\]
\end{exo}

\begin{corr}
\note{à venir}
\end{corr}

\begin{exo}[Exercice 6]
Soit \(n\in\N\).

Calculer \[\sum_{k=1}^{n}k\times k!\]
\end{exo}

\begin{corr}
\note{à venir}
\end{corr}

\begin{exo}[Exercice 7]
Soit \(n\in\N\).

Calculer \[\sum_{k=1}^{n}\dfrac{k}{k^4+k^2+1}.\]

\textit{Indication :} factoriser le dénominateur en remarquant \(k^4+k^2+1=k^4+2k^2+1-k^2\).
\end{exo}

\begin{corr}
\note{à venir}
\end{corr}

\begin{exo}[Exercice 8]
Soit \(n\in\N\).

Calculer \[\sum_{k=1}^{n}k^3.\]

Que remarque-t-on ?

\textit{Indication :} s'inspirer du calcul de \(\sum_{k=1}^{n}k^2\) vu en cours.
\end{exo}

\begin{corr}
\note{à venir}
\end{corr}

\begin{exo}[Exercice 9]
Montrer la proposition suivante : \[\quantifs{\forall n\in\N;\forall x\in\R\excluant\accol{1}}\prod_{k=0}^{n}\paren{x^{2^k}+1}=\dfrac{x^{2^{n+1}}-1}{x-1}.\]
\end{exo}

\begin{corr}
\note{à venir}
\end{corr}

\begin{exo}[Exercice 10]
Soit \(n\in\N\).

Calculer les sommes suivantes : \[S_1=\sum_{i=1}^{n}\sum_{j=1}^{n}2\qquad S_2=\sum_{i=1}^{n}\sum_{j=1}^{n}2^i\qquad S_3=\sum_{i=1}^{n}\sum_{j=1}^{n}2^j\qquad S_4=\sum_{i=1}^{n}\sum_{j=1}^{n}2^{i+j}.\]
\end{exo}

\begin{corr}
\note{à venir}
\end{corr}

\begin{exo}[Exercice 11]
Soit \(n\in\N\).

Calculer les sommes suivantes : \[S_1=\sum_{i=1}^{n}\sum_{j=1}^{n}\dfrac{1}{j}\qquad S_2=\sum_{i=1}^{n}\sum_{j=1}^{n}\dfrac{i+j}{j}\qquad S_3=\sum_{i=1}^{n}\sum_{j=1}^{n}\dfrac{\paren{i+j}^2}{j}.\]
\end{exo}

\begin{corr}
\note{à venir}
\end{corr}

\begin{exo}[Exercice 12]
Soit \(n\in\N\).

Calculer \[\sum_{i=1}^{n}\sum_{j=1}^{n}\min\accol{i;j}.\]
\end{exo}

\begin{corr}
\note{à venir}
\end{corr}

\begin{exo}[Exercice 13]
On définit la suite \(\paren{u_n}_{n\in\Ns}\) en posant : \[u_1=1\qquad u_2=u_3=2\qquad u_4=u_5=u_6=3\qquad u_7=u_8=u_9=u_{10}=4\qquad \dots\]

Combien vaut \(u_{30}\) ?

Exprimer \(u_n\) en fonction de \(n\in\Ns\) à l'aide de la fonction partie entière.
\end{exo}

\begin{corr}
\note{à venir}
\end{corr}

\begin{exo}[Exercice 14]
Soit \(n\in\N\).

Montrer : \[\floor{\paren{\sqrt{n}+\sqrt{n+1}}^2}=4n+1.\]
\end{exo}

\begin{corr}
\note{à venir}
\end{corr}

\begin{exo}[Exercice 15]
Soit \(n\in\Ns\).

Montrer que la partie entière de \(\paren{2+\sqrt{3}}^n\) est un entier impair.
\end{exo}

\begin{corr}
\note{à venir}
\end{corr}

\begin{exo}[Exercice 16, classique]
Soit \(n\in\Ns\).

Calculer \[\prod_{k=1}^{n}2k\qquad\text{puis}\qquad\prod_{k=1}^{n}\paren{2k-1}.\]

\textit{NB : on exprimera le résultat à l'aide de factorielles, sans symbole \(\prod\) ni points de suspension.}
\end{exo}

\begin{corr}
\note{à venir}
\end{corr}

\chapter{Révisions de trigonométrie}

\minitoc

\begin{exo}[Exercice 1]
Calculer \[\cos\dfrac{\pi}{12}\qquad\sin\dfrac{\pi}{12}\qquad\tan\dfrac{\pi}{12}\qquad\tan\dfrac{\pi}{8}\qquad\tan\dfrac{\pi}{16}\qquad\sin\dfrac{\pi}{8}\qquad\cos\dfrac{\pi}{8}.\]
\end{exo}

\begin{corr}
\note{à venir}
\end{corr}

\begin{exo}[Exercice 2]
Résoudre les équations suivantes, d'inconnue \(x\) :

\begin{enumerate}
\item \(\dfrac{2\tan x}{1-\tan^2x}=3\tan x\) \\

\item \(\sin x+\cos x=1\) \\

\item \(\sqrt{3}\sin x+\cos x=1\)
\end{enumerate}
\end{exo}

\begin{corr}
\note{à venir}
\end{corr}

\begin{exo}[Exercice 3]
Soit \(t\) un réel.

On note \(M\) (respectivement \(N\)) le point de coordonnées \(\paren{1,t}\) (respectivement \(\paren{-1,0}\)) et \(\classe{}\) le cercle unité.

\begin{enumerate}
\item Déterminer \(\classe{}\inter\paren{MN}\). \\

\item Retrouver des formules connues.
\end{enumerate}
\end{exo}

\begin{corr}
\note{à venir} % voir Priscilla sur Discord
\end{corr}

\begin{exo}[Exercice 4]
Soient \(n\) un entier relatif et \(x\) un réel tels que \[\cos x+\cos\paren{nx}+\cos\paren{\paren{2n-1}x}\not=0.\]

Simplifier l'expression \[\dfrac{\sin x+\sin\paren{nx}+\sin\paren{\paren{2n-1}x}}{\cos x+\cos\paren{nx}+\cos\paren{\paren{2n-1}x}}.\]
\end{exo}

\begin{corr}
\note{à venir}
\end{corr}

\begin{exo}[Exercice 5]
Montrer \[\quantifs{\forall n\in\N;\forall x\in\R}\abs{\sin\paren{nx}}\leq n\abs{\sin x}.\]
\end{exo}

\begin{corr}
\note{à venir}
\end{corr}

\begin{exo}[Exercice 6]
Montrer \[\quantifs{\forall x,y\in\R}\cos x^2+\cos y^2-\cos\paren{xy}<3.\]
\end{exo}

\begin{corr}
\note{à venir}
\end{corr}

\begin{exo}[Exercice 7]
Soient \(n\) un entier naturel et \(x\) un réel tels que \(x\not\equiv0\croch{\dfrac{\pi}{2^{n+1}}}\).

Montrer que la somme \(S=\sum_{k=0}^{n}2^k\tan\paren{2^kx}\) est bien définie et calculer \(S\).

\textit{Indication :} montrer que pour tout réel \(\theta\) tel que \(\theta\not\equiv0\croch{\dfrac{\pi}{2}}\), on a \[\tan\theta=\cotan\theta-2\cotan\paren{2\theta}\] en posant \(\cotan=\dfrac{\cos}{\sin}\).
\end{exo}

\begin{corr}
\note{à venir}
\end{corr}

\begin{exo}[Exercice 8]
Soient \(n\in\N\) et \(x\in\intervee{0}{\pi}\).

Calculer \[P=\prod_{k=1}^{n}\cos\dfrac{x}{2^k}.\]
\end{exo}

\begin{corr}
\note{à venir}
\end{corr}

\begin{exo}[Exercice 9]
Soient \(n\in\N\) et \(x\in\intervee{0}{\dfrac{\pi}{2^{n+1}}}\).

En utilisant l'exercice précédent, calculer \[Q=\prod_{k=0}^{n}\paren{1+\dfrac{1}{\cos\paren{2^kx}}}.\]
\end{exo}

\begin{corr}
\note{à venir}
\end{corr}

\begin{exo}[Exercice 10]
Soient \(x\) un réel et \(n\) un entier naturel.

Calculer \[S=\sum_{k=0}^{n}\dfrac{1}{\paren{-3}^k}\cos^3\paren{3^kx}.\]

\textit{Indication :} montrer \(\quantifs{\forall\theta\in\R}\cos^3\theta=\dfrac{3}{4}\cos\theta+\dfrac{1}{4}\cos\paren{3\theta}\).
\end{exo}

\begin{corr}
\note{à venir}
\end{corr}

\chapter{Nombres complexes}

\minitoc

\begin{exo}
Écrire les nombres complexes suivants sous forme algébrique et sous forme trigonométrique : \[1+\i\qquad1+j\qquad\dfrac{\i}{1-\i}\qquad\dfrac{\sqrt{3}+3\i}{1+\i}.\]
\end{exo}

\begin{corr}
\note{à venir}
\end{corr}

\begin{exo}
Calculer \[\i^{2022}\qquad j^{2022}\qquad\paren{1+j}^{2022}\qquad\paren{1+\i}^{2022}.\]
\end{exo}

\begin{corr}
\note{à venir}
\end{corr}

\begin{exo}~\\
Écrire \(z=\sqrt{2+\sqrt{2}}+\i\sqrt{2-\sqrt{2}}\) sous forme trigonométrique.

\textit{Indication :} déterminer \(\cos\paren{2\arg z}\).
\end{exo}

\begin{corr}
\note{à venir}
\end{corr}

\begin{exo}
Résoudre les équations suivantes, d'inconnue \(z\in\C\) :

\begin{enumerate}
\item \(z^2+2z+5=0\) \\

\item \(z^2+\paren{1+\i}z-\i=0\) \\

\item \(z^4+z^2+1=0\) \\

\item \(z^4-4\i z^2-4=0\)
\end{enumerate}
\end{exo}

\begin{corr}
\note{à venir}
\end{corr}

\begin{exo}
Résoudre le système suivant, d'inconnues \(x,y\in\C\) : \[\begin{dcases}x+y=1+\i \\ xy=2-\i\end{dcases}\]
\end{exo}

\begin{corr}
\note{à venir}
\end{corr}

\begin{exo}[CCP 2016]
On pose \[\omega=\exp\dfrac{2\i\pi}{7}\qquad S=\omega+\omega^2+\omega^4\qquad T=\omega^3+\omega^5+\omega^6.\]

\begin{enumerate}
\item Calculer \(S+T\) et \(ST\). \\

\item En déduire les valeurs de \(S\) et \(T\).
\end{enumerate}
\end{exo}

\begin{corr}
\note{à venir}
\end{corr}

\begin{exo}
Décrire l'ensemble \(\U_{10}\) des racines dixièmes de l'unité.

Quels éléments de \(\U_{10}\) sont racines carrées de l'unité ? racines cinquièmes ? racines septièmes ? racines quinzièmes ? racines vingtièmes ?
\end{exo}

\begin{corr}
\note{à venir}
\end{corr}

\begin{exo}
Soient \(x,y\in\C\).

Montrer les propositions suivantes :

\begin{enumerate}
\item \(\abs{x}+\abs{y}\leq\abs{x+y}+\abs{x-y}\) \\

\item \(\abs{x+y}^2+\abs{x-y}^2=2\abs{x}^2+2\abs{y}^2\qquad\text{\guillemets{identité du parallélogramme} dans }\C\)
\end{enumerate}
\end{exo}

\begin{corr}
\note{à venir}
\end{corr}

\begin{exo}
Soient \(A\), \(B\) et \(C\) des points du plan d'affixes respectives \(a\), \(b\) et \(c\).

\begin{enumerate}
\item Montrer que le triangle \(ABC\) est équilatéral direct si, et seulement si \[a+jb+j^2c=0.\] \textit{\small Si \(A=B=C\), on convient que le triangle \(ABC\) est équilatéral direct et indirect.} \\

\item Donner une CNS analogue pour que le triangle \(ABC\) soit équilatéral indirect. \\

\item Montrer que le triangle \(ABC\) est équilatéral si, et seulement si \[a^2+b^2+c^2=ab+bc+ca.\]
\end{enumerate}
\end{exo}

\begin{corr}
\note{à venir}
\end{corr}

\begin{exo}
Soit \(\paren{n,a,b}\in\N\times\R\times\R\).

Calculer \[\sum_{k=0}^{n}\sin\paren{a+kb}\qquad\text{et}\qquad\sum_{k=0}^{n}\cos\paren{a+kb}.\]
\end{exo}

\begin{corr}
\note{à venir}
\end{corr}

\begin{exo}
Soit \(\theta\in\R\).

Développer \[\sin\paren{5\theta}\qquad\cos\paren{5\theta}\qquad\tan\paren{5\theta}\qquad\sin\paren{7\theta}\qquad\cos\paren{7\theta}\qquad\tan\paren{7\theta}.\]
\end{exo}

\begin{corr}
\note{à venir}
\end{corr}

\begin{exo}
Soit \(\theta\in\R\).

Linéariser \[\cos^5\theta\qquad\sin^5\theta\qquad\cos^4\theta\sin^2\theta.\]
\end{exo}

\begin{corr}
\note{à venir}
\end{corr}

\begin{exo}
Soit \(n\in\Ns\).

Calculer \[S=\sum_{\omega\in\U_n}\omega\qquad\text{et}\qquad P=\prod_{\omega\in\U_n}\omega.\]
\end{exo}

\begin{corr}
\note{à venir}
\end{corr}

\begin{exo}~\\
Calculer \(\cos\dfrac{2\pi}{5}\).

Indication : on pourra utiliser la somme \(\sum_{\omega\in\U_5}\omega\).
\end{exo}

\begin{corr}
\note{à venir}
\end{corr}

\begin{exo}
Résoudre les équations suivantes, d'inconnue \(z\in\C\) :

\begin{enumerate}
\item \(\e{z}=0\) \\

\item \(\e{z}=1\) \\

\item \(\e{z}=\i\) \\

\item \(\e{z}=2j\)
\end{enumerate}
\end{exo}

\begin{corr}
\note{à venir}
\end{corr}

\begin{exo}
Déterminer les nombres complexes \(z\in\C\) tels que \[z+\conj{z}=\abs{z}.\]
\end{exo}

\begin{corr}
\note{à venir}
\end{corr}

\begin{exo}
Soit \(n\in\Ns\).

Résoudre l'équation d'inconnue \(z\in\C\) : \[\paren{z+1}^n=\paren{z-1}^n.\]
\end{exo}

\begin{corr}
\note{à venir}
\end{corr}

\begin{exo}
Soient \(a,b,c\in\R\) avec \(a\not=0\).

Montrer que les racines du polynôme \(P=aX^2+bX+c\) ont leurs parties réelles respectives strictement négatives si, et seulement si, on a : \[a,b,c\in\Rps\qquad\text{ou}\qquad a,b,c\in\Rms.\]
\end{exo}

\begin{corr}
\note{à venir}
\end{corr}

\begin{exo}
Soit \(z\in\C\).

Donner une CNS pour que les points \(z\), \(z^2\) et \(z^4\) soient alignés.
\end{exo}

\begin{corr}
\note{à venir}
\end{corr}

\begin{exo}
On pose \(\H=\accol{z\in\C\tq\Im z>0}\) et \(\D=\accol{z\in\C\tq\abs{z}<1}\).

Montrer que la fonction \(\fonction{f}{\H}{\D}{z}{\dfrac{z-\i}{z+\i}}\) est une bijection de \(\H\) vers \(\D\).

\textit{NB : On vérifiera que la fonction \(f\) est bien définie.}
\end{exo}

\begin{corr}
\note{à venir}
\end{corr}

\chapter{Notions ensemblistes}

\minitoc

\section{Ensembles}

\begin{exo}
Soient \(A\), \(B\) et \(C\) des ensembles.

Montrer les propositions suivantes :

\begin{enumerate}
\item \(\paren{A\union B}\inter C=\paren{A\inter C}\union\paren{B\inter C}\) \\

\item \(\paren{A\inter B}\union C=\paren{A\union C}\inter\paren{B\union C}\) \\

\item \(C\excluant\paren{A\union B}=\paren{C\excluant A}\inter\paren{C\excluant B}\) \\

\item \(\paren{A\union B}\excluant\paren{A\inter B}=\paren{A\excluant B}\union\paren{B\excluant A}\)
\end{enumerate}
\end{exo}

\begin{corr}
\note{à venir}
\end{corr}

\begin{exo}
Soient \(E\) et \(F\) deux ensembles.

\begin{enumerate}
\item Montrer l'équivalence \[E\subset F\ssi\P{E}\subset\P{F}\] \\

\item La proposition suivante est-elle vraie ? \[\P{E\union F}=\P{E}\union\P{F}\] \\

\item La proposition suivante est-elle vraie ? \[\P{E\inter F}=\P{E}\inter\P{F}\]
\end{enumerate}
\end{exo}

\begin{corr}
\note{à venir}
\end{corr}

\begin{exo}
On pose \(\classe{}=\accol{\paren{x,y}\in\R^2\tq x^2+y^2=1}\).

Existe-t-il deux parties \(A,B\subset\R\) telles que \(\classe{}=A\times B\) ?
\end{exo}

\begin{corr}
\note{à venir}
\end{corr}

\begin{exo}[Pas d'\guillemets{ensemble de tous les ensembles}]
Montrer qu'il n'existe pas d'ensemble dont les éléments sont les ensembles.

Pour cela, on pourra supposer par l'absurde qu'un tel ensemble \(E\) existe, considérer l'ensemble \[A=\accol{X\in E\tq X\not\in X}\] et aboutir à une contradiction.
\end{exo}

\begin{corr}
\note{à venir}
\end{corr}

\section{Fonctions}

\begin{exo}
Soient \(E\) et \(F\) deux ensembles et deux fonctions \(f:E\to F\) et \(g:F\to E\).

On suppose que la fonction \(f\rond g\rond f\) est une bijection de \(E\) dans \(F\).

Montrer que \(f\) et \(g\) sont des bijections.
\end{exo}

\begin{corr}
\note{à venir}
\end{corr}

\begin{exo}[Images directes, images réciproques]
Soient \(E\) et \(F\) deux ensembles, la fonction \(f:E\to F\), la partie \(A\subset E\) et la partie \(B\subset F\).

\begin{enumerate}
\item Montrer l'équivalence \[A\subset f\inv\paren{B}\ssi f\paren{A}\subset B.\] \\

\item Quelle inclusion est toujours vraie entre \(A\) et \(f\inv\paren{f\paren{A}}\) ? Donner un contre-exemple pour l'autre inclusion. \\

\item Même chose entre \(B\) et \(f\paren{f\inv\paren{B}}\).
\end{enumerate}
\end{exo}

\begin{corr}
\note{à venir}
\end{corr}

\begin{exo}
Soient \(a\) et \(b\) des réels.

Donner une CNS sur \(a\) et \(b\) pour que la fonction \[\fonction{f}{\R}{\R}{x}{ax+b}\] soit une bijection.

Déterminer alors sa bijection réciproque.
\end{exo}

\begin{corr}
\note{à venir}
\end{corr}

\begin{exo}
Montrer que la fonction \guillemets{sinus hyperbolique} : \[\fonction{\sh}{\R}{\R}{x}{\dfrac{\e{x}-\e{-x}}{2}}\] est bijective et déterminer sa bijection réciproque.
\end{exo}

\begin{corr}
\note{à venir}
\end{corr}

\begin{exo}
Soit \(E\) un ensemble non-vide et \(A\subset E\) une partie de \(E\).

Montrer que la fonction \[\fonction{f}{\P{E}}{\P{E}^2}{X}{\paren{X\inter A,X\union A}}\] est une injection.

Est-ce une surjection ?
\end{exo}

\begin{corr}
\note{à venir}
\end{corr}

\begin{exo}
Soient \(E\) et \(F\) deux ensembles et deux fonctions \(f:E\to F\) et \(g:F\to E\).

On suppose \[g\rond f=\id{E}\qquad\text{et}\qquad f\rond g=\id{F}.\]

Montrer que \(f\) et \(g\) sont des bijections, réciproques l'une de l'autre.
\end{exo}

\begin{corr}
\note{à venir}
\end{corr}

\begin{exo}[Fonctions indicatrices]
Soient \(E\) un ensemble et \(A,B,C\in\P{E}\).

\begin{enumerate}
\item Montrer que la fonction \[\fonction{\Phi}{\P{E}}{\accol{0;1}^E}{A}{\ind{A}}\] est une bijection et déterminer sa bijection réciproque. \\

\item Déterminer, en fonction de \(\ind{A}\) et \(\ind{B}\), les fonctions indicatrices des parties suivantes : \[A\inter B\qquad A\union B\qquad E\excluant A.\] \\

\item Utiliser ce qui précède pour montrer l'égalité : \[A\inter\paren{B\union C}=\paren{A\inter B}\union\paren{A\inter C}.\]
\end{enumerate}
\end{exo}

\begin{corr}
\note{à venir}
\end{corr}

\begin{exo}
Soient \(E\) et \(F\) deux ensembles non-vides.

Montrer que les deux propositions suivantes sont équivalentes :

\begin{enumerate}
\item Il existe une injection \(f:E\to F\). \\

\item Il existe une surjection \(g:F\to E\).
\end{enumerate}
\end{exo}

\begin{corr}
\note{à venir}
\end{corr}

\begin{exo}
Soit \(E\) un ensemble.

Montrer qu'il n'existe pas de surjection \(f:E\to\P{E}\).
\end{exo}

\begin{corr}
\note{à venir}
\end{corr}

\begin{exo}
Soient \(E\), \(E\prim\), \(F\) et \(F\prim\) des ensembles et \(f:E\to F\).

On suppose que l'ensemble \(E\prim\) est non-vide et que l'ensemble \(F\prim\) possède au moins deux éléments distincts.

On définit les fonctions : \[\fonction{F_1}{\F{E\prim}{E}}{\F{E\prim}{F}}{\phi}{f\rond\phi}\qquad\text{et}\qquad\fonction{F_2}{\F{F}{F\prim}}{\F{E}{F\prim}}{\phi}{\phi\rond f}\]

\begin{enumerate}
\item Donner une CNS sur \(f\) pour que \(F_1\) soit injective. \\

\item Donner une CNS sur \(f\) pour que \(F_1\) soit surjective. \\

\item Donner une CNS sur \(f\) pour que \(F_2\) soit injective. \\

\item Donner une CNS sur \(f\) pour que \(F_2\) soit surjective.
\end{enumerate}
\end{exo}

\begin{corr}
\note{à venir}
\end{corr}

\section{Ensembles ordonnés}

\begin{exo}[Ordre produit sur \(E\times F\)]
Soient \(\groupe{E}[\leq_E]\) et \(\groupe{F}[\leq_F]\) deux ensembles ordonnés.

On définit une relation binaire sur \(E\times F\) en posant : \[\quantifs{\forall\paren{x_1,y_1},\paren{x_2,y_2}\in E\times F}\paren{x_1,y_1}\leq\paren{x_2,y_2}\ssi\begin{dcases}x_1\leq_Ex_2 \\ y_1\leq_Fy_2\end{dcases}\]

Montrer que \(\leq\) est une relation d'ordre sur \(E\times F\).
\end{exo}

\begin{corr}
\note{à venir}
\end{corr}

\begin{exo}[Ordre lexicographique sur \(\R^2\)]
On définit une relation binaire \(\sqsubseteq\) sur \(\R^2\) en posant : \[\quantifs{\forall\paren{x_1,y_1},\paren{x_2,y_2}\in\R^2}\paren{x_1,y_1}\sqsubseteq\paren{x_2,y_2}\ssi\orenv{x_1<x_2 \\ x_1=x_2\quad\text{et}\quad y_1\leq y_2}\]

Montrer que \(\sqsubseteq\) est une relation d'ordre sur \(\R^2\) et que cet ordre est total.
\end{exo}

\begin{corr}
\note{à venir}
\end{corr}

\begin{exo}
Soient \(A,B\in\P{\R}\) deux parties de \(\R\) admettant chacune une borne supérieure.

\begin{enumerate}
\item Montrer que \(A\union B\) admet une borne supérieure, et déterminer cette borne supérieure en fonction de \(\sup A\) et \(\sup B\). \\

\item Montrer que la partie \[A+B=\accol{a+b}_{\paren{a,b}\in A\times B}\] admet une borne supérieure, et déterminer cette borne supérieure en fonction de \(\sup A\) et \(\sup B\).
\end{enumerate}
\end{exo}

\begin{corr}
\note{à venir}
\end{corr}

\begin{exo}
Soit \(X\) un ensemble.

On considère l'ensemble \(E=\P{X}\), ordonné par la relation d'inclusion \(\subset\).

\begin{enumerate}
\item Soient \(A,B\in E\). La partie \(\accol{A;B}\subset E\) admet-elle une borne supérieure dans \(E\) ? une borne inférieure dans \(E\) ? \\

\item Soit \(\paren{A_i}_{i\in I}\in E^I\) une famille d'éléments de \(E\). Cette famille admet-elle une borne supérieure dans \(E\) ? une borne inférieure dans \(E\) ? \\

\textit{NB : dans le cas présent, où tous les ensembles \(A_i\) sont des parties de \(E\), on convient que l'intersection \(\biginter_{i\in I}A_i\) vaut \(E\) si \(I\) est vide.}
\end{enumerate}
\end{exo}

\begin{corr}
\note{à venir}
\end{corr}

\begin{exo}
On considère la droite réelle achevée \(\Rb\), munie de sa relation d'ordre usuelle.

Montrer que toute partie de \(\Rb\) admet une borne supérieure et une borne inférieure dans \(\Rb\).
\end{exo}

\begin{corr}
\note{à venir}
\end{corr}

\section{Entiers naturels}

\begin{exo}[Descente infinie de Fermat]
Montrer qu'il n'existe pas de suite \(\paren{u_n}_{n\in\N}\in\N^\N\) d'entiers naturels strictement décroissante.
\end{exo}

\begin{corr}
\note{à venir}
\end{corr}

\begin{exo}
On considère la suite \(\paren{u_n}_{n\in\N}\in\R^\N\) définie par : \[\begin{dcases}u_0=1 \\ \quantifs{\forall n\in\N}u_{n+1}=\sum_{k=0}^{n}u_k\end{dcases}\]

Montrer la proposition suivante : \[\quantifs{\forall n\in\Ns}u_n=2^{n-1}.\]
\end{exo}

\begin{corr}
\note{à venir}
\end{corr}

\chapter{Suites}

\begin{exo}[Exercice 1, exemples fondamentaux]
Soit \(\lambda\in\R\).

Étudier la limite des suites suivantes en fonction de \(\lambda\) : \[\paren{n^\lambda}_{n\in\Ns}\qquad\paren{\ln^\lambda n}_{n\in\interventierie{2}{\pinf}}\qquad\paren{\lambda^n}_{n\in\N}\qquad\paren{n!}_{n\in\N}\]
\end{exo}

\begin{corr}
\note{à venir}
\end{corr}

\begin{exo}[Exercice 2]
Étudier la convergence (et déterminer la limite éventuelle) des suites de terme général :

\begin{enumerate}
\item \(u_n=\dfrac{n!+\paren{-1}^n\cos n}{n!+\cos n}\) \\

\item \(v_n=\dfrac{\paren{-1}^nn!+\cos n}{n!+\cos n}\) \\

\item \(w_n=\dfrac{\paren{-1}^nn!+\cos n}{\paren{n+1}!+\cos\paren{n+1}}\) \\

\item \(x_n=\int_{-1}^1t^n\sin t\odif{t}\)
\end{enumerate}
\end{exo}

\begin{corr}
\note{à venir}
\end{corr}

\begin{exo}[Exercice 3, moyennes]
On définit, pour tous réels strictement positifs \(a\) et \(b\) :

\begin{itemize}
\item leur moyenne arithmétique : \(m_a\paren{a,b}=\dfrac{a+b}{2}\) ; \\

\item leur moyenne géométrique : \(m_g\paren{a,b}=\sqrt{ab}\) ; \\

\item leur moyenne harmonique : \(m_h\paren{a,b}=\dfrac{2ab}{a+b}\).
\end{itemize}

\begin{enumerate}[series=exmoyennes]
\item Soient \(a,b\in\Rps\). Montrer qu'on a \[m_g\paren{a\inv,b\inv}\inv=m_g\paren{a,b}\qquad\text{et}\qquad m_a\paren{a\inv,b\inv}\inv=m_h\paren{a,b}\] et \[m_h\paren{a,b}\leq m_g\paren{a,b}\leq m_a\paren{a,b}\qquad\text{et}\qquad m_g\paren{m_h\paren{a,b},m_a\paren{a,b}}=m_g\paren{a,b}.\]
\end{enumerate}

On se fixe pour la suite deux réels \(x\) et \(y\) tels que \(0<x<y\).

\begin{enumerate}[resume=exmoyennes]
\item On pose \[\begin{dcases}x_0=x \\ y_0=y\end{dcases}\qquad\text{et}\qquad\quantifs{\forall n\in\N}\begin{dcases}x_{n+1}=m_g\paren{x_n,y_n} \\ y_{n+1}=m_a\paren{x_n,y_n}\end{dcases}\] \\

Montrer que \(\paren{x_n}_{n\in\N}\) et \(\paren{y_n}_{n\in\N}\) sont convergentes et de même limite. \\

\item On pose \[\begin{dcases}u_0=x \\ v_0=y\end{dcases}\qquad\text{et}\qquad\quantifs{\forall n\in\N}\begin{dcases}u_{n+1}=m_h\paren{u_n,v_n} \\ v_{n+1}=m_a\paren{u_n,v_n}\end{dcases}\] \\

Montrer que \(\paren{u_n}_{n\in\N}\) et \(\paren{v_n}_{n\in\N}\) sont convergentes et de même limite. Quelle est cette limite ?
\end{enumerate}
\end{exo}

\begin{corr}
\note{à venir}
\end{corr}

\begin{exo}[Exercice 4, série harmonique]
On pose \[\quantifs{\forall n\in\N}H_n=\sum_{k=1}^{n}\dfrac{1}{k}.\]

\begin{enumerate}
\item Montrer que la suite \(\paren{H_n}_{n\in\N}\) admet une limite. \\

\item Montrer : \[\quantifs{\forall n\in\N}H_{2n}-H_n\geq\dfrac{1}{2}.\] \\

\item En déduire la limite de la suite \(\paren{H_n}_{n\in\N}\).
\end{enumerate}
\end{exo}

\begin{corr}
\note{à venir}
\end{corr}

\begin{exo}[Exercice 5]
Soit \(\paren{u_n}_{n\in\N}\) une suite réelle.

Quelles implications sont vraies entre les propositions suivantes ?

\begin{enumerate}
\item La suite \(\paren{u_n}_{n\in\N}\) est convergente. \\

\item \(\lim_{n\to\pinf}u_{n+1}-u_n=0\)
\end{enumerate}
\end{exo}

\begin{corr}
\note{à venir}
\end{corr}

\begin{exo}[Exercice 6]
Soit \(\paren{z_n}_{n\in\N}\) une suite complexe telle que \[\quantifs{\forall n\in\N}z_{n+1}=\dfrac{z_n+\abs{z_n}}{2}.\]

Montrer que \(\paren{z_n}_{n\in\N}\) est convergente et que sa limite est réelle.
\end{exo}

\begin{corr}
\note{à venir}
\end{corr}

\begin{exo}[Exercice 7]
Soit \(\paren{u_n}_{n\in\N}\) une suite réelle.

\begin{enumerate}
\item Montrer que si \(\paren{u_n}_n\) est décroissante et si on a \(\quantifs{\forall n\in\N}u_n\in\N\) alors \(\paren{u_n}_n\) est stationnaire. \\

\item Montrer que si \(\paren{u_n}_n\) est convergente et si on a \(\quantifs{\forall n\in\N}u_n\in\Z\) alors \(\paren{u_n}_n\) est stationnaire.
\end{enumerate}
\end{exo}

\begin{corr}
\note{à venir}
\end{corr}

\begin{exo}[Exercice 8]\thlabel{exo:5.15}
Soit \(A\subset\R\).

\begin{enumerate}
\item Montrer que si \(A\) n'est pas majorée, alors il existe une suite d'élément de \(A\) qui tend vers \(\pinf\). \\

\item Montrer que si \(A\) est majorée, alors il existe une suite d'éléments de \(A\) qui tend vers \(\sup A\).
\end{enumerate}
\end{exo}

\begin{corr}
\note{à venir}
\end{corr}

\begin{exo}[Exercice 9]
Soit \(\paren{u_n}_{n\in\N}\) une suite réelle.

Montrer que \(\paren{u_n}_n\) ne tend pas vers \(\pinf\) si, et seulement si, elle admet une suite extraite majorée.
\end{exo}

\begin{corr}
\note{à venir}
\end{corr}

\begin{exo}[Exercice 10]
Soient \(\paren{u_n}_{n\in\N}\in\R^\N\) et \(l\in\Rb\).

Montrer que \(\paren{u_n}_n\) tend vers \(l\) si, et seulement si, de toute suite extraite de \(\paren{u_n}_n\) on peut extraire une suite qui tend vers \(l\).
\end{exo}

\begin{corr}
\note{à venir}
\end{corr}

\begin{exo}[Exercice 11, théorème des segments emboîtés]
Soient \(\paren{a_n}_{n\in\N},\paren{b_n}_{n\in\N}\in\R^\N\) deux suites réelles telles que \[\quantifs{\forall n\in\N}a_n\leq b_n.\]

On suppose que la suite des segments \(\paren{\intervii{a_n}{b_n}}_{n\in\N}\) est décroissante pour l'inclusion, \cad : \[\quantifs{\forall n\in\N}\intervii{a_{n+1}}{b_{n+1}}\subset\intervii{a_n}{b_n}\] et que la suite des longueurs des segments tend vers \(0\), \cad : \[\lim_{n\to\pinf}b_n-a_n=0.\]

Montrer que \(\biginter_{n\in\N}\intervii{a_n}{b_n}\) est un singleton.
\end{exo}

\begin{corr}
\note{à venir}
\end{corr}

\begin{exo}[Exercice 12]
\begin{enumerate}
\item Soit \(A\) une partie de \(\R\) telle que \[\quantifs{\forall x\in\R;\exists a,b\in A}a<x<b\qquad\text{et}\qquad\quantifs{\forall a,b\in A}\dfrac{a+b}{2}\in A.\] \\

Montrer que \(A\) est dense dans \(\R\). \\

\item Retrouver le fait que \(\Q\) est dense dans \(\R\).
\end{enumerate}
\end{exo}

\begin{corr}
\note{à venir}
\end{corr}

\begin{exo}[Exercice 13, moyenne de Cesàro d'une suite]\thlabel{exo:moyenneDeCesàroD'UneSuite}
Soient \(\paren{u_n}_{n\in\N}\) et \(\paren{v_n}_{n\in\N}\) deux suites réelles et \(l\) un réel.

On considère la suite \(\paren{U_n}_{n\in\N}\) définie par : \[\quantifs{\forall n\in\N}U_n=\dfrac{1}{n+1}\sum_{k=0}^{n}u_k.\]

\begin{enumerate}
\item Montrer que si \(\paren{u_n}_n\) diverge vers \(\pinf\), alors \(\paren{U_n}_n\) diverge vers \(\pinf\). \\

\item Montrer que si \(\paren{u_n}_n\) converge vers \(l\), alors \(\paren{U_n}_n\) converge vers \(l\). \\

\item Montrer que les implications réciproques des propositions ci-dessus ne sont pas toujours vraies. \\

\item Montrer que si la suite \(\paren{v_{n+1}-v_n}_n\) converge vers \(l\), alors \(\lim_{n\to\pinf}\dfrac{v_n}{n}=l\). \\

\item On suppose qu'on a : \[\quantifs{\forall n\in\N}v_n\in\Rps\qquad\text{et}\qquad\lim_{n\to\pinf}\dfrac{v_{n+1}}{v_n}=\pinf.\] \\

Montrer : \[\lim_{n\to\pinf}\sqrt[n]{v_n}=\pinf.\] \\

\item On suppose qu'on a : \[\quantifs{\forall n\in\N}v_n\in\Rps\qquad\text{et}\qquad\lim_{n\to\pinf}\dfrac{v_{n+1}}{v_n}=l.\] \\

Montrer : \[\lim_{n\to\pinf}\sqrt[n]{v_n}=l.\] \\

\item Calculer \(\lim_{n\to\pinf}\sqrt[n]{n!}\). \\

\item Calculer \(\lim_{n\to\pinf}\sqrt[n]{n^2}\).
\end{enumerate}
\end{exo}

\begin{corr}
\note{à venir}
\end{corr}

\begin{exo}[Exercice 14]
\begin{enumerate}[series=ex14suites]
\item Montrer que pour tout entier \(n\in\Ns\), l'équation d'inconnue \(x\in\Rps\) : \[x+\ln x=n\] admet une unique solution.
\end{enumerate}

Dans la suite, on note \(u_n\) cette unique solution et on s'intéresse à la suite \(\paren{u_n}_{n\in\Ns}\).

\begin{enumerate}[resume=ex14suites]
\item Montrer que \(\paren{u_n}_{n\in\Ns}\) est croissante. \\

\item Déterminer sa limite.
\end{enumerate}
\end{exo}

\begin{corr}
\note{à venir}
\end{corr}

\chapter{Algèbre générale}

\minitoc

\section{Lois de composition internes}

\begin{exo}
On considère la loi de composition interne \(*\) sur \(\R\) définie par : \[\quantifs{\forall x,y\in\R}x*y=\ln\paren{\e{x}+\e{y}}.\]

Est-elle associative ? commutative ? Possède-t-elle un élément neutre ?
\end{exo}

\begin{corr}
\note{à venir}
\end{corr}

\begin{exo}
Soit \(E\) un ensemble muni d'une loi \(*\).

On suppose que la loi \(*\) est associative et qu'elle possède un élément neutre \(e\).

Un élément \(x\in E\) est dit idempotent s'il vérifie \(x*x=x\).

\begin{enumerate}
\item Soient \(x,y\in E\). On suppose que \(x\) et \(y\) sont idempotents et qu'ils commutent. Montrer que \(x*y\) est idempotent. \\

\item Soit \(x\in E\) un élément inversible et idempotent. Montrer que son inverse \(x\inv\) est idempotent.
\end{enumerate}
\end{exo}

\begin{corr}
\note{à venir}
\end{corr}

\begin{exo}[Plus difficile]
Soient \(E\) un ensemble fini et \(*\) une loi de composition interne associative sur \(E\).

Soit \(x\in E\). On suppose que \(x\) est régulier pour la loi \(*\).

\begin{enumerate}
\item Montrer que toute puissance de \(x\) est régulière : \[\quantifs{\forall n\in\Ns}x^n\text{ est régulier}.\] \\

\item Montrer que la loi \(*\) admet un élément neutre. \\

\item Montrer que \(x\) est inversible.
\end{enumerate}
\end{exo}

\begin{corr}
\note{à venir}
\end{corr}

\section{Groupes}

\begin{exo}[Exemple important]
Soit \(E\) un ensemble.

On appelle permutation de \(E\) toute bijection \(f:E\to E\).

On note \(S_E\) l'ensemble des permutations de \(E\).

Montrer que \(\groupe{S_E}[\rond]\) est un groupe. Est-il commutatif ?
\end{exo}

\begin{corr}
\note{à venir}
\end{corr}

\begin{exo}
Montrer que l'ensemble des applications de la forme \[\fonction{f_{ab}}{\C}{\C}{z}{az+b}\] avec \(\paren{a,b}\in\Cs\times\C\) est un groupe pour la composition.
\end{exo}

\begin{corr}
\note{à venir}
\end{corr}

\begin{exo}
On pose \(E=\accol{0;1;2}\).

\begin{enumerate}
\item Combien y a-t-il de lois de composition internes sur \(E\) ? \\

\item Combien y a-t-il de lois de composition internes commutatives sur \(E\) ? \\

\item Combien y a-t-il de lois de composition internes admettant \(0\) comme élément neutre sur \(E\) ? \\

\item Combien y a-t-il de lois de composition internes \guillemets{unitaires} (\cad admettant un élément neutre) sur \(E\) ? \\

\item Combien y a-t-il de structures de groupe pour lesquelles \(0\) est l'élément neutre sur \(E\) ? \\

\item Combien y a-t-il de structures de groupe sur \(E\) ? \\

\item Ces dernières sont-elles commutatives ? \\

\item On munit \(E\) de sa structure de groupe telle que \(0\) soit l'élément neutre. Quels sont les sous-groupes de \(E\) ? Déterminer \(\Aut{E}\).
\end{enumerate}
\end{exo}

\begin{corr}
\note{à venir}
\end{corr}

\begin{exo}
On note \(\U\) l'ensemble des nombres complexes de module \(1\).

Rappeler les structures naturelles de groupe de \(\Cs\) et \(\U\) et montrer que la fonction \[\fonction{f}{\Cs}{\U}{z}{\dfrac{z}{\abs{z}}}\] est un morphisme de groupes.
\end{exo}

\begin{corr}
\note{à venir}
\end{corr}

\begin{exo}[Intersection de sous-groupes]\thlabel{exo:6.15}
Soient \(G\) un groupe et \(\paren{H_i}_{i\in I}\) une famille de sous-groupes de \(G\).

Montrer que l'intersection \(\biginter_{i\in I}H_i\) est un sous-groupe de \(G\).
\end{exo}

\begin{corr}
\note{à venir}
\end{corr}

\begin{exo}
Soit \(\groupe{G}[*]\) un groupe d'élément neutre \(e\) tel que \[\quantifs{\forall g\in G}g*g=e.\]

Montrer que \(G\) est abélien.
\end{exo}

\begin{corr}
\note{à venir}
\end{corr}

\begin{exo}
Soit \(\groupe{G}[*]\) un groupe de cardinal pair et d'élément neutre \(e\).

\begin{enumerate}
\item Vérifier que la relation \(\rel\) sur \(G\) définie par \[\quantifs{\forall g,h\in G}g\rel h\ssi\orenv{g=h \\ g=h\inv}\] est une relation d'équivalence. \\

\item En déduire que \(G\) possède un élément \(g\) tel que \(g\not=e\) et \(g*g=e\).
\end{enumerate}
\end{exo}

\begin{corr}
\note{à venir}
\end{corr}

\begin{exo}
Soit \(G\) un groupe et \(g\) un élément de \(G\).

\begin{enumerate}
\item Montrer qu'il existe un unique morphisme de groupes \(\phi:\Z\to G\) tel que \[\phi\paren{1}=g.\] \\

Que vient-on de montrer à propos de l'application \[\fonction{f}{\Hom{\Z}{G}}{G}{\phi}{\phi\paren{1}}\text{ ?}\] \\

\item Montrer par des exemples que l'image d'un morphisme de groupes \(\phi:\Z\to G\) peut être finie ou infinie.
\end{enumerate}
\end{exo}

\begin{corr}
\note{à venir}
\end{corr}

\section{Anneaux}

\begin{exo}
Quels sont les sous-anneaux de \(\Z\) ?
\end{exo}

\begin{corr}
\note{à venir}
\end{corr}

\begin{exo}
Montrer que toute structure d'anneau sur un ensemble à trois éléments est une structure d'anneau commutatif.
\end{exo}

\begin{corr}
\note{à venir}
\end{corr}

\begin{exo}
Montrer que l'ensemble des suites réelles convergentes (indicées par \(\N\)) est un sous-anneau de \(\R^\N\).

Que dire des suites complexes convergentes ?
\end{exo}

\begin{corr}
\note{à venir}
\end{corr}

\begin{exo}
\begin{enumerate}
\item Les anneaux \(\anneau{\Z}\) et \(\anneau{\Q}\) sont-ils isomorphes ? \\

\item Les anneaux \(\anneau{\R}\) et \(\anneau{\Q}\) sont-ils isomorphes ? \\

\item Les anneaux \(\anneau{\R}\) et \(\anneau{\C}\) sont-ils isomorphes ?
\end{enumerate}
\end{exo}

\begin{corr}
\note{à venir}
\end{corr}

\begin{exo}
On pose : \[\Z\croch{\sqrt{2}}=\accol{x+\sqrt{2}y}_{\paren{x,y}\in\Z^2}=\accol{a\in\R\tq\quantifs{\exists x,y\in\Z}a=x+\sqrt{2}y}.\]

\begin{enumerate}
\item Montrer que \(\Z\croch{\sqrt{2}}\) est un anneau commutatif. \\

\item On pose : \[\quantifs{\forall x,y\in\Z}N\paren{x+\sqrt{2}y}=x^2-2y^2.\] \\

Montrer que cela définit une application \(N:\Z\croch{\sqrt{2}}\to\Z\). \\

\item Montrer que si \(a\) et \(b\) sont des éléments de \(\Z\croch{\sqrt{2}}\) alors \[N\paren{ab}=N\paren{a}N\paren{b}.\] \\

\item Montrer que \[\quantifs{\forall a\in\Z\croch{\sqrt{2}}}a=0\ssi N\paren{a}=0.\] \\

\item Montrer que \[\quantifs{\forall a\in\Z\croch{\sqrt{2}}}a\in\Z\croch{\sqrt{2}}\croix\ssi N\paren{a}\in\Z\croix.\] \\

\item On pose \[\begin{dcases}u_0=1 \\ v_0=0\end{dcases}\qquad\text{et}\qquad\begin{dcases}u_{n+1}=u_n+2v_n \\ v_{n+1}=u_n+v_n\end{dcases}\] \\

Montrer que pour tout entier naturel \(n\), l'élément \(u_n+\sqrt{2}v_n\) est inversible dans \(\Z\croch{\sqrt{2}}\).
\end{enumerate}
\end{exo}

\begin{corr}
\note{à venir}
\end{corr}

\begin{exo}
Soient \(\anneau{A}\) un anneau intègre et \(a,b\in A\).

On dit que \(a\) divise \(b\) et on note \(a\divise b\) si on a : \[\quantifs{\exists c\in A}ac=b.\]

Montrer : \[\paren{\quantifs{\exists\lambda\in A\croix}a=\lambda b}\ssi\begin{dcases}a\divise b \\ b\divise a\end{dcases}\]
\end{exo}

\begin{corr}
\note{à venir}
\end{corr}

\begin{exo}
On note \(\D\) l'ensemble des nombres décimaux : \[\D=\accol{x\in\R\tq\quantifs{\exists\alpha\in\N;\exists\lambda\in\Z}x=\dfrac{\lambda}{10^\alpha}}.\]

\begin{enumerate}
\item Montrer que \(\D\) est naturellement muni d'une structure d'anneau. \\

\item Quels sont ses éléments inversibles ?
\end{enumerate}
\end{exo}

\begin{corr}
\note{à venir}
\end{corr}

\begin{exo}
Soit \(\anneau{A}\) un anneau. On note \(0\) et \(1\) ses éléments neutres.

Un élément \(a\in A\) est dit nilpotent s'il vérifie : \[\quantifs{\exists n\in\Ns}a^n=0.\]

\begin{enumerate}
\item Si \(A\) est intègre, quels sont ses éléments nilpotents ? \\

\item Soient \(a,b\in A\). On suppose que \(a\) et \(b\) sont nilpotents et qu'ils commutent. \\

Montrer que \(a+b\) et \(ab\) sont nilpotents. \\

\item Soit \(a\in A\) un élément nilpotent. \\

Montrer que \(1+a\) est inversible et déterminer son inverse.
\end{enumerate}
\end{exo}

\begin{corr}
\note{à venir}
\end{corr}

\section{Corps}

\begin{exo}
Quels sont les sous-corps de \(\Q\) ?
\end{exo}

\begin{corr}
\note{à venir}
\end{corr}

\begin{exo}
On pose : \[\Q\croch{\sqrt{2}}=\accol{x+\sqrt{2}y}_{\paren{x,y}\in\Q^2}=\accol{a\in\R\tq\quantifs{\exists x,y\in\Q}a=x+\sqrt{2}y}.\]

Montrer que \(\Q\croch{\sqrt{2}}\) est un corps. Quels sont ses sous-corps ?
\end{exo}

\begin{corr}
\note{à venir}
\end{corr}

\begin{exo}
Soient \(K\) un corps et \(A\) un anneau non-nul.

Montrer que tout morphisme d'anneaux \(\phi:F\to A\) est injectif.
\end{exo}

\begin{corr}
\note{à venir}
\end{corr}

\begin{exo}
Montrer que tout anneau intègre fini est un corps.
\end{exo}

\begin{corr}
\note{À venir}
\end{corr}

\chapter{Limites de fonctions, continuité}

\minitoc

\section{Limites \& continuité}

\begin{exo}[Exercice 1]
On admet \(\lim_{t\to0^+}t\ln t=0\).

Déterminer, lorsqu'elle existent, les limites suivantes :

\begin{enumerate}
\item \(\lim_{x\to\pinf}x^2-2x+x\ln x\) \\

\item \(\lim_{x\to a}\dfrac{x^2+x}{x^6+2x^5-2x^3-x^2}\) avec \(a\) valant \(\pinf\), puis \(\minf\), puis \(0\), puis \(1\), puis \(-1\). \\

On commencera par déterminer l'ensemble de définition du quotient. \\

\item \(\lim_{x\to0^+}x^x\) \\

\item \(\lim_{x\to\pinf}x^{\nicefrac{-1}{\ln x}}\) \\

\item \(\lim_{x\to\pinf}\sqrt{x+1}-\sqrt{x-1}\) \\

\item \(\lim_{x\to1^+}\ln x\times\ln\paren{\ln x}\)
\end{enumerate}
\end{exo}

\begin{corr}
\note{À venir}
\end{corr}

\begin{exo}[Exercice 2]
\begin{enumerate}
\item Quel est l'ensemble de définition de la fonction \(f:x\mapsto\dfrac{x\ln x}{x-1}\) ? \\

\item En quels points peut-on prolonger \(f\) par continuité ?
\end{enumerate}
\end{exo}

\begin{corr}
\note{À venir}
\end{corr}

\begin{exo}[Exercice 3]
Soient \(T\in\Rps\) et \(f:\R\to\R\) une fonction \(T\)-périodique.

\begin{enumerate}
\item Montrer que \(f\) est continue si,et seulement si, sa restriction \(\restr{f}{\intervii{0}{T}}\) est continue. \\

\item Montrer que \(f\) admet une limite en \(\pinf\) si, et seulement si, elle est constante. \\

\item Soit \(f\) la fonction \(2\)-périodique telle que \[\quantifs{\forall x\in\intervie{0}{1}}f\paren{x}=1-x\qquad\text{et}\qquad\quantifs{\forall x\in\intervie{1}{2}}f\paren{x}=x-1.\] Que dire de \(f\) (continuité, limite en \(\pinf\), en \(\minf\)) ?
\end{enumerate}
\end{exo}

\begin{corr}
\note{À venir}
\end{corr}

\begin{exo}[Exercice 4]
Soit \(\alpha\in\R\). On pose \[\fonction{f_\alpha}{\Rps}{\R}{x}{x^\alpha\floor{\dfrac{1}{x}}}\]

\begin{enumerate}
\item On suppose dans cette question que \(\alpha=0\). Déterminer en quels points \(f_0\) est continue, continue à droite, continue à gauche. \\

\item Faire la même chose pour \(f_\alpha\) (où \(\alpha\) est un réel quelconque). \\

\item Étudier la limite de \(f_\alpha\) en \(0\) et en \(\pinf\).
\end{enumerate}
\end{exo}

\begin{corr}
\note{À venir}
\end{corr}

\begin{exo}[Exercice 5]
On pose \[\fonction{f}{\Q}{\R}{x}{\cos x}\]

Déterminer les prolongements continus de \(f\) à \(\R\).
\end{exo}

\begin{corr}
\note{À venir}
\end{corr}

\begin{exo}[Exercice 6, CCP MPI 2023]
Soit \(x_0\in\R\). On définit la suite \(\paren{u_n}_n\) en posant : \[\begin{dcases}u_0=x_0 \\ \quantifs{\forall n\in\N}u_{n+1}=\Arctan u_n\end{dcases}\]

\begin{enumerate}
\item Démontrer que la suite \(\paren{u_n}_n\) est monotone et déterminer, en fonction de la valeur de \(x_0\), le sens de variation de \(\paren{u_n}_n\). \\

\item Montrer que \(\paren{u_n}_n\) converge et déterminer sa limite. \\

\item Déterminer l'ensemble des fonctions continues \(h:\R\to\R\) telles que \[\quantifs{\forall x\in\R}h\paren{x}=h\paren{\Arctan x}.\]
\end{enumerate}
\end{exo}

\begin{corr}
\note{À venir}
\end{corr}

\begin{exo}[Exercice 7]
Soit \(\phi:\R\to\R\) un endomorphisme de groupe (la loi étant l'addition usuelle).

On pose \(\alpha=\phi\paren{1}\).

\begin{enumerate}
\item Déterminer, en fonction de \(\alpha\), la restriction de \(\phi\) à \(\Z\). \\

\item Déterminer, en fonction de \(\alpha\), la restriction de \(\phi\) à \(\Q\). \\

\item On suppose \(\phi\) continu. Déterminer \(\phi\) en fonction de \(\alpha\). \\

\item Quels sont les endomorphismes continus du groupe \(\R\) ?
\end{enumerate}
\end{exo}

\begin{corr}
\note{À venir}
\end{corr}

\begin{exo}[Exercice 8]
Soient \(a,b\in\R\) tels que \(a<b\). Soient \(f:\intervii{a}{b}\to\intervii{a}{b}\) continue et \(\paren{u_n}_n\) une suite d'éléments de \(\intervii{a}{b}\) telle que \[\quantifs{\forall n\in\N}u_{n+1}=f\paren{u_n}.\]

On suppose que \(\paren{u_n}_n\) est convergente.

Montrer que sa limite \(l\) est un point fixe de \(f\) (on n'oubliera pas de justifier que \(f\) est bien définie en \(l\)).
\end{exo}

\begin{corr}
\note{À venir}
\end{corr}

\section{Principaux théorèmes}

\begin{exo}[Exercice 9]\thlabel{exo:7.17}
Démontrer le \thref{theo:descriptionDesIntervallesDeR} (inutile de traiter tous les cas).
\end{exo}

\begin{corr}
\note{À venir}
\end{corr}

\begin{exo}[Exercice 10, très classique]
Soit \(f:\intervii{0}{1}\to\intervii{0}{1}\) continue.

Montrer que \(f\) admet un point fixe, \cad : \[\quantifs{\exists x\in\intervii{0}{1}}f\paren{x}=x.\]
\end{exo}

\begin{corr}
\note{À venir}
\end{corr}

\begin{exo}[Exercice 11]
Soit \(f:\R\to\R\) continue et décroissante.

Montrer que \(f\) admet un unique point fixe.
\end{exo}

\begin{corr}
\note{À venir}
\end{corr}

\begin{exo}[Exercice 12]
Montrer que la fonction tangente admet une infinité de points fixes, \cad qu'il existe une infinité de réels \(x\) tels que \(\tan x=x\).
\end{exo}

\begin{corr}
\note{À venir}
\end{corr}

\begin{exo}[Exercice 13]
Soit \(f:\R\to\R\) continue et périodique.

Montrer que \(f\) est bornée.
\end{exo}

\begin{corr}
\note{À venir}
\end{corr}

\begin{exo}[Exercice 14]
Soit \(f:\R\to\R\) continue.

On suppose : \[\lim_{x\to\pinf}f\paren{x}=\pinf\qquad\text{et}\lim_{x\to\minf}f\paren{x}=\pinf.\]

Montrer que \(f\) admet un minimum (global).
\end{exo}

\begin{corr}
\note{À venir}
\end{corr}

\begin{exo}[Exercice 15]
Soient \(f,g\in\F{\R}{\R}\).

On suppose \(f\) continue et \(g\) bornée.

Montrer que \(g\rond f\) et \(f\rond g\) sont bornées.
\end{exo}

\begin{corr}
\note{À venir}
\end{corr}

\begin{exo}[Exercice 16, démonstration du \thref{theo:fonctionContinueEtInjectiveSurUnIntervalleEstMonotone}]\thlabel{exo:7.31}
Soient \(I\) un intervalle de \(\R\) et \(f:I\to\R\) injective et continue.

On veut montrer dans les questions (1) à (3) que \(f\) est monotone en raisonnant par l'absurde. Pour cela, on suppose \(f\) non-monotone.

\begin{enumerate}
\item Montrer qu'il existe des éléments \(a,b,c,d\in I\) tels que \[\begin{dcases}a<b \\ c<d \\ f\paren{a}<f\paren{b} \\ f\paren{c}>f\paren{d}.\end{dcases}\] \\

\item On pose \[\fonction{g}{\intervii{0}{1}}{\R}{t}{f\paren{ta+\paren{1-t}c}-f\paren{tb+\paren{1-t}d}}\] Montrer : \[\quantifs{\exists t_0\in\intervii{0}{1}}g\paren{t_0}=0.\] \\

\item Conclure. \\

\item Montrer par un exemple qu'on ne pourrait conclure sans supposer que \(f\) est définie sur un intervalle (\cad donner une fonction \(f_1:J\to\R\) injective, continue et non-monotone, où \(J\) est une partie quelconque de \(\R\)). \\

\item Montrer par un exemple qu'on ne pourrait pas conclure sans l'hypothèse de continuité de \(f\) (\cad donner une fonction \(f_2:K\to\R\) injective et non-monotone, où \(K\) est un intervalle de \(\R\)).
\end{enumerate}
\end{exo}

\begin{corr}
\note{À venir}
\end{corr}

\begin{exo}[Exercice 17]
Soient \(a,b\in\Rb\) tels que \(a<b\). Soit \(f:\intervee{a}{b}\to\R\) continue.

On suppose que \(f\) possède la même limite en \(a\) et en \(b\).

Montrer que \(f\) n'est pas injective.
\end{exo}

\begin{corr}
\note{À venir}
\end{corr}

\begin{exo}[Exercice 18]
Soient \(\lambda\in\Rps\) et \(f:\R\to\R\) continue et telle que \[\quantifs{\forall x,y\in\R}\abs{f\paren{x}-f\paren{y}}\geq\lambda\abs{x-y}.\]

Montrer que \(f\) est une bijection de \(\R\) dans \(\R\).
\end{exo}

\begin{corr}
\note{À venir}
\end{corr}

\section{Fonctions circulaires réciproques}

\begin{exo}[Exercice 19]
Montrer \[\quantifs{\forall x\in\Rs}\Arctan x+\Arctan\dfrac{1}{x}=\begin{dcases}\dfrac{\pi}{2} &\text{si }x>0 \\ \dfrac{-\pi}{2} &\text{si }x<0\end{dcases}\]
\end{exo}

\begin{corr}
\note{À venir}
\end{corr}

\begin{exo}[Exercice 20]
Soient \(x,y\in\intervee{-1}{1}\).

\begin{enumerate}
\item Montrer \[\Arctan x+\Arctan y=\Arctan\dfrac{x+y}{1-xy}.\] \\

\item Montrer \[\Arctan\dfrac{1}{2}+\Arctan\dfrac{1}{3}=\dfrac{\pi}{4}.\]
\end{enumerate}
\end{exo}

\begin{corr}
\note{À venir}
\end{corr}

\begin{exo}[Exercice 21]
Soit \(x\in\R\).

Dire pour quelles valeurs de \(x\) les expressions suivantes sont bien définies et simplifier ces expressions :

\begin{enumerate}
\item \(\cos\paren{\Arctan x}\) \\

\textit{Indication :} utiliser la relation entre \(\tan^2\) et \(\cos^2\). \\

\item \(\sin\paren{\Arctan x}\) \\

\item \(\sin\paren{2\Arctan x}\) \\

Quelle formule reconnaît-on ? \\

\item \(\cos\paren{2\Arctan x}\) \\

Quelle formule reconnaît-on ? \\

\item \(\Arctan\sqrt{\dfrac{1-\cos x}{1+\cos x}}\)
\end{enumerate}
\end{exo}

\begin{corr}
\note{À venir}
\end{corr}

\section{Fonctions lipschitziennes}

\begin{exo}[Exercice 22]
Soient \(k\in\intervie{0}{1}\) et \(f:\R\to\R\) \(k\)-lipschitzienne.

\begin{enumerate}
\item Montrer que \(f\) admet un point fixe. \\

\item Montrer que ce point fixe est unique. \\

\item Soit \(\paren{u_n}_n\in\R^\N\). On suppose \[\quantifs{\forall n\in\N}u_{n+1}=f\paren{u_n}.\] Montrer que \(\paren{u_n}_n\) converge vers le point fixe de \(f\).
\end{enumerate}
\end{exo}

\begin{corr}
\note{À venir}
\end{corr}

\begin{exo}[Exercice 23]
Soient \(a,b,\lambda\in\R\) tels que \(a\leq\lambda\leq b\). Soit \(f:\intervii{a}{b}\to\intervii{a}{b}\) \(1\)-lipschitzienne.

On considère la suite \(\paren{u_n}_n\) définie par : \(\begin{dcases}u_0=\lambda \\ \quantifs{\forall n\in\N}u_{n+1}=\dfrac{u_n+f\paren{u_n}}{2}\end{dcases}\)

\begin{enumerate}
\item Justifier que la suite \(\paren{u_n}_n\) est bien définie. \\

\item Soit \(n\in\N\). Montrer que \(u_{n+2}-u_{n+1}\) est de même signe (au sens large) que \(u_{n+1}-u_n\). \\

\item Montrer que la suite \(\paren{u_n}_n\) converge vers un point fixe de \(f\).
\end{enumerate}
\end{exo}

\begin{corr}
\note{À venir}
\end{corr}

\chapter{Arithmétique}

\minitoc

\section{Compléments sur les groupes}

\begin{exo}
Soit \(\groupe{G}\) un groupe abélien. On note \(E\) l'ensemble de ses sous-groupes.

On munit \(E\) de la loi de composition interne \(+\) (somme de sous-groupes) et de la relation d'ordre \(\subset\).

Soit \(r\in\Ns\) et \(H_1,\dots,H_r\) des sous-groupes de \(G\).

\begin{enumerate}
\item \(\groupe{E}\) est-il un groupe ? \\

\item Montrer que \(\accol{H_1;\dots;H_r}\) admet une borne inférieure et une borne supérieure dans \(E\).
\end{enumerate}
\end{exo}

\begin{corr}
\note{À venir}
\end{corr}

\begin{exo}[Familles d'entiers presque tous nuls]\thlabel{exo:familleD'EntiersPresqueTousNuls}
\renewcommand{\F}{\mathscr{F}}

Soit \(I\) un ensemble.

On appelle support d'une famille d'entiers relatifs \(\F=\paren{x_i}_{i\in I}\in\Z^I\) l'ensemble : \[\supp\F=\accol{i\in I\tq x_i\not=0}.\]

On dit que \(\F\) est une famille d'entiers presque tous nuls si son support est un ensemble fini, \cad si les termes de \(\F\) sont tous nuls sauf un nombre fini d'entre eux. L'ensemble de ces familles est noté \(\Z^{\paren{I}}\) : \[\Z^{\paren{I}}=\accol{\F\in\Z^I\tq\Card\supp\F<\pinf}.\]

Montrer que \(\Z^{\paren{I}}\) est un sous groupe de \(\groupe{\Z^I}\).
\end{exo}

\begin{corr}
\note{À venir}
\end{corr}

\section{Arithmétique}

\begin{exo}
Calculer :

\begin{enumerate}
\item \(100^{1234567}\) modulo \(13\) \\

\item \(1234^{12345678910}\) modulo \(21\) \\

\item \(1234^{12345^{123456}}\) modulo \(256\) \\

\item \(1000^{1000^{1000}}\) modulo \(17\) \\

\textit{Indication :} on pourra utiliser le petit théorème de Fermat.
\end{enumerate}
\end{exo}

\begin{corr}[1]
On a \(100\equiv9\croch{13}\).

Or on a \(9^2\equiv3\croch{13}\), \(9^3\equiv1\croch{13}\) et \(9^4\equiv9\croch{13}\).

Donc, avec \(N\) un entier relatif, \(9^N\) modulo \(13\) ne dépend que de \(N\) modulo \(3\).

Or \(1234567\equiv1\croch{3}\).

Donc \(100^{1234567}\equiv9^1\croch{13}\equiv9\croch{13}\).
\end{corr}

\begin{exo}
Soient \(a,b\in\Z\).

\begin{enumerate}
\item Montrer que \(8\) divise \(a^2-1\) si, et seulement si, \(a\) est impair. \\

\item Montrer que \(7\) divise \(a^2+b^2\) si, et seulement si, \(7\) divise \(a\) et \(b\).
\end{enumerate}
\end{exo}

\begin{corr}
\note{À venir}
\end{corr}

\begin{exo}
\begin{enumerate}
\item Soient \(n\) et \(\alpha\) deux entiers supérieurs ou égaux à \(2\). Montrer que si \(n^\alpha\) est un nombre premier alors \(n=2\) et \(\alpha\) est un nombre premier. La réciproque est-elle vraie ? \\

\item Soit \(\beta\) un entier naturel. Montrer que si \(2^\beta+1\) est premier alors \(\beta\) est une puissance de \(2\).
\end{enumerate}
\end{exo}

\begin{corr}
\note{À venir}
\end{corr}

\begin{exo}
Pour tout \(n\in\N\), on note \(u_n\) l'entier naturel dont l'écriture en base \(10\) possède \(3^n\) chiffres, tous égaux à \(1\) : \[\quantifs{\forall n\in\N}u_n=\underbrace{1111\dots1}_{3^n\text{ chiffres}}.\]

Déterminer la valuation \(3\)-adique de \(u_n\).

\textit{Indication :} remarquer \(\quantifs{\forall n\in\N}u_{n+1}=u_n\paren{1+10^{3^n}+10^{3^n\times2}}\).
\end{exo}

\begin{corr}
\note{À venir}
\end{corr}

\begin{exo}
Soit \(n\in\N\).

\begin{enumerate}
\item Justifier que \(n+1\) et \(2n+1\) sont premiers entre eux. \\

\item Montrer que \(n+1\) divise \(\binom{n}{2n}\). \\

\textit{Indication :} considérer \(\binom{n+1}{2n+1}\).
\end{enumerate}
\end{exo}

\begin{corr}
\note{À venir}
\end{corr}

\begin{exo}
Soient \(m,n\in\interventierie{2}{\pinf}\).

On suppose \(\dfrac{\ln m}{\ln n}\in\Q\).

Montrer que \(m\) et \(n\) ont les mêmes diviseurs premiers.
\end{exo}

\begin{corr}
\note{À venir}
\end{corr}

\begin{exo}
Trouver un entier \(x\in\Z\) tel que \(\begin{dcases}x\equiv2\croch{7} \\ x\equiv3\croch{9}\end{dcases}\)
\end{exo}

\begin{corr}
\note{À venir}
\end{corr}

\begin{exo}
Trouver un entier \(x\in\Z\) tel que \(\begin{dcases}x\equiv5\croch{7} \\ x\equiv10\croch{16}\end{dcases}\)
\end{exo}

\begin{corr}
\note{À venir}
\end{corr}

\begin{exo}
Trouver un entier \(x\in\Z\) tel que \(\begin{dcases}x\equiv5\croch{9} \\ x\equiv10\croch{15}\end{dcases}\)
\end{exo}

\begin{corr}
\note{À venir}
\end{corr}

\begin{exo}
Soit \(p\in\prem\).

\begin{enumerate}
\item Montrer \(\quantifs{\forall k\in\interventierii{1}{p-1}}p\divise\binom{k}{p}\). \\

\item En déduire \(\quantifs{\forall k\in\interventierii{0}{p-1}}\binom{k}{p-1}\equiv\paren{-1}^k\croch{p}\).
\end{enumerate}
\end{exo}

\begin{corr}
\note{À venir}
\end{corr}

\begin{exo}
Soit \(n\in\interventierie{2}{\pinf}\).

Montrer \[n\in\prem\ssi\quantifs{\forall k\in\interventierii{1}{n-1}}n\divise\binom{k}{n}.\]
\end{exo}

\begin{corr}
\note{À venir}
\end{corr}

\begin{exo}[Valuations \(p\)-adiques des rationnels]
\begin{enumerate}
\item Soit \(a\in\prem\). Montrer qu'on définit une fonction \(w_p:\Qs\to\Z\) en posant : \[\quantifs{\forall a\in\Zs;\forall b\in\Ns}w_p\paren{\dfrac{a}{b}}=\valp{p}{a}-\valp{p}{b}.\] \\

\textit{Indication :} il s'agit de montrer que l'image d'un rationnel ne dépend pas de l'écriture \(\dfrac{a}{b}\) choisie. \\

\item En utilisant l'\thref{exo:familleD'EntiersPresqueTousNuls}, montrer que les groupes \(\groupe{\Qs}[\times]\) et \(\groupe{\Z^{\paren{\prem}}}\) sont isomorphes.
\end{enumerate}
\end{exo}

\begin{corr}
\note{À venir}
\end{corr}

\begin{exo}
On note \(\paren{p_n}_{n\in\N}\) la suite strictement croissante des nombres premiers (\cad \(p_0=2\), \(p_1=3\), \(p_2=5\), ...).

\begin{enumerate}[series=exsuitepremiers]
\item Montrer \(\quantifs{\forall n\in\N}p_{n+1}\leq p_0\times p_1\times\dots\times p_n+1\). \\

\item En déduire \(\quantifs{\forall n\in\N}p_n\leq2^{2^n}\).
\end{enumerate}

Pour tout \(N\in\N\), on note \(\pi\paren{N}\) le nombre de nombres premiers inférieurs à \(N\) : \[\quantifs{\forall N\in\N}\pi\paren{N}=\Card\prem\inter\interventierii{1}{N}.\]

\begin{enumerate}[resume=exsuitepremiers]
\item Montrer l'encadrement \(\quantifs{\forall N\in\interventierie{2}{\pinf}}\log_2\rond\log_2\paren{N}\leq\pi\paren{N}\leq N\).
\end{enumerate}

\textit{Remarque :} le logarithme en base \(2\) est la fonction \(\log_2:\Rps\to\R\) qui est la bijection réciproque de la fonction \[\fonctionlambda{\R}{\Rps}{x}{2^x}\]

On montre facilement : \(\quantifs{\forall x\in\Rps}\log_2 x=\dfrac{\ln x}{\ln2}\).
\end{exo}

\begin{corr}
\note{À venir}
\end{corr}

\begin{exo}
Soit \(p\) un nombre premier impair.

On suppose que \(-1\) est un carré modulo \(p\) : \[\quantifs{\exists x\in\Z}-1\equiv x^2\croch{p}.\]

Montrer \(p\equiv1\croch{4}\).

\textit{Indication :} utiliser le petit théorème de Fermat.

{\small \textit{Remarque :} la CN prouvée est en fait une CNS : \(\croch{\quantifs{\exists x\in\Z}-1\equiv x^2\croch{p}}\ssi p\equiv1\croch{4}\).}
\end{exo}

\begin{corr}
\note{À venir}
\end{corr}

\begin{exo}
\begin{enumerate}
\item Montrer qu'il existe une infinité de nombres premiers \(p\) tels que \(p\equiv3\croch{4}\). \\

\item En utilisant l'exercice précédent, montrer qu'il existe une infinité de nombres premiers \(p\) tels que \(p\equiv1\croch{4}\).
\end{enumerate}

\textit{Indications :}

\begin{enumerate}
\item S'intéresser aux diviseurs premiers de \(4\paren{n!}-1\) avec \(n\in\N\). \\

\item S'intéresser aux diviseurs premiers de \(\paren{n!}^2+1\) avec \(n\in\N\).
\end{enumerate}
\end{exo}

\begin{corr}
\note{À venir}
\end{corr}

\chapter{Fonctions dérivables}

\minitoc

\section{Étude locale}

\begin{exo}[Exercice 1]
Étudier la dérivabilité des fonctions suivantes (en précisant leurs ensembles de définition) :

\begin{enumerate}
\item \(f:x\mapsto\cos\sqrt{x}\) \\

\item \(f:x\mapsto\sin\abs{x}\) \\

\item \(f:x\mapsto\ln\paren{1+\sqrt{x}}\) \\

\item \(f:x\mapsto x\abs{x}\) \\

\item \(f:x\mapsto\sqrt{x^2-x^3}\) \\

\item \(f:x\mapsto\paren{1-x}\Arccos x\)
\end{enumerate}
\end{exo}

\begin{corr}
\note{À venir}
\end{corr}

\begin{exo}[Exercice 2]
Montrer que la fonction \[f:x\mapsto\ln\paren{\dfrac{\e{x}-1}{x}}\] est définie sur \(\Rs\) et se prolonge en une fonction de classe \(\classe{1}\) sur \(\R\).
\end{exo}

\begin{corr}
\note{À venir}
\end{corr}

\begin{exo}[Exercice 3]
Soient \(f:\R\to\R\) et \(a\in\R\).

On suppose que \(f\) est dérivable en \(a\).

Déterminer \(\lim_{\substack{h\to0 \\ h\not=0}}\dfrac{f\paren{a+2h}-f\paren{a-h}}{h}\).
\end{exo}

\begin{corr}
\note{À venir}
\end{corr}

\begin{exo}[Exercice 4]
Pour tout \(k\in\N\), calculer la dérivée \(k\)-ème des fonctions suivantes :

\begin{enumerate}
\item \(f:x\mapsto\ln x\) \\

\item \(f:x\mapsto x^2\ln x\) \\

\item \(f:x\mapsto\cos^4x\) \\

\item \(f:x\mapsto\dfrac{1}{\sqrt{x}}\) \\

\item \(f:x\mapsto\dfrac{x-1}{x+1}\)
\end{enumerate}

On simplifiera les résultats en utilisant des factorielles si c'est utile.
\end{exo}

\begin{corr}
\note{À venir}
\end{corr}

\begin{exo}[Exercice 5]
Soit \(n\in\N\).

\begin{enumerate}
\item Pour tout \(k\in\N\), calculer la dérivée \(k\)-ème de \(f:x\mapsto x^n\). \\

\item En calculant de deux façons la dérivée \(n\)-ème de \(g:x\mapsto x^{2n}\), calculer : \[S=\sum_{k=0}^n\binom{k}{n}^2.\]
\end{enumerate}
\end{exo}

\begin{corr}
\note{À venir}
\end{corr}

\section{Étude globale}

\begin{exo}[Exercice 6]
Étudier la fonction \(f:x\mapsto\Arccos\paren{\dfrac{1-x^2}{1+x^2}}\).

On précisera l'ensemble de définition de \(f\), en quels points \(f\) est continue, en quels points \(f\) est dérivable, les variations de \(f\) (avec ses limites), et le graphe de \(f\).
\end{exo}

\begin{corr}
\note{À venir}
\end{corr}

\begin{exo}[Exercice 7]
Dessiner le graphe de la fonction : \[f:x\mapsto\Arccos\sqrt{\dfrac{1+\sin x}{2}}-\Arcsin\sqrt{\dfrac{1+\cos x}{2}}.\]
\end{exo}

\begin{corr}
\note{À venir}
\end{corr}

\begin{exo}[Exercice 8]
Montrer \[\quantifs{\forall x\in\Rps}\Arcsin\paren{\dfrac{1-x}{1+x}}=\Arctan\paren{\dfrac{1-x}{2\sqrt{x}}}\] de deux façons :

\begin{enumerate}
\item En calculant \(\tan\paren{\Arcsin\paren{\dfrac{1-x}{1+x}}}\). \\

\item En dérivant.
\end{enumerate}
\end{exo}

\begin{corr}
\note{À venir}
\end{corr}

\begin{exo}[Exercice 9]
Soit \(\alpha\in\intervee{1}{\pinf}\).

\begin{enumerate}
\item Déterminer toutes les foncions \(f:\R\to\R\) telles que : \[\quantifs{\forall x,y\in\R}\abs{f\paren{x}-f\paren{y}}\leq\abs{x-y}^{\alpha}.\]

\textit{Indication :} montrer qu'une telle fonction est nécessairement dérivable. \\

\item Déterminer toutes les fonctions \(f:\Rs\to\R\) telles que : \[\quantifs{\forall x,y\in\Rs}\abs{f\paren{x}-f\paren{y}}\leq\abs{x-y}^{\alpha}.\]
\end{enumerate}
\end{exo}

\begin{corr}
\note{À venir}
\end{corr}

\begin{exo}[Exercice 10]
Soit \(f:\Rp\to\R\) une fonction dérivable telle que : \[f\paren{0}=0\qquad\text{et}\qquad\lim_{x\to\pinf}f\paren{x}=0.\]

\begin{enumerate}
\item Montrer que \(f\) est bornée. \\

\item À l'aide du théorème de Rolle, montrer : \[\quantifs{\exists c\in\Rps}f\prim\paren{c}=0.\]
\end{enumerate}
\end{exo}

\begin{corr}
\note{À venir}
\end{corr}

\begin{exo}[Exercice 11]
Soient \(a,b\in\R\) tels que \(a<b\) et \(f,g\in\F{\intervii{a}{b}}{\R}\) continues.

On suppose que \(f\) et \(g\) sont  dérivables sur \(\intervee{a}{b}\).

Montrer : \[\quantifs{\exists c\in\intervee{a}{b}}f\prim\paren{c}\paren{g\paren{b}-g\paren{a}}=g\prim\paren{c}\paren{f\paren{b}-f\paren{a}}.\]
\end{exo}

\begin{corr}
\note{À venir}
\end{corr}

\begin{exo}[Exercice 12]
On pose : \[\fonction{f}{\intervii{0}{1}}{\R}{x}{\dfrac{\e{x}}{x+2}}\]

\begin{enumerate}
\item Soit \(a\in\intervii{0}{1}\). Montrer que l'on définit une suite \(\paren{u_n}_{n\in\N}\in\intervii{0}{1}^\N\) en posant : \[u_0=a\qquad\text{et}\qquad\quantifs{\forall n\in\N}u_{n+1}=f\paren{u_n}.\]

\item Montrer que \(f\) est \(\dfrac{2\e{}}{9}\)-lipschitzienne. \\

\item Montrer que \(f\) admet un unique point fixe \(\alpha\) dans \(\intervii{0}{1}\). \\

\item Montrer que la suite \(\paren{u_n}_n\) converge vers \(\alpha\) et que l'on a : \[\quantifs{\forall n\in\N}\abs{u_n-\alpha}\leq\paren{\dfrac{2\e{}}{9}}^n.\]

\item Donner un rang \(n\in\N\) tel que \(u_n\) soit une approximation de \(\alpha\) à \(10^{-3}\) près.
\end{enumerate}
\end{exo}

\begin{corr}
\note{À venir}
\end{corr}

\section{Convexité}

\begin{exo}[Exercice 13]
Soient \(a,b,c\in\R\).

Donner une CNS pour que la fonction \[\fonction{f}{\R}{\R}{x}{ax^2+bx+c}\] soit convexe.
\end{exo}

\begin{corr}
\note{À venir}
\end{corr}

\begin{exo}[Exercice 14]
Soit \(\alpha\in\Rps\).

On pose : \[\fonction{f_\alpha}{\R}{\R}{x}{\exp\paren{\dfrac{-x^2}{2\alpha^2}}}\]

Étudier la convexité de \(f_\alpha\) en fonction de \(\alpha\).
\end{exo}

\begin{corr}
\note{À venir}
\end{corr}

\begin{exo}[Exercice 15]
Soit \(x\in\intervii{0}{1}\).

Montrer : \[\dfrac{\pi}{4}x\leq\Arctan x\leq x\leq\Arcsin x\leq\dfrac{\pi}{2}x.\]
\end{exo}

\begin{corr}
\note{À venir}
\end{corr}

\begin{exo}[Exercice 16]
Montrer : \[\quantifs{\forall x_1,\dots,x_n\in\Rp}\sqrt{x_1+\dots+x_n}\leq\sqrt{x_1}+\dots+\sqrt{x_n}\leq\sqrt{n\paren{x_1+\dots+x_n}}.\]
\end{exo}

\begin{corr}
\note{À venir}
\end{corr}

\begin{exo}[Exercice 17]
\begin{enumerate}
\item Donner un exemple de fonction dérivable de \(\Rp\) dans \(\R\) qui soit convexe, majorée et non-constante. \\

\item Soit \(f:\R\to\R\) une fonction dérivable, convexe et majorée. Montrer que \(f\) est constante. \\

\item Soit \(f:\R\to\R\) une fonction convexe et majorée. Montrer que \(f\) est constante.
\end{enumerate}
\end{exo}

\begin{corr}
\note{À venir}
\end{corr}

\begin{exo}[Exercice 18]
Soit \(f:\intervii{0}{1}\to\R\) convexe.

\begin{enumerate}
\item Montrer que \(f\) est dérivable à droite et à gauche en tout point de \(\intervee{0}{1}\). \\

\item La fonction \(f\) est-elle nécessairement dérivable en \(0\) ? en \(1\) ? en \(\dfrac{1}{2}\) ? \\

\item La fonction \(f\) est-elle nécessairement continue en \(0\) ? en \(1\) ? en \(\dfrac{1}{2}\) ?
\end{enumerate}
\end{exo}

\begin{corr}
\note{À venir}
\end{corr}

\begin{exo}[Exercice 19]
Soit \(f:\Rp\to\R\) convexe.

\begin{enumerate}
\item Montrer que \(\lim_{x\to\pinf}\dfrac{f\paren{x}}{x}\) existe. On la note \(l\). \\

\item Montrer que si \(l\leq0\) alors \(f\) est décroissante sur \(\Rp\).

\textit{Indication :} raisonner par l'absurde. \\

\item Montrer que \(f\) admet une limite en \(\pinf\). \\

\item Établir, pour chaque valeur de \(l\), quelles sont les différentes limites possibles pour \(f\) en \(\pinf\), en illustrant chaque possibilité par un exemple.
\end{enumerate}
\end{exo}

\begin{corr}
\note{À venir}
\end{corr}

\begin{exo}[Exercice 20]
\begin{enumerate}
\item Montrer que la fonction \(\fonction{f}{\R}{\R}{x}{\ln\paren{1+\e{x}}}\) est convexe. \\

\item En déduire : \[\quantifs{\forall x_1,\dots,x_n\in\Rps}1+\paren{\prod_{k=1}^nx_k}^{\nicefrac{1}{n}}\leq\paren{\prod_{k=1}^n\paren{1+x_k}}^{\nicefrac{1}{n}}\] puis : \[\quantifs{\forall a_1,\dots,a_n,b_1,\dots,b_n\in\Rps}\paren{\prod_{k=1}^na_k}^{\nicefrac{1}{n}}+\paren{\prod_{k=1}^nb_k}^{\nicefrac{1}{n}}\leq\paren{\prod_{k=1}^n\paren{a_k+b_k}}^{\nicefrac{1}{n}}.\]
\end{enumerate}
\end{exo}

\begin{corr}
\note{À venir}
\end{corr}

\begin{exo}[Exercice 21]
Soient \(n\in\Ns\) et \(x\in\intervie{1}{\pinf}\).

Montrer : \[n\paren{x^{\nicefrac{n+1}{2}}-x^{\nicefrac{n-1}{2}}}\leq x^n-1.\]

\textit{Indication :} factoriser par \(x-1\).
\end{exo}

\begin{corr}
\note{À venir}
\end{corr}

\begin{exo}[Exercice 22]
Soient \(a,b\in\R\) tels que \(a<b\) et \(f\in\ensclasse{2}{\intervii{a}{b}}{\R}\) telle que \(f\paren{a}=f\paren{b}=0\).

\begin{enumerate}
\item Justifier que \[M=\max_{\intervii{a}{b}}\abs{f\seconde}\] est bien défini. \\

\item Étudier la convexité des fonctions \[\fonction{g}{\intervii{a}{b}}{\R}{x}{f\paren{x}+M\dfrac{\paren{x-a}\paren{b-x}}{2}}\quad\text{et}\quad\fonction{h}{\intervii{a}{b}}{\R}{x}{f\paren{x}-M\dfrac{\paren{x-a}\paren{b-x}}{2}}\]

\item En déduire : \[\quantifs{\forall x\in\intervii{a}{b}}\abs{f\paren{x}}\leq M\dfrac{\paren{x-a}\paren{b-x}}{2}.\]
\end{enumerate}
\end{exo}

\begin{corr}
\note{À venir}
\end{corr}

\begin{exo}[Exercice 23, inégalités de Hölder et de Minkowski]
Soient \(p,q\in\intervee{1}{\pinf}\) tels que : \[\dfrac{1}{p}+\dfrac{1}{q}=1.\]

\begin{enumerate}
\item Montrer : \[\quantifs{\forall x,y\in\Rps}xy\leq\dfrac{1}{p}x^p+\dfrac{1}{q}y^q.\]

\item Soient \(x_1,\dots,x_n,y_1,\dots,y_n\in\Rps\) tels que \(\sum_{k=1}^nx_k^p=\sum_{k=1}^ny_k^q=1\).

Montrer : \[\sum_{k=1}^nx_ky_k\leq1.\]

\item En déduire l'inégalité de Hölder : \[\quantifs{\forall a_1,\dots,a_n,b_1,\dots,b_n\in\Rp}\sum_{k=1}^na_kb_k\leq\paren{\sum_{k=1}^na_k^p}^{\nicefrac{1}{p}}\paren{\sum_{k=1}^nb_k^q}^{\nicefrac{1}{q}}.\]

\item En déduire l'inégalité de Minkowski : \[\quantifs{\forall a_1,\dots,a_n,b_1,\dots,b_n\in\Rp}\paren{\sum_{k=1}^n\paren{a_k+b_k}^p}^{\nicefrac{1}{p}}\leq\paren{\sum_{k=1}^na_k^p}^{\nicefrac{1}{p}}+\paren{\sum_{k=1}^nb_k^p}^{\nicefrac{1}{p}}.\]
\end{enumerate}
\end{exo}

\begin{corr}
\note{À venir}
\end{corr}

\chapter{Polynômes, fractions rationnelles}

Soit \(\K\) un corps.

\section{Polynômes}

\begin{exo}[Exercice 1]
On pose \[P_1=X^4+3X-1\qquad P_2=X^2+4\qquad P_3=X+1.\]

\begin{enumerate}
\item Faire, pour tout \(\paren{i,j}\in\accol{1;2;3}\), la division euclidienne de \(P_i\) par \(P_j\). \\

\item Calculer \[P_3\paren{P_2}\qquad P_2\paren{P_3}\qquad P_2\paren{P_1}\qquad P_2\prim\paren{P_1}\qquad P_1\prim\paren{P_3}.\]
\end{enumerate}
\end{exo}

\begin{corr}
\note{À venir}
\end{corr}

\begin{exo}[Exercice 2]
\begin{enumerate}
\item Déterminer tous les polynômes \(P\in\poly[\R]\) tels que \[P\paren{X^3}=P^2.\]

\item Déterminer tous les polynômes \(P\in\poly[\R]\) tels que \[X^2P\paren{X^2}=P\paren{X^3}+2\paren{X-1}\paren{X^3+X+1}.\]
\end{enumerate}
\end{exo}

\begin{corr}
\note{À venir}
\end{corr}

\begin{exo}[Exercice 3]
\begin{enumerate}
\item Déterminer les polynômes \(P\) à coefficients complexes tels que \[P=P\prim P\seconde.\]

\item Déterminer les polynômes \(P\) à coefficients complexes tels que \[P=P\deriv{1}P\deriv{2}P\deriv{3}.\]

\item Déterminer les polynômes \(P\) à coefficients complexes tels que \[P=P\deriv{1}P\deriv{2}P\deriv{3}P\deriv{4}.\]
\end{enumerate}
\end{exo}

\begin{corr}
\note{À venir}
\end{corr}

\begin{exo}[Exercice 4]
Déterminer tous les polynômes \(P\in\poly[\R]\) tels que \(\begin{dcases}P\paren{-1}=-17 \\ P\paren{0}=-7 \\ P\paren{1}=-3 \\ P\paren{3}=35\end{dcases}\)

\textit{Indication :} commencer par calculer l'unique polynôme \(P\in\polydeg[\R]{3}\) solution du système.
\end{exo}

\begin{corr}
\note{À venir}
\end{corr}

\begin{exo}[Exercice 5]
Décomposer en produit de polynômes irréductibles dans \(\poly[\C]\) puis dans \(\poly[\R]\) les polynômes suivants : \[P_1=X^4-4\qquad P_2=X^3-2X^2+2X\qquad P_3=X^6+64.\]
\end{exo}

\begin{corr}
\note{À venir}
\end{corr}

\begin{exo}[Exercice 6]
Soit \(n\in\Ns\).

Calculer \[\prod_{\omega\in\U_n\excluant\accol{1}}\paren{X-\omega}.\]
\end{exo}

\begin{corr}
\note{À venir}
\end{corr}

\begin{exo}[Exercice 7]
Soit \(n\in\N\).

Calculer \[S_1=\sum_{k=0}^nk\binom{k}{n}\qquad S_2=\sum_{k=0}^nk^2\binom{k}{n}\qquad S_3=\sum_{k=0}^nk^3\binom{k}{n}.\]
\end{exo}

\begin{corr}
\note{À venir}
\end{corr}

\begin{exo}[Exercice 8, algorithme d'Euclide]
Appliquer l'algorithme d'Euclide étendu à \(A=X^8+X\) et \(B=X^5+X\) pour trouver un PGCD \(D\) de \(A\) et \(B\) et des polynômes \(U\) et \(V\) tels que \(UA+VB=D\).

\textit{Il s'agit d'un simple exercice d'application, ne pas chercher à ruser pour aller plus vite.}
\end{exo}

\begin{corr}
\note{À venir}
\end{corr}

\begin{exo}[Exercice 9]
Soient \(A,B\in\poly[\R]\).

Montrer que \(A\) divise \(B\) dans \(\poly[\R]\) si, et seulement si, \(A\) divise \(B\) dans \(\poly[\C]\).
\end{exo}

\begin{corr}
\note{À venir}
\end{corr}

\begin{exo}[Exercice 10]
Soient \(m,n\in\Ns\).

Montrer que \(X^n-1\) divise \(X^m-1\) dans \(\poly[\R]\) si, et seulement si, \(n\) divise \(m\) dans \(\Z\).
\end{exo}

\begin{corr}
\note{À venir}
\end{corr}

\begin{exo}[Exercice 11]
Déterminer les polynômes \(P\in\poly[\C]\) tels que \(P\prim\) divise \(P\).
\end{exo}

\begin{corr}
\note{À venir}
\end{corr}

\begin{exo}[Exercice 12]
Soit \(P\in\poly\).

Montrer que \(P-X\) divise \(P\rond P-X\).
\end{exo}

\begin{corr}
\note{À venir}
\end{corr}

\begin{exo}[Exercice 13, devinettes]
Soient \(a,b,c\in\R\) et \(P\in\poly[\R]\).

Quel est le reste de la division euclidienne de \(P\) par \(\paren{X-a}\paren{X-b}\paren{X-c}\)

\begin{enumerate}
\item si \(a\), \(b\) et \(c\) sont deux à deux distincts ? \\

\item si \(a=b=c\) ?
\end{enumerate}
\end{exo}

\begin{corr}
\note{À venir}
\end{corr}

\begin{exo}[Exercice 14]
Soit \(n\in\Ns\).

On pose \[P_n=\sum_{k=0}^n\dfrac{X^k}{k!}\].

Montrer que \(P_n\) est scindé à racines simples dans \(\poly[\C]\).
\end{exo}

\begin{corr}
\note{À venir}
\end{corr}

\begin{exo}[Exercice 15]
Soit \(n\in\Ns\).

\begin{enumerate}
\item Calculer \(\sum_{x\in\U_n}x^k\) en fonction de \(k\in\Z\). \\

\item Soient \(P\in\poly[\C]\) et \(M\in\Rp\) tels que \[\deg P<n\qquad\text{et}\qquad\quantifs{\forall x\in\U_n}\abs{P\paren{x}}\leq M.\]

Montrer que les coefficients de \(P\) sont de module majoré par \(M\).
\end{enumerate}
\end{exo}

\begin{corr}
\note{À venir}
\end{corr}

\begin{exo}[Exercice 16, classique]\thlabel{exo:classiqueSurLesPolynômesScindés}
Dans tout l'exercice, \guillemets{scindé} signifie \guillemets{scindé sur \(\R\)}.

Soit \(P\in\poly[\R]\).

\begin{enumerate}
\item Montrer que si \(P\) est scindé à racines simples, alors \(P\prim\) est scindé à racines simples. \\

\item Montrer que si \(P\) est scindé, alors \(P\prim\) est scindé.
\end{enumerate}
\end{exo}

\begin{corr}
\note{À venir}
\end{corr}

\begin{exo}[Exercice 17, suite de l'exercice précédent, moins classique]
Soit \(P\in\poly[\R]\).

On suppose que \(P\) s'écrit \(P=\sum_{k=0}^na_kX^k\) avec \[n\geq3\qquad a_0\not=0\qquad a_n\not=0\qquad\quantifs{\exists k\in\interventierii{1}{n-2}}a_k=a_{k+1}=0.\]

Montrer que \(P\) n'est pas scindé sur \(\R\).
\end{exo}

\begin{corr}
\note{À venir}
\end{corr}

\begin{exo}[Exercice 18]
On pose \(P=X^3+3X^2+3X+9\).

\begin{enumerate}
\item Montrer que toutes les racines complexes de \(P\) sont simples. \\

\item Calculer la somme des racines complexe de \(P\). \\

\item Calculer la somme des carrés des racines complexes de \(P\). \\

\item Calculer la somme des cubes des racines complexes de \(P\).
\end{enumerate}
\end{exo}

\begin{corr}
\note{À venir}
\end{corr}

\begin{exo}[Exercice 19]
On pose \(P=X^6+4X^5-3X^4-32X^3-53X^2-36X-9\).

\begin{enumerate}
\item Montrer que \(-1\) est racine de \(P\) et calculer sa multiplicité. \\

\item Déterminer les autres racines de \(P\).
\end{enumerate}
\end{exo}

\begin{corr}
\note{À venir}
\end{corr}

\section{Fractions rationnelles}

\begin{exo}[Exercice 20]
Soit \(F\in\fracrat[\C]\).

Montrer : \[F=\conj{F}\ssi F\in\fracrat[\R].\]
\end{exo}

\begin{corr}
\note{À venir}
\end{corr}

\begin{exo}[Exercice 21]
Calculer la décomposition en éléments simples des fractions rationnelles suivantes :

\begin{enumerate}
\item \(\dfrac{X^2+2X+5}{X^2-3X+2}\) \\

\item \(\dfrac{X^2+1}{\paren{X-1}\paren{X-2}\paren{X-3}}\) \\

\item \(\dfrac{4}{X^4-1}\) (sur \(\C\) puis sur \(\R\)) \\

\item \(\dfrac{1}{X\paren{X-1}^2}\) \\

\item \(\dfrac{2X}{X^2+1}\) (sur \(\C\)) \\

\item \(\dfrac{3X-1}{X^2\paren{X+1}^2}\)
\end{enumerate}
\end{exo}

\begin{corr}
\note{À venir}
\end{corr}

\begin{exo}[Exercice 22]
\begin{enumerate}
\item Rappeler les lois des groupes \(\fracrat\excluant\accol{0}\) et \(\fracrat\). \\

\item Montrer que \(\fonction{\phi}{\fracrat\excluant\accol{0}}{\fracrat}{F}{\dfrac{F\prim}{F}}\) est un morphisme de groupes. \\

Quel est son noyau ? \\

\item Soit \(F\in\fracrat[\C]\excluant\accol{0}\). \\

Quelle est la décomposition en éléments simples de \(\dfrac{F\prim}{F}\) ? On l'exprimera en fonction des racines et pôles de \(F\), et de leurs multiplicités respectives.
\end{enumerate}
\end{exo}

\begin{corr}
\note{À venir}
\end{corr}

\begin{exo}[Exercice 23]
Soient \(\lambda_1,\dots,\lambda_n\in\K\) et \(\mu\in\K\excluant\accol{0}\).

On pose \(P=\mu\paren{X-\lambda_1}\dots\paren{X-\lambda_n}\).

Exprimer en fonction de \(P\), \(P\prim\) et \(P\seconde\) les fractions rationnelles suivantes : \[F=\sum_{k=1}^n\dfrac{1}{X-\lambda_k}\qquad G=\sum_{k=1}^n\dfrac{1}{\paren{X-\lambda_k}^2}\qquad H=\sum_{1\leq k<l\leq n}\dfrac{1}{\paren{X-\lambda_k}\paren{X-\lambda_l}}.\]
\end{exo}

\begin{corr}
\note{À venir}
\end{corr}

\begin{exo}[Exercice 24]
Soient \(n\in\Ns\) et \(a_0,\dots,a_n\in\R\) tels que \(a_n\not=0\).

On suppose que le polynôme \(P=a_nX^n+\dots+a_0X^0\) est scindé sur \(\R\).

\begin{enumerate}
\item Montrer : \(\quantifs{\forall x\in\R}P\prim\paren{x}^2-P\paren{x}P\seconde\paren{x}\geq0\). \\

\item En déduire : \(\quantifs{\forall k\in\interventierii{0}{n-2}}a_ka_{k+2}\leq a_{k+1}^2\). \\

\textit{Indication :} utiliser l'\thref{exo:classiqueSurLesPolynômesScindés}.
\end{enumerate}
\end{exo}

\begin{corr}
\note{À venir}
\end{corr}

\begin{exo}[Exercice 25]
Soit \(n\in\Ns\).

On pose \(\omega=\e{\frac{2\i\pi}{n}}\).

\begin{enumerate}
\item Soit \(P\in\poly[\C]\) tel que \(P\paren{\omega X}=P\). Montrer : \[\quantifs{\exists Q\in\poly[\C]}P=Q\paren{X^n}.\]

\item En déduire la forme irréductible de la fraction rationnelle : \[F=\sum_{k=0}^{n-1}\dfrac{X+\omega^k}{X-\omega^k}.\]
\end{enumerate}
\end{exo}

\begin{corr}
\note{À venir}
\end{corr}

\begin{exo}[Exercice 26, un peu calculatoire]
Soit \(n\in\N\).

On pose : \[A_n=\sum_{l=0}^{\floor{\frac{n-1}{2}}}\paren{-1}^l\binom{2l+1}{n}X^{2l+1}\qquad B_n=\sum_{l=0}^{\floor{\frac{n}{2}}}\paren{-1}^l\binom{2l}{n}X^{2l}\qquad F_n=\dfrac{A_n}{B_n}.\]

\begin{enumerate}
\item Soit \(\theta\in\R\) tel que \(\theta\not\equiv\dfrac{\pi}{2}\croch{\pi}\) et \(n\theta\not\equiv\dfrac{\pi}{2}\croch{\pi}\). \\

Montrer : \(\tan\paren{n\theta}=F_n\paren{\tan\theta}\). \\

\textit{Indication :} utiliser la formule du binôme de Newton à \(\paren{\cos\theta+\i\sin\theta}^n\) pour calculer \[\tan\paren{n\theta}=\dfrac{\Im\paren{\paren{\cos\theta+\i\sin\theta}^n}}{\Re\paren{\paren{\cos\theta+\i\sin\theta}^n}}.\]

\item Quelle formule retrouve-t-on si \(n=2\) ? \\

\item Quels sont les pôles de \(F_n\) ? En déduire que \(A_n\) et \(B_n\) sont premiers entre eux. \\

\item Quelle est la partie entière de \(F_n\) ? On donnera le résultat en fonction de la parité de \(n\). \\

\item Donner la décomposition en éléments simples de \(F_n\).
\end{enumerate}
\end{exo}

\begin{corr}
\note{À venir}
\end{corr}

\chapter{Intégrales sur un segment}

\minitoc

\begin{exo}[Exercice 1]
Soit \(\alpha\in\Rps\).

Calculer l'intégrale \[\int_0^1\dfrac{\odif{t}}{1+t^{\alpha}}\] pour \(\alpha=1\), \(\alpha=2\), \(\alpha=3\) et \(\alpha=4\).
\end{exo}

\begin{corr}
\note{À venir}
\end{corr}

\begin{exo}[Exercice 2]
Déterminer une primitive pour chacun des fonctions suivantes (on précisera sur quel ensemble de définition) :

\begin{enumerate}
\item \(f:x\mapsto\dfrac{1}{x^4-x^2-2}\) \\

\item \(f:x\mapsto\dfrac{x}{1+x^4}\) \\

\item \(f:x\mapsto\dfrac{x^2}{1+x^3}\) \\

\item \(f:x\mapsto\dfrac{1}{x+\i}\) \\

\item \(f=\cos^3\) \\

\item \(f=\tan\) \\

\item \(f=\Arctan\) \\

\item \(f:x\mapsto x\Arctan x\) \\

\item \(f:x\mapsto\dfrac{1}{\cos x}\) \\

\item \(f:x\mapsto\dfrac{\sin x}{\cos^2x+2}\) \\

\item \(f:x\mapsto x^7\ln x\) \\

\item \(f:x\mapsto\ln^3x\) \\

\item \(f:x\mapsto\dfrac{x}{\sqrt{1+x^2}}\) \\

\item \(f:x\mapsto\dfrac{1}{\sqrt{x}+2x}\) \\

\item \(f:x\mapsto\dfrac{1}{x\ln x}\) \\

\item \(f:x\mapsto\dfrac{1}{1+\e{x}}\)
\end{enumerate}
\end{exo}

\begin{corr}
\note{À venir}
\end{corr}

\begin{exo}[Exercice 3]
On pose \[f:x\mapsto\int_0^{\sin^2x}\Arcsin\sqrt{t}\odif{t}+\int_0^{\cos^2x}\Arccos\sqrt{t}\odif{t}.\]

\begin{enumerate}
\item Quel est l'ensemble de définition de \(f\) ? \\

\item Déterminer \(f\).
\end{enumerate}
\end{exo}

\begin{corr}
\note{À venir}
\end{corr}

\begin{exo}[Exercice 4]
Soient \(a,b\in\R\) tels que \(a<b\).

On pose : \[f:x\mapsto\dfrac{1}{\sqrt{\paren{x-a}\paren{b-x}}}.\]

\begin{enumerate}
\item Quel est l'ensemble de définition de \(f\) ? \\

\item Déterminer une primitive de \(f\).

\textit{Indication :} faire le changement de variable \[y=\dfrac{2}{b-a}\paren{x-\dfrac{a+b}{2}}.\]
\end{enumerate}
\end{exo}

\begin{corr}
\note{À venir}
\end{corr}

\begin{exo}[Exercice 5]
On pose \[I=\int_0^{\frac{\pi}{2}}\dfrac{\sin t}{\sin t+\cos t}\odif{t}\qquad\text{et}\qquad J=\int_0^{\frac{\pi}{2}}\dfrac{\cos t}{\sin t+\cos t}\odif{t}.\]

\begin{enumerate}
\item Justifier que les intégrales \(I\) et \(J\) sont bien définies. \\

\item Montrer \(I=J\) par un changement de variable. \\

\item En déduire la valeur de \(I\) et \(J\). \\

\item En déduire \(\int_0^1\dfrac{\odif{t}}{t+\sqrt{1-t^2}}\).
\end{enumerate}
\end{exo}

\begin{corr}
\note{À venir}
\end{corr}

\begin{exo}[Exercice 6]
Soit \(f\in\ensclasse{2}{\R}{\R}\).

On suppose \[\lim_{x\to\pinf}f\paren{x}=\lim_{x\to\pinf}f\seconde\paren{x}=0.\]

Montrer : \[\lim_{x\to\pinf}f\prim\paren{x}=0.\]

\textit{Indication :} on pourra utiliser la formule de Taylor avec reste intégral.
\end{exo}

\begin{corr}
\note{À venir}
\end{corr}

\begin{exo}[Exercice 7, classique à l'oral, de CCP à l'X]
On pose \[\fonction{f}{\Rps\excluant\accol{1}}{\R}{x}{\int_x^{x^2}\dfrac{\odif{t}}{\ln t}}\]

\begin{enumerate}
\item Vérifier que la fonction \(f\) est bien définie. Quel est son signe ? \\

\item Dresser le tableau de variations de \(f\) (en précisant les limites en \(0^+\), \(1^-\), \(1^+\) et \(\pinf\)).

\textit{Indication :} pour la limite en \(1^+\), montrer : \[\quantifs{\forall x\in\intervee{1}{\pinf}}\int_x^{x^2}\dfrac{x\odif{t}}{t\ln t}\leq f\paren{x}\leq\int_x^{x^2}\dfrac{x^2\odif{t}}{t\ln t}.\]

\item En déduire que \(f\) se prolonge en une fonction continue \(g:\Rp\to\R\). \\

\item Montrer que \(g\) est de classe \(\classe{1}\).
\end{enumerate}
\end{exo}

\begin{corr}
\note{À venir}
\end{corr}

\begin{exo}[Exercice 8, lemme de Lebesgue]
Soient \(a,b\in\R\) tels que \(a<b\) et \(f\in\ensclasse{1}{\intervii{a}{b}}{\C}\).

Montrer : \[\lim_{n\to\pinf}\int_a^bf\paren{t}\cos\paren{nt}\odif{t}=0.\]

\textit{Indication :} faire une intégration par parties.
\end{exo}

\begin{corr}
\note{À venir}
\end{corr}

\begin{exo}[Exercice 9]
À l'aide de la formule de Taylor avec reste intégral, montrer :

\begin{enumerate}
\item \(\quantifs{\forall x\in\intervei{-1}{1}}\ln\paren{1+x}=\lim_{n\to\pinf}\sum_{k=1}^n\dfrac{\paren{-1}^{k-1}x^k}{k}\). \\

\item \(\quantifs{\forall n\in\Ns;\forall x\in\intervei{-1}{0}}\ln\paren{1+x}\leq\sum_{k=1}^n\dfrac{\paren{-1}^{k-1}x^k}{k}\). \\

\item \(\quantifs{\forall n\in\Ns;\forall x\in\intervie{0}{\pinf}}\sum_{k=1}^{2n}\dfrac{\paren{-1}^{k-1}x^k}{k}\leq\ln\paren{1+x}\leq\sum_{k=1}^{2n+1}\dfrac{\paren{-1}^{k-1}x^k}{k}\).
\end{enumerate}
\end{exo}

\begin{corr}
\note{À venir}
\end{corr}

\begin{exo}[Exercice 10]
Soit \(f:\intervii{0}{1}\to\intervii{0}{1}\) une fonction continue telle que : \[\int_0^1f\paren{t}\odif{t}=\int_0^1f\paren{t}^2\odif{t}.\]

Montrer que \(f\) est constante, égale à \(0\) ou \(1\).
\end{exo}

\begin{corr}
\note{À venir}
\end{corr}

\begin{exo}[Exercice 11]
Soit une fonction continue \(f:\intervii{0}{1}\to\R\) telle que : \[\int_0^1f\paren{t}\odif{t}=\dfrac{1}{2}.\]

Montrer que \(f\) admet un point fixe.

Donner un contre-exemple quand \(f\) n'est pas supposée continue (et seulement supposée continue par morceaux).
\end{exo}

\begin{corr}
\note{À venir}
\end{corr}

\begin{exo}[Exercice 12]
Calculer la fonction \[\fonction{f}{\Rs}{\R}{x}{\int_{\frac{1}{x}}^x\dfrac{t}{1+t+t^2+t^3}\odif{t}}\]
\end{exo}

\begin{corr}
\note{À venir}
\end{corr}

\begin{exo}[Exercice 13, \thref{dem:convergenceDesSommesDeRiemannDansLeCasContinu}]\thlabel{exo:convergenceDesSommesDeRiemannDansLeCasContinu}
Soient \(a,b\in\R\) tels que \(a<b\) et \(f:\intervii{a}{b}\to\R\) continue.

On considère les sommes de Riemann de \(f\) : \[\quantifs{\forall n\in\Ns}R_n\paren{f}=\dfrac{b-a}{n}\sum_{k=0}^{n-1}f\paren{a+k\dfrac{b-a}{n}}.\]

Montrer que la suite \(\paren{R_n\paren{f}}_{n\in\Ns}\) converge vers l'intégrale de \(f\) sur \(\intervii{a}{b}\).
\end{exo}

\begin{corr}
\note{À venir}
\end{corr}

\begin{exo}[Exercice 14]
Déterminer \[\lim_{n\to\pinf}\sum_{k=1}^{2n}\dfrac{k}{n^2+k^2}.\]
\end{exo}

\begin{corr}
\note{À venir}
\end{corr}

\begin{exo}[Exercice 15]
Déterminer \[\lim_{n\to\pinf}\sum_{k=n}^{2n}\dfrac{1}{k}.\]
\end{exo}

\begin{corr}
\note{À venir}
\end{corr}

\begin{exo}[Exercice 16]
Déterminer \[\lim_{n\to\pinf}\sum_{k=n}^{2n}\dfrac{1}{2k+1}.\]
\end{exo}

\begin{corr}
\note{À venir}
\end{corr}

\begin{exo}[Exercice 17]
Déterminer \[\lim_{n\to\pinf}\dfrac{1}{n}\sqrt[n]{\dfrac{\paren{2n}!}{n!}}.\]

\textit{Indication :} remarquer que pour tout \(n\in\Ns\), on a : \[\prod_{k=1}^n\paren{n+k}=n^n\prod_{k=1}^n\paren{1+\dfrac{k}{n}}.\]
\end{exo}

\begin{corr}
\note{À venir}
\end{corr}

\begin{exo}[Exercice 18, ENSEA 2018]
\begin{enumerate}
\item À l'aide de la formule de Taylor avec reste intégral, montrer : \[\quantifs{\forall x\in\intervii{0}{1}}0\leq\e{x}-1-x\leq\dfrac{\e{}}{2}x^2.\]

\item En déduire la limite de la suite \(\paren{u_n}_n\) définie par : \[\quantifs{\forall n\in\Ns}u_n=\sum_{k=1}^n\e{\frac{1}{n+k}}-n.\]
\end{enumerate}
\end{exo}

\begin{corr}
\note{À venir}
\end{corr}

\chapter{Espaces vectoriels}

\minitoc

Soit \(\K\) un corps.

\section{Espaces vectoriels}

\begin{exo}[Exercice 1]
Les ensembles suivants sont-ils (naturellement) des \(\R\)-espaces vectoriels ?

\begin{enumerate}
\item \(\accol{\paren{a,b,c,d}\in\R^4\tq a+2b-c-d=\lambda}\) où \(\lambda\) est un réel fixé. \\

\item \(\accol{y\in\ensclasse{2}{\R}{\R}\tq y\seconde+2y\prim+fy=g}\) où \(f\) et \(g\) sont deux fonctions fixées. \\

\item \(\accol{y\in\ensclasse{\infty}{\R}{\R}\tq y\text{ est croissante sur }\R}\). \\

\item \(\accol{y\in\ensclasse{\infty}{\R}{\R}\tq y\text{ est monotone sur }\R}\).
\end{enumerate}
\end{exo}

\begin{corr}
\note{À venir}
\end{corr}

\begin{exo}[Exercice 2, Mines-Telecom PSI 2016]
Soient \(E\) un espace vectoriel et \(F\) et \(G\) deux sous-espaces vectoriels de \(E\).

Montrer : \[F\union G\text{ est un sous-espace vectoriel de }E\ssi\orenv{F\subset G \\ G\subset F}\]
\end{exo}

\begin{corr}
\note{À venir}
\end{corr}

\section{Applications linéaires}

\begin{exo}[Exercice 3]
Les applications suivantes sont-elles des applications linéaires (pour les structures naturelles d'espaces vectoriels) ?

\[\fonction{u_1}{\R}{\R}{x}{x^2}\qquad\fonction{u_2}{\ensclasse{\infty}{\R}{\R}}{\R\times\ensclasse{\infty}{\R}{\R}}{f}{\paren{f\prim\paren{1},f\seconde}}\qquad\fonction{u_3}{\poly[\R]}{\poly[\R]}{P}{P\paren{X^2}}\]
\end{exo}

\begin{corr}
\note{À venir}
\end{corr}

\begin{exo}[Exercice 4]
On note \(B\) l'ensemble des suites réelles bornées et \(C\) l'ensemble des suites réelles convergentes.

\begin{enumerate}
\item Montrer que \(B\) et \(C\) sont naturellement des \(\R\)-espaces vectoriels. \\

\item L'application \[\fonction{u}{B}{\R}{\paren{u_n}_{n\in\N}}{\sup_{n\in\N}u_n}\] est-elle une forme linéaire ? \\

\item L'application \[\fonction{v}{C}{\R}{\paren{u_n}_{n\in\N}}{\lim_{n\to\pinf}u_n}\] est-elle une forme linéaire ?
\end{enumerate}
\end{exo}

\begin{corr}
\note{À venir}
\end{corr}

\begin{exo}[Exercice 5]
Soient \(E\) un \(\K\)-espace vectoriel et \(u,v\in\Lendo{E}\).

On suppose que les endomorphismes \(u\) et \(v\) commutent.

Montrer que \(u\) stabilise \(\ker v\) et \(\Im v\), \cad : \[u\paren{\ker v}\subset\ker v\qquad\text{et}\qquad u\paren{\Im v}\subset\Im v.\]
\end{exo}

\begin{corr}
\note{À venir}
\end{corr}

\begin{exo}[Exercice 6, résultats à bien connaître]
Soient \(E\) un \(\K\)-espace vectoriel et \(u,v\in\Lendo{E}\).

\begin{enumerate}
\item Montrer : \[v\rond u=0\ssi\Im u\subset\ker v.\]

\item Donner les inclusions qui sont vraies entre les sous-espaces vectoriels de \(E\) suivants :

\begin{enumerate}
\item \(\Im\paren{v\rond u}\) et \(\Im v\). \\

\item \(\ker\paren{v\rond u}\) et \(\ker u\). \\

\item \(\Im\paren{u+v}\) et \(\Im u+\Im v\). \\

\item \(\ker\paren{u+v}\) et \(\ker u\inter\ker v\). \\
\end{enumerate}

\item Que peut-on en déduire pour les suites de sous-espaces vectoriels de \(E\) : \[\paren{\Im u^k}_{k\in\N}\qquad\text{et}\qquad\paren{\ker u^k}_{k\in\N}\text{ ?}\]
\end{enumerate}
\end{exo}

\begin{corr}
\note{À venir}
\end{corr}

\begin{exo}[Exercice 7, classique]
Soient \(E\) un \(\K\)-espace vectoriel et \(u\in\Lendo{E}\).

On suppose que pour tout vecteur \(x\in E\), le vecteur \(u\paren{x}\) est colinéaire à \(x\).

\begin{enumerate}
\item Justifier que \(u\) vérifie : \[\quantifs{\forall x\in E;\exists\lambda_x\in\K}u\paren{x}=\lambda_xx.\]

\item Montrer que \(u\) est une homothétie.
\end{enumerate}
\end{exo}

\begin{corr}
\note{À venir}
\end{corr}

\begin{exo}[Exercice 8, Arts \& Métiers PSI 2016]
On pose \[E=\accol{f\in\ensclasse{\infty}{\R}{\R}\tq\quantifs{\forall x\in\R}f\paren{x+2\pi}=f\paren{x}}\qquad\text{et}\qquad\quantifs{\forall f\in E}u\paren{f}=f\seconde.\]

\begin{enumerate}
\item Montrer que \(E\) est un espace vectoriel et que \(u\in\Lendo{E}\). \\

\item Déterminer \(\ker u\). \\

\item Déterminer les vecteurs \(f\in E\) tels que \(u\paren{f}=\sin\) et les vecteurs \(f\in E\) tels que \(u\paren{f}=\sin^2\). \\

\item Déterminer \(\Im u\). \\

\item Montrer \(\ker u\oplus\Im u=E\).
\end{enumerate}
\end{exo}

\begin{corr}
\note{À venir}
\end{corr}

\begin{exo}[Exercice 9]
Soient \(E\), \(F\) et \(G\) des \(\K\)-espaces vectoriels et \(f:E\to F\) et \(g:F\to G\) des applications linéaires.

\begin{enumerate}
\item Montrer : \[f\paren{\ker\paren{g\rond f}}=\Im f\inter\ker g.\]

\item Montrer : \[\ker\paren{g\rond f}=\ker f\ssi\Im f\inter\ker g=\accol{0_F}.\]
\end{enumerate}
\end{exo}

\begin{corr}
\note{À venir}
\end{corr}

\section{Sommes directes, projecteurs}

\begin{exo}[Exercice 10]
Soient \(E\) un espace vectoriel et \(p,q\in\Lendo{E}\) des projecteurs.

Montrer que les propositions suivantes sont équivalentes :

\begin{enumerate}
\item \(p+q\) est un projecteur \\

\item \(pq+qp=0\) \\

\item \(pq=qp=0\)
\end{enumerate}
\end{exo}

\begin{corr}
\note{À venir}
\end{corr}

\begin{exo}[Exercice 11]
Soient \(E\) un espace vectoriel et \(f,g\in\Lendo{E}\) tels que \(f\rond g=\id{E}\).

Montrer que \(g\rond f\) est un projecteur. Déterminer son image et son noyau.
\end{exo}

\begin{corr}
\note{À venir}
\end{corr}

\begin{exo}[Exercice 12]
Soient \(E\) un \(\K\)-espace vectoriel et \(p\) et \(q\) des projecteurs de \(E\) qui commutent.

\begin{enumerate}
\item Montrer que \(pq\) est un projecteur. \\

\item Montrer \(\Im\paren{pq}=\Im p\inter\Im q\). \\

\item Montrer \(\ker\paren{pq}=\ker p+\ker q\).
\end{enumerate}
\end{exo}

\begin{corr}
\note{À venir}
\end{corr}

\begin{exo}[Exercice 13, polynômes interpolateurs de Lagrange]
Soient \(x_0,\dots,x_n\in\K\) deux à deux distincts.

On note \(L_0,\dots,L_n\in\polydeg{n}\) les polynômes définis par : \[\quantifs{\forall j\in\interventierii{0}{n}}L_j=\prod_{k\in\interventierii{0}{n}\excluant\accol{j}}\dfrac{X-x_k}{x_j-x_k}.\]

\begin{enumerate}
\item Montrer que \(\paren{L_0,\dots,L_n}\) est une base de \(\polydeg{n}\). \\

\item Montrer que l'application \[\fonction{u}{\poly}{\poly}{P}{P\paren{x_0}L_0+\dots+P\paren{x_n}L_n}\] est un projecteur. \\

\item Déterminer l'image et le noyau de \(u\).
\end{enumerate}
\end{exo}

\begin{corr}
\note{À venir}
\end{corr}

\begin{exo}[Exercice 14]
Donner un supplémentaire de \(\poly\) dans \(\fracrat\).
\end{exo}

\begin{corr}
\note{À venir}
\end{corr}

\section{Familles libres / génératrices}

\begin{exo}[Exercice 15]
On considère le système linéaire homogène : \[\paren{S}\begin{dcases}
a+\i b+c+\i d=0 \\
\i a-b+\i c-d=0 \\
\paren{1-\i}a+\paren{1+\i}b+\paren{1+\i}c-\paren{1-\i}d=0
\end{dcases}\]

On note \(\fami{S}\subset\C^4\) son ensemble solution.

C'est un \(\C\)-espace vectoriel et donc également un \(\R\)-espace vectoriel.

\begin{enumerate}
\item Donner une base du \(\C\)-espace vectoriel \(\fami{S}\). \\

\item Donner une base du \(\R\)-espace vectoriel \(\fami{S}\).
\end{enumerate}
\end{exo}

\begin{corr}
\note{À venir}
\end{corr}

\begin{exo}[Exercice 16]
On pose \[\quantifs{\forall\lambda\in\R}\fonction{f_\lambda}{\R}{\R}{t}{\e{\lambda t}}\]

On considère la famille \(\fami{F}=\paren{f_\lambda}_{\lambda\in\R}\) de vecteurs de \(\F{\R}{\R}\).

\begin{enumerate}
\item La famille \(\fami{F}\) est-elle une base de \(\F{\R}{\R}\) ? \\

\item La famille \(\fami{F}\) est-elle une base de \(\ensclasse{\infty}{\R}{\R}\) ? \\

\item Montrer que la famille \(\fami{F}\) n'est pas libre

\begin{enumerate}
\item en utilisant des limites en \(\pinf\). \\

\item en utilisant des polynômes interpolateurs de Lagrange.
\end{enumerate}
\end{enumerate}
\end{exo}

\begin{corr}
\note{À venir}
\end{corr}

\begin{exo}[Exercice 17]
On se place dans le \(\R\)-espace vectoriel \(\F{\R}{\R}\).

On définit les vecteurs suivants : \[\quantifs{\forall\lambda\in\R}\fonction{f_\lambda}{\R}{\R}{t}{\sin\paren{\lambda t}}\]

Les familles suivantes sont-elles libres ?

\begin{enumerate}
\item \(\paren{f_\lambda}_{\lambda\in\R}\) \\

\item \(\paren{f_\lambda}_{\lambda\in\Rs}\) \\

\item \(\paren{f_\lambda}_{\lambda\in\Rps}\)
\end{enumerate}
\end{exo}

\begin{corr}
\note{À venir}
\end{corr}

\begin{exo}[Exercice 18]
Donner une base de l'image et du noyau des endomorphismes suivants :

\begin{enumerate}
\item \[\fonction{u}{\R^3}{\R^3}{\paren{a,b,c}}{\paren{a-b,b-c,c-a}}\]

\item \[\fonction{u}{\polydeg[\R]{3}}{\polydeg[\R]{3}}{P}{P\paren{X+1}-P}\]
\end{enumerate}
\end{exo}

\begin{corr}
\note{À venir}
\end{corr}

\begin{exo}[Exercice 19, base duale]\thlabel{exo:baseDuale}
Soient \(E\) un \(\K\)-espace vectoriel et \(\fami{B}=\paren{e_1,\dots,e_n}\) une base de \(E\).

Pour tout \(k\in\interventierii{1}{n}\), on note \(e_k\etoile:E\to\K\) l'application qui associe à tout vecteur de \(E\) sa \(k\)-ème coordonnée dans la base \(\fami{B}\).

Les applications \(e_1\etoile,\dots,e_n\etoile\) sont donc caractérisées par : \[\quantifs{\forall x\in E}x=\sum_{k=1}^ne_k\etoile\paren{x}e_k.\]

Montrer que la famille \(\fami{B}\etoile=\paren{e_1\etoile,\dots,e_n\etoile}\) est une base de \(E\etoile\) (appelée base duale de \(\fami{B}\)).
\end{exo}

\begin{corr}
\note{À venir}
\end{corr}

\begin{exo}[Exercice 20]
Montrer que la famille infinie \(\paren{\ln p}_{p\in\prem}\) est une famille libre du \(\Q\)-espace vectoriel \(\R\).
\end{exo}

\begin{corr}
\note{À venir}
\end{corr}

\begin{exo}[Exercice 21]
On se place dans le \(\Q\)-espace vectoriel \(\R\).

\begin{enumerate}
\item Soit \(p\in\prem\). Montrer que la famille \(\paren{1,\sqrt{p}}\) est libre sur \(\Q\). \\

\item En déduire que la famille \(\paren{1,\sqrt{2},\sqrt{3},\sqrt{6}}\) est libre sur \(\Q\).
\end{enumerate}
\end{exo}

\begin{corr}
\note{À venir}
\end{corr}

\begin{exo}[Exercice 22]
Résoudre les systèmes linéaires suivants en appliquant l'algorithme du pivot de Gauss.

Lorsque l'ensemble solution est un espace affine, on précisera une base de sa direction.

\[\paren{S_1}\begin{dcases}
a+b+2c+d=7 \\
4a+5b+3c+4d=17 \\
a+2b-3c+d=-4
\end{dcases}\qquad\paren{S_2}\begin{dcases}
a+b+2c=7 \\
4a+5b+3c=17 \\
a+2b-3c=7
\end{dcases}\]
\end{exo}

\begin{corr}
\note{À venir}
\end{corr}

\chapter{Équations différentielles}

\section{Équations différentielles linéaires d'ordre 1}

\begin{exo}[Exercice 1]
Résoudre l'équation différentielle : \[\paren{E}~y\prim+y=\sin t.\]
\end{exo}

\begin{corr}
\note{À venir}
\end{corr}

\begin{exo}[Exercice 2]
Résoudre l'équation différentielle : \[\paren{E}~y\prim+\dfrac{\sin t}{\cos t-2}y=\dfrac{2\sin t}{\cos t-2}.\]
\end{exo}

\begin{corr}
\note{À venir}
\end{corr}

\begin{exo}[Exercice 3]
Résoudre le problème de Cauchy : \[\begin{dcases}
y\prim-\dfrac{t^3+t+1}{t^2+1}y=\dfrac{t^4+t-1}{t^2+1} \\
y\paren{0}=1
\end{dcases}\]
\end{exo}

\begin{corr}
\note{À venir}
\end{corr}

\begin{exo}[Exercice 4]
Résoudre l'équation différentielle suivante sur l'intervalle \(\intervee{0}{1}\) : \[\paren{E}~t\ln ty\prim-y=2t^2\ln^2t.\]
\end{exo}

\begin{corr}
\note{À venir}
\end{corr}

\begin{exo}[Exercice 5]
Résoudre l'équation différentielle suivante sur l'intervalle \(\intervee{-1}{1}\) : \[\paren{E}~\sqrt{1-t^2}y\prim-y=1.\]
\end{exo}

\begin{corr}
\note{À venir}
\end{corr}

\begin{exo}[Exercice 6]\thlabel{exo:exempleProblèmesDeRaccordCours}
On considère l'équation différentielle : \[\paren{E}~ty\prim+y=\cos t.\]

\begin{enumerate}
    \item Déterminer les solutions de \(\paren{E}\) sur \(\Rps\) (respectivement \(\Rms\)). \\
    \item En déduire les solutions définies sur \(\R\).
\end{enumerate}
\end{exo}

\begin{corr}[1]
On pose \(I=\Rps\) ou \(\Rms\).

Résolvons \(\paren{E}\) sur \(I\).

Résolvons \(\paren{E_0}~y\prim+\dfrac{1}{t}y=0\).

Une primitive de \(t\mapsto\dfrac{1}{t}\) est \(t\mapsto\ln\abs{t}\).

Donc les solutions de \(\paren{E_0}\) sont les fonctions de la forme \(t\mapsto\lambda\e{-\ln\abs{t}}=\dfrac{\lambda}{\abs{t}}\) avec \(\lambda\in\R\), \cad de la forme : \[t\mapsto\dfrac{\lambda}{t}\] avec \(\lambda\in\R\).

Déterminons une solution de \(\paren{E}\).

Soit \(\lambda\in\ensclasse{1}{I}{\R}\).

On pose \(y_1:t\mapsto\dfrac{\lambda\paren{t}}{t}=\lambda\paren{t}t\inv\).

On a : \[\begin{aligned}
y_1\text{ est solution de }\paren{E}&\ssi\quantifs{\forall t\in I}t\paren{\lambda\prim\paren{t}t\inv-\lambda\paren{t}t^2}+\lambda\paren{t}t\inv=\cos t \\
&\ssi\quantifs{\forall t\in I}\lambda\prim\paren{t}=\cos t \\
&\impr\lambda=\sin.
\end{aligned}\]

Donc \(y_1:t\mapsto\dfrac{\sin t}{t}\) est solution de \(\paren{E}\).

Finalement, les solutions de \(\paren{E}\) sur \(I\) sont les fonctions de la forme : \[t\mapsto\dfrac{\lambda}{t}+\dfrac{\sin t}{t}=\dfrac{\lambda+\sin t}{t}\] où \(\lambda,\mu\in\R\).
\end{corr}

\begin{corr}[2]
\analyse

Soit \(y\in\ensclasse{1}{\R}{\R}\) telle que \(\quantifs{\forall t\in\R}ty\prim\paren{t}+y\paren{t}=\cos t\).

Comme \(y\) est solution de \(\paren{E}\) sur \(\Rps\), il existe \(\lambda\in\R\) tel que \[\quantifs{\forall t\in\Rps}y\paren{t}=\dfrac{\lambda}{t}+\dfrac{\sin t}{t}.\]

Comme \(y\) est solution de \(\paren{E}\) sur \(\Rms\), il existe \(\mu\in\R\) tel que \[\quantifs{\forall t\in\Rms}y\paren{t}=\dfrac{\mu}{t}+\dfrac{\sin t}{t}.\]

De plus, en \(t=0\), on a \(0+y\paren{0}=1\).

D'où : \[\fonction{y}{\R}{\R}{t}{\begin{dcases}
\dfrac{\lambda}{t}+\dfrac{\sin t}{t} &\text{si }t>0 \\
\dfrac{\mu}{t}+\dfrac{\sin t}{t} &\text{si }t<0 \\
1 &\text{si }t=0
\end{dcases}}\]

Comme \(y\) est continue en \(0\), on a \(\lim_{0^+}y=1\).

Or \(\lim_{0^+}y=\begin{dcases}
\pinf &\text{si }\lambda>0 \\
\minf &\text{si }\lambda<0 \\
1 &\text{si }\lambda=0
\end{dcases}\)

Donc \(\lambda=0\).

De même, comme \(\lim_{0^-}y=1\), on a \(\mu=0\).

Donc : \[\fonction{y}{\R}{\R}{t}{\begin{dcases}
\dfrac{\sin t}{t} &\text{si }t\not=0 \\
1 &\text{sinon}
\end{dcases}}\]

\synthese

Posons \(\fonction{y}{\R}{\R}{t}{\begin{dcases}
\dfrac{\sin t}{t} &\text{si }t\not=0 \\
1 &\text{sinon}
\end{dcases}}\)

On a \(y\in\ensclasse{1}{\R}{\R}\) selon le théorème de la limite de la dérivée et \(y\) est solution de \(\paren{E}\).

\conclusion

La fonction \[\fonction{y}{\R}{\R}{t}{\begin{dcases}
\dfrac{\sin t}{t} &\text{si }t\not=0 \\
1 &\text{sinon}
\end{dcases}}\] est l'unique solution de \(\paren{E}\).
\end{corr}

\begin{exo}[Exercice 7, Mines MP 2006]
Résoudre l'équation différentielle suivante sur l'intervalle \(\intervee{0}{\pinf}\) : \[\paren{E}~t\ln ty\prim-\paren{3\ln t+1}y=0.\]
\end{exo}

\begin{corr}
\note{À venir}
\end{corr}

\begin{exo}[Exercice 8]
Trouver les solutions réelles sur \(\R\) à l'équation différentielle : \[\paren{E}~ty\prim-\paren{2t+1}y=2t+1.\]

Quelles sont les solutions vérifiant \(y\paren{0}=0\) ? celles vérifiant \(y\paren{0}=-1\) ?
\end{exo}

\begin{corr}
\note{À venir}
\end{corr}

\section{Équations différentielles linéaires d'ordre 2}

\begin{exo}[Exercice 9]
Résoudre les équations différentielles suivantes :

\begin{enumerate}
    \item \(y\seconde-2y\prim+y=\e{3t}+3t\e{t}\) \\
    \item \(y\seconde+4y\prim-5y=\sh t\) \\
    \item \(y\seconde+4y=t\sin t\).
\end{enumerate}
\end{exo}

\begin{corr}
\note{À venir}
\end{corr}

\begin{exo}[Exercice 10]\thlabel{exo:exempleChangementDeVariableCours}
Résoudre sur l'intervalle \(\Rps\) l'équation différentielle : \[\paren{E}~t^2y\seconde+3ty\prim+y=\paren{t+1}^2\] en faisant le changement de variable \(s=\ln t\).
\end{exo}

\begin{corr}~\\
\begin{brouill}
On a \(s=\ln t\ssi t=\e{s}\) et \(y\paren{t}=y\paren{\e{s}}=z\paren{s}=z\paren{\ln t}\).
\end{brouill}

Soit \(y\in\ensclasse{2}{\Rps}{\R}\).

On pose \(\fonction{z}{\R}{\R}{s}{y\paren{\e{s}}}\)

On a \(\quantifs{\forall t\in\Rps}\begin{dcases}
y\paren{t}=z\paren{\ln t} \\
y\prim\paren{t}=\dfrac{1}{t}z\prim\paren{\ln t} \\
y\seconde\paren{t}=\dfrac{1}{t^2}z\seconde\paren{t}-\dfrac{1}{t^2}z\prim\paren{\ln t}
\end{dcases}\)

Donc : \[\begin{aligned}
y\text{ est solution de }\paren{E}&\ssi\quantifs{\forall t\in\Rps}t^2y\seconde\paren{t}+3ty\prim\paren{t}+y\paren{t}=\paren{t+1}^2 \\
&\ssi\quantifs{\forall t\in\Rps}z\seconde\paren{\ln t}-z\prim\paren{\ln t}+3z\prim\paren{\ln t}+z\paren{\ln t}=\paren{t+1}^2 \\
&\ssi\quantifs{\forall s\in\R}z\seconde\paren{s}+2z\prim\paren{s}+z\paren{s}=\paren{\e{s}+1}^2 \\
&\ssi z\text{ est solution de }\paren{E\prim}~z\seconde+2z\prim+z=\paren{\e{s}+1}^2.
\end{aligned}\]

Résolvons \(\paren{E\prim}\).

Résolvons \(\paren{E_0\prim}~z\seconde+2z\prim+z=0\).

L'équation caractéristique \(x^2+2x+1\) admet \(-1\) comme solution double.

Donc les solutions de \(\paren{E_0\prim}\) sont les fonctions de la forme \(z_0:s\mapsto\paren{\lambda s+\mu}\e{-t}\) avec \(\lambda,\mu\in\R\).

Déterminons une solution de \(\paren{E\prim}\).

On pose : \[\paren{E_1\prim}~y\seconde+2y\prim+y=\e{2s}\qquad\paren{E_2\prim}~y\seconde+2y\prim+y=\e{s}\qquad\paren{E_3\prim}y\seconde+2y\prim+y=1.\]

Déterminons une solution de \(\paren{E_1\prim}\).

\begin{brouill}
On a \(P\paren{s}\e{\gamma s}\) avec \(\begin{dcases}
P=1\text{ donc }\deg P=0 \\
\gamma=2
\end{dcases}\)

Donc on cherche une solution de la forme \(Q\paren{s}\e{\gamma s}\) avec \(\deg Q=0\), \cad de la forme \(a\e{2s}\).
\end{brouill}

Soit \(a\in\R\).

On pose \(z_1:s\mapsto a\e{2s}\).

On a \(\quantifs{\forall s\in\R}\begin{dcases}
z_1\prim\paren{s}=2a\e{2s} \\
z_1\seconde\paren{s}=4a\e{2s}
\end{dcases}\)

Donc \[\begin{aligned}
z_1\text{ est solution de }\paren{E_1\prim}&\ssi\quantifs{\forall s\in\R}4a\e{2s}+4a\e{2s}+a\e{2s}=\e{2s} \\
&\ssi\quantifs{\forall s\in\R}9a\e{2s}=\e{2s} \\
&\impr a=\dfrac{1}{9}.
\end{aligned}\]

Donc \(z_1:s\mapsto\dfrac{1}{9}\e{2s}\) est solution de \(\paren{E_1\prim}\).

Déterminons une solution de \(\paren{E_2\prim}\).

Soit \(a\in\R\).

On pose \(z_2:s\mapsto a\e{s}\).

On a : \[\begin{aligned}
z_2\text{ est solution de }\paren{E_2\prim}&\ssi a\e{s}+2a\e{s}+a\e{s}=\e{s} \\
&\impr a=\dfrac{1}{4}.
\end{aligned}\]

Donc \(z_2:s\mapsto\dfrac{1}{4}\e{s}\) est solution de \(\paren{E_2\prim}\).

On remarque que \(z_3:s\mapsto1\) est solution de \(\paren{E_3\prim}\).

Donc \(z:s\mapsto\dfrac{1}{9}\e{2s}+\dfrac{1}{2}\e{s}+1\) est solution de \(\paren{E\prim}\) selon le principe de superposition.

Ainsi, les solutions de \(\paren{E\prim}\) sont les fonctions de la forme \[s\mapsto\paren{\lambda s+\mu}\e{-s}+\dfrac{1}{9}\e{2s}+\dfrac{1}{2}\e{s}+1\] avec \(\lambda,\mu\in\R\).

Finalement : \[\begin{aligned}
y\text{ est solution de }\paren{E}&\ssi\quantifs{\exists\lambda,\mu\in\R;\forall s\in\R}z\paren{s}=\paren{\lambda s+\mu}\e{-s}+\dfrac{1}{9}\e{2s}+\dfrac{1}{2}\e{s}+1 \\
&\ssi\quantifs{\exists\lambda,\mu\in\R;\forall t\in\Rps}z\paren{\ln t}=\paren{\lambda\ln t+\mu}\e{-\ln t}+\dfrac{1}{9}\e{2\ln t}+\dfrac{1}{2}\e{\ln t}+1 \\
&\ssi\quantifs{\exists\lambda,\mu\in\R;\forall t\in\Rps}y\paren{t}=\dfrac{\lambda\ln t+\mu}{t}+\dfrac{1}{9}t^2+\dfrac{1}{2}t+1.
\end{aligned}\]
\end{corr}

\begin{exo}[Exercice 11]
Résoudre l'équation différentielle : \[\paren{E}~\paren{1+t^2}^2y\seconde+2t\paren{1+t^2}y\prim+y=\Arctan t\] en faisant le changement de variable \(x=\Arctan t\).
\end{exo}

\begin{corr}
\note{À venir}
\end{corr}

\begin{exo}[Exercice 12]
Résoudre sur l'intervalle \(\Rps\) l'équation différentielle : \[\paren{E}~t^2y\seconde+ty\prim-y=t^2\] en faisant le changement de variable \(x=\ln t\).
\end{exo}

\begin{corr}
\note{À venir}
\end{corr}

\begin{exo}[Exercice 13, Mines-Ponts PSI 2017]
Soit \(a\) un réel non-nul.

Résoudre \[\paren{E}~\paren{1+x^2}^2y\seconde+2x\paren{1+x^2}y\prim+a^2y=0\] en faisant le changement de variable \(\theta=\Arctan x\).
\end{exo}

\begin{corr}
\note{À venir}
\end{corr}

\begin{exo}[Exercice 14]
Résoudre sur \(\R\) l'équation différentielle réelle : \[\paren{E}~ty\seconde+2\paren{t+1}y\prim+\paren{t+2}y=0.\]

\textit{Indication :} on pourra considérer la fonction \(z:t\mapsto ty\paren{t}\).
\end{exo}

\begin{corr}
\note{À venir}
\end{corr}

\section{Applications}

\begin{exo}[Exercice 15]
Résoudre\footnote{\Cad déterminer les couples \(\paren{y_1,y_2}\in\ensclasse{1}{\R}{\R}^2\) qui sont solution} de deux façons le système linéaire : \[\paren{E}~\begin{dcases}
y_1\prim=-y_2 \\
y_2\prim=y_1
\end{dcases}\]

\begin{enumerate}
    \item À l'aide d'une équation différentielle d'ordre 1 vérifiée par la fonction complexe \(y_1+\i y_2\). \\
    \item À l'aide d'une équation différentielle d'ordre 2 vérifée par la fonction réelle \(y_1\).
\end{enumerate}
\end{exo}

\begin{corr}
\note{À venir}
\end{corr}

\begin{exo}[Exercice 16]
Résoudre le système linéaire : \[\paren{E}~\begin{dcases}
y_1\prim=2y_1-y_2+\cos t \\
y_2\prim=y_1+2y_2+\sin t
\end{dcases}\]
\end{exo}

\begin{corr}
\note{À venir}
\end{corr}

\begin{exo}[Exercice 17]
Si \(f:\R\to\R\) est une fonction dérivable, on note \(\P{f}\) la propriété : \[\quantifs{\forall x\in\R}f\prim\paren{x}=2f\paren{-x}+x.\]

\begin{enumerate}
    \item Soit \(f:\R\to\R\) une fonction dérivable. Montrer que si \(\P{f}\) est vraie, alors \(f\) est deux fois dérivable et satisfait une certaine équation différentielle linéaire d'ordre 2. \\
    \item Résoudre l'équation différentielle obtenue. \\
    \item Déterminer l'ensemble des fonctions dérivables \(f:\R\to\R\) telles que \(\P{f}\) soit vraie.
\end{enumerate}
\end{exo}

\begin{corr}
\note{À venir}
\end{corr}

\chapter{Espaces vectoriels en dimension finie}

\minitoc

\begin{exo}[Exercice 1]
Si \(p\in\Ns\), on note \(E_p\) l'espace vectoriel des suites \(p\)-périodiques d'éléments de \(\K\) : \[E_p=\accol{\paren{x_n}_n\in\K^\N\tq\quantifs{\forall n\in\N}u_{n+p}=u_n}.\]

On note \(E\) l'espace vectoriel des suites périodiques d'éléments de \(\K\) : \[E=\accol{\paren{x_n}_n\in\K^\N\tq\quantifs{\exists p\in\Ns;\forall n\in\N}u_{n+p}=u_n}.\]

\begin{enumerate}
\item Soit \(p\in\Ns\). Quelle est la dimension de \(E_p\) ? \\

\item Quelle est la dimension de \(E\) ?
\end{enumerate}
\end{exo}

\begin{corr}
\note{À venir}
\end{corr}

\begin{exo}[Exercice 2, TPE MP 2018]
On pose : \[E=\accol{P\in\poly[\C]\tq\paren{X^4+1}P=P\paren{X^2}}.\]

\begin{enumerate}
\item Montrer que \(E\) est un sous-espace vectoriel de \(\poly[\C]\). \\

\item Si \(P\in E\excluant\accol{0}\), déterminer le degré de \(P\). \\

\item Donner un polynôme non-nul appartenant à \(E\). \\

\item Déterminer la dimension de \(E\).
\end{enumerate}
\end{exo}

\begin{corr}
\note{À venir}
\end{corr}

\begin{exo}[Exercice 3, relations de récurrence d'ordre 2]\thlabel{exo:relationsDeRécurrenceD'Ordre2}
Soient \(a,b,c\in\C\) tels que \(a\not=0\).

On note \(E\) le \(\C\)-espace vectoriel des suites \(\paren{u_n}_{n\in\N}\in\C^\N\) à coefficients dans \(\C\) vérifiant la relation de récurrence linéaire homogène d'ordre 2 à coefficients constants suivant : \[\quantifs{\forall n\in\N}au_{n+2}+bu_{n+1}+cu_n=0.\]

\begin{enumerate}
\item Montrer que l'application \[\fonction{\phi}{E}{\C^2}{\paren{u_n}_n}{\paren{u_0,u_1}}\] est un isomorphisme d'espaces vectoriels. \\

\item Quelle est la dimension de \(E\) ? \\

\item En déduire le théorème suivant (à retenir) :

\begin{itemize}
\item Si le polynôme \(aX^2+bX+c\) admet deux racines distinctes \(x,y\in\C\), alors \(E\) est l'ensemble des suites de la forme : \[\paren{\lambda x^n+\mu y^n}_{n\in\N}\] où \(\lambda,\mu\in\C\). \\

\item Si le polynôme \(aX^2+bX+c\) admet une racine double \(x\in\C\), alors \(E\) est l'ensemble des suites de la forme : \[\paren{\paren{\lambda n+\mu}x^n}_{n\in\N}\] où \(\lambda,\mu\in\C\).
\end{itemize}
\end{enumerate}
\end{exo}

\begin{corr}
\note{À venir}
\end{corr}

\begin{exo}[Exercice 4]
Donner un exemple d'espace vectoriel \(E\) et d'endomorphisme \(u\in\Lendo{E}\) tels que \(u\) soit inversible à droite mais pas à gauche (dans l'anneau \(\anneau{\Lendo{E}}[+][\rond]\)).
\end{exo}

\begin{corr}
\note{À venir}
\end{corr}

\begin{exo}[Exercice 5]
Soient \(E\) et \(F\) deux \(\K\)-espaces vectoriels de dimension finie.

\begin{enumerate}
\item Montrer qu'il existe une application linéaire injective \(u:E\to F\) si, et seulement si, \(\dim E\leq\dim F\). \\

\item Montrer qu'il existe une application linéaire surjective \(u:E\to F\) si, et seulement si, \(\dim E\geq\dim F\). \\

\item Montrer qu'il existe un isomorphisme (d'espaces vectoriels) \(u:E\to F\) si, et seulement si, \(\dim E=\dim F\).
\end{enumerate}
\end{exo}

\begin{corr}
\note{À venir}
\end{corr}

\begin{exo}[Exercice 6, endomorphismes de rang 1]
Soient \(E\) un \(\K\)-espace vectoriel et \(u\in\Lendo{E}\).

\begin{enumerate}[series=endoRG1]
\item Justifier que l'endomorphisme \(u\) est de rang 1 si, et seulement si, il s'écrit : \[\fonction{u}{E}{E}{x}{l\paren{x}a}\] où \(a\) est un vecteur non-nul de \(E\) et \(l\in E\etoile\) est une forme linéaire non-nulle. \\
\end{enumerate}

On suppose désormais que \(u\) est de rang 1.

\begin{enumerate}[resume=endoRG1]
\item Montrer : \(\quantifs{\exists\lambda\in\K}u^2=\lambda u\). \\

\item Montrer que les trois propositions suivantes sont équivalentes :

\begin{enumerate}
\item \(\quantifs{\exists\lambda\in\K\excluant\accol{0}}\lambda u\text{ est un projecteur}\) \\

\item \(E=\ker u\oplus\Im u\) \\

\item \(u^2\not=0\).
\end{enumerate}
\end{enumerate}
\end{exo}

\begin{corr}
\note{À venir}
\end{corr}

\begin{exo}[Exercice 7]
Soient \(E\) un \(\K\)-espace vectoriel et \(u\in\Lendo{E}\).

On suppose : \[\dim E=3\qquad\text{et}\qquad u^2=0.\]

Montrer : \(\rg u\leq1\).
\end{exo}

\begin{corr}
\note{À venir}
\end{corr}

\begin{exo}[Exercice 8]
Soient \(E\) un espace vectoriel de dimension finie et \(u,v\in\Lendo{E}\).

Montrer : \[\abs{\rg u-\rg v}\leq\rg\paren{u+v}\leq\rg u+\rg v.\]
\end{exo}

\begin{corr}
\note{À venir}
\end{corr}

\begin{exo}[Exercice 9]
Soient \(E\) un espace vectoriel de dimension finie et \(u,v\in\Lendo{E}\).

Montrer : \[\rg\paren{uv}\leq\min\accol{\rg u;\rg v}.\]
\end{exo}

\begin{corr}
\note{À venir}
\end{corr}

\begin{exo}[Exercice 10]
Soient \(u\in\L{\R^2}{\R^3}\) et \(v\in\L{\R^3}{\R^2}\).

On suppose que \(u\rond v\) est un projecteur de rang 2 de \(\R^3\).

Montrer que \(v\) est une surjection et que \(u\) est une injection puis que \(v\rond u=\id{\R^2}\).
\end{exo}

\begin{corr}
\note{À venir}
\end{corr}

\begin{exo}[Exercice 11]
Soient \(E\) un \(\K\)-espace vectoriel de dimension finie et \(u,v\in\Lendo{E}\).

Montrer l'équivalence : \[\ker u=\Im u\ssi\begin{dcases}
u^2=0 \\
\dim E=2\rg u
\end{dcases}\]
\end{exo}

\begin{corr}
\note{À venir}
\end{corr}

\begin{exo}[Exercice 12, formule de Grassmann]
Soient \(E\) un \(\K\)-espace vectoriel et \(F\) et \(G\) deux sous-espaces vectoriels de \(E\) de dimension finie.

\begin{enumerate}
\item Montrer qu'il existe \(a,b,c\in\N\) et une base de \(F+G\) : \[\paren{e_1,\dots,e_a,e_1\prim,\dots,e_b\prim,e_1\seconde,\dots,e_c\seconde}\] tels que \[\begin{dcases}
F\inter G=\Vect{e_1,\dots,e_a} \\
F=\Vect{x_1,\dots,e_a,e_1\prim,\dots,e_b\prim} \\
G=\Vect{e_1,\dots,e_a,e_1\seconde,\dots,e_b\seconde}
\end{dcases}\]

\item En déduire une nouvelle démonstration de la formule de Grassmann.
\end{enumerate}
\end{exo}

\begin{corr}
\note{À venir}
\end{corr}

\begin{exo}[Exercice 13, Centrale 2008 (extrait)]
Soient \(E\) et \(F\) deux espaces vectoriels de dimension finie, \(f\in\L{E}{F}\) et \(G\) un sous-espace vectoriel de \(F\).

Montrer : \[\dim f\inv\paren{G}=\dim E-\rg f+\dim\paren{\Im f\inter G}.\]
\end{exo}

\begin{corr}
\note{À venir}
\end{corr}

\begin{exo}[Exercice 14]
Soient \(E\) un espace vectoriel de dimension finie et \(u,v\in\Lendo{E}\).

Montrer : \[\rg\paren{u+v}=\rg u+\rg v\ssi\begin{dcases}
\Im u\inter\Im v=\accol{0_E} \\
\ker u+\ker v=E
\end{dcases}\]
\end{exo}

\begin{corr}
\note{À venir}
\end{corr}

\begin{exo}[Exercice 15]
Soient \(E\) un espace vectoriel de dimension finie et \(l_1,\dots,l_m\in E\etoile\).

On considère l'application : \[\fonction{\phi}{E}{\K^m}{x}{\paren{l_1\paren{x},\dots,l_m\paren{x}}}\]

\begin{enumerate}
\item Montrer que la famille \(\paren{l_1,\dots,l_m}\) est une famille libre si, et seulement si, l'application \(\phi\) est surjective. \\

\item Montrer : \[\rg\phi=\rg\paren{l_1,\dots,l_m}.\]

\textit{Indication :} considérer une base \(\paren{l_{i_1},\dots,l_{i_r}}\) de \(\Vect{l_1,\dots,l_m}\). \\

\item En déduire que la famille \(\paren{l_1,\dots,l_m}\) engendre \(E\etoile\) si, et seulement si, l'application \(\phi\) est injective.
\end{enumerate}
\end{exo}

\begin{corr}
\note{À venir}
\end{corr}

\begin{exo}[Exercice 16]
Soient \(E\) un espace vectoriel de dimension finie et \(u\in\Lendo{E}\).

Montrer que les propositions suivantes sont équivalentes :

\begin{enumerate}
\item \(\Im u^2=\Im u\) \\

\item \(\ker u^2=\ker u\) \\

\item \(E=\ker u\oplus\Im u\).
\end{enumerate}
\end{exo}

\begin{corr}
\note{À venir}
\end{corr}

\begin{exo}[Exercice 17]
Soient \(E\) un espace vectoriel de dimension finie et \(u,v\in\Lendo{E}\).

On suppose : \[u\rond v=0\qquad\text{et}\qquad u+v\in\GL{}[E].\]

Montrer : \[\rg u+\rg v=\dim E\qquad\text{et}\qquad\ker u\oplus\ker v=E.\]
\end{exo}

\begin{corr}
\note{À venir}
\end{corr}

\chapter{Matrices I}

\minitoc

\begin{exo}[Exercice 1]
Soient \(a,b,c\in\K\excluant\accol{0}\).

On pose \[A=\begin{pmatrix}
a & 0 & 0 \\
0 & b & 0 \\
0 & 0 & c
\end{pmatrix}\qquad\text{et}\qquad B=\begin{pmatrix}
1 & a \\
0 & 1
\end{pmatrix}.\]

Calculer \(A^n\) et \(B^n\) pour tout \(n\in\Z\).
\end{exo}

\begin{corr}
\note{À venir}
\end{corr}

\begin{exo}[Exercice 2]
Soit \(n\in\Ns\).

L'ensemble des matrices symétriques de \(\M{n}\) est-il stable par le produit matriciel ? Qu'en est-il de l'ensemble des matrices antisymétriques ?
\end{exo}

\begin{corr}
\note{À venir}
\end{corr}

\begin{exo}[Exercice 3]
Dire pour chaque système quel est son rang et ce qu'on le peut en déduire sur l'espace \(\fami{S}\) de ses solutions. Le résoudre ensuite.

\begin{enumerate}
\item Système d'inconnue \(\paren{x,y,z}\in\R^3\) : \[\begin{dcases}
x+2z=1 \\
x+y+z=2
\end{dcases}\]

\item Système d'inconnue \(\paren{a,b,c,d}\in\R^4\), où \(\lambda\) est un paramètre réel fixé : \[\begin{dcases}
a+b+c+d=0 \\
a-b+2c-d=1 \\
3a+b+4c+d=\lambda
\end{dcases}\]
\end{enumerate}
\end{exo}

\begin{corr}
\note{À venir}
\end{corr}

\begin{exo}[Exercice 4]
Soient \(n,p\in\Ns\) et \(A\in\M{np}[\R]\).

On considère les équations matricielles d'inconnue \(X\in\M{p1}[\R]\) : \[\paren{E_1}~AX=\begin{pmatrix}1\\0\\0\end{pmatrix}\qquad\text{et}\qquad\paren{E_2}~AX=\begin{pmatrix}0\\1\\0\end{pmatrix}.\]

On suppose que \(\paren{E_1}\) n'admet aucune solution et que \(\paren{E_2}\) admet une unique solution.

Quelles sont les valeurs possibles pour les entiers \(n\) et \(p\) ?

Donner, pour chaque possibilité, un exemple de matrice \(A\).
\end{exo}

\begin{corr}
\note{À venir}
\end{corr}

\begin{exo}[Exercice 5]
Soit \(\paren{S}\) un système linéaire de \(n\) équations à \(p\) inconnues. On note \(r\) son rang.

Justifier les propositions suivantes :

\begin{enumerate}
\item Si \(r=p\) alors \(\paren{S}\) admet au plus une solution. \\

\item Si \(r=n\) alors \(\paren{S}\) admet au moins une solution.
\end{enumerate}
\end{exo}

\begin{corr}
\note{À venir}
\end{corr}

\begin{exo}[Exercice 6]
Montrer que les matrices \(\begin{pmatrix}
1 & 1 \\
0 & 1
\end{pmatrix}\) et \(\begin{pmatrix}
1 & 0 \\
1 & 1
\end{pmatrix}\) sont semblables.
\end{exo}

\begin{corr}
\note{À venir}
\end{corr}

\begin{exo}[Exercice 7]
Calculer les inverses des matrices suivantes : \[A=\begin{pmatrix}
2 & 5 \\
3 & 7
\end{pmatrix}\qquad B=\begin{pmatrix}
1 & 2 \\
2 & 4
\end{pmatrix}\qquad C=\begin{pmatrix}
0 & 7 & 8 \\
0 & 1 & 1 \\
1 & 2 & 3
\end{pmatrix}\qquad D=\begin{pmatrix}
1 & 0 & 2 \\
1 & 3 & 1 \\
1 & 7 & 0
\end{pmatrix}\qquad E=\begin{pmatrix}
1 & 0 & 2 & 0 \\
3 & 1 & 1 & 0 \\
0 & 1 & 2 & 1 \\
2 & 2 & 2 & 1
\end{pmatrix}\]
\end{exo}

\begin{corr}
\note{À venir}
\end{corr}

\begin{exo}[Exercice 8]\thlabel{exo:matrice1...n}
Soit \(n\in\Ns\).

On pose : \[A=\begin{pmatrix}
1 & \dots & \dots & 1 \\
0 & \ddots &  & \vdots \\
\vdots & \ddots & \ddots & \vdots \\
0 & \dots & 0 & 1
\end{pmatrix}\in\M{n}\qquad\text{et}\qquad B=\begin{pmatrix}
1 & \dots & \dots & \dots & 1 \\
\vdots & 2 & \dots & \dots & 2 \\
\vdots & \vdots & 3 & \dots & 3 \\
\vdots & \vdots & \vdots & \ddots & \vdots \\
1 & 2 & 3 & \dots & n
\end{pmatrix}\]

\begin{enumerate}
\item Déterminer l'inverse de \(A\). \\

\item Calculer \(\trans{A}A\). \\

\item En déduire l'inverse de \(B\).
\end{enumerate}
\end{exo}

\begin{corr}
\note{À venir}
\end{corr}

\begin{exo}[Exercice 9]
Soient \(a,b,c,d,e,f,g,h,i\in\K\) et \(\lambda\in\K\excluant\accol{0}\).

Montrer que les matrices \(\begin{pmatrix}
a & b & c \\
d & e & f \\
g & h & i
\end{pmatrix}\) et \(\begin{pmatrix}
a & b\lambda\inv & c\lambda^{-2} \\
d\lambda & e & f\lambda\inv \\
g\lambda^2 & h\lambda & i
\end{pmatrix}\) sont semblables.
\end{exo}

\begin{corr}
\note{À venir}
\end{corr}

\begin{exo}[Exercice 10]
Montrer que l'application \[\fonction{\phi}{\M{n}[\R]}{\M{n}[\R]\etoile}{A}{B\mapsto\tr\paren{AB}}\] est un isomorphisme de l'espace vectoriel \(\M{n}[\R]\) vers son dual.
\end{exo}

\begin{corr}
\note{À venir}
\end{corr}

\begin{exo}[Exercice 11]
On suppose que le corps \(\K\) est fini et on note \(q\) son cardinal : \[q=\Card\K<\pinf.\]

Soit \(n\in\Ns\).

Déterminer le cardinal des ensembles suivants :

\begin{enumerate}
\item \(\K^n\) \\

\item \(\M{n}\) \\

\item \(\GL{n}\)
\end{enumerate}

\textit{Indications pour le (3) :}

\begin{itemize}
\item Commencer par \(n=2\) ou \(3\). \\

\item Une matrice carrée est inversible si, et seulement si, la famille de ses vecteurs colonnes est libre. Compter les matrices inversibles de taille \(n\) revient donc à compter les familles libres de \(n\) vecteurs de \(\K^n\).
\end{itemize}
\end{exo}

\begin{corr}
\note{À venir}
\end{corr}

\begin{exo}[Exercice 12]
Soient \(n\in\Ns\) et \(f:\M{n}\to\K\) une fonction non-constante telle que : \[\quantifs{\forall A,B\in\M{n}}f\paren{AB}=f\paren{A}f\paren{B}.\]

Montrer que pour toute matrice \(A\in\M{n}\), on a l'équivalence : \[A\in\GL{n}\ssi f\paren{A}\not=0.\]
\end{exo}

\begin{corr}
\note{À venir}
\end{corr}

\chapter{Matrices II}

\minitoc

\begin{exo}
On note \(E\) un espace vectoriel, \(\fami{B}\) une base de \(E\) et \(u\) un endomorphisme de \(E\).

Donner, dans chaque cas, la matrice \(\Mat{u}\) de \(u\) dans \(\fami{B}\).

\begin{enumerate}
\item \(E=\C\), vu comme un \(\R\)-espace vectoriel ; \(\fami{B}=\paren{1,\i}\)

\(\fonction{u}{\C}{\C}{z}{\paren{a+\i b}z}\) où \(a,b\in\R\). \\

\item \(E=\C\), vu comme un \(\R\)-espace vectoriel ; \(\fami{B}=\paren{1,\i}\)

\(\fonction{u}{\C}{\C}{z}{\i\lambda z+2\conj{z}}\) où \(\lambda\in\R\). \\

\item \(E=\C\), vu comme un \(\C\)-espace vectoriel ; \(\fami{B}=\paren{1}\)

\(\fonction{u}{\C}{\C}{z}{\paren{a+\i b}z}\) où \(a,b\in\R\). \\

\item \(E=\C\), vu comme un \(\C\)-espace vectoriel ; \(\fami{B}=\paren{1}\)

\(\fonction{u}{\C}{\C}{z}{\i\lambda z+2\conj{z}}\) où \(\lambda\in\R\). \\

\item \(E=\Vect{\cos,\sin}\), sous-espace vectoriel de \(\ensclasse{\infty}{\R}{\R}\) ; \(\fami{B}=\paren{\cos,\sin}\)

\(\fonction{u}{E}{E}{f}{2f+f\prim}\) \\

\item \(E=\M{2}\) ; \(\fami{B}=\paren{E_{11},E_{21},E_{12},E_{22}}\)

\(\fonction{u}{E}{E}{M}{AM}\) où \(A=\begin{pmatrix}
a & b \\
c & d
\end{pmatrix}\in E\) est une matrice fixée. \\

\item \(E=\M{2}\) ; \(\fami{B}=\paren{E_{11},E_{21},E_{12},E_{22}}\)

\(\fonction{u}{E}{E}{M}{MA}\) où \(A=\begin{pmatrix}
a & b \\
c & d
\end{pmatrix}\in E\) est une matrice fixée. \\

\item \(E=\M{2}\) ; \(\fami{B}=\paren{E_{11},E_{12},E_{21},E_{22}}\)

\(\fonction{u}{E}{E}{M}{MA}\) où \(A=\begin{pmatrix}
a & b \\
c & d
\end{pmatrix}\in E\) est une matrice fixée. \\

\item \(E=\M{n}\) ; \(\fami{B}\) à choisir (judicieusement)

\(\fonction{u}{E}{E}{M}{AM}\) où \(A\in E\) est une matrice fixée. \\

\item \(E=\M{n}\) ; \(\fami{B}\) à choisir (judicieusement)

\(\fonction{u}{E}{E}{M}{MA}\) où \(A\in E\) est une matrice fixée.
\end{enumerate}
\end{exo}

\begin{corr}
\note{À venir}
\end{corr}

\begin{exo}
On pose \[\fonction{u}{\polydeg[\R]{n}}{\polydeg[\R]{n}}{P}{X^2P\seconde-6P}\]

Calculer le rang et la trace de \(u\) :

\begin{enumerate}
\item si \(n=3\). \\

\item si \(n\geq3\).
\end{enumerate}
\end{exo}

\begin{corr}
\note{À venir}
\end{corr}

\begin{exo}
On pose \[\fonction{u}{\M{n}[\R]}{\M{n}[\R]}{A}{\trans{A}}\]

Calculer le rang et la trace de \(u\) :

\begin{enumerate}
\item si \(n=2\). \\

\item si \(n\in\Ns\).
\end{enumerate}
\end{exo}

\begin{corr}
\note{À venir}
\end{corr}

\begin{exo}
Soient \(E\) un espace vectoriel de dimension finie non-nulle et \(\fami{B}\) et \(\fami{B}\prim\) deux bases de \(E\).

Quelle est la relation entre la matrice de passage \(\pass{\fami{B}}{\fami{B}\prim}\) et la matrice \(\Mat[\fami{B},\fami{B}\prim]{\id{E}}\) de l'application linéaire \(\id{E}\) dans les bases \(\fami{B}\) et \(\fami{B}\prim\) ?
\end{exo}

\begin{corr}
\note{À venir}
\end{corr}

\begin{exo}
Soient \(E\) et \(F\) deux espaces vectoriels de dimension finie et \(u\in\L{E}{F}\) une application linéaire de rang \(r\in\N\).

Montrer que \(u\) est la somme de \(r\) applications linéaires de rang 1.
\end{exo}

\begin{corr}
\note{À venir}
\end{corr}

\begin{exo}
Soient \(E\) un espace vectoriel de dimension finie non-nulle et \(u\in\Lendo{E}\) tel que \(u^2=0\).

\begin{enumerate}
\item Quelle inclusion est vraie entre \(\ker u\) et \(\Im u\) ? \\

\item Montrer que \(u\) est de trace nulle.

\textit{Indication :} construire une base de \(E\) adaptée à la situation.

\textit{Remarque :} vous montrerez en deuxième année que tout endomorphisme nilpotent de \(E\) est de trace nulle.
\end{enumerate}
\end{exo}

\begin{corr}
\note{À venir}
\end{corr}

\begin{exo}[Méthode à retenir]
On pose : \[H=\accol{\tcoords{x_1}{x_2}{x_3}\in\C^3\tq x_1+2x_3=0}\qquad\text{et}\qquad D=\Vect{\tcoords{1}{-1}{0}}.\]

\begin{enumerate}
\item Justifier que \(H\) et \(D\) sont deux sous-espaces vectoriels supplémentaires dans \(\C^3\). \\

\item On note \(p_H\) la projection sur \(H\) parallèlement à \(D\) et \(p_D\) la projection sur \(D\) parallèlement à \(H\).

Déterminer les matrices de \(p_H\) et \(p_D\) dans la base canonique.
\end{enumerate}
\end{exo}

\begin{corr}
\note{À venir}
\end{corr}

\begin{exo}[Méthode également à retenir]
On pose \(E=\R^4\).

On note \(F\) et \(G\) les ensembles solutions respectifs des systèmes \[\begin{dcases}
a+b=0 \\
b+c=0
\end{dcases}\qquad\text{et}\qquad\begin{dcases}
a+b+c=0 \\
b+c+d=0
\end{dcases}\]

\begin{enumerate}[series=mat2]
\item Montrer que \(F\) et \(G\) sont supplémentaires dans \(\R^4\). \\

\item Déterminer une base \(\fami{B}_F\) de \(F\) et une base \(\fami{B}_G\) de \(G\).
\end{enumerate}

On note \(\fami{B}_0\) la base canonique de \(\R^4\), \(\fami{B}\) la base de \(\R^4\) obtenue en juxtaposant \(\fami{B}_F\) et \(\fami{B}_G\), \(p\) la projection sur \(F\) parallèlement à \(G\) et \(s\) la symétrie par rapport à \(F\) parallèlement à \(G\).

\begin{enumerate}[resume=mat2]
\item Donner la matrice de \(p\) dans la base \(\fami{B}\). \\

\item En déduire la matrice de \(p\) dans la base \(\fami{B}_0\). \\

\item En déduire la matrice de \(s\) dans la base \(\fami{B}_0\).
\end{enumerate}

NB : il faut savoir refaire parfaitement la question (4) sans les questions intermédiaires (2) et (3).
\end{exo}

\begin{corr}
\note{À venir}
\end{corr}

\begin{exo}
Soit \(M\in\M{n}[\R]\) une matrice de trace différente de \(2\).

On pose : \[E=\accol{X\in\M{n}[\R]\tq X+\trans{X}=\paren{\tr X}M}.\]

\begin{enumerate}
\item Montrer que \(E\) est un \(\R\)-espace vectoriel et déterminer sa dimension. \\

\item Que se passe-t-il si \(\tr M=2\) ?
\end{enumerate}
\end{exo}

\begin{corr}
\note{À venir}
\end{corr}

\begin{exo}
Soient \(E\) un \(\R\)-espace vectoriel de dimension finie, \(\lambda\in\Rs\) et \(u\in\Lendo{E}\).

On suppose \(u^2=\lambda u\).

Donner une relation entre \(\tr u\) et \(\rg u\).
\end{exo}

\begin{corr}
\note{À venir}
\end{corr}

\chapter{Relations de comparaison, développements limités}

\minitoc

\begin{exo}[Exercice 1]
Classer les expressions suivantes de la plus \guillemets{petite} à la plus \guillemets{grande} quand \(x\) tend vers \(\pinf\) (on pourra exprimer la réponse à l'aide des notations de Hardy) : \[x^2\ln x\qquad\dfrac{x^3}{\ln x}\qquad x\ln^7x\qquad\dfrac{\e{x}}{x\ln x}\qquad\dfrac{\e{x^2}}{x^7\ln x}\qquad\dfrac{\e{2x}}{x\ln x}\qquad\e{\sqrt{x}}\]
\end{exo}

\begin{corr}
\note{À venir}
\end{corr}

\begin{exo}[Exercice 2]
\begin{enumerate}
\item Simplifier les expressions suivantes : \[A\underset{x\to\pinf}{=}\paren{x\ln x+\sqrt{x}\ln^2x+O\paren{x}+\sin x+\sqrt{x}}x+o\paren{x^2\ln x}\] et \[B\underset{x\to0}{=}O\paren{x^3\ln x}+x^3\ln^2x+2x^3+\dfrac{x\ln^2x}{1-x}+o\paren{x^3}+O\paren{x^3}\] en utilisant pour \(B\) le développement limité en \(0\) de \(x\mapsto\dfrac{1}{1-x}\). \\

\item Donner un équivalent de \(A\) et \(B\).
\end{enumerate}
\end{exo}

\begin{corr}
\note{À venir}
\end{corr}

\begin{exo}[Exercice 3]
Donner un équivalent des fonctions suivantes :

\begin{enumerate}
\item \(f:x\mapsto\floor{x}\) quand \(x\) tend vers \(\pinf\). \\

\item \(g:x\mapsto\dfrac{x^{\alpha}}{1+x^{\beta}}\) quand \(x\) tend vers \(\pinf\) et quand \(x\) tend vers \(0^+\). \\

\item \(h:x\mapsto\int_0^x\floor{t}\odif{t}\) quand \(x\) tend vers \(\pinf\). \\

\item \(k:x\mapsto\int_x^{x+1}\e{t}\ln t\odif{t}\) quand \(x\) tend vers \(\pinf\).
\end{enumerate}
\end{exo}

\begin{corr}
\note{À venir}
\end{corr}

\begin{exo}[Exercice 4]
Pour tout \(n\in\N\), on note \(u_n\) le nombre de zéros consécutifs situés à droite de l'écriture en base 10 de \(n!\).

Par exemple, l'entier \(20!=2432902008176640000\) se termine par quatre zéros donc \(u_{20}=4\).

Donner un équivalent de \(u_n\) quand \(n\) tend vers \(\pinf\).
\end{exo}

\begin{corr}
Si \(N\in\Ns\), le nombre de zéros à droite de \(N\) est \[\min\accol{\valp{2}{n};\valp{5}{n}}.\]

Soit \(p\in\prem\).

Calculons \[\begin{aligned}
\valp{p}{n!}&=\floor{\dfrac{n}{p}}+\floor{\dfrac{n}{p^2}}+\floor{\dfrac{n}{p^3}}+\dots \\
&=\sum_{\alpha=1}^{\pinf}\floor{\dfrac{n}{p^{\alpha}}}
\end{aligned}\]

Donc \(\valp{2}{n!}\geq\valp{5}{n!}\).

Donc \(u_n=\valp{5}{n!}=\sum_{\alpha=1}^{\pinf}\floor{\dfrac{n}{5^{\alpha}}}\).

On a \[\begin{aligned}
\quantifs{\forall\alpha\in\Ns}\floor{\dfrac{n}{5^{\alpha}}}=0&\ssi\dfrac{n}{5^{\alpha}}<1 \\
&\ssi n<5^{\alpha} \\
&\ssi\alpha>\log_5n \\
&\ssi\alpha>\floor{\log_5n}
\end{aligned}\]

Donc \[\begin{aligned}
u_n&=\sum_{\alpha=1}^{\floor{\log_5n}}\floor{\dfrac{n}{5^{\alpha}}} \\
&\leq\sum_{\alpha=1}^{\floor{\log_5n}}\dfrac{n}{5^{\alpha}} \\
&=n\dfrac{\frac{1}{5}-\frac{1}{5^{\floor{\log_5n}+1}}}{1-\frac{1}{5}} \\
&\leq\dfrac{n}{4}
\end{aligned}\]

De plus, on a \[\begin{WithArrows}
u_n&\geq\sum_{\alpha=1}^{\floor{\log_5n}}\paren{\dfrac{n}{5^{\alpha}}-1} \\
&=n\dfrac{\frac{1}{5}-\frac{1}{5^{\floor{\log_5n}+1}}}{1-\frac{1}{5}}-\floor{\log_5n} \\
&=\dfrac{n}{4}\paren{1-\dfrac{1}{5^{\floor{\log_5n}+1}}}-\floor{\log_5n} \\
&\geq\dfrac{n}{4}\paren{1-\dfrac{1}{5^{\floor{\log_5n}+1}}}-\log_5n \Arrow{\(n\dfrac{1+o\paren{1}}{4}+o\paren{n}=\dfrac{n}{4}+o\paren{n}\)} \\
&\underset{n\to\pinf}{\sim}\dfrac{n}{4}
\end{WithArrows}\]

Donc \(u_n\sim\dfrac{n}{4}\).
\end{corr}

\begin{exo}[Exercice 5]\thlabel{exo:fonctionPaireDLAvecTermesPairs}
Soient \(I\) un intervalle de \(\R\) centré en \(0\), \(f:I\to\R\) et \(n\in\N\).

On suppose que \(f\) admet un développement limité à l'ordre \(n\) en \(0\) : \[f\paren{x}\underset{x\to0}{=}a_0+a_1x+a_2x^2+\dots+a_nx^n+o\paren{x^n}.\]

\begin{enumerate}
\item Montrer que si la fonction \(f\) est paire, alors les termes de degré impair de son développement limité sont tous nuls : \[0=a_1=a_3=a_5=\dots\]

\item Montrer que si la fonction \(f\) est impaire, alors les termes de degré pair de son développement limité sont tous nuls : \[0=a_0=a_2=a_4=\dots\]
\end{enumerate}
\end{exo}

\begin{corr}
\note{À venir}
\end{corr}

\begin{exo}[Exercice 6]
Calculer les développements limités des fonctions suivantes en \(0\) :

\begin{enumerate}
\item \(x\mapsto\sqrt{1+\sin x}\) à l'ordre 4. \\

\item \(x\mapsto\sqrt{1+\cos x}\) à l'ordre 4. \\

\item \(x\mapsto\ln\paren{\e{x}+\cos x}\) à l'ordre 3. \\

\item \(x\mapsto\ln\paren{\cos\paren{2x}}\) à l'ordre 7. \\

\item \(x\mapsto\dfrac{1}{\sqrt{1-x^2}}\) à l'ordre \(n\in\N\). \\

\item \(\Arcsin\) à l'ordre \(n\in\N\). \\

\item \(\Arccos\) à l'ordre \(n\in\N\).
\end{enumerate}

\textit{Indication :} pour le (5), exprimer pour tout \(k\in\N\) le quotient \(\dfrac{\alpha\paren{\alpha-1}\dots\paren{\alpha-k+1}}{k!}\) avec \(\alpha=\dfrac{-1}{2}\) à l'aide de factorielles.
\end{exo}

\begin{corr}
\note{À venir}
\end{corr}

\begin{exo}[Exercice 7]
Calculer les développements limités des fonctions suivantes :

\begin{enumerate}
\item \(x\mapsto\ln\paren{1+x^2}\) à l'ordre 3 en \(2\). \\

\item \(x\mapsto\sin x\cos\paren{3x}\) à l'ordre 2 en \(\dfrac{\pi}{3}\).
\end{enumerate}
\end{exo}

\begin{corr}
\note{À venir}
\end{corr}

\begin{exo}[Exercice 8]
Calculer les limites suivantes :

\begin{enumerate}
\item \[\lim_{\substack{x\to0 \\ x\not=0}}\dfrac{\Arctan\paren{2x}-2\Arctan x}{x^3}\]

\item \[\lim_{x\to\pinf}\paren{\dfrac{\ln\paren{1+x}}{\ln x}}^{x\ln x}\]

\item \[\lim_{n\to\pinf}\paren{1+\dfrac{\lambda}{n}}^n\text{ où \(\lambda\) est un réel fixé}\]

\item \[\lim_{n\to\pinf}\paren{3\sqrt[n]{2}-2\sqrt[n]{3}}^n\]

\item \[\lim_{\substack{x\to0 \\ x\not=0}}\dfrac{\Arctan\paren{x^2-x^2\cos x}}{\paren{1-\sqrt{\cos x}}\ln\frac{\sin x}{x}}\]

\item \[\lim_{\substack{x\to0 \\ x\not=0}}\dfrac{x\cos x-\sin x}{\cos x-1}\]

\item \[\lim_{x\to0^+}\paren{\dfrac{x\cos x-\sin x}{\cos x-1}}^x\]

\item \[\lim_{\substack{x\to0 \\ x\not=0}}\paren{\dfrac{2^x+3^x}{2}}^{\frac{1}{x}}\]

\item \[\lim_{\substack{x\to\frac{\pi}{3} \\ x\not=\frac{\pi}{3}}}\dfrac{3\cos^2x-\sin^2\paren{2x}}{\tan\paren{4x}-\sqrt{3}}\]
\end{enumerate}
\end{exo}

\begin{corr}
\note{À venir}
\end{corr}

\begin{exo}[Exercice 9]
\begin{enumerate}
\item Montrer que l'application \[\fonction{f}{\R}{\R}{x}{x\e{x^2}}\] est une bijection de \(\R\) vers \(\R\). On note \(f\inv\) sa bijection réciproque. \\

\item Montrer que les fonctions \(f\) et \(f\inv\) sont impaires. \\

\item Déterminer le développement limité à l'ordre 5 de \(f\inv\) en \(0\).

On pensera à utiliser l'\thref{exo:fonctionPaireDLAvecTermesPairs} pour alléger les calculs.
\end{enumerate}
\end{exo}

\begin{corr}
\note{À venir}
\end{corr}

\begin{exo}[Exercice 10]
On pose \[\fonction{f}{\R}{\R}{x}{\begin{dcases}
0 &\text{si }x=0 \\
\dfrac{x^2}{\e{x}-\e{-x}} &\text{sinon}
\end{dcases}}\]

\begin{enumerate}
\item Déterminer le développement limité de \(f\) à l'ordre 3 en \(0\). \\

\item Montrer que \(f\) est dérivable en \(0\) et calculer \(f\prim\paren{0}\). \\

\item Déterminer la position relative du graphe de \(f\) par rapport à sa tangente en \(0\).
\end{enumerate}
\end{exo}

\begin{corr}
\note{À venir}
\end{corr}

\begin{exo}[Exercice 11]
Soit \(\lambda\in\R\).

Déterminer un développement asymptotique de la suite de terme général \[u_n=\paren{1+\dfrac{\lambda}{n}}^n\] à la précision \(\dfrac{1}{n^2}\).
\end{exo}

\begin{corr}
\note{À venir}
\end{corr}

\begin{exo}[Exercice 12]
On pose : \[\fonction{f}{\Rs}{\R}{x}{\paren{x+1}\exp\paren{\dfrac{1}{x}}}\]

Déterminer un développement asymptotique de \(f\) en \(\pinf\) à la précision \(\dfrac{1}{x}\).

En déduire que le graphe de \(f\) admet une asymptote en \(\pinf\) et sa position par rapport à son asymptote (au voisinage de \(\pinf\)).
\end{exo}

\begin{corr}
\note{À venir}
\end{corr}

\begin{exo}[Exercice 13]
Trouver un équivalent, quand \(x\) tend vers \(0^+\), de : \[x^{\sin x}-\sin^xx.\]
\end{exo}

\begin{corr}
\note{À venir}
\end{corr}

\begin{exo}[Exercice 14]
Montrer qu'on a, quand \(x\) tend vers \(1\) : \[\Arccos x\sim\sqrt{2\paren{1-x}}.\]
\end{exo}

\begin{corr}
\note{À venir}
\end{corr}

\begin{exo}[Exercice 15]
\begin{enumerate}
\item Montrer : \[\quantifs{\forall n\in\N;\exists!u_n\in\R}u_n\e{nu_n}=1.\] On obtient ainsi une suite de réels \(\paren{u_n}_{n\in\N}\), que l'on étudie dans la suite. \\

\item Étudier la limite de la suite \(\paren{u_n}_n\). \\

\item Donner un équivalent de \(u_n\) quand \(n\) tend vers \(\pinf\).
\end{enumerate}
\end{exo}

\begin{corr}
\note{À venir}
\end{corr}

\begin{exo}[Exercice 16, oral 2016]
\begin{enumerate}
\item Montrer, pour tout \(n\in\N\), l'existence et l'unicité d'un réel \(x_n\) tel que : \[x_n-\e{-x_n}=n.\]

\item Montrer : \[\quantifs{\forall n\in\N}n\leq x_n\leq n+1.\]

\item En déduire un équivalent de \(x_n\), puis un développement asymptotique à deux termes de \(x_n\) quand \(n\) tend vers \(\pinf\). \\

\item Donner un développement asymptotique à 5 termes de \(x_n\) quand \(n\) tend vers \(\pinf\).
\end{enumerate}

NB : la question (4) est ajoutée, elle n'a pas été posée à l'oral dont provient l'exercice.
\end{exo}

\begin{corr}
\note{À venir}
\end{corr}

\begin{exo}[Exercice 17]
On rappelle la formule du multinôme\footnote{La formule est énoncée ici pour les nombres complexes ; elle est en fait valable dans tout anneau si les éléments \(z_1,\dots,z_r\) commutent deux à deux.}, qui généralise la formule du binôme de Newton (obtenue quand \(r=2\)) : \[\quantifs{\forall n,r\in\N;\forall z_1,\dots,z_r\in\C}\paren{z_1+\dots+z_r}^n=\sum_{\alpha_1+\dots+\alpha_r=n}\dfrac{n!}{\alpha_1!\dots\alpha_r!}z_1^{\alpha_1}\dots z_r^{\alpha_r}\] (on sous-entend pour alléger la formule que les \(\alpha_i\) sont des entiers naturels).

Soient \(I\) et \(J\) des intervalles de \(\R\) contenant au moins deux points, \(f\in\ensclasse{n}{I}{J}\), \(g\in\ensclasse{n}{J}{\R}\) et \(a\in I\).

Montrer la formule de Faà di Bruno : \[\paren{g\rond f}\deriv{n}\paren{a}=\sum_{\alpha_1+2\alpha_2+\dots+n\alpha_n=n}\dfrac{n!}{\paren{1!}^{\alpha_1}\dots\paren{n!}^{\alpha_n}\alpha_1!\dots\alpha_n!}g\deriv{\alpha_1+\dots+\alpha_n}\paren{f\paren{a}}f\deriv{1}\paren{a}^{\alpha_1}\dots f\deriv{n}\paren{a}^{\alpha_n}\] qui donne la dérivée \(n\)-ème de \(g\rond f\) en fonction des dérivées successives de \(f\) et \(g\).
\end{exo}

\begin{corr}
\note{À venir}
\end{corr}

\begin{rem}
Les deux derniers exercices sont assez calculatoires ; on pourra s'aider d'une calculatrice ou d'un ordinateur pour obtenir les développements limités utiles.
\end{rem}

\begin{exo}[Exercice 18]
On considère le \(\R\)-espace vectoriel \(\ensclasse{\infty}{\R}{\R}\) et le vecteur \(f=\sin\).

La famille \(\paren{f,f\rond f,f\rond f\rond f}\) est-elle libre ?
\end{exo}

\begin{corr}
\note{À venir}
\end{corr}

\begin{exo}[Exercice 19]
Trouver un équivalent simple quand \(x\) tend vers \(0\) de la fonction : \[x\mapsto\tan\paren{\sin x}-\sin\paren{\tan x}.\]
\end{exo}

\begin{corr}
\note{À venir}
\end{corr}

\chapter{Groupe symétrique}

\minitoc

\begin{exo}[Exercice 1]
On pose \[\sigma_1=\permu{1;2;3;4;5}{3;5;2;4;1}\quad\sigma_2=\permu{1;2;3;4;5}{2;1;5;4;3}\quad\sigma_3=\permu{1;2;3;4;5;6;7}{7;6;5;4;3;2;1}\quad\sigma_4=\permu{1;2;3;4;5;6;7}{5;4;6;3;7;2;1}\]

Pour tout \(k\in\interventierii{1}{4}\) :

\begin{itemize}
\item Décomposer \(\sigma_k\) en produit de cycles à supports disjoints. \\

\item Calculer la signature de \(\sigma_k\). \\

\item Calculer \(\sigma_k^2\), \(\sigma_k\inv\) et \(\sigma_k^{2023}\).
\end{itemize}
\end{exo}

\begin{corr}
\note{À venir}
\end{corr}

\begin{exo}[Exercice 2, bon à retenir]
Soient \(n,k\in\N\) tels que \(2\leq k\leq n\).

Soient \(\sigma\in\S{n}\) et un \(k\)-cycle \[c=\cycle{a_1;a_2;a_3;\dots;a_k}\in\S{n}\] (avec \(a_1,\dots,a_k\in\interventierii{1}{n}\) deux à deux distincts).

Calculer \[\sigma c\sigma\inv.\]
\end{exo}

\begin{corr}
\note{À venir}
\end{corr}

\begin{exo}[Exercice 3]
Soit \(n\in\interventierie{2}{\pinf}\).

\begin{enumerate}
\item Montrer que si \(\tau,\tau\prim\in\S{n}\) sont des transpositions alors \[\quantifs{\exists\sigma\in\S{n}}\sigma\tau\sigma\inv=\tau\prim\] (on dit que les transpositions \(\tau\) et \(\tau\prim\) sont conjuguées dans \(\S{n}\)). \\

\item Quels sont les morphismes de groupes de \(\S{n}\) vers \(\accol{-1;1}\) ?
\end{enumerate}
\end{exo}

\begin{corr}
\note{À venir}
\end{corr}

\begin{exo}[Exercice 4]
Soient \(n\in\interventierie{3}{\pinf}\) et \(\sigma\in\S{n}\).

On suppose que \(\sigma\) commute avec toutes les permutations de \(\S{n}\).

Montrer \[\sigma=\id{}.\]
\end{exo}

\begin{corr}
\note{À venir}
\end{corr}

\begin{exo}[Exercice 5]
Soit \(n\in\interventierie{2}{\pinf}\).

On considère le \(n\)-cycle \[c=\cycle{1;2;3;\dots;n}\in\S{n}.\]

Quelles sont les permutations de \(\S{n}\) qui commutent avec \(c\) ?

On pourra identifier \(\interventierii{1}{n}\) avec \(\Z/n\Z\) de sorte que \[\quantifs{\forall x\in\Z/n\Z}c\paren{x}=x+1.\]
\end{exo}

\begin{corr}
\note{À venir}
\end{corr}

\begin{exo}[Exercice 6]
Soit \(n\in\Ns\).

Montrer que toute permutation \(\sigma\in\S{n}\) est produit de transpositions de la forme \(\cycle{i;i+1}\) où \(i\in\interventierii{1}{n-1}\).
\end{exo}

\begin{corr}
\note{À venir}
\end{corr}

\begin{exo}[Exercice 7, groupe alterné \(\frakA{n}\)]
Soit \(n\in\Ns\).

\begin{enumerate}[series=exoAn]
\item Donner tous les éléments de \(\frakA{4}\) sous forme de produits de cycles à supports disjoints. \\

\item En supposant \(n\geq2\), quel est le cardinal de \(\frakA{n}\) ? \\

\item Donner une CNS sur \(n\) pour que le groupe \(\frakA{n}\) soit commutatif. \\

\item Montrer que tout élément de \(\frakA{n}\) est produit de \(3\)-cycles.
\end{enumerate}

On suppose désormais \(n\geq5\).

\begin{enumerate}[resume=exoAn]
\item Soient deux \(3\)-cycles \(c_1,c_2\in\frakA{n}\). Montrer \[\quantifs{\exists\sigma\in\frakA{n}}c_2=\sigma c_1\sigma\inv\] (on dit que \(c_1\) et \(c_2\) sont conjugués dans \(\frakA{n}\)). \\

\item Soient \(\groupe{G}[\times]\) un groupe abélien et \(\phi:\frakA{5}\to G\) un morphisme de groupes.

\begin{enumerate}
\item Montrer qu'il existe deux permutations \(a,b\in\frakA{5}\) telles que : \[\begin{dcases}a^3=\id{} \\ b^2=\id{} \\ \paren{ab}^5=\id{}\end{dcases}\]

\item Montrer \[a,b\in\ker\phi.\]

\textit{Indication :} utiliser le système et la commutativité de \(G\). \\

\item Montrer que \(\phi\) est constant. \\
\end{enumerate}

\item Montrer que tout morphisme de groupes de \(\frakA{n}\) vers un groupe abélien est constant\footnote{Cela montre que le groupe \(\frakA{n}\) n'est pas \guillemets{résoluble} si \(n\geq5\), proposition importante en \guillemets{théorie de Galois}. Cette démonstration provient de la belle conférence d'Alain Connes \guillemets{Évariste Galois et la théorie de l'ambiguïté} (on la trouve facilement sur internet).}.
\end{enumerate}
\end{exo}

\begin{corr}[1]
On a \[\begin{aligned}
\frakA{4}&=\left\lbrace\id{};\cycle{1;2;3};\cycle{3;2;1};\cycle{1;2;4};\cycle{4;2;1};\cycle{1;3;4};\cycle{4;3;1};\cycle{2;3;4};\right. \\
&\color{white}=\lbrace\color{black}\left.\cycle{4;3;2};\cycle{1;2}\cycle{3;4};\cycle{1;3}\cycle{2;4};\cycle{1;4}\cycle{2;3}\right\rbrace
\end{aligned}\]

On remarque \(\Card\frakA{4}=12\).
\end{corr}

\begin{corr}[2]
Montrons que \(\fonction{f}{\frakA{n}}{\S{n}\excluant\frakA{n}}{\sigma}{\cycle{1;2}\sigma}\) est une bijection.

\(f\) est bien définie car \(\quantifs{\forall\sigma\in\frakA{n}}f\paren{\sigma}\in\S{n}\excluant\frakA{n}\).

Soit \(\sigma\prim\in\S{n}\excluant\frakA{n}\).

On a \[\begin{aligned}
\quantifs{\forall\sigma\in\frakA{n}}f\paren{\sigma}=\sigma\prim&\ssi\cycle{1;2}\sigma=\sigma\prim \\
&\ssi\sigma=\cycle{1;2}\sigma\prim.
\end{aligned}\]

De plus : \[\begin{aligned}
\epsilon\paren{\cycle{1;2}\sigma\prim}&=\epsilon\paren{\cycle{1;2}}\epsilon\paren{\sigma\prim} \\
&=\paren{-1}\paren{-1} \\
&=1.
\end{aligned}\]

Donc \(\cycle{1;2}\sigma\prim\in\frakA{n}\).

Donc \(\cycle{1;2}\sigma\prim\) est l'unique antécédent de \(\sigma\) par \(f\).

Donc \(f\) est une bijection.

Donc \(\Card\frakA{n}=\Card\paren{\S{n}\excluant\frakA{n}}\).

Or \(\S{n}=\frakA{n}\union\paren{\S{n}\excluant\frakA{n}}\) (réunion disjointe).

Donc \[\begin{aligned}
\Card\S{n}&=\Card\frakA{n}+\Card\paren{\S{n}\excluant\frakA{n}} \\
n!&=2\Card\frakA{n} \\
\dfrac{n!}{2}&=\Card\frakA{n}.
\end{aligned}\]
\end{corr}

\begin{corr}[3]
\(\frakA{1}\) et \(\frakA{2}\) sont clairement commutatifs.

De plus, \(\frakA{3}=\accol{\id{};\cycle{1;2;3};\cycle{3;2;1}}\) est commutatif car \(\cycle{3;2;1}^2=\cycle{1;2;3}\).

Enfin, dans \(\frakA{n}\) avec \(n\geq4\), on a \[\begin{dcases}
\cycle{1;2;3}\cycle{1;2;4}\paren{3}=1 \\
\cycle{1;2;4}\cycle{1;2;3}\paren{3}=2
\end{dcases}\] donc \(\cycle{1;2;3}\) et \(\cycle{1;2;4}\) ne commutent pas donc \(\frakA{n}\) n'est pas commutatif.

D'où la CNS : \[\frakA{n}\text{ est commutatif}\ssi n\leq3.\]
\end{corr}

\begin{corr}[4]
Soit \(\sigma\in\frakA{n}\).

Soient \(N\in\N\) et \(a_1,b_1,\dots,a_N,b_N\in\interventierii{1}{N}\) tels que \[\begin{dcases}\sigma=\cycle{a_1;b_1}\dots\cycle{a_N;b_N} \\ \quantifs{\forall j\in\interventierii{1}{n}}a_j\not=b_j\end{dcases}\]

Comme \(\sigma\in\frakA{n}\), \(N\) est pair donc il existe \(k\in\N\) tel que \(N=2k\).

Donc \[\sigma=\underbrace{\cycle{a_1;b_1}\cycle{a_2;b_2}}_{\sigma_1}\dots\underbrace{\cycle{a_{N-1};b_{N-1}}\cycle{a_N;b_N}}_{\sigma_k}.\]

Il suffit de montrer que \(\sigma_1,\dots,\sigma_k\) sont produits de \(3\)-cycles.

Montrons que \(\sigma_1\) est produit de \(3\)-cycles.

Si \(\Card\paren{\accol{a_1;b_1}\inter\accol{a_2;b_2}}=2\) alors \[\cycle{a_1;b_1}\cycle{a_2;b_2}=\id{}.\]

Si \(\Card\paren{\accol{a_1;b_1}\inter\accol{a_2;b_2}}=1\) : supposons par exemple \(b_1=a_2\). Alors \[\cycle{a_1;b_1}\cycle{a_2;b_2}=\cycle{a_1;b_1;b_2}.\]

Si \(\Card\paren{\accol{a_1;b_1}\inter\accol{a_2;b_2}}=0\) alors \[\begin{aligned}
\cycle{a_1;b_1}\cycle{a_2;b_2}&=\cycle{a_1;b_1}\cycle{b_1;a_2}\cycle{b_1;a_2}\cycle{a_2;b_2} \\
&=\cycle{a_1;b_1;a_2}\cycle{b_1;a_2;b_2}.
\end{aligned}\]

Donc \(\sigma_1\) est produit de \(3\)-cycles.

De même, \(\quantifs{\forall i\in\interventierii{1}{k}}\sigma_k\text{ est produit de \(3\)-cycles}\).

Donc \(\sigma\) est produit de \(3\)-cycles.
\end{corr}

\begin{corr}[5]
Soient \(x_1,y_1,z_1,x_2,y_2,z_2\in\interventierii{1}{n}\) tels que \[\begin{dcases}c_1=\cycle{x_1;y_1;z_1} \\ c_2=\cycle{x_2;y_2;z_2}\end{dcases}\]

\(x_1,y_1,z_1\) sont deux à deux distincts et \(x_2,y_2,z_2\) aussi.

Soit \(\sigma\in\S{n}\) telle que \[\begin{dcases}\sigma\paren{x_1}=x_2 \\ \sigma\paren{y_1}=y_2 \\ \sigma\paren{z_1}=z_2\end{dcases}\]

On a \[\begin{aligned}
\sigma c_1\sigma\inv&=\cycle{\sigma\paren{x_1};\sigma\paren{y_1};\sigma\paren{z_1}} \\
&=\cycle{x_2;y_2;z_2} \\
&=c_2.
\end{aligned}\]

Soient \(a,b\in\interventierii{1}{n}\excluant\accol{x_1;y_1;z_1}\) tels que \(a\not=b\).

Posons \(\sigma\prim=\sigma\cycle{a;b}\).

On a \[\begin{dcases}
\sigma\prim\paren{x_1}=x_2 \\
\sigma\prim\paren{y_1}=y_2 \\
\sigma\prim\paren{z_1}=z_2
\end{dcases}\]

Donc \(\sigma\prim c_1{\sigma\prim}\inv=c_2\).

On a \(\epsilon\paren{\sigma\prim}=-\epsilon\paren{\sigma}\).

Donc \(\sigma\) ou \(\sigma\prim\) est pair.

Donc \(c_1\) et \(c_2\) sont conjugués dans \(\frakA{n}\).
\end{corr}

\begin{corr}[6a]
On schématise la situation comme suit :

\begin{center}
\begin{tikzpicture}
\draw (0,0) node[above] {\(1\)} -- (1,-1) node[below] {\(2\)};
\draw (1,0) node[above] {\(2\)} -- (0,-1) node[below] {\(1\)};
\draw (2,0) node[above] {\(3\)} -- (2,-1) node[below] {\(3\)};
\draw (3,0) node[above] {\(4\)} -- (4,-1) node[below] {\(5\)};
\draw (4,0) node[above] {\(5\)} -- (3,-1) node[below] {\(4\)};
\draw (0,-1.6) -- (0,-2.6) node[below] {\(1\)};
\draw (1,-1.6) -- (2,-2.6) node[below] {\(2\)};
\draw (2,-1.6) -- (3,-2.6) node[below] {\(3\)};
\draw (3,-1.6) -- (1,-2.6) node[below] {\(4\)};
\draw (4,-1.6) -- (4,-2.6) node[below] {\(5\)};
\draw[->] (4.5,0) -- (4.5,-1);
\draw[->] (4.5,-1.6) -- (4.5,-2.6);
\node[right] at (4.5,-0.5) {\(b\)};
\node[right] at (4.5,-2.1) {\(a\)};
\end{tikzpicture}
\end{center}

Prenons donc \(\begin{dcases}
a=\cycle{2;3;4} \\
b=\cycle{1;2}\cycle{4;5}
\end{dcases}\)

On a bien \[\begin{dcases}
a^3=\id{} \\
b^2=\id{}
\end{dcases}\]

De plus, on a aussi \[\begin{aligned}
\paren{ab}^5&=\paren{\cycle{2;3;4}\cycle{1;2}\cycle{4;5}}^5 \\
&=\cycle{1;3;4;5;2}^5 \\
&=\id{}.
\end{aligned}\]
\end{corr}

\begin{corr}[6b]~\\
Posons \(\begin{dcases}
a\prim=\phi\paren{a} \\
b\prim=\phi\paren{b}
\end{dcases}\)

On a \({a\prim}^3=\phi\paren{a}^3=\phi\paren{a^3}=\phi\paren{\id{}}=1\).

De même, \({b\prim}^2=1\).

Donc \(\paren{a\prim b\prim}^5=1\).

De plus \(G\) est commutatif donc \(\paren{a\prim b\prim}^5={a\prim}^5{b\prim}^5\).

Donc \(1={a\prim}\inv b\prim\).

Donc \(a\prim=b\prim\).

Donc \({a\prim}^2=1\).

Donc \(a\prim={a\prim}^3{a\prim}^{-2}=1\).

Donc \(b\prim=1\).

Donc \(a,b\in\ker\phi\).
\end{corr}

\begin{corr}[6c]
Soit \(c\) un \(3\)-cycle.

Selon (5), il existe \(\sigma\in\frakA{n}\) telle que \(c=\sigma a\sigma\inv\) car \(a=\cycle{2;3;4}\).

Donc \[\begin{WithArrows}
\phi\paren{c}&=\phi\paren{\sigma a\sigma\inv} \\
&=\phi\paren{\sigma}\phi\paren{a}\phi\paren{\sigma\inv} \Arrow{car \(G\) est commutatif} \\
&=\phi\paren{a}\phi\paren{\sigma}\phi\paren{\sigma\inv} \\
&=\phi\paren{a} \\
&=1.
\end{WithArrows}\]

Soit \(\sigma\prim\in\frakA{n}\).

Selon (4), \(\sigma\prim\) est produit de \(3\)-cycles.

Donc \(\phi\paren{\sigma\prim}=1\).
\end{corr}

\begin{corr}[7]
Soit \(\psi:\frakA{n}\to G\) un morphisme de groupes avec \(G\) abélien.

Posons \(\phi=\restr{\psi}{\frakA{5}}\).

Selon (6), \(\phi\) est constant.

Donc \[\psi\paren{\cycle{1;2;3}}=\phi\paren{\cycle{1;2;3}}=1.\]

Comme précédemment, on en déduit que tout \(3\)-cycle de \(\frakA{n}\) appartient à \(\ker\psi\) puis que \(\psi\) est constant.
\end{corr}

\chapter{Déterminants}

\minitoc

\begin{exo}[Exercice 1]
Soient \(a,b,c,d\in\R\).

Calculer les déterminants suivants quand ils sont définis : \[A=\begin{vmatrix}
a\inv & a & a^3 \\
b\inv & b & b^3 \\
c\inv & c & c^3
\end{vmatrix}\qquad B=\begin{vmatrix}
a & c & c & b \\
c & a & b & c \\
c & b & a & c \\
b & c & c & a
\end{vmatrix}\qquad C=\begin{vmatrix}
a+b & b+c & c+a \\
a^2+b^2 & b^2+c^2 & c^2+a^2 \\
a^3+b^3 & b^3+c^3 & c^3+a^3
\end{vmatrix}\]

On factorisera autant que possible les résultats.
\end{exo}

\begin{corr}
\note{À venir}
\end{corr}

\begin{exo}[Exercice 2]
\begin{enumerate}
\item Donner une CNS sur \(\lambda\in\R\) pour que \[\fonction{u}{\polydeg[\R]{2}}{\polydeg[\R]{2}}{P}{P+\lambda XP\prim}\] soit un automorphisme de \(\polydeg[\R]{2}\). \\

\item Donner une CNS sur \(\lambda\in\R\) pour que \[\fonction{u}{\polydeg[\R]{3}}{\polydeg[\R]{3}}{P}{P+\paren{\lambda X+1}P\prim+\lambda X^2P\paren{0}}\] soit un automorphisme de \(\polydeg[\R]{3}\). \\

\item (Mines-Telecom 2016) Soit \(n\in\Ns\). Calculer le déterminant de \[\fonction{u}{\polydeg[\R]{n}}{\polydeg[\R]{n}}{P}{XP\prim+P}\]
\end{enumerate}
\end{exo}

\begin{corr}
\note{À venir}
\end{corr}

\begin{exo}[Exercice 3]
\begin{enumerate}
\item Donner une CNS sur \(\lambda,\mu\in\R\) pour que \[\fonction{u}{\polydeg[\R]{2}}{\polydeg[\R]{2}}{P}{\paren{\lambda X^2+\mu}P\paren{1}+XP\prim}\] soit un automorphisme de \(\polydeg[\R]{2}\). \\

\item Plus généralement, donner une CNS sur \(\lambda,\mu\in\R\) et \(n\in\Ns\) pour que \[\fonction{u}{\polydeg[\R]{n}}{\polydeg[\R]{n}}{P}{\paren{\lambda X^n+\mu}P\paren{1}+XP\prim}\] soit un automorphisme de \(\polydeg[\R]{n}\).
\end{enumerate}
\end{exo}

\begin{corr}
\note{À venir}
\end{corr}

\begin{exo}[Exercice 4]
Soient \(a,b,c,t\in\R\).

On suppose que \(a\), \(b\) et \(c\) sont deux à deux distincts.

Résoudre le système d'inconnue \(\paren{x,y,z}\in\R^3\) : \[\paren{S}\begin{dcases}
x+y+z=t \\
ax+by+cz=t^2 \\
a^2x+b^2y+c^2z=t^3
\end{dcases}\]
\end{exo}

\begin{corr}
\note{À venir}
\end{corr}

\begin{exo}[Exercice 5]
Soient \(a,b\in\R\).

Résoudre le système d'inconnues \(x,y,z\in\R\) : \[\paren{S}\begin{dcases}
ax+\paren{a-1}y+\paren{a+b}z=1 \\
ax+ay+bz=a \\
bx+by+az=b
\end{dcases}\]
\end{exo}

\begin{corr}
\note{À venir}
\end{corr}

\begin{exo}[Exercice 6]
Soit \(n\in\Ns\).

Calculer le déterminant de la matrice \(A=\paren{a_{ij}}_{\paren{i,j}\in\interventierii{1}{n}^2}\in\M{n}[\R]\) définie par : \[\quantifs{\forall i,j\in\interventierii{1}{n}}a_{ij}=\abs{i-j}.\]
\end{exo}

\begin{corr}
\note{À venir}
\end{corr}

\begin{exo}[Exercice 7]
Soient \(x_1,\dots,x_n\in\C\).

Calculer le déterminant \[\begin{vmatrix}
0 & \dots & 0 & x_1 \\
\vdots & \iddots & \iddots & 0 \\
0 & \iddots & \iddots & \vdots \\
x_n & 0 & \dots & 0
\end{vmatrix}\]
\end{exo}

\begin{corr}
\note{À venir}
\end{corr}

\begin{exo}[Exercice 8]
\begin{enumerate}
\item Soient \(x_1,\dots,x_n\in\C\). On pose \(\quantifs{\forall k\in\interventierii{1}{n}}s_k=\sum_{j=1}^kx_j\).

Calculer le déterminant \[\begin{vmatrix}
s_1 & \dots & \dots & \dots & s_1 \\
\vdots & s_2 & \dots & \dots & s_2 \\
\vdots & \vdots & s_3 & \dots & s_3 \\
\vdots & \vdots & \vdots & \ddots & \vdots \\
s_1 & s_2 & s_3 & \dots & s_n
\end{vmatrix}\]

\item Que vaut le déterminant suivant ? \[\begin{vmatrix}
1 & \dots & \dots & \dots & 1 \\
\vdots & 2 & \dots & \dots & 2 \\
\vdots & \vdots & 3 & \dots & 3 \\
\vdots & \vdots & \vdots & \ddots & \vdots \\
1 & 2 & 3 & \dots & n
\end{vmatrix}\]

Retrouver ce résultat en utilisant l'\thref{exo:matrice1...n}.
\end{enumerate}
\end{exo}

\begin{corr}
\note{À venir}
\end{corr}

\begin{exo}[Exercice 9]
Calculer le déterminant et la trace des endomorphismes \[\fonction{f}{\M{n}[\C]}{\M{n}[\C]}{M}{\trans{M}}\qquad\text{et}\qquad\fonction{g}{\M{n}[\C]}{\M{n}[\C]}{M}{\paren{\tr M}I_n}\]
\end{exo}

\begin{corr}
\note{À venir}
\end{corr}

\begin{exo}[Exercice 10]
Soient \(n\in\Ns\) et \(A\in\M{n}[\R]\) antisymétrique et inversible.

Montrer que l'entier \(n\) est pair.

Donner un exemple de matrice antisymétrique inversible.
\end{exo}

\begin{corr}
\note{À venir}
\end{corr}

\begin{exo}[Exercice 11]
Soient \(n\in\Ns\) et \(A\in\M{n}[\R]\) telle que \(A^2=-I_n\).

Montrer que l'entier \(n\) est pair.

Donner un exemple

\begin{itemize}
\item de matrice \(A\in\M{n}[\R]\) telle que \(A^2=-I_n\). \\

\item de matrice \(A\in\M{n}[\C]\) telle que \(A^2=-I_n\) avec \(n\) impair.
\end{itemize}
\end{exo}

\begin{corr}
\note{À venir}
\end{corr}

\begin{exo}[Exercice 12]
Soient \(A,B\in\M{n}[\K]\) avec \(\K=\R\) ou \(\C\).

Montrer que \(A\) et \(B\) sont semblables sur \(\R\) si, et seulement si, elles sont semblables sur \(\C\), \cad : \[\croch{\quantifs{\exists P\in\GL{n}[\R]}B=PAP\inv}\ssi\croch{\quantifs{\exists Q\in\GL{n}[\C]}B=QAQ\inv}.\]
\end{exo}

\begin{corr}
\note{À venir}
\end{corr}

\begin{exo}[Exercice 13, déterminant d'une matrice \guillemets{tridiagonale}]
Soient \(x,y\in\C\) et \(n\in\interventierie{2}{\pinf}\).

On considère la matrice \(M_n=\paren{m_{ij}}_{\paren{i,j}\in\interventierii{1}{n}^2}\in\M{n}[\C]\) définie par : \[\quantifs{\forall\paren{i,j}\in\interventierii{1}{n}^2}m_{ij}=\begin{dcases}
0 &\text{si }\abs{i-j}\geq2 \\
xy &\text{si }i-j=-1 \\
x+y &\text{si }i-j=0 \\
1 &\text{si }i-j=1
\end{dcases}\]

Calculer le déterminant de \(M_n\).

\textit{Indication :} poser \(u_n=\det M_n\) et montrer que la suite \(\paren{u_n}_n\) vérifie une relation de récurrence linéaire d'ordre 2 : \(\quantifs{\forall n\in\interventierie{2}{\pinf}}au_{n+2}+bu_{n+1}+cu_n=0\) (avec \(a,b,c\in\C\)) puis en déduire la valeur de \(u_n\) en fonction de \(n\) (utiliser l'exercice \thref{exo:relationsDeRécurrenceD'Ordre2}).
\end{exo}

\begin{corr}
\note{À venir}
\end{corr}

\begin{exo}[Exercice 14]
Soit \(x\in\C\).

On considère la matrice \(M=\paren{m_{ij}}_{\paren{i,j}\in\interventierii{1}{n}^2}\in\M{n}[\C]\) définie par : \[\quantifs{\forall\paren{i,j}\in\interventierii{1}{n}^2}m_{ij}=\begin{dcases}
0 &\text{si }\abs{i-j}\geq2 \\
x &\text{si }\abs{i-j}=1 \\
2x &\text{si }i=j
\end{dcases}\]

Calculer le déterminant de \(M\).
\end{exo}

\begin{corr}
\note{À venir}
\end{corr}

\begin{exo}[Exercice 15]
Soit \(n\in\Ns\).

On note \(\fami{B}=\paren{e_1,\dots,e_n}\) la base canonique de \(\R^n\).

\begin{enumerate}[series=det15]
\item Soit \(\sigma\in\S{n}\). Justifier qu'il existe un unique endomorphisme \(u\in\Lendo{\R^n}\) tel que \[\quantifs{\forall j\in\interventierii{1}{n}}u\paren{e_j}=e_{\sigma\paren{j}}.\]
\end{enumerate}

Dans la suite, on note \(u_\sigma\) cet endomorphisme et on note \(M_\sigma\) sa matrice dans \(\fami{B}\). Les matrices de la forme \(M_\sigma\), où \(\sigma\in\S{n}\), sont appelées matrices de permutation.

\begin{enumerate}[resume=det15]
\item On suppose ici \(n=3\). Donner la matrice de permutation \(M_\sigma\) pour toute permutation \(\sigma\in\S{3}\). \\

\item Montrer que l'application \[\fonction{\phi}{\S{n}}{\GL{}[\R^n]}{\sigma}{u_\sigma}\] est bien définie et est un morphisme de groupes. \\

\item En déduire un morphisme de groupes de \(\S{n}\) vers \(\GL{n}[\R]\). \\

\item Soit \(\sigma\in\S{n}\). Calculer le déterminant de \(u_\sigma\).
\end{enumerate}
\end{exo}

\begin{corr}
\note{À venir}
\end{corr}

\begin{exo}[Exercice 16]
Soit \(n\in\Ns\).

On note \(\M{n}[\Z]\) l'ensemble des matrices carrées de taille \(n\) à coefficients dans \(\Z\).

\begin{enumerate}
\item Vérifier que \(\M{n}[\Z]\) est un sous-anneau de \(\anneau{\M{n}[\R]}\). \\

\item Soit \(A\in\M{n}[\Z]\). Donner une CNS sur \(\det A\) pour que \(A\) soit inversible dans \(\M{n}[\Z]\). \\

\item Montrer que \(\M{n}[\Z]\) possède une infinité d'éléments inversibles si \(n\geq2\).
\end{enumerate}
\end{exo}

\begin{corr}
\note{À venir}
\end{corr}

\begin{exo}[Exercice 17, ENSAM PSI 2016 (BEOS 2727)]
Soit \(P\in\poly[\R]\) un polynôme de degré \(3\).

On lui associe la famille de vecteurs de \(\polydeg[\R]{4}\) : \[\fami{F}=\paren{P,XP,P\prim,XP\prim,X^2P\prim}.\]

On note \(D\) le déterminant de cette famille dans la base canonique de \(\polydeg[\R]{4}\).

\begin{enumerate}
\item Montrer que \(D\) est nul si, et seulement si, il existe deux polynômes non-nuls \(U\in\polydeg[\R]{1}\) et \(V\in\polydeg[\R]{2}\) tels que \(PU=P\prim V\). \\

\item Montrer que \(D\) est nul si, et seulement si, le polynôme \(P\) admet une racine multiple. \\

\item Calculer \(D\) si \(P=aX^3+bX^2+cX\).
\end{enumerate}
\end{exo}

\begin{corr}
\note{À venir}
\end{corr}

\begin{exo}[Exercice 18, polynômes interpolateurs de Lagrange]
Soient \(x_0,\dots,x_n\in\K\) des scalaires deux à deux distincts.

On note \(\fami{B}=\paren{1,X,\dots,X^n}\) la base canonique de \(\polydeg{n}\).

On note \(L_0,\dots,L_n\in\polydeg{n}\) les polynômes définis par : \[\quantifs{\forall j\in\interventierii{0}{n}}L_j=\prod_{k\in\interventierii{0}{n}\excluant\accol{j}}\dfrac{X-x_k}{x_j-x_k}.\]

\begin{enumerate}
\item Montrer que \(\paren{L_0,\dots,L_n}\) est une base de \(\polydeg{n}\). \\

\item Calculer \(\detb{\fami{B}}\paren{L_0,\dots,L_n}\).
\end{enumerate}
\end{exo}

\begin{corr}
\note{À venir}
\end{corr}

\begin{exo}[Exercice 19, ENSEA PSI 2018]
Soient \(E\) un espace vectoriel de dimension \(3\), \(u\in\Lendo{E}\) et \(\fami{B}\) une base de \(E\).

Étudier l'application \[\fonction{\phi}{E^3}{\K}{\paren{x,y,z}}{\detb{\fami{B}}\paren{u\paren{x},y,z}+\detb{\fami{B}}\paren{x,u\paren{y},z}+\detb{\fami{B}}\paren{x,y,u\paren{z}}}\]
\end{exo}

\begin{corr}
\note{À venir}
\end{corr}

\begin{exo}[Exercice 20, TPE PSI 2018]
Soient \(a,b,r_1,\dots,r_n\in\R\).

On pose : \[A=\begin{pmatrix}
r_1 & a & \dots & a \\
b & \ddots & \ddots & \vdots \\
\vdots & \ddots & \ddots & a \\
b & \dots & b & r_n
\end{pmatrix}\qquad\text{et}\qquad\quantifs{\forall x\in\R}\Delta\paren{x}=\begin{vmatrix}
r_1+x & a+x & \dots & a+x \\
b+x & \ddots & \ddots & \vdots \\
\vdots & \ddots & \ddots & a+x \\
b+x & \dots & b+x & r_n+x
\end{vmatrix}\]

\begin{enumerate}
\item Montrer que la fonction \(\Delta\) est polynomiale de degré au plus \(1\). \\

\item En déduire \(\det A\) si \(a\not=b\). \\

\item Supposons \(a=b\). Comment calculer \(\det A\) ?
\end{enumerate}
\end{exo}

\begin{corr}
\note{À venir}
\end{corr}

\chapter{Séries, familles sommables}

\minitoc

Dans tout le TD, on admet \(\sum_{n=1}^{\pinf}\dfrac{1}{n^2}=\dfrac{\pi^2}{6}\) et \(\sum_{n=1}^{\pinf}\dfrac{1}{n^4}=\dfrac{\pi^4}{90}\).

\begin{exo}[Exercice 1]
Soient \(a,b,c\in\R\).

Déterminer la nature des séries suivantes, dont on notera \(x_n\) les termes généraux :

\begin{enumerate}
\item \(\sum_n\dfrac{1}{n\sqrt[n]{n}}\). \\

\item \(\sum_n\paren{\exp\dfrac{1}{n}-\exp\dfrac{1}{n+1}}\). \\

\item \(\sum_n\paren{\e{}-\paren{1+\dfrac{1}{n}}^n}\). \\

\item \(\sum_n\ln\dfrac{n^2+n+1}{n^2+n-1}\). \\

\item \(\sum_n\dfrac{1}{n^a4^n}\binom{n}{2n}\). \\

\item \(\sum_n\paren{a\sqrt{n}+b\sqrt{n+1}+c\sqrt{n+2}}\). \\

\item \(\sum_n\dfrac{1}{\ln^nn}\). \\

\item \(\sum_n\dfrac{n!^a}{\paren{2n}!}\). \\

\item \(\sum_n\dfrac{\sqrt{n}\ln n}{\e{n}}\). \\

\item \(\sum_n\dfrac{1}{\paren{\ln\paren{\ln n}}^n}\). \\

\item \(\sum_n\int_0^1\tan^nt\odif{t}\). \\

\item \(\sum_n\int_0^{\frac{\pi}{4}}\tan^{n^2}t\odif{t}\).
\end{enumerate}
\end{exo}

\begin{corr}
\note{À venir}
\end{corr}

\begin{exo}[Exercice 2]
Déterminer si les séries suivantes sont convergentes et calculer alors leur somme : \[\sum_{n\geq1}\dfrac{1}{n\paren{n+1}\paren{n+2}}\qquad\text{et}\qquad\sum_{n\geq0}\ln\paren{\cos\dfrac{x}{2^n}}\text{ où }x\in\intervee{0}{\dfrac{\pi}{2}}.\]
\end{exo}

\begin{corr}
\note{À venir}
\end{corr}

\begin{exo}[Exercice 3, classique]
\begin{enumerate}
\item Soit \(\sum_nu_n\) une série à termes positifs.

On suppose la série \(\sum_nu_n\) convergente. La série \(\sum_nu_n^2\) est-elle convergente ?

Inversement, si l'on suppose \(\sum_nu_n^2\) convergente, la série \(\sum_nu_n\) est-elle convergente ? \\

\item L'implication montrée ci-dessus reste-t-elle vraie si l'on ne suppose pas que les termes de la suite \(\paren{u_n}_n\) sont positifs ?
\end{enumerate}
\end{exo}

\begin{corr}
\note{À venir}
\end{corr}

\begin{exo}[Exercice 4, TPE]
On pose : \[\quantifs{\forall n\in\Ns}u_n=\paren{n\sin\dfrac{1}{n}}^{n^2}.\]

\begin{enumerate}
\item Déterminer \(l=\lim_{n\to\pinf}u_n\). \\

\item Déterminer la nature de la série \(\sum_n\paren{u_n-l}\).
\end{enumerate}
\end{exo}

\begin{corr}
\note{À venir}
\end{corr}

\begin{exo}[Exercice 5, ENSEA]
\begin{enumerate}
\item Montrer qu'on a, quand \(x\) tend vers \(1\) : \[\Arccos x\sim\sqrt{2\paren{1-x}}.\]

\item Donner la nature de la série \[\sum_n\Arccos\dfrac{n^2+n+1}{n^2+n+3}.\]
\end{enumerate}
\end{exo}

\begin{corr}
\note{À venir}
\end{corr}

\begin{exo}[Exercice 6, séries de Bertrand]\thlabel{exo:sériesDeBertrand}
Donner une CNS sur \(\paren{\alpha,\beta}\in\R^2\) pour que la série \[\sum_n\dfrac{1}{n^{\alpha}\ln^\beta n}\] soit convergente.
\end{exo}

\begin{corr}
\note{À venir}
\end{corr}

\begin{exo}[Exercice 7]
\begin{enumerate}
\item Montrer que la série \(\sum_{n\geq1}\sin\paren{\pi n+\dfrac{\pi}{n}}\) est convergente.

On note \(S=\sum_{n=1}^{\pinf}\sin\paren{\pi n+\dfrac{\pi}{n}}\) la somme de cette série. \\

\item Donner un rang \(N\in\Ns\) tel que : \[\abs{S-\sum_{n+1}^N\sin\paren{\pi n+\dfrac{\pi}{n}}}\leq10^{-3}.\]
\end{enumerate}
\end{exo}

\begin{corr}
\note{À venir}
\end{corr}

\begin{exo}[Exercice 8]
Montrer que la somme \[\sum_{n=0}^{\pinf}\dfrac{\paren{-8}^n}{\paren{2n}!}\] est bien définie et donner sa partie entière.
\end{exo}

\begin{corr}
\note{À venir}
\end{corr}

\begin{exo}[Exercice 9, Mines]
La série \[\sum_n\ln\paren{1+\dfrac{\paren{-1}^n}{\sqrt{n}}}\] est-elle convergente ?
\end{exo}

\begin{corr}
\note{À venir}
\end{corr}

\begin{exo}[Exercice 10]
Donner une CNS sur \(x\in\R\) pour que la série suivante converge : \[\sum_n\dfrac{x^n}{1+x^{2n}}.\]
\end{exo}

\begin{corr}
\note{À venir}
\end{corr}

\begin{exo}[Exercice 11]
Donner une CNS sur \(\alpha\in\Rps\) pour que la série suivante converge : \[\sum_{n\geq2}\dfrac{\paren{-1}^n}{n^{\alpha}+\paren{-1}^n}.\]
\end{exo}

\begin{corr}
\note{À venir}
\end{corr}

\begin{exo}[Exercice 12]
Soit \(\paren{u_n}_n\) une suite telle que : \[\quantifs{\forall n\in\N}u_{n+1}=\dfrac{\e{-u_n}}{n+1}.\]

\begin{enumerate}
\item Déterminer la nature de la série \(\sum_nu_n\). \\

\item Déterminer la nature de la série \(\sum_n\paren{-1}^nu_n\).
\end{enumerate}
\end{exo}

\begin{corr}
\note{À venir}
\end{corr}

\begin{exo}[Exercice 13]
Soit \(n\in\N\).

On pose : \[a_n=n!\sum_{k=0}^n\dfrac{\paren{-1}^k}{k!}.\]

\begin{enumerate}
\item Justifier que \(a_n\) est un entier relatif. \\

\item Montrer que la série \(\sum_{k\geq n+1}\dfrac{\paren{-1}^k}{k!}\) converge. On note \(R_n\) sa somme. \\

\item Donner le signe de \(R_n\) en fonction de \(n\). \\

\item On suppose \(n\geq2\). Montrer que \(a_n\) est l'entier relatif le plus proche de \(\dfrac{n!}{\e{}}\).
\end{enumerate}
\end{exo}

\begin{corr}
\note{À venir}
\end{corr}

\begin{exo}[Exercice 14]
On pose : \[\quantifs{\forall n\in\N}R_n=\sum_{k=n+1}^{\pinf}\dfrac{1}{k!}.\]

\begin{enumerate}
\item Quelle est la limite de la suite \(\paren{R_n}_n\) ? \\

\item Montrer : \[\quantifs{\forall n\in\N}0\leq R_n\leq\dfrac{1}{nn!}.\]

\item En déduire un équivalent de \(R_{n-1}\) quand \(n\) tend vers \(\pinf\). \\

\item Déterminer la nature des séries \[\sum_n\sin\paren{2\e{}\pi n!}\qquad\text{et}\qquad\sum_n\sin\paren{\e{}\pi n!}.\]
\end{enumerate}
\end{exo}

\begin{corr}
\note{À venir}
\end{corr}

\begin{exo}[Exercice 15, CCP]
On pose : \[\quantifs{\forall n\in\interventierie{2}{\pinf}}u_n=\paren{\dfrac{\ln\paren{n+1}}{\ln n}}^n.\]

\begin{enumerate}
\item Déterminer la limite de la suite \(\paren{u_n}_n\). \\

\item Déterminer la nature de la série \(\sum_n\dfrac{u_n-1}{n}\).

\textit{On admet\footnote{\Cf \thref{exo:sériesDeBertrand}.} que la série de Bertrand \(\sum_{n\geq2}\dfrac{1}{n\ln n}\) diverge.}
\end{enumerate}
\end{exo}

\begin{corr}
\note{À venir}
\end{corr}

\begin{exo}[Exercice 16]
On considère une suite \(\paren{u_n}_n\) telle que : \[u_0\in\intervee{0}{\pi}\qquad\text{et}\qquad\quantifs{\forall n\in\N}u_{n+1}=\sin u_n.\]

\begin{enumerate}
\item Montrer : \[\quantifs{\forall n\in\N}0<u_n<\pi.\]

\item Étudier la croissance de la suite \(\paren{u_n}_n\) puis sa limite. \\

\item Montrer : \[\lim_{n\to\pinf}\dfrac{1}{u_{n+1}^2}-\dfrac{1}{u_n^2}=\dfrac{1}{3}.\]

\item En déduire un équivalent de \(u_n\) quand \(n\) tend vers \(\pinf\).

\textit{On admet\footnote{\Cf \thref{exo:moyenneDeCesàroD'UneSuite}.} que si \(\paren{a_n}_n\in\R^\N\) converge vers \(l\in\R\) alors sa \guillemets{moyenne de Cesàro} converge aussi vers \(l\) :} \[\lim_{n\to\pinf}\dfrac{1}{n}\sum_{k=1}^na_k=l.\]
\end{enumerate}
\end{exo}

\begin{corr}
\note{À venir}
\end{corr}

\begin{exo}[Exercice 17]
Soit \(\paren{u_n}_n\) une suite décroissante de réels positifs.

On suppose que la série \(\sum_nu_n\) est convergente.

Montrer : \[u_n\simqd{n\to\pinf}\o{\dfrac{1}{n}}.\]
\end{exo}

\begin{corr}
\note{À venir}
\end{corr}

\begin{exo}[Exercice 18, astuces à retenir]
Calculer : \[S_2=\sum_{k=0}^{\pinf}\dfrac{1}{\paren{2k+1}^2}\qquad S_2\prim=\sum_{n=1}^{\pinf}\dfrac{\paren{-1}^{n-1}}{n^2}\qquad S_4=\sum_{k=0}^{\pinf}\dfrac{1}{\paren{2k+1}^4}\qquad S_4\prim=\sum_{n=1}^{\pinf}\dfrac{\paren{-1}^{n-1}}{n^4}.\]
\end{exo}

\begin{corr}
\note{À venir}
\end{corr}

\begin{exo}[Exercice 19]
Calculer \[S=\sum_{a=1}^{\pinf}\sum_{b=a}^{\pinf}\dfrac{1}{b^3}.\]
\end{exo}

\begin{corr}
\note{À venir}
\end{corr}

\begin{exo}[Exercice 20]
On pose : \[I=\accol{\paren{k,n}\in\N^2\tq1\leq k\leq n}\qquad I\prim=\accol{\paren{k,n}\in\N^2\tq1\leq n\leq k}\qquad I\seconde=\paren{\Ns}^2.\]

Les familles suivantes sont-elles sommables ? Le cas échéant, calculer leur somme : \[\paren{\dfrac{\paren{-1}^{n-1}}{nk\paren{k+1}}}_{\paren{k,n}\in I}\qquad\paren{\dfrac{\paren{-1}^{n-1}}{nk\paren{k+1}}}_{\paren{k,n}\in I\prim}\qquad\paren{\dfrac{\paren{-1}^{n-1}}{nk\paren{k+1}}}_{\paren{k,n}\in I\seconde}.\]
\end{exo}

\begin{corr}
\note{À venir}
\end{corr}

\begin{exo}[Exercice supplémentaire]
On pose : \[I=\accol{\paren{k,n}\in\N^2\tq0<k<n}.\]

La famille suivante est-elle sommable ? Le cas échéant, calculer sa somme : \[\fami{G}=\paren{\dfrac{1}{2^n}\exp\dfrac{2\i k\pi}{n}}_{\paren{k,n}\in I}.\]
\end{exo}

\begin{corr}
\note{À venir}
\end{corr}

\begin{exo}[Exercice 21]
Déterminer si la série de terme général \(u_n\) est convergente et calculer alors sa somme dans les cas suivants :

\begin{enumerate}
\item \(\quantifs{\forall n\in\interventierie{2}{\pinf}}u_n=\sum_{p=1}^{n-1}\dfrac{1}{p^2\paren{n-p}^2}\). \\

\item \(\quantifs{\forall n\in\N}u_n=\sum_{k=0}^n\dfrac{1}{2^{n-k}k!}\).
\end{enumerate}
\end{exo}

\begin{corr}
\note{À venir}
\end{corr}

\begin{exo}[Exercice 22]
\begin{enumerate}
\item Calculer \[S=\sum_{\paren{a,b}\in\Ns\times\N}\dfrac{1}{\paren{a^2+b}\paren{a^2+b+1}}.\]

\item En déduire : \[S\prim=\sum_{n\in\Ns}\dfrac{\floor{\sqrt{n}}}{n\paren{n+1}}.\]
\end{enumerate}
\end{exo}

\begin{corr}
\note{À venir}
\end{corr}

\begin{exo}[Exercice 23, X MP]
On note \(I\) l'ensemble des entiers naturels non-nuls dont l'écriture décimale ne comporte pas le chiffre \(9\).

La famille \(\paren{\dfrac{1}{n}}_{n\in I}\) est-elle sommable ?
\end{exo}

\begin{corr}
\note{À venir}
\end{corr}

\begin{exo}[Exercice 24]
On note \(P\) (comme \guillemets{puissances}) l'ensemble des entiers qui s'écrivent sous la forme \(a^b\) où \(a,b\in\interventierie{2}{\pinf}\).

\begin{enumerate}
\item Calculer \(\sum_{a=2}^{\pinf}\sum_{b=2}^{\pinf}\dfrac{1}{a^b}\). \\

\item En déduire \(\sum_{m\in P}\dfrac{1}{m-1}\) (formule due à Euler).
\end{enumerate}
\end{exo}

\begin{corr}
\note{À venir}
\end{corr}

\begin{exo}[Exercice 25]
Calculer la partie entière des sommes suivantes : \[S_1=\sum_{n=1}^{10^9}\dfrac{1}{n^{\nicefrac{2}{3}}}\qquad\text{et}\qquad S_2=\sum_{n=1}^{10^{12}}\dfrac{1}{n^{\nicefrac{2}{3}}}.\]
\end{exo}

\begin{corr}
\note{À venir}
\end{corr}

\chapter{Espaces préhilbertiens}

\minitoc

\begin{exo}[Exercice 1]
Soit \(n\in\Ns\).

On munit \(\M{n}[\R]\) de son produit scalaire canonique : \[\fonctionlambda{\M{n}[\R]\times\M{n}[\R]}{\R}{\paren{A,B}}{\tr\paren{\trans{A}B}=\sum_{i=1}^n\sum_{j=1}^na_{ij}b_{ij}}\] et on note \(\norme{\cdot}\) la norme associée.

\begin{enumerate}
\item Calculer \(\norme{I_n}\). \\

\item Déterminer l'orthogonal de \(\sym{n}[\R]\) dans \(\M{n}[\R]\). \\

\item Montrer : \[\quantifs{\forall A,B\in\M{n}[\R]}\norme{AB}\leq\norme{A}\norme{B}.\]
\end{enumerate}
\end{exo}

\begin{corr}
\note{À venir}
\end{corr}

\begin{exo}[Exercice 2]
Soient \(\groupe{E}[\ps{\cdot}{\cdot}]\) un espace préhilbertien réel et \(F\) et \(G\) deux sous-espaces vectoriels de \(E\).

\begin{enumerate}[series=exoOrthos]
\item Montrer \(\paren{F+G}\ortho=F\ortho\inter G\ortho\). \\

\item Montrer \(F\subset G\imp G\ortho\subset F\ortho\).
\end{enumerate}

On suppose désormais que \(E\) est de dimension finie.

\begin{enumerate}[resume=exoOrthos]
\item Montrer \(\paren{F\ortho}\ortho=F\). \\

\item Montrer \(\paren{F\inter G}\ortho=F\ortho+G\ortho\). \\

\item Montrer \(F\subset G\ssi G\ortho\subset F\ortho\).
\end{enumerate}
\end{exo}

\begin{corr}
\note{À venir}
\end{corr}

\begin{exo}[Exercice 3]
\begin{enumerate}
\item Montrer que l'application \[\fonction{\phi}{\poly[\R]\times\poly[\R]}{\R}{\paren{P,Q}}{\int_{-1}^1P\paren{t}Q\paren{t}\odif{t}}\] est un produit scalaire sur le \(\R\)-espace vectoriel \(\poly[\R]\). \\

\item Appliquer l'algorithme de Gram-Schmidt à la base canonique de \(\polydeg[\R]{3}\).
\end{enumerate}
\end{exo}

\begin{corr}
\note{À venir}
\end{corr}

\begin{exo}[Exercice 4, CCP PSI 2012]
Déterminer les réels \(a,b,c,d\) tels que l'intégrale \[\int_{\frac{-\pi}{2}}^{\frac{\pi}{2}}\paren{\sin x-ax^3-bx^2-cx-d}^2\odif{x}\] soit la plus petite possible.
\end{exo}

\begin{corr}
\note{À venir}
\end{corr}

\begin{exo}[Exercice 5, CCP PSI]
On munit \(\R^4\) de son produit scalaire canonique et de sa base canonique \(\fami{B}_0\).

Donner la matrice dans \(\fami{B}_0\) de la projection orthogonale sur le plan \[\Pi:\begin{dcases}
x+2y+z+2t=0 \\
x-y+z-t=0
\end{dcases}\]
\end{exo}

\begin{corr}
\note{À venir}
\end{corr}

\begin{exo}[Exercice 6]
Calculer : \[\inf_{a,b,c\in\R}\int_{-1}^1\paren{t^3-at^2-bt-c}^2\odif{t}.\]
\end{exo}

\begin{corr}
\note{À venir}
\end{corr}

\begin{exo}[Exercice 7]
Calculer : \[\inf_{a,b,c\in\R}\int_0^1\paren{\e{t}-at^2-bt-c}^2\odif{t}.\]
\end{exo}

\begin{corr}
\note{À venir}
\end{corr}

\begin{exo}[Exercice 8]
On munit \(\R^3\) de son produit scalaire canonique.

On note \(H\) l'hyperplan de \(\R^3\) d'équation cartésienne \[x+y-z=0\] (dans la base canonique).

Déterminer la matrice de la projection orthogonale sur \(H\).
\end{exo}

\begin{corr}
\note{À venir}
\end{corr}

\begin{exo}[Exercice 9]
Soient \(E\) un espace euclidien, \(u\in\Lendo{E}\) et \(x\in E\).

On note \(S\) la sphère unité de \(E\) : \[S=\accol{y\in E\tq\norme{y}=1}.\]

Montrer : \[\norme{x}=\max_{y\in S}\ps{y}{x}.\]
\end{exo}

\begin{corr}
\note{À venir}
\end{corr}

\begin{exo}[Exercice 10]
Soient \(E\) et \(F\) des espaces euclidiens dont on note \(\norme{\cdot}_E\) et \(\norme{\cdot}_F\) les normes, \(\fami{B}=\paren{e_1,\dots,e_n}\) une base orthonormée de \(E\) et \(u\in\L{E}{F}\).

On suppose que la famille \(\paren{u\paren{e_1},\dots,u\paren{e_n}}\in F^n\) est orthonormale.

Montrer : \[\quantifs{\forall x\in E}\norme{u\paren{x}}_F=\norme{x}_E.\]
\end{exo}

\begin{corr}
\note{À venir}
\end{corr}

\begin{exo}[Exercice 11]
Soit \(f\in\ensclasse{0}{\intervii{0}{1}}{\R}\) positive et non-nulle.

On pose : \[\quantifs{\forall n\in\N}I_n=\int_0^1f^n\paren{t}\odif{t}\] et : \[\quantifs{\forall n\in\N}u_n=\dfrac{I_{n+1}}{I_n}.\]

\begin{enumerate}
\item Justifier : \(\quantifs{\forall n\in\N}I_n>0\). \\

\item Montrer que \(\paren{u_n}_{n\in\N}\) est croissante.
\end{enumerate}
\end{exo}

\begin{corr}
\note{À venir}
\end{corr}

\begin{exo}[Exercice 12, Centrale]
Soient \(E\) un espace euclidien dont on note le produit scalaire \(\paren{x,y}\mapsto\ps{x}{y}\) et \(f\in\Lendo{E}\).

On considère l'application \[\fonction{g}{E^2}{\R}{\paren{x,y}}{\ps{f\paren{x}}{f\paren{y}}}\]

\begin{enumerate}
\item Donner une CNS sur \(f\) pour que \(g\) soit un produit scalaire. \\

\item On suppose désormais : \[\quantifs{\forall x,y\in E}\ps{x}{y}=0\ssi\ps{f\paren{x}}{f\paren{y}}=0.\]

\begin{enumerate}
\item Montrer que \(f\) est inversible. \\

\item Montrer : \[\quantifs{\exists\lambda\in\Rps}\ps{f\paren{x}}{f\paren{y}}=\lambda\ps{x}{y}.\]
\end{enumerate}
\end{enumerate}
\end{exo}

\begin{corr}
\note{À venir}
\end{corr}

\begin{exo}[Exercice 13, CCP PSI 2016]
On pose : \[\quantifs{\forall P,Q\in\polydeg[\R]{3}}\phi\paren{P,Q}=\int_{-1}^1P\paren{t}Q\paren{t}\odif{t}.\]

\begin{enumerate}
\item Montrer que \(\phi\) est un produit scalaire. \\

\item Calculer le projeté orthogonal de \(X^3\) sur \(\polydeg[\R]{2}\).
\end{enumerate}
\end{exo}

\begin{corr}
\note{À venir}
\end{corr}

\begin{exo}[Exercice supplémentaire]
Soient \(E\) un espace préhilbertien réel dont on note \(\norme{\cdot}\) la norme et \(p\in\Lendo{E}\) un projecteur.

Montrer : \[p\text{ est un projecteur orthogonal}\ssi\quantifs{\forall x\in E}\norme{p\paren{x}}\leq\norme{x}.\]
\end{exo}

\begin{corr}
\note{À venir}
\end{corr}

\begin{exo}[Exercice 14, sous-espace vectoriel sans supplémentaire orthogonal]
On considère le \(\R\)-espace vectoriel \(E=\ensclasse{0}{\intervii{0}{1}}{\R}\) muni de son produit scalaire usuel : \[\quantifs{\forall f,g\in E}\ps{f}{g}=\int_0^1f\paren{t}g\paren{t}\odif{t}\] et de la norme associée : \[\quantifs{\forall f\in E}\norme{f}=\sqrt{\int_0^1f^2\paren{t}\odif{t}}.\]

\begin{enumerate}[series=ssevsanssupportho]
\item Justifier que \(F=\accol{f\in E\tq f\paren{0}=0}\) est un hyperplan de \(E\).
\end{enumerate}

Soit \(f\in E\). On pose : \[\quantifs{\forall n\in\Ns}\fonction{f_n}{\intervii{0}{1}}{\R}{x}{\begin{dcases}
nf\paren{\dfrac{1}{n}}x &\text{si }x\leq\dfrac{1}{n} \\
f\paren{x} &\text{sinon}
\end{dcases}}\]

\begin{enumerate}[resume=ssevsanssupportho]
\item Montrer : \[\lim_{n\to\pinf}\norme{f_n-f}=0.\]

\item En déduire : \[F\ortho=\accol{0_E}.\]

\item \(F\) admet-il un supplémentaire orthogonal dans \(E\) ? \\

\item Que vaut \(\paren{F\ortho}\ortho\) ?
\end{enumerate}
\end{exo}

\begin{corr}
\note{À venir}
\end{corr}

\begin{exo}[Exercice 15]
Soient \(E\) un espace euclidien et \(v_1,\dots,v_p\) des vecteurs de \(E\) tels que : \[\quantifs{\forall i,j\in\interventierii{1}{p}}i\not=j\imp\ps{v_i}{v_j}<0.\]

\begin{enumerate}
\item Montrer : \[\quantifs{\forall\lambda_1,\dots,\lambda_p\in\R}\norme{\sum_{i=1}^p\abs{\lambda_i}v_i}\leq\norme{\sum_{i=1}^p\lambda_iv_i}.\]

\item Montrer que toute sous-famille à \(p-1\) éléments de \(\paren{v_1,\dots,v_p}\) est libre.
\end{enumerate}
\end{exo}

\begin{corr}
\note{À venir}
\end{corr}

\begin{exo}[Exercice 16, endomorphismes \guillemets{symétriques}]
Soit \(E\) un espace euclidien de dimension \(n\in\Ns\).

On dit qu'un endomorphisme \(u\in\Lendo{E}\) est symétrique si l'on a : \[\quantifs{\forall x,y\in E}\ps{u\paren{x}}{y}=\ps{x}{u\paren{y}}.\]

\begin{enumerate}
\item Les homothéties de \(E\) sont-elles des endomorphismes symétriques ? \\

\item Soit \(u\in\Lendo{E}\) un endomorphisme symétrique. Montrer que \(\ker u\) et \(\Im u\) sont des supplémentaires orthogonaux. \\

\item Soit \(p\in\Lendo{E}\) un projecteur. Montrer l'équivalence : \[p\text{ est un endomorphisme symétrique}\ssi p\text{ est un projecteur orthogonal}.\]

\item Soient \(u\in\Lendo{E}\) et \(\fami{B}\) une base orthonormée de \(E\). Montrer : \[u\text{ est un endomorphisme symétrique}\ssi\Mat{u}\text{ est une matrice symétrique}.\]
\end{enumerate}
\end{exo}

\begin{corr}
\note{À venir}
\end{corr}

\begin{exo}[Exercice 17, matrices \guillemets{orthogonales}]
Soient \(n\in\Ns\) et \(A=\paren{a_{ij}}_{\paren{i,j}}\in\M{n}[\R]\).

On note \(\paren{C_1,\dots,C_n}\in\paren{\R^n}^n\) la famille des colonnes de \(A\) : \[A=\begin{pmatrix}
a_{11} & \dots & a_{1n} \\
\vdots &  & \vdots \\
a_{n1} & \dots & a_{nn}
\end{pmatrix}=\begin{pmatrix}
C_1 & \dots & C_n
\end{pmatrix}.\]

La matrice \(A\) est dite orthogonale si la famille \(\paren{C_1,\dots,C_n}\) est une base orthonormée de \(\R^n\) pour le produit scalaire usuel.

\begin{enumerate}
\item Exprimer la matrice suivante en fonction de \(A\) par une formule simple : \[M=\begin{pmatrix}
\ps{C_1}{C_1} & \dots & \ps{C_1}{C_n} \\
\vdots &  & \vdots \\
\ps{C_n}{C_1} & \dots & \ps{C_n}{C_n}
\end{pmatrix}.\]

\item Compléter : \[A\text{ est une matrice orthogonale}\ssi A\text{ est inversible, d'inverse ...}\]
\end{enumerate}
\end{exo}

\begin{corr}
\note{À venir}
\end{corr}

\begin{exo}[Exercice 18]
Soit \(\sigma\in\S{\N}\).

\begin{enumerate}
\item Déterminer la nature de la série \(\sum_n\dfrac{1}{n\sigma\paren{n}}\). \\

\item Montrer que la série \(\sum_n\dfrac{\sigma\paren{n}}{n^2}\) est divergente.

\textit{Indication :} considérer les sommes partielles de la série : \(\quantifs{\forall N\in\N}S_N=\sum_{n=1}^N\dfrac{\sigma\paren{n}}{n^2}\) et minorer \(S_{2N}-S_N\).
\end{enumerate}
\end{exo}

\begin{corr}
\note{À venir}
\end{corr}

\chapter{Fonctions de deux variables réelles}

\minitoc

\begin{exo}[Exercice 1, ouverts de \(\R^2\)]
\begin{enumerate}
    \item Soit \(\paren{U_i}_{i\in I}\) une famille d'ouverts de \(\R^2\) (où \(I\) est un ensemble). Montrer que \(\bigunion_{i\in I}U_i\) est un ouvert de \(\R^2\). \\
    \item Soient \(U_1,\dots,U_n\) des ouverts de \(\R^2\) (où \(n\in\Ns\)). Montrer que \(\biginter_{k=1}^n U_i\) est un ouvert de \(\R^2\).
\end{enumerate}
\end{exo}

\begin{corr}
\note{À venir}
\end{corr}

\begin{exo}[Exercice 2]
En quels points la fonction \[\fonction{f}{\R^2}{\R}{\paren{x,y}}{\begin{dcases}
\dfrac{y^2}{x^2+y^2} &\text{si }\paren{x,y}\not=\paren{0,0} \\
0 &\text{sinon}
\end{dcases}}\] est-elle continue ?
\end{exo}

\begin{corr}
\note{À venir}
\end{corr}

\begin{exo}[Exercice 3]
En quels points la fonction \[\fonction{f}{\R^2}{\R}{\paren{x,y}}{\begin{dcases}
\dfrac{xy}{x^2+y^2} &\text{si }\paren{x,y}\not=\paren{0,0} \\
0 &\text{sinon}
\end{dcases}}\]

\begin{enumerate}
    \item En quels points \(f\) est-elle continue ? \\
    \item En quels points \(f\) admet-elle des dérivées partielles ? \\
    \item \(f\) est-elle de classe \(\classe{1}\) ?
\end{enumerate}
\end{exo}

\begin{corr}
\note{À venir}
\end{corr}

\begin{exo}[Exercice 4, à propos du théorème de Schwarz]
On pose : \[\quantifs{\forall x,y\in\R}f\paren{x,y}=\begin{dcases}
\dfrac{xy\paren{x^2-y^2}}{x^2+y^2} &\text{si }\paren{x,y}\not=\paren{0,0} \\
0 &\text{sinon}
\end{dcases}\]

\begin{enumerate}
    \item La fonction \(f:\R^2\to\R\) est-elle continue ? \\
    \item Admet-elle des dérivées partielles ? Le cas échéant, les calculer. \\
    \item Est-elle de classe \(\classe{1}\) ? \\
    \item Calculer \(\pdv{f}{y,x}\paren{0,0}\) et \(\pdv{f}{x,y}\paren{0,0}\).
\end{enumerate}
\end{exo}

\begin{corr}
\note{À venir}
\end{corr}

\begin{exo}[Exercice 5]
On pose : \[f\paren{x,y}=\Arctan x+\Arctan y-\Arctan\dfrac{x+y}{1-xy}.\]

\begin{enumerate}
    \item Donner trois ouverts \guillemets{naturels} \(U_1\), \(U_2\) et \(U_3\) tels que l'ensemble de définition de \(f\) soit \(U_1\union U_2\union U_3\). \\
    \item Montrer que \(f\) est de classe \(\classe{1}\). \\
    \item Déterminer \(f\).
\end{enumerate}
\end{exo}

\begin{corr}
\note{À venir}
\end{corr}

\begin{exo}[Exercice 6]
Dire pour chacune des équations suivantes s'il existe une solution (fonction de classe \(\classe{1}\) sur l'ensemble indiqué).

\[\quantifs{\forall\paren{x,y}\in\R^2\excluant\accol{\paren{0,0}}}\paren{E_1}~\nabla f\paren{x,y}=\dfrac{1}{x^2+y^2}\dcoords{-y}{x}\]

\[\quantifs{\forall\paren{x,y}\in\Rps\times\R}\paren{E_2}~\nabla f\paren{x,y}=\dfrac{1}{x^2+y^2}\dcoords{-y}{x}\]

\[\quantifs{\forall\paren{x,y}\in\R^2}\paren{E_3}~\nabla f\paren{x,y}=\dcoords{-y}{x}\]

\[\quantifs{\forall\paren{x,y}\in\Rps\times\R}\paren{E_4}~\nabla f\paren{x,y}=\dcoords{-y}{x}\]

\textit{Indication :} pour \(\paren{E_1}\), supposer par l'absurde qu'une telle fonction \(f\) existe et calculer la dérivée de \[\fonction{g}{\R}{\R}{\theta}{f\paren{\cos\theta,\sin\theta}}\]

\textit{Remarque :} concernant \(\paren{E_3}\) et \(\paren{E_4}\), la façon la plus simple de conclure serait d'utiliser le théorème de Schwarz que vous verrez en deuxième année.
\end{exo}

\begin{corr}
\note{À venir}
\end{corr}

\begin{exo}[Exercice 7]
Soit \(f\in\ensclasse{1}{\R^2}{\R}\) une fonction telle que : \[\quantifs{\forall t,x,y\in\R}f\paren{tx,ty}=tf\paren{x,y}.\]

\begin{enumerate}
    \item Montrer : \[\quantifs{\forall t,x,y\in\R}f\paren{x,y}=x\pdv{f}{x}\paren{tx,ty}+y\pdv{f}{y}\paren{tx,ty}.\] \\ \textit{Indication :} dériver de deux façons la fonction \[\fonction{g}{\R}{\R}{t}{f\paren{tx,ty}}\]
    \item En déduire que la fonction \(f\) est linéaire.
\end{enumerate}
\end{exo}

\begin{corr}
\note{À venir}
\end{corr}

\begin{rem}[Compléments de cours]
\begin{itemize}
    \item On dit qu'une partie \(A\subset\R^2\) est un fermé de \(\R^2\) si son complémentaire est un ouvert de \(\R^2\). \\
    \item On admet le résultat suivant, qui sera un théorème du cours de deuxième année.
\end{itemize}
\end{rem}

\begin{theo}
Soit une fonction continue \(f:A\to\R\) sur une partie \(A\subset\R^2\) non-vide, fermée et bornée.

Alors \(f\) est \guillemets{bornée et atteint ses bornes}, ce qui signifie qu'elle admet un maximum et un minimum (globaux).
\end{theo}

\begin{exo}[Exercice 8]
Étudier les extrema de la fonction \[\fonction{f}{\R^2}{\R}{\paren{x,y}}{x^2+xy+y^2-2x-y}\]
\end{exo}

\begin{corr}
\note{À venir}
\end{corr}

\begin{exo}[Exercice 9]
Étudier les extrema de la fonction \[\fonction{f}{\R^2}{\R}{\paren{x,y}}{\dfrac{x+y}{\paren{1+x^2}\paren{1+y^2}}}\]
\end{exo}

\begin{corr}
\note{À venir}
\end{corr}

\begin{exo}[Exercice 10]
Étudier les extrema de la fonction \[\fonction{f}{\R^2}{\R}{\paren{x,y}}{x^3y^2\paren{1+x-y}}\]
\end{exo}

\begin{corr}
\note{À venir}
\end{corr}

\begin{exo}[Exercice 11, CCP PSI 2015 BEOS]
Soient les ensembles \[K=\accol{\paren{x,y}\in\R^2\tq0\leq x\leq\pi\text{ et }0\leq y\leq\pi}\qquad\text{et}\qquad T=\accol{\paren{x,y}\in\R^2\tq0<x<y<\pi}.\]

On pose : \[\quantifs{\forall\paren{x,y}\in K}F\paren{x,y}=\begin{dcases}
x\paren{\pi-y} &\text{si }0\leq x\leq y\leq\pi \\
y\paren{\pi-x} &\text{si }0\leq y\leq x\leq\pi
\end{dcases}\]

\begin{enumerate}
    \item Représenter \(K\) et \(T\). \\
    \item La fonction \(F\) admet-elle des extrema locaux sur \(T\) ? \\
    \item La fonction \(F\) admet-elle un minimum sur \(K\) ? un maximum sur \(K\) ? Le cas échéant, déterminer leur valeur.
\end{enumerate}
\end{exo}

\begin{corr}
\note{À venir}
\end{corr}

\begin{exo}[Exercice 12]\thlabel{exo:EDPChangementDeVariable}
Résoudre l'équation aux dérivées partielles : \[\paren{E}~x\pdv{f}{x}+y\pdv{f}{y}=x^3y^3\] d'inconnue \(f\in\ensclasse{1}{\paren{\Rps}^2}{\R}\) à l'aide du changement de variables : \[\paren{u,v}=\paren{xy,\dfrac{x}{y}}.\]
\end{exo}

\begin{corr}
\note{À venir}
\end{corr}

\begin{exo}[Exercice 13]
Résoudre l'équation aux dérivées partielles : \[\paren{E}~x\pdv{f}{x}-y\pdv{f}{y}=\paren{x^2+y^2}\dfrac{y}{x}\] d'inconnue \(f\in\ensclasse{1}{\Rps\times\R}{\R}\), en utilisant les coordonnées polaires : \[\paren{x,y}=\paren{r\cos\theta,r\sin\theta}.\] 
\end{exo}

\begin{corr}
\note{À venir}
\end{corr}

\chapter{Probabilités}

\minitoc

\begin{exo}[Exercice 1]
Donner le nombre d'anagrammes de chacun des noms suivants : \[\text{UGO}\qquad\text{ROBIN}\qquad\text{EMMY}\qquad\text{METEHAN}\qquad\text{PRISCILLA}\]
\end{exo}

\begin{corr}
\note{À venir}
\end{corr}

\begin{exo}[Exercice 2]
Soient \(n\in\Ns\) et \(\groupe{\Omega}[\prem]\) un espace probabilisé.

On note \(\Sigma:\Omega\to\S{n}\) une permutation aléatoire telle que \(\Sigma\sim\loiuniforme{\S{n}}\).

Calculer les probabilités suivantes :

\begin{enumerate}
    \item \(\proba{\Sigma\text{ est une transposition}}\) \\
    \item \(\proba{\Sigma\text{ est un 3-cycle}}\) \\
    \item \(\proba{\Sigma\text{ est un cycle}}\)
\end{enumerate}
\end{exo}

\begin{corr}
\note{À venir}
\end{corr}

\begin{exo}[Exercice 3, suite de l'\thref{exo:td15exo11}]
On suppose que le corps \(\K\) est fini et on note \(q\) son cardinal : \[q=\Card\K<\pinf.\]

Soient \(r,n\in\Ns\) tels que \(r\leq n\) et \(E\) un \(\K\)-espace vectoriel de dimension \(n\).

\begin{enumerate}
    \item Combien \(E\) possède-t-il de bases ? \\
    \item Combien \(E\) possède-t-il de sous-espaces vectoriels de dimension \(r\) ? \\
    \item Soit \(F\) un sous-espace vectoriel de \(E\) de dimension \(r\). \\ Combien possède-t-il de supplémentaires dans \(E\) ? \\
    \item Combien y a-t-il de projecteurs de rang \(r\) dans \(\Lendo{E}\) ? \\
    \item Combien y a-t-il d'endomorphismes de rang \(r\) dans \(\Lendo{E}\) ?
\end{enumerate}
\end{exo}

\begin{corr}
\note{À venir}
\end{corr}

\begin{exo}[Exercice 4]
Il existe dans ma ville deux compagnies de taxis : les \guillemets{taxis-rouges} (5000 taxis) et les \guillemets{taxis-verts} (250 taxis).

Je suis légèrement daltonien, et j'ai oublié hier mon parapluie dans un taxi.

Il me semble que le taxi était vert, mais je sais que je ne suis pas complètement fiable : je reconnais le rouge 3 fois sur 4, et le vert 4 fois sur 5.

À quelle compagnie devrais-je téléphoner en premier ?
\end{exo}

\begin{corr}
\note{À venir}
\end{corr}

\begin{exo}[Exercice 5]
Dans une population donnée, 80\% des individus programment en Python.

Parmi ceux qui programment en Python, 10\% programment aussi en OCaml.

Parmi ceux qui programment en OCaml, 90\% programment aussi en Python.

Quelle proportion de la population programme en OCaml ?
\end{exo}

\begin{corr}
\note{À venir}
\end{corr}

\begin{exo}[Exercice 6, TPE EIVP 2018]
Un enfant possède vingt figurines qu'il place sur une étagère. Parmi celles-ci se trouvent des oiseaux et d'autres animaux.

\begin{enumerate}
    \item S'il y a deux oiseaux, quelle est la probabilité qu'ils soient côte à côte sur l'étagère ? \\
    \item S'il y a cinq oiseaux, quelle est la probabilité qu'au moins deux oiseaux soient côte à côte sur l'étagère ?
\end{enumerate}
\end{exo}

\begin{corr}
\note{À venir}
\end{corr}

\begin{exo}[Exercice 7]
On dispose de dix pièces de 1 euro.

L'une des dix pièces est fausse : la probabilité d'obtenir \guillemets{pile} lorsqu'on la lance est \(\dfrac{2}{3}\).

Les neuf vraies pièces sont équilibrées (on obtient \guillemets{pile} et \guillemets{face} avec la même probabilité lorsqu'on les lance).

\begin{enumerate}
    \item On choisit l'une des dix pièces, on la lance, et on obtient \guillemets{pile}. \\ Quelle est la probabilité que la pièce choisie soit fausse ? \\
    \item On choisit une seconde pièce (différente de la première), on la lance, et on obtient encore \guillemets{pile}. \\ Quelle est la probabilité que la seconde pièce choisie soit fausse ?
\end{enumerate}
\end{exo}

\begin{corr}
\note{À venir}
\end{corr}

\begin{exo}[Exercice 8, Centrale PSI 2015 (BEOS 1743)]
Au rez-de-chaussée d'un immeuble à \(n\) étages, \(p\) personnes prennent l'ascenseur et s'arrêtent à un étage au hasard et de manière indépendante.

On note \(X\) la variable aléatoire qui donne le nombre d'arrêts de l'ascenseur.

On note \(X_i\) la variable aléatoire qui vaut \(1\) si l'ascenseur s'arrête à l'étage \(i\) et \(0\) sinon.

\begin{enumerate}
    \item \begin{enumerate}
        \item Donner la loi des \(X_i\). \\
        \item Donner l'expression de \(X\) en fonction des \(X_i\). \\
        \item Calculer l'espérance de \(X\). \\
    \end{enumerate}
    \item Vérification avec Python du résultat obtenu. \\ \begin{enumerate}
        \item Écrire une fonction qui simule la variable aléatoire \(X\). \\
        \item Pour \(n\) variant de 3 à 20 et avec \(p=10\), vérifier le résultat de la question (1c) pour 1000 répétitions de l'expérience.
    \end{enumerate}
\end{enumerate}
\end{exo}

\begin{corr}
\note{À venir}
\end{corr}

\begin{exo}[Exercice 9]
Soient \(n,s\in\interventierie{2}{\pinf}\).

On considère une urne contenant des boules de couleurs \(C_1,\dots,C_s\).

On effectue \(n\) tirages successifs d'une boule avec remise.

Pour tout \(i\in\interventierii{1}{s}\), on note : \begin{description}
    \item[] \(p_i\) la proportion des boules de couleur \(C_i\)
    \item[] \(X_i\) le nombre de boules de couleur \(C_i\) obtenues à l'issue des \(n\) tirages.
\end{description}

\begin{enumerate}
    \item Que valent les sommes \(\sum_{i=1}^sX_i\) et \(\sum_{i=1}^sp_i\) ? \\
    \item Pour tout \(i\in\interventierii{1}{s}\), déterminer la loi de \(X_i\), son espérance et sa variance. \\
    \item Soient \(i,j\in\interventierii{1}{s}\) tels que \(i\not=j\). \\ Déterminer la loi de \(X_i+X_j\) et sa variance. \\ En déduire \(\cov{X_i}{X_j}=-np_ip_j\).
\end{enumerate}
\end{exo}

\begin{corr}
\note{À venir}
\end{corr}

\begin{exo}[Exercice 10, X-Cachan PSI 2015 (BEOS 1204)]
\underline{Exercice 1}

On considère une suite de \(n\) convertisseurs numériques fonctionnant de manière indépendante et placés en série. Chaque convertisseur restitue correctement le bit qu'on lui fournit avec la probabilité \(p\) et renvoie le bit opposé avec la probabilité \(1-p\), où \(p\in\intervii{0}{1}\).

On note \(X_k\) le bit en sortie du \(k\)-ème convertisseur et \(X_0\) le bit en entrée de chaîne.

On définit la suite finie \(\paren{A_k}_{k\in\interventierii{0}{n}}\) par \(A_k=\dcoords{\proba{X_k=1}}{\proba{X_k=0}}\).

\begin{enumerate}
    \item Déterminer une relation de récurrence pour la suite \(\paren{A_k}_{k\in\interventierii{0}{n}}\). \\
    \item En déduire la probabilité que le bit initial soit correctement rendu en sortie du \(n\)-ème convertisseur. Que se passe-t-il lorsqu'on fait tendre \(n\) vers \(\pinf\) ?
\end{enumerate}

\underline{Exercice 2} (modifié)

On considère un dé non-pipé à six faces numérotées de 1 à 6.

On considère une suite de \(n\) lancers.

\begin{enumerate}
    \item On note \(N_k\) le nombre d'apparitions de la face \(k\in\interventierii{1}{6}\) dans la suite des \(n\) lancers. \\ Intuitivement, que peut-on dire de \(N_k\) lorsque \(n\) tend vers \(\pinf\) ? \\ Rigoureusement, que donne la loi faible des grands nombres ? \\
    \item On suppose que \(n=6m\) est un multiple de \(6\). Quelle est la probabilité d'obtenir une suite de lancers telle que \[\quantifs{\forall k\in\interventierii{1}{6}}N_k=m\text{ ?}\]
    \item Vérifier le résultat de la question précédente avec Python pour \(m\in\interventierii{1}{10}\), en faisant pour chaque valeur de \(m\) une série de \(10^5\) lancers. \\ Que remarque-t-on ?
\end{enumerate}
\end{exo}

\begin{corr}
\note{À venir}
\end{corr}

\begin{exo}[Exercice 11, Centrale maths 1 2016]
On lance \(N\) fois une pièce de monnaie équilibrée.

On note \(\groupe{\Omega}[\prem]\) un espace probabilisé modélisant le problème, \(X\) le nombre de \guillemets{pile} et \(Y\) le nombre de \guillemets{face} obtenus.

\begin{enumerate}
    \item On suppose que \(N\) est un entier naturel non-nul fixé. \\ \begin{enumerate}
        \item Calculer la covariance de \(X\) et de \(Y\). \\
        \item Les variables aléatoires \(X\) et \(Y\) sont-elles indépendantes ? \\
    \end{enumerate}
    \item ... \\
    \item ...
\end{enumerate}
\end{exo}

\begin{corr}
\note{À venir}
\end{corr}

\begin{exo}[Exercice 12]
Soient \(n\in\Ns\).

On note \(\Sigma:\Omega\to\S{n}\) une permutation aléatoire telle que \(\Sigma\sim\loiuniforme{\S{n}}\) et \(N\) le nombre de points fixes de \(\Sigma\) : c'est un entier aléatoire à valeurs dans \(\interventierii{0}{n}\).

Dans la suite, on calcule l'espérance et la variance de \(N\).

Pour cela, on définit des variables aléatoires \(X_1,\dots,X_n\) en posant : \[\quantifs{\forall k\in\interventierii{1}{n}}X_k=\begin{dcases}
1 &\text{si }k\text{ est un point fixe de }\Sigma \\
0 &\text{sinon}
\end{dcases}\] de sorte qu'on a : \[N=X_1+\dots+X_n.\]

\begin{enumerate}
    \item Que vaut \(\proba{N=n}\) ? \\
    \item Donner les lois des variables aléatoires \(X_1,\dots,X_n\). \\ Sont-elles mutuellement indépendantes ? \\
    \item Calculer l'espérance de \(N\). \\
    \item Soient deux entiers distincts \(i,j\in\interventierii{1}{n}\). Calculer l'espérance de \(X_iX_j\). \\ En déduire la covariance de \(X_i\) et \(X_j\). \\ Sont-elles indépendantes ? \\
    \item Calculer la variance de \(N\). \\
    \item Montrer : \[\proba{N\geq4}\leq\dfrac{1}{9}.\]
\end{enumerate}
\end{exo}

\begin{corr}
\note{À venir}
\end{corr}
\end{document}
