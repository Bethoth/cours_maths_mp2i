\chapter{Intégrales sur un segment}

\minitoc

Dans ce chapitre, on définit l'intégrale d'une fonction continue par morceaux sur un segment. L'étude des intégrales de fonctions sur un intervalle autre qu'un segment sera faite en deuxième année.

Dans tout le chapitre, on pose \(\K=\R\) ou \(\C\) et on considère un segment \(\intervii{a}{b}\) de \(\R\) (avec \(a,b\in\R\) tels que \(a<b\)).

\section{Définitions}

\subsection{Subdivisions d'un segment}

\begin{defi}
On appelle subdivision de \(\intervii{a}{b}\) toute suite finie strictement croissante \(\sigma=\paren{x_0,x_1,\dots,x_n}\) d'éléments de \(\intervii{a}{b}\) telle que \[a=x_0<x_1<\dots<x_{n-1}<x_n=b.\]

L'application qui à une telle subdivision associe l'ensemble fini \(\accol{x_1;x_2;\dots;x_{n-1}}\) est une bijection de l'ensemble des subdivisions de \(\intervii{a}{b}\) vers l'ensemble des parties finies de \(\intervee{a}{b}\).

En pratique, on s'autorise à voir les subdivisions comme des ensembles finis et à parler, par exemple, de la subdivision réunion de deux subdivisions.
\end{defi}

\begin{defi}
Soient \(\sigma\) et \(\sigma\prim\) deux subdivisions du segment \(\intervii{a}{b}\).

On dit que \(\sigma\prim\) est plus fine que \(\sigma\) si on a \[\sigma\subset\sigma\prim.\]

\Cad, du point de vue des suites finies : \(\sigma\) est une suite extraite de \(\sigma\prim\).
\end{defi}

\begin{rem}
Pour toutes subdivisions \(\sigma_1\) et \(\sigma_2\) de \(\intervii{a}{b}\), il existe une subdivision \(\sigma_3\) plus fine que \(\sigma_1\) et \(\sigma_2\).
\end{rem}

\begin{defi}
Soit \(n\in\Ns\).

On appelle subdivision régulière de pas \(\dfrac{1}{n}\) la subdivision \(a=x_0<x_1<\dots<x_{n-1}<x_n=b\) définie par : \[\quantifs{\forall k\in\interventierii{0}{n}}x_k=a+\dfrac{k}{n}\paren{b-a}.\]
\end{defi}

\subsection{Fonctions en escalier}

\begin{defi}
On dit qu'une fonction \(\phi:\intervii{a}{b}\to\K\) est en escalier s'il existe une subdivision \(a=x_0<x_1<\dots<x_{n-1}<x_n=b\) de \(\intervii{a}{b}\) telle que : \[\quantifs{\forall j\in\interventierii{0}{n-1}}\restr{\phi}{\intervee{x_j}{x_{j+1}}}\text{ est constante},\] \cad si : \[\quantifs{\forall j\in\interventierii{0}{n-1};\forall t\in\intervee{x_j}{x_{j+1}}}\phi\paren{t}=\phi\paren{\dfrac{x_j+x_{j+1}}{2}}.\]

Une telle subdivision est dite adaptée à \(\phi\).
\end{defi}

\begin{ex}
La restriction à tout segment de la fonction \guillemets{partie entière} est en escalier.

La restriction à tout segment de la fonction \guillemets{signum} est en escalier.
\end{ex}

\begin{nota}[Non-officielle]
On notera \(\Esc\) l'ensemble des fonctions en escalier de \(\intervii{a}{b}\) vers \(\K\).

C'est un sous-anneau de \(\anneau{\F{\intervii{a}{b}}{\K}}\).
\end{nota}

\begin{rem}
Soient \(\phi_1,\phi_2\in\F{\intervii{a}{b}}{\R}\).

On pose \(\phi=\phi_1+\i\phi_2\in\F{\intervii{a}{b}}{\C}\).

Alors la fonction \(\phi\) est en escalier si, et seulement si, \(\phi_1\) et \(\phi_2\) sont en escalier.
\end{rem}

\begin{rem}
Soit \(\phi\in\F{\intervii{a}{b}}{\R}\) une fonction en escalier.

Alors l'ensemble image \(\Im\phi\) est fini.

En particulier, la fonction \(\phi\) est bornée.
\end{rem}

\subsection{Fonctions continues par morceaux}

\begin{defi}[Fonction continue par morceaux sur un segment]
On dit qu'une fonction \(f:\intervii{a}{b}\to\K\) est continue par morceaux s'il existe une subdivision \(a=x_0<x_1<\dots<x_{n-1}<x_n=b\) de \(\intervii{a}{b}\) telle que : \[\quantifs{\forall j\in\interventierii{0}{n-1}}\restr{f}{\intervee{x_j}{x_{j+1}}}\text{ admet un prolongement continu à }\intervii{x_j}{x_{j+1}},\] \cad si \[\begin{dcases}\quantifs{\forall j\in\interventierii{0}{n-1}}\restr{f}{\intervee{x_j}{x_{j+1}}}\text{ est continue} \\ \quantifs{\forall j\in\interventierii{0}{n-1}}f\text{ admet une limite finie à droite en }x_j \\ \quantifs{\forall j\in\interventierii{1}{n}}f\text{ admet une limite finie à gauche en }x_j\end{dcases}\]

Une telle subdivision est dite adaptée à \(f\).
\end{defi}

\begin{ex}\thlabel{ex:fonctionsContinuesParMorceauxOuPas}
Toute fonction en escalier est continue par morceaux.

Toute fonction continue est continue par morceaux.

La fonction \(\fonction{f}{\intervii{0}{1}}{\R}{x}{\begin{dcases}\sin\dfrac{1}{x} &\text{si }x\not=0 \\ 0 &\text{si }x=0\end{dcases}}\) n'est pas continue par morceaux (elle est continue sur \(\intervei{0}{1}\) mais n'admet pas de limite finie en \(0^+\)).
\end{ex}

\begin{nota}[Non-officielle]
On notera \(\contm\) l'ensemble des fonctions continues par morceaux de \(\intervii{a}{b}\) vers \(\K\).

C'est un sous-anneau de l'anneau \(\anneau{\F{\intervii{a}{b}}{\K}}\).
\end{nota}

\begin{rem}
Soient \(\phi_1,\phi_2\in\F{\intervii{a}{b}}{\R}\).

On pose \(\phi=\phi_1+\i\phi_2\in\F{\intervii{a}{b}}{\C}\).

Alors la fonction \(\phi\) est continue par morceaux si, et seulement si, \(\phi_1\) et \(\phi_2\) sont continues par morceaux.
\end{rem}

\begin{rem}
Toute fonction continue par morceaux sur un segment est bornée (mais n'atteint pas forcément ses bornes).
\end{rem}

\begin{rem}
Toute fonction continue par morceaux est la somme d'une fonction continue et d'une fonction en escalier.
\end{rem}

\begin{dem}
On raisonne par récurrence sur le nombre (nécessairement fini) de points où la fonction continue par morceaux n'est pas continue.

\note{Exercice} : montrer que l'écriture comme somme d'une fonction continue et d'une fonction en escalier est unique à une constante additive près.
\end{dem}

\subsection{Intégrales}

\subsubsection{Intégrale d'une fonction en escalier}

\begin{defi}
Soient \(\phi\in\Esc\) et \(a=x_0<x_1<\dots<x_{n-1}<x_n=b\) une subdivision adaptée à \(\phi\).

La valeur de la somme \[\sum_{j=0}^{n-1}\paren{x_{j+1}-x_j}\phi\paren{\dfrac{x_j+x_{j+1}}{2}}\] ne dépend pas de la subdivision choisie.

On l'appelle intégrale de \(\phi\) sur \(\intervii{a}{b}\) et on la note : \[\int_{\intervii{a}{b}}\phi\qquad\text{ou}\qquad\int_a^b\phi\paren{t}\odif{t}\qquad\text{ou}\qquad\int_a^b\phi.\]
\end{defi}

\begin{ex}
On a : \[\begin{aligned}
\int_{-1}^2\floor{t}\odif{t}&=\sum_{j=0}^1\paren{x_{j+1}-x_j}\floor{\dfrac{x_j+x_{j+1}}{2}} \\
&=\paren{0+1}\floor{\dfrac{0-1}{2}}+\paren{1-0}\floor{\dfrac{1+0}{2}}+\paren{2-1}\floor{\dfrac{2+1}{2}} \\
&=-1+0+1 \\
&=0
\end{aligned}\]
\end{ex}

\begin{rem}
Les propriétés suivantes (laissées en exercice) sont utiles pour définir l'intégrale d'une fonction continue par morceaux.
\end{rem}

\begin{prop}[Linéarité de l'intégrale]
On a : \[\quantifs{\forall\lambda,\mu\in\K;\forall\phi,\psi\in\Esc}\int_{\intervii{a}{b}}\paren{\lambda\phi+\mu\psi}=\lambda\int_{\intervii{a}{b}}\phi+\mu\int_{\intervii{a}{b}}\psi.\]

On dit que l'application \(\fonctionlambda{\Esc}{\K}{\phi}{\int_{\intervii{a}{b}}\phi}\) est linéaire (cela signifie que l'image d'une combinaison linéaire est la combinaison linéaire des images).
\end{prop}

\begin{prop}[Positivité et croissance de l'intégrale]\thlabel{prop:positivitéEtCroissanceDeLIntégraleDUneFonctionEnEscalier}
Soient \(a,b\in\R\) tels que \(a\leq b\) et \(\phi,\psi\in\Esc\).

On a \[\phi\geq0\imp\int_a^b\phi\paren{t}\odif{t}\geq0\qquad\text{(positivité)}\] et \[\phi\leq\psi\imp\int_a^b\phi\paren{t}\odif{t}\leq\int_a^b\psi\paren{t}\odif{t}\qquad\text{(croissance).}\]
\end{prop}

\subsubsection{Intégrale d'une fonction continue par morceaux}

\begin{deftheo}[Cas d'une fonction à valeurs réelles]
\newcommand{\Em}{\mathcal{E}_-}
\newcommand{\Ep}{\mathcal{E}_+}
Soit \(f\in\contm\).

On note \(\Em\) l'ensemble des fonctions en escalier inférieurs à \(f\) : \[\Em=\accol{\phi\in\Esc\tq\phi\leq f}\] et \(\Ep\) l'ensemble des fonctions en escalier supérieures à \(f\) : \[\Ep=\accol{\psi\in\Esc\tq f\leq\psi}.\]

Les bornes supérieure et inférieure suivantes sont bien définies : \[I_-\paren{f}=\sup_{\phi\in\Em}\int_{\intervii{a}{b}}\phi\qquad\text{et}\qquad I_+\paren{f}=\inf_{\psi\in\Ep}\int_{\intervii{a}{b}}\psi.\]

De plus, elles ont la même valeur, que l'on appelle intégrale de \(f\) sur \(\intervii{a}{b}\) et qu'on note : \[\int_{\intervii{a}{b}}f\qquad\text{ou}\qquad\int_a^bf\paren{t}\odif{t}\qquad\text{ou}\qquad\int_a^bf.\]

En résumé : \[\int_{\intervii{a}{b}}f=I_-\paren{f}=I_+\paren{f}.\]
\end{deftheo}

\begin{dem}~\\
\newcommand{\Em}{\mathcal{E}_-}
\newcommand{\Ep}{\mathcal{E}_+}
L'ensemble \(\accol{\int_{\intervii{a}{b}}\phi}_{\phi\in\Em}\) est une partie non-vide de \(\R\) car \(f\) est continue par morceaux sur son segment, donc bornée, donc minorée par un réel \(m\in\R\) et donc la fonction constante égale à \(m\) appartient à \(\Em\), et majorée car \(f\) est majorée par un réel \(M\in\R\) qui vérifie : \[\quantifs{\forall\phi\in\Em}\phi\leq f\leq M\] et donc, selon la \thref{prop:positivitéEtCroissanceDeLIntégraleDUneFonctionEnEscalier} : \[\quantifs{\forall\phi\in\Em}\int_{\intervii{a}{b}}\phi\leq\int_{\intervii{a}{b}}M=\paren{b-a}M.\] Donc cet ensemble admet une borne supérieure. Donc \(I_-\paren{f}\) est bien définie.

Idem pour \(I_+\paren{f}\).

Montrons que \(I_-\paren{f}=I_+\paren{f}\).

\leqbox

On a \[\quantifs{\forall\phi\in\Em;\forall\psi\in\Ep}\phi\leq f\leq\psi.\]

D'où, selon la \thref{prop:positivitéEtCroissanceDeLIntégraleDUneFonctionEnEscalier} : \[\quantifs{\forall\phi\in\Em;\forall\psi\in\Ep}\int_{\intervii{a}{b}}\phi\leq\int_{\intervii{a}{b}}\psi.\]

Donc, par définition de \(I_+\paren{f}\) : \[\quantifs{\forall\phi\in\Em}\int_{\intervii{a}{b}}\phi\leq I_+\paren{f}.\]

Et donc, par définition de \(I_-\) : \[I_-\paren{f}\leq I_+\paren{f}.\]

\geqbox

Soit \(\epsilon\in\Rps\).

Comme \(f\) est continue par morceaux, il existe \(f_0\in\ensclasse{0}{\intervii{a}{b}}{\R}\) et \(\phi_0\in\Esc\) telles que \(f=f_0+\phi_0\).

Comme \(f_0\) est continue sur un segment, elle est uniformément continue d'après le théorème de Heine.

Il existe donc \(\delta\in\Rps\) tel que \[\quantifs{\forall x,y\in\intervii{a}{b}}\abs{x-y}\leq\delta\imp\abs{f_0\paren{x}-f_0\paren{y}}\leq\epsilon.\]

Soit \(n\in\Ns\) tel que \(\dfrac{b-a}{n}\leq\delta\) (un tel entier existe car \(\lim_n\dfrac{b-a}{n}=0\)).

On définit \(\phi\in\Em\) et \(\psi\in\Ep\) en posant \(\quantifs{\forall k\in\interventierii{0}{n}}x_k=a+k\dfrac{b-a}{n}\) et  \(\phi\paren{b}=\psi\paren{b}=f\paren{b}\) et \[\quantifs{\forall k\in\interventierii{0}{n-1};\forall t\in\intervie{x_k}{x_{k+1}}}\begin{dcases}\phi\paren{t}=\min_{\intervie{x_k}{x_{k+1}}}f_0 \\ \psi\paren{t}=\max_{\intervie{x_k}{x_{k+1}}}f_0\end{dcases}\]

De plus, on a : \[\quantifs{\forall k\in\interventierii{0}{n-1};\forall x,y\in\intervie{x_k}{x_{k+1}}}\abs{x-y}\leq\dfrac{b-a}{n}\leq\delta.\]

Donc, par définition de \(\delta\) : \[\quantifs{\forall k\in\interventierii{0}{n-1}}\abs{\min_{\intervie{x_k}{x_{k+1}}}f_0-\max_{\intervie{x_k}{x_{k+1}}}f_0}\leq\epsilon.\]

D'où \(\quantifs{\forall t\in\intervii{a}{b}}\abs{\phi\paren{t}-\psi\paren{t}}\leq\epsilon\) donc \(\quantifs{\forall t\in\intervii{a}{b}}\psi\paren{t}-\phi\paren{t}\leq\epsilon\).

D'où \(\psi\leq\phi+\epsilon\).

Finalement, on a obtenu : \[\begin{dcases}\phi\leq f_0\leq\psi \\ \psi\leq\phi+\epsilon\end{dcases}\]

D'où, en posant \(\phi_1=\phi+\phi_0\) et \(\psi_1=\psi+\psi_0\) : \[\begin{dcases}\phi_1\leq f\leq\psi_1 \\ \psi_1\leq\phi_1+\epsilon\end{dcases}\]

On a donc \(\phi_1\in\Em\) et \(\psi_1\in\Ep\) donc \[\begin{dcases}\int_a^b\phi_1\leq I_-\paren{f} \\ \int_a^b\psi_1\geq I_+\paren{f}\end{dcases}\]

Or \(\psi_1\leq\phi_1+\epsilon\) donc \[\int_a^b\psi_1\leq\int_a^b\phi_1+\paren{b-a}\epsilon.\]

D'où \[I_+\paren{f}\leq\int_a^b\psi_1\leq\int_a^b\phi_1+\paren{b-a}\epsilon\leq I_-\paren{f}+\paren{b-a}\epsilon.\]

D'où, avec \(\epsilon\to0\) : \[I_+\paren{f}\leq I_-\paren{f}.\]

Donc \(I_-\paren{f}=I_+\paren{f}\).
\end{dem}

\begin{rem}
\newcommand{\Em}{\mathcal{E}_-}
\newcommand{\Ep}{\mathcal{E}_+}
Seule l'intégration des fonctions continues par morceaux est au programme, mais la construction précédente (due au mathématicien Gaston Darboux) s'applique à toute fonction \(f\) bornée sur \(\intervii{a}{b}\) (afin que les ensembles \(\Em\) et \(\Ep\) ne soient pas vides) et telle que \(I_-\paren{f}=I_+\paren{f}\). Une telle fonction est dite Riemann-intégrable ou Darboux-intégrable.

Par exemple, la fonction \(f\) de l'\thref{ex:fonctionsContinuesParMorceauxOuPas} n'est pas continue par morceaux mais est Riemann-intégrable.

Il existe une autre théorie plus récente (1901) et plus satisfaisante pour définir l'intégrale d'une fonction réelle ou complexe, due à Henri Lebesgue (hors-programme).
\end{rem}

\begin{defi}[Cas d'une fonction à valeurs complexes]
Soit \(f\in\contm[\intervii{a}{b}][\C]\).

On note \(f_1,f_2\in\contm[\intervii{a}{b}][\R]\) les fonctions réelles telles que \(f=f_1+\i f_2\).

L'intégrale de \(f\) sur \(\intervii{a}{b}\) est définie par : \[\int_{\intervii{a}{b}}f=\int_{\intervii{a}{b}}f_1+\i\int_{\intervii{a}{b}}f_2.\]
\end{defi}

\begin{rem}[Pour se ramener à des fonctions continues]
Soient \(f\in\contm\) et \(a=x_0<x_1<\dots<x_{n-1}<x_n=b\) une subdivision adaptée à \(f\), \cad telle que pour tout \(j\in\interventierii{0}{n-1}\), la fonction \(\restr{f}{\intervee{x_j}{x_{j+1}}}\) admet un prolongement continu \(g_j:\intervii{x_j}{x_{j+1}}\to\K\).

L'intégrale de \(f\) sur \(\intervii{a}{b}\) est alors \[\int_{\intervii{a}{b}}f=\int_a^bf\paren{t}\odif{t}=\sum_{j=0}^{n-1}\int_{x_j}^{x_{j+1}}g_j\paren{t}\odif{t}.\]
\end{rem}

\begin{nota}[Généralisation]
Soit \(f\in\contm\).

On pose : \[\quantifs{\forall x,y\in\intervii{a}{b}}\int_x^yf\paren{t}\odif{t}=\begin{dcases}\int_{\intervii{x}{y}}f &\text{si }x<y \\ -\int_{\intervii{x}{y}}f &\text{si }x>y \\ 0 &\text{si }x=y\end{dcases}\]
\end{nota}

\section{Propriétés}

\subsection{Premières propriétés}

\begin{prop}[Linéarité de l'intégrale]
On a : \[\quantifs{\forall\lambda,\mu\in\K;\forall f,g\in\contm}\int_{\intervii{a}{b}}\paren{\lambda f+\mu g}=\lambda\int_{\intervii{a}{b}}f+\mu\int_{\intervii{a}{b}}g.\]

On dit que l'application \(\fonctionlambda{\contm}{\K}{f}{\int_{\intervii{a}{b}}f}\) est linéaire.
\end{prop}

\begin{rem}
Soient \(f,g\in\contm\).

Si \(f\) et \(g\) coïncident sauf en un nombre fini de points, alors \[\int_{\intervii{a}{b}}f=\int_{\intervii{a}{b}}g.\]
\end{rem}

\begin{dem}
\(g-f\) est une fonction en escalier nulle partout sauf en un nombre fini de points donc \(\int_a^b\paren{g-f}=0\).

D'où, par linéarité de l'intégrale : \(\int_a^bg-\int_a^bf=0\).
\end{dem}

\begin{prop}[Relation de Chasles]
Soit \(f\in\contm\).

On a : \[\quantifs{\forall x,y,z\in\intervii{a}{b}}\int_x^yf\paren{t}\odif{t}=\int_x^zf\paren{t}\odif{t}+\int_z^yf\paren{t}\odif{t}.\]
\end{prop}

\begin{prop}[Inégalité triangulaire intégrale\protect\footnotemark]
\footnotetext{Dans les programmes précédents (donc dans les annales de concours), cette inégalité était appelée \guillemets{inégalité de la moyenne}.}
On a : \[\quantifs{\forall f\in\contm}\abs{\int_{\intervii{a}{b}}f}\leq\int_{\intervii{a}{b}}\abs{f}.\]
\end{prop}

\begin{prop}[Positivité et croissance de l'intégrale]
Soient \(a,b\in\R\) tels que \(a\leq b\) et \(f,g\in\contm\).

On a \[f\geq0\imp\int_a^bd\paren{t}\odif{t}\geq0\qquad\text{(positivité)}\] et \[f\leq g\imp\int_a^bf\paren{t}\odif{t}\leq\int_a^bg\paren{t}\odif{t}\qquad\text{(croissance)}.\]
\end{prop}

\begin{dem}[Positivité]
\newcommand{\Em}{\mathcal{E}_-}
Supposons \(f\geq0\).

La fonction nulle \(\fonction{\phi_0}{\intervii{a}{b}}{\K}{t}{0}\) appartient à \(\Em\) donc on a \[\int_a^bf=\sup_{\phi\in\Em}\int_a^b\phi\geq\int_a^b\phi_0.\]
\end{dem}

\begin{dem}[Croissance]
Supposons \(f\leq g\).

Alors \(0\leq g-f\).

Donc selon la positivité de l'intégrale, on a \(0\leq\int_a^b\paren{g-f}\).

D'où, par linéarité de l'intégrale : \(0\leq\int_a^bg-\int_a^bf\).
\end{dem}

\begin{prop}
On suppose \(a<b\).

Soit \(f:\intervii{a}{b}\to\R\) continue.

On a \[\croch{f\geq0\quad\text{et}\quad\int_a^bf\paren{t}\odif{t}=0}\imp f=0.\]

NB : l'implication est fausse pour les fonctions continues par morceaux.
\end{prop}

\begin{dem}
On suppose \(f\) continue et positive.

Montrons que \(\int_a^bf\paren{t}\odif{t}=0\imp f=0\) par contraposée.

Supposons \(f\not=0\).

Montrons que \(\int_a^bf\paren{t}\odif{t}\not=0\).

Soit \(x_0\in\intervii{a}{b}\) tel que \(f\paren{x_0}\not=0\).

Supposons \(x_0\not=a\) et \(x_0\not=b\).

On a \(\lim_{x\to x_0}f\paren{x}=f\paren{x_0}\) car \(f\) est continue en \(x_0\) et \(f\paren{x_0}>0\) car \(f\paren{x_0}\not=0\) et \(f\geq0\).

Donc \(\dfrac{f\paren{x_0}}{2}<\lim_{x\to x_0}f\paren{x}\).

Il existe \(\delta\in\Rps\) tel que \[\begin{dcases}\intervee{x_0-\delta}{x_0+\delta}\subset\intervii{a}{b} \\ \quantifs{\forall t\in\intervee{x_0-\delta}{x_0+\delta}}\dfrac{f\paren{x_0}}{2}<f\paren{t}\end{dcases}\] car \(\intervii{a}{b}\) est un voisinage de \(x_0\) et car \(f\) est continue en \(x_0\).

Posons \(\fonction{\phi}{\intervii{a}{b}}{\R}{t}{\begin{dcases}\dfrac{f\paren{x_0}}{2} &\text{si }\abs{t-x_0}\leq\delta \\ 0 &\text{sinon}\end{dcases}}\)

On a \(\phi\leq f\) donc par croissance de l'intégrale, on a \[0<\delta f\paren{x_0}=2\delta\dfrac{f\paren{x_0}}{2}=\int_a^b\phi\leq\int_a^bf.\]

De même si \(x_0=a\) ou \(x_0=b\), on obtient \(\int_a^bf\geq\delta\dfrac{f\paren{x_0}}{2}>0\).
\end{dem}

\subsection{Sommes de Riemann}

\begin{prop}[Convergence des sommes de Riemann]
Soit \(f\in\contm\).

On a : \[\lim_{n\to\pinf}\dfrac{b-a}{n}\sum_{k=1}^{n}f\paren{a+k\dfrac{b-a}{n}}=\lim_{n\to\pinf}\dfrac{b-a}{n}\sum_{k=0}^{n-1}f\paren{a+k\dfrac{b-a}{n}}=\int_a^bf\paren{t}\odif{t}.\]
\end{prop}

\begin{dem}[Cas où \(f\) est une fonction lipschitzienne]~\\
On pose \(\quantifs{\forall k\in\interventierii{0}{n-1}}x_k=a+k\dfrac{b-a}{n}\).

On pose \(\quantifs{\forall n\in\Ns}R_n\paren{f}=\dfrac{b-a}{n}\sum_{k=0}^{n-1}f\paren{x_k}\).

On a : \[\begin{aligned}
\int_a^bf\paren{t}\odif{t}-R_n\paren{f}&=\sum_{k=0}^{n-1}\int_{x_k}^{x_{k+1}}f\paren{t}\odif{t}-\sum_{k=0}^{n-1}\int_{x_k}^{x_{k+1}}f\paren{x_k}\odif{t} \\
&=\sum_{k=0}^{n-1}\int_{x_k}^{x_{k+1}}\croch{f\paren{t}-f\paren{x_k}}\odif{t}.
\end{aligned}\]

Soit \(K\in\Rp\) tel que \(f\) est \(K\)-lipschitzienne.

On a : \[\quantifs{\forall k\in\interventierii{0}{n-1};\forall t\in\intervii{x_k}{x_{k+1}}}\abs{f\paren{t}-f\paren{x_k}}\leq K\abs{t-x_k}.\]

Donc on a : \[\begin{aligned}
\quantifs{\forall k\in\interventierii{0}{n-1}}\abs{\int_{x_k}^{x_{k+1}}\croch{f\paren{t}-f\paren{x_k}}\odif{t}}&\leq\int_{x_k}^{x_{k+1}}\abs{f\paren{t}-f\paren{x_k}}\odif{t}\qquad\text{(inégalité triangulaire intégrale)} \\
&\leq\int_{x_k}^{x_{k+1}}K\paren{t-x_k}\odif{t}\qquad\text{(croissance de l'intégrale)} \\
&=\croch{K\dfrac{\paren{t-x_k}^2}{2}}_{x_k}^{x_{k+1}} \\
&=\dfrac{K}{2}\paren{\dfrac{b-a}{n}}^2
\end{aligned}\]

D'où : \[\begin{aligned}
\abs{\int_a^bf\paren{t}\odif{t}-R_n\paren{f}}&\leq\sum_{k=0}^{n-1}\abs{\int_{x_k}^{x_{k+1}}\croch{f\paren{t}-f\paren{x_k}}\odif{t}} \\
&\leq\sum_{k=0}^{n-1}\dfrac{K}{2}\dfrac{\paren{b-a}^2}{n^2} \\
&=\dfrac{K}{2}\dfrac{\paren{b-a}^2}{n}.
\end{aligned}\]

Or \(\lim_n\dfrac{K}{2}\dfrac{\paren{b-a}^2}{n}=0\).

Donc selon le \hyperref[theo:gendarmes]{théorème des gendarmes}, \(\lim_nR_n\paren{f}=\int_a^bf\paren{t}\odif{t}\).

L'autre limite en découle car on a \[\dfrac{b-a}{n}\sum_{k=1}^{n}f\paren{x_k}=R_n\paren{f}+\dfrac{b-a}{n}\paren{f\paren{b}-f\paren{a}}\quad\text{et}\quad\begin{dcases}\lim_nR_n\paren{f}=\int_a^bf\paren{t}\odif{t} \\ \lim_n\dfrac{b-a}{n}\paren{f\paren{b}-f\paren{a}}=0\end{dcases}\].
\end{dem}

\begin{dem}[Cas où \(f\) est une fonction continue]\thlabel{dem:convergenceDesSommesDeRiemannDansLeCasContinu}
Cf. \thref{exo:convergenceDesSommesDeRiemannDansLeCasContinu}.
\end{dem}

\subsection{Théorème fondamental}

\begin{defi}
On appelle primitive d'une fonction \(f\) toute fonction dérivable \(F\) dont la dérivée est \(f\) : \[F\prim=f.\]
\end{defi}

\begin{prop}[Unicité à une constante additive près]
Soient \(I\) un intervalle de \(\R\) et \(f:I\to\K\).

On suppose que \(F_1,F_2\in\F{I}{\K}\) sont des primitives de \(f\).

Alors \[\quantifs{\exists c\in\K}F_2=F_1+c.\]
\end{prop}

\begin{dem}
On a \(\paren{F_2-F_1}\prim=f-f=0\).

Donc la fonction \(F_2-F_1\) est constante sur l'intervalle \(I\) : \(\quantifs{\exists c\in\K}F_2-F_1=c\).

Donc \(\quantifs{\exists c\in\K}F_2=F_1+c\).
\end{dem}

\begin{theo}[Théorème fondamental de l'analyse]
Soient \(I\) un intervalle de \(\R\), \(f:I\to\K\) continue et \(x_0\in I\).

Alors la fonction \[\fonction{F}{I}{\K}{x}{\int_{x_0}^xf\paren{t}\odif{t}}\] est une primitive de \(f\) sur \(I\) (c'est l'unique primitive de \(f\) qui s'annule en \(x_0\)).
\end{theo}

\begin{dem}
Soit \(x\in I\).

Montrons que \(\lim_{\substack{x\prim\to x \\ x\prim\not=x}}\dfrac{F\paren{x\prim}-F\paren{x}}{x\prim-x}=f\paren{x}\), \cad \[\quantifs{\forall\epsilon\in\Rps;\exists\delta\in\Rps;\forall x\prim\in I}\abs{x\prim-x}\leq\delta\imp\abs{\dfrac{F\paren{x\prim}-F\paren{x}}{x\prim-x}-f\paren{x}}\leq\epsilon.\]

Soit \(\epsilon\in\Rps\).

Soit \(\delta\in\Rps\) tel que \[\quantifs{\forall x\seconde\in I}\abs{x-x\seconde}\leq\delta\imp\abs{f\paren{x}-f\paren{x\seconde}}\leq\epsilon.\]

Un tel \(\delta\) existe car \(f\) est continue en \(x\).

Soit \(x\prim\in I\) tel que \(\abs{x-x\prim}\leq\delta\).

On a, selon la relation de Chasles : \[F\paren{x\prim}-F\paren{x}=\int_x^{x\prim}f\paren{t}\odif{t}.\]

D'où \[\begin{aligned}
\abs{F\paren{x\prim}-F\paren{x}-\paren{x\prim-x}f\paren{x}}&=\abs{\int_x^{x\prim}f\paren{t}\odif{t}-\int_x^{x\prim}f\paren{x}\odif{t}} \\
&=\abs{\int_x^{x\prim}\paren{f\paren{t}-f\paren{x}}\odif{t}} \\
&\leq\abs{\int_x^{x\prim}\abs{f\paren{t}-f\paren{x}}\odif{t}} \\
&\leq\abs{\int_x^{x\prim}\epsilon\odif{t}} \\
&=\epsilon\abs{x\prim-x}.
\end{aligned}\]

D'où, si \(x\not=x\prim\) : \[\abs{\dfrac{F\paren{x\prim}-F\paren{x}}{x\prim-x}}\leq\epsilon.\]

Donc \(\delta\) convient.

D'où \(F\prim\paren{x}=f\paren{x}\).
\end{dem}

\begin{cor}[Existence de primitives]
Toute fonction continue sur un intervalle admet une primitive.
\end{cor}

\begin{exo}
Donner une primitive des fonctions suivantes (à l'aide d'une intégrale) :

\begin{enumerate}
\item \(\Arctan:\R\to\R\) \\

\item \(f:x\mapsto\dfrac{1}{\sqrt{\Arcsin x}}\)
\end{enumerate}
\end{exo}

\begin{corr}[1]
On a la primitive de \(\Arctan\) suivante : \[\fonctionlambda{\R}{\R}{x}{\int_0^x\Arctan t\odif{t}}\]
\end{corr}

\begin{corr}[2]
\(f\) est bien définie sur \(\intervei{0}{1}\).

On a la primitive suivante : \[\fonctionlambda{\intervei{0}{1}}{\R}{x}{\int_1^x\dfrac{\odif{t}}{\sqrt{\Arcsin t}}}\]
\end{corr}

\begin{cor}\thlabel{cor:intégraleEgaleDifférencePrimitive}
Soient \(f:\intervii{a}{b}\to\K\) continue et \(F:\intervii{a}{b}\to\K\) une primitive de \(f\).

Alors \[\int_a^bf\paren{t}\odif{t}=F\paren{b}-F\paren{a}.\]

L'accroissement \(F\paren{b}-F\paren{a}\) est noté : \[\croch{F\paren{t}}_a^b\qquad\text{ou}\qquad\croch{F\paren{t}}_{t=a}^b.\]
\end{cor}

\begin{dem}
Posons \(\fonction{G}{\intervii{a}{b}}{\K}{x}{\int_a^xf\paren{t}\odif{t}}\)

Les fonctions \(F\) et \(G\) sont des primitives de \(f\) sur l'intervalle \(\intervii{a}{b}\).

Il existe donc \(c\in\K\) tel que \(\quantifs{\forall x\in\intervii{a}{b}}G\paren{x}=F\paren{x}+c\).

En particulier, si \(x=a\) : \(0=G\paren{a}=F\paren{a}+c\).

Donc \(c=-F\paren{a}\) et \(\quantifs{\forall x\in\intervii{a}{b}}G\paren{x}=F\paren{x}-F\paren{a}\).

D'où, en prenant \(x=b\) : \[\int_a^bf\paren{t}\odif{t}=G\paren{b}=F\paren{b}-F\paren{a}.\]
\end{dem}

\begin{cor}
Soit \(g\in\ensclasse{1}{\intervii{a}{b}}{\K}\).

On a : \[g\paren{b}-g\paren{a}=\int_a^bg\prim\paren{t}\odif{t}.\]
\end{cor}

\begin{dem}
Découle du \thref{cor:intégraleEgaleDifférencePrimitive} en prenant \(\begin{dcases}f=g\prim \\ F=g\end{dcases}\)
\end{dem}

\subsection{Intégration par parties}

\begin{prop}
Soient \(f,g\in\ensclasse{1}{\intervii{a}{b}}{\K}\).

On a \[\int_a^bf\paren{t}g\prim\paren{t}\odif{t}=\croch{f\paren{t}g\paren{t}}_a^b-\int_a^bf\prim\paren{t}g\paren{t}\odif{t}.\]
\end{prop}

\begin{dem}
La fonction \(fg\) est de classe \(\classe{1}\).

On a donc : \[\begin{aligned}
\croch{f\paren{t}g\paren{t}}_a^b&=\int_a^b\paren{fg}\prim\paren{t}\odif{t} \\
&=\int_a^bf\prim\paren{t}g\paren{t}\odif{t}+\int_a^bg\prim\paren{t}f\paren{t}\odif{t}.
\end{aligned}\]
\end{dem}

\begin{exoex}
\begin{enumerate}
\item Calculer une primitive de \(\ln\) sur \(\Rps\). \\

\item Calculer \(\int_0^1\Arctan t\odif{t}\). \\

\item Calculer une primitive de \(t\mapsto t^2\ln t\) sur \(\Rps\). \\

\item Calculer une primitive de \(t\mapsto t\e{t}\) sur \(\R\). \\

\item Calculer une primitive de \(t\mapsto t^2\cos t\) sur \(\R\).
\end{enumerate}
\end{exoex}

\begin{corr}[1]
On a la primitive \[\begin{aligned}
x\mapsto\int_1^x\ln t\odif{t}&=\croch{t\ln t}_1^x-\int_1^xt\dfrac{1}{t}\odif{t} \\
&=x\ln x-\paren{x-1}
\end{aligned}\]

Donc une autre primitive est \(x\mapsto x\ln x-x\).
\end{corr}

\begin{corr}[2]
On a : \[\begin{aligned}
\int_0^1\Arctan t\odif{t}&=\croch{t\Arctan t}_0^1-\int_0^1\dfrac{t}{1+t^2}\odif{t} \\
&=\dfrac{\pi}{4}-\dfrac{1}{2}\int_0^1\dfrac{2t}{t^2+1}\odif{t} \\
&=\dfrac{\pi}{4}-\dfrac{1}{2}\croch{\ln\paren{t^2+1}}_0^1 \\
&=\dfrac{\pi}{4}-\dfrac{1}{2}\ln2.
\end{aligned}\]
\end{corr}

\begin{corr}[3]
On a la primitive \[\begin{aligned}
x\mapsto\int_1^xt^2\ln t\odif{t}&=\croch{\dfrac{t^3}{3}\ln t}_1^x-\int_1^x\dfrac{t^3}{3}\times\dfrac{1}{t}\odif{t} \\
&=\dfrac{x^3\ln x}{3}-\croch{\dfrac{t^3}{9}}_1^x \\
&=\dfrac{x^3\ln x}{3}-\dfrac{x^3}{9}+\dfrac{1}{9} \\
&=\dfrac{x^3\paren{3\ln x-1}+1}{9}.
\end{aligned}\]
\end{corr}

\begin{corr}[4]
On a la primitive \[\begin{aligned}
x\mapsto\int_0^xt\e{t}\odif{t}&=\croch{t\e{t}}_0^x-\int_0^x\e{t}\odif{t} \\
&=x\e{x}-\e{x}+1 \\
&=\e{x}\paren{x-1}+1.
\end{aligned}\]
\end{corr}

\begin{corr}[5]
On a la primitive \[\begin{aligned}
x\mapsto\int_0^xt^2\cos t\odif{t}&=\croch{t^2\sin t}_0^x-\int_0^x2t\sin t\odif{t} \\
&=x^2\sin x-\croch{-2t\cos t}_0^x+\int_0^x-2\cos t\odif{t} \\
&=x^2\sin x+2x\cos x-2\sin x.
\end{aligned}\]
\end{corr}

\subsection{Formules de Taylor}

\begin{theo}[Formule de Taylor avec reste intégral]
Soient \(n\in\N\) et \(f\in\ensclasse{n+1}{\intervii{a}{b}}{\K}\).

On a \[f\paren{b}=\sum_{k=0}^n\dfrac{f\deriv{k}\paren{a}}{k!}\paren{b-a}^k+\int_a^b\dfrac{\paren{b-t}^n}{n!}f\deriv{n+1}\paren{t}\odif{t}.\]
\end{theo}

\begin{dem}
On raisonne par récurrence finie.

On note \[\quantifs{\forall m\in\interventierii{0}{n}}\underbrace{f\paren{b}=\sum_{k=0}^m\dfrac{f\deriv{k}\paren{a}}{k!}\paren{b-a}^k+\int_a^b\dfrac{\paren{b-t}^m}{m!}f\deriv{m+1}\paren{t}\odif{t}}_{\P{m}}.\]

On a bien \(f\paren{b}=f\paren{a}+\int_a^bf\prim\paren{t}\odif{t}\) car \(f\in\ensclasse{1}{\intervii{a}{b}}{\K}\), d'où \(\P{0}\).

Soit \(m\in\interventierii{0}{n-1}\) tel que \(\P{m}\).

Montrons \(\P{m+1}\).

On peut intégrer la fonction \(t\mapsto\dfrac{\paren{b-t}^m}{m!}f\deriv{m+1}\paren{t}\) car les fonctions \(t\mapsto\dfrac{-\paren{b-t}^{m+1}}{\paren{m+1}!}\) et \(f\deriv{m+1}\) sont de classe \(\classe{1}\).

D'où \[\begin{aligned}
\int_a^b\dfrac{\paren{b-t}^m}{m!}f\deriv{m+1}\paren{t}\odif{t}&=\croch{\dfrac{-\paren{b-t}^{m+1}}{\paren{m+1}!}f\deriv{m+1}\paren{t}}_a^b-\int_a^b\dfrac{-\paren{b-t}^{m+1}}{\paren{m+1}!}f\deriv{m+2}\paren{t}\odif{t} \\
&=0-\dfrac{-\paren{b-a}^{m+1}}{\paren{m+1}!}f\deriv{m+1}\paren{a}+\int_a^b\dfrac{\paren{b-t}^{m+1}}{\paren{m+1}!}f\deriv{m+2}\paren{t}\odif{t}.
\end{aligned}\]

D'où, selon \(\P{m}\) : \[\begin{aligned}
f\paren{b}&=\sum_{k=0}^m\dfrac{f\deriv{k}\paren{a}}{k!}\paren{b-a}^k+\dfrac{\paren{b-a}^{m+1}}{\paren{m+1}!}f\deriv{m+1}\paren{a}+\int_a^b\dfrac{\paren{b-t}^{m+1}}{\paren{m+1}!}f\deriv{m+2}\paren{t}\odif{t} \\
&=\sum_{k=0}^{m+1}\dfrac{f\deriv{k}\paren{a}}{k!}\paren{b-a}^k+\int_a^b\dfrac{\paren{b-t}^{m+1}}{\paren{m+1}!}f\deriv{m+2}\paren{t}\odif{t}.
\end{aligned}\]

D'où \(\P{m+1}\).

D'où, par récurrence : \(\P{n}\).
\end{dem}

\begin{cor}[Inégalité de Taylor-Lagrange]\thlabel{cor:inégalitéDeTaylorLagrange}
Soient \(n\in\N\), \(f\in\ensclasse{n+1}{\intervii{a}{b}}{\K}\) et \(M\in\Rp\) tel que \(\quantifs{\forall x\in\intervii{a}{b}}\abs{f\deriv{n+1}\paren{x}}\leq M\).

Alors \[\abs{f\paren{b}-\sum_{k=0}^n\dfrac{f\deriv{k}\paren{a}}{k!}\paren{b-a}^k}\leq\dfrac{M}{\paren{n+1}!}\paren{b-a}^{n+1}.\]
\end{cor}

\begin{dem}
On a, selon la formule de Taylor avec reste intégral : \[\begin{WithArrows}
\abs{f\paren{b}-\sum_{k=0}^n\dfrac{f\deriv{k}\paren{a}}{k!}\paren{b-a}^k}&=\abs{\int_a^b\dfrac{\paren{b-t}^n}{n!}f\deriv{n+1}\paren{t}\odif{t}} \Arrow[tikz={text width=4cm}]{inégalité triangulaire intégrale} \\
&\leq\int_a^b\abs{\dfrac{\paren{b-t}^n}{n!}f\deriv{n+1}\paren{t}}\odif{t} \Arrow{croissance de l'intégrale} \\
&\leq\int_a^b\dfrac{\paren{b-t}^n}{n!}M\odif{t} \\
&\leq\dfrac{M}{n!}\croch{\dfrac{\paren{b-t}^{n+1}}{n+1}}_a^b \\
&=\dfrac{M}{n!}\paren{0-\dfrac{\paren{b-a}^{n+1}}{n+1}} \\
&=\dfrac{M}{\paren{n+1}!}\paren{b-a}^{n+1}.
\end{WithArrows}\]
\end{dem}

\begin{ex}
Soit \(x\in\Rp\).

Prenons \(a=0\), \(b=x\), \(f=\exp\) et \(n=3\).

On a \[\begin{aligned}
\abs{\e{x}-\paren{1+x+\dfrac{x^2}{2}+\dfrac{x^3}{6}}}&\leq\dfrac{\ds\max_{\intervii{0}{x}}\exp}{4!}x^4 \\
&=\dfrac{\e{x}x^4}{24}.
\end{aligned}\]
\end{ex}

\subsection{Changements de variable}

\begin{theo}
Soient \(I\) un intervalle de \(\R\), \(f\in\ensclasse{1}{I}{\R}\) et \(u\in\ensclasse{1}{\intervii{a}{b}}{\R}\) telle que \(\Im u\subset I\).

On a \[\int_{u\paren{a}}^{u\paren{b}}f\paren{t}\odif{t}=\int_a^bf\paren{u\paren{s}}u\prim\paren{s}\odif{s}.\]

On dit qu'on fait le changement de variable \(\begin{dcases}t=u\paren{s} \\ \odif{t}=u\prim\paren{s}\odif{s}\end{dcases}\)
\end{theo}

\begin{dem}
Posons \(\fonction{g}{\intervii{a}{b}}{\R}{x}{\int_{u\paren{a}}^{u\paren{x}}f\paren{t}\odif{t}-\int_a^xf\paren{u\paren{s}}u\prim\paren{s}\odif{s}}\)

Montrons que \(g\) est dérivable.

La fonction \(\fonctionlambda{\intervii{a}{b}}{\K}{s}{f\paren{u\paren{s}}u\prim\paren{s}}\) est continue car \(\begin{dcases}f\text{ est continue} \\ u\text{ est de classe }\classe{1}\end{dcases}\)

Donc \(x\mapsto\int_a^xf\paren{u\paren{s}}u\prim\paren{s}\odif{s}\) est dérivable de dérivée \(x\mapsto f\paren{u\paren{x}}u\prim\paren{x}\).

La fonction \(\fonctionlambda{I}{\K}{y}{\int_{u\paren{a}}^yf\paren{t}\odif{t}}\) est dérivable car \(f\) est continue, de dérivée \(y\mapsto f\paren{y}\).

De plus, \(u\) est de classe \(\classe{1}\) donc par composition de fonctions dérivables : \(\fonctionlambda{\intervii{a}{b}}{\K}{x}{\int_{u\paren{a}}^{u\paren{x}}f\paren{t}\odif{t}}\) est dérivable de dérivée \(\fonctionlambda{\intervii{a}{b}}{\K}{x}{u\prim\paren{x}f\paren{u\paren{x}}}\)

Finalement, \(g\) est dérivable de dérivée nulle.

Donc \(g\) est constante sur l'intervalle \(\intervii{a}{b}\).

Enfin, \(g\paren{a}=0-0=0\).

Donc \(g\) est nulle, d'où la formule énoncée.
\end{dem}

\begin{rem}
Si \(f\) était continue par morceaux alors \(s\mapsto f\paren{u\paren{s}}u\prim\paren{s}\) pourrait ne pas être continue par morceaux.
\end{rem}

\begin{exoex}~\\
\begin{enumerate}
\item Calculer \(I=\int_{-1}^0\dfrac{t}{t^2+2t+2}\odif{t}\). \\

\item Calculer une primitive de \(\dfrac{1}{\ch}\). \\

\item Calculer une primitive de \(f:x\mapsto\dfrac{1}{x\ln x\ln\paren{\ln x}}\).
\end{enumerate}
\end{exoex}

\begin{corr}[1]~\\
On a \(I=\int_{-1}^0\dfrac{t}{\paren{t+1}^2+1}\odif{t}\).

On fait le changement de variable \[\begin{dcases}u=t+1 \\ \odif{u}=\odif{t}\end{dcases}\ssi\begin{dcases}t=u-1 \\ \odif{t}=\odif{u}\end{dcases}\]

Donc \[\begin{aligned}
I&=\int_0^1\dfrac{u-1}{u^2+1}\odif{u} \\
&=\dfrac{1}{2}\int_0^1\dfrac{2u}{u^2+1}\odif{u}-\int_0^1\dfrac{\odif{u}}{u^2+1} \\
&=\dfrac{1}{2}\croch{\ln\paren{u^2+1}}_0^1-\croch{\Arctan u}_0^1 \\
&=\dfrac{1}{2}\ln2-\dfrac{\pi}{4}.
\end{aligned}\]
\end{corr}

\begin{corr}[2]~\\
Une primitive de \(\dfrac{1}{\ch}\) est \(F:x\mapsto\int_0^x\dfrac{1}{\ch t}\odif{t}=\int_0^x\dfrac{2}{\e{t}+\e{-t}}\odif{t}\).

On fait le changement de variable \[\begin{dcases}u=\e{t} \\ \odif{u}=\e{t}\odif{t}\end{dcases}\ssi\begin{dcases}t=\ln u \\ \odif{t}=\dfrac{\odif{u}}{u}\end{dcases}\]

On a \[\begin{aligned}
F:x\mapsto\int_0^x\dfrac{2}{\e{t}+\e{-t}}\odif{t}&=\int_1^{\e{x}}\dfrac{2}{u^2+1}\odif{u} \\
&=2\croch{\Arctan u}_1^{\e{x}} \\
&=2\Arctan\e{x}-\dfrac{\pi}{2}.
\end{aligned}\]
\end{corr}

\begin{corr}[3]
Une primitive de \(f\) est \(F:x\mapsto\int_{10}^x\dfrac{1}{t\ln t\ln\paren{\ln t}}\odif{t}\).

On fait le changement de variable \[\begin{dcases}u=\ln t \\ \odif{u}=\dfrac{\odif{t}}{t}\end{dcases}\ssi\begin{dcases}t=\e{u} \\ \odif{t}=\e{u}\odif{u}\end{dcases}\]

Donc \[\begin{aligned}
F:x\mapsto\int_{\ln10}^{\ln x}\dfrac{\e{u}}{\e{u}u\ln u}\odif{u}&=\int_{\ln10}^{\ln x}\dfrac{1}{u\ln u}\odif{u} \\
&=\int_{\ln10}^{\ln x}\dfrac{\frac{1}{u}}{\ln u}\odif{u} \\
&=\croch{\ln\abs{\ln u}}_{\ln10}^{\ln x} \\
&=\ln\paren{\ln\paren{\ln x}}-\ln\paren{\ln\paren{\ln10}}.
\end{aligned}\]

Donc une primitive de \(f\) est \(x\mapsto\ln\paren{\ln\paren{\ln x}}\).
\end{corr}

\begin{nota}[Officielle]~\\
On peut noter \(x\mapsto\int^xf\paren{t}\odif{t}\) une primitive de la fonction continue \(f\).
\end{nota}

\subsection{Symétries}

\begin{prop}
Soient \(\alpha\in\Rp\) et \(f\in\contm[\intervii{-\alpha}{\alpha}]\).

Si \(f\) est impaire alors \[\int_{-\alpha}^{\alpha}f\paren{t}\odif{t}=0.\]

Si \(f\) est paire, alors \[\int_{-\alpha}^{\alpha}f\paren{t}\odif{t}=2\int_0^{\alpha}f\paren{t}\odif{t}.\]
\end{prop}

\begin{dem}[Dans le cas continu]
On a \[\begin{aligned}
\int_{-\alpha}^0f\paren{t}\odif{t}&=\int_{\alpha}^0-f\paren{-u}\odif{u}\qquad\text{avec }\begin{dcases}u=-t \\ \odif{u}=-\odif{t}\end{dcases}\text{ car }\begin{dcases}f\text{ est de classe }\classe{0} \\ t\mapsto-t\text{ est de classe }\classe{1}\end{dcases} \\
&=\int_0^{\alpha}f\paren{-u}\odif{u} \\
&=\begin{dcases}-\int_0^{\alpha}f\paren{u}\odif{u} &\text{si }f\text{ est impaire} \\ \int_0^{\alpha}f\paren{u}\odif{u} &\text{si }f\text{  est paire}\end{dcases}
\end{aligned}\]

D'où le résultat (car \(\int_{-\alpha}^{\alpha}f\paren{t}\odif{t}=\int_{-\alpha}^0f\paren{t}\odif{t}+\int_0^{\alpha}f\paren{t}\odif{t}\)).
\end{dem}

\begin{prop}
Soient \(f\in\contm[\R]\) et \(T\in\Rps\).

On suppose que \(f\) est \(T\)-périodique.

Alors \[\quantifs{\forall x\in\R}\int_x^{x+T}f\paren{t}\odif{t}=\int_0^Tf\paren{t}\odif{t}.\]
\end{prop}

\begin{dem}[Dans le cas continu]
On pose \(\fonction{g}{\R}{\K}{x}{\int_x^{x+T}f\paren{t}\odif{t}}\)

On a \(\quantifs{\forall x\in\R}g\paren{x}=\int_0^{x+T}f\paren{t}\odif{t}-\int_0^xf\paren{t}\odif{t}\).

Donc \(g\) est dérivable et \(\quantifs{\forall x\in\R}g\prim\paren{x}=f\paren{x+T}-f\paren{x}=0\).

Donc \(g\) est constante sur l'intervalle \(\R\).

D'où le résultat.
\end{dem}

\section{Techniques de calcul}

\subsection{Utiliser les nombres complexes}

\begin{rappel}
Soit \(\alpha\in\C\).

La fonction \(\fonctionlambda{\R}{\C}{x}{\e{\alpha x}}\) est dérivable de dérivée \(\fonctionlambda{\R}{\C}{x}{\alpha\e{\alpha x}}\)
\end{rappel}

\begin{ex}
Calculons une primitive de \(f:x\mapsto\e{2x}\sin x\).

On a \(f=\Im g\) en posant \(\fonction{g}{\R}{\C}{x}{\e{\paren{2+\i}x}}\)

On a \[\begin{aligned}
\quantifs{\forall x\in\R}\int^xg\paren{t}\odif{t}&=\int^x\e{\paren{2+\i}t}\odif{t} \\
&=\croch{\dfrac{1}{2+\i}\e{\paren{2+\i}t}}^x \\
&=\dfrac{1}{2+\i}\e{\paren{2+\i}x} \\
&=\dfrac{\paren{2-\i}\e{\paren{2+\i}x}}{5} \\
&=\dfrac{2}{5}\e{2x}\cos x+\dfrac{2\i}{5}\e{2x}\sin x-\dfrac{\i}{5}\e{2x}\cos x+\dfrac{1}{5}\e{2x}\sin x.
\end{aligned}\]

Donc \(\int^xf\paren{t}\odif{t}=\dfrac{\e{2x}}{5}\paren{2\sin x-\cos x}\).
\end{ex}

\subsection{Primitives des fonctions rationnelles}

\begin{ex}
Calculons une primitive de \(f:x\mapsto\dfrac{3}{x^3-1}\) (sur \(\R\excluant\accol{1}\)).

Calculons la décomposition en éléments simples de \(\dfrac{3}{X^3-1}\).

On a \(X^3-1=\paren{X-1}\paren{X^2+X+1}\).

Donc \(\dfrac{3}{X^3-1}=\dfrac{a}{X-1}+\dfrac{bX+c}{X^2+X+1}\) avec \(a,b,c\in\R\).

Donc \(\dfrac{3}{X^3-1}=\dfrac{a\paren{X^2+X+1}+\paren{X-1}\paren{bX+c}}{X^3-1}\).

Donc \(3=aX^2+aX+a+bX^2+\paren{c-b}X-c\).

Donc \(\begin{dcases}a+b=0 \\ a-b+c=0 \\ a-c=3\end{dcases}\)

Donc \(\begin{dcases}a=1 \\ b=-1 \\ c=-2\end{dcases}\)

Donc \(\dfrac{3}{X^3-1}=\dfrac{1}{X-1}-\dfrac{X+2}{X^2+X+1}\).

D'où \[\int^x\dfrac{3}{t^3-1}\odif{t}=\int^x\dfrac{\odif{t}}{t-1}-\int^x\dfrac{t+2}{t^2+t+1}\odif{t}.\]

Or \(\int^x\dfrac{\odif{t}}{t-1}=\croch{\ln\abs{t-1}}^x=\ln\abs{x-1}\).

De plus, \(\int^x\dfrac{t+2}{t^2+t+1}\odif{t}=\int^x\dfrac{t+2}{\paren{t+\frac{1}{2}}^2+\frac{3}{4}}\odif{t}=\int^x\dfrac{t+2}{\frac{3}{4}\paren{\paren{\frac{2}{\sqrt{3}}\paren{t+\frac{1}{2}}}^2}+1}\odif{t}\).

On fait le changement de variable \[\begin{dcases}u=\dfrac{2}{\sqrt{3}}\paren{t+\dfrac{1}{2}} \\ \odif{u}=\dfrac{2}{\sqrt{3}}\odif{t}\end{dcases}\ssi\begin{dcases}t=\dfrac{\sqrt{3}}{2}u-\dfrac{1}{2} \\ \odif{t}=\dfrac{\sqrt{3}}{2}\odif{u}\end{dcases}\]

D'où \[\begin{aligned}
\int^x\dfrac{t+2}{t^2+t+1}\odif{t}&=\int^{\frac{2}{\sqrt{3}}\paren{x+\frac{1}{2}}}\dfrac{\frac{\sqrt{3}}{2}u-\frac{1}{2}+2}{\frac{3}{4}\paren{u^2+1}}\odif{u} \\
&=\dfrac{2}{\sqrt{3}}\int^{\frac{2}{\sqrt{3}}\paren{x+\frac{1}{2}}}\dfrac{\frac{\sqrt{3}}{2}+\frac{3}{2}}{u^2+1}\odif{u} \\
&=\dfrac{2}{\sqrt{3}}\croch{\dfrac{\sqrt{3}}{4}\ln\paren{u^2+1}+\dfrac{3}{2}\Arctan u}^{\frac{2}{\sqrt{3}}\paren{x+\frac{1}{2}}} \\
&=\dfrac{1}{2}\ln\paren{\dfrac{4}{3}\paren{x+\dfrac{1}{2}}^2+1}+\sqrt{3}\Arctan\paren{\dfrac{2}{\sqrt{3}}\paren{x+\dfrac{1}{2}}} \\
&=\dfrac{1}{2}\ln\paren{\dfrac{4}{3}\paren{x^2+x+1}}+\sqrt{3}\Arctan\dfrac{2x+1}{\sqrt{3}}.
\end{aligned}\]

Donc une primitive de \(f\) est \[x\mapsto\ln\abs{x-1}-\dfrac{1}{2}\ln\paren{\dfrac{4}{3}\paren{x^2+x+1}}-\sqrt{3}\Arctan\dfrac{2x+1}{\sqrt{3}}.\]
\end{ex}

\begin{prop}
On peut primitiver toute fonction rationnelle en écrivant sa décomposition en éléments simples sur \(\R\).
\end{prop}

\begin{rem}
C'est suffisant, pas nécessaire.

Par exemple : \(\int_0^1\dfrac{t^7}{t^8+1}\odif{t}=\croch{\dfrac{1}{8}\ln\paren{t^8+1}}_0^1=\dfrac{\ln2}{8}\).
\end{rem}

\begin{dem}
On peut primitiver chaque terme de la décomposition en éléments simples sur \(\R\) :

La partie entière : c'est un polynôme.

Les termes de la forme \(t\mapsto\dfrac{1}{t-\lambda}\) avec \(\lambda\in\R\) : primitive \[t\mapsto\ln\abs{t-\lambda}.\]

Les termes de la forme \(t\mapsto\dfrac{1}{\paren{t-\lambda}^{\alpha}}=\paren{t-\lambda}^{-\alpha}\) où \(\begin{dcases}\lambda\in\R \\ \alpha\in\interventierie{2}{\pinf}\end{dcases}\) : primitive \[t\mapsto\dfrac{1}{1-\alpha}\paren{t-\lambda}^{1-\alpha}.\]

Les termes de la forme \(t\mapsto\dfrac{\lambda t+\mu}{t^2+bt+c}\) avec \(\begin{dcases}\lambda,\mu\in\R \\ b,c\in\R\text{ tels que }\Delta=b^2-4c<0\end{dcases}\) :

Par un changement de variable affine, on se ramène à \[\int^y\dfrac{\lambda\prim u+\mu\prim}{u^2+1}\odif{u}=\dfrac{\lambda\prim}{2}\ln\paren{y^2+1}+\mu\prim\Arctan y\] et \[\begin{aligned}
\int^x\dfrac{\odif{t}}{t^2+bt+c}&=\int^x\dfrac{\odif{t}}{\paren{t+\frac{b}{2}}^2-\frac{b^2}{4}+c} \\
&=\int^x\dfrac{\odif{t}}{\paren{t+\frac{b}{2}}^2\underbrace{-\frac{\Delta}{4}}_{>0}} \\
&=\int^x\dfrac{\odif{t}}{-\frac{\Delta}{4}\paren{\paren{\frac{\sqrt{-\Delta}}{2}\paren{t+\frac{b}{2}}}^2+1}}.
\end{aligned}\]

D'où ce qu'on voulait par le changement de variable \(u=\dfrac{\sqrt{-\Delta}}{2}\paren{t+\dfrac{b}{2}}\).

Les termes de la forme \(t\mapsto\dfrac{\lambda t+\mu}{\paren{t^2+bt+c}^{\alpha}}\) avec \(\begin{dcases}\alpha\in\interventierie{2}{\pinf} \\ \lambda,\mu,b,c\in\R \\ b^2-4c<0\end{dcases}\) :

Quitte à faire un changement de variable affine, on peut supposer \(b=0\) et \(c=1\). Il suffit donc de savoir calculer \[\int_0^x\dfrac{\lambda t+\mu}{\paren{t^2+1}^{\alpha}}\odif{t}=\underbrace{\lambda\int_0^xt\paren{t^2+1}^{-\alpha}\odif{t}}_{\frac{\lambda}{2\paren{1-\alpha}}\paren{x^2+1}^{1-\alpha}}+\underbrace{\mu\int_0^x\dfrac{\odif{t}}{\paren{t^2+1}^{\alpha}}}_{I_\alpha\paren{x}}.\]

Reste à calculer \(I_\alpha\paren{x}\) par récurrence sur \(\alpha\), à l'aide d'une intégration par partie.

En effet : \[\begin{aligned}
\int_0^x\paren{t^2+1}^{-\alpha}\odif{t}&=\croch{t\paren{t^2+1}^{-\alpha}}_0^x-\int_0^xt\times2t\paren{-\alpha}\paren{t^2+1}^{-\alpha-1}\odif{t} \\
&=\dfrac{x}{\paren{x^2+1}^{\alpha}}+2\alpha\int_0^x\paren{t^2+1-1}\paren{t^2+1}^{-\alpha-1}\odif{t} \\
&=\dfrac{x}{\paren{x^2+1}^{\alpha}}+2\alpha I_\alpha\paren{x}-2\alpha I_{\alpha+1}\paren{x}.
\end{aligned}\]

D'où \(\paren{1-2\alpha}I_\alpha\paren{x}=\dfrac{x}{\paren{x^2+1}^{\alpha}}-2\alpha I_{\alpha+1}\paren{x}\).

On connaît \(I_1\paren{x}\) (\cf termes de la forme \(t\mapsto\dfrac{1}{\paren{t-\lambda}^{\alpha}}\)).

On en déduit \(I_2\), \(I_3\), ...
\end{dem}

\subsection{Fonctions rationnelles en \(\e{t}\)}

\begin{ex}
\[t\mapsto\dfrac{1}{\ch t}=\dfrac{2}{\e{t}+\e{-t}}=\dfrac{2\e{t}}{\e{2t}+1}\qquad\text{et}\qquad t\mapsto\dfrac{\e{2t}+1}{\e{t}-1}\]
\end{ex}

\begin{prop}
On remarque \[\quantifs{\forall F\in\fracrat}\int^xF\paren{\e{t}}\odif{t}=\int^{\e{x}}\dfrac{F\paren{u}}{u}\odif{u}\] en opérant le changement de variable \[\begin{dcases}u=\e{t} \\ \odif{u}=\e{t}\odif{t}\end{dcases}\ssi\begin{dcases}t=\ln u \\ \odif{t}=\dfrac{\odif{u}}{u}\end{dcases}\]

Or \(\dfrac{F}{X}\in\fracrat\) donc on sait primitiver.
\end{prop}

\subsection{Règle de Bioche}

\begin{meth}~\\
Pour calculer \(\int F\paren{\cos\theta,\sin\theta}\odif{\theta}\) :

\begin{itemize}
\item Si \(F\paren{\cos\theta,\sin\theta}\odif{\theta}\) est invariant quand on remplace \(\theta\) par \(-\theta\) : \[\text{faire le changement de variable }t=\cos\theta.\]

\item Si \(F\paren{\cos\theta,\sin\theta}\odif{\theta}\) est invariant quand on remplace \(\theta\) par \(\pi-\theta\) : \[\text{faire le changement de variable }t=\sin\theta.\]

\item Si \(F\paren{\cos\theta,\sin\theta}\odif{\theta}\) est invariant quand on remplace \(\theta\) par \(\pi+\theta\) : \[\text{faire le changement de variable }t=\tan\theta.\]

\item Dans tous les cas, on peut \[\text{faire le changement de variable }t=\tan\dfrac{\theta}{2}.\]
\end{itemize}
\end{meth}

\begin{exoex}
Calculer une primitive de \(x\mapsto\dfrac{\sin^3x}{1+\cos^2x}\).
\end{exoex}

\begin{corr}
On a la primitive \[x\mapsto\int^x\dfrac{\sin^3\theta}{1+\cos^2\theta}\odif{\theta}.\]

On fait le changement de variable \(\begin{dcases}t=\cos\theta \\ \odif{t}=-\sin\theta\odif{\theta}\end{dcases}\)

Donc \[\begin{aligned}
x\mapsto\int^x\dfrac{\sin^3\theta}{1+\cos^2\theta}\odif{\theta}&=\int^x\dfrac{-1+\cos^2\theta}{1+\cos^2\theta}\paren{-\sin\theta}\odif{\theta} \\
&=\int^{\cos x}\dfrac{-1+t^2}{1+t^2}\odif{t} \\
&=\int^{\cos x}\dfrac{t^2+1-2}{t^2+1}\odif{t} \\
&=\int^{\cos x}\paren{1-\dfrac{2}{t^2+1}}\odif{t} \\
&=\cos x-2\Arctan\paren{\cos x}.
\end{aligned}\]
\end{corr}