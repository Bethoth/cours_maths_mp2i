\chapter{Probabilités}

\minitoc

\section{Rappels et compléments sur les ensembles}

\subsection{Ensembles finis}

\begin{defi}
Un ensemble \(E\) est dit fini s'il contient un nombre fini d'éléments.

Le nombre d'éléments de \(E\) est appelé cardinal de \(E\) et est noté \(\abs{E}\), \(\Card E\) ou encore \(\#E\).
\end{defi}

\begin{theo}
Soient \(E\) un ensemble fini et \(A\subset E\) une partie de \(E\).

Alors \(A\) est un ensemble fini.

De plus, on a : \[\Card A\leq\Card E\] avec égalité si, et seulement si, \(A=E\).
\end{theo}

\begin{theo}
Soient \(E\) et \(F\) deux ensembles finis de même cardinal et une fonction \(f:E\to F\).

Les trois propositions suivantes sont équivalentes :

\begin{enumerate}
    \item \(f\) est une bijection \\
    \item \(f\) est une injection \\
    \item \(f\) est une surjection.
\end{enumerate}
\end{theo}

\subsection{Dénombrement des ensembles finis}

Soient \(E\) un ensemble fini et \(A,B\in\P{E}\).

On a :

\begin{description}
    \item[] \(\Card A\union B=\Card A+\Card B\) si \(A\) et \(B\) sont disjoints \\
    \item[] \(\Card A\union B=\Card A+\Card B-\Card A\inter B\) \\
    \item[] \(\Card E\excluant A=\Card E-\Card A\) \\
    \item[] \(\Card A\times B=\Card A\times\Card B\) \\
    \item[] \(\Card B^A=\Card\F{A}{B}=\paren{\Card B}^{\Card A}\) \\
    \item[] \(\Card\P{E}=2^{\Card E}\) \\
    \item[] \(\Card E^p=\paren{\Card E}^p\).
\end{description}

Soit \(p\in\interventierii{0}{n}\).

On appelle :

\begin{description}
    \item[] \(p\)-arrangement de \(E\) tout \(p\)-uplet d'éléments de \(E\) deux à deux distincts ; \\
    \item[] \(p\)-combinaison de \(E\) toute partie de \(E\) contenant \(p\) éléments.
\end{description}

On a alors, en notant \(n\) le cardinal de \(E\) :

\begin{description}
    \item[] le nombre de \(p\)-arrangements de \(E\) est \(\arr{p}{n}=\dfrac{n!}{\paren{n-p}!}\) \\
    \item[] le nombre de \(p\)-combinaisons de \(E\) est \(\comb{p}{n}=\binom{p}{n}=\dfrac{n!}{p!\,\paren{n-p}!}\)
\end{description}

Enfin, \(\arr{p}{n}\) est aussi le nombre d'injections d'un ensemble de cardinal \(p\) vers \(E\). En particulier, si \(p=n\) et en notant \(\S{E}\) l'ensemble des bijections de \(E\) vers \(E\) : \[\Card\S{E}=n!\]

\begin{exoex}
Soient \(k,n\in\Ns\) tels que \(k\leq n\).

Une urne contient \(n\) boules numérotées de \(1\) à \(n\). On tire \(k\) boules dans cette urne.

Donner le nombre de résultats possibles en fonction du type de tirage :

\begin{enumerate}
    \item On tire \(k\) boules simultanément. \\
    \item On tire \(k\) boules une par une, avec remise. \\
    \item On tire \(k\) boules une par une, sans remise.
\end{enumerate}
\end{exoex}

\begin{corr}~\\
\begin{enumerate}
    \item Il y a \(\binom{k}{n}\) résultats possibles. \\
    \item Il y a \(n^k\) résultats possibles. \\
    \item Il y a \(\arr{k}{n}\) résultats possibles.
\end{enumerate}
\end{corr}

\begin{exoex}[Mines-Télécom 2016]
Soit \(E\) un ensemble fini dont on note \(n\) le cardinal.

Donner le cardinal des ensembles suivants :

\begin{enumerate}
    \item \(E_1=\accol{\paren{X,Y}\in\P{E}^2\tq\accol{X;Y}\text{ est une partition de }E}\) ; \\
    \item \(E_2=\accol{\paren{X,Y}\in\P{E}^2\tq X\inter Y=\ensvide}\) ; \\
    \item \(E_3=\accol{\paren{X,Y}\in\P{E}^2\tq X\union Y=E}\) ; \\
    \item \(E_4=\accol{\paren{X,Y}\in\P{E}^2\tq X\subset Y}\) ; \\
    \item \(E_5=\accol{\paren{X,Y,Z}\in\P{E}^3\tq X\union Y\union Z=E}\).
\end{enumerate}
\end{exoex}

\begin{corr}
On a :

\begin{enumerate}
    \item \(\Card E_1=2^n-2\) \\
    \item \(\Card E_2=\sum_{k=0}^{n}\binom{k}{n}2^{n-k}=3^n\) \\
    \item \(\Card E_3=3^n\) \\
    \item \(\Card E_4=3^n\) \\
    \item \(\Card E_5=7^n\)
\end{enumerate}
\end{corr}