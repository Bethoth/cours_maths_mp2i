\chapter{Probabilités}

\minitoc

\section{Rappels et compléments sur les ensembles}

\subsection{Ensembles finis}

\begin{defi}
Un ensemble \(E\) est dit fini s'il contient un nombre fini d'éléments.

Le nombre d'éléments de \(E\) est appelé cardinal de \(E\) et est noté \(\abs{E}\), \(\Card E\) ou encore \(\#E\).
\end{defi}

\begin{theo}
Soient \(E\) un ensemble fini et \(A\subset E\) une partie de \(E\).

Alors \(A\) est un ensemble fini.

De plus, on a : \[\Card A\leq\Card E\] avec égalité si, et seulement si, \(A=E\).
\end{theo}

\begin{theo}
Soient \(E\) et \(F\) deux ensembles finis de même cardinal et une fonction \(f:E\to F\).

Les trois propositions suivantes sont équivalentes :

\begin{enumerate}
    \item \(f\) est une bijection \\
    \item \(f\) est une injection \\
    \item \(f\) est une surjection.
\end{enumerate}
\end{theo}