\chapter{Probabilités}

\minitoc

\section{Rappels et compléments sur les ensembles}

\subsection{Ensembles finis}

\begin{defi}
Un ensemble \(E\) est dit fini s'il contient un nombre fini d'éléments.

Le nombre d'éléments de \(E\) est appelé cardinal de \(E\) et est noté \(\abs{E}\), \(\Card E\) ou encore \(\#E\).
\end{defi}

\begin{theo}
Soient \(E\) un ensemble fini et \(A\subset E\) une partie de \(E\).

Alors \(A\) est un ensemble fini.

De plus, on a : \[\Card A\leq\Card E\] avec égalité si, et seulement si, \(A=E\).
\end{theo}

\begin{theo}
Soient \(E\) et \(F\) deux ensembles finis de même cardinal et une fonction \(f:E\to F\).

Les trois propositions suivantes sont équivalentes :

\begin{enumerate}
    \item \(f\) est une bijection \\
    \item \(f\) est une injection \\
    \item \(f\) est une surjection.
\end{enumerate}
\end{theo}

\subsection{Dénombrement des ensembles finis}

Soient \(E\) un ensemble fini et \(A,B\in\P{E}\).

On a :

\begin{description}
    \item[] \(\Card A\union B=\Card A+\Card B\) si \(A\) et \(B\) sont disjoints \\
    \item[] \(\Card A\union B=\Card A+\Card B-\Card A\inter B\) \\
    \item[] \(\Card E\excluant A=\Card E-\Card A\) \\
    \item[] \(\Card A\times B=\Card A\times\Card B\) \\
    \item[] \(\Card B^A=\Card\F{A}{B}=\paren{\Card B}^{\Card A}\) \\
    \item[] \(\Card\P{E}=2^{\Card E}\) \\
    \item[] \(\Card E^p=\paren{\Card E}^p\).
\end{description}

Soit \(p\in\interventierii{0}{n}\).

On appelle :

\begin{description}
    \item[] \(p\)-arrangement de \(E\) tout \(p\)-uplet d'éléments de \(E\) deux à deux distincts ; \\
    \item[] \(p\)-combinaison de \(E\) toute partie de \(E\) contenant \(p\) éléments.
\end{description}

On a alors, en notant \(n\) le cardinal de \(E\) :

\begin{description}
    \item[] le nombre de \(p\)-arrangements de \(E\) est \(\arr{p}{n}=\dfrac{n!}{\paren{n-p}!}\) \\
    \item[] le nombre de \(p\)-combinaisons de \(E\) est \(\comb{p}{n}=\binom{p}{n}=\dfrac{n!}{p!\,\paren{n-p}!}\)
\end{description}

Enfin, \(\arr{p}{n}\) est aussi le nombre d'injections d'un ensemble de cardinal \(p\) vers \(E\). En particulier, si \(p=n\) et en notant \(\S{E}\) l'ensemble des bijections de \(E\) vers \(E\) : \[\Card\S{E}=n!\]

\begin{exoex}
Soient \(k,n\in\Ns\) tels que \(k\leq n\).

Une urne contient \(n\) boules numérotées de \(1\) à \(n\). On tire \(k\) boules dans cette urne.

Donner le nombre de résultats possibles en fonction du type de tirage :

\begin{enumerate}
    \item On tire \(k\) boules simultanément. \\
    \item On tire \(k\) boules une par une, avec remise. \\
    \item On tire \(k\) boules une par une, sans remise.
\end{enumerate}
\end{exoex}

\begin{corr}~\\
\begin{enumerate}
    \item Il y a \(\binom{k}{n}\) résultats possibles. \\
    \item Il y a \(n^k\) résultats possibles. \\
    \item Il y a \(\arr{k}{n}\) résultats possibles.
\end{enumerate}
\end{corr}

\begin{exoex}[Mines-Télécom 2016]
Soit \(E\) un ensemble fini dont on note \(n\) le cardinal.

Donner le cardinal des ensembles suivants :

\begin{enumerate}
    \item \(E_1=\accol{\paren{X,Y}\in\P{E}^2\tq\accol{X;Y}\text{ est une partition de }E}\) ; \\
    \item \(E_2=\accol{\paren{X,Y}\in\P{E}^2\tq X\inter Y=\ensvide}\) ; \\
    \item \(E_3=\accol{\paren{X,Y}\in\P{E}^2\tq X\union Y=E}\) ; \\
    \item \(E_4=\accol{\paren{X,Y}\in\P{E}^2\tq X\subset Y}\) ; \\
    \item \(E_5=\accol{\paren{X,Y,Z}\in\P{E}^3\tq X\union Y\union Z=E}\).
\end{enumerate}
\end{exoex}

\begin{corr}
On a :

\begin{enumerate}
    \item \(\Card E_1=2^n-2\) \\
    \item \(\Card E_2=\sum_{k=0}^{n}\binom{k}{n}2^{n-k}=3^n\) \\
    \item \(\Card E_3=3^n\) \\
    \item \(\Card E_4=3^n\) \\
    \item \(\Card E_5=7^n\)
\end{enumerate}
\end{corr}

\section{Espaces probabilisés}

\subsection{Cadre formel}

\subsubsection{Univers}

On considère un ensemble fini non-vide \(\Omega\) appelé l'univers.

\begin{ex}
\begin{itemize}
    \item On lance un dé une fois : \(\Omega=\interventierii{1}{6}\). \\
    \item On lance un dé deux fois : \(\Omega=\interventierii{1}{6}^2\). \\
    \item On lance un dé \(n\) fois : \(\Omega=\interventierii{1}{6}^n\).
\end{itemize}
\end{ex}

\begin{rem}
En pratique, on ne précisera pas quel est l'univers (il y a beaucoup de façons de modéliser une situation et la modélisation choisie ne change pas les probabilités obtenues).
\end{rem}

\subsubsection{Événements}

Les parties de \(\Omega\) sont appelées les événements.

\begin{ex}
\begin{itemize}
    \item Expérience aléatoire : on lance un dé une fois. \\ Univers : \(\Omega=\interventierii{1}{6}\). \\ Événement \guillemets{le résultat vaut \(6\)} : \(\accol{6}\). \\ Événement \guillemets{le résultat est pair} : \(\accol{2;4;6}\). \\
    \item Expérience aléatoire : on lance un dé deux fois. \\ Univers : \(\Omega=\interventierii{1}{6}^2\). \\ Événement \guillemets{les deux lancers donnent le même résultat} : \(\accol{\paren{1,1};\paren{2,2};\paren{3,3};\paren{4,4};\paren{5,5};\paren{6,6}}\).
\end{itemize}
\end{ex}

\begin{defi}
L'ensemble vide \(\ensvide\) est un événement, appelé l'événement impossible.

L'événement \(\Omega\) est appelé l'événement certain.

Les événements qui ne contiennent qu'un seul élément (\ie les singletons) sont appelés événements élémentaires.
\end{defi}

\begin{defi}
Deux événements sont dits incompatibles (ou disjoints) si leur intersection est vide.
\end{defi}

\begin{defi}[Système complet d'événements]
Soient \(\Omega\) un univers et \(N\in\Ns\).

On appelle système complet d'événements toute famille \(\paren{A_1,\dots,A_N}\in\P{\Omega}^N\) d'événements deux à deux incompatibles telle que \[\bigunion_{n=1}^NA_n=\Omega.\]
\end{defi}

\subsubsection{Probabilité}

\begin{defi}[Probabilité]
Soit \(\Omega\) un univers.

On appelle probabilité sur \(\Omega\) toute application \(\prem:\P{\Omega}\to\intervii{0}{1}\) telle que :

\begin{enumerate}
    \item \(\proba{\Omega}=1\) \\
    \item Pour tout entier \(N\in\Ns\) et toute famille \(\paren{A_1,\dots,A_N}\in\P{\Omega}^N\) d'événements deux à deux incompatibles : \[\proba{\bigunion_{n=1}^NA_n}=\sum_{n=1}^N\proba{A_n}.\]
\end{enumerate}
\end{defi}

\begin{defi}[Espace probabilisé]
Un couple \(\groupe{\Omega}[\prem]\), où \(\Omega\) est un univers et \(\prem\) est une probabilité sur \(\Omega\), est appelé un espace probabilisé.
\end{defi}

\begin{rem}
Soit \(\groupe{\Omega}[\prem]\) un espace probabilisé.

On a : \[\proba{\ensvide}=0.\]
\end{rem}

\begin{dem}
On a \(\ensvide=\ensvide\union\ensvide\) (événements incompatibles) donc \(\proba{\ensvide}=\proba{\ensvide}+\proba{\ensvide}\).

Donc \(\proba{\ensvide}=0\).
\end{dem}

\begin{rem}
Soit \(\Omega\) un univers.

Dans ce chapitre, on note généralement \(\conj{A}\) le complémentaire dans \(\Omega\) de l'événement \(A\in\P{\Omega}\).

Le complémentaire d'une partie n'est jamais noté ainsi dans le cadre de la topologie (notation alors réservée à l'adhérence).
\end{rem}

\begin{prop}
Soient \(\groupe{\Omega}[\prem]\) un espace probabilisé et \(A,B\in\P{\Omega}\) deux événements.

\begin{enumerate}
    \item Si \(A\subset B\) alors \(\proba{B\excluant A}=\proba{B}-\proba{A}\). En particulier, on a \(\proba{A}\leq\proba{B}\). \\
    \item On a \(\proba{\conj{A}}=1-\proba{A}\). \\
    \item On a \(\proba{A\union B}=\proba{A}+\proba{B}-\proba{A\inter B}\).
\end{enumerate}
\end{prop}

\begin{dem}[1]
On a \(B=A\union\paren{B\excluant A}\) (réunion disjointe) donc \(\proba{B}=\proba{A}+\proba{B\excluant A}\).
\end{dem}

\begin{dem}[2]
Découle du (1) en prenant \(B=\Omega\).
\end{dem}

\begin{dem}[3]
On a \(A\union B=\paren{A\excluant B}\union\paren{A\inter B}\union\paren{B\excluant A}\) (réunion disjointe).

Donc \(\proba{A\union B}=\proba{A\excluant B}+\proba{A\inter B}+\proba{B\excluant A}\).

D'autre part, on a \(\begin{dcases}
A=\paren{A\excluant B}\union\paren{A\inter B} \\
B=\paren{B\excluant A}\union\paren{A\inter B}
\end{dcases}\) (réunions disjointes).

Donc on a \(\begin{dcases}
\proba{A}=\proba{A\excluant B}+\proba{A\inter B} \\
\proba{B}=\proba{B\excluant A}+\proba{A\inter B}
\end{dcases}\)

D'où la formule.
\end{dem}

\begin{prop}[Sous-additivité]
Soient \(\groupe{\Omega}[\prem]\) un espace probabilisé et \(N\in\Ns\).

Pour toute famille d'événements \(\paren{A_1,\dots,A_N}\in\P{\Omega}^N\), on a : \[\proba{\bigunion_{n=1}^NA_n}\leq\sum_{n=1}^N\proba{A_n}.\]
\end{prop}

\begin{dem}
On pose \(\quantifs{\forall k\in\interventierii{1}{N}}A_k\prim=A_k\excluant\bigunion_{i=1}^{k-1}A_i\) de sorte que \(\bigunion_{k=1}^NA_k=\bigunion_{k=1}^NA_k\prim\) (réunion disjointe).

On a alors : \[\begin{WithArrows}
\proba{\bigunion_{k=1}^NA_k}&=\sum_{k=1}^N\proba{A_k\prim} \Arrow{car \(\quantifs{\forall k\in\interventierii{1}{N}}A_k\prim\subset A_k\)} \\
&\leq\sum_{k=1}^N\proba{A_k}
\end{WithArrows}\]
\end{dem}

\subsubsection{Distribution de probabilités}

\begin{defi}
Soit \(\Omega\) un univers.

On appelle distribution de probabilités sur \(\Omega\) toute famille \(\paren{p_\omega}_{\omega\in\Omega}\) de réels positifs de somme \(1\) : \[\quantifs{\forall\omega\in\Omega}p_\omega\geq0\qquad\text{et}\qquad\sum_{\omega\in\Omega}p_\omega=1.\]
\end{defi}

\begin{prop}
Soit \(\Omega\) un univers.

La donnée d'une probabilité sur \(\Omega\) revient à la donnée d'une distribution de probabilités sur \(\Omega\).
\end{prop}

\begin{dem}
\analyse

Soient \(\prem\) une probabilité sur \(\Omega\) et \(A\subset\Omega\).

On a \(A=\bigunion_{\omega\in A}\accol{\omega}\) (réunion disjointe).

Donc, en posant \(\quantifs{\forall\omega\in\Omega}p_\omega=\proba{\accol{\omega}}\), on a : \[\proba{A}=\sum_{\omega\in A}\proba{\accol{\omega}}=p_\omega.\]

On a bien \(\begin{dcases}
\quantifs{\forall\omega\in\Omega}p_\omega\geq0 \\
\sum_{\omega\in\Omega}p_\omega=\proba{\Omega}=1
\end{dcases}\)

Donc \(\paren{p_\omega}_{\omega\in\Omega}\) est une distribution de probabilités sur \(\Omega\).

\synthese

Soit \(\paren{p_\omega}_{\omega\in\Omega}\) une distribution de probabilités sur \(\Omega\).

On pose : \[\fonction{\prem}{\P{\Omega}}{\R}{A}{\sum_{\omega\in A}p_\omega}\]

Montrons que \(\prem\) est une probabilité sur \(\Omega\).

Soient \(N\in\Ns\) et \(A_1,\dots,A_N\in\P{\Omega}\) des événements deux à deux incompatibles.

On a : \[\begin{WithArrows}
\proba{\bigunion_{i=1}^NA_i}&=\sum_{\omega\in\bigunion_{i=1}^NA_i}p_\omega \Arrow{car \(\bigunion_{i=1}^NA_i\) est une réunion disjointe} \\
&=\sum_{i=1}^N\sum_{\omega\in A_i}p_\omega \\
&=\sum_{i=1}^N\proba{A_i}.
\end{WithArrows}\]

De plus, on a \(\proba{\Omega}=\sum_{\omega\in\Omega}p_\omega=1\).

Donc \(\prem\) est une probabilité sur \(\Omega\).
\end{dem}

\subsection{Exemple : probabilité uniforme sur un ensemble fini}

\begin{defi}[Probabilité uniforme]
Soit \(\Omega\) un univers.

On appelle probabilité uniforme sur \(\Omega\) la probabilité \(\prem\) définie par : \[\quantifs{\forall A\in\P{\Omega}}\proba{A}=\dfrac{\Card A}{\Card\Omega}.\]

Pour cette probabilité, tous les événements élémentaires sont équiprobables, de probabilité \(\dfrac{1}{\Card\Omega}\).
\end{defi}

\begin{ex}[Lancer d'un dé]
On lance un dé (non-pipé, à six faces).

Le résultat peut être modélisé par l'univers \(\Omega=\interventierii{1}{6}\) muni de la probabilité uniforme.
\end{ex}

\begin{exoex}
On tire simultanément trois cartes dans un jeu de trente-deux cartes. On suppose que tous les tirages sont équiprobables.

\begin{enumerate}
    \item Quelle est la probabilité de tirer trois as ? \\
    \item Quelle est la probabilité de tirer trois cartes qui se suivent ? \\
    \item Quelle est la probabilité de tirer deux rois et une dame ?
\end{enumerate}
\end{exoex}

\begin{corr}[1]~\\
On a \(\Card\Omega=\binom{3}{32}\).

On note \(A\) l'événement \guillemets{on tire trois as}.

On a \(\Card A=\binom{3}{4}\).

Donc : \[\begin{aligned}
\proba{A}&=\dfrac{\Card A}{\Card\Omega} \\
&=\dfrac{\binom{3}{4}}{\binom{3}{32}} \\
&=\dfrac{\frac{4\times3\times2}{3\times2\times1}}{\frac{32\times31\times30}{3\times2\times1}} \\
&=\dfrac{4\times3\times2}{32\times31\times30} \\
&=\dfrac{1}{4\times31\times10} \\
&=\dfrac{1}{1240}.
\end{aligned}\]
\end{corr}

\begin{corr}[2]
On note \(B\) l'événement \guillemets{on tire trois cartes qui se suivent}.

On a \(\Card B=6\times4^3\).

Donc : \[\begin{aligned}
\proba{B}&=\dfrac{6\times4^3}{\frac{32\times31\times30}{3\times2\times1}} \\
&=\dfrac{2\times3\times6\times2^6}{30\times31\times32} \\
&=\dfrac{2^2\times3}{5\times31} \\
&=\dfrac{12}{155}.
\end{aligned}\]
\end{corr}

\begin{corr}[3]
On note \(C\) l'événement \guillemets{on tire deux rois et une dame}.

On a \(\Card C=\binom{2}{4}\times\binom{1}{4}=4\times6\).

Donc : \[\begin{aligned}
\proba{C}&=\dfrac{6\times4}{\frac{32\times31\times30}{3\times2\times1}} \\
&=\dfrac{2^4\times3^2}{2^5\times31\times2\times5\times3} \\
&=\dfrac{3}{2^2\times5\times31} \\
&=\dfrac{3}{620}.
\end{aligned}\]
\end{corr}

\section{Conditionnement}

\subsection{Définitions}

On considère un espace probabilisé \(\groupe{\Omega}[\prem]\).

\begin{defi}
Soient \(A,B\in\P{\Omega}\) deux événements tels que \(\proba{B}\not=0\).

On appelle probabilité conditionnelle de \(A\) sachant \(B\) le réel : \[\probacond{A}{B}=\proba{A\tq B}=\dfrac{\proba{A\inter B}}{\proba{B}}.\]
\end{defi}

\begin{rem}
On a donc \(\proba{A\inter B}=\proba{B}\proba{A\tq B}\).
\end{rem}

\begin{exoex}
On lance deux dés.

\begin{enumerate}
    \item Quelle est la probabilité d'avoir au total au moins \(10\) ? \\
    \item Quelle est la probabilité d'avoir au total au moins \(10\) sachant que l'un des deux dés donne \(2\) ? \\
    \item Quelle est la probabilité d'avoir au total au moins \(10\) sachant que l'un des deux dés donne \(6\) ?
\end{enumerate}
\end{exoex}

\begin{corr}[1]
On a l'univers \(\Omega=\interventierii{1}{6}^2\) de cardinal \(36\).

On note \(A\) l'événement \guillemets{avoir au total au moins 10}.

On a \(A=\accol{\paren{4,6};\paren{5,5};\paren{5,6};\paren{6,4};\paren{6,5};\paren{6,6}}\) donc \(\Card A=6\).

Donc \(\proba{A}=\dfrac{6}{36}=\dfrac{1}{6}\).
\end{corr}

\begin{corr}[2]
La probabilité est nulle.
\end{corr}

\begin{corr}[3]
On note \(A\) l'événement \guillemets{avoir au total au moins 10} et \(B\) l'événement \guillemets{l'un des deux dés donne 6}.

On a : \[\proba{A\tq B}=\dfrac{\proba{A\inter B}}{\proba{B}}=\dfrac{5}{11}.\]
\end{corr}

\begin{prop}
Soit \(B\in\P{\Omega}\) un événement de probabilité non-nulle.

L'application \(\prem_B:\P{\Omega}\to\intervii{0}{1}\) est une probabilité sur \(\Omega\).
\end{prop}