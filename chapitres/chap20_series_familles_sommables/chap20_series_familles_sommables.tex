\chapter{Séries, familles sommables}

\minitoc

On pose \(\K=\R\) ou \(\C\).

Dans ce chapitre, on définit la valeur d'une \guillemets{somme infinie} dans \(\K\) de la forme \(x_0+x_1+x_2+\dots\) de deux façons :

\begin{itemize}
    \item une première façon au \hyperref[sec:séries]{paragraphe 1} où elle est notée \(\sum_{n=0}^{\pinf}x_n\) ; \\
    \item une seconde au paragraphe \hyperref[sec:famillesSommablesDeRéelsPositifs]{paragraphe 5} où elle est notée \(\sum_{n\in\N}x_n\).
\end{itemize}

La grande différence entre les deux façons est que l'ordre dans lequel sont énumérés les termes de la somme est important dans la première ; pas dans la seconde.

\section{Séries}\label{sec:séries}

\subsection{Deux exemples}

\begin{ex}[Série géométrique]
On a la série géométrique de raison \(\dfrac{1}{2}\) : \[\dfrac{1}{2}+\dfrac{1}{4}+\dfrac{1}{8}+\dfrac{1}{16}+\dots=1.\]
\end{ex}

\begin{ex}[Série harmonique]
On a la série harmonique : \[\dfrac{1}{1}+\dfrac{1}{2}+\dfrac{1}{3}+\dfrac{1}{4}+\dfrac{1}{5}+\dots=\pinf.\]
\end{ex}