\chapter{Séries, familles sommables}

\minitoc

On pose \(\K=\R\) ou \(\C\).

Dans ce chapitre, on définit la valeur d'une \guillemets{somme infinie} dans \(\K\) de la forme \(x_0+x_1+x_2+\dots\) de deux façons :

\begin{itemize}
    \item une première façon au \hyperref[sec:séries]{paragraphe 1} où elle est notée \(\sum_{n=0}^{\pinf}x_n\) ; \\
    \item une seconde au paragraphe \hyperref[sec:famillesSommablesDeRéelsPositifs]{paragraphe 5} où elle est notée \(\sum_{n\in\N}x_n\).
\end{itemize}

La grande différence entre les deux façons est que l'ordre dans lequel sont énumérés les termes de la somme est important dans la première ; pas dans la seconde.

\section{Séries}\label{sec:séries}

\subsection{Deux exemples}

\begin{ex}[Série géométrique]
On a la série géométrique de raison \(\dfrac{1}{2}\) : \[\dfrac{1}{2}+\dfrac{1}{4}+\dfrac{1}{8}+\dfrac{1}{16}+\dots=1.\]
\end{ex}

\begin{ex}[Série harmonique]
On a la série harmonique : \[\dfrac{1}{1}+\dfrac{1}{2}+\dfrac{1}{3}+\dfrac{1}{4}+\dfrac{1}{5}+\dots=\pinf.\]
\end{ex}

\subsection{Séries convergentes, séries divergentes}

\begin{defi}[Série convergente, série divergente]
À toute suite \(\paren{x_n}_{n\in\N}\in\K^\N\) on associe une série notée \[\sum x_n\qquad\text{ou}\qquad\sum_nx_n\qquad\text{ou}\qquad\sum_{n\geq0}x_n\] et appelée série de terme général \(x_n\).

La suite \(\paren{S_n}_n\) des sommes partielles de la série \(\sum_nx_n\) est définie par : \[\quantifs{\forall n\in\N}S_n=\sum_{k=0}^nx_k.\]

On dit que la série \(\sum_nx_n\) converge ou est convergente si la suite \(\paren{S_n}_n\) de ses sommes partielles est convergente.

On définit alors la somme de la série \(\sum_nx_n\) par : \[\sum_{n=0}^{\pinf}x_n=\lim_{N\to\pinf}\sum_{n=0}^Nx_n.\]

Sinon, on dit que la série \(\sum_nx_n\) diverge ou est divergente.

Déterminer la nature d'une série ou étudier sa convergence, c'est décider si elle est convergente ou divergente.
\end{defi}

\begin{rem}
(Mêmes notations)

La série \(\sum_nx_n\) est toujours définie ; sa somme \(\sum_{n=0}^{\pinf}x_n\) n'est définie que si la série est convergente.
\end{rem}

\begin{rem}
En pratique, la suite \(\paren{x_n}_n\) n'est pas toujours définie à partir du rang \(n=0\).

Si la suite \(\paren{x_n}_{n\in\interventierie{n_0}{\pinf}}\in\K^{\interventierie{n_0}{\pinf}}\) est définie à partir d'un certain rang \(n_0\in\N\), on lui associe la série notée \[\sum x_n\qquad\text{ou}\qquad\sum_nx_n\qquad\text{ou}\qquad\sum_{n\geq n_0}x_n\] dont la suite \(\paren{S_n}_{n\geq n_0}\) des sommes partielles est définie par : \[\quantifs{\forall n\in\interventierie{n_0}{\pinf}}S_n=\sum_{k=n_0}^nx_k.\]

On dit que la série \(\sum_{n\geq n_0}x_n\) converge ou est convergente si la suite \(\paren{S_n}_n\) de ses sommes partielles est convergente.

On définit alors la somme de la série \(\sum_{n\geq n_0}x_n\) par : \[\sum_{n=n_0}^{\pinf}x_n=\lim_{N\to\pinf}\sum_{n=n_0}^Nx_n.\]

Dans la suite du cours, on suppose que la suite \(\paren{x_n}_n\) est définie à partir de \(n=0\) pour simplifier l'exposé.
\end{rem}

\begin{rem}[Tronquer une série]\thlabel{rem:tronquerUneSérie}
Soient \(\sum_{n\geq0}x_n\) une série et \(n_0\in\N\).

Les séries \(\sum_{n\geq0}x_n\) et \(\sum_{n\geq n_0}x_n\) sont de même nature (on dit qu'\guillemets{on ne change pas la nature d'une série en la tronquant}).

Si elles convergent, leurs sommes respectives vérifient : \[\sum_{n=0}^{\pinf}x_n=\sum_{n=0}^{n_0-1}x_n+\sum_{n=n_0}^{\pinf}x_n.\]
\end{rem}

\begin{dem}
On note \(\paren{S_n}_n\) et \(\paren{S_n\prim}_n\) les suites des sommes partielles respectives des séries \(\sum_{n\geq0}x_n\) et \(\sum_{n\geq n_0}x_n\).

On a : \[\quantifs{\forall n\geq n_0}S_n=\sum_{k=0}^nx_k=\sum_{k=0}^{n_0-1}x_k+S_n\prim.\]

Donc : \[\paren{S_n}_n\text{ converge}\ssi\paren{S_n\prim}_n\text{ converge}.\]

D'où : \[\sum_{n\geq0}x_n\text{ converge}\ssi\sum_{n\geq n_0}x_n\text{ converge}\] et, en cas de convergence, \(\lim_nS_n=\sum_{k=0}^{n_0-1}x_k+\lim_nS_n\prim\), \cad : \[\sum_{n=0}^{\pinf}x_n=\sum_{n=0}^{n_0-1}x_n+\sum_{n=n_0}^{\pinf}x_n.\]
\end{dem}

\begin{defprop}[Restes d'une série convergente]
Soit \(\sum_{n\geq0}x_n\) une série convergente.

On définit la suite \(\paren{R_n}_{n\in\N}\) des restes de cette série : \[\quantifs{\forall n\in\N}R_n=\sum_{k=n+1}^{\pinf}x_k.\]

Celle-ci vérifie : \[\quantifs{\forall n\in\N}S_n+R_n=\sum_{k=0}^{\pinf}x_k\] et : \[\lim_{n\to\pinf}R_n=0.\]
\end{defprop}

\begin{dem}
Pour tout \(n\in\N\), \(R_n\) est bien défini car la série \(\sum_nx_n\) converge.

D'après la \thref{rem:tronquerUneSérie}, on a bien : \[\quantifs{\forall n\in\N}S_n+R_n=\sum_{k=0}^{\pinf}x_k.\]

D'où : \[\begin{aligned}
\quantifs{\forall n\in\N}R_n&=\sum_{k=0}^{\pinf}-S_n \\
&\tendqd{n\to\pinf}\sum_{k=0}^{\pinf}x_k-\sum_{k=0}^{\pinf}x_k \\
&=0.
\end{aligned}\]
\end{dem}

\begin{prop}[Linéarité de la somme d'une série]
Soient \(\lambda,\mu\in\K\) et \(\paren{x_n}_n,\paren{y_n}_n\in\K^\N\).

Si les séries \(\sum_{n\geq0}x_n\) et \(\sum_{n\geq0}y_n\) sont convergentes alors la série \(\sum_{n\geq0}\paren{\lambda x_n+\mu y_n}\) l'est aussi et sa somme est : \[\sum_{n=0}^{\pinf}\paren{\lambda x_n+\mu y_n}=\lambda\sum_{n=0}^{\pinf}x_n+\mu\sum_{n=0}^{\pinf}y_n.\]
\end{prop}

\begin{dem}
\note{Exercice}
\end{dem}

\begin{rem}
\begin{itemize}
    \item La somme de deux séries convergentes est une série convergente. \\
    \item La somme d'une série convergente et d'une série divergente est une série divergente. \\
    \item Dans le cas de la somme de deux séries divergentes, on ne peut pas conclure a priori.
\end{itemize}
\end{rem}

\subsection{Séries grossièrement divergentes}

\begin{prop}\thlabel{prop:sérieConvergenteImpliqueLimiteDuTermeGénéralNulle}
Soit \(\paren{x_n}_{n\in\N}\in\K^\N\).

On a : \[\sum_nx_n\text{ converge}\imp\lim_{n\to\pinf}x_n=0.\]
\end{prop}

\begin{dem}
Supposons \(\sum_nx_n\) convergente, \cad \(\lim_nS_n=\sum_{k=0}^{\pinf}x_k\in\K\).

On remarque \(\quantifs{\forall n\in\Ns}x_n=S_n-S_{n-1}\) donc : \[\lim_nx_n=\lim_nS_n-\lim_nS_{n-1}=0.\]
\end{dem}

\begin{defi}
Soit \(\paren{x_n}_{n\in\N}\in\K^\N\).

Si la suite \(\paren{x_n}_n\) ne tend pas vers \(0\), la série \(\sum_nx_n\) est dite grossièrement divergente.

Toute série grossièrement divergente est divergente.
\end{defi}

\begin{rem}
L'implication réciproque de la \thref{prop:sérieConvergenteImpliqueLimiteDuTermeGénéralNulle} est fausse.

Autrement dit : pour qu'une série soit convergente, il ne suffit pas qu'elle ne soit pas grossièrement divergente.
\end{rem}

\begin{dem}
La série harmonique \(\sum_{n\geq1}\dfrac{1}{n}\) n'est pas grossièrement divergente mais divergente.
\end{dem}

\subsection{Exemples}

\subsubsection{Séries géométriques}

\begin{defi}[Série géométrique]
Soit \(\paren{x_n}_{n\in\N}\in\K^\N\).

On dit que \(\sum_nx_n\) est une série géométrique si la suite \(\paren{x_n}_n\) est une suite géométrique : \[\quantifs{\exists q\in\C;\forall n\in\N}x_{n+1}=qx_n.\]

L'élément \(q\) est appelé raison de la série.
\end{defi}

\begin{prop}[Convergence des séries géométriques]
Soit \(q\in\C\).

On a : \[\sum_nq^n\text{ converge}\ssi\abs{q}<1.\]

Sa somme vaut alors \[\sum_{k=0}^{\pinf}q^k=\dfrac{1}{1-q}.\]

Si \(\abs{q}\geq1\), la série géométrique est en fait grossièrement divergente.
\end{prop}

\begin{dem}
Rappel : la suite géométrique \(\paren{q^n}_{n\in\N}\) \(\begin{dcases}
\text{converge vers }0 &\text{si }\abs{q}<1 \\
\text{converge vers }1 &\text{si }\abs{q}=1 \\
\text{diverge} &\text{si }\begin{dcases}
\abs{q}\geq1 \\
q\not=1
\end{dcases}
\end{dcases}\)

Si \(\abs{q}\geq1\) alors \(\sum_nq^n\) est grossièrement divergente donc divergente.

Si \(\abs{q}<1\), on a : \[\begin{aligned}
\quantifs{\forall n\in\N}S_n&=\sum_{k=0}^nq^k \\
&=\dfrac{1-q^{n+1}}{1-q} \\
&\tendqd{n\to\pinf}\dfrac{1}{1-q}.
\end{aligned}\]
\end{dem}

\begin{rem}
Soient \(q\in\C\) tel que \(\abs{q}<1\) et \(n_0\in\N\).

On a : \[\sum_{n=n_0}^{\pinf}q^n=q^{n_0}\sum_{n=n_0}^{\pinf}q^{n-n_0}=q^{n_0}\sum_{k=0}^{\pinf}q^k=\dfrac{q^{n_0}}{1-q}.\]
\end{rem}

\subsubsection{Séries télescopiques}

\begin{defi}[Série télescopique]
Une série télescopique est une série écrite sous la forme \[\sum_n\paren{u_{n+1}-u_n}\] où \(\paren{u_n}_n\in\K^\N\).
\end{defi}

\begin{prop}[Convergence des séries télescopiques]
Soit \(\paren{u_n}_n\in\K^\N\).

On a : \[\sum_n\paren{u_{n+1}-u_n}\text{ converge}\ssi\paren{u_n}_n\text{ converge}.\]

Sa somme vaut alors \[\sum_{k=0}^{\pinf}\paren{u_{k+1}-u_k}=\lim_nu_n-u_0.\]
\end{prop}

\begin{dem}
On remarque \(\quantifs{\forall n\in\N}\sum_{k=0}^n\paren{u_{k+1}-u_k}=u_{n+1}-u_0\) donc : \[\begin{aligned}
\sum_n\paren{u_{n+1}-u_n}\text{ converge}&\ssi\paren{u_{n+1}-u_0}_n\text{ converge} \\
&\ssi\paren{u_n}_n\text{ converge}.
\end{aligned}\]

On a alors : \[\begin{aligned}
\sum_{k=0}^{\pinf}\paren{u_{k+1}-u_k}&=\lim_n\paren{u_{n+1}-u_0} \\
&=\lim_nu_n-u_0.
\end{aligned}\]
\end{dem}

\begin{exoex}
Calculer : \[\sum_{n=1}^{\pinf}\dfrac{1}{n\paren{n+1}}.\]
\end{exoex}

\begin{corr}
On a : \[\begin{aligned}
\quantifs{\forall n\in\Ns}\sum_{k=1}^n\dfrac{1}{k\paren{k+1}}&=\sum_{k=1}^n\paren{\dfrac{1}{k}-\dfrac{1}{k+1}} \\
&=1-\dfrac{1}{n+1} \\
&\tendqd{n\to\pinf}1.
\end{aligned}\]

Donc \(\sum_n\dfrac{1}{n\paren{n+1}}\) converge et on a \(\sum_{n=1}^{\pinf}\dfrac{1}{n\paren{n+1}}=1\).
\end{corr}

\begin{exoex}
Calculer : \[\sum_{n=1}^{\pinf}\dfrac{n}{\paren{n+1}!}.\]
\end{exoex}

\begin{corr}
On a : \[\begin{aligned}
\quantifs{\forall n\in\Ns}\sum_{k=1}^n\dfrac{k}{\paren{k+1}!}&=\sum_{k=1}^n\paren{\dfrac{1}{k!}-\dfrac{1}{\paren{k+1}!}} \\
&=1-\dfrac{1}{\paren{n+1}!} \\
&\tendqd{n\to\pinf}1.
\end{aligned}\]

Donc \(\sum_n\dfrac{n}{\paren{n+1}!}\) converge et on a \(\sum_{n=1}^{\pinf}\dfrac{n}{\paren{n+1}!}=1\).
\end{corr}

\begin{rem}
Toute suite peut être vue comme la suite des sommes partielles d'une série.

Cela permet d'appliquer aux suites les outils dont on dispose pour les séries (\cf TD).
\end{rem}

\begin{dem}
Soit \(\paren{u_n}_n\in\K^\N\).

Posons \(u_{-1}=0\).

La suite des sommes partielles de la série \(\sum_{n\geq0}\paren{u_n-u_{n-1}}\) est \(\paren{u_n}_{n\geq0}\).

En effet : \[\quantifs{\forall n\in\N}\sum_{k=0}^n\paren{u_k-u_{k-1}}=u_n-u_{-1}=u_n.\]
\end{dem}

\subsubsection{Séries de Riemann}

\begin{defi}[Série de Riemann]
On appelle série de Riemann une série de la forme \(\sum_n\dfrac{1}{n^\alpha}\) où \(\alpha\in\R\).

Si \(\alpha=1\), la série de Riemann \(\sum_n\dfrac{1}{n}\) est appelée série harmonique.
\end{defi}

\begin{ex}
On a vu au DS 9, question 21 : \[\sum_{n=1}^{\pinf}\dfrac{1}{n^2}=\dfrac{\pi^2}{6}.\]
\end{ex}

\begin{rem}
Soit \(\alpha\in\R\).

On montrera au \hyperref[subsubsec:applicationAuxSériesDeRiemann]{paragraphe 2.3.2} que la série de Riemann \(\sum_{n}\dfrac{1}{n^\alpha}\) est convergente si, et seulement si, \(\alpha>1\).

Elle est grossièrement divergente si, et seulement si, \(\alpha\leq0\).
\end{rem}

\begin{nota}[Fonction zêta de Riemann]
On pose : \[\quantifs{\forall\alpha\in\intervee{1}{\pinf}}\zeta\paren{\alpha}=\sum_{n=1}^{\pinf}\dfrac{1}{n^\alpha}.\]
\end{nota}

\subsubsection{Autres exemples}

\begin{rappel}[Inégalité de Taylor-Lagrange (\thref{cor:inégalitéDeTaylorLagrange})]
Soient \(I\) un intervalle de \(\R\), \(n\in\N\), \(f\in\ensclasse{n+1}{I}{\K}\), \(M\in\Rp\) et \(a,b\in I\).

On suppose : \[\quantifs{\forall t\in I}\abs{f\deriv{n+1}\paren{t}}\leq M.\]

On a : \[\abs{f\paren{b}-\sum_{k=0}^n\dfrac{f\deriv{k}\paren{a}}{k!}\paren{b-a}^k}\leq\dfrac{M\abs{b-a}^{n+1}}{\paren{n+1}!}.\]
\end{rappel}

\begin{prop}
On a, pour tous \(x\in\R\) : \[\e{x}=\sum_{n=0}^{\pinf}\dfrac{x^n}{n!}\qquad\cos x=\sum_{n=0}^{\pinf}\dfrac{\paren{-1}^nx^{2n}}{\paren{2n}!}\qquad\sin x=\sum_{n=0}^{\pinf}\dfrac{\paren{-1}^nx^{2n+1}}{\paren{2n+1}!}.\]
\end{prop}

\begin{dem}
Soit \(x\in\R\).

Montrons que \(\sum_{n=0}^{\pinf}\dfrac{x^n}{n!}=\e{x}\), \cad \(\sum_n\dfrac{x^n}{n!}\) converge et sa somme vaut \(\e{x}\), \cad \(\lim_n\sum_{k=0}^n\dfrac{x^k}{k!}=\e{x}\).

Soit \(n\in\N\).

On a : \[\quantifs{\forall t\in\intervii{-\abs{x}}{\abs{x}}}\abs{\exp\deriv{n+1}\paren{t}}=\e{t}\leq\e{\abs{x}}.\]

Donc, selon l'inégalité de Taylor-Lagrange appliquée entre \(0\) et \(x\) : \[\abs{\e{x}-\sum_{k=0}^n\dfrac{1}{k!}\paren{x-0}^k}\leq\dfrac{\e{\abs{x}}\abs{x-0}^{n+1}}{\paren{n+1}!}.\]

Or \(x^{n+1}\egqd{n\to\pinf}\o{\paren{n+1}!}\) donc : \[\lim_n\dfrac{\e{\abs{x}}\abs{x-0}^{n+1}}{\paren{n+1}!}=0.\]

Donc, selon le théorème des gendarmes : \[\lim_n\sum_{k=0}^n\dfrac{x^k}{k!}=\sum_{k=0}^{\pinf}\dfrac{x^k}{k!}=\e{x}.\]

De plus, on a : \[\quantifs{\forall t\in\R}\abs{\cos\deriv{2n+1}\paren{t}}\leq1.\]

Donc, selon l'inégalité de Taylor-Lagrange appliquée entre \(0\) et \(x\) : \[\abs{\cos x-\sum_{k=0}^{2n}\dfrac{\cos\deriv{k}\paren{0}}{k!}\paren{x-0}^k}\leq\dfrac{\abs{x-0}^{2n+1}}{\paren{2n+1}!}.\]

Or \(x^{2n+1}\egqd{n\to\pinf}\o{\paren{2n+1}!}\) donc : \[\lim_n\dfrac{\abs{x-0}^{2n+1}}{\paren{2n+1}!}=0.\]

Donc, selon le théorème des gendarmes : \[\lim_n\sum_{k=0}^n\dfrac{\paren{-1}^kx^{2k}}{\paren{2k}!}=\sum_{k=0}^{\pinf}\dfrac{\paren{-1}^kx^{2k}}{\paren{2k}!}=\cos x.\]

Idem pour \(\sin\).
\end{dem}

\section{Séries à termes positifs}

\subsection{Convergence des séries à termes positifs}

\begin{defi}
Une série à termes positifs est une série de la forme \(\sum_nx_n\) où \(\paren{x_n}_n\in\paren{\Rp}^\N\) est une suite de réels positifs.
\end{defi}

\begin{rappel}[Théorème de la limite monotone (\thref{theo:théorèmeDeLaLimiteMonotone})]\thlabel{rappel:théorèmeDeLaLimiteMonotone}
Soit \(\paren{u_n}_n\in\R^\N\) une suite réelle croissante.

Alors elle admet une limite et on a : \[\lim_nu_n=\begin{dcases}
\sup_{n\in\N}u_n &\text{si la suite est majorée} \\
\pinf &\text{sinon}
\end{dcases}\]
\end{rappel}

\begin{theo}\thlabel{theo:convergenceDesSériesATermesPositifs}
Une série à termes positifs converge si, et seulement si, la suite de ses sommes partielles est majorée.
\end{theo}

\begin{dem}
Notons \(\paren{S_n}_n\) la suite des sommes partielles de la série \(\sum_nx_n\).

On remarque : \[\quantifs{\forall n\in\N}S_{n+1}=\sum_{k=0}^{n+1}x_k=S_n+x_{n+1}\geq S_n.\]

Donc \(\paren{S_n}_n\) est croissante.

D'où : \[\begin{WithArrows}
\sum_nx_n\text{ converge}&\ssi\paren{S_n}_n\text{ converge} \Arrow{\thref{rappel:théorèmeDeLaLimiteMonotone}} \\
&\ssi\paren{S_n}_n\text{ est majorée}.
\end{WithArrows}\]
\end{dem}

\begin{nota}
Soit \(\sum_nx_n\) une série à termes positifs.

Si la série est divergente, on s'autorise à écrire : \[\sum_{n=0}^{\pinf}x_n=\pinf.\]

Cette notation est relativement naturelle car la suite des sommes partielles de la série tend vers \(\pinf\) (mais elle est réservée au cas des séries à termes positifs divergentes).
\end{nota}

\subsection{Comparaison des séries à termes positifs}

\begin{theo}[Théorème de comparaison des séries à termes positifs]\thlabel{theo:théorèmeDeComparaisonDesSériesATermesPositifs}
Soient \(\sum_nx_n\) et \(\sum_ny_n\) deux séries à termes positifs.

\begin{enumerate}
    \item Si \(\begin{dcases}
        \quantifs{\forall n\in\N}x_n\leq y_n \\
        \sum_ny_n\text{ converge}
    \end{dcases}\) alors \(\sum_nx_n\) converge. \\
    \item Si \(\begin{dcases}
        x_n\egqd{n\to\pinf}\O{y_n} \\
        \sum_ny_n\text{ converge}
    \end{dcases}\) alors \(\sum_nx_n\) converge. \\
    \item Si \(x_n\simqd{n\to\pinf}y_n\) alors \(\sum_nx_n\) et \(\sum_ny_n\) sont de même nature. \\
    \item Si \(\begin{dcases}
        \quantifs{\forall n\in\N}x_n\leq y_n \\
        \sum_nx_n\text{ diverge}
    \end{dcases}\) alors \(\sum_ny_n\) diverge. \\
    \item Si \(\begin{dcases}
        x_n\egqd{n\to\pinf}\O{y_n} \\
        \sum_nx_n\text{ diverge}
    \end{dcases}\) alors \(\sum_ny_n\) diverge.
\end{enumerate}
\end{theo}

\begin{dem}[1]
Supposons \(\quantifs{\forall n\in\N}x_n\leq y_n\) et \(\sum_ny_n\) converge.

Selon le \thref{theo:convergenceDesSériesATermesPositifs}, il existe \(M\in\Rp\) tel que \[\quantifs{\forall n\in\N}\sum_{k=0}^ny_k\leq M.\]

Donc : \[\quantifs{\forall n\in\N}\sum_{k=0}^nx_k\leq M.\]

Donc \(\sum_nx_n\) converge car c'est une série à termes positifs dont la suite des sommes partielles est majorée.
\end{dem}

\begin{dem}[2]
Découle du (1) car si \(x_n\egqd{n\to\pinf}\O{y_n}\) alors il existe \(K\in\Rp\) tel que \(\quantifs{\forall n\in\N}x_n\leq Ky_n\) et si \(\sum_ny_n\) converge alors \(\sum_nKy_n\) aussi.
\end{dem}

\begin{dem}[3]
Découle de (2) car \(x_n\simqd{n\to\pinf}y_n\imp\begin{dcases}
x_n\egqd{n\to\pinf}\O{y_n} \\
y_n\egqd{n\to\pinf}\O{x_n}
\end{dcases}\)
\end{dem}

\begin{dem}[4]
Contraposée de (1).
\end{dem}

\begin{dem}[5]
Contraposée de (2).
\end{dem}

\begin{rem}
Le (1) est suffisant pour énoncer le théorème.
\end{rem}

\begin{rem}
Attention à ne surtout pas utiliser le \thref{theo:convergenceDesSériesATermesPositifs} pour comparer des séries quelconques (\cf \thref{exoex:contreExempleUtilisationDuThéorèmeDeComparaisonDesSériesATermesPositifsSurDesSériesQuelconques} pour un contre-exemple).

On peut comparer des séries à termes positifs, ou plus généralement, des séries dont les termes sont de signe constant à partir d'un certain rang.
\end{rem}

\begin{exoex}
Soit \(q\in\Rps\).

Étudier la nature des séries suivantes :

\begin{enumerate}
    \item \(\sum_n\sin\dfrac{1}{2^n}\) \\
    \item \(\sum_n\dfrac{q^n}{1+q^n}\)
\end{enumerate}
\end{exoex}

\begin{corr}[1]~\\
On a \(\begin{dcases}
\sin h\simqd{h\to0}h \\
\lim_n\dfrac{1}{2^n}=0
\end{dcases}\) donc \(\sin\dfrac{1}{2^n}\simqd{n\to\pinf}\dfrac{1}{2^n}\).

Or la série géométrique \(\sum_n\dfrac{1}{2^n}\) converge donc selon le théorème de comparaison des séries à termes positifs, \(\sum_n\sin\dfrac{1}{2^n}\) converge.
\end{corr}

\begin{corr}[2]
On a, quand \(n\to\pinf\) : \[1+q^n\sim\begin{dcases}
1 &\text{si }q<1 \\
2 &\text{si }q=1 \\
q^n &\text{si }q>1
\end{dcases}\] donc : \[\dfrac{q^n}{1+q^n}\sim\begin{dcases}
q^n &\text{si }q<1 \\
\dfrac{1}{2} &\text{si }q=1 \\
1 &\text{si }q>1
\end{dcases}\]

Si \(q\geq1\) : \(\sum_n\dfrac{q^n}{1+q^n}\) diverge grossièrement.

Sinon, selon le théorème de comparaison des séries à termes positifs : \[\sum_n\dfrac{q^n}{1+q^n}\text{ converge}\ssi\sum_nq^n\text{ converge}.\]

Or la série géométrique \(\sum_nq^n\) converge car \(\abs{q}<1\) donc \(\sum_n\dfrac{q^n}{1+q^n}\) converge.

Conclusion : \[\sum_n\dfrac{q^n}{1+q^n}\text{ converge}\ssi q<1.\]
\end{corr}

\subsection{Utilisation des intégrales}

\subsubsection{Méthode}

\begin{rem}
La proposition suivante est en fait une méthode pour encadrer les sommes partielles d'une série. Vous énoncerez en deuxième année le \guillemets{théorème de comparaison série-intégrale} qui repose sur cette méthode.
\end{rem}

\begin{prop}
Soit une fonction décroissante \(f:\Rp\to\R\).

On a : \[\quantifs{\forall n\in\N}\sum_{k=1}^nf\paren{k}\leq\int_0^nf\paren{t}\odif{t}\leq\sum_{k=0}^{n-1}f\paren{k}.\]

D'où l'encadrement des sommes partielles de la série \(\sum_{n\geq0}f\paren{n}\) : \[\quantifs{\forall n\in\N}\int_0^{n+1}f\paren{t}\odif{t}\leq\sum_{k=0}^nf\paren{k}\leq f\paren{0}+\int_0^nf\paren{t}\odif{t}.\]
\end{prop}

\begin{dem}
Comme \(f\) est décroissante, on a : \[\quantifs{\forall k\in\N;\forall t\in\intervii{k}{k+1}}f\paren{k+1}\leq f\paren{t}\leq f\paren{k}.\]

D'où, par croissance de l'intégrale : \[\quantifs{\forall k\in\N}\underbrace{\int_k^{k+1}f\paren{k+1}\odif{t}}_{=f\paren{k+1}}\leq\int_k^{k+1}f\paren{t}\odif{t}\leq\underbrace{\int_k^{k+1}f\paren{k}\odif{t}}_{=f\paren{k}}.\]

D'où, par somme : \[\quantifs{\forall n\in\N}\sum_{k=0}^{n-1}f\paren{k+1}\leq\sum_{k=0}^{n-1}\int_k^{k+1}f\paren{t}\odif{t}\leq\sum_{k=0}^{n-1}\paren{k},\] \cad : \[\quantifs{\forall n\in\N}\sum_{k=1}^nf\paren{k}\leq\int_0^nf\paren{t}\odif{t}\leq\sum_{k=0}^{n-1}f\paren{k}.\]
\end{dem}

\begin{rem}
\begin{itemize}
    \item Ne pas retenir l'énoncé de la proposition : ce qui importe est la méthode. En pratique, redémontrer les encadrements utiles, en s'adaptant au contexte (notamment quand la fonction n'est pas définie en \(0\), \cf \hyperref[subsubsec:applicationAuxSériesDeRiemann]{paragraphe 2.3.2}). \\
    \item La méthode est très intéressante dans le cas (très fréquent) où l'on ne sait pas calculer les sommes partielles alors qu'on sait calculer les intégrales. Elle permet : \begin{itemize}
        \item de montrer la convergence d'une série (\cf \hyperref[subsubsec:applicationAuxSériesDeRiemann]{paragraphe 2.3.2}) ;
        \item d'obtenir un équivalent des sommes partielles d'une série ;
        \item d'obtenir un équivalent des restes d'une série convergente.
    \end{itemize}
\end{itemize}
\end{rem}

\subsubsection{Application aux séries de Riemann}\label{subsubsec:applicationAuxSériesDeRiemann}

\begin{theo}[Convergence des séries de Riemann]
Soit \(\alpha\in\R\).

On a : \[\sum_n\dfrac{1}{n^\alpha}\text{ converge}\ssi\alpha>1.\]
\end{theo}

\begin{dem}~\\
Si \(\alpha\leq0\) alors \(\sum_n\dfrac{1}{n^\alpha}\) est grossièrement divergente.

Sinon, la fonction \(\fonction{f}{\Rps}{\R}{t}{\dfrac{1}{t^\alpha}}\) est décroissante et continue.

On a donc : \[\quantifs{\forall k\in\Ns;\forall t\in\intervii{k}{k+1}}\dfrac{1}{\paren{k+1}^\alpha}\leq\dfrac{1}{t^\alpha}\leq\dfrac{1}{k^\alpha}.\]

D'où, par croissance de l'intégrale : \[\quantifs{\forall k\in\Ns}\dfrac{1}{\paren{k+1}^\alpha}\leq\int_k^{k+1}\dfrac{\odif{t}}{t^\alpha}\leq\dfrac{1}{k^\alpha}.\]

D'où, par somme : \[\quantifs{\forall n\in\interventierie{2}{\pinf}}\sum_{k=1}^{n-1}\dfrac{1}{\paren{k+1}^\alpha}\leq\int_1^n\dfrac{\odif{t}}{t^\alpha}\leq\sum_{k=1}^{n-1}\dfrac{1}{k^\alpha}.\]

D'où : \[\quantifs{\forall n\in\Ns}\int_1^{n+1}\dfrac{\odif{t}}{t^\alpha}\leq\sum_{k=1}^n\dfrac{1}{k^\alpha}\leq1+\int_1^n\dfrac{\odif{t}}{t^\alpha}.\]

Si \(\alpha=1\) (série harmonique), on obtient : \[\quantifs{\forall n\in\Ns}\ln\paren{n+1}\leq\sum_{k=1}^n\dfrac{1}{k}\leq1+\ln n.\]

Or \(\lim_n\ln\paren{n+1}=\lim_n\paren{1+\ln n}=\pinf\) donc selon le théorème des gendarmes, on a \(\lim_n\sum_{k=1}^n\dfrac{1}{k}=\pinf\).

Donc la suite des sommes partielles de la série à termes positifs \(\sum_n\dfrac{1}{n}\) n'est pas majorée donc la série diverge.

Si \(\alpha<1\), on a \(\dfrac{1}{n}\egqd{n\to\pinf}\O{\dfrac{1}{n^\alpha}}\) et \(\sum_n\dfrac{1}{n}\) diverge donc selon le théorème de comparaison des séries à termes positifs, la série \(\sum_n\dfrac{1}{n^\alpha}\) diverge.

Si \(\alpha>1\), on obtient : \[\quantifs{\forall n\in\Ns}\dfrac{1}{\alpha-1}\paren{1-\dfrac{1}{\paren{n+1}^{\alpha-1}}}\leq\sum_{k=1}^n\dfrac{1}{k^\alpha}\leq1+\int_1^nt^{-\alpha}\odif{t}=\croch{\dfrac{t^{1-\alpha}}{1-\alpha}}_1^n=1+\dfrac{1}{\alpha-1}\paren{1-\dfrac{1}{n^{\alpha-1}}}.\]

D'où : \[\quantifs{\forall n\in\Ns}\sum_{k=1}^n\dfrac{1}{k^\alpha}\leq1+\dfrac{1}{\alpha-1}.\]

Donc \(\sum_n\dfrac{1}{n^\alpha}\) converge car c'est une série à termes positifs dont la suite des sommes partielles est majorée.
\end{dem}

\begin{exoex}
Étudier la nature des séries suivantes :

\begin{enumerate}
    \item \(\sum_n\paren{\tan\dfrac{1}{n}-\sin\dfrac{1}{n}}\) \\
    \item \(\sum_n\dfrac{1}{n^{1+\nicefrac{1}{n}}}\)
\end{enumerate}
\end{exoex}

\begin{corr}[1]
On a : \[\sin x\egqd{x\to0}x-\dfrac{x^3}{6}+\o{x^3}\qquad\text{et}\qquad\tan x\egqd{x\to0}x+\dfrac{x^3}{3}+\o{x^3}.\]

Donc : \[\begin{aligned}
\tan\dfrac{1}{n}-\sin\dfrac{1}{n}&\egqd{n\to\pinf}\dfrac{1}{3n^3}+\dfrac{1}{6n^3}+\o{\dfrac{1}{n^3}} \\
&\egqd{n\to\pinf}\dfrac{1}{2n^3}+\o{\dfrac{1}{n^3}} \\
&\simqd{n\to\pinf}\dfrac{1}{2n^3}.
\end{aligned}\]

En particulier, \(\sum_n\paren{\tan\dfrac{1}{n}-\sin\dfrac{1}{n}}\) est une série à termes positifs à partir d'un certain rang.

Selon le théorème de comparaison des séries à termes positifs (à partir d'un certain rang) : \[\sum_n\paren{\tan\dfrac{1}{n}-\sin\dfrac{1}{n}}\text{ et }\sum_n\dfrac{1}{2n^3}\text{ sont de même nature}.\]

Or la série de Riemann \(\sum_n\dfrac{1}{n^3}\) converge donc \(\sum_n\paren{\tan\dfrac{1}{n}-\sin\dfrac{1}{n}}\) converge.
\end{corr}

\begin{corr}[2]
On a : \[\begin{aligned}
n^{1+\nicefrac{1}{n}}&=\exp\paren{\paren{1+\dfrac{1}{n}}\ln n} \\
&=\exp\paren{\ln n+\dfrac{\ln n}{n}} \\
&=n\underbrace{\exp\underbrace{\dfrac{\ln n}{n}}_{\tendqd{n\to\pinf}0}}_{\simqd{n\to\pinf}1} \\
&\simqd{n\to\pinf}n.
\end{aligned}\]

Donc \(\dfrac{1}{n^{1+\nicefrac{1}{n}}}\simqd{n\to\pinf}\dfrac{1}{n}\).

Donc \(\sum_n\dfrac{1}{n^{1+\nicefrac{1}{n}}}\) et \(\sum_n\dfrac{1}{n}\) sont de même nature selon le théorème de comparaison des séries à termes positifs.

Or la série harmonique diverge donc \(\sum_n\dfrac{1}{n^{1+\nicefrac{1}{n}}}\) diverge.
\end{corr}

\section{Séries absolument convergentes}

\begin{defi}[Série absolument convergente]
Soit \(\paren{x_n}_n\in\K^\N\).

On dit que la série \(\sum_nx_n\) est absolument convergente ou converge absolument si la série \(\sum_n\abs{x_n}\) est convergente.
\end{defi}

\begin{rem}
Pour une série à termes positifs, la convergence absolue équivaut à la convergence.
\end{rem}

\begin{rem}
Une série convergente qui ne converge pas absolument est dite semi-convergente.
\end{rem}

\begin{theo}[Convergence absolue \(\imp\) convergence]
Soit \(\paren{x_n}_n\in\K^\N\).

Si la série \(\sum_nx_n\) est absolument convergente alors elle est convergente.
\end{theo}

\begin{dem}
Supposons \(\sum_nx_n\) absolument convergente, \cad \(\sum_x\abs{x_n}\) convergente.

Si \(\K=\R\) :

On remarque : \[\begin{WithArrows}
\quantifs{\forall n\in\N}&-\abs{x_n}\leq x_n\leq\abs{x_n} \Arrow{\(+\abs{x_n}\)} \\
&0\leq x_n+\abs{x_n}\leq2\abs{x_n}.
\end{WithArrows}\]

Or \(\sum_n2\abs{x_n}\) est convergente et \(\sum_n\paren{x_n+\abs{x_n}}\) et \(\sum_n2\abs{x_n}\) sont des séries à termes positifs.

Donc selon le théorème de comparaison des séries à termes positifs, \(\sum_n\paren{x_n+\abs{x_n}}\) converge.

Or \(\sum_n\abs{x_n}\) converge donc \(\sum_nx_n\) converge (différence de deux séries convergentes).

Si \(\K=\C\) :

On a : \[\quantifs{\forall n\in\N}\begin{dcases}
\abs{\Re x_n}\leq\abs{x_n} \\
\abs{\Im x_n}\leq\abs{x_n}
\end{dcases}\]

Donc les séries \(\sum_n\abs{\Re x_n}\) et \(\sum_n\abs{\Im x_n}\) sont convergentes selon le théorème de comparaison des séries à termes positifs.

Donc \(\sum_n\Re x_n\) et \(\sum_n\Im x_n\) sont absolument convergentes et donc convergentes selon le cas \(\K=\R\).

Donc \(\sum_nx_n=\sum_n\Re x_n+\i\sum_n\Im x_n\) est convergente.
\end{dem}

\begin{bilan}[Convergence absolue \(\imp\) convergence \(\imp\) non divergence grossière]
Pour toute série \(\sum_nx_n\), on a : \[\sum_n\abs{x_n}\text{ converge}\imp\sum_nx_n\text{ converge}\imp\lim_{n\to\pinf}x_n=0.\]
\end{bilan}

\begin{rem}[Inégalité triangulaire pour les séries absolument convergentes]
Soit \(\sum_nx_n\) une série absolument convergente.

On a : \[\abs{\sum_{n=0}^{\pinf}x_n}\leq\sum_{n=0}^{\pinf}\abs{x_n}.\]
\end{rem}

\begin{dem}
On a, selon l'inégalité triangulaire dans \(\K\) : \[\quantifs{\forall N\in\N}\abs{\sum_{n=0}^Nx_n}\leq\sum_{n=0}^N\abs{x_n}.\]

Or, quand \(N\to\pinf\) : \[\begin{dcases}
\abs{\sum_{n=0}^Nx_n}\to\abs{\sum_{n=0}^{\pinf}x_n} &\text{car }\sum_nx_n\text{ converge absolument donc converge} \\
\sum_{n=0}^N\abs{x_n}\to\sum_{n=0}^{\pinf}\abs{x_n} &\text{car }\sum_nx_n\text{ converge absolument}
\end{dcases}\]

D'où le résultat, par passage à la limite dans l'inégalité.
\end{dem}

\begin{rem}
Pour montrer la convergence absolue d'une série, on peut utiliser le théorème de comparaison des séries à termes positifs :
\end{rem}

\begin{rem}[Corollaire du \thref{theo:théorèmeDeComparaisonDesSériesATermesPositifs}]
Soient \(\paren{x_n}_n\in\K^\N\) et \(\paren{y_n}_n\in\paren{\Rp}^\N\).

Supposons \[\begin{dcases}
x_n\egqd{n\to\pinf}\O{y_n} \\
\sum_ny_n\text{ converge}
\end{dcases}\]

Alors, selon le théorème de comparaison des séries à termes positifs, \(\sum_n\abs{x_n}\) converge.

Donc la série \(\sum_nx_n\) converge absolument, donc elle converge.
\end{rem}

\begin{exoex}
Les séries suivantes sont-elles absolument convergentes ?

\begin{enumerate}
    \item \(\sum_n\dfrac{\paren{-1}^n}{n}\) \\
    \item \(\sum_n\dfrac{\cos n}{n^2\Arctan n}\) \\
    \item \(\sum_n\paren{\tan\dfrac{1}{n}-\sin\dfrac{1}{n}}\)
\end{enumerate}
\end{exoex}

\begin{corr}[1]
On a : \[\sum_n\abs{\dfrac{\paren{-1}^n}{n}}=\sum_n\dfrac{1}{n}.\]

Or, la série harmonique diverge donc \(\sum_n\dfrac{\paren{-1}^n}{n}\) ne converge pas absolument.
\end{corr}

\begin{corr}[2]
On a : \[\quantifs{\forall n\in\Ns}0\leq\dfrac{1}{\Arctan n}\leq\dfrac{1}{\Arctan1}=\dfrac{4}{\pi}.\]

Donc : \[\quantifs{\forall n\in\Ns}\abs{\dfrac{\cos n}{n^2\Arctan n}}\leq\dfrac{4}{\pi n^2}.\]

Or \(\sum_n\dfrac{4}{\pi n^2}\) converge donc selon le théorème de comparaison des séries à termes positifs, \(\sum_n\abs{\dfrac{\cos n}{n^2\Arctan n}}\) converge et donc \(\sum_n\dfrac{\cos n}{n^2\Arctan n}\) converge absolument.
\end{corr}

\begin{corr}[3]
On a, quand \(x\to0\) : \[\tan x=x+\O{x^2}\qquad\text{et}\qquad\sin x=x+\O{x^2}.\]

Donc, quand \(n\to\pinf\) : \[\abs{\tan\dfrac{1}{n}-\sin\dfrac{1}{n}}=\O{\dfrac{1}{n^2}}.\]

Or \(\sum_n\dfrac{1}{n^2}\) est une série de Riemann convergente donc selon le théorème de comparaison des séries à termes positifs, \(\sum_n\abs{\tan\dfrac{1}{n}-\sin\dfrac{1}{n}}\) converge et donc \(\sum_n\paren{\tan\dfrac{1}{n}-\sin\dfrac{1}{n}}\) converge absolument.
\end{corr}

\section{Séries alternées}

\begin{rappel}[\thref{prop:suiteConvergenteSsiSuitesExtraitesPairesEtImpairesConvergentVersLaMêmeLimite}]
Soient \(\paren{u_n}_n\in\K^\N\) et \(l\in\K\).

On a : \[\lim_{n\to\pinf}u_n=l\ssi\lim_{n\to\pinf}u_{2n}=\lim_{n\to\pinf}u_{2n+1}=l.\]
\end{rappel}

\begin{rappel}[Suites adjacentes (\thref{theo:théorèmeDesSuitesAdjacentes})]
Deux suites réelles \(\paren{a_n}_n,\paren{b_n}_n\in\R^\N\) sont dites adjacentes si l'une est croissante, l'autre est décroissante et la limite de leur différence est nulle.

Les deux suites sont alors convergentes et de même limite.
\end{rappel}

\begin{defi}
On appelle série alternée une série de la forme \(\sum_n\paren{-1}^na_n\) où \(\paren{a_n}_{n\in\N}\) est une suite de réelle de signe constant : \[\orenv{\quantifs{\forall n\in\N}a_n\geq0 \\ \quantifs{\forall n\in\N}a_n\leq0}\]
\end{defi}

\begin{theo}[Critère spécial des séries alternées (CSSA)]
Soit \(\paren{a_n}_n\in\R^\N\).

On suppose :

\begin{enumerate}
    \item \(\paren{a_n}_n\) est positive : \(\quantifs{\forall n\in\N}a_n\geq0\) ; \\
    \item \(\paren{a_n}_n\) est décroissante : \(\quantifs{\forall n\in\N}a_{n+1}\leq a_n\) ; \\
    \item \(\paren{a_n}_n\) tend vers \(0\) : \(\lim_{n\to\pinf}a_n=0\).
\end{enumerate}

Alors la série \(\sum_n\paren{-1}^na_n\) est convergente.

De plus, on a les encadrements : \[\quantifs{\forall n\in\N}-a_{2n+1}\leq R_{2n}=\sum_{k=2n+1}^{\pinf}\paren{-1}^ka_k\leq0\] et : \[\quantifs{\forall n\in\N}0\leq R_{2n+1}=\sum_{k=2n+2}^{\pinf}\paren{-1}^ka_k\leq a_{2n+2}.\]

En d'autres termes, le reste est du signe de son premier terme et on a la \guillemets{majoration du reste} : \[\quantifs{\forall N\in\N}\abs{R_N}\leq a_{N+1}.\]
\end{theo}

\begin{dem}[Convergence de la série]
Montrons que \(\paren{S_{2n}}_n\) et \(\paren{S_{2n+1}}_n\) sont adjacentes, où \(\paren{S_N}_N\) est la suite des sommes partielles de la série : \[\quantifs{\forall N\in\N}S_N=\sum_{k=0}^N\paren{-1}^ka_k.\]

On a : \[\begin{aligned}
\quantifs{\forall n\in\N}S_{2\paren{n+1}}&=S_{2n}+\paren{-1}^{2n+1}a_{2n+1}+\paren{-1}^{2n+2}a_{2n+2} \\
&=S_{2n}-a_{2n+1}+a_{2n+2} \\
&\leq S_{2n}.
\end{aligned}\]

Donc \(\paren{S_{2n}}_n\) est décroissante.

De même : \[\begin{aligned}
\quantifs{\forall n\in\N}S_{2n+3}&=S_{2n+1}+a_{2n+2}-a_{2n+3} \\
&\geq S_{2n+1}.
\end{aligned}\]

Donc \(\paren{S_{2n+1}}_n\) est croissante.

Enfin, on a : \[\quantifs{\forall n\in\N}S_{2n+1}-S_{2n}=-a_{2n+1}\tendqd{n\to\pinf}0.\]

Donc \(\paren{S_{2n}}_n\) et \(\paren{S_{2n+1}}_n\) sont adjacentes et donc convergentes de même limite \(l\in\R\).

Donc \(\sum_n\paren{-1}^na_n\) est convergente (de somme \(l\)).
\end{dem}

\begin{dem}[Encadrements]
Comme \(\paren{S_{2n+1}}_n\) est croissante de limite \(l\) et \(\paren{S_{2n}}_n\) est décroissante de limite \(l\), on a : \[\quantifs{\forall n\in\N}S_{2n+1}\leq l\leq S_{2n},\] \cad : \[\quantifs{\forall n\in\N}\sum_{k=0}^{2n+1}\paren{-1}^ka_k\leq\sum_{k=0}^{\pinf}\paren{-1}^ka_k\leq\sum_{k=0}^{2n}\paren{-1}^ka_k.\]

D'où, en soustrayant \(\sum_{k=0}^{2n}\paren{-1}^ka_k\) : \[\quantifs{\forall n\in\N}-a_{2n+1}\leq\underbrace{\sum_{k=2n+1}^{\pinf}\paren{-1}^ka_k}_{=R_{2n}}\leq0.\]

Idem pour l'autre encadrement en partant de \(S_{2n+1}\leq l\leq S_{2n+2}\).
\end{dem}

\begin{ex}~\\
La série \(\sum_{n}\dfrac{\paren{-1}^n}{n}\) est semi-convergente.
\end{ex}

\begin{dem}~\\
La suite \(\paren{\dfrac{1}{n}}_n\) est positive, décroissante et de limite nulle donc d'après le critère spécial des séries alternées, \(\sum_n\dfrac{\paren{-1}^n}{n}\) est convergente.

Or, on a vu qu'elle n'est pas absolument convergente donc elle est semi-convergente.
\end{dem}

\begin{exoex}\thlabel{exoex:contreExempleUtilisationDuThéorèmeDeComparaisonDesSériesATermesPositifsSurDesSériesQuelconques}~\\
\begin{enumerate}
    \item La série \(\sum_n\paren{\dfrac{\paren{-1}^n}{\sqrt{n}}+\dfrac{1}{n}}\) est-elle convergente ? \\
    \item Donner un équivalent de son terme général. Que remarque-t-on ?
\end{enumerate}
\end{exoex}

\begin{corr}[1]~\\
La suite \(\paren{\dfrac{1}{\sqrt{n}}}_n\) est positive, décroissante et de limite nulle donc d'après le critère spécial des séries alternées, \(\sum_n\dfrac{\paren{-1}^n}{\sqrt{n}}\) converge.

Or, \(\sum_n\dfrac{1}{n}\) diverge donc \(\sum_n\paren{\dfrac{\paren{-1}^n}{\sqrt{n}}+\dfrac{1}{n}}\) diverge (somme d'une série convergente et d'une série divergente).
\end{corr}

\begin{corr}[2]~\\
On a \(\dfrac{1}{n}\egqd{n\to\pinf}\o{\dfrac{\paren{-1}^n}{\sqrt{n}}}\) donc : \[\dfrac{1}{n}+\dfrac{\paren{-1}^n}{\sqrt{n}}\simqd{n\to\pinf}\dfrac{\paren{-1}^n}{\sqrt{n}}.\]

Pourtant, \(\sum_n\paren{\dfrac{\paren{-1}^n}{\sqrt{n}}+\dfrac{1}{n}}\) et \(\sum_n\dfrac{\paren{-1}^n}{\sqrt{n}}\) ne sont pas de même nature.
\end{corr}