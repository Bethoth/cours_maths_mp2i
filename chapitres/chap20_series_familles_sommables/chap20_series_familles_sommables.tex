\chapter{Séries, familles sommables}

\minitoc

On pose \(\K=\R\) ou \(\C\).

Dans ce chapitre, on définit la valeur d'une \guillemets{somme infinie} dans \(\K\) de la forme \(x_0+x_1+x_2+\dots\) de deux façons :

\begin{itemize}
    \item une première façon au \hyperref[sec:séries]{paragraphe 1} où elle est notée \(\sum_{n=0}^{\pinf}x_n\) ; \\
    \item une seconde au paragraphe \hyperref[sec:famillesSommablesDeRéelsPositifs]{paragraphe 5} où elle est notée \(\sum_{n\in\N}x_n\).
\end{itemize}

La grande différence entre les deux façons est que l'ordre dans lequel sont énumérés les termes de la somme est important dans la première ; pas dans la seconde.

\section{Séries}\label{sec:séries}

\subsection{Deux exemples}

\begin{ex}[Série géométrique]
On a la série géométrique de raison \(\dfrac{1}{2}\) : \[\dfrac{1}{2}+\dfrac{1}{4}+\dfrac{1}{8}+\dfrac{1}{16}+\dots=1.\]
\end{ex}

\begin{ex}[Série harmonique]
On a la série harmonique : \[\dfrac{1}{1}+\dfrac{1}{2}+\dfrac{1}{3}+\dfrac{1}{4}+\dfrac{1}{5}+\dots=\pinf.\]
\end{ex}

\subsection{Séries convergentes, séries divergentes}

\begin{defi}[Série convergente, série divergente]
À toute suite \(\paren{x_n}_{n\in\N}\in\K^\N\) on associe une série notée \[\sum x_n\qquad\text{ou}\qquad\sum_nx_n\qquad\text{ou}\qquad\sum_{n\geq0}x_n\] et appelée série de terme général \(x_n\).

La suite \(\paren{S_n}_n\) des sommes partielles de la série \(\sum_nx_n\) est définie par : \[\quantifs{\forall n\in\N}S_n=\sum_{k=0}^nx_k.\]

On dit que la série \(\sum_nx_n\) converge ou est convergente si la suite \(\paren{S_n}_n\) de ses sommes partielles est convergente.

On définit alors la somme de la série \(\sum_nx_n\) par : \[\sum_{n=0}^{\pinf}x_n=\lim_{N\to\pinf}\sum_{n=0}^Nx_n.\]

Sinon, on dit que la série \(\sum_nx_n\) diverge ou est divergente.

Déterminer la nature d'une série ou étudier sa convergence, c'est décider si elle est convergente ou divergente.
\end{defi}

\begin{rem}
(Mêmes notations)

La série \(\sum_nx_n\) est toujours définie ; sa somme \(\sum_{n=0}^{\pinf}x_n\) n'est définie que si la série est convergente.
\end{rem}

\begin{rem}
En pratique, la suite \(\paren{x_n}_n\) n'est pas toujours définie à partir du rang \(n=0\).

Si la suite \(\paren{x_n}_{n\in\interventierie{n_0}{\pinf}}\in\K^{\interventierie{n_0}{\pinf}}\) est définie à partir d'un certain rang \(n_0\in\N\), on lui associe la série notée \[\sum x_n\qquad\text{ou}\qquad\sum_nx_n\qquad\text{ou}\qquad\sum_{n\geq n_0}x_n\] dont la suite \(\paren{S_n}_{n\geq n_0}\) des sommes partielles est définie par : \[\quantifs{\forall n\in\interventierie{n_0}{\pinf}}S_n=\sum_{k=n_0}^nx_k.\]

On dit que la série \(\sum_{n\geq n_0}x_n\) converge ou est convergente si la suite \(\paren{S_n}_n\) de ses sommes partielles est convergente.

On définit alors la somme de la série \(\sum_{n\geq n_0}x_n\) par : \[\sum_{n=n_0}^{\pinf}x_n=\lim_{N\to\pinf}\sum_{n=n_0}^Nx_n.\]

Dans la suite du cours, on suppose que la suite \(\paren{x_n}_n\) est définie à partir de \(n=0\) pour simplifier l'exposé.
\end{rem}

\begin{rem}[Tronquer une série]\thlabel{rem:tronquerUneSérie}
Soient \(\sum_{n\geq0}x_n\) une série et \(n_0\in\N\).

Les séries \(\sum_{n\geq0}x_n\) et \(\sum_{n\geq n_0}x_n\) sont de même nature (on dit qu'\guillemets{on ne change pas la nature d'une série en la tronquant}).

Si elles convergent, leurs sommes respectives vérifient : \[\sum_{n=0}^{\pinf}x_n=\sum_{n=0}^{n_0-1}x_n+\sum_{n=n_0}^{\pinf}x_n.\]
\end{rem}

\begin{dem}
On note \(\paren{S_n}_n\) et \(\paren{S_n\prim}_n\) les suites des sommes partielles respectives des séries \(\sum_{n\geq0}x_n\) et \(\sum_{n\geq n_0}x_n\).

On a : \[\quantifs{\forall n\geq n_0}S_n=\sum_{k=0}^nx_k=\sum_{k=0}^{n_0-1}x_k+S_n\prim.\]

Donc : \[\paren{S_n}_n\text{ converge}\ssi\paren{S_n\prim}_n\text{ converge}.\]

D'où : \[\sum_{n\geq0}x_n\text{ converge}\ssi\sum_{n\geq n_0}x_n\text{ converge}\] et, en cas de convergence, \(\lim_nS_n=\sum_{k=0}^{n_0-1}x_k+\lim_nS_n\prim\), \cad : \[\sum_{n=0}^{\pinf}x_n=\sum_{n=0}^{n_0-1}x_n+\sum_{n=n_0}^{\pinf}x_n.\]
\end{dem}

\begin{defprop}[Restes d'une série convergente]
Soit \(\sum_{n\geq0}x_n\) une série convergente.

On définit la suite \(\paren{R_n}_{n\in\N}\) des restes de cette série : \[\quantifs{\forall n\in\N}R_n=\sum_{k=n+1}^{\pinf}x_k.\]

Celle-ci vérifie : \[\quantifs{\forall n\in\N}S_n+R_n=\sum_{k=0}^{\pinf}x_k\] et : \[\lim_{n\to\pinf}R_n=0.\]
\end{defprop}

\begin{dem}
Pour tout \(n\in\N\), \(R_n\) est bien défini car la série \(\sum_nx_n\) converge.

D'après la \thref{rem:tronquerUneSérie}, on a bien : \[\quantifs{\forall n\in\N}S_n+R_n=\sum_{k=0}^{\pinf}x_k.\]

D'où : \[\begin{aligned}
\quantifs{\forall n\in\N}R_n&=\sum_{k=0}^{\pinf}-S_n \\
&\tendqd{n\to\pinf}\sum_{k=0}^{\pinf}x_k-\sum_{k=0}^{\pinf}x_k \\
&=0.
\end{aligned}\]
\end{dem}

\begin{prop}[Linéarité de la somme d'une série]
Soient \(\lambda,\mu\in\K\) et \(\paren{x_n}_n,\paren{y_n}_n\in\K^\N\).

Si les séries \(\sum_{n\geq0}x_n\) et \(\sum_{n\geq0}y_n\) sont convergentes alors la série \(\sum_{n\geq0}\paren{\lambda x_n+\mu y_n}\) l'est aussi et sa somme est : \[\sum_{n=0}^{\pinf}\paren{\lambda x_n+\mu y_n}=\lambda\sum_{n=0}^{\pinf}x_n+\mu\sum_{n=0}^{\pinf}y_n.\]
\end{prop}

\begin{dem}
\note{Exercice}
\end{dem}

\begin{rem}
\begin{itemize}
    \item La somme de deux séries convergentes est une série convergente. \\
    \item La somme d'une série convergente et d'une série divergente est une série divergente. \\
    \item Dans le cas de la somme de deux séries divergentes, on ne peut pas conclure a priori.
\end{itemize}
\end{rem}

\subsection{Séries grossièrement divergentes}

\begin{prop}\thlabel{prop:sérieConvergenteImpliqueLimiteDuTermeGénéralNulle}
Soit \(\paren{x_n}_{n\in\N}\in\K^\N\).

On a : \[\sum_nx_n\text{ converge}\imp\lim_{n\to\pinf}x_n=0.\]
\end{prop}

\begin{dem}
Supposons \(\sum_nx_n\) convergente, \cad \(\lim_nS_n=\sum_{k=0}^{\pinf}x_k\in\K\).

On remarque \(\quantifs{\forall n\in\Ns}x_n=S_n-S_{n-1}\) donc : \[\lim_nx_n=\lim_nS_n-\lim_nS_{n-1}=0.\]
\end{dem}

\begin{defi}
Soit \(\paren{x_n}_{n\in\N}\in\K^\N\).

Si la suite \(\paren{x_n}_n\) ne tend pas vers \(0\), la série \(\sum_nx_n\) est dite grossièrement divergente.

Toute série grossièrement divergente est divergente.
\end{defi}

\begin{rem}
L'implication réciproque de la \thref{prop:sérieConvergenteImpliqueLimiteDuTermeGénéralNulle} est fausse.

Autrement dit : pour qu'une série soit convergente, il ne suffit pas qu'elle ne soit pas grossièrement divergente.
\end{rem}

\begin{dem}
La série harmonique \(\sum_{n\geq1}\dfrac{1}{n}\) n'est pas grossièrement divergente mais divergente.
\end{dem}