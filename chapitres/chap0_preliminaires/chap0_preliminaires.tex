\chapter{Préliminaires}

\minitoc

\section{Logique}

\begin{defi}
Une proposition est une affirmation qui peut être vraie ou fausse.

Exemples : \(1+1=2\) est une proposition vraie ; \(0>1\) est une proposition fausse.
\end{defi}

\begin{defi}
Soient \(P\) et \(Q\) deux propositions. On définit :

\begin{itemize}
\item la proposition \guillemets{\(P\) et \(Q\)} (notée aussi \(P\et Q\)) de table de vérité : \begin{tabular}{c|c||c}
\(P\) & \(Q\) & \(P\et Q\) \\
\hline
V & V & V \\
V & F & F \\
F & V & F \\
F & F & F
\end{tabular}

\item la proposition \guillemets{\(P\) ou \(Q\)} (notée aussi \(P\ou Q\)) de table de vérité : \begin{tabular}{c|c||c}
\(P\) & \(Q\) & \(P\ou Q\) \\
\hline
V & V & V \\
V & F & V \\
F & V & V \\
F & F & F
\end{tabular}

\item la proposition \guillemets{\(P\) implique \(Q\)} (notée aussi \(P\imp Q\)) de table de vérité : \begin{tabular}{c|c||c}
\(P\) & \(Q\) & \(P\imp Q\) \\
\hline
V & V & V \\
V & F & F \\
F & V & V \\
F & F & V
\end{tabular}

Exemple : Soit \(x\) un nombre réel. La proposition \(x\geq0\imp x^2\geq0\) est vraie. La proposition \(x=4\imp x=5\) est fausse si \(x=4\) et vraie sinon.

\item la proposition \guillemets{\(P\) équivaut à \(Q\)} (notée aussi \(P\ssi Q\)) de table de vérité \begin{tabular}{c|c||c}
\(P\) & \(Q\) & \(P\ssi Q\) \\
\hline
V & V & V \\
V & F & F \\
F & V & F \\
F & F & V
\end{tabular}

\item la proposition \guillemets{non \(P\)} (notée aussi \(\non P\)) de table de vérité \begin{tabular}{c||c}
\(P\) & \(\non P\) \\
\hline
V & F \\
F & V
\end{tabular}
\end{itemize}
\end{defi}

\begin{rem}[Contraposition]
Soient \(P\) et \(Q\) deux propositions. Les propositions \(P\imp Q\) et \(\non Q\imp\non P\) sont équivalentes.
\end{rem}

\begin{dem}[Méthode 1]
Il suffit de remarquer que les deux propositions ont la même table de vérité :

\begin{tabular}{c|c||c||c|c||c}
\(P\) & \(Q\) & \(P\imp Q\) & \(\non Q\) & \(\non P\) & \(\non Q\imp\non Q\) \\
\hline
V & V & V & F & F & V \\
V & F & F & V & F & F \\
F & V & V & F & V & V \\
F & F & V & V & V & V
\end{tabular}

~
\end{dem}

\begin{dem}[Méthode 2]
Montrons que \(\paren{P\imp Q}\ssi\paren{\non Q\imp\non P}\) :
\begin{itemize}
\item[\impdir] Supposons \(P\imp Q\). Montrons que \(\non Q\imp\non P\).

Supposons \(\non Q\).

Si \(P\) était vraie, \(Q\) serait vraie aussi. Donc \(P\) est fausse. Donc \(\non P\).

D'où \(\non Q\imp\non P\).

\item[\imprec] Supposons \(\non Q\imp\non P\).

D'après \impdir on a \(\non\paren{\non P}\imp\non\paren{\non Q}\), c'est à dire \(P\imp Q\).
\end{itemize}
\end{dem}

\begin{defi}
Soient \(P\) et \(Q\) deux propositions. Démontrer l'implication \(P\imp Q\) par contraposition, c'est démontrer l'implication contraposée \(\non Q\imp\non P\), c'est à dire supposer \(\non Q\) et montrer \(\non P\).
\end{defi}

\begin{defi}
CS : condition suffisante ; CN : condition nécessaire ; CNS : condition nécessaire et suffisante.
\end{defi}

\begin{ex}
CS pour avoir \(x\geq0\) : \(x=10\) ; CN pour avoir \(x\geq\) : \(x\geq-1\) ; CNS pour avoir \(x\geq0\) : \(x+1\geq1\).
\end{ex}

\begin{rem}
Ne pas utiliser les symboles \(\imp\) et \(\ssi\) comme des abréviations dans du texte en français.
\end{rem}

\begin{ex}
Soit \(x\in\Rp\). Montrons que \(x+1\geq1\).

Écrire \guillemets{On a \(x\geq0\ssi x+1\geq1\)} est faux.

Écrire \guillemets{On a \(x\geq0\) donc \(x+1\geq1\)} est correct.
\end{ex}

\begin{rem}
Abréviations autorisées : CS, CN, CNS et ssi.
\end{rem}

\section{Quantificateurs}

\begin{nota}
\(\in\) : \guillemets{appartient}

\(\exists\) : \guillemets{il existe ... tel que} (quantificateur existentiel)

\(\forall\) : \guillemets{pour tout} (quantificateur universel)

\(\exists!\) : \guillemets{il existe un unique ... tel que}
\end{nota}

\begin{ex}
\(\quantifs{\forall a,b\in\R}\paren{a+b}^2=a^2+2ab+b^2\)

\(\quantifs{\forall x\in\Rp;\exists y\in\R}x=y^2\)

\(\quantifs{\forall x\in\Rp;\exists!y\in\Rp}x=y^2\)
\end{ex}

\begin{rem}
Les quantificateurs se lisent de gauche à droite et leur ordre est important.
\end{rem}

\begin{ex}
On associe à tous réels \(x,y\) une proposition \(P\paren{x,y}\).

Alors \(\quantifs{\exists x\in\R;\forall y\in\R}P\paren{x,y}\imp\quantifs{\forall y\in\R;\exists x\in\R}P\paren{x,y}\).

Dans la proposition de gauche, le \(x\) est valable pour tout \(y\). Dans la proposition de droite, le \(x\) dépend du \(y\).

L'implication réciproque est généralement fausse.

Par exemple, la proposition \(\quantifs{\exists x\in\R;\forall y\in\R}y=x+1\) est fausse mais la proposition \(\quantifs{\forall y\in\R;\exists x\in\R}y=x+1\) est vraie.
\end{ex}

\begin{defi}[Produit cartésien d'ensembles]
Soient \(A,B\) deux ensembles. On note \(A\times B\) l'ensemble des couples de la forme \(\paren{x,y}\) où \(x\in A\) et \(y\in B\).

Plus généralement, soient \(A_1,\dots,A_n\) des ensembles avec \(n\in\Ns\). \(A_1\times\dots\times A_n\) est l'ensemble des n-uplets de la forme \(\paren{x_1,\dots,x_n}\) où \(x_1\in A_1,\dots,x_n\in A_n\).

Soient \(C\) un ensemble et \(m\in\Ns\). On pose \(C^m=\underbrace{C\times\dots\times C}_\text{\(m\) facteurs}\).

On s'autorise alors les identifications suivantes : \(A\times B\times C=A\times\paren{B\times C}=\paren{A\times B}\times C\) et \(\paren{x,y,z}=\paren{x,\paren{y,z}}=\paren{\paren{x,y},z}\).
\end{defi}

\begin{ex}
Dessiner les ensembles \(A=\intervii{1}{2}\times\Rp\), \(B=\N\times\R\) et \(C=\paren{\intervii{0}{1}\cup\accol{2}}\times\Z\).

On a \(\quantifs{\forall\paren{x,y}\in\R^2}\paren{x,y}\in A\ssi\begin{dcases}x\in\intervii{1}{2} \\ y\in\Rp\end{dcases}\) donc :

\begin{center}
\begin{tkz}
\draw[->,gray] (-1,0) -- (3,0);
\draw[->,gray] (0,-1) -- (0,3);
\node[below left] at (0,0) {\(0\)};
\node[below] at (1,0) {\(1\)};
\node[below] at (2,0) {\(2\)};
\draw[blue] (1,0) -- (1,3);
\draw[blue] (2,0) -- (2,3);
\fill[pattern=north east lines,pattern color=blue] (1,0) -- (1,3) -- (2,3) node[below right,blue] {\(A\)} -- (2,0);
\end{tkz}
\end{center}

De même, on a \(\quantifs{\forall\paren{x,y}\in\R^2}\paren{x,y}\in B\ssi\begin{dcases}x\in\N \\ y\in\R\end{dcases}\) donc :

\begin{center}
\begin{tkz}
\draw[->,gray] (-1,0) -- (3,0);
\draw[->,gray] (0,-3) -- (0,3);
\node[below left] at (0,0) {\(0\)};
\node[below left] at (1,0) {\(1\)};
\node[below left] at (2,0) {\(2\)};
\draw[blue] (0,-3) -- (0,3);
\draw[blue] (1,-3) -- (1,3) node[above right] {\(B\)};
\draw[blue] (2,-3) -- (2,3);
\draw[blue] (3,-3) -- (3,3);
\end{tkz}
\end{center}

Enfin, on a \(\quantifs{\forall\paren{x,y}\in\R^2}\paren{x,y}\in C\ssi\begin{dcases}x\in\intervii{0}{1}\cup\accol{2} \\ y\in\Z\end{dcases}\) donc :

\begin{center}
\begin{tkz}
\draw[->,gray] (-1,0) -- (3,0);
\draw[->,gray] (0,-3) -- (0,3);
\node[below left] at (0,0) {\(0\)};
\node[below] at (1,0) {\(1\)};
\node[below] at (2,0) {\(2\)};
\draw[blue] (0,3) -- (1,3);
\draw[blue] (0,2) -- (1,2);
\draw[blue] (0,1) -- (1,1);
\draw[blue] (0,0) -- (1,0);
\draw[blue] (0,-1) -- (1,-1);
\draw[blue] (0,-2) -- (1,-2);
\draw[blue] (0,-3) -- (1,-3);
\filldraw[blue] (2,3) circle (3pt) node[above left,blue] {\(C\)};
\filldraw[blue] (2,2) circle (3pt);
\filldraw[blue] (2,1) circle (3pt);
\filldraw[blue] (2,0) circle (3pt);
\filldraw[blue] (2,-1) circle (3pt);
\filldraw[blue] (2,-2) circle (3pt);
\filldraw[blue] (2,-3) circle (3pt);
\end{tkz}
\end{center}
\end{ex}

\begin{rem}
Les notations \guillemets{\(\forall x,y\in\R\)} et \guillemets{\(\forall\paren{x,y}\in\R^2\)} sont équivalentes.

En revanche, les produits cartésiens sont nécessaires pour le quantificateur \(\exists!\). En effet, on a \(\quantifs{\exists!\paren{x,y}\in\R^2}P\paren{x,y}\imp\quantifs{\exists! x\in\R;\exists! y\in\R}P\paren{x,y}\).
\end{rem}

\section{Raisonnements par analyse-synthèse}

Ils sont utiles pour trouver toutes les solutions à un problème.

\begin{ex}
Soit \(f:\R\to\R\). Montrons que \(f\) s'écrit de façon unique comme la somme d'une fonction paire et d'une fonction impaire. Autrement dit, en notant \(E_0\) l'ensemble des fonctions paires de \(\R\) dans \(\R\) et \(E_1\) l'ensemble des fonctions impaires de \(\R\) dans \(\R\), on a \(\quantifs{\exists!\paren{g,h}\in E_0\times E_1}f=g+h\).

\analyse

Soit \(\paren{g,h}\in E_0\times E_1\) tel que \(f=g+h\).

On a \(\quantifs{\forall x\in\R}\begin{dcases}f\paren{x}=g\paren{x}+h\paren{x} \\ f\paren{-x}=g\paren{x}-h\paren{x}\end{dcases}\)

Donc par somme et différence, \(\quantifs{\forall x\in\R}\begin{dcases}f\paren{x}+f\paren{-x}=2g\paren{x} \\ f\paren{x}-f\paren{-x}=2h\paren{x}\end{dcases}\)

Donc \(\quantifs{\forall x\in\R}\begin{dcases}g\paren{x}=\dfrac{f\paren{x}+f\paren{-x}}{2} \\ h\paren{x}=\dfrac{f\paren{x}-f\paren{-x}}{2}\end{dcases}\)

\synthese

On définit les fonctions \(\fonction{g}{\R}{\R}{x}{\dfrac{f\paren{x}+f\paren{-x}}{2}}\) et \(\fonction{h}{\R}{\R}{x}{\dfrac{f\paren{x}-f\paren{-x}}{2}}\).

On remarque \(\quantifs{\forall x\in\R}\begin{dcases}g\paren{-x}=\dfrac{f\paren{-x}+f\paren{-\paren{-x}}}{2}=\dfrac{f\paren{-x}+f\paren{x}}{2}=g\paren{x} \\ h\paren{-x}=\dfrac{f\paren{-x}-f\paren{-\paren{-x}}}{2}=\dfrac{f\paren{-x}-f\paren{x}}{2}=-h\paren{x}\end{dcases}\)

Donc \(g\) est paire et \(h\) est impaire.

Et \(\quantifs{\forall x\in\R}g\paren{x}+h\paren{x}=\dfrac{f\paren{x}+f\paren{-x}}{2}+\dfrac{f\paren{x}-f\paren{-x}}{2}=f\paren{x}\).

\conclusion

Le seul couple \(\paren{g,h}\in E_0\times E_1\) est \(\paren{\fonctionlambda{\R}{\R}{x}{\dfrac{f\paren{x}+f\paren{-x}}{2}},\fonctionlambda{\R}{\R}{x}{\dfrac{f\paren{x}-f\paren{-x}}{2}}}\).
\end{ex}

\section{Congruences}

\begin{defi}
Soient \(x,y\in\R\) et \(T\in\Rps\).

On dit que \(x\) et \(y\) sont congrus modulo \(T\) si on a \(\quantifs{\exists k\in\Z}x=y+kT\).

On note alors \(x\equiv y\croch{T}\).
\end{defi}

\begin{rem}
Soient \(x,y,z\in\R\) et \(T\in\Rps\). On a \begin{itemize}
\item \(x\equiv y\croch{T}\ssi y\equiv x\croch{T}\) : symétrie

\item \(x\equiv x\croch{T}\) : réflexivité

\item \(\begin{dcases}x\equiv y\croch{T} \\ y\equiv z\croch{T}\end{dcases}\imp x\equiv z\croch{T}\) : transitivité
\end{itemize}
\end{rem}