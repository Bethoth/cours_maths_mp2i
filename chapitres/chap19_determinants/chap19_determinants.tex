\chapter{Déterminants}

\minitoc

On considère un corps \(\K\) (en pratique, \(\K=\R\) ou \(\C\)).

\section{Multilinéarité}

\subsection{Formes multilinéaires}

\begin{defi}[Application multilinéaire]
Soient \(E_1,\dots,E_r,F\) des \(\K\)-espaces vectoriels où \(r\in\Ns\) et une fonction \[\fonction{f}{E_1\times\dots\times E_r}{F}{\paren{x_1,\dots,x_r}}{f\paren{x_1,\dots,x_r}}\]

On dit que \(f\) est une fonction multilinéaire (ou, plus précisément, est \(r\)-linéaire) si elle est linéaire par rapport à chacune de ses \(r\) variables, \cad si l'on a : \[\begin{aligned}
&\quantifs{\forall j\in\interventierii{1}{r};\forall\lambda,\mu\in\K;\forall\paren{x_1,\dots,x_r}\in E_1\times\dots\times E_r;\forall y_j\in E_j} \\
&f\paren{x_1,\dots,x_{j-1},\lambda x_j+\mu y_j,x_{j+1},\dots,x_r}=\lambda f\paren{x_1,\dots,x_{j-1},x_j,x_{j+1},\dots,x_r}+\mu f\paren{x_1,\dots,x_{j-1},y_j,x_{j+1},\dots,x_r}.
\end{aligned}\]

Si, de plus, \(F=\K\) alors on dit que \(f\) est une forme \(r\)-linéaire.
\end{defi}

\begin{rem}
Soient \(E_1,\dots,E_r,F\) des \(\K\)-espaces vectoriels où \(r\in\Ns\) et une fonction \(r\)-linéaire \[f:E_1\times\dots\times E_r\to F.\]

\begin{itemize}
    \item On a : \[\quantifs{\forall\paren{x_1,\dots,x_r}\in E_1\times\dots\times E_r}\paren{\quantifs{\exists i\in\interventierii{1}{r}}x_i=0_{E_i}}\imp f\paren{x_1,\dots,x_r}=0_F.\] En français : \guillemets{Si (au moins) l'un des \(x_i\) est nul alors \(f\paren{x_1,\dots,x_r}\) est nul.} \\
    \item On a : \[\quantifs{\forall\lambda\in\K;\forall\paren{x_1,\dots,x_r}\in E_1\times\dots\times E_r}f\paren{\lambda x_1,\dots,\lambda x_r}=\lambda^rf\paren{x_1,\dots,x_r}.\] En particulier, une application \(r\)-linéaire avec \(r\geq2\) n'est pas linéaire.
\end{itemize}
\end{rem}

\begin{ex}
Les applications suivantes sont multilinéaires :

\begin{itemize}
    \item Toute application linéaire est \(1\)-linéaire. \\
    \item Le produit matriciel et la composition des applications linéaires sont \(2\)-linéaires (on dit aussi bilinéaires). Par exemple, si \(n\in\Ns\) et \(E\) est un espace vectoriel : \[\fonctionlambda{\M{n}\times\M{n}}{\M{n}}{\paren{A,B}}{A\times B}\qquad\text{et}\qquad\fonctionlambda{\Lendo{E}\times\Lendo{E}}{\Lendo{E}}{\paren{u,v}}{u\rond v}\] sont bilinéaires. \\
    \item L'application \[\fonctionlambda{\M{n}^3}{\M{n}}{\paren{A,B,C}}{A\times B\times C}\] est \(3\)-linéaire (on dit aussi trilinéaire).
\end{itemize}
\end{ex}

\subsection{Formes multilinéaires alternées}

\begin{defi}[Forme multilinéaire alternée]
Soient \(r\in\Ns\), \(E\) un \(\K\)-espace vectoriel et une forme \(r\)-linéaire \[\fonction{f}{E^r}{\K}{\paren{x_1,\dots,x_r}}{f\paren{x_1,\dots,x_r}}\]

On dit que \(f\) est une forme multilinéaire alternée si l'on a : \[\quantifs{\forall\paren{x_1,\dots,x_r}\in E^r}\paren{\quantifs{\exists i,j\in\interventierii{1}{r}}i\not=j\quad\text{et}\quad x_i=x_j}\imp f\paren{x_1,\dots,x_r}=0.\]

En français : \guillemets{\(f\paren{x_1,\dots,x_r}\) est nul s'il existe (au moins) deux vecteurs égaux parmi \(x_1,\dots,x_r\).}
\end{defi}

\begin{prop}
Soient \(E\) un \(\K\)-espace vectoriel, \(r\in\interventierie{2}{\pinf}\), \(f:E^r\to\K\) une forme \(r\)-linéaire alternée et \(i,j\in\interventierii{1}{r}\) tels que \(i<j\).

On a : \[\quantifs{\forall\paren{x_1,\dots,x_r}\in E^r}f\paren{x_1,\dots,x_{i-1},x_j,x_{i+1},\dots,x_{j-1},x_i,x_{j+1},\dots,x_r}=-f\paren{x_1,\dots,x_r}.\]
\end{prop}

\begin{dem}
Soient \(x_1,\dots,x_r\in E\).

On pose \(\fonction{g}{E^2}{\K}{\paren{x,y}}{f\paren{x_1,\dots,x_{i-1},x,x_{i+1},\dots,x_{j-1},y,x_{j+1},\dots,x_r}}\)

Montrons que \(g\paren{x_j,x_i}=-g\paren{x_i,x_j}\).

Comme \(g\) est une forme bilinéaire alternée, on a : \[\underbrace{g\paren{x_i+x_j,x_i+x_j}}_{=0}=\underbrace{g\paren{x_i,x_i}}_{=0}+g\paren{x_i,x_j}+g\paren{x_j,x_i}+\underbrace{g\paren{x_j,x_j}}_{=0}.\]

D'où le résultat.
\end{dem}

\begin{cor}
Soient \(E\) un \(\K\)-espace vectoriel, \(r\in\Ns\) et \(f:E^r\to\K\) une forme \(r\)-linéaire alternée.

On a : \[\quantifs{\forall\sigma\in\S{n};\forall\paren{x_1,\dots,x_r}\in E^r}f\paren{x_{\sigma\paren{1}},\dots,x_{\sigma\paren{x_r}}}=\epsilon\paren{\sigma}f\paren{x_1,\dots,x_r}.\]
\end{cor}

\begin{prop}\thlabel{prop:l'ImageD'UneFamilleLiéeParUneFormeMultilinéaireAlternéeEstNulle}
Soient \(E\) un \(\K\)-espace vectoriel, \(r\in\Ns\), \(f:E^r\to\K\) une forme \(r\)-linéaire alternée et \(\paren{x_1,\dots,x_r}\in E^r\) une famille liée.

On a : \[f\paren{x_1,\dots,x_r}=0.\]
\end{prop}

\begin{dem}
Comme \(\paren{x_1,\dots,x_r}\) est liée, il existe \(i\in\interventierii{1}{r}\) tel que \(x_i\in\Vect{x_1,\dots,x_{i-1},x_{i+1},\dots,x_r}\).

Quitte à permuter les vecteurs \(x_1,\dots,x_r\), on peut supposer \(i=1\).

Soient \(\lambda_2,\dots,\lambda_r\in\K\) tels que \(x_1=\lambda_2x_2+\dots+\lambda_rx_r\).

On a : \[\begin{WithArrows}
f\paren{x_1,\dots,x_r}&=f\paren{\lambda_2x_2+\dots+\lambda_rx_r,x_2,\dots,x_r} \Arrow{car \(f\) est \(r\)-linéaire} \\
&=\sum_{k=2}^r\lambda_kf\paren{x_k,x_2,\dots,x_r} \Arrow{car \(f\) est alternée} \\
&=0.
\end{WithArrows}\]
\end{dem}

\section{Déterminant d'une famille de vecteurs dans une base}

Soit \(n\in\Ns\).

On s'intéresse désormais aux formes \(n\)-linéaires sur un espace vectoriel de dimension finie \(n\).

\begin{lem}[Calcul d'une forme \(n\)-linéaire en dimension \(n\)]\thlabel{lem:calculFormeN-LinéaireEnDimensionN}
Soient \(E\) un espace vectoriel de dimension \(n\), \(f:E^n\to\K\) une forme \(n\)-linéaire alternée, \(\fami{B}=\paren{e_1,\dots,e_n}\) une base de \(E\) et \(\paren{x_1,\dots,x_n}\in E^n\).

On considère la matrice de cette famille dans la base \(\fami{B}\) : \[\Mat{x_1,\dots,x_n}=\begin{pmatrix}
a_{11} & \dots & a_{1n} \\
\vdots &  & \vdots \\
a_{n1} & \dots & a_{nn}
\end{pmatrix}\in\M{n}.\]

On a : \[\begin{aligned}
f\paren{x_1,\dots,x_n}&=\sum_{\sigma\in\S{n}}\epsilon\paren{\sigma}a_{1\sigma\paren{1}}\dots a_{n\sigma\paren{n}}f\paren{e_1,\dots,e_n} \\
&=\sum_{\sigma\in\S{n}}\epsilon\paren{\sigma}a_{\sigma\paren{1}1}\dots a_{\sigma\paren{n}n}f\paren{e_1,\dots,e_n}.
\end{aligned}\]
\end{lem}

\begin{dem}
On a : \[\quantifs{\forall j\in\interventierii{1}{n}}x_j=\sum_{i=1}^na_{ij}e_i.\]

Donc, comme \(f\) est une forme \(n\)-linéaire, on a : \[\begin{WithArrows}
f\paren{x_1,\dots,x_n}&=f\paren{\sum_{i_1=1}^na_{i_11}e_{i_1},\dots,\sum_{i_n=1}^na_{i_nn}e_{i_n}} \\
&=\underbrace{\sum_{i_1=1}^n\dots\sum_{i_n=1}^n}_{n^n\text{ termes}}a_{i_11}\dots a_{i_nn}f\paren{e_{i_1},\dots,e_{i_n}} \\
&=\underbrace{\sum_{\alpha\in\F{\interventierii{1}{n}}{\interventierii{1}{n}}}}_{\star}a_{\alpha\paren{1}1}\dots a_{\alpha\paren{n}n}f\paren{e_{\alpha\paren{1}},\dots,e_{\alpha\paren{n}}} \\
&=\sum_{\sigma\in\S{n}}a_{\sigma\paren{1}1}\dots a_{\sigma\paren{n}n}f\paren{e_{\sigma\paren{1}},\dots,e_{\sigma\paren{n}}} \Arrow{car \(f\) est alternée} \\
&=\sum_{\sigma\in\S{n}}a_{\sigma\paren{1}1}\dots a_{\sigma\paren{n}n}\epsilon\paren{\sigma}f\paren{e_1,\dots,e_n}.
\end{WithArrows}\]

\(\star\) : \(n^n\) termes dont beaucoup sont nuls : tous ceux pour lesquels \(\alpha\) n'est pas injective et donc pas bijective, car \(f\) est alternée.
\end{dem}

\begin{deftheo}[Déterminant d'une famille de vecteurs dans une base]
Soient \(E\) un espace vectoriel de dimension \(n\) et \(\fami{B}=\paren{e_1,\dots,e_n}\) une base de \(E\).

Il existe une unique forme \(n\)-linéaire alternée \(\detb:E^n\to\K\) telle que \[\detb\paren{e_1,\dots,e_n}=1.\]

On l'appelle le déterminant dans la base \(\fami{B}\).
\end{deftheo}

\begin{dem}
\note{Existence non-exigible}

\analyse (unicité)

Soit \(f:E^n\to\K\) une forme \(n\)-linéaire alternée telle que \[\detb\paren{e_1,\dots,e_n}=1.\]

Selon le \thref{lem:calculFormeN-LinéaireEnDimensionN}, l'image de toute famille \(\paren{x_1,\dots,x_n}\in E^n\) de matrice \[\Mat{x_1,\dots,x_n}=\begin{pmatrix}
a_{11} & \dots & a_{1n} \\
\vdots &  & \vdots \\
a_{n1} & \dots & a_{nn}
\end{pmatrix}\] dans la base \(\fami{B}\) est : \[f\paren{x_1,\dots,x_n}=\sum_{\sigma\in\S{n}}\epsilon\paren{\sigma}a_{1\sigma\paren{1}}\dots a_{n\sigma\paren{n}}.\]

\synthese (existence)

Considérons la fonction \(f:E^n\to\K\) qui envoie toute famille \(\paren{x_1,\dots,x_n}\in E^n\) de matrice \[\Mat{x_1,\dots,x_n}=\begin{pmatrix}
a_{11} & \dots & a_{1n} \\
\vdots &  & \vdots \\
a_{n1} & \dots & a_{nn}
\end{pmatrix}\] dans la base \(\fami{B}\) sur : \[\sum_{\sigma\in\S{n}}\epsilon\paren{\sigma}a_{1\sigma\paren{1}}\dots a_{n\sigma\paren{n}}.\]

Alors \(f\) est clairement une forme \(n\)-linéaire sur \(E\).

Montrons qu'elle est alternée.

Soit une famille \(\paren{x_1,\dots,x_n}\in E^n\) de matrice dans la base \(\fami{B}\) : \[\Mat{x_1,\dots,x_n}=\begin{pmatrix}
a_{11} & \dots & a_{1n} \\
\vdots &  & \vdots \\
a_{n1} & \dots & a_{nn}
\end{pmatrix}.\]

On suppose qu'il existe \(k,l\in\interventierii{1}{n}\) tels que \(k<l\) et \(x_k=x_l\).

On a donc : \[\quantifs{\forall i\in\interventierii{1}{n}}a_{ik}=a_{il}.\]

Donc, en posant \(\tau=\cycle{k;l}\), on a : \[\quantifs{\forall\sigma\in\S{n}}a_{1\sigma\paren{1}}\dots a_{k\sigma\paren{k}}\dots a_{l\sigma\paren{l}}\dots a_{n\sigma\paren{n}}=a_{1\,\sigma\tau\paren{1}}\dots a_{k\,\sigma\tau\paren{k}}\dots a_{l\,\sigma\tau\paren{l}}\dots a_{n\,\sigma\tau\paren{n}}.\]

Or, le groupe symétrique s'écrit comme une réunion disjointe : \[\S{n}=\underbrace{\frakA{n}}_{\substack{\text{permutations} \\ \text{paires}}}\union\underbrace{\frakA{n}\tau}_{\substack{\text{permutations} \\ \text{impaires}}}.\]

D'où : \[\begin{aligned}
f\paren{x_1,\dots,x_n}&=\sum_{\sigma\in\frakA{n}}\epsilon\paren{\sigma}a_{1\sigma\paren{1}}\dots a_{n\sigma\paren{n}}+\sum_{\sigma\in\frakA{n}}\epsilon\paren{\sigma\tau}a_{1\,\sigma\tau\paren{1}}\dots a_{n\,\sigma\tau\paren{n}} \\
&=\sum_{\sigma\in\frakA{n}}a_{1\sigma\paren{1}}\dots a_{n\sigma\paren{n}}-\sum_{\sigma\in\frakA{n}}a_{1\sigma\paren{1}}\dots a_{n\sigma\paren{n}} \\
&=0.
\end{aligned}\]

Donc \(f\) est une forme \(n\)-linéaire alternée.

Enfin, il est clair que \(f\paren{\fami{B}}=1\), d'où l'existence.
\end{dem}

\begin{prop}\thlabel{prop:formuleDéterminantDansUneBase}
Soient \(E\) un espace vectoriel de dimension \(n\), \(\fami{B}=\paren{e_1,\dots,e_n}\) une base de \(E\) et \(\paren{x_1,\dots,x_n}\in E^n\).

On considère la matrice de cette famille dans la base \(\fami{B}\) : \[\Mat{x_1,\dots,x_n}=\begin{pmatrix}
a_{11} & \dots & a_{1n} \\
\vdots &  & \vdots \\
a_{n1} & \dots & a_{nn}
\end{pmatrix}.\]

On a : \[\detb\paren{x_1,\dots,x_n}=\sum_{\sigma\in\S{n}}\epsilon\paren{\sigma}a_{1\sigma\paren{1}}\dots a_{n\sigma\paren{n}}=\sum_{\sigma\in\S{n}}\epsilon\paren{\sigma}a_{\sigma\paren{1}1}\dots a_{\sigma\paren{n}n}.\]
\end{prop}

\begin{exoex}
Soient \(a,b,c,d,e,f,g,h,i\in\K\).

\begin{enumerate}
    \item On note \(\fami{B}_0\) la base canonique de \(\K^2\). Calculer : \[\detb[\fami{B}_0]\paren{\dcoords{a}{b},\dcoords{c}{d}}.\]
    \item On note \(\fami{B}_0\) la base canonique de \(\K^3\). Calculer : \[\detb[\fami{B}_0]\paren{\tcoords{a}{b}{c},\tcoords{d}{e}{f},\tcoords{g}{h}{i}}.\]
\end{enumerate}
\end{exoex}

\begin{corr}[1]
On a \(\S{2}=\accol{\id{};\cycle{1;2}}\) donc : \[\detb[\fami{B}_0]\paren{\dcoords{a}{b},\dcoords{c}{d}}=ad-bc.\]
\end{corr}

\begin{corr}[2]
On a \(\S{3}=\accol{\id{};\cycle{1;2};\cycle{1;3};\cycle{2;3};\cycle{1;2;3};\cycle{3;2;1}}\) donc : \[\detb[\fami{B}_0]\paren{\tcoords{a}{b}{c},\tcoords{d}{e}{f},\tcoords{g}{h}{i}}=aei-bdi-ceg-afh+bfg+cdh.\]
\end{corr}

\begin{prop}\thlabel{prop:formeMultilinéaireAlternéeProportionnelleAuDéterminantDansUneBase}
Soient \(E\) un espace vectoriel de dimension \(n\), \(\fami{B}\) une base de \(E\) et \(f:E^n\to\K\) une forme \(n\)-linéaire alternée.

On a : \[\quantifs{\exists\lambda\in\K}f=\lambda\detb.\]
\end{prop}

\begin{dem}
Découle du \thref{lem:calculFormeN-LinéaireEnDimensionN} et de la \thref{prop:formuleDéterminantDansUneBase}.
\end{dem}

\begin{rem}
Soient \(E\) un espace vectoriel de dimension \(n\) et \(\fami{B}\) une base de \(E\).

On pourrait montrer\footnote{C'est très facile mais le programme se limite à la proposition précédente.} que l'ensemble des formes \(n\)-linéaires alternées sur \(E\) est un espace vectoriel, de base \(\paren{\detb}\).

En particulier, cet espace vectoriel est de dimension \(1\).
\end{rem}

\begin{prop}[Changement de base]
Soient \(E\) un espace vectoriel de dimension \(n\) et \(\fami{B}\) et \(\fami{B}\prim\) deux bases de \(E\).

On a : \[\detb[\fami{B}\prim]=\detb[\fami{B}\prim]\paren{\fami{B}}\detb,\] \cad : \[\quantifs{\forall x_1,\dots,x_n\in E}\detb[\fami{B}\prim]\paren{x_1,\dots,x_n}=\detb[\fami{B}\prim]\paren{\fami{B}}\detb\paren{x_1,\dots,x_n}.\]
\end{prop}

\begin{dem}
Comme \(\detb[\fami{B}\prim]\) est une forme \(n\)-linéaire alternée sur \(E\), selon la \thref{prop:formeMultilinéaireAlternéeProportionnelleAuDéterminantDansUneBase}, il existe \(\lambda\in\K\) tel que \(\detb[\fami{B}\prim]=\lambda\detb\).

D'où, en appliquant à \(\fami{B}\) : \[\detb[\fami{B}\prim]\paren{\fami{B}}=\lambda\detb\paren{\fami{B}}=\lambda.\]

D'où : \[\detb[\fami{B}\prim]=\detb[\fami{B}\prim]\paren{\fami{B}}\detb.\]
\end{dem}

\begin{ex}
Si \(\fami{B}=\paren{e_1,\dots,e_n}\) et \(\fami{B}\prim=\paren{2e_1,\dots,2e_n}\) alors : \[\detb[\fami{B}\prim]\fami{B}=\detb[\fami{B}\prim]\paren{e_1,\dots,e_n}=\dfrac{1}{2^n}\detb[\fami{B}\prim]\paren{2e_1,\dots,2e_n}=\dfrac{1}{2^n}\] et : \[\detb[\fami{B}\prim]=\dfrac{1}{2^n}\detb.\]
\end{ex}

\begin{prop}[Caractérisation des bases]
Soient \(E\) un espace vectoriel de dimension \(n\), \(\fami{B}\) une base de \(E\) et \(x_1,\dots,x_n\in E\).

On a : \[\paren{x_1,\dots,x_n}\text{ est une base de }E\ssi\detb\paren{x_1,\dots,x_n}\not=0.\]
\end{prop}

\begin{dem}
\imprec Par contraposée :

Supposons que \(\paren{x_1,\dots,x_n}\) n'est pas une base de \(E\).

Comme \(\dim E=n\) et \(\paren{x_1,\dots,x_n}\) possède \(n\) vecteurs, \(\paren{x_1,\dots,x_n}\) est liée.

Donc \(\detb\paren{x_1,\dots,x_n}=0\) selon la \thref{prop:l'ImageD'UneFamilleLiéeParUneFormeMultilinéaireAlternéeEstNulle}.

\impdir

Supposons que \(\fami{B}\prim=\paren{x_1,\dots,x_n}\) est une base de \(E\).

On a \(\detb=\detb\paren{\fami{B}\prim}\detb[\fami{B}\prim]\) donc en appliquant à \(\fami{B}\) : \[1=\detb\paren{\fami{B}\prim}\times\detb[\fami{B}\prim]\paren{\fami{B}}.\]

Donc \(\detb\paren{\fami{B}\prim}\not=0\).
\end{dem}

\section{Déterminant d'un endomorphisme}

Soit \(n\in\Ns\).

\subsection{Définition}

\begin{deftheo}[Déterminant d'un endomorphisme]
Soient \(E\) un espace vectoriel de dimension \(n\) et \(u\in\Lendo{E}\).

Il existe un unique scalaire \(\lambda\in\K\) tel que : \[\quantifs{\forall\fami{B}\text{ base de }E;\forall x_1,\dots,x_n\in E}\detb\paren{u\paren{x_1},\dots,u\paren{x_n}}=\lambda\detb\paren{x_1,\dots,x_n}.\]

Ce scalaire \(\lambda\) est appelé déterminant de \(u\) et est noté \(\det u\).
\end{deftheo}

\begin{dem}
Soit \(\fami{B}\) une base de \(E\).

On pose \(\fonction{f}{E^n}{\K}{\paren{x_1,\dots,x_n}}{\detb\paren{u\paren{x_1},\dots,u\paren{x_n}}}\) une forme \(n\)-linéaire alternée.

Selon la \thref{prop:formeMultilinéaireAlternéeProportionnelleAuDéterminantDansUneBase}, il existe \(\lambda\in\K\) tel que : \[\quantifs{\forall x_1,\dots,x_n\in E}f\paren{x_1,\dots,x_n}=\lambda\detb\paren{x_1,\dots,x_n}.\]

Soit \(\fami{B}\prim\) une autre base de \(E\).

On a : \[\quantifs{\forall x_1,\dots,x_n\in E}\detb\paren{u\paren{x_1},\dots,u\paren{x_n}}=\lambda\detb\paren{x_1,\dots,x_n}.\]

Donc : \[\quantifs{\forall x_1,\dots,x_n\in E}\detb[\fami{B}\prim]\paren{\fami{B}}\detb\paren{u\paren{x_1},\dots,u\paren{x_n}}=\lambda\detb[\fami{B}\prim]\paren{\fami{B}}\detb\paren{x_1,\dots,x_n}.\]

Donc : \[\quantifs{\forall x_1,\dots,x_n\in E}\detb[\fami{B}\prim]\paren{u\paren{x_1},\dots,u\paren{x_n}}=\lambda\detb[\fami{B}\prim]\paren{x_1,\dots,x_n}.\]

Donc \(\lambda\) est indépendant du choix de la base.
\end{dem}

\begin{rem}\thlabel{rem:déterminantD'UnEndomorphismeVautLeDéterminantDeL'ImageD'uneBaseParCetEndomorphismeDansCetteBase}
Soient \(E\) un espace vectoriel de dimension \(n\), \(u\in\Lendo{E}\) et \(\fami{B}=\paren{e_1,\dots,e_n}\) une base de \(E\).

On a : \[\det u=\detb\paren{u\paren{e_1},\dots,u\paren{e_n}}.\]
\end{rem}

\begin{dem}
On a, par définition : \[\detb\paren{u\paren{e_1},\dots,u\paren{e_n}}=\det u\times\detb\paren{e_1,\dots,e_n}=\det u.\]
\end{dem}

\begin{ex}[Déterminant d'une homothétie]
Soit \(E\) un espace vectoriel de dimension \(n\).

On a : \[\quantifs{\forall\lambda\in\K}\det\paren{\lambda\id{E}}=\lambda^n.\]
\end{ex}

\begin{dem}
On applique la \thref{rem:déterminantD'UnEndomorphismeVautLeDéterminantDeL'ImageD'uneBaseParCetEndomorphismeDansCetteBase}.

Soit \(\fami{B}=\paren{e_1,\dots,e_n}\) une base de \(E\).

On a : \[\begin{aligned}
\det\paren{\lambda\id{E}}&=\detb\paren{\lambda\id{E}\paren{e_1},\dots,\lambda\id{E}\paren{e_n}} \\
&=\detb\paren{\lambda e_1,\dots,\lambda e_n} \\
&=\lambda^n\detb\paren{e_1,\dots,e_n} \\
&=\lambda^n\detb\paren{\fami{B}} \\
&=\lambda^n.
\end{aligned}\]
\end{dem}