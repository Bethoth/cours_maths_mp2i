\chapter{Déterminants}

\minitoc

On considère un corps \(\K\) (en pratique, \(\K=\R\) ou \(\C\)).

\section{Multilinéarité}

\subsection{Formes multilinéaires}

\begin{defi}[Application multilinéaire]
Soient \(E_1,\dots,E_r,F\) des \(\K\)-espaces vectoriels où \(r\in\Ns\) et une fonction \[\fonction{f}{E_1\times\dots\times E_r}{F}{\paren{x_1,\dots,x_r}}{f\paren{x_1,\dots,x_r}}\]

On dit que \(f\) est une fonction multilinéaire (ou, plus précisément, est \(r\)-linéaire) si elle est linéaire par rapport à chacune de ses \(r\) variables, \cad si l'on a : \[\begin{aligned}
&\quantifs{\forall j\in\interventierii{1}{r};\forall\lambda,\mu\in\K;\forall\paren{x_1,\dots,x_r}\in E_1\times\dots\times E_r;\forall y_j\in E_j} \\
&f\paren{x_1,\dots,x_{j-1},\lambda x_j+\mu y_j,x_{j+1},\dots,x_r}=\lambda f\paren{x_1,\dots,x_{j-1},x_j,x_{j+1},\dots,x_r}+\mu f\paren{x_1,\dots,x_{j-1},y_j,x_{j+1},\dots,x_r}.
\end{aligned}\]

Si, de plus, \(F=\K\) alors on dit que \(f\) est une forme \(r\)-linéaire.
\end{defi}

\begin{rem}
Soient \(E_1,\dots,E_r,F\) des \(\K\)-espaces vectoriels où \(r\in\Ns\) et une fonction \(r\)-linéaire \[f:E_1\times\dots\times E_r\to F.\]

\begin{itemize}
    \item On a : \[\quantifs{\forall\paren{x_1,\dots,x_r}\in E_1\times\dots\times E_r}\paren{\quantifs{\exists i\in\interventierii{1}{r}}x_i=0_{E_i}}\imp f\paren{x_1,\dots,x_r}=0_F.\] En français : \guillemets{Si (au moins) l'un des \(x_i\) est nul alors \(f\paren{x_1,\dots,x_r}\) est nul.} \\
    \item On a : \[\quantifs{\forall\lambda\in\K;\forall\paren{x_1,\dots,x_r}\in E_1\times\dots\times E_r}f\paren{\lambda x_1,\dots,\lambda x_r}=\lambda^rf\paren{x_1,\dots,x_r}.\] En particulier, une application \(r\)-linéaire avec \(r\geq2\) n'est pas linéaire.
\end{itemize}
\end{rem}

\begin{ex}
Les applications suivantes sont multilinéaires :

\begin{itemize}
    \item Toute application linéaire est \(1\)-linéaire. \\
    \item Le produit matriciel et la composition des applications linéaires sont \(2\)-linéaires (on dit aussi bilinéaires). Par exemple, si \(n\in\Ns\) et \(E\) est un espace vectoriel : \[\fonctionlambda{\M{n}\times\M{n}}{\M{n}}{\paren{A,B}}{A\times B}\qquad\text{et}\qquad\fonctionlambda{\Lendo{E}\times\Lendo{E}}{\Lendo{E}}{\paren{u,v}}{u\rond v}\] sont bilinéaires. \\
    \item L'application \[\fonctionlambda{\M{n}^3}{\M{n}}{\paren{A,B,C}}{A\times B\times C}\] est \(3\)-linéaire (on dit aussi trilinéaire).
\end{itemize}
\end{ex}

\subsection{Formes multilinéaires alternées}

\begin{defi}[Forme multilinéaire alternée]
Soient \(r\in\Ns\), \(E\) un \(\K\)-espace vectoriel et une forme \(r\)-linéaire \[\fonction{f}{E^r}{\K}{\paren{x_1,\dots,x_r}}{f\paren{x_1,\dots,x_r}}\]

On dit que \(f\) est une forme multilinéaire alternée si l'on a : \[\quantifs{\forall\paren{x_1,\dots,x_r}\in E^r}\paren{\quantifs{\exists i,j\in\interventierii{1}{r}}i\not=j\quad\text{et}\quad x_i=x_j}\imp f\paren{x_1,\dots,x_r}=0.\]

En français : \guillemets{\(f\paren{x_1,\dots,x_r}\) est nul s'il existe (au moins) deux vecteurs égaux parmi \(x_1,\dots,x_r\).}
\end{defi}

\begin{prop}
Soient \(E\) un \(\K\)-espace vectoriel, \(r\in\interventierie{2}{\pinf}\), \(f:E^r\to\K\) une forme \(r\)-linéaire alternée et \(i,j\in\interventierii{1}{r}\) tels que \(i<j\).

On a : \[\quantifs{\forall\paren{x_1,\dots,x_r}\in E^r}f\paren{x_1,\dots,x_{i-1},x_j,x_{i+1},\dots,x_{j-1},x_i,x_{j+1},\dots,x_r}=-f\paren{x_1,\dots,x_r}.\]
\end{prop}

\begin{dem}
Soient \(x_1,\dots,x_r\in E\).

On pose \(\fonction{g}{E^2}{\K}{\paren{x,y}}{f\paren{x_1,\dots,x_{i-1},x,x_{i+1},\dots,x_{j-1},y,x_{j+1},\dots,x_r}}\)

Montrons que \(g\paren{x_j,x_i}=-g\paren{x_i,x_j}\).

Comme \(g\) est une forme bilinéaire alternée, on a : \[\underbrace{g\paren{x_i+x_j,x_i+x_j}}_{=0}=\underbrace{g\paren{x_i,x_i}}_{=0}+g\paren{x_i,x_j}+g\paren{x_j,x_i}+\underbrace{g\paren{x_j,x_j}}_{=0}.\]

D'où le résultat.
\end{dem}

\begin{cor}
Soient \(E\) un \(\K\)-espace vectoriel, \(r\in\Ns\) et \(f:E^r\to\K\) une forme \(r\)-linéaire alternée.

On a : \[\quantifs{\forall\sigma\in\S{n};\forall\paren{x_1,\dots,x_r}\in E^r}f\paren{x_{\sigma\paren{1}},\dots,x_{\sigma\paren{x_r}}}=\epsilon\paren{\sigma}f\paren{x_1,\dots,x_r}.\]
\end{cor}

\begin{prop}
Soient \(E\) un \(\K\)-espace vectoriel, \(r\in\Ns\), \(f:E^r\to\K\) une forme \(r\)-linéaire alternée et \(\paren{x_1,\dots,x_r}\in E^r\) une famille liée.

On a : \[f\paren{x_1,\dots,x_r}=0.\]
\end{prop}

\begin{dem}
Comme \(\paren{x_1,\dots,x_r}\) est liée, il existe \(i\in\interventierii{1}{r}\) tel que \(x_i\in\Vect{x_1,\dots,x_{i-1},x_{i+1},\dots,x_r}\).

Quitte à permuter les vecteurs \(x_1,\dots,x_r\), on peut supposer \(i=1\).

Soient \(\lambda_2,\dots,\lambda_r\in\K\) tels que \(x_1=\lambda_2x_2+\dots+\lambda_rx_r\).

On a : \[\begin{WithArrows}
f\paren{x_1,\dots,x_r}&=f\paren{\lambda_2x_2+\dots+\lambda_rx_r,x_2,\dots,x_r} \Arrow{car \(f\) est \(r\)-linéaire} \\
&=\sum_{k=2}^r\lambda_kf\paren{x_k,x_2,\dots,x_r} \Arrow{car \(f\) est alternée} \\
&=0.
\end{WithArrows}\]
\end{dem}