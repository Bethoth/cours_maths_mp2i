\chapter{Algèbre générale}

\minitoc

\section{Lois de composition internes}

\subsection{Définition}

\begin{defi}[Loi de composition interne]
Soit \(E\) un ensemble.

On appelle loi de composition interne sur \(E\) toute application de \(E\times E\) dans \(E\).
\end{defi}

\begin{rem}
Les lois de composition internes sont en général notées avec des symboles tels que : \[+\quad\times\quad\cdot\quad\land\quad\rond\quad*\quad\oplus\quad\otimes.\]

Si \(*\) est une loi de composition interne sur un ensemble \(E\) et \(x,y\) sont deux éléments de \(E\), on préfère noter \(x*y\) plutôt que \(*\paren{x,y}\) l'image du couple \(\paren{x,y}\) par \(*\).
\end{rem}

\begin{ex}\thlabel{ex:loisDeCompositionInternes}
\begin{enumerate}
\item L'addition et la multiplication sont des lois de composition internes sur \(\N\), sur \(\Z\), sur \(\Q\), sur \(\R\), sur \(\C\), sur \(\R^\N\), sur \(\C^\N\)... \\

\item Si \(E\) est un ensemble, alors : \begin{itemize}
\item L'intersection est une loi de composition interne sur \(\P{E}\). \\

\item La réunion est une loi de composition interne sur \(\P{E}\). \\

\item La composition est une loi de composition interne sur \(\F{E}{E}\). \\
\end{itemize}

\item Le produit vectoriel est une loi de composition interne sur l'ensemble des vecteurs d'un espace euclidien orienté de dimension 3. \\
\end{enumerate}
\end{ex}

\subsection{Associativité, commutativité}

\begin{defi}
Soit \(E\) un ensemble et \(*\) une loi de composition interne sur \(E\).

On dit que \(*\) est associative si on a : \[\quantifs{\forall x,y,z\in E}x*\paren{y*z}=\paren{x*y}*z.\]
\end{defi}

\begin{ex}
Parmi les lois de composition internes données à l'\thref{ex:loisDeCompositionInternes}, toutes sont associatives, excepté le produit vectoriel.

Par exemple, on a bien \[\quantifs{\forall x,y,z\in\N}\paren{x+y}+z=x+\paren{y+z}.\]

En revanche, si l'on considère une base orthonormée directe \(\paren{i,j,k}\), on a : \[\begin{dcases}\paren{i\vecto j}\vecto j=k\vecto j=-i \\ i\vecto\paren{j\vecto j}=i\vecto0=0\end{dcases}\]
\end{ex}

\begin{defi}
Soient \(E\) un ensemble et \(*\) une loi de composition interne sur \(E\).

On dit que \(*\) est commutative si on a : \[\quantifs{\forall x,y\in E}x*y=y*x.\]
\end{defi}

\begin{ex}
Parmi les lois de composition internes données à l'\thref{ex:loisDeCompositionInternes}, toutes sont commutatives, exceptés le produit vectoriel et la composition.
\end{ex}

\subsection{Élément neutre, inverse}

\begin{defi}[Élément neutre]
Soient \(E\) un ensemble, \(*\) une loi de composition interne sur \(E\) et \(e\) un élément de \(E\).

On dit que \(e\) est un élément neutre pour \(*\) si on a : \[\quantifs{\forall x\in E}e*x=x*e=x.\]

On dit alors que la loi \(*\) admet un élément neutre (ou, par abus, que \(E\) admet un élément neutre).
\end{defi}

\begin{ex}
Parmi les lois de composition internes données à l'\thref{ex:loisDeCompositionInternes}, toutes admettent un élément neutre, excepté le produit vectoriel :

\begin{enumerate}
\item \begin{itemize}
\item La loi \(+\) admet \(0\) (ou \(\paren{0}_n\)) comme neutre. \\

\item La loi \(\times\) admet \(1\) (ou \(\paren{1}_n\)) comme neutre. \\
\end{itemize}

\item \begin{itemize}
\item La loi \(\inter\) admet \(E\) comme neutre. \\

\item La loi \(\union\) admet \(\ensvide\) comme neutre. \\

\item La loi \(\rond\) admet \(\id{E}\) comme neutre. \\
\end{itemize}

\item La loi \(\vecto\) n'admet pas de neutre. Par l'absurde, soit \(v\) un élément neutre. On a \(\quantifs{\forall w\text{ vecteur}}w\vecto v=v\vecto w=w\). D'où, en prenant \(w=v\), \(0=v\). D'où \(\quantifs{\forall w\text{ vecteur}}w=w\vecto0=0\) : contradiction. \\
\end{enumerate}
\end{ex}

\begin{rem}
Soient \(E\) un ensemble, \(*\) une loi de composition interne sur \(E\) et \(e\) un élément de \(E\).

On dit que \(e\) est un élément neutre à droite pour \(*\) si on a : \[\quantifs{\forall x\in E}x*e=x.\]

Il n'est généralement pas suffisant que \(e\) soit un élément neutre à droite pour que \(e\) soit un élément neutre (cela suffit si \(*\) est commutative), comme le montre l'exemple suivant :

\begin{center}
\begin{tikzpicture}
\matrix (mymatrix) [matrix of nodes, nodes={draw, minimum size=6mm, outer sep=0pt}, column sep=-\pgflinewidth, row sep=-\pgflinewidth]
    {  & 0 & 1\\
     0 & 0 & 1\\
     1 & 0 & 0\\};
\draw[->, shorten <=1mm, shorten >=1mm, looseness=1.2]
    (mymatrix-2-1.north west)to[out=90, in=180]node[below right=-3pt]{\(\oplus\)}(mymatrix-1-2.north west);
\end{tikzpicture}
\end{center}

\(\oplus\) est bien une loi de composition interne sur \(\accol{0;1}\) et \(0\) est neutre à gauche mais pas à droite.
\end{rem}

\begin{prop}[Unicité de l'élément neutre]
Soient \(E\) un ensemble et \(*\) une loi de composition interne sur \(E\).

Alors il existe au plus un élément neutre pour \(*\).
\end{prop}

\begin{dem}
Soient \(e,e\prim\in E\) deux éléments neutres pour \(*\). On a : \[\quantifs{\forall x\in E}x*e=e*x=x=e\prim*x=x*e\prim.\]

Montrons que \(e=e\prim\).

On a : \[\begin{aligned}
e&=e*e\prim&\text{car }e\prim\text{ neutre} \\
&=e\prim&\text{car }e\text{ neutre}
\end{aligned}\]
\end{dem}

\begin{defprop}[Inverse d'un élément]
Soient \(E\) un ensemble et \(*\) une loi de composition interne sur \(E\).

On suppose que la loi \(*\) est associative et qu'elle admet un élément neutre noté \(e\).

Soit \(x\) un élément de \(E\).

On appelle inverse de \(x\) tout élément \(y\in E\) tel que : \[y*x=x*y=e.\]

Il existe au plus un tel élément \(y\).

S'il existe, on l'appelle donc l'inverse de \(x\) et on le note \(x\inv\).

On dit alors que \(x\) est inversible.
\end{defprop}

\begin{dem}
Soient \(y_1,y_2\in E\) tels que \(\begin{dcases}y_1*x=x*y_1=e \\ y_2*x=x*y_2=e\end{dcases}\)

Montrons que \(y_1=y_2\).

On a \[\begin{aligned}
\paren{y_1*x}*y_2&=y_1*\paren{x*y_2} \\
e*y_2&=y_1*e \\
y_2&=y_1.
\end{aligned}\]
\end{dem}

\begin{ex}\thlabel{ex:inverseElementNeutreEgalElementNeutre}
Soient \(E\) un ensemble et \(*\) une loi de composition interne associative sur \(E\) et admettant un élément neutre \(e\).

Alors \(e\) est inversible, d'inverse lui-même : \[e\inv=e.\]
\end{ex}

\begin{dem}
On a \(e*e=e*e=e\). Donc \(e\) inversible, d'inverse \(e\).
\end{dem}

\begin{ex}\thlabel{ex:inverseInverseEgalElement}
Soient \(E\) un ensemble et \(*\) une loi de composition interne associative sur \(E\) admettant un élément neutre \(e\).

Soit \(a\) un élément inversible de \(E\).

Alors l'inverse \(a\inv\) de \(a\) est inversible, d'inverse \(a\) : \[\paren{a\inv}\inv=a.\]
\end{ex}

\begin{dem}
On a \(a*a\inv=a\inv*a=e\). Donc \(a\inv\) est inversible, d'inverse \(a\).
\end{dem}

\begin{ex}
Étudions les éléments inversibles des lois de composition internes associatives de l'\thref{ex:loisDeCompositionInternes} qui admettent un élément neutre :

\begin{itemize}
\item Pour \(\paren{\N,+}\), le neutre est \(0\) et il est le seul inversible. \\

\item Pour \(\paren{\N,\times}\), le neutre est \(1\) et il est le seul inversible. \\

\item Pour \(\paren{\Z,+}\), le neutre est \(0\) et tout élément admet un opposé. \\

\item Pour \(\paren{\Z,\times}\), le neutre est \(1\) et \(1\) et \(-1\) sont les seuls éléments inversibles. \\

\item Tout élément admet un opposé dans \(\paren{\Q,+}\), \(\paren{\R,+}\), \(\paren{\C,+}\), \(\paren{\R^\N,+}\) et \(\paren{\C^\N,+}\). \\

\item \(0\) n'est pas inversible par \(\times\) dans \(\Q\), \(\R\), \(\C\), \(\R^\N\) et \(\C^\N\). \\

\item Pour \(\paren{\P{E},\inter}\), le neutre est \(E\). Soit \(A\in\P{E}\). On a \[\begin{aligned}
A\text{ inversible par }\inter&\ssi\quantifs{\exists B\in\P{E}}A\inter B=B\inter A=E \\
&\ssi A=E
\end{aligned}\] donc \(E\) est le seul inversible. \\

\item Pour \(\paren{\P{E},\union}\), de même, \(\ensvide\) est le seul inversible. \\

\item Pour \(\paren{\F{E}{E},\rond}\), le neutre est \(\id{E}\). Soit \(f\in\F{E}{E}\). On a \[\begin{aligned}
f\text{ inversible par }\rond&\ssi\quantifs{\exists g\in\F{E}{E}}f\rond g=g\rond f=\id{E} \\
&\ssi f\text{ est bijective}
\end{aligned}\] donc l'inverse de \(f\) est sa bijection réciproque \(f\inv\). \\
\end{itemize}
\end{ex}

\begin{prop}\thlabel{prop:inverseProduitEgalProduitInversesInverse}
Soient \(E\) un ensemble et \(*\) une loi de composition interne associative sur \(E\) admettant un élément neutre \(e\).

Si \(x\) et \(y\) sont inversibles, alors \(x*y\) est aussi inversible, d'inverse : \[\paren{x*y}\inv=y\inv*x\inv.\]
\end{prop}

\begin{dem}
On a : \[\begin{aligned}
\paren{x*y}*\paren{y\inv*x\inv}&=x*\paren{y*y\inv}*x\inv \\
&=x*e*x\inv \\
&=x*x\inv \\
&=e
\end{aligned}\]

De même : \[\begin{aligned}
y\inv*x\inv*x*y&=y\inv*y \\
&=e
\end{aligned}\]

Donc \(x*y\) est inversible, d'inverse \(y\inv*x\inv\).
\end{dem}

\begin{prop}[Inversible \(\imp\) régulier]
Soit \(E\) un ensemble muni d'une loi de composition interne associative \(*\) et admettant un élément neutre \(e\).

Soient \(x,y,z\in E\). On suppose que \(x\) est inversible.

Alors les conditions suivantes sont équivalentes :

\begin{enumerate}
\item \(y=z\) \\

\item \(xy=xz\) \\

\item \(yx=zx\)
\end{enumerate}

On dit que \(x\) est un élément régulier ou simplifiable.
\end{prop}

\begin{dem}
(1) \(\imp\) (2) et (1) \(\imp\) (3) : Clair.

(2) \(\imp\) (1) : on multiplie à gauche par \(x\inv\).

(3) \(\imp\) (1) : on multiplie à droite par \(x\inv\).
\end{dem}

\begin{nota}
Lorsqu'une loi de composition interne est notée \(*\), \(\times\), \(\cdot\), \(\otimes\) ou \(\rond\), on dit qu'on utilise une notation multiplicative.

Au contraire, les notations telles que \(+\) ou \(\oplus\) sont dites additives.

L'usage est de n'employer des notations additives que pour des lois de composition commutatives.

L'usage est également :

\begin{itemize}
\item de réserver les notations \(1\) ou \(1_E\) pour l'élément neutre aux lois notées multiplicativement. Dans le cas d'une loi notée additivement, on préfère noter l'élément neutre \(0\) ou \(0_E\). \\

\item de réserver la notation \(x\inv\) pour l'inverse d'un élément \(x\) aux lois notées multiplicativement. Dans le cas d'une loi notée additivement, on préfère noter \(-x\) l'inverse de \(x\) et on l'appelle alors plutôt l'opposé de \(x\).
\end{itemize}

Soit \(E\) un ensemble muni d'une loi de composition interne associative notée multiplicativement, par exemple \(*\). Soient \(x\in E\) et \(n\in\Ns\).

Le produit \(x*\dots*x\) de \(n\) facteurs tous égaux à \(x\) est noté \(x^n\).

Si, de plus, \(*\) admet un élément neutre \(1_E\), on pose \(x^0=1_E\).

Enfin, si \(x\) est inversible, on pose \(x^{-n}=\paren{x\inv}^n\).

On fait de même pour les lois notées additivement, mais en utilisant encore des notations différentes :

Soit \(E\) un ensemble muni d'une loi de composition interne associative notée additivement, par exemple \(+\). Soient \(x\in E\) et \(n\in\Ns\).

La somme \(x+\dots+x\) de \(n\) termes tous égaux à \(x\) est notée \(nx\) (ou \(n\cdot x\)).

Si, de plus, \(+\) admet un élément neutre \(0_E\), on pose \(0x=0_E\).

Enfin, si \(x\) est \guillemets{inversible}, on pose \(\paren{-n}x=n\paren{-x}\).

Enfin, on utilisera pour une loi additive la notation \(\sum\) pour une somme et pour une loi multiplicative la notation \(\prod\) pour un produit.

Toutes ces différences entre lois notées multiplicativement et lois notées additivement ne reposent que dans la manière de noter les objets. Il n'y a aucune différence dans les concepts.
\end{nota}

\begin{rem}
Soit \(E\) un ensemble muni d'une loi de composition interne associative \(*\).

Soient \(x,y\in E\) et \(n\in\N\).

Attention à ne pas écrire en général \(\paren{xy}^n=x^ny^n\).

Cette formule n'est vraie que si \(x\) et \(y\) commutent.
\end{rem}

\subsection{Distributivité}

\begin{defi}[Distributivité]
Soit \(E\) un ensemble muni de deux lois de composition internes \(\oplus\) et \(\otimes\).

On dit que la loi \(\otimes\) est distributive par rapport à la loi \(\oplus\) si on a : \[\quantifs{\forall x,y,z\in E}\begin{dcases}x\otimes\paren{y\oplus z}=x\otimes y\oplus x\otimes z \\ \paren{y\oplus z}\otimes x=y\otimes x\oplus z\otimes x\end{dcases}\]
\end{defi}

\begin{rem}
Soit \(E\) un ensemble muni de deux lois de composition internes \(\oplus\) et \(\otimes\).

Si la loi \(\otimes\) est commutative, alors : \[\otimes\text{ est distributive par rapport à }\oplus\ssi\quantifs{\forall x,y,z\in E}x\otimes\paren{y\oplus z}=x\otimes y\oplus x\otimes z.\]
\end{rem}

\begin{dem}
\impdir Claire.

\imprec

Soient \(x,y,z\in E\).

On a \[\begin{aligned}
\paren{y\oplus z}\otimes x&=x\otimes\paren{y\oplus z} \\
&=x\otimes y\oplus x\otimes z \\
&=y\otimes x\oplus z\otimes x
\end{aligned}\]
\end{dem}

\begin{ex}
\begin{itemize}
\item Dans \(\C\), \(\times\) est distributif par rapport à \(+\). \\

\item Dans \(\P{E}\) : \(\quantifs{\forall A,B,C\in\P{E}}\begin{dcases}A\union\paren{B\inter C}=\paren{A\union B}\inter\paren{A\union C} \\ A\inter\paren{B\union C}=\paren{A\inter B}\union\paren{A\inter C}\end{dcases}\)

Donc \(\inter\) est distributive par rapport à \(\union\) et \(\union\) est distributive par rapport à \(\inter\). \\

\item Dans \(\N\), \(+\) n'est pas distributive par rapport à \(\times\) : \[1+\paren{2\times3}\not=\paren{1+2}\times\paren{1+3}.\]
\end{itemize}
\end{ex}

\subsection{Parties stables}

\begin{defi}[Partie stable]
Soient \(E\) un ensemble et \(*\) une loi de composition interne sur \(E\).

On dit qu'un partie \(A\subset E\) est stable par la loi \(*\) si on a : \[\quantifs{\forall x,y\in A}x*y\in A.\]

On peut alors définir une loi de composition interne sur \(A\) : \[\fonctionlambda{A\times A}{A}{\paren{x,y}}{x*y}\] qui est appelée la loi de composition interne induite par \(*\) sur \(A\).

Cette loi est souvent encore notée \(*\) (abusivement).
\end{defi}

\begin{ex}
\begin{itemize}
\item \(\Z\) est une partie stable de \(\R\) par \(+\) et \(\times\). \\

\item Soit \(E\prim\in\P{E}\). Alors \(\P{E\prim}\in\P{\P{E}}\) est stable par \(\union\) et \(\inter\). \\

\item Notons \(\mathscr{P}_p\paren{\interventierii{1}{10}}\) les parties de \(\interventierii{1}{10}\) de cardinal pair. Ce n'est une partie stable de \(\P{\interventierii{1}{10}}\) ni pour \(\union\) ni pour \(\inter\). En effet, on a : \[\accol{1;2}\union\accol{1;3}=\accol{1;2;3}\not\in\mathscr{P}_p\paren{\interventierii{1}{10}}\quad\text{et}\quad\accol{1;2}\inter\accol{1;3}=\accol{1}\not\in\mathscr{P}_p\paren{\interventierii{1}{10}}.\]
\end{itemize}
\end{ex}

\section{Groupes}

\subsection{Définition}

\begin{defi}[Groupe]
Un groupe est un couple \(\groupe{G}[*]\) où \(G\) est un ensemble et \(*\) une loi de composition interne sur \(G\) respectant les conditions suivantes :

\begin{enumerate}
\item La loi \(*\) est associative. \\

\item La loi \(*\) admet un élément neutre. \\

\item Tout élément de \(G\) possède un inverse.
\end{enumerate}

On dit aussi que \(G\) est muni d'une structure de groupe, ou, par abus, que \(G\) est un groupe.

Si, de plus, la loi \(*\) est commutative, on dit que le groupe est abélien ou commutatif.
\end{defi}

\begin{ex}[Groupes abéliens]
\[\groupe{\Qs}[\times]\quad\groupe{\Rs}[\times]\quad\groupe{\Cs}[\times]\quad\groupe{\paren{\Rs}^\N}[\times]\quad\groupe{\paren{\Cs}^\N}[\times]\]
\end{ex}

\begin{ex}
Il existe une unique structure de groupe sur \(\accol{0}\) :

\begin{center}
\begin{tikzpicture}
\matrix (mymatrix) [matrix of nodes, nodes={draw, minimum size=6mm, outer sep=0pt}, column sep=-\pgflinewidth, row sep=-\pgflinewidth]
    {  & 0 \\
     0 & 0 \\};
\draw[->, shorten <=1mm, shorten >=1mm, looseness=1.2]
    (mymatrix-2-1.north west)to[out=90, in=180]node[below right=-3pt]{\(+\)}(mymatrix-1-2.north west);
\end{tikzpicture}
\end{center}

En effet, on a \(0+\paren{0+0}=\paren{0+0}+0=0\) donc \(+\) est une loi de groupe. On a aussi \(0\) neutre et \(0\) opposé de \(0\).

Sur \(\accol{0;1}\), il existe deux structures de groupe :

Si \(0\) neutre :

\begin{center}
\begin{tikzpicture}
\matrix (mymatrix) [matrix of nodes, nodes={draw, minimum size=6mm, outer sep=0pt}, column sep=-\pgflinewidth, row sep=-\pgflinewidth]
    {  & 0 & 1\\
     0 & 0 & 1\\
     1 & 1 & 0\\};
\draw[->, shorten <=1mm, shorten >=1mm, looseness=1.2]
    (mymatrix-2-1.north west)to[out=90, in=180]node[below right=-3pt]{\(\oplus\)}(mymatrix-1-2.north west);
\end{tikzpicture}
\end{center}

Si \(1\) neutre :

\begin{center}
\begin{tikzpicture}
\matrix (mymatrix) [matrix of nodes, nodes={draw, minimum size=6mm, outer sep=0pt}, column sep=-\pgflinewidth, row sep=-\pgflinewidth]
    {  & 0 & 1\\
     0 & 1 & 0\\
     1 & 0 & 1\\};
\draw[->, shorten <=1mm, shorten >=1mm, looseness=1.2]
    (mymatrix-2-1.north west)to[out=90, in=180]node[below right=-3pt]{\(\otimes\)}(mymatrix-1-2.north west);
\end{tikzpicture}
\end{center}
\end{ex}

\begin{ex}
Structure de groupe sur \(\accol{0;1;2}\) dont \(0\) est le neutre :

\begin{center}
\begin{tikzpicture}
\matrix (mymatrix) [matrix of nodes, nodes={draw, minimum size=6mm, outer sep=0pt}, column sep=-\pgflinewidth, row sep=-\pgflinewidth]
    {  & 0 & 1 & 2\\
     0 & 0 & 1 & 2\\
     1 & 1 & 2 & 0\\
     2 & 2 & 0 & 1\\};
\draw[->, shorten <=1mm, shorten >=1mm, looseness=1.2]
    (mymatrix-2-1.north west)to[out=90, in=180]node[below right=-3pt]{\(\oplus\)}(mymatrix-1-2.north west);
\end{tikzpicture}
\end{center}

On remarque que cette loi est commutative.
\end{ex}

\begin{ex}[Produit de deux groupes]
Soient \(\groupe{G_1}[*_1]\) et \(\groupe{G_2}[*_2]\) deux groupes.

L'ensemble \(G_1\times G_2\) est naturellement muni d'une structure de groupe, de loi : \[\fonctionlambda{\paren{G_1\times G_2}\times\paren{G_1\times G_2}}{G_1\times G_2}{\paren{\paren{g_1,g_2},\paren{g_1\prim,g_2\prim}}}{\paren{g_1*_1g_1\prim,g_2*_2g_2\prim}}\]

Déterminons son neutre et l'inverse d'un élément \(\paren{g_1,g_2}\) donné.
\end{ex}

\begin{dem}
On note \(*\) la loi.

Montrons que \(*\) est associative :

Soient \(\paren{g_1,g_2},\paren{g_1\prim,g_2\prim},\paren{g_1\seconde,g_2\seconde}\in G_1\times G_2\).

On a : \[\begin{aligned}
\paren{g_1,g_2}*\paren{\paren{g_1\prim,g_2\prim}*\paren{g_1\seconde,g_2\seconde}}&=\paren{g_1,g_2}*\paren{g_1\prim*_1g_1\seconde,g_2\prim*_2g_2\seconde} \\
&=\paren{g_1*_1\paren{g_1\prim*_1g_1\seconde},g_2*_2\paren{g_2\prim*_2g_2\seconde}} \\
&=\paren{\paren{g_1*_1g_1\prim}*_1g_1\seconde,\paren{g_2*_2g_2\prim}*_2g_2\seconde} \\
&=\paren{g_1*_1g_1\prim,g_2*_2g_2\prim}*\paren{g_1\seconde,g_2\seconde} \\
&=\paren{\paren{g_1,g_2}*\paren{g_1\prim,g_2\prim}}*\paren{g_1\seconde,g_2\seconde}
\end{aligned}\]

Montrons que \(*\) admet un neutre :

On note \(e_1\) le neutre de \(G_1\) et \(e_2\) le neutre de \(G_2\).

On remarque : \[\quantifs{\forall\paren{g_1,g_2}\in G_1\times G_2}\begin{dcases}\paren{g_1,g_2}*\paren{e_1,e_2}=\paren{g_1*_1e_1,g_2*_2e_2}=\paren{g_1,g_2} \\ \paren{e_1,e_2}*\paren{g_1,g_2}=\paren{e_1*_1g_1,e_2*_2g_2}=\paren{g_1,g_2}\end{dcases}\]

Donc \(\paren{e_1,e_2}\) est le neutre de \(*\).

Montrons que tout élément admet un inverse :

Soit \(\paren{g_1,g_2}\in G_1\times G_2\).

On remarque : \[\begin{dcases}\paren{g_1,g_2}*\paren{g_1\inv,g_2\inv}=\paren{g_1*_1g_1\inv,g_2*_2g_2\inv}=\paren{e_1,e_2} \\ \paren{g_1\inv,g_2\inv}*\paren{g_1,g_2}=\paren{g_1\inv*_1g_1,g_2\inv*_2g_2}=\paren{e_1,e_2}\end{dcases}\]

Donc \(\paren{g_1,g_2}\) est inversible, d'inverse \(\paren{g_1\inv,g_2\inv}\).
\end{dem}

\begin{rem}
On a : \[G_1\times G_2\text{ abélien}\ssi\begin{dcases}G_1\text{ abélien} \\ G_2\text{ abélien}\end{dcases}\]
\end{rem}

\begin{ex}[Produit de groupes]
Soient \(\groupe{G_1}[*_1],\dots,\groupe{G_n}[*_n]\) des groupes.

Le produit cartésien \(G_1\times\dots\times G_n\) est naturellement muni d'une structure de groupe, de loi : \[\fonctionlambda{\paren{G_1\times\dots\times G_n}\times\paren{G_1\times\dots\times G_n}}{G_1\times\dots\times G_n}{\paren{\paren{g_1,\dots,g_n},\paren{g_1\prim,\dots,g_n\prim}}}{\paren{g_1*_1g_1\prim,\dots,g_n*_ng_n\prim}}\]

Son neutre est \(\paren{e_1,\dots,e_n}\) où \(\quantifs{\forall i\in\interventierii{1}{n}}e_i\text{ neutre de }G_i\).

L'inverse d'un élément \(\paren{g_1,\dots,g_n}\) est \(\paren{g_1,\dots,g_n}\inv=\paren{g_1\inv,\dots,g_n\inv}\).
\end{ex}

\begin{dem}
Idem.
\end{dem}

\begin{ex}\thlabel{ex:fonctionsDeIDansGroupeEstUnGroupe}
Soit \(I\) un ensemble et \(G\) un groupe.

L'ensemble \(\F{I}{G}\) est naturellement muni d'une structure de groupe.

Notons \(\times\) la loi de \(G\) et \(e\) son neutre.

On pose \[\fonction{*}{\F{I}{G}^2}{\F{I}{G}}{\paren{f,g}}{\fonctionlambda{I}{G}{t}{f\paren{t}\times g\paren{t}}}\]

Montrons que \(*\) est associative :

Soient \(f,g,h\in\F{I}{G}\).

On a : \[\begin{WithArrows}
\quantifs{\forall t\in I}f*\paren{g*h}\paren{t}&=f\paren{t}\times\paren{g*h}\paren{t} \\
&=f\paren{t}\times\paren{g\paren{t}\times h\paren{t}}\Arrow{car \(\times\) est associative} \\
&=\paren{f\paren{t}\times g\paren{t}}\times h\paren{t} \\
&=\paren{f*g}\paren{t}\times h\paren{t} \\
&=\paren{f*g}*h\paren{t}
\end{WithArrows}\]

Donc \(*\) est associative.

Montrons que \(*\) admet un élément neutre :

On pose \(\fonction{f_1}{I}{G}{t}{e}\). Montrons que \(f_1\) est l'élément neutre de \(*\).

Soit \(f\in\F{I}{G}\). Montrons que \(f_1*f=f*f_1=f\).

On a \[\quantifs{\forall t\in I}\begin{dcases}\paren{f_1*f}\paren{t}=f_1\paren{t}\times f\paren{t}=e\times f\paren{t}=f\paren{t} \\ \paren{f*f_1}\paren{t}=f\paren{t}\times f_1\paren{t}=f\paren{t}\times e=f\paren{t}\end{dcases}\]

Donc \(f_1\) est l'élément neutre de \(*\).

Montrons que tout élément de \(\F{I}{G}\) est inversible.

Soit \(f\in\F{I}{G}\). On pose \(\fonction{g}{I}{G}{t}{f\paren{t}\inv}\).

Montrons que \(f*g=g*f=f_1\).

On a : \[\quantifs{\forall t\in I}\begin{dcases}\paren{f*g}\paren{t}=f\paren{t}\times g\paren{t}=f\paren{t}\times f\paren{t}\inv=e=f_1\paren{t} \\ \paren{g*f}\paren{t}=g\paren{t}\times f\paren{t}=f\paren{t}\inv\times f\paren{t}=e=f_1\paren{t}\end{dcases}\]

Donc tout élément de \(\F{I}{G}\) est inversible, d'inverse \(g\).
\end{ex}

\begin{rem}
On a : \[\F{I}{G}\text{ abélien}\ssi\orenv{G\text{ abélien} \\ I=\ensvide}\]
\end{rem}

\subsection{Sous-groupes}

\begin{defi}[Sous-groupe]
Soit \(\groupe{G}[*]\) un groupe. On note \(e\) son élément neutre.

Un sous-groupe de \(\groupe{G}[*]\) (ou, par abus, de \(G\)) est une partie \(H\) de \(G\) telle que :

\begin{enumerate}
\item L'élément neutre \(e\) appartient à \(H\). \\

\item La partie \(H\) est stable par produit : \(\quantifs{\forall h_1,h_2\in H}h_1*h_2\in H\). \\

\item La partie \(H\) est stable par passage à l'inverse : \(\quantifs{\forall h\in H}h\inv\in H\).
\end{enumerate}
\end{defi}

\begin{prop}
Tout sous-groupe d'un groupe \(G\) est naturellement un groupe. Sa loi est la loi induite par celle de \(G\).
\end{prop}

\begin{dem}
On note \(*\) la loi de \(G\).

Soit \(H\) un sous-groupe de \(G\).

Comme \(H\) est stable par \(*\), on peut définir : \(\fonction{*_H}{H^2}{H}{\paren{h_1,h_2}}{h_1*h_2}\)

Montrons que \(*_H\) est associative :

On a bien \[\quantifs{\forall h_1,h_2,h_3\in H}h_1*_H\paren{h_2*_Hh_3}=h_1*\paren{h_2*h_3}=\paren{h_1*h_2}*h_3=\paren{h_1*_Hh_2}*_Hh_3.\]

Montrons que \(*_H\) admet un élément neutre :

On a \(e\in H\) et \[\quantifs{\forall h\in H}\begin{dcases}e*_Hh=e*h=h \\ h*_He=h*e=h\end{dcases}\] donc \(e\) est le neutre de \(*_H\).

Montrons que tout élément de \(H\) est inversible par \(*_H\) :

Soit \(h\in H\). On note \(h\inv\) son inverse dans \(G\).

On a \[\begin{dcases}h\inv\in H \\ h*_Hh\inv=h\inv*_Hh=e\end{dcases}\]

Donc \(\groupe{H}[*_H]\) est un groupe.
\end{dem}

\begin{rem}
On a \[H\text{ abélien}\impr G\text{ abélien}\]
\end{rem}

\begin{rem}
En pratique, on note également \(*\) la loi \(*_H\) de \(H\).
\end{rem}

\begin{prop}
Soit \(\groupe{G}[*]\) un groupe. On note \(e\) son élément neutre.

Soit \(H\) une partie de \(G\).

Alors \(H\) est un sous-groupe de \(G\) si, et seulement si, les conditions suivantes sont vérifiées :

\begin{enumerate}\setcounter{enumi}{3}
\item \(e\in H\) \\

\item \(\quantifs{\forall h_1,h_2\in H}h_1*h_2\inv\in H\) \\
\end{enumerate}
\end{prop}

\begin{dem}
\impdir

Supposons \(H\) sous-groupe de \(G\).

On a clairement (4) selon (1).

Montrons (5).

Soient \(h_1,h_2\in H\).

On a \(h_2\inv\in H\) selon (3) donc \(h_1*h_2\inv\in H\) selon (2).

\imprec

Supposons (4) et (5).

On a (1) selon (4).

Montrons (3).

Soit \(h\in H\).

On a \(e\in H\) selon (4) donc selon (5), \(e*h\inv\in H\) donc \(h\inv\in H\).

Montrons (2).

Soient \(h_1,h_2\in H\).

On a \(h_2\inv\in H\) selon (3) donc \(h_1*\paren{h_2\inv}\inv\in H\) donc \(h_1*h_2\in H\).
\end{dem}

\begin{ex}
Soit \(n\in\Ns\).

L'ensemble \(\U\) des nombres complexes de module \(1\) est un sous-groupe de \(\Cs\).

L'ensemble \(\U_n\) des racines \(n\)-ièmes de l'unité est un sous-groupe de \(\U\).
\end{ex}

\begin{dem}
On sait que \(\groupe{\Cs}[\times]\) est un groupe. Montrons que \(\U\) est un sous-groupe de \(\Cs\).

On a \(1\in\U\) et \(\quantifs{\forall z_1,z_2\in\U}z_1\times z_2\inv\in\U\) (car \(\abs{z_1\times z_2\inv}=\dfrac{\abs{z_1}}{\abs{z_2}}=\dfrac{1}{1}=1\)).

Donc \(\U\) est un sous-groupe de \(\Cs\). Donc \(\U\) est un groupe.

Montrons que \(\U_n\) est un sous-groupe de \(\U\).

On a \(\U_n\subset\U\) et \(1\in\U_n\) et \(\quantifs{\forall z_1,z_2\in\U_n}z_1\times z_2\inv\in\U_n\) car \(\paren{\dfrac{z_1}{z_2}}^n=\dfrac{z_1^n}{z_2^n}=\dfrac{1}{1}=1\).

Donc \(\U_n\) est un sous-groupe de \(\U\). Donc c'est un groupe.
\end{dem}

\subsection{Morphismes de groupe}

\begin{defi}[Morphisme de groupes]
Soient \(\groupe{G_1}[*_1]\) et \(\groupe{G_2}[*_2]\) deux groupes.

On appelle morphisme de groupes de \(\groupe{G_1}[*_1]\) vers \(\groupe{G_2}[*_2]\) (ou, par abus, de \(G_1\) vers \(G_2\)) toute application \(\phi:G_1\to G_2\) telle que : \[\quantifs{\forall g,g\prim\in G_1}\phi\paren{g*_1g\prim}=\phi\paren{g}*_2\phi\paren{g\prim}.\]

L'ensemble des morphismes de groupes de \(G_1\) vers \(G_2\) est noté : \[\Hom{G_1}{G_2}.\]
\end{defi}

\begin{prop}\thlabel{prop:neutreEtInverseParUnMorphismeDeGroupes}
Soient \(\groupe{G_1}[*_1]\) et \(\groupe{G_2}[*_2]\) deux groupes et \(\phi:G_1\to G_2\) un morphisme de groupes de \(\groupe{G_1}[*_1]\) vers \(\groupe{G_2}[*_2]\). On note \(e_1\) et \(e_2\) les éléments neutres respectifs de \(G_1\) et \(G_2\).

On a alors : \[\phi\paren{e_1}=e_2\] et : \[\quantifs{\forall g\in G}\phi\paren{g\inv}=\phi\paren{g}\inv.\]
\end{prop}

\begin{dem}
On a \(e_1*_1e_1=e_1\) donc : \[\begin{aligned}
\phi\paren{e_1*_1e_1}&=\phi\paren{e_1} \\
\phi\paren{e_1}*_2\phi\paren{e_1}&=\phi\paren{e_1} \\
\phi\paren{e_1}\inv*_2\phi\paren{e_1}*_2\phi\paren{e_1}&=\phi\paren{e_1}\inv*_2\phi\paren{e_1} \\
e_2*_2\phi\paren{e_1}&=e_2 \\
\phi\paren{e_1}&=e_2
\end{aligned}\]

Soit \(g\in G_1\).

On a \(g*_1g\inv=e_1\) donc : \[\begin{aligned}
\phi\paren{g*_1g\inv}&=\phi\paren{e_1} \\
\phi\paren{g}*_2\phi\paren{g\inv}&=e_2 \\
\phi\paren{g}\inv*_2\phi\paren{g}*_2\phi\paren{g\inv}&=\phi\paren{g}\inv*_2e_2 \\
\phi\paren{g\inv}&=\phi\paren{g}\inv
\end{aligned}\]
\end{dem}

\begin{prop}\thlabel{prop:imagesDirectesEtReciproquesMorphismeSousGroupes}
Soient \(\groupe{G_1}[*_1]\) et \(\groupe{G_2}[*_2]\) deux groupes et \(\phi:G_1\to G_2\) un morphisme de groupes de \(\groupe{G_1}[*_1]\) vers \(\groupe{G_2}[*_2]\).

Soient \(H_1\) un sous-groupe de \(G_1\) et \(H_2\) un sous-groupe de \(G_2\).

Alors :

\begin{enumerate}
\item L'image directe \(\phi\paren{H_1}\) est un sous-groupe de \(G_2\). \\

\item L'image réciproque \(\phi\inv\paren{H_2}\) est un sous-groupe de \(G_1\).
\end{enumerate}
\end{prop}

\begin{dem}[1]
Montrons que \(\phi\paren{H_1}\) est un sous-groupe de \(G_2\).

On a \(\phi\paren{H_1}\subset G_2\).

On note \(e_1\) l'élément neutre de \(G_1\) et \(e_2\) l'élément neutre de \(G_2\).

On a \(e_1\in H_1\) donc \(\phi\paren{e_1}\in\phi\paren{H_1}\) donc \(e_2\in\phi\paren{H_1}\).

Montrons que \(\quantifs{\forall y_1,y_2\in\phi\paren{H_1}}y_1*_2y_2\inv\in\phi\paren{H_1}\).

Soient \(y_1,y_2\in\phi\paren{H_1}\). Soient \(x_1,x_2\in H_1\) tels que \(\begin{dcases}\phi\paren{x_1}=y_1 \\ \phi\paren{x_2}=y_2\end{dcases}\)

On a \(x_1*_1x_2\inv\in H_1\) donc \(\phi\paren{x_1*_1x_2\inv}\in\phi\paren{H_1}\).

Or \(\phi\paren{x_1*_1x_2\inv}=\phi\paren{x_1}*_2\phi\paren{x_2\inv}=\phi\paren{x_1}*_2\phi\paren{x_2}\inv=y_1*_2y_2\inv\).

Donc \(y_1*_2y_2\inv\in\phi\paren{H_1}\).

Donc \(\phi\paren{H_1}\) est un sous-groupe de \(G_2\).
\end{dem}

\begin{dem}[2]
Montrons que \(\phi\inv\paren{H_2}\) est un sous-groupe de \(G_1\).

On a \(\phi\inv\paren{H_2}\subset G_1\).

On note \(e_1\) l'élément neutre de \(G_1\) et \(e_2\) l'élément neutre de \(G_2\).

On a \(e_1\in\phi\inv\paren{H_2}\) car \(\phi\paren{e_1}=e_2\in H_2\).

Enfin, on a \(\quantifs{\forall x_1,x_2\in\phi\inv\paren{H_2}}\phi\paren{x_1*_1x_2\inv}=\phi\paren{x_1}*_2\phi\paren{x_2\inv}=\phi\paren{x_1}*_2\phi\paren{x_2}\inv\in H_2\) car \(H_2\) est un groupe.

Donc \(\quantifs{\forall x_1,x_2\in\phi\inv\paren{H_2}}x_1*_1x_2\inv\in\phi\inv\paren{H_2}\).

Donc \(\phi\inv\paren{H_2}\) est un sous-groupe de \(G_1\).
\end{dem}

\begin{prop}
Soit \(G\) un groupe et \(H\) un sous-groupe de \(G\).

L'application \[\fonction{i}{H}{G}{h}{h}\] est un morphisme de groupes.
\end{prop}

\begin{dem}
On note \(*\) la loi de \(G\) et \(*_H\) la loi de \(H\) (induite).

On remarque : \[\quantifs{\forall h_1,h_2\in H}i\paren{h_1*_Hh_2}=h_1*_Hh_2=h_1*h_2.\]
\end{dem}

\begin{ex}
Soit \(I\) un ensemble, \(J\) une partie de \(I\) et \(G\) un groupe.

Alors l'application de restriction \[\fonctionlambda{\F{I}{G}}{\F{J}{G}}{f}{\restr{f}{J}}\] est un morphisme de groupes.
\end{ex}

\begin{ex}
La fonction \(\fonction{\phi}{\R}{\U}{\theta}{\e{\i\theta}}\) est un morphisme de groupes de \(\groupe{\R}\) vers \(\groupe{\U}[\times]\).
\end{ex}

\begin{dem}
On a : \[\quantifs{\forall\theta_1,\theta_2\in\R}\phi\paren{\theta_1+\theta_2}=\e{\i\paren{\theta_1+\theta_2}}=\e{\i\theta_1}\e{\i\theta_2}=\phi\paren{\theta_1}\phi\paren{\theta_2}.\]
\end{dem}

\begin{prop}[Composition de morphismes de groupes]\thlabel{prop:composeeMorphismesEstUnMorphisme}
Soient \(G_1\), \(G_2\) et \(G_3\) des groupes et \(\phi:G_1\to G_2\) et \(\psi:G_2\to G_3\) des morphismes de groupes.

Alors \[\psi\rond\phi:G_1\to G_3\] est un morphisme de groupes.
\end{prop}

\begin{dem}
On note \(*_1\) la loi de \(G_1\), \(*_2\) la loi de \(G_2\) et \(*_3\) la loi de \(G_3\).

On a : \[\begin{aligned}
\quantifs{\forall g,g\prim\in G_1}\psi\rond\phi\paren{g*_1g\prim}&=\psi\paren{\phi\paren{g*_1g\prim}} \\
&=\psi\paren{\phi\paren{g}*_2\phi\paren{g\prim}} \\
&=\psi\paren{\phi\paren{g}}*_3\psi\paren{\phi\paren{g\prim}} \\
&=\psi\rond\phi\paren{g}*_3\psi\rond\phi\paren{g\prim}
\end{aligned}\]

Donc \(\psi\rond\phi\) est un morphisme de groupes.
\end{dem}

\begin{defi}[Noyau et image d'un morphisme]
Soient \(G_1\) et \(G_2\) deux groupes et \(\phi:G_1\to G_2\) un morphisme de groupes.

On note \(e_2\) le neutre de \(G_2\).

L'image du morphisme \(\phi\) est l'ensemble image de l'application \(\phi\) : \[\Im\phi=\accol{\phi\paren{g_1}}_{g_1\in G_1}=\accol{g_2\in G_2\tq\quantifs{\exists g_1\in G_1}\phi\paren{g_1}=g_2}.\]

Le noyau du morphisme \(\phi\) est l'ensemble des éléments de \(G_1\) dont l'image par \(\phi\) est le neutre de \(G_2\) : \[\ker\phi=\phi\inv\paren{\accol{e_2}}=\accol{g_1\in G_1\tq\phi\paren{g_1}=e_2}.\]
\end{defi}

\begin{prop}
Soient \(G_1\) et \(G_2\) deux groupes et \(\phi:G_1\to G_2\) un morphisme de groupes.

Alors \(\ker\phi\) est un sous-groupe de \(G_1\) et \(\Im\phi\) est un sous-groupe de \(G_2\).
\end{prop}

\begin{dem}
On a \(\Im\phi=\phi\paren{G_1}\) donc \(\Im\phi\) est l'image directe de \(G_2\) par \(\phi\).

Donc \(\Im\phi\) est un sous-groupe de \(G_2\) selon la \thref{prop:imagesDirectesEtReciproquesMorphismeSousGroupes} car \(\phi\) est un morphisme de groupes et \(G_1\) est un sous-groupe de \(G_1\).

\(\ker\phi\) est l'image réciproque de \(\accol{e_2}\) par \(\phi\).

Donc \(\ker\phi\) est un sous-groupe de \(G_1\) selon la \thref{prop:imagesDirectesEtReciproquesMorphismeSousGroupes} car \(\phi\) est un morphisme de groupes et \(\accol{e_2}\) est un sous-groupe de \(G_2\).
\end{dem}

\begin{theo}\thlabel{theo:morphismeDeGroupeInjectifSsiNoyauNul}
Soit \(\phi:G_1\to G_2\) un morphisme de groupes.

On note \(e_1\) le neutre de \(G_1\).

Alors \(\phi\) est une application injective si, et seulement si, son noyau est \guillemets{nul}, \cad : \[\ker\phi=\accol{e_1}.\]
\end{theo}

\begin{dem}
On note \(e_2\) le neutre de \(G_2\).

\impdir

Supposons \(\phi\) injectif.

Montrons que \(\ker\phi=\accol{e_1}\).

\increc On a \(\phi\paren{e_1}=e_2\) donc \(e_1\in\phi\inv\paren{\accol{e_2}}\) donc \(\accol{e_1}\subset\ker\phi\).

\incdir Soit \(x\in\ker\phi\). On a \(\phi\paren{x}=e_2=\phi\paren{e_1}\). Donc \(x=e_1\) car \(\phi\) est injectif. Donc \(\ker\phi\subset\accol{e_1}\).

Finalement, on a \(\ker\phi=\accol{e_1}\).

\imprec

Supposons \(\ker\phi=\accol{e_1}\).

Montrons que \(\phi\) est injectif.

Soient \(x,y\in G_1\) tels que \(\phi\paren{x}=\phi\paren{y}\). Montrons que \(x=y\).

On a : \[\begin{aligned}
\phi\paren{x}&=\phi\paren{y} \\
\phi\paren{x}\phi\paren{y}\inv&=e_2 \\
\phi\paren{x}\phi\paren{y\inv}&=e_2 \\
\phi\paren{xy\inv}&=e_2
\end{aligned}\]

Donc \(xy\inv\in\ker\phi\) donc \(xy\inv=e_1\) car \(\ker\phi=\accol{e_1}\).

Donc \(x=y\).

Donc \(\phi\) est injectif.
\end{dem}

\subsection{Isomorphismes, endomorphismes, automorphismes}

\begin{defi}[Isomorphisme]
Soient \(G_1\) et \(G_2\) deux groupes.

Un isomorphisme de groupes de \(G_1\) vers \(G_2\) est un morphisme de groupes \(\phi:G_1\to G_2\) bijectif.
\end{defi}

\begin{theo}
Soient \(G_1\), \(G_2\) et \(G_3\) des groupes.

Soient \(\phi:G_1\to G_2\) et \(\phi:G_2\to G_3\) deux isomorphismes de groupes.

Alors \(\psi\rond\phi:G_1\to G_3\) est un isomorphisme de groupes.
\end{theo}

\begin{dem}
Comme \(\psi\) et \(\phi\) sont des morphismes de groupes, d'après la \thref{prop:composeeMorphismesEstUnMorphisme}, \(\psi\rond\phi\) est un morphisme de groupes.

Comme \(\psi\) et \(\phi\) sont des bijections, d'après la \thref{prop:composeeBijectionsEstUneBijection}, \(\psi\rond\phi\) est une bijection.

Donc \(\psi\rond\phi\) est un isomorphisme de groupes.
\end{dem}

\begin{theo}
Soit \(\phi:G_1\to G_2\) un isomorphisme de groupes.

Alors la bijection réciproque \(\phi\inv:G_2\to G_1\) est un isomorphisme de groupes.
\end{theo}

\begin{dem}
On note \(*_1\) la loi de \(G_1\) et \(*_2\) la loi de \(G_2\).

On sait déjà que \(\phi\inv\) est une bijection (\thref{defprop:bijRec}).

Montrons que \(\phi\inv:G_2\to G_1\) est un morphisme de groupes.

Soient \(y,y\prim\in G_2\).

On a \(\phi\rond\phi\inv\paren{y}=y\) et \(\phi\rond\phi\inv\paren{y\prim}=y\prim\).

Donc, comme \(\phi\) est un morphisme de groupes, on a \(\phi\paren{\phi\inv\paren{y}*_1\phi\inv\paren{y\prim}}=y*_2y\prim\).

Donc \(\phi\inv\paren{y}*_1\phi\inv\paren{y\prim}\) est l'antécédent de \(y*_2y\prim\) par \(\phi\), \cad \(\phi\inv\paren{y}*_1\phi\inv\paren{y\prim}=\phi\inv\paren{y*_2y\prim}\).

Donc \(\phi\inv\) est un morphisme de groupes.

Donc \(\phi\inv\) est un isomorphisme de groupes.
\end{dem}

\begin{defi}[Endomorphisme]
Soit \(G\) un groupe.

Un endomorphisme de groupe de \(G\) est un morphisme de groupes \(\phi:G\to G\).
\end{defi}

\begin{defi}[Automorphisme]
Soit \(G\) un groupe.

Un automorphisme de groupe de \(G\) est un isomorphisme de groupes \(\phi:G\to G\), \cad un endomorphisme de \(G\) bijectif.

L'ensemble des automorphismes de groupe de \(G\) est noté \(\Aut{G}\).
\end{defi}

\begin{ex}
Soit \(G\) un groupe.

Alors \(\groupe{\Aut{G}}[\rond]\) est un groupe appelé le groupe des automorphismes de \(G\).
\end{ex}

\begin{dem}
\note{EXERCICE}
\end{dem}

\section{Anneaux}

\subsection{Définition}

\begin{defi}[Anneau]
Un anneau est un triplet \(\anneau{A}\) où \(A\) est un ensemble et \(+\) et \(\times\) sont deux lois de composition internes sur \(A\) tels que les conditions suivantes soient vérifiées :

\begin{enumerate}
\item \(\groupe{A}\) est un groupe commutatif. \\

\item La loi \(\times\) est associative et admet un élément neutre. \\

\item La loi \(\times\) est distributive par rapport à la loi \(+\), \cad : \[\quantifs{\forall a,b,c\in A}a\times\paren{b+c}=a\times b+a\times c\quad\text{et}\quad\paren{b+c}\times a=b\times a+c\times a.\]
\end{enumerate}

On dit aussi que \(A\) est muni d'une structure d'anneau, ou, par abus, que \(A\) est un anneau.

Si, de plus, la loi \(\times\) est commutative, on dit que \(A\) est un anneau commutatif.
\end{defi}

\begin{rem}
La loi \(+\) est appelée l'addition de l'anneau.

La loi \(\times\) est appelée sa multiplication. On s'autorise à ne pas écrire son symbole (\cad écrire \(ab\) au lieu de \(a\times b\)).

L'élément neutre de \(+\) est généralement noté \(0\) ou \(0_A\) ; celui de \(\times\) est noté \(1\) ou \(1_A\).

Si \(a\) est un élément de \(A\), son \guillemets{élément inverse} pour \(+\) est appelé son opposé et est noté \(-a\) (il existe car \(\groupe{A}\) est un groupe commutatif) ; son élément inverse pour \(\times\) est appelé son inverse et est noté \(a\inv\) (s'il existe.)
\end{rem}

\begin{ex}
On vérifie facilement que \(\anneau{\Z}\), \(\anneau{\Q}\), \(\anneau{\R}\), \(\anneau{\C}\), \(\anneau{\R^\N}\) et \(\anneau{\C^\N}\) sont des anneaux commutatifs.
\end{ex}

\begin{ex}\thlabel{ex:produitD'anneauxEstUnAnneau}
Soient \(\anneau{A_1}[+_1][\times_1]\) et \(\anneau{A_2}[+_2][\times_2]\) deux anneaux de neutres \(0_1,1_1\) et \(0_2,1_2\) respectivement.

L'ensemble \(A_1\times A_2\) est naturellement muni d'une structure d'anneau, de lois \[\fonctionlambda{\paren{A_1\times A_2}\times\paren{A_1\times A_2}}{A_1\times A_2}{\paren{\paren{a_1,a_2},\paren{a_1\prim,a_2\prim}}}{\paren{a_1+_1a_1\prim,a_2+_2a_2\prim}}\] et \[\fonctionlambda{\paren{A_1\times A_2}\times\paren{A_1\times A_2}}{A_1\times A_2}{\paren{\paren{a_1,a_2},\paren{a_1\prim,a_2\prim}}}{\paren{a_1\times_1a_1\prim,a_2\times_2a_2\prim}}\]

Ses éléments neutres sont \(\paren{0_1,0_2}\) et \(\paren{1_1,1_2}\).
\end{ex}

\begin{ex}\thlabel{ex:fonctionsD'unEnsembleDansUnAnneauEstUnAnneau}
Soient \(I\) un ensemble et \(A\) un anneau.

\(\F{I}{A}\) est naturellement muni d'une structure d'anneau.

On note \(\begin{dcases}+\text{ et }\times\text{ les lois de }A \\ 0\text{ et }1\text{ les lois de }A\end{dcases}\)

Comme \(\groupe{A}\) est un groupe commutatif, on sait que \(\F{I}{A}\) est un groupe commutatif pour la loi \[\fonction{\oplus}{\F{I}{A}^2}{\F{I}{A}}{\paren{f,g}}{\fonctionlambda{I}{A}{x}{f\paren{x}+g\paren{x}}}\] dont le neutre est \(\fonctionlambda{I}{A}{x}{0}\) (\cf \thref{ex:fonctionsDeIDansGroupeEstUnGroupe}).

On pose d'autre part \[\fonction{\otimes}{\F{I}{A}^2}{\F{I}{A}}{\paren{f,g}}{\fonctionlambda{I}{A}{x}{f\paren{x}\times g\paren{x}}}\]

On vérifie facilement que \(\otimes\) est associative et qu'elle admet pour neutre \(\fonctionlambda{I}{A}{x}{1}\).

Vérifions maintenant que \(\otimes\) est distributive par rapport à \(\oplus\).

Soient \(f,g,h\in\F{I}{A}\).

Montrons que \(\begin{dcases}f\otimes\paren{g\oplus h}=f\otimes g\oplus f\otimes h &\text{(1)} \\ \paren{g\oplus h}\otimes f=g\otimes f\oplus h\otimes f &\text{(2)}\end{dcases}\)

Soit \(x\in I\).

On a : \[\begin{WithArrows}
\croch{f\otimes\paren{g\oplus h}}\paren{x}&=f\paren{x}\times\paren{g\oplus h}\paren{x} \Arrow{par définition de \(\oplus\)} \\
&=f\paren{x}\times\paren{g\paren{x}+h\paren{x}} \Arrow{car \(A\) est un anneau} \\
&=f\paren{x}\times g\paren{x}+f\paren{x}\times h\paren{x} \Arrow{par définition de \(\otimes\)} \\
&=\croch{f\otimes g}\paren{x}+\croch{f\otimes h}\paren{x} \Arrow{par définition de \(\oplus\)} \\
&=\croch{\paren{f\otimes g}\oplus\paren{f\otimes h}}\paren{x}
\end{WithArrows}\]

D'où (1).

On montre de même (2).

Donc \(\otimes\) est distributive par rapport à \(\oplus\).

Donc \(\anneau{\F{I}{A}}\) est un anneau.
\end{ex}

\subsection{Calculs dans un anneau}

\begin{prop}
Soit \(\anneau{A}\) un anneau.

On note \(0\) son neutre pour \(+\).

On a, pour tout élément \(a\) de \(A\) : \[0\times a=a\times 0=0.\]

On dit que l'élément \(0\) est absorbant pour le produit.
\end{prop}

\begin{dem}
Soit \(a\in A\).

On a : \[\begin{WithArrows}
1+0&=1 \Arrow{on multiplie à gauche par \(a\)} \\
a\paren{1+0}&=a\times1 \Arrow{car \(\times\) est distributive par rapport à \(+\)} \\
a\times1+a\times0&=a\times1 \\
a+a\times0&=a \Arrow{on ajoute \(-a\) de chaque côté} \\
a\times0&=0
\end{WithArrows}\]
\end{dem}

\begin{prop}
Soit \(\anneau{A}\) un anneau.

On a : \[\quantifs{\forall a,b\in A}-\paren{a\times b}=\paren{-a}\times b=a\times\paren{-b}\quad\text{et}\quad\paren{-a}\times\paren{-b}=ab.\]
\end{prop}

\begin{dem}
Soient \(a,b\in A\).

On a d'une part \(a\paren{b-b}=a\times0=0\) et d'autre part \(a\paren{b-b}=ab+a\paren{-b}\).

Ainsi, on a \(ab+a\paren{-b}=0\).

D'où, en ajoutant l'opposé de \(ab\) de chaque côté : \(a\paren{-b}=-ab\).

De même, on a : \(\paren{-a}b+ab=\paren{-a+a}b=0\times b=0\) donc \(\paren{-a}b=-ab\).

Enfin, selon ce qui précède : \(\paren{-a}\paren{-b}=-a\paren{-b}=-\paren{-ab}=ab\).
\end{dem}

\begin{defi}[Anneau nul]
Un anneau \(\anneau{A}\) est dit nul si \(A\) est un singleton.
\end{defi}

\begin{prop}
Un anneau \(\anneau{A}\) d'éléments neutres \(0\) et \(1\) est nul si, et seulement si, on a : \(0=1\).
\end{prop}

\begin{dem}
\impdir Si \(A\) est nul alors \(0=1\).

\imprec

Supposons \(0=1\). Montrons que \(A\) est nul.

On a \(\quantifs{\forall a\in A}a=a\times1=a\times0=0\).

Donc \(\Card A=1\).
\end{dem}

\begin{prop}
Soit \(n\in\Ns\). Soient \(A\) un anneau et \(a\) et \(b\) deux éléments de \(A\) tels que : \[ab=ba\] (on dit que les éléments \(a\) et \(b\) commutent).

On a :

\begin{enumerate}
\item La formule du binôme de Newton : \[\paren{a+b}^n=\sum_{k=0}^{n}\binom{k}{n}a^kb^{n-k}\]

\item La factorisation de \(a^n-b^n\) : \[a^n-b^n=\paren{a-b}\sum_{k=0}^{n-1}a^kb^{n-k-1}\]
\end{enumerate}
\end{prop}

\begin{dem}[1]
Pour tout \(n\in\N\) on pose \(\P{n}:\paren{a+b}^n=\sum_{k=0}^{n}\binom{k}{n}a^kb^{n-k}\).

Si \(n=0\), on a : \(\paren{a+b}^0=1\) et \(\sum_{k=0}^{0}=\binom{k}{0}a^kb^{0-k}=\binom{0}{0}a^0b^0=1\) donc \(\P{0}\) est vraie.

Soit \(n\in\N\) tel que \(\P{n}\). Montrons \(\P{n+1}\) :

On a : \[\begin{WithArrows}
\paren{a+b}^{n+1}&=\paren{a+b}\paren{a+b}^n \Arrow{selon \(\P{n}\)} \\
&=\paren{a+b}\sum_{k=0}^{n}\binom{k}{n}a^kb^{n-k} \\
&=\sum_{k=0}^{n}\binom{k}{n}a^{k+1}b^{n-k}+\sum_{k=0}^{n}\binom{k}{n}ba^kb^{n-k} \Arrow{car \(ba=ab\)} \\
&=\sum_{k=1}^{n+1}\binom{k-1}{n}a^kb^{n-k+1}+\sum_{k=0}^{n}\binom{k}{n}a^kb^{n-k+1} \Arrow{car \(\binom{-1}{n}=\binom{n+1}{n}=0\)} \\
&=\sum_{k=1}^{n+1}\binom{k-1}{n}a^kb^{n-k+1}+\sum_{k=0}^{n+1}\binom{k}{n}a^kb^{n-k+1} \\
&=\sum_{k=0}^{n+1}\paren{\binom{k-1}{n}+\binom{k}{n}}a^kb^{n-k+1} \\
&=\sum_{k=0}^{n+1}\binom{k}{n+1}a^kb^{n-k+1}
\end{WithArrows}\]

D'où \(\P{n+1}\).

Donc pour tout \(n\in\N\), on a \(\P{n}\) par récurrence.
\end{dem}

\begin{dem}[2]
On a : \[\begin{WithArrows}
\paren{a-b}\sum_{k=0}^{n-1}a^kb^{n-k-1}&=\sum_{k=0}^{n-1}\paren{a^{k+1}b^{n-k-1}-ba^kb^{n-k-1}} \Arrow{car \(ba=ab\)} \\
&=\sum_{k=0}^{n-1}\paren{a^{k+1}b^{n-k-1}-a^kb^{n-k}} \Arrow{somme téléscopique} \\
&=a^nb^0-a^0b^n \\
&=a^n-b^n
\end{WithArrows}\]
\end{dem}

\begin{ex}
Déterminons toutes les structures d'anneau sur \(\accol{0}\), puis sur \(\accol{0;1}\).

L'unique structure d'anneau sur \(\accol{0}\) est la suivante :

\begin{center}
\begin{tikzpicture}
\matrix (mymatrix) [matrix of nodes, nodes={draw, minimum size=6mm, outer sep=0pt}, column sep=-\pgflinewidth, row sep=-\pgflinewidth]
    {  & 0\\
     0 & 0\\};
\draw[->, shorten <=1mm, shorten >=1mm, looseness=1.2]
    (mymatrix-2-1.north west)to[out=90, in=180]node[below right=-3pt]{\(+\)}(mymatrix-1-2.north west);
\end{tikzpicture}
\hspace{2cm}
\begin{tikzpicture}
\matrix (mymatrix) [matrix of nodes, nodes={draw, minimum size=6mm, outer sep=0pt}, column sep=-\pgflinewidth, row sep=-\pgflinewidth]
    {  & 0\\
     0 & 0\\};
\draw[->, shorten <=1mm, shorten >=1mm, looseness=1.2]
    (mymatrix-2-1.north west)to[out=90, in=180]node[below right=-3pt]{\(\times\)}(mymatrix-1-2.north west);
\end{tikzpicture}
\end{center}

On obtient bien une structure d'anneau :

\begin{itemize}
\item \(\groupe{\accol{0}}\) est bien un groupe commutatif (le groupe nul). \\

\item \(\times\) est associative et possède un neutre. \\

\item \(\times\) est bien distributive par rapport à \(+\) car \[\quantifs{\forall a,b,c\in\accol{0}}a\paren{b+c}=0=0+0=ab+ac.\]
\end{itemize}

Déterminons maintenant les structures d'anneau sur \(A=\accol{0;1}\).

\analyse

Comme \(\Card A=2\), les lois \(+\) et \(\times\) n'admettent pas le même élément neutre.

Si le neutre de \(+\) est \(0\) et celui de \(\times\) est \(1\) :

\begin{center}
\begin{tikzpicture}
\matrix (mymatrix) [matrix of nodes, nodes={draw, minimum size=6mm, outer sep=0pt}, column sep=-\pgflinewidth, row sep=-\pgflinewidth]
    {  & 0 & 1\\
     0 & 0 & 1\\
	 1 & 1 & 0\\};
\draw[->, shorten <=1mm, shorten >=1mm, looseness=1.2]
    (mymatrix-2-1.north west)to[out=90, in=180]node[below right=-3pt]{\(+\)}(mymatrix-1-2.north west);
\end{tikzpicture}
\hspace{2cm}
\begin{tikzpicture}
\matrix (mymatrix) [matrix of nodes, nodes={draw, minimum size=6mm, outer sep=0pt}, column sep=-\pgflinewidth, row sep=-\pgflinewidth]
    {  & 0 & 1\\
     0 & 0 & 0\\
	 1 & 0 & 1\\};
\draw[->, shorten <=1mm, shorten >=1mm, looseness=1.2]
    (mymatrix-2-1.north west)to[out=90, in=180]node[below right=-3pt]{\(\times\)}(mymatrix-1-2.north west);
\end{tikzpicture}
\end{center}

Si le neutre de \(+\) est \(1\) et celui de \(\times\) est \(0\) :

\begin{center}
\begin{tikzpicture}
\matrix (mymatrix) [matrix of nodes, nodes={draw, minimum size=6mm, outer sep=0pt}, column sep=-\pgflinewidth, row sep=-\pgflinewidth]
    {  & 0 & 1\\
     0 & 1 & 0\\
	 1 & 0 & 1\\};
\draw[->, shorten <=1mm, shorten >=1mm, looseness=1.2]
    (mymatrix-2-1.north west)to[out=90, in=180]node[below right=-3pt]{\(+\)}(mymatrix-1-2.north west);
\end{tikzpicture}
\hspace{2cm}
\begin{tikzpicture}
\matrix (mymatrix) [matrix of nodes, nodes={draw, minimum size=6mm, outer sep=0pt}, column sep=-\pgflinewidth, row sep=-\pgflinewidth]
    {  & 0 & 1\\
     0 & 0 & 1\\
	 1 & 1 & 1\\};
\draw[->, shorten <=1mm, shorten >=1mm, looseness=1.2]
    (mymatrix-2-1.north west)to[out=90, in=180]node[below right=-3pt]{\(\times\)}(mymatrix-1-2.north west);
\end{tikzpicture}
\end{center}

\synthese

On admet qu'on obtient bien deux structures d'anneau (on obtient deux anneaux isomorphes à \(\nicefrac{\Z}{2\Z}\)).
\end{ex}

\begin{defi}[Élément inversible d'un anneau]
Soient \(\anneau{A}\) un anneau et \(a\) un élément de \(A\).

On dit que \(a\) est inversible dans l'anneau \(A\) s'il est inversible pour la loi \(\times\).

L'ensemble des éléments inversibles est souvent noté \(A\croix\).
\end{defi}

\begin{defprop}[Groupe des inversibles d'un anneau]\thlabel{defprop:groupeDesInversiblesD'UnAnneau}
Soit \(\anneau{A}\) un anneau.

On appelle groupe des inversibles de l'anneau \(A\) l'ensemble des éléments inversibles de \(A\), muni de la loi \(\times\).

C'est un groupe commutatif si l'anneau \(A\) est commutatif.
\end{defprop}

\begin{dem}
Montrons que l'application \(\fonctionlambda{A\croix\times A\croix}{A\croix}{\paren{a,b}}{ab}\) est bien définie, \cad que \(A\croix\) est une partie de \(A\) stable par \(\times\).

Soient \(a,b\in A\croix\). On sait que \(ab\) est inversible, d'inverse \(\paren{ab}\inv=b\inv a\inv\) (\cf \thref{prop:inverseProduitEgalProduitInversesInverse}).

Donc \(ab\in A\croix\) donc \(A\croix\) est stable par \(\times\).

On a \(\quantifs{\forall a,b,c\in A\croix}a\paren{bc}=\paren{ab}c\) donc \(\times\) est associative.

La loi \(\times\) admet \(1\) comme élément neutre dans \(A\croix\) car \(1\in A\croix\) (\cf \thref{ex:inverseElementNeutreEgalElementNeutre}).

Tout élément de \(A\croix\) est inversible dans \(A\croix\) (\cf \thref{ex:inverseInverseEgalElement}).

Donc \(A\croix\) est un groupe.
\end{dem}

\begin{ex}
Déterminons \(\Z\croix\), \(\R\croix\), \(\C\croix\) et \(\paren{\R^\N}\croix\).

Dans \(\anneau{\Z}\) (anneau de neutres \(0\) et \(1\)), les seuls éléments inversibles sont \(1\) et \(-1\) donc \(\Z\croix=\accol{-1;1}\). Sa loi de groupe est :

\begin{center}
\begin{tikzpicture}
\matrix (mymatrix) [matrix of nodes, nodes={draw, minimum size=6mm, outer sep=0pt}, column sep=-\pgflinewidth, row sep=-\pgflinewidth]
    {  & 1 & -1\\
     1 & 1 & -1\\
	 -1 & -1 & 1\\};
\draw[->, shorten <=1mm, shorten >=1mm, looseness=1.2]
    (mymatrix-2-1.north west)to[out=90, in=180]node[below right=-3pt]{\(\times\)}(mymatrix-1-2.north west);
\end{tikzpicture}
\end{center}

Dans \(\anneau{\Q}\) (anneau de neutres \(0\) et \(1\)), tous les éléments sont inversibles sauf \(0\) donc \(\Q\croix=\Q\excluant\accol{0}\).

Idem dans \(\anneau{\R}\) et \(\anneau{\C}\) : \(\R\croix=\R\excluant\accol{0}\) et \(\C\croix=\C\excluant\accol{0}\).

Dans \(\anneau{\R^\N}\) (anneau de neutres \(\paren{0}_{n\in\N}\) et \(\paren{1}_{n\in\N}\)), tous les éléments sont inversibles sauf les suites dont au moins un terme est nul donc \(\paren{\R^\N}\croix=\paren{\Rs}^\N\).
\end{ex}

\subsection{Sous-anneaux}

\begin{defi}[Sous-anneau]
Soit \(\anneau{A}\) un anneau. On note \(0\) et \(1\) ses neutres.

Un sous-anneau de \(\anneau{A}\) (ou, par abus, de \(A\)) est une partie \(B\) de \(A\) telle que :

\begin{enumerate}[series=sousanneau]
\item Les éléments neutres \(0\) et \(1\) appartiennent à \(B\). \\

\item La partie \(B\) est stable par somme : \[\quantifs{\forall b_1,b_2\in B}b_1+b_2\in B.\]

\item La partie \(B\) est stable par produit : \[\quantifs{\forall b_1,b_2\in B}b_1\times b_2\in B.\]

\item La partie \(B\) est stable par passage à l'opposé : \[\quantifs{\forall b\in B}-b\in B.\]
\end{enumerate}
\end{defi}

\begin{prop}
Tout sous-anneau d'un anneau \(\anneau{A}\) d'éléments neutres \(0\) et \(1\) est naturellement un anneau dont les lois sont celles induites par celles de \(A\).
\end{prop}

\begin{dem}
Soit \(B\) un sous-anneau de \(A\).

Selon (2) et (3), \(B\) est une partie de \(A\) stable par \(+\) et \(\times\) donc \(+\) et \(\times\) induisent des lois \(\oplus\) et \(\otimes\) sur \(B\).

Montrons que \(\groupe{B}[\oplus]\) est un groupe abélien.

On remarque que \(\groupe{B}[\oplus]\) est un sous-groupe de \(\groupe{A}\) selon (1), (2) et (4) donc \(\groupe{B}[\oplus]\) est un groupe abélien.

Montrons que \(\otimes\) est associative et possède un élément neutre :

\(\otimes\) est associative car \(\times\) est associative et \(\otimes\) possède un neutre car \(1\in B\) et \(\quantifs{\forall b\in B}\begin{dcases}1\otimes b=1\times b=b \\ b\otimes1=b\times1=b\end{dcases}\)

\(\otimes\) est distributive par rapport à \(\oplus\) car \(\times\) est distributive par rapport à \(+\).

Donc \(\anneau{B}[\oplus][\otimes]\) est un anneau.
\end{dem}

\begin{prop}
Soient \(\anneau{A}\) d'éléments neutres \(0\) et \(1\) et \(B\) une partie de \(A\).

Alors \(B\) est un sous-anneau de \(A\) si, et seulement si, les conditions suivantes sont vérifiées :

\begin{enumerate}[resume=sousanneau]
\item \(1\in B\) \\

\item \(\quantifs{\forall b_1,b_2\in B}b_1-b_2\in B\) \\

\item \(\quantifs{\forall b_1,b_2\in B}b_1\times b_2\in B\) \\
\end{enumerate}
\end{prop}

\begin{dem}
\impdir

Supposons (1), (2), (3) et (4).

On a clairement (5) et (7) selon (1) et (3).

Montrons (6) :

Soient \(b_1,b_2\in B\). On a \(-b_2\in B\) selon (4) puis \(b_1-b_2=b_1+\paren{-b_2}\in B\) selon (2).

D'où (6).

\imprec

Supposons (5), (6) et (7).

On a clairement (3) selon (7).

On a \(1\in B\) selon (5) donc \(1-1\in B\) selon (6), \cad \(0\in B\). D'où (1).

Montrons (4) :

Soit \(b\in B\). On a \(0-b\in B\) selon (6) donc \(-b\in B\). D'où (4).

Montrons (2) :

Soient \(b_1,b_2\in B\). On a \(-b_2\in B\) selon (4). Donc \(b_1-\paren{-b_2}\in B\) selon (6). Donc \(b_1+b_2\in B\). D'où (2).
\end{dem}

\begin{ex}
On vérifie facilement que \(\Z\) est un sous-anneau de \(\Q\) et de \(\R\), que \(\Q\) est un sous-anneau de \(\R\), que \(\R\) est un sous-anneau de \(\C\), que \(\R^\N\) est un sous-anneau de \(\C^\N\), ...
\end{ex}

\subsection{Morphismes d'anneaux}

\begin{defi}[Morphisme d'anneaux]
Soient \(\anneau{A_1}[+_1][\times_1]\) et \(\anneau{A_2}[+_2][\times_2]\) deux anneaux d'éléments neutres respectifs \(0_1\) et \(1_1\) et \(0_2\) et \(1_2\).

On appelle morphisme d'anneaux de \(\anneau{A_1}[+_1][\times_1]\) vers \(\anneau{A_2}[+_2][\times_2]\) (ou, par abus, de \(A_1\) vers \(A_2\)) toute application \(\phi:A_1\to A_2\) telle que les conditions suivantes sont vérifiées :

\begin{enumerate}
\item \(\phi\paren{1_1}=1_2\) \\

\item \(\quantifs{\forall a,a\prim\in A_1}\phi\paren{a+_1a\prim}=\phi\paren{a}+_2\phi\paren{a\prim}\) \\

\item \(\quantifs{\forall a,a\prim\in A_1}\phi\paren{a\times_1a\prim}=\phi\paren{a}\times_2\phi\paren{a\prim}\) \\
\end{enumerate}
\end{defi}

\begin{prop}
Soient \(\anneau{A_1}[+_1][\times_1]\) et \(\anneau{A_2}[+_2][\times_2]\) deux anneaux d'éléments neutres pour l'addition respectifs \(0_1\) et \(0_2\).

Soit \(\phi:A_1\to A_2\) un morphisme d'anneaux.

Alors on a : \[\phi\paren{0_1}=0_2\] et \[\quantifs{\forall a\in A_1}\phi\paren{-a}=-\phi\paren{a}.\]
\end{prop}

\begin{dem}
Montrons que \(\phi\paren{0_1}=0_2\). Comme \(\phi\) est un morphisme d'anneaux de \(A_1\) vers \(A_2\), c'est un morphisme de groupes de \(\groupe{A_1}[+_1]\) vers \(\groupe{A_2}[+_2]\). On a alors \(\phi\paren{0_1}=0_2\) (\cf \thref{prop:neutreEtInverseParUnMorphismeDeGroupes}).

De même, on a \(\quantifs{\forall a\in A_1}\phi\paren{-a}=-\phi\paren{a}\) car \(\phi\) est un morphisme de groupes de \(\groupe{A_1}[+_1]\) vers \(\groupe{A_2}[+_2]\) (\cf \thref{prop:neutreEtInverseParUnMorphismeDeGroupes}).
\end{dem}

\begin{prop}
Soit \(\anneau{A}\) un anneau dont on note l'élément neutre pour la multiplication \(1\) et \(B\) un sous-anneau de \(A\).

L'application \[\fonction{i}{B}{A}{b}{b}\] est un morphisme d'anneaux.
\end{prop}

\begin{dem}
On a \(i\paren{1}=1\) et \[\quantifs{\forall b_1,b_2\in B}\begin{dcases}i\paren{b_1+b_2}=b_1+b_2=i\paren{b_1}+i\paren{b_2} \\ i\paren{b_1b_2}=b_1b_2=i\paren{b_1}i\paren{b_2}\end{dcases}\]
\end{dem}

\begin{ex}
Soient \(\anneau{A_1}[+_1][\times_1]\) et \(\anneau{A_2}[+_2][\times_2]\) deux anneaux.

Les projections \(p_1:A_1\times A_2\to A_1\) et \(p_2:A_1\times A_2\to A_2\) sont des morphismes d'anneaux.
\end{ex}

\begin{ex}
Soit \(I\) un ensemble, \(J\) une partie de \(I\) et \(A\) un anneau.

Alors l'application de restriction \[\fonctionlambda{\F{I}{A}}{\F{J}{A}}{f}{\restr{f}{J}}\] est un morphisme d'anneaux.
\end{ex}

\begin{prop}[Composition de morphismes d'anneaux]
Soient \(A_1\), \(A_2\) et \(A_3\) des anneaux et \(\phi:A_1\to A_2\) et \(\psi:A_2\to A_3\) des morphismes d'anneaux.

Alors \[\psi\rond\phi:A_1\to A_3\] est un morphisme d'anneaux.
\end{prop}

\begin{dem}
On note \(1_1\), \(1_2\) et \(1_3\) les neutres respectifs de \(A_1\), \(A_2\) et \(A_3\) pour la multiplication et \(+_i\) et \(\times_i\) les lois de \(A_i\) pour tout \(i\in\interventierii{1}{3}\).

On a \(\psi\rond\phi\paren{1_1}=\psi\paren{\phi\paren{1_1}}=\psi\paren{1_2}=1_3\).

Soient \(a,b\in A_1\).

On a : \[\begin{dcases}\psi\rond\phi\paren{a+_1b}=\psi\paren{\phi\paren{a+_1b}}=\psi\paren{\phi\paren{a}+_2\phi\paren{b}}=\psi\paren{\phi\paren{a}}+_3\psi\paren{\phi\paren{b}} \\ \psi\rond\phi\paren{a\times_1b}=\psi\paren{\phi\paren{a}\times_2\phi\paren{b}}=\psi\paren{\phi\paren{a}}\times_3\psi\paren{\phi\paren{b}}\end{dcases}\]

Donc \(\psi\rond\phi\) est un morphisme d'anneaux de \(A_1\) vers \(A_3\).
\end{dem}

\subsection{Isomorphismes, endomorphismes, automorphismes}

\begin{defi}[Isomorphisme]
Soient \(A_1\) et \(A_2\) deux anneaux.

Un isomorphisme d'anneaux de \(A_1\) vers \(A_2\) est un morphisme d'anneaux \(\phi:A_1\to A_2\) bijectif.
\end{defi}

\begin{oubli}
Un composée d'isomorphismes d'anneaux est un isomorphisme d'anneaux.
\end{oubli}

\begin{theo}
Soit un isomorphisme d'anneaux \(\phi:A_1\to A_2\). Alors la bijection réciproque \(\phi\inv:A_2\to A_1\) est un isomorphisme d'anneaux.
\end{theo}

\begin{dem}
On note \(+_1\) et \(\times_1\) les lois de \(A_1\) et \(+_2\) et \(\times_2\) les lois de \(A_2\). On note \(1_1\) et \(1_2\) les neutres de \(\times_1\) et \(\times_2\).

On sait que \(\phi\inv\) est une bijection de \(A_2\) vers \(A_1\).

Montrons que \(\phi\inv\) est un morphisme d'anneaux.

On a \(\phi\paren{1_1}=1_2\) donc \(\phi\inv\paren{1_2}=1_1\).

Soient \(y,y\prim\in A_2\). On a \[\phi\paren{\phi\inv\paren{y}+_1\phi\inv\paren{y\prim}}=\phi\paren{\phi\inv\paren{y}}+_2\phi\paren{\phi\inv\paren{y\prim}}=y+_2y\prim\] donc \(\phi\inv\paren{y}+_1\phi\inv\paren{y\prim}\) est l'antécédent de \(y+_2y\prim\) par \(\phi\), \cad \[\phi\inv\paren{y}+_1\phi\inv\paren{y\prim}=\phi\inv\paren{y+_2y\prim}.\]

On montre de même \(\phi\inv\paren{y}\times_1\phi\inv\paren{y\prim}=\phi\inv\paren{y\times_2y\prim}\).

Donc \(\phi\inv\) est un morphisme d'anneaux.

Donc \(\phi\inv\) est un isomorphisme d'anneaux.
\end{dem}

\begin{defi}[Endomorphisme]
Soit \(A\) un anneau.

Un endomorphisme d'anneau de \(A\) est un morphisme d'anneaux \(\phi:A\to A\).
\end{defi}

\begin{defi}[Automorphisme]
Soit \(A\) un anneau.

Un automorphisme de \(A\) est un isomorphisme d'anneaux \(\phi:A\to A\), \cad un endomorphisme de \(A\) bijectif.

L'ensemble des automorphismes d'anneau de \(A\) est noté \(\Aut{A}\).
\end{defi}

\begin{ex}
Soit \(A\) un anneau.

Alors \(\groupe{\Aut{A}}[\rond]\) est un groupe, appelé le groupe des automorphismes de \(A\).
\end{ex}

\begin{dem}
\note{EXERCICE}
\end{dem}

\subsection{Anneaux intègres}

\begin{defi}[Anneau intègre]
Soit \(\anneau{A}\) un anneau. On note \(0\) son élément neutre pour \(+\).

On dit que \(\anneau{A}\) est un anneau intègre s'il est non-nul, commutatif et \guillemets{sans diviseurs de \(0\)} : \[\begin{dcases}0\not=1 \\ \quantifs{\forall a,b\in A}ab=ba \\ \quantifs{\forall a,b\in A}ab=0\ssi\orenv{a=0 \\ b=0}\end{dcases}\]
\end{defi}

\begin{ex}
Si \(A=\Z^2\) alors \(\paren{1,0}\times\paren{0,1}=\paren{0,0}\) donc \(A\) est un anneau non-intègre.

Si \(A=\F{\R}{\R}\) alors \(\ind{\Rps}\times\ind{\Rms}=0\) donc \(A\) est un anneau non-intègre.

Les anneaux usuels \(\Z\), \(\Q\), \(\R\) et \(\C\) sont intègres.
\end{ex}

\begin{rem}
Un anneau est intègre si, et seulement si, \(\paren{S}:\begin{dcases}\text{il est non-nul} \\ \text{il est commutatif} \\ \text{tout élément non-nul est régulier pour le produit}\end{dcases}\)
\end{rem}

\begin{dem}
\impdir

Soit \(\anneau{A}\) un anneau intègre.

Soit \(a\in A\excluant\accol{0}\). Montrons que \(a\) est régulier pour \(\times\).

Soient \(b,c\in A\) tels que \(ab=ac\).

On a : \[\begin{WithArrows}
ab-ac&=0 \\
a\paren{b-c}&=0 \Arrow{car \(A\) est intègre et \(a\not=0\)} \\
b-c&=0 \\
b&=c
\end{WithArrows}\]

\imprec

Soit \(A\) un anneau tel que \(\paren{S}\) soit vérifié.

Montrons que \(A\) est intègre.

Soient \(a,b\in A\) tels que \(ab=0\). Montrons que \(a=0\) ou \(b=0\).

Si \(a=0\) : OK.

Si \(a\not=0\) alors \(ab=a\times0\) donc \(b=0\) car \(a\) est régulier et non-nul.

Donc \(A\) est intègre.
\end{dem}

\section{Corps}

\begin{defi}[Corps]
Un corps est un anneau non-nul, commutatif et dont tout élément non-nul est inversible, \cad un anneau \(\anneau{K}\) tel que les conditions suivantes soient satisfaites :

\begin{enumerate}
\item \(K\) n'est pas un singleton, \cad \(0\not=1\). \\

\item \(K\) est un anneau commutatif. \\

\item \(K\croix=K\excluant\accol{0}\).
\end{enumerate}

On dit aussi que \(K\) est muni d'une structure de corps, ou, par abus, que \(K\) est un corps.
\end{defi}

\begin{prop}
Si \(K\) est un corps alors \(K\excluant\accol{0}\) est un groupe commutatif.
\end{prop}

\begin{dem}
On sait que le groupe des inversibles d'un anneau est un groupe pour la loi \(\times\) donc \(\groupe{K\croix}[\times]\) est un groupe (\cf \thref{defprop:groupeDesInversiblesD'UnAnneau}).

De plus, on a \(K\croix=K\excluant\accol{0}\) et \(\times\) est commutative.

Donc \(\groupe{K\excluant\accol{0}}[\times]\) est commutatif.
\end{dem}

\begin{prop}
Tout corps est un anneau intègre.
\end{prop}

\begin{dem}
C'est clair car le fait que tous les éléments non-nuls soient inversibles implique le fait que tous les éléments non-nuls soient réguliers.
\end{dem}

\begin{ex}
On vérifie facilement que \(\corps{\Q}\), \(\corps{\R}\) et \(\corps{\C}\) sont des corps.
\end{ex}

\begin{ex}
Soient \(\corps{K_1}[+_1][\times_1]\) et \(\corps{K_2}[+_2][\times_2]\) deux corps (donc deux anneaux dont on note \(0_1\) et \(1_1\) et \(0_2\) et \(1_2\) les neutres).

On a vu que \(K_1\times K_2\) est naturellement muni d'une structure d'anneau (\cf \thref{ex:produitD'anneauxEstUnAnneau}).

Cependant, ce n'est pas un corps.

En effet, il n'est pas intègre : \(\paren{1_1,0_2}\paren{0_1,1_2}=\paren{0_1,0_2}\).
\end{ex}

\begin{ex}
Soit \(I\) un ensemble et \(K\) un corps.

On a vu que \(\F{I}{K}\) est naturellement muni d'une structure d'anneau (\cf \thref{ex:fonctionsD'unEnsembleDansUnAnneauEstUnAnneau}).

Si \(I=\ensvide\) alors \(\F{I}{K}\) n'est pas un corps car c'est l'anneau nul.

Si \(I=\accol{x}\) est un singleton alors \(\fonction{\phi}{K}{\F{\accol{x}}{K}}{\lambda}{\fonctionlambda{\accol{x}}{K}{x}{\lambda}}\) est un isomorphisme d'anneaux. Or \(K\) est un corps donc \(\F{I}{K}\) est un corps.

Si \(I\) contient au moins deux élément \(x\) et \(y\) avec \(x\not=y\) alors \(\F{I}{K}\) est un anneau non-intègre car \(\fonction{f}{I}{K}{z}{\begin{dcases}1&\text{si }z=x \\ 0&\text{sinon}\end{dcases}}\) et \(\fonction{g}{I}{K}{z}{\begin{dcases}1&\text{si }z=y \\ 0&\text{sinon}\end{dcases}}\) vérifient \(f\not=0\), \(g\not=0\) et \(fg=0\).

Donc \(\F{I}{K}\) n'est pas un corps.
\end{ex}