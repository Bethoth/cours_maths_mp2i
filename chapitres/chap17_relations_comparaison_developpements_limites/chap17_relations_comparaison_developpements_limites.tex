\chapter{Relations de comparaison, développements limités}

\minitoc

Dans tout le chapitre, on pose : \(\K=\R\) ou \(\C\).

\section{Relations de comparaison : cas des suites}

\subsection{Relation de domination \(\mathscr{O}\)}

\begin{defprop}
Soient \(\paren{u_n}_n,\paren{v_n}_n\in\K^\N\).

Les propositions suivantes sont équivalentes :

\begin{enumerate}
    \item \(\quantifs{\exists M\in\Rp;\exists N\in\N;\forall n\in\interventierie{N}{\pinf}}\abs{u_n}\leq M\abs{v_n}\) \\
    \item Il existe une suite \(\paren{\lambda_n}_n\in\K^\N\) telle que : \[\paren{\lambda_n}_n\text{ est bornée}\qquad\text{et}\qquad\quantifs{\exists N\in\N;\forall n\in\interventierie{N}{\pinf}}u_n=\lambda v_n.\]
\end{enumerate}

Lorsqu'elles sont vérifiées, on dit que \[\paren{v_n}_n\text{ domine }\paren{u_n}_n\] ou que \[v_n\text{ domine }u_n\text{ quand }n\text{ tend vers }\pinf\] et on note : \[u_n=\O{v_n}\text{ quand }n\text{ tend vers }\pinf\] ou : \[u_n\egqd{n\to\pinf}\O{v_n}.\]
\end{defprop}

\begin{prop}
Soient \(\paren{u_n}_n,\paren{v_n}_n\in\K^\N\).

Si les termes de \(\paren{v_n}_n\) sont non-nuls à partir d'un certain rang \(N\in\N\) : \[\quantifs{\forall n\in\interventierie{N}{\pinf}}v_n\not=0,\] alors : \[u_n\egqd{n\to\pinf}\O{v_n}\ssi\text{la suite }\paren{\dfrac{u_n}{v_n}}_{n\geq N}\text{ est bornée}.\]

Si les termes de \(\paren{v_n}_n\) sont tous non-nuls : \[\quantifs{\forall n\in\N}v_n\not=0,\] alors : \[u_n\egqd{n\to\pinf}\O{v_n}\ssi\text{la suite }\paren{\dfrac{u_n}{v_n}}_{n\in\N}\text{ est bornée}.\]
\end{prop}

\begin{ex}
Soient \(\alpha,\beta\in\R\). On a : \[n^{\alpha}\egqd{n\to\pinf}\O{n^\beta}\ssi\alpha\leq\beta.\]

Soient \(a,b\in\Rps\). On a : \[a^n\egqd{n\to\pinf}\O{b^n}\ssi a\leq b.\]

Soit \(\paren{u_n}_n\in\K^\N\). On a : \[u_n\egqd{n\to\pinf}\O{0}\ssi\paren{u_n}_n\text{ est nulle à partir d'un certain rang}\] et : \[u_n\egqd{n\to\pinf}\O{1}\ssi\paren{u_n}_n\text{ est bornée}.\]
\end{ex}

\subsection{Relation de négligeabilité \(o\)}

\begin{defprop}
Soient \(\paren{u_n}_n,\paren{v_n}_n\in\K^\N\).

Les propositions suivantes sont équivalentes :

\begin{enumerate}
    \item \(\quantifs{\forall\epsilon\in\Rps;\exists N\in\N;\forall n\in\interventierie{N}{\pinf}}\abs{u_n}\leq\epsilon\abs{v_n}\) \\
    \item Il existe une suite \(\paren{\epsilon_n}_n\in\K^\N\) telle que : \[\lim_{n\to\pinf}\epsilon_n=0\qquad\text{et}\qquad\quantifs{\exists N\in\N;\forall n\in\interventierie{N}{\pinf}}u_n=\epsilon_nv_n.\]
\end{enumerate}

Lorsqu'elles sont vérifiées, on dit que \[\paren{u_n}_n\text{ est négligeable devant }\paren{v_n}_n\] ou que \[u_n\text{ est négligeable devant }v_n\text{ quand }n\text{ tend vers }\pinf\] et on note : \[u_n=\o{v_n}\text{ quand }n\text{ tend vers }\pinf\] ou : \[u_n\egqd{n\to\pinf}\o{v_n}.\]
\end{defprop}

\begin{prop}
Soient \(\paren{u_n}_n,\paren{v_n}_n\in\K^\N\).

Si les termes de \(\paren{v_n}_n\) sont non-nuls à partir d'un certain rang : \[\quantifs{\exists N\in\N;\forall n\in\interventierie{N}{\pinf}}v_n\not=0,\] alors : \[u_n\egqd{n\to\pinf}\o{v_n}\ssi\lim_{n\to\pinf}\dfrac{u_n}{v_n}=0.\]
\end{prop}

\begin{ex}
Soient \(\alpha,\beta\in\R\). On a : \[n^\alpha\egqd{n\to\pinf}\o{n^\beta}\ssi\alpha<\beta.\]

Soient \(a,b\in\Rps\). On a : \[a^n\egqd{n\to\pinf}\o{b^n}\ssi a<b.\]

Soit \(\paren{u_n}_n\in\K^\N\). On a : \[u_n\egqd{n\to\pinf}\o{0}\ssi\paren{u_n}_n\text{ est nulle à partir d'un certain rang}\] et : \[u_n\egqd{n\to\pinf}\o{1}\ssi\lim_{n\to\pinf}u_n=0.\]
\end{ex}