\chapter{Relations de comparaison, développements limités}

\minitoc

Dans tout le chapitre, on pose : \(\K=\R\) ou \(\C\).

\section{Relations de comparaison : cas des suites}

\subsection{Relation de domination \(\mathscr{O}\)}

\begin{defprop}
Soient \(\paren{u_n}_n,\paren{v_n}_n\in\K^\N\).

Les propositions suivantes sont équivalentes :

\begin{enumerate}
    \item \(\quantifs{\exists M\in\Rp;\exists N\in\N;\forall n\in\interventierie{N}{\pinf}}\abs{u_n}\leq M\abs{v_n}\) \\
    \item Il existe une suite \(\paren{\lambda_n}_n\in\K^\N\) telle que : \[\paren{\lambda_n}_n\text{ est bornée}\qquad\text{et}\qquad\quantifs{\exists N\in\N;\forall n\in\interventierie{N}{\pinf}}u_n=\lambda v_n.\]
\end{enumerate}

Lorsqu'elles sont vérifiées, on dit que \[\paren{v_n}_n\text{ domine }\paren{u_n}_n\] ou que \[v_n\text{ domine }u_n\text{ quand }n\text{ tend vers }\pinf\] et on note : \[u_n=\O{v_n}\text{ quand }n\text{ tend vers }\pinf\] ou : \[u_n\egqd{n\to\pinf}\O{v_n}.\]
\end{defprop}

\begin{prop}
Soient \(\paren{u_n}_n,\paren{v_n}_n\in\K^\N\).

Si les termes de \(\paren{v_n}_n\) sont non-nuls à partir d'un certain rang \(N\in\N\) : \[\quantifs{\forall n\in\interventierie{N}{\pinf}}v_n\not=0,\] alors : \[u_n\egqd{n\to\pinf}\O{v_n}\ssi\text{la suite }\paren{\dfrac{u_n}{v_n}}_{n\geq N}\text{ est bornée}.\]

Si les termes de \(\paren{v_n}_n\) sont tous non-nuls : \[\quantifs{\forall n\in\N}v_n\not=0,\] alors : \[u_n\egqd{n\to\pinf}\O{v_n}\ssi\text{la suite }\paren{\dfrac{u_n}{v_n}}_{n\in\N}\text{ est bornée}.\]
\end{prop}

\begin{ex}
Soient \(\alpha,\beta\in\R\). On a : \[n^{\alpha}\egqd{n\to\pinf}\O{n^\beta}\ssi\alpha\leq\beta.\]

Soient \(a,b\in\Rps\). On a : \[a^n\egqd{n\to\pinf}\O{b^n}\ssi a\leq b.\]

Soit \(\paren{u_n}_n\in\K^\N\). On a : \[u_n\egqd{n\to\pinf}\O{0}\ssi\paren{u_n}_n\text{ est nulle à partir d'un certain rang}\] et : \[u_n\egqd{n\to\pinf}\O{1}\ssi\paren{u_n}_n\text{ est bornée}.\]
\end{ex}

\subsection{Relation de négligeabilité \(o\)}

\begin{defprop}
Soient \(\paren{u_n}_n,\paren{v_n}_n\in\K^\N\).

Les propositions suivantes sont équivalentes :

\begin{enumerate}
    \item \(\quantifs{\forall\epsilon\in\Rps;\exists N\in\N;\forall n\in\interventierie{N}{\pinf}}\abs{u_n}\leq\epsilon\abs{v_n}\) \\
    \item Il existe une suite \(\paren{\epsilon_n}_n\in\K^\N\) telle que : \[\lim_{n\to\pinf}\epsilon_n=0\qquad\text{et}\qquad\quantifs{\exists N\in\N;\forall n\in\interventierie{N}{\pinf}}u_n=\epsilon_nv_n.\]
\end{enumerate}

Lorsqu'elles sont vérifiées, on dit que \[\paren{u_n}_n\text{ est négligeable devant }\paren{v_n}_n\] ou que \[u_n\text{ est négligeable devant }v_n\text{ quand }n\text{ tend vers }\pinf\] et on note : \[u_n=\o{v_n}\text{ quand }n\text{ tend vers }\pinf\] ou : \[u_n\egqd{n\to\pinf}\o{v_n}.\]
\end{defprop}

\begin{prop}
Soient \(\paren{u_n}_n,\paren{v_n}_n\in\K^\N\).

Si les termes de \(\paren{v_n}_n\) sont non-nuls à partir d'un certain rang : \[\quantifs{\exists N\in\N;\forall n\in\interventierie{N}{\pinf}}v_n\not=0,\] alors : \[u_n\egqd{n\to\pinf}\o{v_n}\ssi\lim_{n\to\pinf}\dfrac{u_n}{v_n}=0.\]
\end{prop}

\begin{ex}
Soient \(\alpha,\beta\in\R\). On a : \[n^\alpha\egqd{n\to\pinf}\o{n^\beta}\ssi\alpha<\beta.\]

Soient \(a,b\in\Rps\). On a : \[a^n\egqd{n\to\pinf}\o{b^n}\ssi a<b.\]

Soit \(\paren{u_n}_n\in\K^\N\). On a : \[u_n\egqd{n\to\pinf}\o{0}\ssi\paren{u_n}_n\text{ est nulle à partir d'un certain rang}\] et : \[u_n\egqd{n\to\pinf}\o{1}\ssi\lim_{n\to\pinf}u_n=0.\]
\end{ex}

\subsection{Propriétés de \(\mathscr{O}\) et \(o\)}

\begin{prop}
Soient \(\paren{u_n}_n,\paren{v_n}_n\in\K^\N\).

On a : \[u_n\egqd{n\to\pinf}\o{v_n}\imp u_n\egqd{n\to\pinf}\O{v_n}.\]
\end{prop}

\begin{prop}
Soient \(\paren{u_n}_n,\paren{v_n}_n\in\K^\N\) et \(l,l\prim\in\intervii{0}{\pinf}\) tels que : \[\lim_{n\to\pinf}\abs{u_n}=l\qquad\text{et}\qquad\lim_{n\to\pinf}\abs{v_n}=l\prim.\]

On a, quand \(n\) tend vers \(\pinf\) :

\begin{enumerate}
    \item Si \(l=0\) et \(l\prim\not=0\) alors \(u_n=\o{v_n}\). \\
    \item Si \(l\in\intervie{0}{\pinf}\) et \(l\prim=\pinf\) alors \(u_n=\o{v_n}\). \\
    \item Si \(l,l\prim\in\intervee{0}{\pinf}\) alors \(u_n=\O{v_n}\). \\
    \item Si \(l=l\prim=0\) ou \(l=l\prim=\pinf\) alors on ne peut rien dire.
\end{enumerate}
\end{prop}

\begin{prop}[Transitivités]
Soient \(\paren{u_n}_n,\paren{v_n}_n,\paren{w_n}_n\in\K^\N\).

On a, quand \(n\) tend vers \(\pinf\) :

\begin{enumerate}
    \item Si \(\begin{dcases}
        u_n=\O{v_n} \\
        v_n=\O{w_n}
    \end{dcases}\) alors \(u_n=\O{w_n}\). \\\\ Autrement dit : la relation de domination est transitive. \\
    \item Si \(\begin{dcases}
        u_n=\O{v_n} \\
        v_n=\o{w_n}
    \end{dcases}\) ou \(\begin{dcases}
        u_n=\o{v_n} \\
        v_n=\O{w_n}
    \end{dcases}\) alors \(u_n=\o{w_n}\). \\
    \item En particulier, la relation de négligeabilité est transitive.
\end{enumerate}
\end{prop}

\begin{prop}[Sommes]
Soient \(\paren{u_n}_n,\paren{v_n}_n,\paren{w_n}_n\in\K^\N\).

On a, quand \(n\) tend vers \(\pinf\) :

\begin{enumerate}
    \item Si \(\begin{dcases}
        u_n=\O{w_n} \\
        v_n=\O{w_n}
    \end{dcases}\) alors \(u_n+v_n=\O{w_n}\). \\
    \item Si \(\begin{dcases}
        u_n=\o{w_n} \\
        v_n=\o{w_n}
    \end{dcases}\) alors \(u_n+v_n=\o{w_n}\).
\end{enumerate}
\end{prop}

\begin{prop}[Produits]
Soient \(\paren{a_n}_n,\paren{b_n}_n,\paren{c_n}_n,\paren{d_n}_n\in\K^\N\).

On a, quand \(n\) tend vers \(\pinf\) :

\begin{enumerate}
    \item Si \(\begin{dcases}
        a_n=\O{b_n} \\
        c_n=\O{d_n}
    \end{dcases}\) alors \(a_nc_n=\O{b_nd_n}\). \\
    \item En particulier : si \(a_n=\O{b_n}\) alors \(a_nc_n=\O{b_nc_n}\). \\
    \item Si \(\begin{dcases}
        a_n=\o{b_n} \\
        c_n=\O{d_n}
    \end{dcases}\) alors \(a_nc_n=\o{b_nd_n}\). \\
    \item En particulier : si \(a_n=\o{b_n}\) alors \(a_nc_n=\o{b_nc_n}\).
\end{enumerate}
\end{prop}

\begin{prop}[Puissances]
Soient \(\paren{u_n}_n,\paren{v_n}_n\in\paren{\Rps}^\N\) et \(\alpha\in\R\).

On a, quand \(n\) tend vers \(\pinf\) :

\begin{enumerate}
    \item Si \(\begin{dcases}
        u_n=\O{v_n} \\
        \alpha\geq0
    \end{dcases}\) alors \(u_n^\alpha=\O{v_n^\alpha}\). \\
    \item Si \(\begin{dcases}
        u_n=\O{v_n} \\
        \alpha\leq0
    \end{dcases}\) alors \(v_n^\alpha=\O{u_n^\alpha}\). \\
    \item Si \(\begin{dcases}
        u_n=\o{v_n} \\
        \alpha>0
    \end{dcases}\) alors \(u_n^\alpha=\o{v_n^\alpha}\). \\
    \item Si \(\begin{dcases}
        u_n=\o{v_n} \\
        \alpha<0
    \end{dcases}\) alors \(v_n^\alpha=\o{u_n^\alpha}\).
\end{enumerate}

On retient en particulier : \[u_n=\O{v_n}\imp\dfrac{1}{v_n}=\O{\dfrac{1}{u_n}}\qquad\text{et}\qquad u_n=\o{v_n}\imp\dfrac{1}{v_n}=\o{\dfrac{1}{u_n}}.\]

La même proposition est vraie en prenant \(\paren{u_n}_n,\paren{v_n}_n\in\C^\N\) et \(\alpha\in\N\) ou en prenant \(\paren{u_n}_n,\paren{v_n}_n\in\paren{\Cs}^\N\) et \(\alpha\in\Z\).
\end{prop}

\begin{prop}[Suites extraites]
Soient \(\paren{u_n}_n,\paren{v_n}_n\in\C^\N\) et \(\phi:\N\to\N\) une fonction strictement croissante.

On a, quand \(n\) tend vers \(\pinf\) : \[u_n=\O{v_n}\imp u_{\phi\paren{n}}=\O{v_{\phi\paren{n}}}\qquad\text{et}\qquad u_n=\o{v_n}\imp u_{\phi\paren{n}}=\o{v_{\phi\paren{n}}}.\]
\end{prop}