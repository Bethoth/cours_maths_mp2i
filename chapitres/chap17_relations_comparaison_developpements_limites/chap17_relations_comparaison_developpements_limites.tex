\chapter{Relations de comparaison, développements limités}

\minitoc

Dans tout le chapitre, on pose : \(\K=\R\) ou \(\C\).

\section{Relations de comparaison : cas des suites}

\subsection{Relation de domination \(\mathscr{O}\)}

\begin{defprop}
Soient \(\paren{u_n}_n,\paren{v_n}_n\in\K^\N\).

Les propositions suivantes sont équivalentes :

\begin{enumerate}
    \item \(\quantifs{\exists M\in\Rp;\exists N\in\N;\forall n\in\interventierie{N}{\pinf}}\abs{u_n}\leq M\abs{v_n}\) \\
    \item Il existe une suite \(\paren{\lambda_n}_n\in\K^\N\) telle que : \[\paren{\lambda_n}_n\text{ est bornée}\qquad\text{et}\qquad\quantifs{\exists N\in\N;\forall n\in\interventierie{N}{\pinf}}u_n=\lambda v_n.\]
\end{enumerate}

Lorsqu'elles sont vérifiées, on dit que \[\paren{v_n}_n\text{ domine }\paren{u_n}_n\] ou que \[v_n\text{ domine }u_n\text{ quand }n\text{ tend vers }\pinf\] et on note : \[u_n=\O{v_n}\text{ quand }n\text{ tend vers }\pinf\] ou : \[u_n\egqd{n\to\pinf}\O{v_n}.\]
\end{defprop}

\begin{prop}
Soient \(\paren{u_n}_n,\paren{v_n}_n\in\K^\N\).

Si les termes de \(\paren{v_n}_n\) sont non-nuls à partir d'un certain rang \(N\in\N\) : \[\quantifs{\forall n\in\interventierie{N}{\pinf}}v_n\not=0,\] alors : \[u_n\egqd{n\to\pinf}\O{v_n}\ssi\text{la suite }\paren{\dfrac{u_n}{v_n}}_{n\geq N}\text{ est bornée}.\]

Si les termes de \(\paren{v_n}_n\) sont tous non-nuls : \[\quantifs{\forall n\in\N}v_n\not=0,\] alors : \[u_n\egqd{n\to\pinf}\O{v_n}\ssi\text{la suite }\paren{\dfrac{u_n}{v_n}}_{n\in\N}\text{ est bornée}.\]
\end{prop}

\begin{ex}
Soient \(\alpha,\beta\in\R\). On a : \[n^{\alpha}\egqd{n\to\pinf}\O{n^\beta}\ssi\alpha\leq\beta.\]

Soient \(a,b\in\Rps\). On a : \[a^n\egqd{n\to\pinf}\O{b^n}\ssi a\leq b.\]

Soit \(\paren{u_n}_n\in\K^\N\). On a : \[u_n\egqd{n\to\pinf}\O{0}\ssi\paren{u_n}_n\text{ est nulle à partir d'un certain rang}\] et : \[u_n\egqd{n\to\pinf}\O{1}\ssi\paren{u_n}_n\text{ est bornée}.\]
\end{ex}

\subsection{Relation de négligeabilité \(o\)}

\begin{defprop}
Soient \(\paren{u_n}_n,\paren{v_n}_n\in\K^\N\).

Les propositions suivantes sont équivalentes :

\begin{enumerate}
    \item \(\quantifs{\forall\epsilon\in\Rps;\exists N\in\N;\forall n\in\interventierie{N}{\pinf}}\abs{u_n}\leq\epsilon\abs{v_n}\) \\
    \item Il existe une suite \(\paren{\epsilon_n}_n\in\K^\N\) telle que : \[\lim_{n\to\pinf}\epsilon_n=0\qquad\text{et}\qquad\quantifs{\exists N\in\N;\forall n\in\interventierie{N}{\pinf}}u_n=\epsilon_nv_n.\]
\end{enumerate}

Lorsqu'elles sont vérifiées, on dit que \[\paren{u_n}_n\text{ est négligeable devant }\paren{v_n}_n\] ou que \[u_n\text{ est négligeable devant }v_n\text{ quand }n\text{ tend vers }\pinf\] et on note : \[u_n=\o{v_n}\text{ quand }n\text{ tend vers }\pinf\] ou : \[u_n\egqd{n\to\pinf}\o{v_n}.\]
\end{defprop}

\begin{prop}
Soient \(\paren{u_n}_n,\paren{v_n}_n\in\K^\N\).

Si les termes de \(\paren{v_n}_n\) sont non-nuls à partir d'un certain rang : \[\quantifs{\exists N\in\N;\forall n\in\interventierie{N}{\pinf}}v_n\not=0,\] alors : \[u_n\egqd{n\to\pinf}\o{v_n}\ssi\lim_{n\to\pinf}\dfrac{u_n}{v_n}=0.\]
\end{prop}

\begin{ex}
Soient \(\alpha,\beta\in\R\). On a : \[n^\alpha\egqd{n\to\pinf}\o{n^\beta}\ssi\alpha<\beta.\]

Soient \(a,b\in\Rps\). On a : \[a^n\egqd{n\to\pinf}\o{b^n}\ssi a<b.\]

Soit \(\paren{u_n}_n\in\K^\N\). On a : \[u_n\egqd{n\to\pinf}\o{0}\ssi\paren{u_n}_n\text{ est nulle à partir d'un certain rang}\] et : \[u_n\egqd{n\to\pinf}\o{1}\ssi\lim_{n\to\pinf}u_n=0.\]
\end{ex}

\subsection{Propriétés de \(\mathscr{O}\) et \(o\)}

\begin{prop}
Soient \(\paren{u_n}_n,\paren{v_n}_n\in\K^\N\).

On a : \[u_n\egqd{n\to\pinf}\o{v_n}\imp u_n\egqd{n\to\pinf}\O{v_n}.\]
\end{prop}

\begin{prop}
Soient \(\paren{u_n}_n,\paren{v_n}_n\in\K^\N\) et \(l,l\prim\in\intervii{0}{\pinf}\) tels que : \[\lim_{n\to\pinf}\abs{u_n}=l\qquad\text{et}\qquad\lim_{n\to\pinf}\abs{v_n}=l\prim.\]

On a, quand \(n\) tend vers \(\pinf\) :

\begin{enumerate}
    \item Si \(l=0\) et \(l\prim\not=0\) alors \(u_n=\o{v_n}\). \\
    \item Si \(l\in\intervie{0}{\pinf}\) et \(l\prim=\pinf\) alors \(u_n=\o{v_n}\). \\
    \item Si \(l,l\prim\in\intervee{0}{\pinf}\) alors \(u_n=\O{v_n}\). \\
    \item Si \(l=l\prim=0\) ou \(l=l\prim=\pinf\) alors on ne peut rien dire.
\end{enumerate}
\end{prop}

\begin{prop}[Transitivités]
Soient \(\paren{u_n}_n,\paren{v_n}_n,\paren{w_n}_n\in\K^\N\).

On a, quand \(n\) tend vers \(\pinf\) :

\begin{enumerate}
    \item Si \(\begin{dcases}
        u_n=\O{v_n} \\
        v_n=\O{w_n}
    \end{dcases}\) alors \(u_n=\O{w_n}\). \\\\ Autrement dit : la relation de domination est transitive. \\
    \item Si \(\begin{dcases}
        u_n=\O{v_n} \\
        v_n=\o{w_n}
    \end{dcases}\) ou \(\begin{dcases}
        u_n=\o{v_n} \\
        v_n=\O{w_n}
    \end{dcases}\) alors \(u_n=\o{w_n}\). \\
    \item En particulier, la relation de négligeabilité est transitive.
\end{enumerate}
\end{prop}

\begin{prop}[Sommes]
Soient \(\paren{u_n}_n,\paren{v_n}_n,\paren{w_n}_n\in\K^\N\).

On a, quand \(n\) tend vers \(\pinf\) :

\begin{enumerate}
    \item Si \(\begin{dcases}
        u_n=\O{w_n} \\
        v_n=\O{w_n}
    \end{dcases}\) alors \(u_n+v_n=\O{w_n}\). \\
    \item Si \(\begin{dcases}
        u_n=\o{w_n} \\
        v_n=\o{w_n}
    \end{dcases}\) alors \(u_n+v_n=\o{w_n}\).
\end{enumerate}
\end{prop}

\begin{prop}[Produits]
Soient \(\paren{a_n}_n,\paren{b_n}_n,\paren{c_n}_n,\paren{d_n}_n\in\K^\N\).

On a, quand \(n\) tend vers \(\pinf\) :

\begin{enumerate}
    \item Si \(\begin{dcases}
        a_n=\O{b_n} \\
        c_n=\O{d_n}
    \end{dcases}\) alors \(a_nc_n=\O{b_nd_n}\). \\
    \item En particulier : si \(a_n=\O{b_n}\) alors \(a_nc_n=\O{b_nc_n}\). \\
    \item Si \(\begin{dcases}
        a_n=\o{b_n} \\
        c_n=\O{d_n}
    \end{dcases}\) alors \(a_nc_n=\o{b_nd_n}\). \\
    \item En particulier : si \(a_n=\o{b_n}\) alors \(a_nc_n=\o{b_nc_n}\).
\end{enumerate}
\end{prop}

\begin{prop}[Puissances]
Soient \(\paren{u_n}_n,\paren{v_n}_n\in\paren{\Rps}^\N\) et \(\alpha\in\R\).

On a, quand \(n\) tend vers \(\pinf\) :

\begin{enumerate}
    \item Si \(\begin{dcases}
        u_n=\O{v_n} \\
        \alpha\geq0
    \end{dcases}\) alors \(u_n^\alpha=\O{v_n^\alpha}\). \\
    \item Si \(\begin{dcases}
        u_n=\O{v_n} \\
        \alpha\leq0
    \end{dcases}\) alors \(v_n^\alpha=\O{u_n^\alpha}\). \\
    \item Si \(\begin{dcases}
        u_n=\o{v_n} \\
        \alpha>0
    \end{dcases}\) alors \(u_n^\alpha=\o{v_n^\alpha}\). \\
    \item Si \(\begin{dcases}
        u_n=\o{v_n} \\
        \alpha<0
    \end{dcases}\) alors \(v_n^\alpha=\o{u_n^\alpha}\).
\end{enumerate}

On retient en particulier : \[u_n=\O{v_n}\imp\dfrac{1}{v_n}=\O{\dfrac{1}{u_n}}\qquad\text{et}\qquad u_n=\o{v_n}\imp\dfrac{1}{v_n}=\o{\dfrac{1}{u_n}}.\]

La même proposition est vraie en prenant \(\paren{u_n}_n,\paren{v_n}_n\in\C^\N\) et \(\alpha\in\N\) ou en prenant \(\paren{u_n}_n,\paren{v_n}_n\in\paren{\Cs}^\N\) et \(\alpha\in\Z\).
\end{prop}

\begin{prop}[Suites extraites]
Soient \(\paren{u_n}_n,\paren{v_n}_n\in\C^\N\) et \(\phi:\N\to\N\) une fonction strictement croissante.

On a, quand \(n\) tend vers \(\pinf\) : \[u_n=\O{v_n}\imp u_{\phi\paren{n}}=\O{v_{\phi\paren{n}}}\qquad\text{et}\qquad u_n=\o{v_n}\imp u_{\phi\paren{n}}=\o{v_{\phi\paren{n}}}.\]
\end{prop}

\subsection{Croissances comparées}

\begin{rem}[Culturelle]
Il existe d'autres notations que les \guillemets{notations de Landau} \(\mathscr{O}\), \(o\) et \(\sim\) qu'on utilise en mathématiques de CPGE.

Soient \(\paren{u_n}_n,\paren{v_n}_n\in\K^\N\).

\begin{itemize}
    \item \guillemets{Notations de Hardy} (peu utilisées mais très commodes pour énoncer la proposition suivante) : \[u_n\prec v_n\ssi u_n\egqd{n\to\pinf}\o{v_n}\] et : \[u_n\preccurlyeq v_n\ssi u_n\egqd{n\to\pinf}\O{v_n}.\]
    \item \guillemets{Notation de Vinogradov} : \[u_n\ll v_n\ssi u_n\egqd{n\to\pinf}\O{v_n}.\]
    \item \guillemets{Notation \(\Theta\)} : \[\begin{aligned}
        u_n\egqd{n\to\pinf}\Theta\paren{v_n}&\ssi\croch{u_n\egqd{n\to\pinf}\O{v_n}\quad\text{et}\quad v_n\egqd{n\to\pinf}\O{u_n}} \\
        &\ssi\quantifs{\exists M_1,M_2\in\Rps;\exists N\in\N;\forall n\in\interventierie{N}{\pinf}}M_1\abs{v_n}\leq\abs{u_n}\leq M_2\abs{v_n}.
    \end{aligned}\] Cette notation est surtout utilisée en informatique (pour étudier la complexité des algorithmes). On verra qu'on a : \[u_n\egqd{n\to\pinf}\Theta\paren{v_n}\imp u_n\simqd{n\to\pinf}v_n.\]
\end{itemize}
\end{rem}

\begin{prop}[Croissances comparées]
On utilise exceptionnellement les notations de Hardy pour gagner en concision.

Soient \(a_1,a_2,\alpha_1,\alpha_2,\beta_1,\beta_2\in\R\) tels que : \[0<a_1<1<a_2\qquad\text{et}\qquad\alpha_1<0<\alpha_2\qquad\text{et}\qquad\beta_1<0<\beta_2.\]

Alors on a, quand \(n\) tend vers \(\pinf\) : \[\underbrace{a_1^n\quad\prec\quad n^{\alpha_1}\quad\prec\quad\ln^{\beta_1}n}_{\text{suites de limite nulle}}\quad\prec\quad1\quad\prec\quad\underbrace{\ln^{\beta_2}n\quad\prec\quad n^{\alpha_2}\quad\prec\quad a_2^n\quad\prec\quad n!}_{\text{suites de limite }\pinf}\]
\end{prop}

\begin{dem}
Montrons que \(a^n\egqd{n\to\pinf}\o{n!}\) avec \(a\in\Rps\), \cad \(\lim_{n\to\pinf}\dfrac{a^n}{n!}=0\).

On a : \[\quantifs{\forall n\in\N}a\times\dfrac{a}{2}\times\dots\times\dfrac{a}{n}=\dfrac{a^n}{n!}.\]

Soit \(N\in\Ns\) tel que \(\dfrac{a}{N}\leq\dfrac{1}{2}\) (un tel \(N\) existe car \(\lim_{n\to\pinf}\dfrac{a}{n}=0\)).

Posons \(M=\dfrac{a^{N-1}}{\paren{N-1}!}>0\).

On a : \[\quantifs{\forall n\in\interventierie{N}{\pinf}}\dfrac{a^n}{n!}=\underbrace{a\times\dots\times\dfrac{a}{N-1}}_{=M}\times\underbrace{\dfrac{a}{N}}_{\leq\frac{1}{2}}\times\dots\times\underbrace{\dfrac{a}{n}}_{\leq\frac{1}{2}}.\]

Donc : \[\quantifs{\forall n\in\interventierie{N}{\pinf}}0\leq\dfrac{a^n}{n!}\leq M\dfrac{1}{2^{n-N+1}}.\]

Or \(\lim_{n\to\pinf}\dfrac{M}{2^{n-N+1}}=0\).

Donc selon le théorème des gendarmes, on a \(\lim_{n\to\pinf}\dfrac{a^n}{n!}=0\) donc : \[a^n\egqd{n\to\pinf}\o{n!}.\]
\end{dem}

\begin{dem}
Soient \(a\in\intervee{1}{\pinf}\) et \(\alpha\in\R\).

Montrons que \(n^\alpha\egqd{n\to\pinf}\o{a^n}\), \cad \(\lim_{n\to\pinf}\dfrac{n^\alpha}{a^n}=0\).

Posons \(\fonction{f}{\Rps}{\R}{x}{\dfrac{x^\alpha}{a^x}=x^\alpha\e{-x\ln a}}\)

On a : \[\begin{aligned}
\quantifs{\forall x\in\Rps}f\prim\paren{x}&=\alpha x^{\alpha-1}a^{-x}-x^\alpha a^{\alpha x}\ln a \\
&=x^{\alpha-1}a^{-x}\paren{\alpha-x\ln a}.
\end{aligned}\]

On a \(\quantifs{\forall x\in\Rps}f\prim\paren{x}\geq0\ssi x\leq\dfrac{\alpha}{\ln a}\) donc \(f\) est décroissante sur \(\intervie{\dfrac{\alpha}{\ln a}}{\pinf}\).

Donc \(f\) admet une limite en \(\pinf\).

De plus, \(f\) est minorée par \(0\) donc \(l=\lim_{\pinf}f\in\Rps\).

De plus : \[\quantifs{\forall x\in\Rps}f\paren{2x}=\dfrac{2^\alpha}{a^x}f\paren{x}.\]

D'où, en passant à la limite quand \(x\to\pinf\) : \[l=0\times l=0\text{ car }l\text{ est finie}.\]

Donc \(n^\alpha\egqd{n\to\pinf}\o{a^n}\).
\end{dem}

\begin{dem}
Soient \(\alpha\in\Rps\) et \(\beta\in\R\).

Montrons que \(\ln^\beta n\egqd{n\to\pinf}\o{n^\alpha}\), \cad \(\lim_{n\to\pinf}\dfrac{\ln^\beta n}{n^\alpha}=0\).

Posons \(\fonction{f}{\intervee{1}{\pinf}}{\R}{x}{\dfrac{\ln^\beta x}{x^\alpha}=x^{-\alpha}\ln^\beta x}\)

On a : \[\begin{aligned}
\quantifs{\forall x\in\intervee{1}{\pinf}}f\prim\paren{x}&=\dfrac{\beta}{x}x^{-\alpha}\ln^{\beta-1}x-\alpha x^{-\alpha-1}\ln^\beta x \\
&=x^{-\alpha-1}\ln^{\beta-1}x\paren{\beta-\alpha\ln x}.
\end{aligned}\]

On a : \[\begin{aligned}
\quantifs{\forall x\in\intervee{1}{\pinf}}f\prim\paren{x}\geq0&\ssi\alpha\ln x\leq\beta \\
&\ssi x\leq\e{\nicefrac{\beta}{\alpha}}.
\end{aligned}\]

Donc \(f\) est décroissante sur \(\intervie{\e{\nicefrac{\beta}{\alpha}}}{\pinf}\).

Donc \(f\) admet une limite en \(\pinf\).

De plus, \(f\) est minorée par \(0\) donc \(l=\lim_{\pinf}f\in\Rp\).

De plus, on a \(\quantifs{\forall x\in\intervee{1}{\pinf}}f\paren{x^2}=\dfrac{2^\beta}{x^\alpha}f\paren{x}\).

Donc en passant à la limite quand \(x\to\pinf\) : \[l=0\times l=0\text{ car }l\text{ est finie}.\]

D'où \(\ln^\beta n\egqd{n\to\pinf}\o{n^{\alpha}}\).
\end{dem}

\begin{dem}
Soit \(\beta\in\Rps\).

On a \(\lim_n\ln^\beta n=\pinf\) donc \(1\egqd{n\to\pinf}\o{\ln^\beta n}\) car \(\lim_n\dfrac{1}{\ln^\beta n}=0\).
\end{dem}

\begin{dem}
Soit \(\beta\in\Rms\).

On a \(-\beta>0\) donc \(1\egqd{n\to\pinf}\o{\ln^{-\beta}n}\).

Donc \(\ln^\beta n\egqd{n\to\pinf}\o{1}\).
\end{dem}

\begin{dem}
Soient \(\alpha,\beta\in\Rms\).

On a \(-\alpha>0\) et \(-\beta>0\).

Donc selon ce qui précède, on a, quand \(n\to\pinf\) : \(\ln^{-\beta}n=\o{n^{-\alpha}}\).

D'où : \[n^\alpha\egqd{n\to\pinf}\o{\ln^\beta n}.\]
\end{dem}

\begin{dem}
Soient \(a\in\intervee{0}{1}\) et \(\alpha\in\Rms\).

De même, \(a^n\egqd{n\to\pinf}\o{n^\alpha}\) découle de ce qui précède.
\end{dem}

\subsection{Relation d'équivalence \(\sim\)}

\begin{defprop}
Soient \(\paren{u_n}_n,\paren{v_n}_n\in\K^\N\).

Les propositions suivantes sont équivalentes :

\begin{enumerate}
    \item \(v_n\egqd{n\to\pinf}u_n+\o{u_n}\) \\
    \item \(u_n\egqd{n\to\pinf}v_n+\o{v_n}\) \\
    \item Il existe une suite \(\paren{\lambda_n}_n\in\K^\N\) telle que : \[\lim_{n\to\pinf}\lambda_n=1\qquad\text{et}\qquad\quantifs{\exists N\in\N;\forall n\in\interventierie{N}{\pinf}}u_n=\lambda_nv_n\]
    \item Il existe une suite \(\paren{\mu_n}_n\in\K^\N\) telle que : \[\lim_{n\to\pinf}\mu_n=1\qquad\text{et}\qquad\quantifs{\exists N\in\N;\forall n\in\interventierie{N}{\pinf}}v_n=\mu_nu_n.\]
\end{enumerate}

Lorsqu'elles sont vérifiées, on dit que \[\paren{u_n}_n\text{ est équivalente à }\paren{v_n}_n\] ou que \[\begin{aligned}
u_n\text{ est équivalent à }v_n\text{ quand }n\text{ tend vers }\pinf \\
u_n\text{ est un équivalent de }v_n\text{ quand }n\text{ tend vers }\pinf
\end{aligned}\] et on note : \[u_n\sim v_n\text{ quand }n\text{ tend vers }\pinf\] ou : \[u_n\simqd{n\to\pinf}v_n.\]
\end{defprop}

\begin{dem}[(2) \(\ssi\) (3)]
On a : \[\begin{aligned}
u_n\egqd{n\to\pinf}v_n+\o{v_n}&\ssi\quantifs{\exists\paren{\epsilon_n}_n\in\K^\N}\begin{dcases}
\quantifs{\exists N\in\N;\forall n\in\interventierie{N}{\pinf}}u_n=v_n+\epsilon_nv_n \\
\lim_n\epsilon_n=0
\end{dcases} \\
&\ssi\quantifs{\exists\paren{\epsilon_n}_n\in\K^\N}\begin{dcases}
\quantifs{\exists N\in\N;\forall n\in\interventierie{N}{\pinf}}u_n=\paren{1+\epsilon_n}v_n \\
\lim_n\epsilon_n=0
\end{dcases} \\
&\ssi\quantifs{\exists\paren{\lambda_n}_n\in\K^\N}\begin{dcases}
\quantifs{\exists N\in\N;\forall n\in\interventierie{N}{\pinf}}u_n=\lambda_nv_n \\
\lim_n\lambda_n=1
\end{dcases}
\end{aligned}\]
\end{dem}

\begin{dem}[(1) \(\ssi\) (4)]
Idem.
\end{dem}

\begin{dem}[(3) \(\ssi\) (4)]~\\
Claire en prenant \(\mu_n=\dfrac{1}{\lambda_n}\) à partir d'un certain rang.
\end{dem}

\begin{prop}
Soient \(\paren{u_n}_n,\paren{v_n}_n\in\K^\N\).

Si les termes de \(\paren{v_n}_n\) sont non-nuls à partir d'un certain rang : \[\quantifs{\exists N\in\N;\forall n\in\interventierie{N}{\pinf}}v_n\not=0,\] alors : \[u_n\simqd{n\to\pinf}v_n\ssi\lim_{n\to\pinf}\dfrac{u_n}{v_n}=1.\]
\end{prop}

\begin{ex}
Soient \(\alpha,\beta\in\R\). On a : \[n^\alpha\simqd{n\to\pinf}n^\beta\ssi\alpha=\beta.\]

Soient \(a,b\in\Rps\). On a : \[a^n\simqd{n\to\pinf}b^n\ssi a=b.\]

Soient \(d\in\N\), \(a_0,\dots,a_d\in\C\) tels que \(a_d\not=0\) et \(P=a_dX^d+\dots+a_0X^0\in\poly[\C]\). On a : \[P\paren{n}\simqd{n\to\pinf}a_dn^d.\]

Soit \(\paren{u_n}_n\in\K^\N\). On a : \[u_n\simqd{n\to\pinf}0\ssi\paren{u_n}_n\text{ est nulle à partir d'un certain rang}\] et : \[\quantifs{\forall l\in\K\excluant\accol{0}}u_n\simqd{n\to\pinf}l\ssi\lim_{n\to\pinf}u_n=l.\]

NB : l'équivalence précédente est valable pour toute limite finie non-nulle. En particulier, deux suites admettant une même limite finie non-nulle sont équivalentes.
\end{ex}

\subsection{Propriétés de \(\sim\)}

\begin{prop}
La relation \(\sim\) est une relation d'équivalence sur \(\K^\N\).
\end{prop}

\begin{prop}
Soient \(\paren{u_n}_n,\paren{v_n}_n\in\K^\N\).

On a : \[u_n\simqd{n\to\pinf}v_n\imp\begin{dcases}
u_n\egqd{n\to\pinf}\O{v_n} \\
v_n\egqd{n\to\pinf}\O{u_n}
\end{dcases}\]
\end{prop}

\begin{prop}[Cas réel : limite et signe]
Soient \(\paren{u_n}_n,\paren{v_n}_n\in\R^\N\) et \(l\in\Rb\).

Si \(\begin{dcases}
\lim_{n\to\pinf}u_n=l \\
u_n\simqd{n\to\pinf}v_n
\end{dcases}\) alors \(\lim_{n\to\pinf}v_n=l\).

Si \(u_n\simqd{n\to\pinf}v_n\) alors \(\paren{u_n}_n\) et \(\paren{v_n}_n\) sont de même signe au sens strict à partir d'un certain rang : \[\quantifs{\exists N\in\N;\forall n\in\interventierie{N}{\pinf}}\sg u_n=\sg v_n.\]
\end{prop}

\begin{dem}
On suppose \(u_n\simqd{n\to\pinf}v_n\).

Soit \(\paren{\lambda_n}_n\in\R^\N\) telle que \(\begin{dcases}
\quantifs{\exists N\in\N;\forall n\in\interventierie{N}{\pinf}}v_n=\lambda_nu_n \\
\lim_{n\to\pinf}\lambda_n=1
\end{dcases}\)

Supposons \(\lim_nu_n=l\). Alors \(\lim_n\lambda_nu_n=l\) donc \(\lim_nv_n=l\).

Comme \(\lim_n\lambda_n=1\), il existe \(N\in\N\) tel que \(\quantifs{\forall n\in\interventierie{N}{\pinf}}\begin{dcases}
\lambda_n\geq\dfrac{1}{2} \\
v_n=\lambda_nu_n
\end{dcases}\) donc \[\quantifs{\forall n\in\interventierie{N}{\pinf}}\sg v_n=\sg u_n.\]
\end{dem}

\begin{prop}[Cas complexe : limite]
Soient \(\paren{u_n}_n,\paren{v_n}_n\in\C^\N\) et \(l\in\C\).

Si \(\begin{dcases}
\lim_{n\to\pinf}u_n=l \\
u_n\simqd{n\to\pinf}v_n
\end{dcases}\) alors \(\lim_{n\to\pinf}v_n=l\).
\end{prop}

\begin{prop}[Produits]
Soient \(\paren{a_n}_n,\paren{b_n}_n,\paren{c_n}_n,\paren{d_n}_n\in\K^\N\).

On a, quand \(n\) tend vers \(\pinf\) : \[\begin{dcases}
a_n\sim b_n \\
c_n\sim d_n
\end{dcases}\imp a_nc_n\sim b_nd_n.\]
\end{prop}

\begin{prop}[Puissances]
Soient \(\paren{u_n}_n,\paren{v_n}_n\in\paren{\Rps}^\N\) et \(\alpha\in\R\).

On a, quand \(n\) tend vers \(\pinf\) : \[u_n\sim v_n\imp u_n^\alpha\sim v_n^\alpha.\]

On retient en particulier : \[u_n\sim v_n\imp\dfrac{1}{u_n}\sim\dfrac{1}{v_n}.\]

La même proposition est vraie en prenant \(\paren{u_n}_n,\paren{v_n}_n\in\C^\N\) et \(\alpha\in\N\) ou en prenant \(\paren{u_n}_n,\paren{v_n}_n\in\paren{\Cs}^\N\) et \(\alpha\in\Z\).
\end{prop}

\begin{prop}[Suites extraites]
Soient \(\paren{u_n}_n,\paren{v_n}_n\in\C^\N\) et \(\phi:\N\to\N\) une fonction strictement croissante.

On a : \[u_n\simqd{n\to\pinf}\imp u_{\phi\paren{n}}\simqd{n\to\pinf}v_{\phi\paren{n}}.\]
\end{prop}

\begin{prop}[Sommes]
On ne fait pas de sommes d'équivalents.
\end{prop}

\begin{dem}
Quand \(n\) tend vers \(\pinf\) :

On a \(\begin{dcases}
1+\dfrac{1}{n}\sim1 \\
-1+\dfrac{1}{n}\sim-1
\end{dcases}\) mais on n'a pas \(\dfrac{2}{n}\sim0\).
\end{dem}

\subsection{Formule de Stirling}

\begin{theo}[Formule de Stirling]
On a : \[n!\simqd{n\to\pinf}\sqrt{2\pi n}\paren{\dfrac{n}{\e{}}}^n.\]
\end{theo}

\begin{dem}
\note{Admise}
\end{dem}