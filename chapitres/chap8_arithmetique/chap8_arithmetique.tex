\chapter{Arithmétique}

\minitoc

\section{Rappels \& compléments}

\subsection{Division euclidienne}

\begin{defprop}[Division euclidienne dans \(\Z\)]
Soient \(a\in\Z\) et \(b\in\Ns\).

Il existe un unique couple \(\paren{q,r}\in\Z^2\) tel que : \[\begin{dcases}a=qb+r \\ 0\leq r<b\end{dcases}\]

L'entier \(q\) est appelé le quotient de la division euclidienne de \(a\) par \(b\).

L'entier \(r\) est appelé le reste de la division euclidienne de \(a\) par \(b\).
\end{defprop}

\begin{ex}
Divisions euclidiennes de \(37\) et \(-37\) par \(10\) : \[37=3\times10+7\quad\text{et}\quad-37=\paren{-4}\times10+3.\]
\end{ex}

\begin{dem}
\textit{Cf.} \thref{dem:divEucli}.
\end{dem}

\subsection{Divisibilité}

\begin{defi}[Divisibilité dans \(\Z\)]
Soient \(a,b\in\Z\). Si on a \[\quantifs{\exists c\in\Z}ac=b\] alors on dit que \begin{itemize}
\item \(a\) divise \(b\) ;

\item \(a\) est un diviseur de \(b\) ;

\item \(b\) est un multiple de \(a\) ;

\item \(b\) est divisible par \(a\).
\end{itemize}
\end{defi}

\begin{nota}
Soient \(a,b\in\Z\).

\begin{itemize}
\item La notation \(a\divise b\) signifie \guillemets{\(a\) divise \(b\)} ;

\item L'ensemble \(\accol{ac}_{c\in\Z}\) des multiples de \(a\) est noté \(a\Z\) : \[a\Z=\accol{ac}_{c\in\Z}=\accol{x\in\Z\tq a\divise x}.\]

\item Dans ce cours, on notera \(\ensdiv{b}\) l'ensemble des diviseurs positifs de \(b\) : \[\ensdiv{b}=\accol{x\in\N\tq x\divise b}.\]
\end{itemize}
\end{nota}

\begin{prop}[Divisibilité dans \(\Z\)]
La relation binaire \(\divise\) sur \(\Z\) est réflexive et transitive (mais pas antisymétrique donc ce n'est pas une relation d'ordre sur \(\Z\)).

On a : \(\quantifs{\forall a,b\in\Z}\paren{a\divise b\quad\text{et}\quad b\divise a}\ssi a=\pm b\).

On a : \begin{itemize}
\item \(\quantifs{\forall a,b\in\Z;\forall n\in\Z}a\divise b\imp na\divise nb\) ;

\item \(\quantifs{\forall a,b\in\Z;\forall n\in\Zs}a\divise b\ssi na\divise nb\).
\end{itemize}
\end{prop}

\begin{prop}[Divisibilité dans \(\N\)]\thlabel{prop:divisibilitéDansN}
La relation binaire \(\divise\) sur \(\N\) est une relation d'ordre sur \(\N\).

Pour cette relation d'ordre, \(0\) est le plus grand élément de \(\N\) et \(1\) le plus petit : \(\quantifs{\forall n\in\N}n\divise0\quad\text{et}\quad1\divise n\).

On a : \(\quantifs{\forall a,b\in\Z}a\divise b\imp\paren{a\leq b\quad\text{ou}\quad b=0}\).
\end{prop}

\subsection{Congruences}

\begin{defi}
Soient \(a,b,n\in\Z\).

On dit que \guillemets{\(a\) est congru à \(b\) modulo \(n\)} et on note \(a\equiv b\croch{n}\) si on a : \[\quantifs{\exists k\in\Z}a=b+kn.\]

On a donc : \[\begin{aligned}
a\equiv b\croch{n}&\ssi\quantifs{\exists k\in\Z}a=b+kn \\
&\ssi n\divise\paren{a-b}.
\end{aligned}\]
\end{defi}

\begin{rem}
Soit \(n\in\Z\).

La relation \guillemets{être congrus modulo \(n\)} est une relation d'équivalence sur \(\Z\).
\end{rem}

\begin{rem}
Soient \(x\in\Z\) et \(n\in\Ns\).

On note \(q\) et \(r\) le quotient et le reste de la division euclidienne de \(x\) par \(n\).

On a : \begin{itemize}
\item \(x\equiv r\croch{n}\) ;

\item \(n\divise x\ssi r=0\).
\end{itemize}
\end{rem}

\begin{prop}[Opérations sur les congruences]
Soient \(a,b,c,d,n\in\Z\).

On suppose \(a\equiv b\croch{n}\text{ et }c\equiv d\croch{n}\).

On a : \begin{itemize}
\item somme : \(a+c\equiv b+d\croch{n}\) ;

\item produit : \(ac\equiv bd\croch{n}\) ;

\item puissance : \(\quantifs{\forall k\in\N}a^k\equiv b^k\croch{n}\).
\end{itemize}
\end{prop}

\begin{prop}
Soient \(a,b,n\in\Z\).

\begin{itemize}
\item Pour tout \(\lambda\in\Zs\), on a : \(a\equiv b\croch{n}\ssi\lambda a\equiv\lambda b\croch{\lambda n}\).

\item Pour tout diviseur \(d\) de \(n\), on a : \(a\equiv b\croch{n}\imp a\equiv b\croch{d}\).
\end{itemize}
\end{prop}

\begin{ex}[Élément non-régulier modulo \(n\)]
Trouvons \(\lambda,x,y,n\in\Z\) tels que : \[\lambda\not=0\quad\text{et}\quad\lambda x\equiv\lambda y\croch{n}\quad\text{et}\quad x\not\equiv y\croch{n}.\]

On remarque que \(\lambda=2\), \(x=3\), \(y=0\) et \(n=6\) conviennent.

En effet, on a \[2\not=0\quad\text{et}\quad2\times3\equiv2\times0\croch{6}\quad\text{et}\quad3\not\equiv0\croch{6}.\]
\end{ex}

\begin{ex}
Calculons \(20.222.023\) modulo \(9\), modulo \(11\) et modulo \(999\).

On a \[\begin{aligned}
20.222.023&=2\E{7}+2\E{5}+2\E{4}+2\E{3}+2\E{1}+3\E{0} \\
&\equiv2+2+2+2+2+3\croch{9}\quad\text{car }10\equiv1\croch{9} \\
&\equiv13\croch{9} \\
&\equiv4\croch{9}.
\end{aligned}\]

De plus, \(10\equiv-1\croch{11}\) donc on a \[\begin{aligned}
20.222.023&\equiv-2-2-2+2-2+3\croch{11} \\
&\equiv-3\croch{11} \\
&\equiv8\croch{11}.
\end{aligned}\]

Enfin, comme \(1.000\equiv1\croch{999}\), on a \[\begin{aligned}
20.222.023&=20\times1000^2+222\times1000^1+23\times1000^0 \\
&\equiv20+222+23\croch{999} \\
&\equiv265\croch{999}.
\end{aligned}\]
\end{ex}

\subsection{Sous-groupes}

\begin{prop}[Intersection de sous-groupes]
Soit \(\paren{G,*}\) un groupe et \(\paren{H_i}_{i\in I}\) une famille de sous-groupes de \(G\).

Alors l'intersection \(\biginter_{i\in I}H_i\) de ces sous-groupes est un sous-groupe de \(G\).
\end{prop}

\begin{dem}
\Cf \thref{exo:6.15}.
\end{dem}

\begin{defprop}[Somme de sous-groupes]\thlabel{defprop:sommeDeSousGroupesEstUnSousGroupe}
Soit \(\paren{G,+}\) un groupe abélien. Soient \(H_1,H_2\) des sous-groupes de \(G\).

On appelle somme de \(H_1\) et \(H_2\) et on note \(H_1+H_2\) l'ensemble : \[\begin{aligned}
H_1+H_2&=\accol{g\in G\tq\quantifs{\exists h_1\in H_1,\exists h_2\in H_2}g=h_1+h_2} \\
&=\accol{h_1+h_2}_{\paren{h_1,h_2}\in H_1\times H_2}
\end{aligned}\]

On a : \begin{enumerate}
\item l'ensemble \(H_1+H_2\) est un sous-groupe de \(G\) ;

\item la loi \(+\) est une loi de composition interne sur l'ensemble des sous-groupes de \(G\) ;

\item cette loi \(+\) est associative et commutative.
\end{enumerate}
\end{defprop}

\begin{dem}[1]
Montrons que \(H_1+H_2\) est un sous-groupe de \(G\).

On note \(0\) le neutre de \(G\).

On a \(H_1+H_2\subset G\).

Comme \(H_1\) et \(H_2\) sont des sous-groupes de \(G\), \(0\in H_1\) et \(0\in H_2\) donc \(0=0+0\in H_1+H_2\).

Soient \(h,h\prim\in H_1+H_2\).

Montrons que \(h-h\prim\in H_1+H_2\).

Soient \(h_1,h_1\prim\in H_1\) et \(h_2,h_2\prim\in H_2\) tels que \[h=h_1+h_2\qquad\text{et}\qquad h\prim=h_1\prim+h_2\prim.\]

On a \[\begin{WithArrows}
h-h\prim&=h_1+h_2-h_1\prim-h_2\prim \Arrow{car groupes abéliens} \\
&=\underbrace{h_1-h_1\prim}_{\in H_1}+\underbrace{h_2-h_2\prim}_{\in H_2}
\end{WithArrows}\]

Donc \(h-h\prim\in H_1+H_2\).

Donc \(H_1+H_2\) est un sous-groupe de \(G\).
\end{dem}

\begin{dem}[2]
Clair.
\end{dem}

\begin{dem}[3]
Montrons que \(+\) est associative :

Soient \(H_1,H_2,H_3\) des sous-groupes de \(G\).

Montrons que \(H_1+\paren{H_2+H_3}=\paren{H_1+H_2}+H_3\).

On a \(H_1+\paren{H_2+H_3}\subset G\) et \(\paren{H_1+H_2}+H_3\subset G\).

Soit \(g\in G\).

On a \[\begin{aligned}
g\in\paren{H_1+H_2}+H_3&\ssi\quantifs{\exists h\in H_1+H_2;\exists h_3\in H_3}g=h+h_3 \\
&\ssi\quantifs{\exists h_1\in H_1;\exists h_2\in H_2;\exists h_3\in H_3}g=\paren{h_1+h_2}+h_3 \\
&\ssi\quantifs{\exists h_1\in H_1;\exists h_2\in H_2;\exists h_3\in H_3}g=h_1+h_2+h_3 \\
&\ssi\quantifs{\exists h_1\in H_1;\exists h\in H_2+H_3}g=h_1+h \\
&\ssi g\in H_1+\paren{H_2+H_3}
\end{aligned}\]

Donc \(+\) est associative.

Montrons que \(+\) est commutative :

On a \(H_1+H_2\subset G\) et \(H_2+H_1\subset G\).

Soit \(g\in G\).

On a \[\begin{aligned}
g\in H_1+H_2&\ssi\quantifs{\exists h_1\in H_1;\exists h_2\in H_2}g=h_1+h_2 \\
&\ssi\quantifs{\exists h_2\in H_2;\exists h_1\in H_1}g=h_2+h_1 \\
&\ssi g\in H_2+H_1
\end{aligned}\]

Donc \(+\) est commutative.
\end{dem}

\begin{theo}[Sous-groupes de \(\Z\)]\thlabel{theo:nZSontLesSousGroupesDeZ}
Les sous-groupes de \(\paren{\Z,+}\) sont les ensembles de la forme \(n\Z\) où \(n\in\N\).
\end{theo}

\begin{dem}
Montrons que les sous-groupes de \(\groupe{\Z}\) sont les parties de la forme \(n\Z=\accol{nk}_{k\in\Z}\) où \(n\in\N\).

On a \[n\Z=\accol{\dots;-2n;-n;0;n;2n;\dots}.\]

\increc

Soit \(n\in\N\).

Montrons que \(n\Z\) est un sous-groupe de \(\groupe{\Z}\).

On a \(n\Z\subset\Z\).

On a \(0\in n\Z\) car \(0=n\times0\).

Soient \(x,y\in n\Z\).

Montrons que \(x-y\in n\Z\).

Soient \(x\prim,y\prim\in\Z\) tels que \(x=nx\prim\) et \(y=ny\prim\).

On a \[x-y=nx\prim-ny\prim=n\underbrace{\paren{x-x\prim}}_{\in\Z}.\]

Donc \(x-y\in n\Z\).

Donc \(n\Z\) est un sous-groupe de \(\groupe{\Z}\).

\incdir

Soit \(H\) un sous-groupe de \(\groupe{\Z}\).

On pose \(E=H\inter\Ns\).

Si \(E=\ensvide\) :

Montrons que \(H=0\Z=\accol{0}\).

Par l'absurde, soit \(x\in H\excluant\accol{0}\).

Si \(x>0\) alors \(x\in E\) : contradiction car \(E=\ensvide\).

Si \(x<0\) alors \(-x\in E\) : contradiction car \(E=\ensvide\).

Donc \(H=\accol{0}=0\Z\).

Supposons \(E\not=\ensvide\).

On pose \(n=\min E\) (\(n\) existe car \(E\) est une partie non-vide de \(\N\)).

Montrons que \(H=n\Z\).

\increc

Montrons que \(\quantifs{\forall k\in\N}nk\in H\) par récurrence sur \(k\).

Initialisation : on a \(0\in H\) et \(0=0\times n\).

Hérédité : soit \(k\in\N\) tel que \(nk\in H\).

On a \(n\paren{k+1}=nk+n\in H\) car \(H\) est un sous-groupe de \(\Z\) et \(nk,n\in H\).

Conclusion : \(\quantifs{\forall k\in\N}nk\in H\).

On en déduit que \(\quantifs{\forall k\in\N}-nk\in H\) car \(H\) est un sous-groupe de \(\Z\).

Donc \(\quantifs{\forall k\in\Z}nk\in H\).

D'où \(n\Z\subset H\).

\incdir

Soit \(h\in H\).

On a \(n\in\Ns\).

On note \(q\) et \(r\) le quotient et le reste de la division euclidienne de \(h\) par \(n\).

On a donc \(\begin{dcases}h=nq+r \\ 0\leq r<n\end{dcases}\)

On a \(h,n\in H\) donc \(r=h-nq\in H\).

Donc \(r=0\) car \(n\) est le plus petit élément strictement positif de \(H\).

Donc \(h=qn\in n\Z\).
\end{dem}

\section{PGCD}

\subsection{PGCD de deux entiers}

\subsubsection{Définition}

\begin{defi}
Soient \(a,b\in\Z\).

On rappelle qu'on note \(\ensdiv{a}\) l'ensemble des diviseurs positifs de \(a\).

Les diviseurs positifs communs à \(a\) et \(b\) sont les éléments de l'ensemble \(\ensdiv{a}\inter\ensdiv{b}\).

Si celui-ci possède un plus grand élément pour la relation d'ordre \(\divise\), on l'appelle le plus grand commun diviseur de \(a\) et \(b\) et on le note \(a\et b\).
\end{defi}

\begin{ex}
On a \[\ensdiv{12}=\accol{1;2;3;4;6;12}\qquad\text{et}\qquad\ensdiv{15}=\accol{1;3;5;15}.\]

Donc les diviseurs communs à \(12\) et \(15\) sont : \[\ensdiv{12}\inter\ensdiv{15}=\accol{1;3}.\]

Donc le PGCD de \(12\) et \(15\) est : \[12\et15=3.\]
\end{ex}

\begin{rem}
Soient \(a,b\in\Z\).

Le PGCD de \(a\) et \(b\) existe si, et seulement si, le PGCD de \(\abs{a}\) et \(\abs{b}\) existe.

On a alors \[a\et b=\abs{a}\et\abs{b}.\]
\end{rem}

\begin{dem}
Clair car \(\ensdiv{a}\inter\ensdiv{b}=\ensdiv{\abs{a}}\inter\ensdiv{\abs{b}}\).
\end{dem}

\begin{rem}
On a, pour tout \(a\in\Z\) : \[a\et0=\abs{a}\qquad\text{et}\qquad a\et1=1.\]
\end{rem}

\begin{rem}
Soient \(a,b\in\Z\).

\begin{enumerate}
\item Le PGCD de \(a\) et \(b\), s'il existe, est le diviseur commun à \(a\) et \(b\) positif et divisible par tous leurs autres diviseurs communs. \\

\item Il est donc caractérisé par : \[a\et b\geq0\qquad\text{et}\qquad\ensdiv{a}\inter\ensdiv{b}=\ensdiv{a\et b}.\] \\

\item Si \(a=b=0\) alors \(0\et0=0\). \\ Sinon, le plus grand diviseur commun à \(a\) et \(b\) pour l'ordre \(\divise\) est aussi le plus grand diviseur commun à \(a\) et \(b\) pour l'ordre \(\leq\) selon la \thref{prop:divisibilitéDansN}.
\end{enumerate}
\end{rem}

\begin{dem}[2]
Soit \(n\in\Z\) tel que \(n\geq0\) et \(\ensdiv{a}\inter\ensdiv{b}=\ensdiv{n}\).

Montrons que \(n\) est le PGCD de \(a\) et \(b\).

On a \(n\in\ensdiv{n}\) donc \(n\in\ensdiv{a}\inter\ensdiv{b}\) donc \[n\divise a\qquad\text{et}\qquad n\divise b.\]

Donc \(n\) est un diviseur commun à \(a\) et \(b\).

Montrons que \(n\) est le plus grand d'entre eux.

Soit \(d\) un diviseur commun à \(a\) et \(b\).

On a \(d\in\ensdiv{a}\inter\ensdiv{b}\) donc \(d\in\ensdiv{n}\).

Donc \(d\divise n\).
\end{dem}

\begin{lem}
Soient \(a,b,n\in\N\) tels que \(a\equiv b\croch{n}\).

Alors \(a\) et \(n\) admettent un PGCD si, et seulement si, \(b\) et \(n\) admettent un PGCD.

On a alors \[a\et n=b\et n.\]
\end{lem}

\begin{dem}
Il suffit de montrer que \(\ensdiv{a}\inter\ensdiv{b}=\ensdiv{b}\inter\ensdiv{a}\).

\incdir

Soient \(d\in\ensdiv{a}\inter\ensdiv{b}\) et \(k\in\Z\) tel que \(b=a+kn\).

On a \(d\divise a\) et \(d\divise kn\) donc \(d\divise b\).

Donc \(d\in\ensdiv{b}\inter\ensdiv{a}\).

\increc Idem.
\end{dem}

\begin{rem}
Soient \(a,b\in\Z\).

On a \[a\et b=\abs{a}\ssi a\divise b.\]
\end{rem}

\begin{dem}
\impdir

Supposons \(a\et b=\abs{a}\).

On a \(a\divise\abs{a}\) et \(\abs{a}\divise b\).

Donc \(a\divise b\).

\imprec

Supposons \(a\divise b\).

On a \(\abs{a}\divise a\) et \(\abs{a}\divise b\).

Donc \(\abs{a}\in\ensdiv{a}\inter\ensdiv{b}\).

De plus, on a \(\quantifs{\forall d\in\ensdiv{a}\inter\ensdiv{b}}d\divise\abs{a}\).

Donc \(\abs{a}=a\et b\).
\end{dem}

\begin{rem}
Soient \(a,b\in\N\).

Alors \(a\et b=\inf\accol{a;b}\) pour l'ordre \(\divise\), sous réserve d'existence.
\end{rem}

\subsubsection{Preuve algébrique}

\begin{prop}
Soient \(a,b\in\Z\).

Alors \(a\) et \(b\) admettent un PGCD.
\end{prop}

\begin{dem}
L'ensemble \(a\Z+b\Z\) est un sous-groupe de \(\Z\) (\cf \thref{defprop:sommeDeSousGroupesEstUnSousGroupe} et \thref{theo:nZSontLesSousGroupesDeZ}).

Il existe donc \(n\in\Z\) tel que \(a\Z+b\Z=n\Z\).

Montrons que \(n\) est le PGCD de \(a\) et \(b\).

On a \(n\geq0\).

De plus, on a \(a\in a\Z+b\Z\) car \(a=1\times a+0\times b\) donc \(a\in n\Z\) donc \(n\divise a\).

De même, \(n\divise b\).

Soient \(d\in\ensdiv{a}\inter\ensdiv{b}\) et \(k,l\in\Z\) tels que \(dk=a\) et \(dl=b\).

On a \(n\in n\Z\) donc \(n\in a\Z+b\Z\).

Soient \(u,v\in\Z\) tels que \(n=ua+vb\).

On a \(n=udk+vdl=d\paren{uk+vl}\) donc \(d\divise n\).

Donc \(n\) est le PGCD de \(a\) et \(b\).
\end{dem}

\subsubsection{Preuve algorithmique, algorithme d'Euclide}

On définit une fonction récursive \(\pgcd:\Z^2\to\N\).

Montrons que la fonction termine toujours et que \(\pgcd[a][b]\) renvoie le PGCD de \(a\) et \(b\) pour tout \(\paren{a,b}\in\Z^2\).

\begin{algo}[Algorithme d'Euclide]\thlabel{algo:EuclideEntiers}
Si \(a<0\) ou \(b<0\), on renvoie \(\pgcd[\abs{a}][\abs{b}]\).

Si \(b=0\), on renvoie \(a\).

Sinon : \begin{itemize}
\item on fait la division euclidienne de \(a\) par \(b\) : \(\begin{dcases}a=qb+r \\ 0\leq r<b\end{dcases}\) ; \\

\item on a \(a\equiv r\croch{b}\) donc on renvoie \(\pgcd[b][r]\).
\end{itemize}
\end{algo}

\begin{dem}[Terminaison]
Montrons que l'algorithme termine toujours.

Soit \(\paren{a,b}\in\Z^2\).

Par l'absurde, supposons que l'algorithme ne termine jamais : \[\pgcd[a][b]\to\underbrace{\pgcd[a_1][b_1]\to\pgcd[a_2][b_2]\to\dots}_{\text{arguments }\geq0}\]

Pour tout \(k\in\Ns\), on a \(\begin{dcases}a_{k+1}=b_k \\ 0\leq b_{k+1}<b_k\end{dcases}\)

Donc \(\paren{b_k}_{k\in\Ns}\) est une suite strictement décroissante d'entiers naturels : contradiction.

Donc l'algorithme termine toujours.
\end{dem}

\begin{ex}
Calculons le PGCD de \(1024\) et \(1000\) : \[\begin{aligned}
1024&=1\times1000+24 \\
1000&=41\times24+16 \\
24&=1\times16+8 \\
16&=2\times8+0
\end{aligned}\]

Donc \(1024\et1000=8\).
\end{ex}

\subsubsection{Propriétés}

\begin{prop}\thlabel{prop:PGCDFoisLambdaEgal}
Soient \(a,b,\lambda\in\Z\).

On a : \[\paren{\lambda a}\et\paren{\lambda b}=\abs{\lambda}\paren{a\et b}.\]
\end{prop}

\begin{dem}
Montrons que \(\lambda\paren{a\et b}\divise\paren{\lambda a}\et\paren{\lambda b}\).

On a \(a\et b\) divise \(a\) et \(b\).

Donc \(\lambda\paren{a\et b}\) divise \(\lambda a\) et \(\lambda b\).

Donc \(\lambda\paren{a\et b}\divise\paren{\lambda a}\et\paren{\lambda b}\).

Montrons que \(\paren{\lambda a}\et\paren{\lambda b}\divise\lambda\paren{a\et b}\).

On a \(\lambda\) divise \(\lambda a\) et \(\lambda b\).

Donc \(\lambda\divise\paren{\lambda a}\et\paren{\lambda b}\).

Soit \(d\in\Z\) tel que \(\paren{\lambda a}\et\paren{\lambda b}=d\lambda\).

On a \(\paren{\lambda a}\et\paren{\lambda b}\) divise \(\lambda a\) et \(\lambda b\).

Donc \(d\lambda\) divise \(\lambda a\) et \(\lambda b\).

Supposons \(\lambda\not=0\).

Alors \(d\) divise \(a\) et \(b\).

Donc \(d\divise a\et b\).

Donc \(\lambda d\divise\lambda\paren{a\et b}\).

Donc \(\paren{\lambda a}\et\paren{\lambda b}\divise\lambda\paren{a\et b}\).

Si \(\lambda=0\) : la proposition est vraie aussi car \(\begin{dcases}\paren{\lambda a}\et\paren{\lambda b}=0 \\ \lambda\paren{a\et b}=0\end{dcases}\)

Finalement, les entiers \(\lambda\paren{a\et b}\) et \(\paren{\lambda a}\et\paren{\lambda b}\) se divisent mutuellement.

Donc \(\abs{\lambda\paren{a\et b}}=\abs{\paren{\lambda a}\et\paren{\lambda b}}\).

Donc \(\abs{\lambda}\paren{a\et b}=\paren{\lambda a}\et\paren{\lambda b}\).
\end{dem}

\subsection{Relation de Bézout}

\subsubsection{Définition}

\begin{defprop}[Relation de Bézout]\thlabel{defprop:relationDeBezout}
Soient \(a,b\in\Z\).

Alors il existe des entiers \(u,v\in\Z\) tels que : \[ua+vb=a\et b.\]

Une telle écriture s'appelle une relation de Bézout.

Elle n'est pas unique.
\end{defprop}

\begin{dem}[Absence d'unicité]
Soient \(u,v\in\Z\) tels que \[ua+vb=a\et b.\]

On a alors \[\paren{u-b}a+\paren{v+a}b=a\et b.\]

Donc le couple \(\paren{u\prim,v\prim}=\paren{u-b,v+a}\) convient aussi.
\end{dem}

\subsubsection{Preuve algébrique}

\begin{prop}
Soient \(a,b\in\Z\).

Alors \(a\) et \(b\) admettent une relation de Bézout.
\end{prop}

\begin{dem}
On a vu que \(a\Z+b\Z=\paren{a\et b}\Z\).

On a \(a\et b\in\paren{a\et b}\Z\) donc \(a\et b\in a\Z+b\Z\).

Donc on a \[\quantifs{\exists u,v\in\Z}a\et b=ua+bv.\]
\end{dem}

\subsubsection{Preuve algorithmique, algorithme d'Euclide étendu}

Écrivons \(\bezout[a][b]\) où \(a,b\in\Z\) et qui renvoie \(u,v,d\) tels que \(\begin{dcases}d=a\et b \\ ua+vb=a\et b\end{dcases}\)

\begin{algo}[Algorithme d'Euclide étendu, en langage Python]\thlabel{algo:EuclideEtenduEntiers}
\begin{code}
def bezout(a, b):
	if a < 0:
		u, v, d = bezout(-a, b)
		return -u, v, d
	if b < 0:
		u, v, d = bezout(a, -b)
		return u, -v, d
	elif b == 0:
		return 1, 0, a
	else:
		q, r = a // b, a % b
		u, v, d = bezout(b, r)
		return v, u - v * q, d
\end{code}
\end{algo}

\begin{ex}
Déterminons le PGCD et une relation de Bézout de \(14\) et \(9\) : \[\begin{WithArrows}
&\bezout[14][9]\text{ (\(q=1\), \(r=5\))} \Arrow[jump=9]{} \\
&\hookrightarrow\bezout[9][5]\text{ (\(q=1\), \(r=4\))} \Arrow[jump=7]{} \\
&\color{white}\hookrightarrow\color{black}\hookrightarrow\bezout[5][4]\text{ (\(q=1\), \(r=1\))} \Arrow[jump=5]{}  \\
&\color{white}\hookrightarrow\hookrightarrow\color{black}\hookrightarrow\bezout[4][1]\text{ (\(q=4\), \(r=0\))} \Arrow[jump=3]{} \\
&\color{white}\hookrightarrow\hookrightarrow\hookrightarrow\color{black}\hookrightarrow\bezout[1][0] \Arrow{} \\
&\color{white}\hookrightarrow\hookrightarrow\hookrightarrow\hookrightarrow\color{black}1,0,1 \\
&\color{white}\hookrightarrow\hookrightarrow\hookrightarrow\color{black}0,1,1 \\
&\color{white}\hookrightarrow\hookrightarrow\color{black}1,-1,1 \\
&\color{white}\hookrightarrow\color{black}-1,2,1 \\
&2,-3,1
\end{WithArrows}\]

Donc \[14\et9=1\qquad\text{et}\qquad1=2\times14-3\times9.\]
\end{ex}

\subsection{PGCD de plusieurs entiers}

\begin{rem}
La loi \(\et\) est une loi de composition interne associative et commutative sur \(\Z\).
\end{rem}

\begin{dem}\thlabel{dem:pgcdLCIsurZ}
La loi \(\et\) est clairement commutative.

Montrons qu'elle est associative.

Soient \(a,b,c\in\Z\).

On a \[\ensdiv{a}\inter\ensdiv{b}\inter\ensdiv{c}=\ensdiv{a}\inter\ensdiv{b\et c}=\ensdiv{a\et\paren{b\et c}}\] et \[\ensdiv{a}\inter\ensdiv{b}\inter\ensdiv{c}=\ensdiv{a\et b}\inter\ensdiv{c}=\ensdiv{\paren{a\et b}\et c}.\]

Donc \(a\paren{b\et c}\) et \(\paren{a\et b}\et c\) se divisent mutuellement.

De plus, ce sont des entiers positifs.

Donc ils sont égaux.
\end{dem}

\begin{defprop}
Soient \(r\in\Ns\) et \(a_1,\dots,a_r\in\Z\).

Les diviseurs positifs communs à \(a_1,\dots,a_r\) sont les diviseurs de \(a_1\et\dots\et a_r\).

Ce nombre est le PGCD de \(a_1,\dots,a_r\).
\end{defprop}

\begin{prop}
Soient \(r\in\Ns\) et \(a_1,\dots,a_r,\lambda\in\Z\).

On a : \[\paren{\lambda a_1}\et\dots\et\paren{\lambda a_r}=\abs{\lambda}\paren{a_1\et\dots\et a_r}.\]
\end{prop}

\begin{dem}
Découle de la \thref{prop:PGCDFoisLambdaEgal} par récurrence sur \(r\in\Ns\).
\end{dem}

\section{Entiers premiers entre eux}\label{sec:entiersPremiersEntreEux}

\subsection{Cas de deux entiers}

\begin{defi}[Entiers premiers entre eux]
Deux entiers \(a,b\in\Z\) sont dits premiers entre eux s'ils vérifient : \[a\et b=1.\]

Cela signifie que leurs seuls diviseurs communs sont \(1\) et \(-1\).
\end{defi}

\begin{theo}[Théorème de Bézout]
Soient \(a,b\in\Z\).

On a : \[a\text{ et }b\text{ premiers entre eux}\ssi\quantifs{\exists u,v\in\Z}ua+vb=1.\]
\end{theo}

\begin{dem}
\impdir Déjà vu (\cf \thref{defprop:relationDeBezout}).

\imprec

Supposons qu'il existe \(u,v\in\Z\) tels que \(ua+bv=1\).

On a \(a\et b\) divise \(a\) et \(b\).

Donc \(a\et b\divise ua+vb\).

Donc \(a\et b\divise1\).

Donc \(a\et b=1\) car \(a\et b\in\N\).
\end{dem}

\begin{prop}
Soient \(\lambda,x,y,n\in\Z\).

On suppose que \(\lambda\) et \(n\) sont premiers entre eux.

Alors : \[x\equiv y\croch{n}\ssi\lambda x\equiv\lambda y\croch{n}.\]
\end{prop}

\begin{dem}
Soient \(u,v\in\Z\) tels que \(u\lambda+vn=1\).

On a \(u\lambda\equiv1\croch{n}\).

Montrons l'équivalence.

\impdir Vraie : il suffit de multiplier par \(\lambda\).

\imprec Vraie : il suffit de multiplier par \(u\).
\end{dem}

\begin{lem}[Lemme de Gauss]
Soient \(a,b,n\in\Z\).

On suppose : \[n\divise ab\qquad\text{et}\qquad n\et b=1.\]

Alors : \[n\divise a.\]
\end{lem}

\begin{dem}
Soient \(u,v\in\Z\) tels que \(un+vb=1\).

On a \(n\divise ab\). Soit \(k\in\Z\) tel que \(nk=ab\).

On a \(a\paren{un+vb}=a\) donc \(nau+vnk=a\).

Donc \(n\paren{au+vk}=a\).

Donc \(n\divise a\).
\end{dem}

\begin{rem}[Forme irréductible d'un rationnel]
Soit \(q\in\Q\).

Il existe un unique couple \(\paren{a,b}\in\Z\times\Ns\) tel que : \[q=\dfrac{a}{b}\qquad\text{et}\qquad a\et b=1.\]
\end{rem}

\begin{dem}\thlabel{dem:formeIrreductibleD'unRationnel}
Soit \(\paren{a,b}\in\Z\times\Ns\) tel que \(q=\dfrac{a}{b}\).

\existence

On pose \(a\prim=\dfrac{a}{a\et b}\) et \(b\prim=\dfrac{b}{a\et b}\) (on a \(a\prim\in\Z\) et \(b\prim\in\Ns\) car \(a\et b\) divise \(a\) et \(b\)).

On a \(q=\dfrac{a\prim}{b\prim}\).

Montrons que \(a\prim\et b\prim=1\).

Comme \(a\et b\geq0\), d'après la \thref{prop:PGCDFoisLambdaEgal}, on a \[a\et b=\paren{\paren{a\et b}a\prim}\et\paren{\paren{a\et b}\et b}=\paren{a\et b}\paren{a\prim\et b\prim}.\]

Or \(a\et b\not=0\) donc \(1=a\prim\et b\prim\).

\unicite

Soit \(\paren{a\seconde,b\seconde}\in\Z\times\Ns\) tel que \(\begin{dcases}a\seconde\et b\seconde=1 \\ q=\dfrac{a\seconde}{b\seconde}\end{dcases}\)

Montrons que \(\paren{a\prim,b\prim}=\paren{a\seconde,b\seconde}\).

On a \(\dfrac{a\prim}{b\prim}=\dfrac{a\seconde}{b\seconde}\) donc \(a\prim b\seconde=b\prim a\seconde\).

Donc \(b\seconde\divise b\prim a\seconde\) et \(b\seconde\et a\seconde=1\) donc \(b\seconde\divise b\prim\) selon le lemme de Gauss.

De même, \(b\prim\divise b\seconde\) donc \(b\prim=b\seconde\) donc \(a\prim=a\seconde\).
\end{dem}

\begin{prop}
Soient \(a,b,n\in\Z\).

On suppose \[a\et n=1\qquad\text{et}\qquad b\et n=1.\]

Alors \[ab\et n=1.\]
\end{prop}

\begin{dem}
Soient \(u,v,u\prim,v\prim\in\Z\) tels que \(\begin{dcases}ua+vn=1 \\ u\prim b+v\prim n=1\end{dcases}\)

On a, par produit : \[\underbrace{uu\prim}_{\in\Z}ab+n\underbrace{\paren{uav\prim+vbu\prim+vv\prim n}}_{\in\Z}=1\]

Donc \(ab\et n=1\).
\end{dem}

\begin{cor}
Soient \(r\in\Ns\) et \(a_1,\dots,a_r,n\in\Z\).

On suppose \[\quantifs{\forall k\in\interventierii{1}{r}}a_k\et n=1.\]

Alors \[\paren{a_1\dots a_r}\et n=1.\]
\end{cor}

\begin{dem}
Par récurrence immédiate sur \(r\in\Ns\).
\end{dem}

\begin{prop}\thlabel{prop:deuxEntiersPremiersEntreEuxDivisantUnNombreImpliqueProduitDiviseCeNombre}
Soient \(a,b,n\in\Z\).

On suppose \[a\divise n\qquad\text{et}\qquad b\divise n\qquad\text{et}\qquad a\et b=1.\]

Alors \[ab\divise n.\]
\end{prop}

\begin{dem}
Comme \(a\divise n\), il existe \(\lambda\in\Z\) tel que \(a\lambda=n\).

On a \(b\divise a\lambda\) et \(b\et a=1\).

Donc selon le lemme de Gauss : \(b\divise\lambda\).

Donc \(ab\divise a\lambda\).

Donc \(ab\divise n\).
\end{dem}

\subsection{Cas de plusieurs entiers}

\begin{defi}
Soient \(r\in\Ns\) et \(a_1,\dots,a_r\in\Z\).

On dit que les entiers \(a_1,\dots,a_r\) sont premiers entre eux deux à deux si on a : \[\quantifs{\forall i,j\in\interventierii{1}{r}}i\not=j\imp a_i\et a_j=1.\]

On dit que les entiers \(a_1,\dots,a_r\) sont premiers entre eux dans leur ensemble si \(a_1\et\dots\et a_r=1\). Cela signifie que leurs seuls diviseurs communs sont \(1\) et \(-1\).
\end{defi}

\begin{rem}
Soient \(r\in\N\excluant\accol{0;1}\) et \(a_1,\dots,a_r\in\Z\).

\guillemets{\(a_1,\dots,a_r\) sont premiers entre eux deux à deux} implique \guillemets{\(a_1,\dots,a_r\) sont premiers entre eux dans leur ensemble}.

L'implication réciproque est fausse, par exemple : \(6\), \(10\) et \(15\) sont premiers entre eux dans leur ensemble mais pas deux à deux.
\end{rem}

\section{PPCM}

\begin{defprop}
Soient \(a,b\in\Z\).

Dans l'ensemble ordonné \(\groupe{\N}[\divise]\), il existe un plus petit multiple commun positif à \(a\) et \(b\) appelé le \guillemets{Plus Petit Commun Multiple} (PPCM) et noté \(a\ou b\).

On a \(\begin{dcases}a\ou b\geq0 \\ a\divise a\ou b \\ b\divise a\ou b \\ \quantifs{\forall m\in\N}\croch{a\divise m\quad\text{et}\quad b\divise m}\imp a\ou b\divise m\end{dcases}\)
\end{defprop}

\begin{dem}
L'ensemble \(a\Z\inter b\Z\) est un sous-groupe de \(\groupe{\Z}\).

Donc il existe \(\mu\in\N\) tel que \(a\Z\inter b\Z=\mu\Z\).

On a \(\mu\geq0\).

Comme \(\mu\in a\Z\inter b\Z\), on a \(\begin{dcases}a\divise\mu \\ b\divise\mu\end{dcases}\)

Enfin, soit \(m\in\N\) tel que \(\begin{dcases}a\divise m \\ b\divise m\end{dcases}\)

On a \(m\in a\Z\inter b\Z\) donc \(m\in\mu\Z\).

Donc \(\mu\divise m\).
\end{dem}

\begin{rem}
On a :

\begin{itemize}
\item \(\quantifs{\forall a,b\in\Z}a\ou b=\abs{a}\ou\abs{b}\) \\

\item \(\quantifs{\forall a,b\in\N}a\ou b=\sup\accol{a;b}\) dans \(\groupe{\N}[\divise]\)
\end{itemize}
\end{rem}

\begin{prop}
\(\ou\) est une loi de composition interne associative et commutative sur \(\Z\).
\end{prop}

\begin{dem}
\(\ou\) est clairement commutative.

Montrons que \(\ou\) est associative.

Soient \(a,b,c\in\Z\).

D'une part, \[a\Z\inter b\Z\inter c\Z=\paren{a\ou b}\Z\inter c\Z=\paren{\paren{a\ou b}\ou c}\Z.\]

D'autre part, \[a\Z\inter b\Z\inter c\Z=a\Z\inter\paren{b\ou c}\Z=\paren{a\ou\paren{b\ou c}}\Z.\]

Donc \(\paren{a\ou b}\ou c\) et \(a\ou\paren{b\ou c}\) se divisent mutuellement.

De plus, ils appartiennent à \(\N\) donc ils égaux.
\end{dem}

\begin{defi}
Soient \(r\in\Ns\) et \(a_1,\dots,a_r\in\Z\).

L'élément \(a_1\ou\dots\ou a_r\) est le plus petit (pour \(\divise\)) multiple commun à \(a_1,\dots,a_r\).

On l'appelle le PPCM de \(a_1,\dots,a_r\).
\end{defi}

\begin{ex}
\(9\ou12=36\) car \(9\N=\accol{0;9;18;27;36;\dots}\) et \(12\N=\accol{0;12;24;36;\dots}\).

On a \[\quantifs{\forall n\in\Z}\begin{dcases}n\ou1=\abs{n} \\ n\ou0=0 \\ n\ou2=\begin{dcases}n &\text{si \(n\) est pair} \\ 2n &\text{sinon}\end{dcases}\end{dcases}\]
\end{ex}

\begin{rem}
Soient \(a,b\in\Z\).

On a \[a\ou b=\abs{b}\ssi a\divise b.\]
\end{rem}

\begin{dem}
\impdir

Supposons \(a\ou b=\abs{b}\).

On a \(a\divise\abs{b}\).

Or \(\abs{b}\divise b\).

Donc \(a\divise b\).

\imprec

Supposons \(a\divise b\).

On a \[\quantifs{\forall m\in\N}\croch{a\divise m\quad\text{et}\quad b\divise m}\ssi b\divise m.\]

Donc \(a\ou b=\abs{b}\).
\end{dem}

\begin{prop}
On a

\begin{enumerate}
\item \(\quantifs{\forall a,b,\lambda\in\Z}\paren{\lambda a}\ou\paren{\lambda b}=\abs{\lambda}\paren{a\ou b}\) \\

\item \(\quantifs{\forall r\in\Ns;\forall a_1,\dots,a_r,\lambda\in\Z}\paren{\lambda a_1}\ou\dots\ou\paren{\lambda a_r}=\abs{\lambda}\paren{a_1\ou\dots\ou a_r}\)
\end{enumerate}
\end{prop}

\begin{dem}[1]
Montrons que \(\paren{\lambda a}\ou\paren{\lambda b}\divise\lambda\paren{a\ou b}\).

On a \(\lambda a\) et \(\lambda b\) divisent \(\lambda\paren{a\ou b}\) donc \(\paren{\lambda a}\ou\paren{\lambda b}\divise\lambda\paren{a\ou b}\).

Montrons que \(\lambda\paren{a\ou b}\divise\paren{\lambda a}\ou\paren{\lambda b}\).

On a \(\lambda\divise\paren{\lambda a}\ou\paren{\lambda b}\) (car \(\lambda\divise\lambda a\divise\paren{\lambda a}\ou\paren{\lambda b}\)).

Soit \(m\in\Z\) tel que \(\lambda m=\paren{\lambda a}\ou\paren{\lambda b}\).

\Cad, en supposant \(\lambda\not=0\) : \(a\ou b\divise m\).

On a \(\lambda a\) et \(\lambda b\) divisent \(\paren{\lambda a}\ou\paren{\lambda b}\) donc \(\lambda a\) et \(\lambda b\) divisent \(\lambda m\).

Donc \(a\) et \(b\) divisent \(m\) car \(\lambda\not=0\).

Donc \(a\ou b\divise m\).

Donc \(\lambda\paren{a\ou b}\divise\lambda m\).

Finalement, \(\paren{\lambda a}\ou\paren{\lambda b}\) et \(\lambda\paren{a\ou b}\) se divisent mutuellement donc ils ont même valeur absolue.
\end{dem}

\begin{dem}[2]
Découle de (1) par récurrence sur \(r\in\Ns\).
\end{dem}

\begin{prop}
Soient \(a,b\in\Z\).

On a \[\paren{a\ou b}\paren{a\et b}=\abs{ab}.\]
\end{prop}

\begin{dem}
Si \(a\et b=1\) :

On a, selon la \thref{prop:deuxEntiersPremiersEntreEuxDivisantUnNombreImpliqueProduitDiviseCeNombre} : \[\quantifs{\forall m\in\Z}\begin{dcases}a\divise m \\ b\divise m\end{dcases}\ssi ab\divise m.\]

Donc \(a\ou b=\abs{ab}\) donc \(\paren{a\ou b}\paren{a\et b}=\abs{ab}\).

Sinon :

On pose \(a\prim=\dfrac{a}{a\et b}\) et \(b\prim=\dfrac{b}{a\et b}\).

Alors \(a\prim\in\Z\) et \(b\prim\in\Ns\) (\cf \thref{dem:formeIrreductibleD'unRationnel}).

On a donc, selon le cas précédent : \(\paren{a\prim\ou b\prim}\paren{a\prim\et b\prim}=\abs{a\prim b\prim}\).

Puis, en multipliant de chaque côté par \(\paren{a\et b}^2\) : \[\croch{\underbrace{\paren{\paren{a\et b}a\prim}}_{a}\et\underbrace{\paren{\paren{a\et b}b\prim}}_b}\croch{\underbrace{\paren{\paren{a\et b}a\prim}}_a\ou\underbrace{\paren{\paren{a\et b}b\prim}}_b}=\abs{\paren{a\et b}a\prim\paren{a\et b}b\prim}=\abs{ab}.\]
\end{dem}

\section{Nombres premiers}

\subsection{Définition}

\begin{defi}[Nombre premier]
On appelle nombre premier tout entier \(p\in\interventierie{2}{\pinf}\) dont les seuls diviseurs positifs sont \(1\) et lui-même.

L'ensemble des nombres premiers est souvent noté \(\prem\) : \[\quantifs{\forall p\in\Z}p\in\prem\ssi\croch{p\geq2\quad\text{et}\quad\ensdiv{p}=\accol{1;p}}.\]
\end{defi}

\begin{defi}[Nombre composé]
On appelle nombre composé tout entier \(n\in\interventierie{2}{\pinf}\) qui n'est pas un nombre premier, \cad vérifiant : \[\quantifs{\exists a,b\in\interventierie{2}{\pinf}}ab=n.\]
\end{defi}

\begin{ex}
Les entiers \(0\) et \(1\) ne sont ni des nombres premiers, ni des nombres composés.

Les entiers \(2\), \(3\), \(5\), \(7\), \(11\) sont des nombres premiers.

Les entiers \(4\), \(6\), \(8\), \(9\), \(10\) sont des nombres composés.
\end{ex}

\begin{rem}
Soient \(p\in\prem\) et \(n\in\Z\).

On a \[p\not\divise n\ssi p\et n=1.\]
\end{rem}

\begin{dem}
Si \(p\divise n\) alors \(p\et n=\abs{n}\).

Si \(p\not\divise n\) alors \(\begin{dcases}p\et n\in\ensdiv{p} \\ p\et n\in\ensdiv{n}\end{dcases}\)

Donc \(\begin{dcases}p\et n\in\accol{1;p} \\ p\et n\not=p\end{dcases}\)

Donc \(p\et n=1\).

Finalement, \(p\et n=1\ssi p\not\divise n\).
\end{dem}

\begin{lem}
Soit \(n\in\interventierie{2}{\pinf}\).

Alors \(n\) admet un diviseur premier.
\end{lem}

\begin{dem}
On raisonne par récurrence forte sur \(n\).

On pose \[\quantifs{\forall n\in\interventierie{2}{\pinf}}\P{n}:\croch{\quantifs{\exists p\in\prem}p\divise n}.\]

On a \(2\divise 2\). D'où \(\P{2}\).

Soit \(n\in\interventierie{3}{\pinf}\) tel que \(\quantifs{\forall k\in\interventierii{2}{n-1}}\P{k}\).

Montrons \(\P{n}\).

Si \(n\in\prem\) alors on a \(\P{n}\) car \(n\divise n\).

Sinon, \(n\) admet un diviseur positif autre que \(1\) et lui-même qu'on note \(d\).

On a \(1<d<n\).

Selon \(\P{d}\), il existe \(p\in\prem\) tel que \(p\divise d\).

D'où \(p\divise n\).

D'où, par récurrence forte, \(\quantifs{\forall n\in\interventierie{2}{\pinf}}\P{n}\).
\end{dem}

\begin{theo}
Il existe une infinité de nombres premiers.
\end{theo}

\begin{dem}
Soit \(n\in\N\).

Posons \(N=n!+1\). On a \(N\geq2\).

Soit \(p\in\prem\) tel que \(p\divise N\) (un tel \(p\) existe selon le lemme précédent).

On a \(p\divise n!+1\).

Montrons que \(p>n\).

Par l'absurde, si \(p\leq n\) alors \(p\divise n!\) donc \(p\divise1\) : contradiction.

On a donc montré \(\quantifs{\forall n\in\N;\exists p\in\prem}p>n\).

Donc \(\prem\) n'est pas majorée.

Donc \(\prem\) n'est pas finie.
\end{dem}

\begin{rem}[Crible d'Eratosthène]
Soit \(N\in\interventierie{2}{\pinf}\).

Tout nombre composé \(n\in\interventierii{1}{N}\) admet un diviseur premier \(p\) tel que \(p\leq N\).
\end{rem}

\begin{dem}
Soit un nombre composé \(n\in\interventierii{1}{N}\).

Il existe \(a,b\in\N\) tels que \(\begin{dcases}ab=n \\ 2\leq a<n \\ 2\leq b<n\end{dcases}\)

Quitte à échanger \(a\) et \(b\), on peut supposer \(a\leq\sqrt{n}\).

Soit, selon le lemme précédent, \(p\in\prem\) tel que \(p\divise a\).

On a \(\begin{dcases}p\divise n &\text{car }p\divise a\text{ et }a\divise n \\ p\leq\sqrt{n} &\text{car }p\leq a\leq\sqrt{n}\end{dcases}\)

On peut ainsi déterminer tous les nombres premiers de \(\interventierii{1}{N}\).
\end{dem}

\begin{ex}
Avec \(N=25\) (et donc \(\sqrt{N}=5\)) :

\begin{center}
\begin{tabular}{ccccc}
\textcolor{red}{1} & \textcolor{green}{2} & \textcolor{green}{3} & \textcolor{red}{4} & \textcolor{green}{5} \\
\textcolor{red}{6} & \textcolor{green}{7} & \textcolor{red}{8} & \textcolor{red}{9} & \textcolor{red}{10} \\
\textcolor{green}{11} & \textcolor{red}{12} & \textcolor{green}{13} & \textcolor{red}{14} & \textcolor{red}{15} \\
\textcolor{red}{16} & \textcolor{green}{17} & \textcolor{red}{18} & \textcolor{green}{19} & \textcolor{red}{20} \\
\textcolor{red}{21} & \textcolor{red}{22} & \textcolor{green}{23} & \textcolor{red}{24} & \textcolor{red}{25}
\end{tabular}
\end{center}
\end{ex}

\subsection{Théorème fondamental}

Le théorème suivant affirme que tout entier strictement positif s'écrit comme produit de nombres premiers, de façon unique à l'ordre des facteurs près.

\begin{theo}[Théorème fondamental de l'arithmétique]
Soit \(n\in\Ns\).

Alors \(n\) s'écrit de façon unique sous la forme : \[n=p_1^{\alpha_1}p_2^{\alpha_2}\dots p_r^{\alpha_r}\qquad\text{où}\qquad\begin{dcases}r\in\N \\ p_1,p_2,\dots,p_r\in\prem \\ p_1<p_2<\dots<p_r \\ \alpha_1,\alpha_2,\dots,\alpha_r\in\Ns\end{dcases}\]
\end{theo}

\begin{dem}
Pour tout \(n\in\Ns\), on note \(\P{n}\) la proposition \guillemets{le théorème est vrai pour \(n\)}.

On raisonne par récurrence forte.

On a clairement \(\P{1}\) car \(1\) est le \guillemets{produit vide} (avec \(r=0\)), et seulement le produit vide.

On a clairement \(\P{2}\) car \(2=p_1^{\alpha_1}\) avec \(r=1\), \(p_1=2\) et \(\alpha_1=1\).

Soit \(n\in\interventierie{3}{\pinf}\) tel que \(\quantifs{\forall k\in\interventierii{1}{n-1}}\P{k}\).

Montrons \(\P{n}\).

\existence

Si \(n\in\prem\) alors l'existence est claire.

Sinon, il existe \(a,b\in\N\) tels que \(\begin{dcases}n=ab \\ 2\leq a<n \\ 2\leq b<n\end{dcases}\)

Selon \(\P{a}\), l'entier \(a\) est produit de nombres premiers.

Selon \(\P{b}\), l'entier \(b\) est produit de nombres premiers.

Donc \(n\) est produit de nombres premiers.

\unicite

Soient \(r,s\in\N\), \(p_1,\dots,p_r,q_1,\dots,q_s\in\prem\) et \(\alpha_1,\dots,\alpha_r,\beta_1,\dots,\beta_s\in\Ns\) tels que \[\begin{dcases}n=p_1^{\alpha_1}\dots p_r^{\alpha_r}=q_1^{\beta_1}\dots q_s^{\beta_s} \\ p_1<\dots<p_r \\ q_1<\dots<q_s\end{dcases}\]

Montrons que \(p_1=q_1\).

Par l'absurde, supposons \(p_1\not=q_1\).

Si \(p_1<q_1\) alors \(\quantifs{\forall l\in\interventierii{1}{s}}p_1<q_l\).

Donc \(\quantifs{\forall l\in\interventierii{1}{s}}p_1\et q_l=1\).

D'où par produit, selon la \thref{prop:deuxEntiersPremiersEntreEuxDivisantUnNombreImpliqueProduitDiviseCeNombre}, \(p_1\et\paren{q_1^{\beta_1}\dots q_s^{\beta_s}}=1\).

Donc \(p_1\et n=1\) et \(p_1\divise n\) : contradiction.

Si \(q_1<p_1\), on montre de même que \(q_1\et n=1\) : contradiction.

Donc \(p_1=q_1\).

Ainsi, \(\dfrac{n}{p_1}=p_1^{\alpha_1-1}\dots p_r^{\alpha_r}=q_1^{\beta_1-1}\dots q_s^{\beta_s}\).

Selon la partie \unicite de \(\P{\dfrac{n}{p_1}}\), on a \[\begin{dcases}r=t \\ \quantifs{\forall k\in\interventierii{2}{s}}\begin{dcases}p_k=q_k \\ \alpha_k=\beta_k\end{dcases} \\ \alpha_1=\beta_1\end{dcases}\]

D'où l'unicité.

D'où \(\P{n}\).

Donc \(\quantifs{\forall n\in\Ns}\P{n}\).
\end{dem}

\subsection{Valuations \(p\)-adiques}

\begin{defi}[Valuation \(p\)-adique]
Pour tout entier strictement positif \(n\in\Ns\) et tout nombre premier \(p\in\prem\), on note \(\valp{p}{n}\) l'exposant de \(p\) dans la décomposition de \(\abs{n}\) en produit de nombres premiers.
\end{defi}

\begin{ex}
On a \(40=2^3\times5\).

Donc \[\valp{2}{40}=3\qquad\valp{5}{40}=1\qquad\quantifs{\forall p\in\prem\excluant\accol{2;5}}\valp{p}{40}=0.\]
\end{ex}

\begin{nota}[Fonction signum]
On pose \[\fonction{\sg}{\R}{\R}{x}{\begin{dcases}1 &\text{si }x>0 \\ 0 &\text{si }x=0 \\ -1 &\text{si }x<0\end{dcases}}\]
\end{nota}

\begin{nota}
Soit \(n\in\Zs\).

La valuation \(p\)-adique de \(n\) est nulle pour tous les nombres premiers \(p\in\prem\) sauf un nombre fini d'entre eux (on dit que la famille \(\paren{\valp{p}{n}}_{p\in\prem}\) est une famille d'entiers \guillemets{presque tous nuls}).

Pour cette raison, on s'autorise à définir le \guillemets{produit infini} : \[\prod_{p\in\prem}p^{\valp{p}{n}}\] comme étant le produit de ses facteurs différents de \(1\).

Sa valeur est donc en fait un produit fini : \[\prod_{p\in\prem}p^{\valp{p}{n}}=\prod_{\substack{p\in\prem \\ \valp{p}{n}\not=0}}p^{\valp{p}{n}}.\]
\end{nota}

\begin{cor}
On a, selon le théorème fondamental de l'arithmétique : \[\quantifs{\forall n\in\Zs}n=\sg\paren{n}\times\prod_{p\in\prem}p^{\valp{p}{n}}.\]
\end{cor}

\begin{prop}[Valuations d'un produit]
Soient \(a,b\in\Zs\).

On a : \[\quantifs{\forall p\in\prem}\valp{p}{ab}=\valp{p}{a}+\valp{p}{b}.\]
\end{prop}

\begin{dem}~\\
On a \(\begin{dcases}a=\prod_{p\in\prem}p^{\valp{p}{a}} \\ b=\prod_{p\in\prem}p^{\valp{p}{b}}\end{dcases}\) donc \[ab=\prod_{p\in\prem}p^{\valp{p}{a}+\valp{p}{b}}=\prod_{p\in\prem}p^{\valp{p}{ab}}.\]

D'où l'égalité, par unicité de l'écriture de \(ab\) en produit de nombres premiers.
\end{dem}

\begin{prop}[Caractérisation de la divisibilité]\thlabel{prop:caractérisationDeLaDivisibilité}
Soient \(a,b\in\Zs\).

On a \[a\divise b\ssi\quantifs{\forall p\in\prem}\valp{p}{a}\leq\valp{p}{b}.\]
\end{prop}

\begin{dem}
\imprec

Supposons \(\quantifs{\forall p\in\prem}\valp{p}{a}\leq\valp{p}{b}\).

Alors on a \[b=\prod_{p\in\prem}p^{\valp{p}{b}}=\prod_{p\in\prem}p^{\valp{p}{a}}\times\prod_{p\in\prem}p^{\valp{p}{b}-\valp{p}{a}}=\abs{a}\times\underbrace{\prod_{p\in\prem}p^{\overbrace{\valp{p}{b}-\valp{p}{a}}^{\text{entiers presque tous nuls}}}}_{\in\N}\]

Donc \(\abs{a}\divise\abs{b}\).

Donc \(a\divise b\).

\imprec

Supposons \(a\divise b\).

Soit \(c\in\Z\) tel que \(ac=b\).

Comme \(b\not=0\), on a \(c\not=0\).

On a \[\paren{\sg\paren{a}\times\prod_{p\in\prem}p^{\valp{p}{a}}}\paren{\sg\paren{c}\times\prod_{p\in\prem}p^{\valp{p}{c}}}=\sg\paren{b}\times\prod_{p\in\prem}p^{\valp{p}{b}}\]

D'où, selon l'unicité de l'écriture d'un entier en produit de nombres premiers : \[\quantifs{\forall p\in\prem}\valp{p}{a}+\underbrace{\valp{p}{c}}_{\in\N}=\valp{p}{b}.\]

Donc \(\quantifs{\forall p\in\prem}\valp{p}{a}\leq\valp{p}{b}\).
\end{dem}

\begin{rem}[Caractérisation de \(\valp{p}{n}\)]
En particulier, si \(n\in\Zs\) et \(p\in\prem\), alors \[\quantifs{\forall\alpha\in\N}\alpha=\valp{p}{n}\ssi\begin{dcases}p^\alpha\divise n \\ p^{\alpha+1}\not\divise n\end{dcases}\]
\end{rem}

\begin{prop}[PGCD et PPCM]
Soient \(a,b\in\Zs\).

On a : \[a\et b=\prod_{p\in\prem}p^{\min\accol{\valp{p}{a};\valp{p}{b}}}\qquad\text{et}\qquad a\ou b=\prod_{p\in\prem}p^{\max\accol{\valp{p}{a};\valp{p}{b}}}\]
\end{prop}

\begin{dem}
Posons \(d=\prod_{p\in\prem}p^{\min\accol{\valp{p}{a};\valp{p}{b}}}\).

Montrons que \(a\et b=d\).

On a \(d\divise a\) et \(d\divise b\) selon la \thref{prop:caractérisationDeLaDivisibilité}.

On a \(d\geq0\).

Enfin, soit \(e\in\N\) tel que \(e\divise a\) et \(e\divise b\).

On a \(e\not=0\) (car \(a\not=0\)).

On a donc, selon la \thref{prop:caractérisationDeLaDivisibilité} : \[\quantifs{\forall p\in\prem}\begin{dcases}\valp{p}{e}\leq\valp{p}{a} \\ \valp{p}{e}\leq\valp{p}{b}\end{dcases}\]

Donc \(\quantifs{\forall p\in\prem}\valp{p}{e}\leq\min\accol{\valp{p}{a};\valp{p}{b}}=\valp{p}{d}\).

D'où \(e\divise d\) selon la \thref{prop:caractérisationDeLaDivisibilité}.

Finalement, \(d=a\et b\).

Idem pour \(a\ou b\).
\end{dem}

\section{Petit théorème de Fermat}

\begin{lem}
Soient \(p\in\prem\) et \(k\in\interventierii{1}{p-1}\).

Alors \(p\divise\binom{k}{p}\).
\end{lem}

\begin{dem}~\\
On a \(\binom{k}{p}=\dfrac{p!}{k!\,\paren{p-k}!}\).

Donc \(p!=\binom{k}{p}k!\,\paren{p-k}!\).

Donc \(p\divise\binom{k}{p}k!\,\paren{p-k}!\).

Or \(p\et k!=1\) car \(k<p\) et \(p\et\paren{p-k}!=1\) car \(k>0\).

Donc \(p\et k!\,\paren{p-k}!=1\).

Donc \(p\divise\binom{k}{p}\) selon le lemme de Gauss.
\end{dem}

\begin{theo}[Petit théorème de Fermat]\thlabel{theo:petitThéorèmeDeFermat}
Soit \(p\in\prem\).

On a, pour tout \(x\in\Z\) : \[x^p\equiv x\croch{p}.\]

De plus, pour tout entier \(x\) qui n'est pas divisible par \(p\) : \[x^{p-1}\equiv1\croch{p}.\]
\end{theo}

\begin{dem}
Montrons que \(\quantifs{\forall x\in\N}\underbrace{x^p\equiv x\croch{p}}_{\P{x}}\) par récurrence sur \(x\in\N\).

On a clairement \(\P{0}\) et \(\P{1}\).

Soit \(x\in\N\) tel que \(\P{x}\).

Montrons \(\P{x+1}\).

On a \[\begin{WithArrows}
\paren{x+1}^p&\equiv\sum_{k=0}^p\binom{k}{p}x^k\croch{p} \Arrow{car \(\binom{k}{p}\equiv0\croch{p}\) si \(1\leq k\leq p-1\)} \\
&\equiv x^0+x^p\croch{p} \Arrow{selon \(\P{x}\)} \\
&\equiv1+x\croch{p}
\end{WithArrows}\]

D'où \(\P{x+1}\).

Ainsi, \(\quantifs{\forall x\in\N}x^p\equiv x\croch{p}\).

Enfin, si \(x\in\Z\) alors il existe \(y\in\N\) tel que \(x\equiv y\croch{p}\).

On a alors \(x^p\equiv y^p\equiv y\equiv x\croch{p}\).
\end{dem}