\chapter{Matrices II}

\minitoc

On considère un corps \(\K\) (en pratique \(\K=\R\) ou \(\C\)).

\section{Matrice d'une famille de vecteurs}

\subsection{Définition}

\begin{defi}[Matrice d'une famille de vecteurs dans une base]
Soient \(E\) un \(\K\)-espace vectoriel de dimension \(n\in\Ns\), \(\fami{B}=\paren{e_1,\dots,e_n}\) une base de \(E\) et \(\paren{x_1,\dots,x_p}\in E^p\) une famille d'éléments de \(E\).

La matrice de la famille \(\paren{x_1,\dots,x_p}\) dans la base \(\fami{B}\) est la matrice : \[\Mat{x_1,\dots,x_p}=\begin{pmatrix}
a_{11} & \dots & a_{1p} \\
\vdots &  & \vdots \\
a_{n1} & \dots & a_{np}
\end{pmatrix}\in\M{np}\] dont les colonnes sont les coordonnées respectives des vecteurs de la famille \(\paren{x_1,\dots,x_p}\) : \[\quantifs{\forall j\in\interventierii{1}{n}}x_j=\sum_{i=1}^{n}a_{ij}e_i.\]
\end{defi}

\begin{rem}
On garde les notations de la définition précédente.

Si \(p=1\) (\cad si la famille de vecteurs ne possède qu'un seul vecteur), alors la matrice \(\Mat{x_1}\) de la famille \(\paren{x_1}\) dans la base \(\fami{B}\) est la matrice-colonne (\cad le \(n\)-uplet) des coordonnées de \(x_1\) dans la base \(\fami{B}\).
\end{rem}

\begin{exoex}
On note \(T_0\), \(T_1\) et \(T_2\) les trois premiers polynômes de Tchebychev.

Écrire la matrice de la famille \(\paren{T_0,T_1,T_2}\) dans la base canonique de \(\polydeg[\R]{2}\) puis dans la base canonique de \(\polydeg[\R]{3}\).
\end{exoex}

\begin{corr}
On a : \[\Mat[\paren{1,X,X^2}]{1,X,2X^2-1}=\begin{pmatrix}
1 & 0 & -1 \\
0 & 1 & 0 \\
0 & 0 & 2
\end{pmatrix}\qquad\text{et}\qquad\Mat[\paren{1,X,X^2,X^3}]{1,X,2X^2-1}=\begin{pmatrix}
1 & 0 & -1 \\
0 & 1 & 0 \\
0 & 0 & 2 \\
0 & 0 & 0
\end{pmatrix}.\]
\end{corr}

\begin{defi}[Matrice de passage]
Soient \(E\) un \(\K\)-espace vectoriel de dimension \(n\in\Ns\) et \(\fami{B}=\paren{e_1,\dots,e_n}\) et \(\fami{B}\prim=\paren{x_1,\dots,x_n}\) deux bases de \(E\).

La matrice \(\Mat{x_1,\dots,x_n}\) est appelée matrice de passage de la base \(\fami{B}\) à la base \(\fami{B}\prim\) et est notée \(\pass{\fami{B}}{\fami{B}\prim}\) : \[\pass{\fami{B}}{\fami{B}\prim}=\Mat{\fami{B}\prim}=\Mat{x_1,\dots,x_n}.\]
\end{defi}

\begin{rem}
Soient \(E\) un \(\K\)-espace vectoriel de dimension \(n\in\Ns\) et \(\fami{B}\) une base de \(E\).

On a : \[\pass{\fami{B}}{\fami{B}}=I_n.\]
\end{rem}

\subsection{Formules de changement de base}

\begin{prop}[Formule de changement de base pour les coordonnées d'un vecteur]
Soient \(E\) un \(\K\)-espace vectoriel de dimension \(n\in\Ns\), \(\fami{B}\) et \(\fami{B}\prim\) deux bases de \(E\) et \(x\) un vecteur de \(E\).

On considère les coordonnées \(X\in\K^n\) et \(X\prim\in\K^n\) de \(x\) dans les bases \(\fami{B}\) et \(\fami{B}\prim\) : \[X=\tcoords{x_1}{\vdots}{x_n}=\Mat{x}\qquad\text{et}\qquad X\prim=\tcoords{x_1\prim}{\vdots}{x_n\prim}=\Mat[\fami{B}\prim]{x}.\]

On a : \[X=\pass{\fami{B}}{\fami{B}\prim}X\prim,\] \cad : \[\Mat{x}=\pass{\fami{B}}{\fami{B}\prim}\Mat[\fami{B}\prim]{x}.\]
\end{prop}

\begin{dem}
On note \(\fami{B}=\paren{e_1,\dots,e_n}\), \(\fami{B}\prim=\paren{e_1\prim,\dots,e_n\prim}\) et \(\pass{\fami{B}}{\fami{B}\prim}=\paren{a_{ij}}_{\paren{i,j}}\).

On a : \[x=\sum_{k=1}^{n}x_ke_k=\sum_{k=1}^{n}x_k\prim e_k\prim\qquad\text{et}\qquad\quantifs{\forall j\in\interventierii{1}{n}}e_j\prim=\sum_{i=1}^{n}a_{ij}e_i.\]

Donc : \[x=\sum_{j=1}^{n}x_j\prim\sum_{i=1}^{n}a_{ij}e_i=\sum_{i=1}^{n}e_i\sum_{j=1}^{n}a_{ij}x_j\prim.\]

D'où \(\quantifs{\forall i\in\interventierii{1}{n}}x_i=\sum_{j=1}^{n}a_{ij}x_j\prim\), \cad : \[X=\pass{\fami{B}}{\fami{B}\prim}X\prim.\]
\end{dem}

\begin{exoex}
On note \(\fami{C}\) le cercle unité de \(\R^2\), d'équation cartésienne (dans la base canonique \(\fami{B}\)) : \[\fami{C}:x^2+y^2=1.\]

\begin{enumerate}
    \item Justifier que la famille \(\fami{B}\prim=\paren{\dcoords{2}{0},\dcoords{1}{1}}\) est une base de \(\R^2\). \\
    \item Donner une équation cartésienne de \(\fami{C}\) dans la base \(\fami{B}\prim\), \cad, pour tout point \(M\in\R^2\), une CNS en fonction des coordonnées \(\paren{x\prim,y\prim}\) de \(M\) dans \(\fami{B}\prim\) pour que \(M\) appartienne à \(\fami{C}\).
\end{enumerate}
\end{exoex}

\begin{corr}[1]
\(\fami{B}\prim\) est clairement libre et possède deux éléments donc c'est une base de \(\R^2\).
\end{corr}

\begin{corr}[2]
Soit \(M\in\R^2\) dont on note \(\paren{x,y}\) les coordonnées dans \(\fami{B}\) et \(\paren{x\prim,y\prim}\) les coordonnées dans \(\fami{B}\prim\).

On a : \[\dcoords{x}{y}=\pass{\fami{B}}{\fami{B}\prim}\dcoords{x\prim}{y\prim}=\begin{pmatrix}
2 & 1 \\
0 & 1
\end{pmatrix}\dcoords{x\prim}{y\prim}=\dcoords{2x\prim+y\prim}{y\prim}.\]

Donc \(\begin{dcases}
x=2x\prim+y\prim \\
y=y\prim
\end{dcases}\)

D'où : \[\begin{aligned}
M\in\fami{C}&\ssi x^2+y^2=1 \\
&\ssi\paren{2x\prim+y\prim}^2+{y\prim}^2=1 \\
&\ssi4{x\prim}^2+4x\prim y\prim+2{y\prim}^2=1.
\end{aligned}\]
\end{corr}

\begin{prop}\thlabel{prop:formuleDeChangementDeBasePourLaMatriceD'UneFamilleDeVecteurs}
{\normalfont\bfseries(Formule de changement de base pour la matrice d'une famille de vecteurs)}

Soient \(E\) un \(\K\)-espace vectoriel de dimension \(n\in\Ns\), \(\fami{B}\) et \(\fami{B}\prim\) deux bases de \(E\) et \(\fami{F}\) une famille de vecteurs.

On a : \[\Mat{\fami{F}}=\pass{\fami{B}}{\fami{B}\prim}\Mat[\fami{B}\prim]{\fami{F}}.\]
\end{prop}

\begin{dem}
On note \(\fami{F}=\paren{x_1,\dots,x_p}\in E^p\), \(\fami{B}=\paren{e_1,\dots,e_n}\), \(\fami{B}\prim=\paren{e_1\prim,\dots,e_n\prim}\), \(\Mat{\fami{F}}=\begin{pmatrix}C_1 & \dots & C_p\end{pmatrix}\) et \(\Mat[\fami{B}\prim]{\fami{F}}=\begin{pmatrix}C_1\prim & \dots & C_p\prim\end{pmatrix}\).

On a \(\quantifs{\forall j\in\interventierii{1}{p}}C_j=\pass{\fami{B}}{\fami{B}\prim}C_j\prim\) donc : \[\begin{aligned}
\Mat{\fami{F}}&=\begin{pmatrix}\pass{\fami{B}}{\fami{B}\prim}C_1\prim & \dots & \pass{\fami{B}}{\fami{B}\prim}C_p\prim\end{pmatrix} \\
&=\pass{\fami{B}}{\fami{B}\prim}\begin{pmatrix}C_1\prim & \dots & C_p\prim\end{pmatrix} \\
&=\pass{\fami{B}}{\fami{B}\prim}\Mat[\fami{B}\prim]{\fami{F}}.
\end{aligned}\]
\end{dem}

\begin{cor}
Soient \(E\) un \(\K\)-espace vectoriel de dimension \(n\in\Ns\) et \(\fami{B}\), \(\fami{B}\prim\) et \(\fami{B}\seconde\) trois bases de \(E\).

On a : \[\pass{\fami{B}}{\fami{B}\seconde}=\pass{\fami{B}}{\fami{B}\prim}\pass{\fami{B}\prim}{\fami{B}\seconde}.\]
\end{cor}

\begin{dem}
On a, selon la \thref{prop:formuleDeChangementDeBasePourLaMatriceD'UneFamilleDeVecteurs} : \[\Mat{\fami{B}\seconde}=\pass{\fami{B}}{\fami{B}\prim}\Mat[\fami{B}\prim]{\fami{B}\seconde}.\]

Comme \(\fami{B}\seconde\) est aussi une base, cette formule s'écrit : \[\pass{\fami{B}}{\fami{B}\seconde}=\pass{\fami{B}}{\fami{B}\prim}\pass{\fami{B}\prim}{\fami{B}\seconde}.\]
\end{dem}

\begin{prop}[Inversibilité des matrices de passage]
Soient \(E\) un \(\K\)-espace vectoriel de dimension \(n\in\Ns\) et \(\fami{B}\) et \(\fami{B}\prim\) deux bases de \(E\).

La matrice de passage \(\pass{\fami{B}}{\fami{B}\prim}\) est inversible, d'inverse : \[\pass{\fami{B}}{\fami{B}\prim}\inv=\pass{\fami{B}\prim}{\fami{B}}.\]
\end{prop}

\begin{dem}
On a : \[I_n=\pass{\fami{B}}{\fami{B}}=\pass{\fami{B}}{\fami{B}\prim}\pass{\fami{B}\prim}{\fami{B}}.\]

Donc \(\pass{\fami{B}}{\fami{B}\prim}\) est inversible, d'inverse \(\pass{\fami{B}\prim}{\fami{B}}\).
\end{dem}