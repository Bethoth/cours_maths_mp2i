\chapter{Matrices II}

\minitoc

On considère un corps \(\K\) (en pratique \(\K=\R\) ou \(\C\)).

\section{Matrice d'une famille de vecteurs}

\subsection{Définition}

\begin{defi}[Matrice d'une famille de vecteurs dans une base]
Soient \(E\) un \(\K\)-espace vectoriel de dimension \(n\in\Ns\), \(\fami{B}=\paren{e_1,\dots,e_n}\) une base de \(E\) et \(\paren{x_1,\dots,x_p}\in E^p\) une famille d'éléments de \(E\).

La matrice de la famille \(\paren{x_1,\dots,x_p}\) dans la base \(\fami{B}\) est la matrice : \[\Mat{x_1,\dots,x_p}=\begin{pmatrix}
a_{11} & \dots & a_{1p} \\
\vdots &  & \vdots \\
a_{n1} & \dots & a_{np}
\end{pmatrix}\in\M{np}\] dont les colonnes sont les coordonnées respectives des vecteurs de la famille \(\paren{x_1,\dots,x_p}\) : \[\quantifs{\forall j\in\interventierii{1}{n}}x_j=\sum_{i=1}^{n}a_{ij}e_i.\]
\end{defi}

\begin{rem}
On garde les notations de la définition précédente.

Si \(p=1\) (\cad si la famille de vecteurs ne possède qu'un seul vecteur), alors la matrice \(\Mat{x_1}\) de la famille \(\paren{x_1}\) dans la base \(\fami{B}\) est la matrice-colonne (\cad le \(n\)-uplet) des coordonnées de \(x_1\) dans la base \(\fami{B}\).
\end{rem}

\begin{exoex}
On note \(T_0\), \(T_1\) et \(T_2\) les trois premiers polynômes de Tchebychev.

Écrire la matrice de la famille \(\paren{T_0,T_1,T_2}\) dans la base canonique de \(\polydeg[\R]{2}\) puis dans la base canonique de \(\polydeg[\R]{3}\).
\end{exoex}

\begin{corr}
On a : \[\Mat[\paren{1,X,X^2}]{1,X,2X^2-1}=\begin{pmatrix}
1 & 0 & -1 \\
0 & 1 & 0 \\
0 & 0 & 2
\end{pmatrix}\qquad\text{et}\qquad\Mat[\paren{1,X,X^2,X^3}]{1,X,2X^2-1}=\begin{pmatrix}
1 & 0 & -1 \\
0 & 1 & 0 \\
0 & 0 & 2 \\
0 & 0 & 0
\end{pmatrix}.\]
\end{corr}

\begin{defi}[Matrice de passage]
Soient \(E\) un \(\K\)-espace vectoriel de dimension \(n\in\Ns\) et \(\fami{B}=\paren{e_1,\dots,e_n}\) et \(\fami{B}\prim=\paren{x_1,\dots,x_n}\) deux bases de \(E\).

La matrice \(\Mat{x_1,\dots,x_n}\) est appelée matrice de passage de la base \(\fami{B}\) à la base \(\fami{B}\prim\) et est notée \(\pass{\fami{B}}{\fami{B}\prim}\) : \[\pass{\fami{B}}{\fami{B}\prim}=\Mat{\fami{B}\prim}=\Mat{x_1,\dots,x_n}.\]
\end{defi}

\begin{rem}
Soient \(E\) un \(\K\)-espace vectoriel de dimension \(n\in\Ns\) et \(\fami{B}\) une base de \(E\).

On a : \[\pass{\fami{B}}{\fami{B}}=I_n.\]
\end{rem}