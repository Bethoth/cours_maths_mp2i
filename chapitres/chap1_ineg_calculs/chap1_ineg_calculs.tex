\chapter{Inégalités, calculs}

\minitoc

\section{Inégalités dans \(\R\)}

\subsection{Parties de \(\R\)}

\begin{defi}
Soient \(A\subset\R\) et \(m,M\in\R\).

On dit que \(M\) est un majorant de \(A\) et que \(M\) majore \(A\) si on a \(\quantifs{\forall x\in A}x\leq M\).

On dit que \(m\) est un minorant de \(A\) et que \(m\) minore \(A\) si on a \(\quantifs{\forall x\in A}m\leq x\).

On dit que la partie \(A\) est majorée si elle admet un majorant.

On dit que la partie \(A\) est minorée si elle admet un minorant.

On dit que la partie \(A\) est bornée si elle admet un majorant et un minorant.

Ainsi, \begin{itemize}
\item \(A\) est majorée \(\ssi\quantifs{\exists\lambda\in\R;\forall x\in A}x\leq\lambda\)
\item \(A\) est minorée \(\ssi\quantifs{\exists\mu\in\R;\forall x\in A}\mu\leq x\)
\item \(A\) est bornée \(\ssi\quantifs{\exists\lambda,\mu\in\R;\forall x\in A}\mu\leq x\leq\lambda\ssi\quantifs{\exists\lambda\in\Rp;\forall x\in A}\abs{x}\leq x\)
\end{itemize}
\end{defi}

\begin{ex}
\begin{itemize}
\item \(\intervii{0}{1}\) est bornée (majorants possibles : \(1\), \(\pi\), ... et minorants possibles : \(0\), \(-10\), ...).

\item \(\N\) est minorée par \(0\) mais non-majorée et donc non-bornée.

\item \(\Z\) n'est ni majorée ni minorée et donc non-bornée.
\end{itemize}
\end{ex}

\begin{rem}
Il n'y a jamais unicité du majorant ou du minorant.
\end{rem}

\begin{defi}
Soient \(A\subset\R\) et \(a\in A\).

On dit que \(a\) est le plus grand élément de \(A\) (ou maximum de \(A\)) si on a \(\quantifs{\forall b\in A}b\leq a\).

On dit que \(a\) est le plus petit élément de \(A\) (ou minimum de \(A\)) si on a \(\quantifs{\forall b\in A}a\leq b\).
\end{defi}

\begin{prop}
S'il existe, le plus grand élément de \(A\) est unique et est noté \(\max A\).

S'il existe, le plus petit élément de \(A\) est unique et est noté \(\min A\).
\end{prop}

\begin{dem}
Montrons l'unicité du plus grand élément de \(A\).

Soient \(a_1,a_2\in A\) tels que \(\quantifs{\forall b\in A}\begin{dcases}b\leq a_1 \\ b\leq a_2\end{dcases}\)

On a en particulier \(\begin{dcases}a_1\leq a_2\text{ avec \(b=a_1\)} \\ a_2\leq a_1\text{ avec \(b=a_2\)}\end{dcases}\)

Donc \(a_1=a_2\). Donc le plus grand élément est unique.

On montre de même l'unicité du plus petit élément.
\end{dem}

\begin{rem}[Plus grand élément \(\imp\) majorant]
Soient \(A\subset\R\) et \(x\in\R\).

\(x\) est le plus grand élément de \(A\) ssi \(\begin{dcases}x\in A \\ x\text{ majore \(A\)}\end{dcases}\)

En particulier, pour que \(A\) admette un plus grand élément, il faut que \(A\) soit majorée.
\end{rem}

\begin{ex}
\begin{itemize}
\item \(\intervii{0}{1}\) admet \(1\) comme plus grand élément et \(0\) comme plus petit élément.

\item \(\Rp\) admet \(0\) comme plus petit élément mais n'admet pas de plus grand élément (car \(\Rp\) n'est pas majorée).

\item \(\intervie{0}{1}\) admet \(0\) comme plus petit élément et est majorée (par \(1\)) mais n'admet pas de plus grand élément.

En effet, par l'absurde :

Soit \(a\in\intervie{0}{1}\). Supposons que \(a\) est le plus grand élément de \(\intervie{0}{1}\).

Comme \(a\in\intervie{0}{1}\), on a \(0\leq a<1\).

Posons \(b=\dfrac{a+1}{2}\).

On a d'une part \(0\leq a+1<2\) donc \(0\leq\dfrac{a+1}{2}<1\) donc \(0\leq b<1\) donc \(b\in\intervie{0}{1}\).

D'autre part, \(a+a<a+1\) donc \(\dfrac{a+a}{2}<\dfrac{a+1}{2}\) donc \(a<b\).

Donc \(a\) ne majore pas \(\intervie{0}{1}\) : contradiction.
\end{itemize}
\end{ex}

\begin{rem}
Toute partie finie admet un plus grand élément.
\end{rem}

\begin{theo}
Toute partie non-vide de \(\N\) admet un plus petit élément.
\end{theo}

\begin{dem}
\note{ADMIS} (fait partie de la définition de \(\N\), hors programme).
\end{dem}

\begin{cor}
Toute partie non-vide de \(\Z\) minorée admet un plus petit élément.

Toute partie non-vide de \(\Z\) majorée admet un plus grand élément.
\end{cor}

\begin{dem}
Soit \(A\subset\Z\) non-vide et minorée.

Soit \(m\in\R\) un minorant de \(A\).

Soit \(m\prim\in\Z\) tel que \(m\prim\leq m\).

On a \(\quantifs{\forall x\in A}m\prim\leq m\leq x\).

Donc \(m\prim\) minore \(A\).

Posons \(B=\accol{x-m\prim}_{x\in A}\).

On remarque \(B\not=\ensvide\) car \(A\not=\ensvide\) et \(B\subset\N\) car \(\quantifs{\forall x\in A}\begin{dcases}x-m\prim\in\Z\text{ car \(x,m\prim\in\Z\)} \\ x-m\prim\geq0\text{ car \(m\prim\leq x\)}\end{dcases}\)

Ainsi, \(B\) est une partie non-vide de \(\N\).

Donc \(B\) admet un plus petit élément \(b_0\in B\) : \(\quantifs{\forall b\in B}b_0\leq b\).

Cet élément \(b_0\) s'écrit \(b_0=a_0-m\prim\) par définition de \(B\).

On a \(\quantifs{\forall x\in A}a_0-m\prim\leq x-m\prim\) donc \(\quantifs{\forall x\in A}a_0\leq x\).

Donc \(a_0=\min A\).

On montre de même que toute partie \(C\) majorée non-vide de \(\Z\) admet un maximum, en considérant un majorant \(M\in\Z\) de \(C\) puis l'ensemble \(D=\accol{M-x}_{x\in C}\).
\end{dem}

\begin{defi}[Intervalle]
Soit \(I\subset\R\).

On dit que \(I\) est un intervalle de \(\R\) si on a \(\quantifs{\forall x,y\in I;\forall z\in\R}x\leq z\leq y\imp z\in I\).

Ie : \(\quantifs{\forall x,y\in I}\intervii{x}{y}\subset I\), en notant \(\intervii{x}{y}=\accol{z\in\R\tq x\leq z\leq y}\) avec donc \(\intervii{x}{y}=\ensvide\) si \(x>y\).
\end{defi}

\begin{ex}
\begin{itemize}
\item \(\ensvide\) et \(\R\) sont des intervalles de \(\R\).

\item \(\intervii{a}{b}\), \(\intervie{a}{b}\), \(\intervee{a}{+\infty}\), ... sont des intervalles de \(\R\) pour tous \(a,b\in\R\).

\item \(\N\), \(\Z\), \(\Q\), ... ne sont pas des intervalles de \(\R\). Par exemple, on a \(0\leq\dfrac{1}{2}\leq1\) mais \(\dfrac{1}{2}\not\in\N\).
\end{itemize}
\end{ex}

\subsection{Manipulation d'inégalités, fonctions}

\begin{prop}
Soit \(I\) un intervalle de \(\R\).

Soit \(f:I\to\R\) dérivable.

\begin{itemize}
\item \(f\) est constante \(\ssi\quantifs{\forall x\in I}f\prim\paren{x}=0\)

\item \(f\) est croissante \(\ssi\quantifs{\forall x\in I}f\prim\paren{x}\geq0\)

\item \(f\) est strictement croissante \(\impr\quantifs{\forall x\in I}f\prim\paren{x}>0\)

\(f\) est strictement croissante \(\ssi\begin{dcases}\quantifs{\forall x\in I}f\prim\paren{x}\geq0 \\ \quantifs{\forall x,y\in I}x<y\imp\quantifs{\exists z\in\intervii{x}{y}}f\prim\paren{z}>0\end{dcases}\)
\end{itemize}
\end{prop}

\begin{dem}
\note{ADMIS} temporairement.
\end{dem}

\begin{rem}
Il ne faut pas oublier l'hypothèse selon laquelle \(I\) est un intervalle.

Par exemple, si \(I=\Rs\) alors \(I\) n'est pas un intervalle de \(\R\) (on a par exemple \(-1\leq0\leq1\) mais \(0\not\in\Rs\)) et les fonctions \(\fonction{f}{\Rs}{\R}{x}{\begin{dcases}1&\text{ si \(x>0\)} \\ -1&\text{ si \(x<0\)}\end{dcases}}\) et \(\fonction{g}{\Rs}{\R}{x}{\dfrac{1}{x}}\) vérifient : \begin{itemize}
\item \(f\prim=0\) mais \(f\) non-constante

\item \(\quantifs{\forall x\in\Rs}g\prim\paren{x}=\dfrac{-1}{x^2}<0\) mais \(g\) n'est pas strictement décroissante.
\end{itemize}
\end{rem}

\begin{rem}
Soient \(A\subset\R\) et \(f:A\to\R\). On considère les propositions suivantes :

\begin{enumerate}
\item \(f\) croissante

\item \(f\) strictement croissante

\item \(\quantifs{\forall x,y\in A}x\leq y\imp f\paren{x}\leq f\paren{y}\)

\item \(\quantifs{\forall x,y\in A}x<y\imp f\paren{x}<f\paren{y}\)

\item \(\quantifs{\forall x,y\in A}x\leq y\ssi f\paren{x}\leq f\paren{y}\)

\item \(\quantifs{\forall x,y\in A}x<y\ssi f\paren{x}<f\paren{y}\)
\end{enumerate}

Les propositions (1) et (3) sont équivalentes.

Les propositions (2), (4), (5) et (6) sont équivalentes.
\end{rem}

\begin{rem}
Quand on rédige un raisonnement par équivalences, la façon correcte d'écrire est la suivante : \guillemets{\(x\leq y\ssi f\paren{x}\leq f\paren{y}\) car \(f\) est strictement croissante}.
\end{rem}

\begin{dem}
\begin{itemize}
\item (1) \(\ssi\) (3) : par définition

\item (2) \(\ssi\) (4) : par définition

\item (6) \(\ssi\) (5) : par contraposition

\item (6) \(\imp\) (4) : clair

\item Il ne reste plus qu'à montrer que (4) \(\imp\) (6).

Supposons (4), montrons (6).

Soient \(x,y\in A\). Montrons que \(x<y\ssi f\paren{x}<f\paren{y}\).

\begin{itemize}
\item[\impdir] Ok selon (4)

\item[\imprec] Supposons \(f\paren{x}<f\paren{y}\). Montrons que \(x<y\).

Si \(x=y\) alors \(f\paren{x}=f\paren{y}\) : impossible.

Si \(x>y\) alors \(f\paren{x}>f\paren{y}\) selon (4) : impossible.

Donc \(x<y\), d'où (6).
\end{itemize}
\end{itemize}
\end{dem}

\begin{prop}
Soient \(a,b,c,d,\lambda\in\R\).

On a \begin{enumerate}
\item \(\begin{dcases}a\leq b \\ c\leq d\end{dcases}\imp a+c\leq b+d\)

\item \(\begin{dcases}a\leq b \\ \lambda\geq0\end{dcases}\imp\lambda a\leq\lambda b\)

\item \(\begin{dcases}a\leq b \\ \lambda\leq0\end{dcases}\imp\lambda a\geq\lambda b\)

\item \(\begin{dcases}0\leq a\leq b \\ 0\leq c\leq d\end{dcases}\imp ac\leq bd\)

\item \(\begin{dcases}a\leq b \\ ab>0\end{dcases}\imp\dfrac{1}{a}\geq\dfrac{1}{b}\)
\end{enumerate}
\end{prop}

\begin{dem}
\begin{enumerate}
\item Supposons \(a\leq b\) et \(c\leq d\).

On a \(a+c\leq b+c\) car \(x\mapsto x+c\) croissante.

Donc \(a+c\leq b+d\) car \(c\leq d\).

\item Supposons \(\lambda\geq0\).

La fonction \(x\mapsto\lambda x\) est de dérivée positive sur l'intervalle \(\R\) donc elle est croissante.

\item Idem.

\item Idem.

\item C'est la décroissance de la fonction inverse sur les intervalles \(\Rps\) et \(\Rms\) (en effet, \(ab>0\ssi\paren{a,b\in\Rps\text{ ou }a,b\in\Rms}\)).
\end{enumerate}
\end{dem}

\subsection{Valeur absolue}

\begin{nota}
On pose \(\quantifs{\forall x\in\R}\abs{x}=\max\accol{x;-x}=\begin{dcases}x&\text{ si \(x\geq0\)} \\ -x&\text{ si \(x<0\)}\end{dcases}\)
\end{nota}

\begin{rem}
On a \(\quantifs{\forall x\in\Rp}\sqrt{x}^2=x\) et \(\quantifs{\forall x\in\R}\sqrt{x^2}=\abs{x}\).

Ainsi, on a \(\quantifs{\forall x,y\in\R}\begin{dcases}x^2=y^2\ssi\abs{x}=\abs{y} \\ x^2\leq y^2\ssi\abs{x}\leq\abs{y}\text{ car \(x\mapsto\sqrt{x}\) strictement croissante}\end{dcases}\)
\end{rem}

\begin{prop}[Inégalité triangulaire]
Soient \(x,y\in\R\).

On a \(\abs{\abs{x}-\abs{y}}\leq\abs{x+y}\leq\abs{x}+\abs{y}\).

Et donc \(\abs{\abs{x}-\abs{y}}\leq\abs{x-y}\leq\abs{x}+\abs{y}\) en remplaçant \(y\) par \(-y\).
\end{prop}

\begin{dem}
\begin{enumerate}
\item Montrons que \(\abs{x+y}\leq\abs{x}+\abs{y}\).

Si \(x\geq0\) et \(y\geq0\) alors \(x+y\geq0\) donc \(\abs{x+y}=x+y=\abs{x}+\abs{y}\).

Si \(x<0\) et \(y<0\) alors \(x+y<0\) donc \(\abs{x+y}=-x-y=\abs{x}+\abs{y}\).

Si \(x\geq0\) et \(y<0\) alors \(\begin{dcases}x+y\leq x-y=\abs{x}+\abs{y}\text{ car \(y<0\)} \\ -x-y\leq x-y=\abs{x}+\abs{y}\text{ car \(x\geq0\)}\end{dcases}\) donc \(\abs{x+y}=\max\accol{x+y;-x-y}\leq\abs{x}+\abs{y}\).

Si \(x<0\) et \(y\geq0\) : idem en échangeant \(x\) et \(y\).

\item On remarque \(\begin{aligned}[t]
\abs{x} &= \abs{x+y-y} \\
&\leq \abs{x+y}+\abs{-y}\text{ selon (1)} \\
&= \abs{x+y}+\abs{y}
\end{aligned}\)

Donc \(\abs{x}-\abs{y}\leq\abs{x+y}\).

De même, \(\abs{y}=\abs{y+x-x}\leq\abs{y+x}+\abs{x}\).

Donc \(\abs{y}-\abs{x}\leq\abs{y+x}\).

Finalement, \(\abs{\abs{x}-\abs{y}}=\max\accol{\abs{x}-\abs{y};\abs{y}-\abs{x}}\leq\abs{x+y}\).
\end{enumerate}
\end{dem}

\begin{prop}
Soient \(x,y\in\R\).

On a \(\abs{xy}\leq\dfrac{x^2+y^2}{2}\).
\end{prop}

\begin{dem}
On a \(\paren{x-y}^2\geq0\) et \(\paren{x+y}^2\geq0\).

Donc \(\begin{dcases}xy\leq\dfrac{x^2+y^2}{2} \\ -xy\leq\dfrac{x^2+y^2}{2}\end{dcases}\)

D'où \(\abs{xy}=\max\accol{xy;-xy}\leq\dfrac{x^2+y^2}{2}\).
\end{dem}

\subsection{Partie entière d'un réel}

\begin{defi}
Soit \(x\in\R\).

On appelle partie entière de \(x\) et on note \(\floor{x}\) le plus grand entier relatif \(n\in\Z\) tel que \(n\leq x\).

Un tel entier existe car \(\accol{n\in\Z\tq n\leq x}\) est une partie non-vide et majorée de \(\Z\).
\end{defi}

\begin{ex}
\(\quantifs{\forall n\in\Z}\floor{n}=n\) ; \(\floor{\dfrac{1}{2}}=1\) ; \(\floor{-\dfrac{1}{2}}=-2\).
\end{ex}

\begin{prop}
Soit \(x\in\R\).

On a \begin{enumerate}
\item \(\floor{x}\leq x<\floor{x}+1\)

\item \(\quantifs{\forall n\in\Z}n\leq x\ssi n\leq\floor{x}\)
\end{enumerate}
\end{prop}

\begin{dem}
\begin{enumerate}
\item On a \(\floor{x}\leq x\) par définition.

De plus, \(\floor{x}\) est le plus grand entier inférieur à \(x\) donc \(\floor{x}+1\) n'est pas un entier inférieur à \(x\).

\item Soit \(n\in\Z\).

\begin{itemize}
\item[\imprec] Claire car \(\floor{x}\leq x\).

\item[\impdir] Si \(n\leq x\) alors \(n\) est plus petit que le plus grand entier inférieur à \(x\).

Donc \(n\leq\floor{x}\).
\end{itemize}
\end{enumerate}
\end{dem}

\section{Sommes et produits}

\begin{nota}
On pose \(\quantifs{\forall a,b\in\Z}\interventierii{a}{b}=\accol{n\in\Z\tq a\leq n\leq b}\).
\end{nota}

\begin{nota}
Soient \(a,b\in\Z\) et \(f:\Z\to\R\).

\begin{itemize}
\item On pose \(\sum_{k=a}^b f\paren{k}=\begin{dcases}f\paren{a}+f\paren{a+1}+\dots+f\paren{b}&\text{ si \(a\leq b\)} \\ 0&\text{ si \(a>b\)}\end{dcases}\)

\item On pose \(\prod_{k=a}^b f\paren{k}=\begin{dcases}f\paren{a}\times f\paren{a+1}\times\dots\times f\paren{b}&\text{ si \(a\leq b\)} \\ 1&\text{ si \(a>b\)}\end{dcases}\)

\item Soit \(E\) un ensemble fini et \(f:E\to\R\).

On note \(\sum_{x\in E}f\paren{x}\) la somme des \(f\paren{x}\) pour \(x\in E\) (\(0\) si \(E=\ensvide\)).

Et \(\prod_{x\in E}f\paren{x}\) le produit des \(f\paren{x}\) pour \(x\in E\) (\(1\) si \(E=\ensvide\)).
\end{itemize}
\end{nota}

\begin{rem}
Les indices \(k\) et \(x\) sont des \guillemets{variables locales}, non définies en dehors des \(\sum\) et \(\prod\).

Ce sont aussi des \guillemets{variables muettes}, la somme ou le produit ne dépend pas du nom de la variable : \(\sum_{k=a}^b f\paren{k}=\sum_{l=a}^b f\paren{l}\).
\end{rem}

\begin{rem}[Changements d'indice]
\begin{itemize}
\item Soit \(n\in\interventierie{2}{\pinf}\).

On a \(\sum_{k=2}^n2^{k-2}\ln\paren{k+1}=\sum_{k=0}^{n-2}2^k\ln\paren{k+3}\).

\item Soit \(\paren{u_k}_{k\in\N}\) une suite de réels.

On a \(\sum_{k=0}^{n-1}u_{k+1}=\sum_{k=1}^n u_k\).
\end{itemize}
\end{rem}

\begin{prop}
Soient \(E\) et \(F\) deux ensembles finis. On suppose que \(E\) et \(F\) sont disjoints (ie \(E\cap F=\ensvide\)).

Soit \(f:E\cup F\to\R\).

Alors \(\sum_{x\in E\cup F}f\paren{x}=\sum_{x\in E}f\paren{x}+\sum_{x\in F}f\paren{x}\).

Et \(\prod_{x\in E\cup F}f\paren{x}=\prod_{x\in E}f\paren{x}\times\prod_{x\in F}f\paren{x}\).
\end{prop}

\begin{prop}[Sommes télescopiques]
Soient \(a,b\in\Z\) tels que \(a\leq b\) et \(\paren{u_n}_{n\in\N}\) une suite de réels.

On a \(\sum_{k=a}^b\paren{u_{k+1}-u_k}=u_{b+1}-u_a\).
\end{prop}

\begin{dem}~\\
On a \(\sum_{k=a}^b\paren{u_{k+1}-u_k}=\sum_{k=a}^b u_{k+1}-\sum_{k=a}^b u_k=\sum_{k=a+1}^{b+1}u_k-\sum_{k=a}^b u_k=u_{b+1}-u_a\).
\end{dem}

\begin{ex}
Soit \(n\in\Ns\). Calculer \(\sum_{k=1}^n\ln\paren{1+\dfrac{1}{k}}\) et \(\sum_{k=1}^n\dfrac{1}{k\paren{k+1}}\).

\begin{itemize}
\item On a \(\begin{aligned}[t]
\sum_{k=1}^n\ln\paren{1+\dfrac{1}{k}}&=\sum_{k=1}^n\ln\dfrac{k+1}{k} \\
&=\sum_{k=1}^n\paren{\ln\paren{k+1}-\ln k} \\
&=\sum_{k=1}^n\ln\paren{k+1}-\sum_{k=1}^n\ln k \\
&=\sum_{k=2}^{n+1}\ln k-\sum_{k=1}^n\ln k \\
&=\ln\paren{n+1}-\ln1 \\
&=\ln\paren{n+1}
\end{aligned}\)

\item On a \(\begin{aligned}[t]
\sum_{k=1}^n\dfrac{1}{k\paren{k+1}}&=\sum_{k=1}^n\dfrac{1+k-k}{k\paren{k+1}} \\
&=\sum_{k=1}^n\paren{\dfrac{1+k}{k\paren{k+1}}-\dfrac{k}{k\paren{k+1}}} \\
&=\sum_{k=1}^n\paren{\dfrac{1}{k}-\dfrac{1}{k+1}} \\
&=\sum_{k=1}^n\dfrac{1}{k}-\sum_{k=1}^n\dfrac{1}{k+1} \\
&=\sum_{k=1}^n\dfrac{1}{k}-\sum_{k=2}^{n+1}\dfrac{1}{k} \\
&=1-\dfrac{1}{n+1} \\
&=\dfrac{n}{n+1}
\end{aligned}\)
\end{itemize}
\end{ex}

\begin{prop}[Produits télescopiques]
Soient \(a,b\in\N\) tels que \(a\leq b\) et \(\paren{u_n}_{n\in\N}\) une suite de réels non-nuls.

Alors, \(\prod_{k=a}^b\dfrac{u_{k+1}}{u_k}=\dfrac{u_{b+1}}{u_a}\).
\end{prop}

\begin{prop}[Doubles sommes]
Soient \(a,b,c,d\in\Z\) et \(f:\Z^2\to\R\).

Alors, \(\sum_{i=a}^b\sum_{j=c}^d f\paren{i,j}=\sum_{\paren{i,j}\in\interventierii{a}{b}\times\interventierii{c}{d}}f\paren{i,j}=\sum_{j=c}^d\sum_{i=a}^b f\paren{i,j}\).

On appelle interversion le passage du membre de gauche au membre de droite.
\end{prop}

\begin{prop}[Produits doubles]
Soient \(a,b,c,d\in\Z\) et \(f:\Z^2\to\R\).

Alors, \(\prod_{i=a}^b\prod_{j=c}^d f\paren{i,j}=\prod_{j=c}^d\prod_{i=a}^b f\paren{i,j}\) (interversion).
\end{prop}

\begin{ex}~\\
On a \(\sum_{i=0}^1\sum_{j=0}^2 j^i=\paren{1+1+1}+\paren{0+1+2}=6\) et \(\sum_{j=0}^2\sum_{i=0}^1 j^i=\paren{1+0}+\paren{1+1}+\paren{1+2}=6\).
\end{ex}

\begin{prop}
Soient \(E\) et \(F\) deux ensembles finis et \(f:E\to\R\) et \(g:F\to\R\).

On a \(\sum_{x\in E}f\paren{x}\times\sum_{y\in F}g\paren{y}=\sum_{x\in E}\sum_{y\in F}f\paren{x}g\paren{y}\).
\end{prop}

\begin{ex}
Soient \(n\in\N\) et \(x_1,\dots,x_n\in\R\).

On a \(\paren{\sum_{k=1}^n x_k}^2=\sum_{k=1}^n x_k^2+2\sum_{1\leq k<l\leq n}x_kx_l\).
\end{ex}

\begin{rem}
La dernière somme pourrait s'écrire \(\sum_{\paren{k,l}\in E}\dots\ \) où \(E=\accol{\paren{k,l}\in\N^2\tq1\leq k<l\leq n}\) ou encore \(\sum_{k=1}^n\sum_{l=k+1}^n\dots\).
\end{rem}

\begin{dem}~\\
On a \(\begin{aligned}[t]
\paren{\sum_{k=1}^n x_k}^2&=\sum_{k=1}^n x_k\times\sum_{l=1}^n x_l \\
&=\sum_{k=1}^n\sum_{l=1}^n x_kx_l \\
&=\sum_{1\leq k=l\leq n}x_kx_l+\sum_{1\leq k<l\leq n}x_kx_l+\sum_{1\leq l<k\leq n}x_kx_l \\
&=\sum_{k=1}^n x_k^2+2\sum_{1\leq k<l\leq n}x_kx_l
\end{aligned}\)

~
\end{dem}

\section{Factorielle, coefficients binomiaux}

\begin{defi}[Factorielle]
Soit \(n\in\N\).

On pose \(n!=\prod_{k=1}^n k\).
\end{defi}

\begin{defi}[Coefficient binomial]
Soient \(n\in\N\) et \(p\in\Z\).

On pose \(\binom{p}{n}=\begin{dcases}\dfrac{n!}{p!\,\paren{n-p}!}&\text{ si \(0\leq p\leq n\)} \\ 0&\text{ si \(0>p\) ou \(p>n\)}\end{dcases}\)

\(\binom{p}{n}\) est le nombre de parties de cardinal \(p\) contenues dans un ensemble de cardinal \(n\).

Supposons \(0\leq p\leq n\). On a \(\dfrac{n!}{\paren{n-p}!}=n\paren{n-1}\dots\paren{n-p+1}\).

Donc \(\binom{p}{n}=\dfrac{n\paren{n-1}\dots\paren{n-p+1}}{p!}\).

On a \(\binom{0}{n}=\dfrac{1}{1}=1\) ; \(\binom{1}{n}=\dfrac{n}{1}=n\) ; \(\binom{2}{n}=\dfrac{n\paren{n-1}}{1\times2}=\dfrac{n\paren{n-1}}{2}\).
\end{defi}

\begin{prop}
Soient \(n\in\N\) et \(p\in\Z\).

On a \begin{enumerate}
\item \(\binom{p}{n}=\binom{n-p}{n}\)

\item \(\binom{p}{n}+\binom{p+1}{n}=\binom{p+1}{n+1}\)

\item Si \(p\not=0\) alors \(\binom{p}{n}=\dfrac{n}{p}\binom{p-1}{n-1}\)
\end{enumerate}
\end{prop}

\begin{dem}
\begin{enumerate}
\item On a \(0\leq p\leq n\ssi0\leq n-p\leq n\).

Si \(p<0\) ou \(n<p\), les deux coefficients binomiaux sont nuls.

Sinon, \(\binom{n-p}{n}=\dfrac{n!}{\paren{n-p}!\,\paren{n-\paren{n-p}}!}=\dfrac{n!}{\paren{n-p}!\,p!}=\binom{p}{n}\).

\item Si \(p<-1\), les trois coefficients binomiaux sont nuls.

Si \(p=-1\), on a bien \(0+1=1\).

Si \(p>n\), les trois coefficients binomiaux sont nuls.

Si \(p=n\), on a bien \(1+0=1\).

Supposons désormais \(0\leq p\leq n-1\).

Les trois coefficients binomiaux sont non-nuls et on a :

\(\begin{aligned}[t]
\binom{p}{n}+\binom{p+1}{n}&=\dfrac{n!}{p!\,\paren{n-p}!}+\dfrac{n!}{\paren{p+1}!\,\paren{n-p+1}!} \\
&=\dfrac{\paren{p+1}n!+\paren{n-p}n!}{\paren{n-p}!\,\paren{p+1}!} \\
&=\dfrac{\paren{n+1}n!}{\paren{n-p}!\,\paren{p+1}!} \\
&=\dfrac{\paren{n+1}!}{\paren{\paren{n+1}-\paren{p+1}}!\,\paren{p+1}!} \\
&=\binom{p+1}{n+1}
\end{aligned}\)

\item Si \(p<0\), on a bien \(0=0\).

Si \(p>n\), idem.

Supposons \(1\leq p\leq n\).

On a \(\binom{p}{n}=\dfrac{n\paren{n-1}\dots\paren{n-p+1}}{p!}=\dfrac{n}{p}\times\dfrac{\paren{n-1}\dots\paren{n-p+1}}{\paren{p-1}!}=\dfrac{n}{p}\binom{p-1}{n-1}\).
\end{enumerate}
\end{dem}

\begin{rem}
Le triangle de Pascal est complété à l'aide de (2) :

\(\begin{array}{c|ccccc}
&0&1&2&3&4 \\
\hline
0&\binom{0}{0}&&&& \\[1em]
1&\binom{0}{1}&\binom{1}{1}&&& \\[1em]
2&\binom{0}{2}&\binom{1}{2}&\binom{2}{2}&& \\[1em]
3&\binom{0}{3}&\binom{1}{3}&\binom{2}{3}&\binom{3}{3}&
\end{array}\) donnant \(\begin{array}{c|ccccc}
&0&1&2&3&4 \\
\hline
0&1&0&0&0&0 \\
1&1&1&0&0&0 \\
2&1&2&1&0&0 \\
3&1&3&3&1&0
\end{array}\)
\end{rem}

\begin{prop}[Formule du binôme de Newton]
Soient \(x,y\in\R\) et \(n\in\N\).

On a \(\paren{x+y}^n=\sum_{k=0}^n\binom{k}{n}x^ky^{n-k}=\sum_{k=0}^n\binom{k}{n}y^kx^{n-k}\)
\end{prop}

\begin{dem}
Par récurrence sur \(n\in\N\).

Pour tout \(n\in\N\), on note \(P\paren{n}\) la proposition \guillemets{\(\paren{x+y}^n=\sum_{k=0}^n\binom{k}{n}x^ky^{n-k}\)}.

\underline{Initialisation :}

On a bien \(\paren{x+y}^0=1\) et \(\sum_{k=0}^0\binom{k}{0}x^ky^{0-k}=\binom{0}{0}x^0y^0=1\).

D'où \(P\paren{0}\).

\underline{Hérédité :}

Soit \(n\in\N\) tel que \(P\paren{n}\). Montrons \(P\paren{n+1}\). On a :

\(\begin{aligned}[t]
\paren{x+y}^{n+1}&=\paren{x+y}\paren{x+y}^n \\
&=\paren{x+y}\sum_{k=0}^n\binom{k}{n}x^ky^{n-k}\text{ selon \(P\paren{n}\)} \\
&=\sum_{k=0}^n\binom{k}{n}x^{k+1}y^{n-k}+\sum_{k=0}^n\binom{k}{n}x^ky^{n-k+1} \\
&=\sum_{k=1}^{n+1}\binom{k-1}{n}x^ky^{n-k+1}+\sum_{k=0}^n\binom{k}{n}x^ky^{n-k+1} \\
&=\sum_{k=0}^{n+1}\binom{k-1}{n}x^ky^{n-k+1}+\sum_{k=0}^{n+1}\binom{k}{n}x^ky^{n-k+1}\text{ car \(\binom{-1}{n}=\binom{n+1}{n}=0\)} \\
&=\sum_{k=0}^{n+1}\paren{\binom{k-1}{n}+\binom{k}{n}}x^ky^{n-k+1} \\
&=\sum_{k=0}^{n+1}\binom{k}{n+1}x^ky^{n-k+1}
\end{aligned}\)

D'où \(P\paren{n+1}\).

\underline{Conclusion :}

On a \(\quantifs{\forall n\in\N}P\paren{n}\).
\end{dem}