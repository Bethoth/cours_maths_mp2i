\chapter{Polynômes, fractions rationnelles}

Soit \(\K\) un corps.

\section{Polynômes}

\begin{exo}[Exercice 1]
On pose \[P_1=X^4+3X-1\qquad P_2=X^2+4\qquad P_3=X+1.\]

\begin{enumerate}
\item Faire, pour tout \(\paren{i,j}\in\accol{1;2;3}\), la division euclidienne de \(P_i\) par \(P_j\). \\

\item Calculer \[P_3\paren{P_2}\qquad P_2\paren{P_3}\qquad P_2\paren{P_1}\qquad P_2\prim\paren{P_1}\qquad P_1\prim\paren{P_3}.\]
\end{enumerate}
\end{exo}

\begin{corr}
\note{À venir}
\end{corr}

\begin{exo}[Exercice 2]
\begin{enumerate}
\item Déterminer tous les polynômes \(P\in\poly[\R]\) tels que \[P\paren{X^3}=P^2.\]

\item Déterminer tous les polynômes \(P\in\poly[\R]\) tels que \[X^2P\paren{X^2}=P\paren{X^3}+2\paren{X-1}\paren{X^3+X+1}.\]
\end{enumerate}
\end{exo}

\begin{corr}
\note{À venir}
\end{corr}

\begin{exo}[Exercice 3]
\begin{enumerate}
\item Déterminer les polynômes \(P\) à coefficients complexes tels que \[P=P\prim P\seconde.\]

\item Déterminer les polynômes \(P\) à coefficients complexes tels que \[P=P\deriv{1}P\deriv{2}P\deriv{3}.\]

\item Déterminer les polynômes \(P\) à coefficients complexes tels que \[P=P\deriv{1}P\deriv{2}P\deriv{3}P\deriv{4}.\]
\end{enumerate}
\end{exo}

\begin{corr}
\note{À venir}
\end{corr}

\begin{exo}[Exercice 4]
Déterminer tous les polynômes \(P\in\poly[\R]\) tels que \(\begin{dcases}P\paren{-1}=-17 \\ P\paren{0}=-7 \\ P\paren{1}=-3 \\ P\paren{3}=35\end{dcases}\)

\textit{Indication :} commencer par calculer l'unique polynôme \(P\in\polydeg[\R]{3}\) solution du système.
\end{exo}

\begin{corr}
\note{À venir}
\end{corr}

\begin{exo}[Exercice 5]
Décomposer en produit de polynômes irréductibles dans \(\poly[\C]\) puis dans \(\poly[\R]\) les polynômes suivants : \[P_1=X^4-4\qquad P_2=X^3-2X^2+2X\qquad P_3=X^6+64.\]
\end{exo}

\begin{corr}
\note{À venir}
\end{corr}

\begin{exo}[Exercice 6]
Soit \(n\in\Ns\).

Calculer \[\prod_{\omega\in\U_n\excluant\accol{1}}\paren{X-\omega}.\]
\end{exo}

\begin{corr}
\note{À venir}
\end{corr}

\begin{exo}[Exercice 7]
Soit \(n\in\N\).

Calculer \[S_1=\sum_{k=0}^nk\binom{k}{n}\qquad S_2=\sum_{k=0}^nk^2\binom{k}{n}\qquad S_3=\sum_{k=0}^nk^3\binom{k}{n}.\]
\end{exo}

\begin{corr}
\note{À venir}
\end{corr}

\begin{exo}[Exercice 8, algorithme d'Euclide]
Appliquer l'algorithme d'Euclide étendu à \(A=X^8+X\) et \(B=X^5+X\) pour trouver un PGCD \(D\) de \(A\) et \(B\) et des polynômes \(U\) et \(V\) tels que \(UA+VB=D\).

\textit{Il s'agit d'un simple exercice d'application, ne pas chercher à ruser pour aller plus vite.}
\end{exo}

\begin{corr}
\note{À venir}
\end{corr}

\begin{exo}[Exercice 9]
Soient \(A,B\in\poly[\R]\).

Montrer que \(A\) divise \(B\) dans \(\poly[\R]\) si, et seulement si, \(A\) divise \(B\) dans \(\poly[\C]\).
\end{exo}

\begin{corr}
\note{À venir}
\end{corr}

\begin{exo}[Exercice 10]
Soient \(m,n\in\Ns\).

Montrer que \(X^n-1\) divise \(X^m-1\) dans \(\poly[\R]\) si, et seulement si, \(n\) divise \(m\) dans \(\Z\).
\end{exo}

\begin{corr}
\note{À venir}
\end{corr}

\begin{exo}[Exercice 11]
Déterminer les polynômes \(P\in\poly[\C]\) tels que \(P\prim\) divise \(P\).
\end{exo}

\begin{corr}
\note{À venir}
\end{corr}

\begin{exo}[Exercice 12]
Soit \(P\in\poly\).

Montrer que \(P-X\) divise \(P\rond P-X\).
\end{exo}

\begin{corr}
\note{À venir}
\end{corr}

\begin{exo}[Exercice 13, devinettes]
Soient \(a,b,c\in\R\) et \(P\in\poly[\R]\).

Quel est le reste de la division euclidienne de \(P\) par \(\paren{X-a}\paren{X-b}\paren{X-c}\)

\begin{enumerate}
\item si \(a\), \(b\) et \(c\) sont deux à deux distincts ? \\

\item si \(a=b=c\) ?
\end{enumerate}
\end{exo}

\begin{corr}
\note{À venir}
\end{corr}

\begin{exo}[Exercice 14]
Soit \(n\in\Ns\).

On pose \[P_n=\sum_{k=0}^n\dfrac{X^k}{k!}\].

Montrer que \(P_n\) est scindé à racines simples dans \(\poly[\C]\).
\end{exo}

\begin{corr}
\note{À venir}
\end{corr}

\begin{exo}[Exercice 15]
Soit \(n\in\Ns\).

\begin{enumerate}
\item Calculer \(\sum_{x\in\U_n}x^k\) en fonction de \(k\in\Z\). \\

\item Soient \(P\in\poly[\C]\) et \(M\in\Rp\) tels que \[\deg P<n\qquad\text{et}\qquad\quantifs{\forall x\in\U_n}\abs{P\paren{x}}\leq M.\]

Montrer que les coefficients de \(P\) sont de module majoré par \(M\).
\end{enumerate}
\end{exo}

\begin{corr}
\note{À venir}
\end{corr}

\begin{exo}[Exercice 16, classique]\thlabel{exo:classiqueSurLesPolynômesScindés}
Dans tout l'exercice, \guillemets{scindé} signifie \guillemets{scindé sur \(\R\)}.

Soit \(P\in\poly[\R]\).

\begin{enumerate}
\item Montrer que si \(P\) est scindé à racines simples, alors \(P\prim\) est scindé à racines simples. \\

\item Montrer que si \(P\) est scindé, alors \(P\prim\) est scindé.
\end{enumerate}
\end{exo}

\begin{corr}
\note{À venir}
\end{corr}

\begin{exo}[Exercice 17, suite de l'exercice précédent, moins classique]
Soit \(P\in\poly[\R]\).

On suppose que \(P\) s'écrit \(P=\sum_{k=0}^na_kX^k\) avec \[n\geq3\qquad a_0\not=0\qquad a_n\not=0\qquad\quantifs{\exists k\in\interventierii{1}{n-2}}a_k=a_{k+1}=0.\]

Montrer que \(P\) n'est pas scindé sur \(\R\).
\end{exo}

\begin{corr}
\note{À venir}
\end{corr}

\begin{exo}[Exercice 18]
On pose \(P=X^3+3X^2+3X+9\).

\begin{enumerate}
\item Montrer que toutes les racines complexes de \(P\) sont simples. \\

\item Calculer la somme des racines complexe de \(P\). \\

\item Calculer la somme des carrés des racines complexes de \(P\). \\

\item Calculer la somme des cubes des racines complexes de \(P\).
\end{enumerate}
\end{exo}

\begin{corr}
\note{À venir}
\end{corr}

\begin{exo}[Exercice 19]
On pose \(P=X^6+4X^5-3X^4-32X^3-53X^2-36X-9\).

\begin{enumerate}
\item Montrer que \(-1\) est racine de \(P\) et calculer sa multiplicité. \\

\item Déterminer les autres racines de \(P\).
\end{enumerate}
\end{exo}

\begin{corr}
\note{À venir}
\end{corr}

\section{Fractions rationnelles}

\begin{exo}[Exercice 20]
Soit \(F\in\fracrat[\C]\).

Montrer : \[F=\conj{F}\ssi F\in\fracrat[\R].\]
\end{exo}

\begin{corr}
\note{À venir}
\end{corr}

\begin{exo}[Exercice 21]
Calculer la décomposition en éléments simples des fractions rationnelles suivantes :

\begin{enumerate}
\item \(\dfrac{X^2+2X+5}{X^2-3X+2}\) \\

\item \(\dfrac{X^2+1}{\paren{X-1}\paren{X-2}\paren{X-3}}\) \\

\item \(\dfrac{4}{X^4-1}\) (sur \(\C\) puis sur \(\R\)) \\

\item \(\dfrac{1}{X\paren{X-1}^2}\) \\

\item \(\dfrac{2X}{X^2+1}\) (sur \(\C\)) \\

\item \(\dfrac{3X-1}{X^2\paren{X+1}^2}\)
\end{enumerate}
\end{exo}

\begin{corr}
\note{À venir}
\end{corr}

\begin{exo}[Exercice 22]
\begin{enumerate}
\item Rappeler les lois des groupes \(\fracrat\excluant\accol{0}\) et \(\fracrat\). \\

\item Montrer que \(\fonction{\phi}{\fracrat\excluant\accol{0}}{\fracrat}{F}{\dfrac{F\prim}{F}}\) est un morphisme de groupes. \\

Quel est son noyau ? \\

\item Soit \(F\in\fracrat[\C]\excluant\accol{0}\). \\

Quelle est la décomposition en éléments simples de \(\dfrac{F\prim}{F}\) ? On l'exprimera en fonction des racines et pôles de \(F\), et de leurs multiplicités respectives.
\end{enumerate}
\end{exo}

\begin{corr}
\note{À venir}
\end{corr}

\begin{exo}[Exercice 23]
Soient \(\lambda_1,\dots,\lambda_n\in\K\) et \(\mu\in\K\excluant\accol{0}\).

On pose \(P=\mu\paren{X-\lambda_1}\dots\paren{X-\lambda_n}\).

Exprimer en fonction de \(P\), \(P\prim\) et \(P\seconde\) les fractions rationnelles suivantes : \[F=\sum_{k=1}^n\dfrac{1}{X-\lambda_k}\qquad G=\sum_{k=1}^n\dfrac{1}{\paren{X-\lambda_k}^2}\qquad H=\sum_{1\leq k<l\leq n}\dfrac{1}{\paren{X-\lambda_k}\paren{X-\lambda_l}}.\]
\end{exo}

\begin{corr}
\note{À venir}
\end{corr}

\begin{exo}[Exercice 24]
Soient \(n\in\Ns\) et \(a_0,\dots,a_n\in\R\) tels que \(a_n\not=0\).

On suppose que le polynôme \(P=a_nX^n+\dots+a_0X^0\) est scindé sur \(\R\).

\begin{enumerate}
\item Montrer : \(\quantifs{\forall x\in\R}P\prim\paren{x}^2-P\paren{x}P\seconde\paren{x}\geq0\). \\

\item En déduire : \(\quantifs{\forall k\in\interventierii{0}{n-2}}a_ka_{k+2}\leq a_{k+1}^2\). \\

\textit{Indication :} utiliser l'\thref{exo:classiqueSurLesPolynômesScindés}.
\end{enumerate}
\end{exo}

\begin{corr}
\note{À venir}
\end{corr}

\begin{exo}[Exercice 25]
Soit \(n\in\Ns\).

On pose \(\omega=\e{\frac{2\i\pi}{n}}\).

\begin{enumerate}
\item Soit \(P\in\poly[\C]\) tel que \(P\paren{\omega X}=P\). Montrer : \[\quantifs{\exists Q\in\poly[\C]}P=Q\paren{X^n}.\]

\item En déduire la forme irréductible de la fraction rationnelle : \[F=\sum_{k=0}^{n-1}\dfrac{X+\omega^k}{X-\omega^k}.\]
\end{enumerate}
\end{exo}

\begin{corr}
\note{À venir}
\end{corr}

\begin{exo}[Exercice 26, un peu calculatoire]
Soit \(n\in\N\).

On pose : \[A_n=\sum_{l=0}^{\floor{\frac{n-1}{2}}}\paren{-1}^l\binom{2l+1}{n}X^{2l+1}\qquad B_n=\sum_{l=0}^{\floor{\frac{n}{2}}}\paren{-1}^l\binom{2l}{n}X^{2l}\qquad F_n=\dfrac{A_n}{B_n}.\]

\begin{enumerate}
\item Soit \(\theta\in\R\) tel que \(\theta\not\equiv\dfrac{\pi}{2}\croch{\pi}\) et \(n\theta\not\equiv\dfrac{\pi}{2}\croch{\pi}\). \\

Montrer : \(\tan\paren{n\theta}=F_n\paren{\tan\theta}\). \\

\textit{Indication :} utiliser la formule du binôme de Newton à \(\paren{\cos\theta+\i\sin\theta}^n\) pour calculer \[\tan\paren{n\theta}=\dfrac{\Im\paren{\paren{\cos\theta+\i\sin\theta}^n}}{\Re\paren{\paren{\cos\theta+\i\sin\theta}^n}}.\]

\item Quelle formule retrouve-t-on si \(n=2\) ? \\

\item Quels sont les pôles de \(F_n\) ? En déduire que \(A_n\) et \(B_n\) sont premiers entre eux. \\

\item Quelle est la partie entière de \(F_n\) ? On donnera le résultat en fonction de la parité de \(n\). \\

\item Donner la décomposition en éléments simples de \(F_n\).
\end{enumerate}
\end{exo}

\begin{corr}
\note{À venir}
\end{corr}