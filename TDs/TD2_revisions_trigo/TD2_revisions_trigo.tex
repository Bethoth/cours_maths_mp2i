\chapter{Révisions de trigonométrie}

\minitoc

\begin{exo}
Calculer \[\cos\dfrac{\pi}{12}\qquad\sin\dfrac{\pi}{12}\qquad\tan\dfrac{\pi}{12}\qquad\tan\dfrac{\pi}{8}\qquad\tan\dfrac{\pi}{16}\qquad\sin\dfrac{\pi}{8}\qquad\cos\dfrac{\pi}{8}.\]
\end{exo}

\begin{corr}
\note{à venir}
\end{corr}

\begin{exo}
Résoudre les équations suivantes, d'inconnue \(x\) :

\begin{enumerate}
\item \(\dfrac{2\tan x}{1-\tan^2x}=3\tan x\) \\

\item \(\sin x+\cos x=1\) \\

\item \(\sqrt{3}\sin x+\cos x=1\)
\end{enumerate}
\end{exo}

\begin{corr}
\note{à venir}
\end{corr}

\begin{exo}
Soit \(t\) un réel.

On note \(M\) (respectivement \(N\)) le point de coordonnées \(\paren{1,t}\) (respectivement \(\paren{-1,0}\)) et \(\classe{}\) le cercle unité.

\begin{enumerate}
\item Déterminer \(\classe{}\inter\paren{MN}\). \\

\item Retrouver des formules connues.
\end{enumerate}
\end{exo}

\begin{corr}
\note{à venir} % voir Priscilla sur Discord
\end{corr}

\begin{exo}
Soient \(n\) un entier relatif et \(x\) un réel tels que \[\cos x+\cos\paren{nx}+\cos\paren{\paren{2n-1}x}\not=0.\]

Simplifier l'expression \[\dfrac{\sin x+\sin\paren{nx}+\sin\paren{\paren{2n-1}x}}{\cos x+\cos\paren{nx}+\cos\paren{\paren{2n-1}x}}.\]
\end{exo}

\begin{corr}
\note{à venir}
\end{corr}

\begin{exo}
Montrer \[\quantifs{\forall n\in\N;\forall x\in\R}\abs{\sin\paren{nx}}\leq n\abs{\sin x}.\]
\end{exo}

\begin{corr}
\note{à venir}
\end{corr}

\begin{exo}
Montrer \[\quantifs{\forall x,y\in\R}\cos x^2+\cos y^2-\cos\paren{xy}<3.\]
\end{exo}

\begin{corr}
\note{à venir}
\end{corr}

\begin{exo}
Soient \(n\) un entier naturel et \(x\) un réel tels que \(x\not\equiv0\croch{\dfrac{\pi}{2^{n+1}}}\).

Montrer que la somme \(S=\sum_{k=0}^{n}2^k\tan\paren{2^kx}\) est bien définie et calculer \(S\).

\textit{Indication :} montrer que pour tout réel \(\theta\) tel que \(\theta\not\equiv0\croch{\dfrac{\pi}{2}}\), on a \[\tan\theta=\cotan\theta-2\cotan\paren{2\theta}\] en posant \(\cotan=\dfrac{\cos}{\sin}\).
\end{exo}

\begin{corr}
\note{à venir}
\end{corr}

\begin{exo}
Soient \(n\in\N\) et \(x\in\intervee{0}{\pi}\).

Calculer \[P=\prod_{k=1}^{n}\cos\dfrac{x}{2^k}.\]
\end{exo}

\begin{corr}
\note{à venir}
\end{corr}

\begin{exo}
Soient \(n\in\N\) et \(x\in\intervee{0}{\dfrac{\pi}{2^{n+1}}}\).

En utilisant l'exercice précédent, calculer \[Q=\prod_{k=0}^{n}\paren{1+\dfrac{1}{\cos\paren{2^kx}}}.\]
\end{exo}

\begin{corr}
\note{à venir}
\end{corr}

\begin{exo}
Soient \(x\) un réel et \(n\) un entier naturel.

Calculer \[S=\sum_{k=0}^{n}\dfrac{1}{\paren{-3}^k}\cos^3\paren{3^kx}.\]

\textit{Indication :} montrer \(\quantifs{\forall\theta\in\R}\cos^3\theta=\dfrac{3}{4}\cos\theta+\dfrac{1}{4}\cos\paren{3\theta}\).
\end{exo}

\begin{corr}
\note{à venir}
\end{corr}