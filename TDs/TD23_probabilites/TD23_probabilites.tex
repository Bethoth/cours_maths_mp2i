\chapter{Probabilités}

\minitoc

\begin{exo}[Exercice 1]
Donner le nombre d'anagrammes de chacun des noms suivants : \[\text{UGO}\qquad\text{ROBIN}\qquad\text{EMMY}\qquad\text{METEHAN}\qquad\text{PRISCILLA}\]
\end{exo}

\begin{corr}
UGO : il y a \(3!=6\) anagrammes.

ROBIN : il y a \(5!=120\) anagrammes.

EMMY : il y a \(\dfrac{4!}{2}=12\) anagrammes.

METEHAN : il y a \(\dfrac{7!}{2}=2520\) anagrammes.

PRISCILLA : il y a \(\dfrac{9!}{4}=90720\) anagrammes.
\end{corr}

\begin{exo}[Exercice 2]
Soient \(n\in\Ns\) et \(\groupe{\Omega}[\prem]\) un espace probabilisé.

On note \(\Sigma:\Omega\to\S{n}\) une permutation aléatoire telle que \(\Sigma\sim\loiuniforme{\S{n}}\).

Calculer les probabilités suivantes :

\begin{enumerate}
    \item \(\proba{\Sigma\text{ est une transposition}}\) \\
    \item \(\proba{\Sigma\text{ est un 3-cycle}}\) \\
    \item \(\proba{\Sigma\text{ est un cycle}}\)
\end{enumerate}
\end{exo}

\begin{corr}[1]~\\
Il y a \(\binom{2}{n}\) transpositions dans \(\S{n}\) donc \[\proba{\Sigma\text{ est une transposition}}=\dfrac{\binom{2}{n}}{n!}=\dfrac{1}{2\paren{n-2}!}.\]
\end{corr}

\begin{corr}[2]~\\
Il y a \(2\times\binom{3}{n}\) \(3\)-cycles dans \(\S{n}\) donc \[\proba{\Sigma\text{ est un 3-cycle}}=\dfrac{2\times\binom{3}{n}}{n!}=\dfrac{1}{3\paren{n-3}!}.\]
\end{corr}

\begin{corr}[3]~\\
Il y a \(\sum_{l\in\interventierii{2}{n}}\binom{l}{n}\times\paren{l-1}!=\sum_{l=2}^n\dfrac{n!}{l!\,\paren{n-l}!}\times\paren{l-1}!=\sum_{l=2}^n\dfrac{n!}{l\paren{n-l}!}\) cycles dans \(\S{n}\) donc \[\proba{\Sigma\text{ est un cycle}}=\sum_{l=2}^n\dfrac{1}{l\paren{n-l}!}.\]
\end{corr}

\begin{exo}[Exercice 3, suite de l'\thref{exo:td15exo11}]
On suppose que le corps \(\K\) est fini et on note \(q\) son cardinal : \[q=\Card\K<\pinf.\]

Soient \(r,n\in\Ns\) tels que \(r\leq n\) et \(E\) un \(\K\)-espace vectoriel de dimension \(n\).

\begin{enumerate}
    \item Combien \(E\) possède-t-il de bases ? \\
    \item Combien \(E\) possède-t-il de sous-espaces vectoriels de dimension \(r\) ? \\
    \item Soit \(F\) un sous-espace vectoriel de \(E\) de dimension \(r\). \\ Combien possède-t-il de supplémentaires dans \(E\) ? \\
    \item Combien y a-t-il de projecteurs de rang \(r\) dans \(\Lendo{E}\) ? \\
    \item Combien y a-t-il d'endomorphismes de rang \(r\) dans \(\Lendo{E}\) ?
\end{enumerate}
\end{exo}

\begin{corr}
\note{À venir}
\end{corr}

\begin{exo}[Exercice 4]
Il existe dans ma ville deux compagnies de taxis : les \guillemets{taxis-rouges} (5000 taxis) et les \guillemets{taxis-verts} (250 taxis).

Je suis légèrement daltonien, et j'ai oublié hier mon parapluie dans un taxi.

Il me semble que le taxi était vert, mais je sais que je ne suis pas complètement fiable : je reconnais le rouge 3 fois sur 4, et le vert 4 fois sur 5.

À quelle compagnie devrais-je téléphoner en premier ?
\end{exo}

\begin{corr}
On note \(R\) l'événement \guillemets{le taxi était rouge}, \(V\) l'événement \guillemets{le taxi était vert}, \(R\prim\) l'événement \guillemets{j'ai cru voir un taxi rouge} et \(V\prim\) l'événement \guillemets{j'ai cru voir un taxi vert}.

On a : \[\proba{R\prim\mid R}=\dfrac{3}{4}\qquad\proba{V\prim\mid V}=\dfrac{4}{5}\qquad\proba{R}=\dfrac{5000}{5250}\qquad\proba{V}=\dfrac{250}{5250}.\]

On a : \[\begin{aligned}
\proba{V\mid V\prim}&=\dfrac{\proba{V\prim\mid V}\proba{V}}{\proba{V\prim\mid V}\proba{V}+\proba{V\prim\mid R}\proba{R}} \\
&=\dfrac{\frac{4}{5}\times\frac{250}{5250}}{\frac{4}{5}\times\frac{250}{5250}+\frac{1}{4}\times\frac{5000}{5250}} \\
&=\dfrac{200}{200+1250} \\
&=\dfrac{4}{29} \\
&<\dfrac{1}{2}.
\end{aligned}\]

Donc il vaut mieux appeler la compagnie \guillemets{taxis rouges}.
\end{corr}

\begin{exo}[Exercice 5]
Dans une population donnée, 80\% des individus programment en Python.

Parmi ceux qui programment en Python, 10\% programment aussi en OCaml.

Parmi ceux qui programment en OCaml, 90\% programment aussi en Python.

Quelle proportion de la population programme en OCaml ?
\end{exo}

\begin{corr}
On note \(P\) l'événement \guillemets{l'individu programme en Python} et \(O\) l'événement \guillemets{l'individu programme en OCaml}.

On a : \[\proba{P}=\dfrac{8}{10}\qquad\proba{P\mid O}=\dfrac{9}{10}\qquad\proba{O\mid P}=\dfrac{1}{10}.\]

On a \(\proba{P\mid O}=\dfrac{\proba{P}\proba{O\mid P}}{\proba{O}}\) donc : \[\begin{aligned}
\proba{O}&=\dfrac{\proba{P}\proba{O\mid P}}{\proba{P\mid O}} \\
&=\dfrac{\frac{8}{10}\times\frac{1}{10}}{\frac{9}{10}} \\
&=\dfrac{4}{45}.
\end{aligned}\]
\end{corr}

\begin{exo}[Exercice 6, TPE EIVP 2018]
Un enfant possède vingt figurines qu'il place sur une étagère. Parmi celles-ci se trouvent des oiseaux et d'autres animaux.

\begin{enumerate}
    \item S'il y a deux oiseaux, quelle est la probabilité qu'ils soient côte à côte sur l'étagère ? \\
    \item S'il y a cinq oiseaux, quelle est la probabilité qu'au moins deux oiseaux soient côte à côte sur l'étagère ?
\end{enumerate}
\end{exo}

\begin{corr}
\note{À venir}
\end{corr}

\begin{exo}[Exercice 7]
On dispose de dix pièces de 1 euro.

L'une des dix pièces est fausse : la probabilité d'obtenir \guillemets{pile} lorsqu'on la lance est \(\dfrac{2}{3}\).

Les neuf vraies pièces sont équilibrées (on obtient \guillemets{pile} et \guillemets{face} avec la même probabilité lorsqu'on les lance).

\begin{enumerate}
    \item On choisit l'une des dix pièces, on la lance, et on obtient \guillemets{pile}. \\ Quelle est la probabilité que la pièce choisie soit fausse ? \\
    \item On choisit une seconde pièce (différente de la première), on la lance, et on obtient encore \guillemets{pile}. \\ Quelle est la probabilité que la seconde pièce choisie soit fausse ?
\end{enumerate}
\end{exo}

\begin{corr}
\note{À venir}
\end{corr}

\begin{exo}[Exercice 8, Centrale PSI 2015 (BEOS 1743)]
Au rez-de-chaussée d'un immeuble à \(n\) étages, \(p\) personnes prennent l'ascenseur et s'arrêtent à un étage au hasard et de manière indépendante.

On note \(X\) la variable aléatoire qui donne le nombre d'arrêts de l'ascenseur.

On note \(X_i\) la variable aléatoire qui vaut \(1\) si l'ascenseur s'arrête à l'étage \(i\) et \(0\) sinon.

\begin{enumerate}
    \item \begin{enumerate}
        \item Donner la loi des \(X_i\). \\
        \item Donner l'expression de \(X\) en fonction des \(X_i\). \\
        \item Calculer l'espérance de \(X\). \\
    \end{enumerate}
    \item Vérification avec Python du résultat obtenu. \\ \begin{enumerate}
        \item Écrire une fonction qui simule la variable aléatoire \(X\). \\
        \item Pour \(n\) variant de 3 à 20 et avec \(p=10\), vérifier le résultat de la question (1c) pour 1000 répétitions de l'expérience.
    \end{enumerate}
\end{enumerate}
\end{exo}

\begin{corr}[1a]
On pose \(\quantifs{\forall i\in\interventierii{1}{p}}Y_i\in\interventierii{1}{n}\) l'étage où s'arrête la personne \no \(i\).

Soit \(i\in\interventierii{1}{n}\).

\(X_i\) suit une loi de Bernoulli.

On a : \[\begin{WithArrows}
\proba{X_i=0}&=\proba{Y_1\not=i\text{ et }\dots\text{ et }Y_p\not=i} \Arrow{mutuelle indépendance} \\
&=\prod_{j=1}^p\proba{Y_j\not=i} \Arrow{\(Y_1,\dots,Y_p\sim\loiuniforme{\interventierii{1}{n}}\)} \\
&=\paren{\dfrac{n-1}{n}}^p.
\end{WithArrows}\]

Donc \(\proba{X_i=1}=1-\proba{X_i=0}=1-\paren{\dfrac{n-1}{n}}^p\).
\end{corr}

\begin{corr}[1b]
On a : \[X=\sum_{i=1}^nX_i.\]
\end{corr}

\begin{corr}[1c]
On a : \[\begin{aligned}
\esp{X}&=\esp{\sum_{i=1}^nX_i} \\
&=\sum_{i=1}^n\esp{X_i} \\
&=\sum_{i=1}^n\paren{1-\paren{\dfrac{n-1}{n}}^p} \\
&=n-n\paren{\dfrac{n-1}{n}}^p.
\end{aligned}\]
\end{corr}

\begin{corr}[2a et 2b]
\begin{code}
from random import randint

def X(n, p):
    arrets = [0] * n
    for _ in range(p):
        arrets[randint(1, n) - 1] = 1
    return sum(arrets)

def esperance_theorique(n, p):
    return n * (1 - (1 - 1 / n) ** p)

def esperance_empirique(n, p, N):
    return sum(X(n, p) for _ in range(N)) / N

for n in range(3, 21):
    printf(f"""n = {n} :
{esperance_theorique(n, 10)}
{esperance_empirique(n, 10, 1000)}""")
\end{code}
\end{corr}

\begin{exo}[Exercice 9]
Soient \(n,s\in\interventierie{2}{\pinf}\).

On considère une urne contenant des boules de couleurs \(C_1,\dots,C_s\).

On effectue \(n\) tirages successifs d'une boule avec remise.

Pour tout \(i\in\interventierii{1}{s}\), on note : \begin{description}
    \item[] \(p_i\) la proportion des boules de couleur \(C_i\)
    \item[] \(X_i\) le nombre de boules de couleur \(C_i\) obtenues à l'issue des \(n\) tirages.
\end{description}

\begin{enumerate}
    \item Que valent les sommes \(\sum_{i=1}^sX_i\) et \(\sum_{i=1}^sp_i\) ? \\
    \item Pour tout \(i\in\interventierii{1}{s}\), déterminer la loi de \(X_i\), son espérance et sa variance. \\
    \item Soient \(i,j\in\interventierii{1}{s}\) tels que \(i\not=j\). \\ Déterminer la loi de \(X_i+X_j\) et sa variance. \\ En déduire \(\cov{X_i}{X_j}=-np_ip_j\).
\end{enumerate}
\end{exo}

\begin{corr}
\note{À venir}
\end{corr}

\begin{exo}[Exercice 10, X-Cachan PSI 2015 (BEOS 1204)]
\underline{Exercice 1}

On considère une suite de \(n\) convertisseurs numériques fonctionnant de manière indépendante et placés en série. Chaque convertisseur restitue correctement le bit qu'on lui fournit avec la probabilité \(p\) et renvoie le bit opposé avec la probabilité \(1-p\), où \(p\in\intervii{0}{1}\).

On note \(X_k\) le bit en sortie du \(k\)-ème convertisseur et \(X_0\) le bit en entrée de chaîne.

On définit la suite finie \(\paren{A_k}_{k\in\interventierii{0}{n}}\) par \(A_k=\dcoords{\proba{X_k=1}}{\proba{X_k=0}}\).

\begin{enumerate}
    \item Déterminer une relation de récurrence pour la suite \(\paren{A_k}_{k\in\interventierii{0}{n}}\). \\
    \item En déduire la probabilité que le bit initial soit correctement rendu en sortie du \(n\)-ème convertisseur. Que se passe-t-il lorsqu'on fait tendre \(n\) vers \(\pinf\) ?
\end{enumerate}

\underline{Exercice 2} (modifié)

On considère un dé non-pipé à six faces numérotées de 1 à 6.

On considère une suite de \(n\) lancers.

\begin{enumerate}
    \item On note \(N_k\) le nombre d'apparitions de la face \(k\in\interventierii{1}{6}\) dans la suite des \(n\) lancers. \\ Intuitivement, que peut-on dire de \(N_k\) lorsque \(n\) tend vers \(\pinf\) ? \\ Rigoureusement, que donne la loi faible des grands nombres ? \\
    \item On suppose que \(n=6m\) est un multiple de \(6\). Quelle est la probabilité d'obtenir une suite de lancers telle que \[\quantifs{\forall k\in\interventierii{1}{6}}N_k=m\text{ ?}\]
    \item Vérifier le résultat de la question précédente avec Python pour \(m\in\interventierii{1}{10}\), en faisant pour chaque valeur de \(m\) une série de \(10^5\) lancers. \\ Que remarque-t-on ?
\end{enumerate}
\end{exo}

\begin{corr}[Exercice 1, 1]
On a : \[\begin{aligned}
\quantifs{\forall k\in\interventierii{0}{n-1}}\proba{X_{k+1}=1}&=\proba{X_{k+1}=1\text{ et }X_k=1}+\proba{X_{k+1}=1\text{ et }X_k=0} \\
&=\proba{X_{k+1}=1\mid X_k=1}\proba{X_k=1}+\proba{X_{k+1}=1\mid X_k=0}\proba{X_k=0} \\
&=p\proba{X_k=1}+\paren{1-p}\proba{X_k=0} \\
&=\cycle{p;1-p}A_k
\end{aligned}\] et : \[\begin{aligned}
\quantifs{\forall k\in\interventierii{0}{n-1}}\proba{X_{k+1}=0}&=\proba{X_{k+1}=0\text{ et }X_k=1}+\proba{X_{k+1}=0\text{ et }X_k=0} \\
&=\proba{X_{k+1}=0\mid X_k=1}\proba{X_k=1}+\proba{X_{k+1}=0\mid X_k=0}\proba{X_k=0} \\
&=\paren{1-p}\proba{X_k=1}+p\proba{X_k=0} \\
&=\cycle{1-p;p}A_k.
\end{aligned}\]

Donc \[\quantifs{\forall k\in\interventierii{0}{n}}A_{k+1}=\begin{pmatrix}
p & 1-p \\
1-p & p
\end{pmatrix}A_k.\]
\end{corr}

\begin{corr}[Exercice 1, 2]~\\
On pose \(M=\begin{pmatrix}
p & 1-p \\
1-p & p
\end{pmatrix}\) et \(P=\begin{pmatrix}
1 & 1 \\
1 & -1
\end{pmatrix}\).

On a : \[\begin{aligned}
D=P\inv MP&=\dfrac{-1}{2}\begin{pmatrix}
-1 & -1 \\
-1 & 1
\end{pmatrix}\begin{pmatrix}
p & 1-p \\
1-p & p
\end{pmatrix}\begin{pmatrix}
1 & 1 \\
1 & -1
\end{pmatrix} \\
&=\dfrac{-1}{2}\begin{pmatrix}
-1 & -1 \\
1-2p & 2p-1
\end{pmatrix}\begin{pmatrix}
1 & 1 \\
1 & -1
\end{pmatrix} \\
&=\dfrac{-1}{2}\begin{pmatrix}
-2 & 0 \\
0 & 2-4p
\end{pmatrix} \\
&=\begin{pmatrix}
1 & 0 \\
0 & 2p-1
\end{pmatrix}.
\end{aligned}\]

Donc \[\begin{aligned}
M^n&=\paren{PDP\inv}^n \\
&=PD^nP\inv \\
&=\begin{pmatrix}
1 & 1 \\
1 & -1
\end{pmatrix}\begin{pmatrix}
1 & 0 \\
0 & \paren{2p-1}^n
\end{pmatrix}\begin{pmatrix}
1 & 1 \\
1 & -1
\end{pmatrix}\inv \\
&=\dfrac{-1}{2}\begin{pmatrix}
1 & \paren{2p-1}^n \\
1 & -\paren{2p-1}^n
\end{pmatrix}\begin{pmatrix}
-1 & -1 \\
-1 & 1
\end{pmatrix} \\
&=\dfrac{-1}{2}\begin{pmatrix}
-1-\paren{2p-1}^n & \paren{2p-1}^n-1 \\
\paren{2p-1}^n-1 & -1-\paren{2p-1}^n
\end{pmatrix}.
\end{aligned}\]

Ainsi, en supposant \(A_0=\dcoords{1}{0}\), on a : \[\begin{aligned}
A_n&=\dfrac{-1}{2}\begin{pmatrix}
-1-\paren{2p-1}^n & \paren{2p-1}^n-1 \\
\paren{2p-1}^n-1 & -1-\paren{2p-1}^n
\end{pmatrix}A_0 \\
&=\dfrac{-1}{2}\dcoords{-1-\paren{2p-1}^n}{\paren{2p-1}^n-1}.
\end{aligned}\]

Quand \(n\to\pinf\) : \[A_n\to\begin{dcases}
\dcoords{\nicefrac{1}{2}}{\nicefrac{1}{2}} &\text{si }p\in\intervee{0}{1} \\
\dcoords{1}{0} &\text{si }p=1
\end{dcases}\]
\end{corr}

\begin{corr}[Exercice 2, 1]
Intuitivement : \[\quantifs{\forall k\in\interventierii{1}{6}}N_k\simqd{n\to\pinf}\dfrac{n}{6}.\]

\note{En fait, la loi des grands nombres est hors-programme}
\end{corr}

\begin{corr}[Exercice 2, 2]
On a : \[\begin{aligned}
\binom{m}{6m}\binom{m}{5m}\binom{m}{4m}\binom{m}{3m}\binom{m}{2m}\binom{m}{m}&=\dfrac{\paren{6m}!}{m!\,\paren{5m}!}\times\dfrac{\paren{5m}!}{m!\,\paren{4m}!}\times\dots\times\dfrac{\paren{2m}!}{m!\,m!} \\
&=\dfrac{\paren{6m}!}{\paren{m!}^6}.
\end{aligned}\]

D'où la probabilité : \[\dfrac{\frac{\paren{6m}!}{\paren{m!}^6}}{6^{6m}}=\dfrac{\paren{6m}!}{6^{6m}\paren{m!}^6}.\]
\end{corr}

\begin{corr}[Exercice 2, 3]
\begin{code}
from random import randint
from math import factorial

def simul(m):
    lancer = [randint(1, 6) for _ in range(6 * m)]
    d = {i : 0 for i in range(1, 7)}
    for e in lancer:
        d[e] += 1
    return all(d[i] == m for i in range(1, 7))

def proba_theorique(m):
    return factorial(6 * m) / (6 ** (6 * m) * factorial(m) ** 6)

def proba_empirique(m, N):
    compt = 0
    for _ in range(N):
        if simul(m):
            compt += 1
    return compt / N

for m in range(1, 11):
    print(f"""m = {m} :
{proba_theorique(m)}
{proba_empirique(m, 10 ** 5)}""")
\end{code}
\end{corr}

\begin{exo}[Exercice 11, Centrale maths 1 2016]
On lance \(N\) fois une pièce de monnaie équilibrée.

On note \(\groupe{\Omega}[\prem]\) un espace probabilisé modélisant le problème, \(X\) le nombre de \guillemets{pile} et \(Y\) le nombre de \guillemets{face} obtenus.

\begin{enumerate}
    \item On suppose que \(N\) est un entier naturel non-nul fixé. \\ \begin{enumerate}
        \item Calculer la covariance de \(X\) et de \(Y\). \\
        \item Les variables aléatoires \(X\) et \(Y\) sont-elles indépendantes ? \\
    \end{enumerate}
    \item ... \\
    \item ...
\end{enumerate}
\end{exo}

\begin{corr}
\note{À venir}
\end{corr}

\begin{exo}[Exercice 12]
Soient \(n\in\Ns\).

On note \(\Sigma:\Omega\to\S{n}\) une permutation aléatoire telle que \(\Sigma\sim\loiuniforme{\S{n}}\) et \(N\) le nombre de points fixes de \(\Sigma\) : c'est un entier aléatoire à valeurs dans \(\interventierii{0}{n}\).

Dans la suite, on calcule l'espérance et la variance de \(N\).

Pour cela, on définit des variables aléatoires \(X_1,\dots,X_n\) en posant : \[\quantifs{\forall k\in\interventierii{1}{n}}X_k=\begin{dcases}
1 &\text{si }k\text{ est un point fixe de }\Sigma \\
0 &\text{sinon}
\end{dcases}\] de sorte qu'on a : \[N=X_1+\dots+X_n.\]

\begin{enumerate}
    \item Que vaut \(\proba{N=n}\) ? \\
    \item Donner les lois des variables aléatoires \(X_1,\dots,X_n\). \\ Sont-elles mutuellement indépendantes ? \\
    \item Calculer l'espérance de \(N\). \\
    \item Soient deux entiers distincts \(i,j\in\interventierii{1}{n}\). Calculer l'espérance de \(X_iX_j\). \\ En déduire la covariance de \(X_i\) et \(X_j\). \\ Sont-elles indépendantes ? \\
    \item Calculer la variance de \(N\). \\
    \item Montrer : \[\proba{N\geq4}\leq\dfrac{1}{9}.\]
\end{enumerate}
\end{exo}

\begin{corr}
\note{À venir}
\end{corr}