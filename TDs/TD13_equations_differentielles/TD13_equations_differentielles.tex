\chapter{Équations différentielles}

\section{Équations différentielles linéaires d'ordre 1}

\begin{exo}[Exercice 1]
Résoudre l'équation différentielle : \[\paren{E}~y\prim+y=\sin t.\]
\end{exo}

\begin{corr}
\note{À venir}
\end{corr}

\begin{exo}[Exercice 2]
Résoudre l'équation différentielle : \[\paren{E}~y\prim+\dfrac{\sin t}{\cos t-2}y=\dfrac{2\sin t}{\cos t-2}.\]
\end{exo}

\begin{corr}
\note{À venir}
\end{corr}

\begin{exo}[Exercice 3]
Résoudre le problème de Cauchy : \[\begin{dcases}
y\prim-\dfrac{t^3+t+1}{t^2+1}y=\dfrac{t^4+t-1}{t^2+1} \\
y\paren{0}=1
\end{dcases}\]
\end{exo}

\begin{corr}
\note{À venir}
\end{corr}

\begin{exo}[Exercice 4]
Résoudre l'équation différentielle suivante sur l'intervalle \(\intervee{0}{1}\) : \[\paren{E}~t\ln ty\prim-y=2t^2\ln^2t.\]
\end{exo}

\begin{corr}
\note{À venir}
\end{corr}

\begin{exo}[Exercice 5]
Résoudre l'équation différentielle suivante sur l'intervalle \(\intervee{-1}{1}\) : \[\paren{E}~\sqrt{1-t^2}y\prim-y=1.\]
\end{exo}

\begin{corr}
\note{À venir}
\end{corr}

\begin{exo}[Exercice 6]\thlabel{exo:exempleProblèmesDeRaccordCours}
On considère l'équation différentielle : \[\paren{E}~ty\prim+y=\cos t.\]

\begin{enumerate}
    \item Déterminer les solutions de \(\paren{E}\) sur \(\Rps\) (respectivement \(\Rms\)). \\
    \item En déduire les solutions définies sur \(\R\).
\end{enumerate}
\end{exo}

\begin{corr}
\note{À venir}
\end{corr}

\begin{exo}[Exercice 7, Mines MP 2006]
Résoudre l'équation différentielle suivante sur l'intervalle \(\intervee{0}{\pinf}\) : \[\paren{E}~t\ln ty\prim-\paren{3\ln t+1}y=0.\]
\end{exo}

\begin{corr}
\note{À venir}
\end{corr}

\begin{exo}[Exercice 8]
Trouver les solutions réelles sur \(\R\) à l'équation différentielle : \[\paren{E}~ty\prim-\paren{2t+1}y=2t+1.\]

Quelles sont les solutions vérifiant \(y\paren{0}=0\) ? celles vérifiant \(y\paren{0}=-1\) ?
\end{exo}

\begin{corr}
\note{À venir}
\end{corr}

\section{Équations différentielles linéaires d'ordre 2}

\begin{exo}[Exercice 9]
Résoudre les équations différentielles suivantes :

\begin{enumerate}
    \item \(y\seconde-2y\prim+y=\e{3t}+3t\e{t}\) \\
    \item \(y\seconde+4y\prim-5y=\sh t\) \\
    \item \(y\seconde+4y=t\sin t\).
\end{enumerate}
\end{exo}

\begin{corr}
\note{À venir}
\end{corr}

\begin{exo}[Exercice 10]\thlabel{exo:exempleChangementDeVariableCours}
Résoudre sur l'intervalle \(\Rps\) l'équation différentielle : \[\paren{E}~t^2y\seconde+3ty\prim+y=\paren{t+1}^2\] en faisant le changement de variable \(s=\ln t\).
\end{exo}

\begin{corr}~\\
\begin{brouill}
On a \(s=\ln t\ssi t=\e{s}\) et \(y\paren{t}=y\paren{\e{s}}=z\paren{s}=z\paren{\ln t}\).
\end{brouill}

Soit \(y\in\ensclasse{2}{\Rps}{\R}\).

On pose \(\fonction{z}{\R}{\R}{s}{y\paren{\e{s}}}\)

On a \(\quantifs{\forall t\in\Rps}\begin{dcases}
y\paren{t}=z\paren{\ln t} \\
y\prim\paren{t}=\dfrac{1}{t}z\prim\paren{\ln t} \\
y\seconde\paren{t}=\dfrac{1}{t^2}z\seconde\paren{t}-\dfrac{1}{t^2}z\prim\paren{\ln t}
\end{dcases}\)

Donc : \[\begin{aligned}
y\text{ est solution de }\paren{E}&\ssi\quantifs{\forall t\in\Rps}t^2y\seconde\paren{t}+3ty\prim\paren{t}+y\paren{t}=\paren{t+1}^2 \\
&\ssi\quantifs{\forall t\in\Rps}z\seconde\paren{\ln t}-z\prim\paren{\ln t}+3z\prim\paren{\ln t}+z\paren{\ln t}=\paren{t+1}^2 \\
&\ssi\quantifs{\forall s\in\R}z\seconde\paren{s}+2z\prim\paren{s}+z\paren{s}=\paren{\e{s}+1}^2 \\
&\ssi z\text{ est solution de }\paren{E\prim}~z\seconde+2z\prim+z=\paren{\e{s}+1}^2.
\end{aligned}\]

Résolvons \(\paren{E\prim}\).

Résolvons \(\paren{E_0\prim}~z\seconde+2z\prim+z=0\).

L'équation caractéristique \(x^2+2x+1\) admet \(-1\) comme solution double.

Donc les solutions de \(\paren{E_0\prim}\) sont les fonctions de la forme \(z_0:s\mapsto\paren{\lambda s+\mu}\e{-t}\) avec \(\lambda,\mu\in\R\).

Déterminons une solution de \(\paren{E\prim}\).

On pose : \[\paren{E_1\prim}~y\seconde+2y\prim+y=\e{2s}\qquad\paren{E_2\prim}~y\seconde+2y\prim+y=\e{s}\qquad\paren{E_3\prim}y\seconde+2y\prim+y=1.\]

Déterminons une solution de \(\paren{E_1\prim}\).

\begin{brouill}
On a \(P\paren{s}\e{\gamma s}\) avec \(\begin{dcases}
P=1\text{ donc }\deg P=0 \\
\gamma=2
\end{dcases}\)

Donc on cherche une solution de la forme \(Q\paren{s}\e{\gamma s}\) avec \(\deg Q=0\), \cad de la forme \(a\e{2s}\).
\end{brouill}

Soit \(a\in\R\).

On pose \(z_1:s\mapsto a\e{2s}\).

On a \(\quantifs{\forall s\in\R}\begin{dcases}
z_1\prim\paren{s}=2a\e{2s} \\
z_1\seconde\paren{s}=4a\e{2s}
\end{dcases}\)

Donc \[\begin{aligned}
z_1\text{ est solution de }\paren{E_1\prim}&\ssi\quantifs{\forall s\in\R}4a\e{2s}+4a\e{2s}+a\e{2s}=\e{2s} \\
&\ssi\quantifs{\forall s\in\R}9a\e{2s}=\e{2s} \\
&\impr a=\dfrac{1}{9}.
\end{aligned}\]

Donc \(z_1:s\mapsto\dfrac{1}{9}\e{2s}\) est solution de \(\paren{E_1\prim}\).

Déterminons une solution de \(\paren{E_2\prim}\).

Soit \(a\in\R\).

On pose \(z_2:s\mapsto a\e{s}\).

On a : \[\begin{aligned}
z_2\text{ est solution de }\paren{E_2\prim}&\ssi a\e{s}+2a\e{s}+a\e{s}=\e{s} \\
&\impr a=\dfrac{1}{4}.
\end{aligned}\]

Donc \(z_2:s\mapsto\dfrac{1}{4}\e{s}\) est solution de \(\paren{E_2\prim}\).

On remarque que \(z_3:s\mapsto1\) est solution de \(\paren{E_3\prim}\).

Donc \(z:s\mapsto\dfrac{1}{9}\e{2s}+\dfrac{1}{2}\e{s}+1\) est solution de \(\paren{E\prim}\) selon le principe de superposition.

Ainsi, les solutions de \(\paren{E\prim}\) sont les fonctions de la forme \[s\mapsto\paren{\lambda s+\mu}\e{-s}+\dfrac{1}{9}\e{2s}+\dfrac{1}{2}\e{s}+1\] avec \(\lambda,\mu\in\R\).

Finalement : \[\begin{aligned}
y\text{ est solution de }\paren{E}&\ssi\quantifs{\exists\lambda,\mu\in\R;\forall s\in\R}z\paren{s}=\paren{\lambda s+\mu}\e{-s}+\dfrac{1}{9}\e{2s}+\dfrac{1}{2}\e{s}+1 \\
&\ssi\quantifs{\exists\lambda,\mu\in\R;\forall t\in\Rps}z\paren{\ln t}=\paren{\lambda\ln t+\mu}\e{-\ln t}+\dfrac{1}{9}\e{2\ln t}+\dfrac{1}{2}\e{\ln t}+1 \\
&\ssi\quantifs{\exists\lambda,\mu\in\R;\forall t\in\Rps}y\paren{t}=\dfrac{\lambda\ln t+\mu}{t}+\dfrac{1}{9}t^2+\dfrac{1}{2}t+1.
\end{aligned}\]
\end{corr}

\begin{exo}[Exercice 11]
Résoudre l'équation différentielle : \[\paren{E}~\paren{1+t^2}^2y\seconde+2t\paren{1+t^2}y\prim+y=\Arctan t\] en faisant le changement de variable \(x=\Arctan t\).
\end{exo}

\begin{corr}
\note{À venir}
\end{corr}

\begin{exo}[Exercice 12]
Résoudre sur l'intervalle \(\Rps\) l'équation différentielle : \[\paren{E}~t^2y\seconde+ty\prim-y=t^2\] en faisant le changement de variable \(x=\ln t\).
\end{exo}

\begin{corr}
\note{À venir}
\end{corr}

\begin{exo}[Exercice 13, Mines-Ponts PSI 2017]
Soit \(a\) un réel non-nul.

Résoudre \[\paren{E}~\paren{1+x^2}^2y\seconde+2x\paren{1+x^2}y\prim+a^2y=0\] en faisant le changement de variable \(\theta=\Arctan x\).
\end{exo}

\begin{corr}
\note{À venir}
\end{corr}

\begin{exo}[Exercice 14]
Résoudre sur \(\R\) l'équation différentielle réelle : \[\paren{E}~ty\seconde+2\paren{t+1}y\prim+\paren{t+2}y=0.\]

\textit{Indication :} on pourra considérer la fonction \(z:t\mapsto ty\paren{t}\).
\end{exo}

\begin{corr}
\note{À venir}
\end{corr}

\section{Applications}

\begin{exo}[Exercice 15]
Résoudre\footnote{\Cad déterminer les couples \(\paren{y_1,y_2}\in\ensclasse{1}{\R}{\R}^2\) qui sont solution} de deux façons le système linéaire : \[\paren{E}~\begin{dcases}
y_1\prim=-y_2 \\
y_2\prim=y_1
\end{dcases}\]

\begin{enumerate}
    \item À l'aide d'une équation différentielle d'ordre 1 vérifiée par la fonction complexe \(y_1+\i y_2\). \\
    \item À l'aide d'une équation différentielle d'ordre 2 vérifée par la fonction réelle \(y_1\).
\end{enumerate}
\end{exo}

\begin{corr}
\note{À venir}
\end{corr}

\begin{exo}[Exercice 16]
Résoudre le système linéaire : \[\paren{E}~\begin{dcases}
y_1\prim=2y_1-y_2+\cos t \\
y_2\prim=y_1+2y_2+\sin t
\end{dcases}\]
\end{exo}

\begin{corr}
\note{À venir}
\end{corr}

\begin{exo}[Exercice 17]
Si \(f:\R\to\R\) est une fonction dérivable, on note \(\P{f}\) la propriété : \[\quantifs{\forall x\in\R}f\prim\paren{x}=2f\paren{-x}+x.\]

\begin{enumerate}
    \item Soit \(f:\R\to\R\) une fonction dérivable. Montrer que si \(\P{f}\) est vraie, alors \(f\) est deux fois dérivable et satisfait une certaine équation différentielle linéaire d'ordre 2. \\
    \item Résoudre l'équation différentielle obtenue. \\
    \item Déterminer l'ensemble des fonctions dérivables \(f:\R\to\R\) telles que \(\P{f}\) soit vraie.
\end{enumerate}
\end{exo}

\begin{corr}
\note{À venir}
\end{corr}