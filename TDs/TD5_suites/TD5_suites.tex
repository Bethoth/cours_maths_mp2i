\chapter{Suites}

\begin{exo}[Exercice 1, exemples fondamentaux]
Soit \(\lambda\in\R\).

Étudier la limite des suites suivantes en fonction de \(\lambda\) : \[\paren{n^\lambda}_{n\in\Ns}\qquad\paren{\ln^\lambda n}_{n\in\interventierie{2}{\pinf}}\qquad\paren{\lambda^n}_{n\in\N}\qquad\paren{n!}_{n\in\N}\]
\end{exo}

\begin{corr}
\note{à venir}
\end{corr}

\begin{exo}[Exercice 2]
Étudier la convergence (et déterminer la limite éventuelle) des suites de terme général :

\begin{enumerate}
\item \(u_n=\dfrac{n!+\paren{-1}^n\cos n}{n!+\cos n}\) \\

\item \(v_n=\dfrac{\paren{-1}^nn!+\cos n}{n!+\cos n}\) \\

\item \(w_n=\dfrac{\paren{-1}^nn!+\cos n}{\paren{n+1}!+\cos\paren{n+1}}\) \\

\item \(x_n=\int_{-1}^1t^n\sin t\odif{t}\)
\end{enumerate}
\end{exo}

\begin{corr}
\note{à venir}
\end{corr}

\begin{exo}[Exercice 3, moyennes]
On définit, pour tous réels strictement positifs \(a\) et \(b\) :

\begin{itemize}
\item leur moyenne arithmétique : \(m_a\paren{a,b}=\dfrac{a+b}{2}\) ; \\

\item leur moyenne géométrique : \(m_g\paren{a,b}=\sqrt{ab}\) ; \\

\item leur moyenne harmonique : \(m_h\paren{a,b}=\dfrac{2ab}{a+b}\).
\end{itemize}

\begin{enumerate}[series=exmoyennes]
\item Soient \(a,b\in\Rps\). Montrer qu'on a \[m_g\paren{a\inv,b\inv}\inv=m_g\paren{a,b}\qquad\text{et}\qquad m_a\paren{a\inv,b\inv}\inv=m_h\paren{a,b}\] et \[m_h\paren{a,b}\leq m_g\paren{a,b}\leq m_a\paren{a,b}\qquad\text{et}\qquad m_g\paren{m_h\paren{a,b},m_a\paren{a,b}}=m_g\paren{a,b}.\]
\end{enumerate}

On se fixe pour la suite deux réels \(x\) et \(y\) tels que \(0<x<y\).

\begin{enumerate}[resume=exmoyennes]
\item On pose \[\begin{dcases}x_0=x \\ y_0=y\end{dcases}\qquad\text{et}\qquad\quantifs{\forall n\in\N}\begin{dcases}x_{n+1}=m_g\paren{x_n,y_n} \\ y_{n+1}=m_a\paren{x_n,y_n}\end{dcases}\] \\

Montrer que \(\paren{x_n}_{n\in\N}\) et \(\paren{y_n}_{n\in\N}\) sont convergentes et de même limite. \\

\item On pose \[\begin{dcases}u_0=x \\ v_0=y\end{dcases}\qquad\text{et}\qquad\quantifs{\forall n\in\N}\begin{dcases}u_{n+1}=m_h\paren{u_n,v_n} \\ v_{n+1}=m_a\paren{u_n,v_n}\end{dcases}\] \\

Montrer que \(\paren{u_n}_{n\in\N}\) et \(\paren{v_n}_{n\in\N}\) sont convergentes et de même limite. Quelle est cette limite ?
\end{enumerate}
\end{exo}

\begin{corr}
\note{à venir}
\end{corr}

\begin{exo}[Exercice 4, série harmonique]
On pose \[\quantifs{\forall n\in\N}H_n=\sum_{k=1}^{n}\dfrac{1}{k}.\]

\begin{enumerate}
\item Montrer que la suite \(\paren{H_n}_{n\in\N}\) admet une limite. \\

\item Montrer : \[\quantifs{\forall n\in\N}H_{2n}-H_n\geq\dfrac{1}{2}.\] \\

\item En déduire la limite de la suite \(\paren{H_n}_{n\in\N}\).
\end{enumerate}
\end{exo}

\begin{corr}
\note{à venir}
\end{corr}

\begin{exo}[Exercice 5]
Soit \(\paren{u_n}_{n\in\N}\) une suite réelle.

Quelles implications sont vraies entre les propositions suivantes ?

\begin{enumerate}
\item La suite \(\paren{u_n}_{n\in\N}\) est convergente. \\

\item \(\lim_{n\to\pinf}u_{n+1}-u_n=0\)
\end{enumerate}
\end{exo}

\begin{corr}
\note{à venir}
\end{corr}

\begin{exo}[Exercice 6]
Soit \(\paren{z_n}_{n\in\N}\) une suite complexe telle que \[\quantifs{\forall n\in\N}z_{n+1}=\dfrac{z_n+\abs{z_n}}{2}.\]

Montrer que \(\paren{z_n}_{n\in\N}\) est convergente et que sa limite est réelle.
\end{exo}

\begin{corr}
\note{à venir}
\end{corr}

\begin{exo}[Exercice 7]
Soit \(\paren{u_n}_{n\in\N}\) une suite réelle.

\begin{enumerate}
\item Montrer que si \(\paren{u_n}_n\) est décroissante et si on a \(\quantifs{\forall n\in\N}u_n\in\N\) alors \(\paren{u_n}_n\) est stationnaire. \\

\item Montrer que si \(\paren{u_n}_n\) est convergente et si on a \(\quantifs{\forall n\in\N}u_n\in\Z\) alors \(\paren{u_n}_n\) est stationnaire.
\end{enumerate}
\end{exo}

\begin{corr}
\note{à venir}
\end{corr}

\begin{exo}[Exercice 8]\thlabel{exo:5.15}
Soit \(A\subset\R\).

\begin{enumerate}
\item Montrer que si \(A\) n'est pas majorée, alors il existe une suite d'élément de \(A\) qui tend vers \(\pinf\). \\

\item Montrer que si \(A\) est majorée, alors il existe une suite d'éléments de \(A\) qui tend vers \(\sup A\).
\end{enumerate}
\end{exo}

\begin{corr}
\note{à venir}
\end{corr}

\begin{exo}[Exercice 9]
Soit \(\paren{u_n}_{n\in\N}\) une suite réelle.

Montrer que \(\paren{u_n}_n\) ne tend pas vers \(\pinf\) si, et seulement si, elle admet une suite extraite majorée.
\end{exo}

\begin{corr}
\note{à venir}
\end{corr}

\begin{exo}[Exercice 10]
Soient \(\paren{u_n}_{n\in\N}\in\R^\N\) et \(l\in\Rb\).

Montrer que \(\paren{u_n}_n\) tend vers \(l\) si, et seulement si, de toute suite extraite de \(\paren{u_n}_n\) on peut extraire une suite qui tend vers \(l\).
\end{exo}

\begin{corr}
\note{à venir}
\end{corr}

\begin{exo}[Exercice 11, théorème des segments emboîtés]
Soient \(\paren{a_n}_{n\in\N},\paren{b_n}_{n\in\N}\in\R^\N\) deux suites réelles telles que \[\quantifs{\forall n\in\N}a_n\leq b_n.\]

On suppose que la suite des segments \(\paren{\intervii{a_n}{b_n}}_{n\in\N}\) est décroissante pour l'inclusion, \cad : \[\quantifs{\forall n\in\N}\intervii{a_{n+1}}{b_{n+1}}\subset\intervii{a_n}{b_n}\] et que la suite des longueurs des segments tend vers \(0\), \cad : \[\lim_{n\to\pinf}b_n-a_n=0.\]

Montrer que \(\biginter_{n\in\N}\intervii{a_n}{b_n}\) est un singleton.
\end{exo}

\begin{corr}
\note{à venir}
\end{corr}

\begin{exo}[Exercice 12]
\begin{enumerate}
\item Soit \(A\) une partie de \(\R\) telle que \[\quantifs{\forall x\in\R;\exists a,b\in A}a<x<b\qquad\text{et}\qquad\quantifs{\forall a,b\in A}\dfrac{a+b}{2}\in A.\] \\

Montrer que \(A\) est dense dans \(\R\). \\

\item Retrouver le fait que \(\Q\) est dense dans \(\R\).
\end{enumerate}
\end{exo}

\begin{corr}
\note{à venir}
\end{corr}

\begin{exo}[Exercice 13, moyenne de Cesàro d'une suite]\thlabel{exo:moyenneDeCesàroD'UneSuite}
Soient \(\paren{u_n}_{n\in\N}\) et \(\paren{v_n}_{n\in\N}\) deux suites réelles et \(l\) un réel.

On considère la suite \(\paren{U_n}_{n\in\N}\) définie par : \[\quantifs{\forall n\in\N}U_n=\dfrac{1}{n+1}\sum_{k=0}^{n}u_k.\]

\begin{enumerate}
\item Montrer que si \(\paren{u_n}_n\) diverge vers \(\pinf\), alors \(\paren{U_n}_n\) diverge vers \(\pinf\). \\

\item Montrer que si \(\paren{u_n}_n\) converge vers \(l\), alors \(\paren{U_n}_n\) converge vers \(l\). \\

\item Montrer que les implications réciproques des propositions ci-dessus ne sont pas toujours vraies. \\

\item Montrer que si la suite \(\paren{v_{n+1}-v_n}_n\) converge vers \(l\), alors \(\lim_{n\to\pinf}\dfrac{v_n}{n}=l\). \\

\item On suppose qu'on a : \[\quantifs{\forall n\in\N}v_n\in\Rps\qquad\text{et}\qquad\lim_{n\to\pinf}\dfrac{v_{n+1}}{v_n}=\pinf.\] \\

Montrer : \[\lim_{n\to\pinf}\sqrt[n]{v_n}=\pinf.\] \\

\item On suppose qu'on a : \[\quantifs{\forall n\in\N}v_n\in\Rps\qquad\text{et}\qquad\lim_{n\to\pinf}\dfrac{v_{n+1}}{v_n}=l.\] \\

Montrer : \[\lim_{n\to\pinf}\sqrt[n]{v_n}=l.\] \\

\item Calculer \(\lim_{n\to\pinf}\sqrt[n]{n!}\). \\

\item Calculer \(\lim_{n\to\pinf}\sqrt[n]{n^2}\).
\end{enumerate}
\end{exo}

\begin{corr}
\note{à venir}
\end{corr}

\begin{exo}[Exercice 14]
\begin{enumerate}[series=ex14suites]
\item Montrer que pour tout entier \(n\in\Ns\), l'équation d'inconnue \(x\in\Rps\) : \[x+\ln x=n\] admet une unique solution.
\end{enumerate}

Dans la suite, on note \(u_n\) cette unique solution et on s'intéresse à la suite \(\paren{u_n}_{n\in\Ns}\).

\begin{enumerate}[resume=ex14suites]
\item Montrer que \(\paren{u_n}_{n\in\Ns}\) est croissante. \\

\item Déterminer sa limite.
\end{enumerate}
\end{exo}

\begin{corr}
\note{à venir}
\end{corr}