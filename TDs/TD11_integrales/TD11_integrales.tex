\chapter{Intégrales sur un segment}

\minitoc

\begin{exo}[Exercice 1]
Soit \(\alpha\in\Rps\).

Calculer l'intégrale \[\int_0^1\dfrac{\odif{t}}{1+t^{\alpha}}\] pour \(\alpha=1\), \(\alpha=2\), \(\alpha=3\) et \(\alpha=4\).
\end{exo}

\begin{corr}
\note{À venir}
\end{corr}

\begin{exo}[Exercice 2]
Déterminer une primitive pour chacun des fonctions suivantes (on précisera sur quel ensemble de définition) :

\begin{enumerate}
\item \(f:x\mapsto\dfrac{1}{x^4-x^2-2}\) \\

\item \(f:x\mapsto\dfrac{x}{1+x^4}\) \\

\item \(f:x\mapsto\dfrac{x^2}{1+x^3}\) \\

\item \(f:x\mapsto\dfrac{1}{x+\i}\) \\

\item \(f=\cos^3\) \\

\item \(f=\tan\) \\

\item \(f=\Arctan\) \\

\item \(f:x\mapsto x\Arctan x\) \\

\item \(f:x\mapsto\dfrac{1}{\cos x}\) \\

\item \(f:x\mapsto\dfrac{\sin x}{\cos^2x+2}\) \\

\item \(f:x\mapsto x^7\ln x\) \\

\item \(f:x\mapsto\ln^3x\) \\

\item \(f:x\mapsto\dfrac{x}{\sqrt{1+x^2}}\) \\

\item \(f:x\mapsto\dfrac{1}{\sqrt{x}+2x}\) \\

\item \(f:x\mapsto\dfrac{1}{x\ln x}\) \\

\item \(f:x\mapsto\dfrac{1}{1+\e{x}}\)
\end{enumerate}
\end{exo}

\begin{corr}
\note{À venir}
\end{corr}

\begin{exo}[Exercice 3]
On pose \[f:x\mapsto\int_0^{\sin^2x}\Arcsin\sqrt{t}\odif{t}+\int_0^{\cos^2x}\Arccos\sqrt{t}\odif{t}.\]

\begin{enumerate}
\item Quel est l'ensemble de définition de \(f\) ? \\

\item Déterminer \(f\).
\end{enumerate}
\end{exo}

\begin{corr}
\note{À venir}
\end{corr}

\begin{exo}[Exercice 4]
Soient \(a,b\in\R\) tels que \(a<b\).

On pose : \[f:x\mapsto\dfrac{1}{\sqrt{\paren{x-a}\paren{b-x}}}.\]

\begin{enumerate}
\item Quel est l'ensemble de définition de \(f\) ? \\

\item Déterminer une primitive de \(f\).

\textit{Indication :} faire le changement de variable \[y=\dfrac{2}{b-a}\paren{x-\dfrac{a+b}{2}}.\]
\end{enumerate}
\end{exo}

\begin{corr}
\note{À venir}
\end{corr}

\begin{exo}[Exercice 5]
On pose \[I=\int_0^{\frac{\pi}{2}}\dfrac{\sin t}{\sin t+\cos t}\odif{t}\qquad\text{et}\qquad J=\int_0^{\frac{\pi}{2}}\dfrac{\cos t}{\sin t+\cos t}\odif{t}.\]

\begin{enumerate}
\item Justifier que les intégrales \(I\) et \(J\) sont bien définies. \\

\item Montrer \(I=J\) par un changement de variable. \\

\item En déduire la valeur de \(I\) et \(J\). \\

\item En déduire \(\int_0^1\dfrac{\odif{t}}{t+\sqrt{1-t^2}}\).
\end{enumerate}
\end{exo}

\begin{corr}
\note{À venir}
\end{corr}

\begin{exo}[Exercice 6]
Soit \(f\in\ensclasse{2}{\R}{\R}\).

On suppose \[\lim_{x\to\pinf}f\paren{x}=\lim_{x\to\pinf}f\seconde\paren{x}=0.\]

Montrer : \[\lim_{x\to\pinf}f\prim\paren{x}=0.\]

\textit{Indication :} on pourra utiliser la formule de Taylor avec reste intégral.
\end{exo}

\begin{corr}
\note{À venir}
\end{corr}

\begin{exo}[Exercice 7, classique à l'oral, de CCP à l'X]
On pose \[\fonction{f}{\Rps\excluant\accol{1}}{\R}{x}{\int_x^{x^2}\dfrac{\odif{t}}{\ln t}}\]

\begin{enumerate}
\item Vérifier que la fonction \(f\) est bien définie. Quel est son signe ? \\

\item Dresser le tableau de variations de \(f\) (en précisant les limites en \(0^+\), \(1^-\), \(1^+\) et \(\pinf\)).

\textit{Indication :} pour la limite en \(1^+\), montrer : \[\quantifs{\forall x\in\intervee{1}{\pinf}}\int_x^{x^2}\dfrac{x\odif{t}}{t\ln t}\leq f\paren{x}\leq\int_x^{x^2}\dfrac{x^2\odif{t}}{t\ln t}.\]

\item En déduire que \(f\) se prolonge en une fonction continue \(g:\Rp\to\R\). \\

\item Montrer que \(g\) est de classe \(\classe{1}\).
\end{enumerate}
\end{exo}

\begin{corr}
\note{À venir}
\end{corr}

\begin{exo}[Exercice 8, lemme de Lebesgue]
Soient \(a,b\in\R\) tels que \(a<b\) et \(f\in\ensclasse{1}{\intervii{a}{b}}{\C}\).

Montrer : \[\lim_{n\to\pinf}\int_a^bf\paren{t}\cos\paren{nt}\odif{t}=0.\]

\textit{Indication :} faire une intégration par parties.
\end{exo}

\begin{corr}
\note{À venir}
\end{corr}

\begin{exo}[Exercice 9]
À l'aide de la formule de Taylor avec reste intégral, montrer :

\begin{enumerate}
\item \(\quantifs{\forall x\in\intervei{-1}{1}}\ln\paren{1+x}=\lim_{n\to\pinf}\sum_{k=1}^n\dfrac{\paren{-1}^{k-1}x^k}{k}\). \\

\item \(\quantifs{\forall n\in\Ns;\forall x\in\intervei{-1}{0}}\ln\paren{1+x}\leq\sum_{k=1}^n\dfrac{\paren{-1}^{k-1}x^k}{k}\). \\

\item \(\quantifs{\forall n\in\Ns;\forall x\in\intervie{0}{\pinf}}\sum_{k=1}^{2n}\dfrac{\paren{-1}^{k-1}x^k}{k}\leq\ln\paren{1+x}\leq\sum_{k=1}^{2n+1}\dfrac{\paren{-1}^{k-1}x^k}{k}\).
\end{enumerate}
\end{exo}

\begin{corr}
\note{À venir}
\end{corr}

\begin{exo}[Exercice 10]
Soit \(f:\intervii{0}{1}\to\intervii{0}{1}\) une fonction continue telle que : \[\int_0^1f\paren{t}\odif{t}=\int_0^1f\paren{t}^2\odif{t}.\]

Montrer que \(f\) est constante, égale à \(0\) ou \(1\).
\end{exo}

\begin{corr}
\note{À venir}
\end{corr}

\begin{exo}[Exercice 11]
Soit une fonction continue \(f:\intervii{0}{1}\to\R\) telle que : \[\int_0^1f\paren{t}\odif{t}=\dfrac{1}{2}.\]

Montrer que \(f\) admet un point fixe.

Donner un contre-exemple quand \(f\) n'est pas supposée continue (et seulement supposée continue par morceaux).
\end{exo}

\begin{corr}
\note{À venir}
\end{corr}

\begin{exo}[Exercice 12]
Calculer la fonction \[\fonction{f}{\Rs}{\R}{x}{\int_{\frac{1}{x}}^x\dfrac{t}{1+t+t^2+t^3}\odif{t}}\]
\end{exo}

\begin{corr}
\note{À venir}
\end{corr}

\begin{exo}[Exercice 13, \thref{dem:convergenceDesSommesDeRiemannDansLeCasContinu}]\thlabel{exo:convergenceDesSommesDeRiemannDansLeCasContinu}
Soient \(a,b\in\R\) tels que \(a<b\) et \(f:\intervii{a}{b}\to\R\) continue.

On considère les sommes de Riemann de \(f\) : \[\quantifs{\forall n\in\Ns}R_n\paren{f}=\dfrac{b-a}{n}\sum_{k=0}^{n-1}f\paren{a+k\dfrac{b-a}{n}}.\]

Montrer que la suite \(\paren{R_n\paren{f}}_{n\in\Ns}\) converge vers l'intégrale de \(f\) sur \(\intervii{a}{b}\).
\end{exo}

\begin{corr}
\note{À venir}
\end{corr}

\begin{exo}[Exercice 14]
Déterminer \[\lim_{n\to\pinf}\sum_{k=1}^{2n}\dfrac{k}{n^2+k^2}.\]
\end{exo}

\begin{corr}
\note{À venir}
\end{corr}

\begin{exo}[Exercice 15]
Déterminer \[\lim_{n\to\pinf}\sum_{k=n}^{2n}\dfrac{1}{k}.\]
\end{exo}

\begin{corr}
\note{À venir}
\end{corr}

\begin{exo}[Exercice 16]
Déterminer \[\lim_{n\to\pinf}\sum_{k=n}^{2n}\dfrac{1}{2k+1}.\]
\end{exo}

\begin{corr}
\note{À venir}
\end{corr}

\begin{exo}[Exercice 17]
Déterminer \[\lim_{n\to\pinf}\dfrac{1}{n}\sqrt[n]{\dfrac{\paren{2n}!}{n!}}.\]

\textit{Indication :} remarquer que pour tout \(n\in\Ns\), on a : \[\prod_{k=1}^n\paren{n+k}=n^n\prod_{k=1}^n\paren{1+\dfrac{k}{n}}.\]
\end{exo}

\begin{corr}
\note{À venir}
\end{corr}

\begin{exo}[Exercice 18, ENSEA 2018]
\begin{enumerate}
\item À l'aide de la formule de Taylor avec reste intégral, montrer : \[\quantifs{\forall x\in\intervii{0}{1}}0\leq\e{x}-1-x\leq\dfrac{\e{}}{2}x^2.\]

\item En déduire la limite de la suite \(\paren{u_n}_n\) définie par : \[\quantifs{\forall n\in\Ns}u_n=\sum_{k=1}^n\e{\frac{1}{n+k}}-n.\]
\end{enumerate}
\end{exo}

\begin{corr}
\note{À venir}
\end{corr}