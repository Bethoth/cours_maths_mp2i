\chapter{Inégalités, calculs}

\minitoc

\begin{exo}
\begin{enumerate}
\item Montrer : \[\quantifs{\forall x,y\in\Rps}\dfrac{xy}{x+y}\leq\dfrac{x+y}{4}.\] \\

\item En déduire : \[\quantifs{\forall x,y,z\in\Rps}\dfrac{xy}{x+y}+\dfrac{yz}{y+z}+\dfrac{xz}{x+z}\leq\dfrac{x+y+z}{2}.\]
\end{enumerate}
\end{exo}

\begin{corr}
\note{à venir}
\end{corr}

\begin{exo}
Résoudre les inéquations d'inconnue \(x\in\R\) :

\begin{enumerate}
\item \(\ln\paren{1+x}\leq1+\ln x\) \\

\item \(\sqrt{\ln\paren{1+x^2}}\geq2\) \\

\item \(\dfrac{1}{x}<-1\) \\

\item \(\dfrac{1}{x}\leq1\) \\

\item \(\sqrt{3-x}-\sqrt{x+1}>\dfrac{1}{2}\) \\

\item \(\floor{\sqrt{x^2+1}}\leq2\) \\

\item \(\floor{x^2-4x}=0\)
\end{enumerate}
\end{exo}

\begin{corr}
\note{à venir}
\end{corr}

\begin{exo}
Soit \(x\in\intervie{1}{\pinf}\).

Montrer : \[\dfrac{\paren{x-1}^2}{8x}\leq\dfrac{x+1}{2}-\sqrt{x}\leq\dfrac{\paren{x-1}^2}{8}.\]
\end{exo}

\begin{corr}
\note{à venir}
\end{corr}

\begin{exo}
Soit \(n\in\Ns\).

Calculer \[\sum_{k=1}^n\ln\paren{1+\dfrac{1}{k}}\quad\text{et}\quad\sum_{k=1}^{n}\ln\paren{1+\dfrac{2}{k}}.\]
\end{exo}

\begin{corr}
\note{à venir}
\end{corr}

\begin{exo}
Soit \(n\in\Ns\).

Calculer \[\sum_{k=1}^{n}\dfrac{1}{k\paren{k+1}}.\]
\end{exo}

\begin{corr}
\note{à venir}
\end{corr}

\begin{exo}
Soit \(n\in\N\).

Calculer \[\sum_{k=1}^{n}k\times k!\]
\end{exo}

\begin{corr}
\note{à venir}
\end{corr}

\begin{exo}
Soit \(n\in\N\).

Calculer \[\sum_{k=1}^{n}\dfrac{k}{k^4+k^2+1}.\]

\textit{Indication :} factoriser le dénominateur en remarquant \(k^4+k^2+1=k^4+2k^2+1-k^2\).
\end{exo}

\begin{corr}
\note{à venir}
\end{corr}

\begin{exo}
Soit \(n\in\N\).

Calculer \[\sum_{k=1}^{n}k^3.\]

Que remarque-t-on ?

\textit{Indication :} s'inspirer du calcul de \(\sum_{k=1}^{n}k^2\) vu en cours.
\end{exo}

\begin{corr}
\note{à venir}
\end{corr}

\begin{exo}
Montrer la proposition suivante : \[\quantifs{\forall n\in\N;\forall x\in\R\excluant\accol{1}}\prod_{k=0}^{n}\paren{x^{2^k}+1}=\dfrac{x^{2^{n+1}}-1}{x-1}.\]
\end{exo}

\begin{corr}
\note{à venir}
\end{corr}

\begin{exo}
Soit \(n\in\N\).

Calculer les sommes suivantes : \[S_1=\sum_{i=1}^{n}\sum_{j=1}^{n}2\qquad S_2=\sum_{i=1}^{n}\sum_{j=1}^{n}2^i\qquad S_3=\sum_{i=1}^{n}\sum_{j=1}^{n}2^j\qquad S_4=\sum_{i=1}^{n}\sum_{j=1}^{n}2^{i+j}.\]
\end{exo}

\begin{corr}
\note{à venir}
\end{corr}

\begin{exo}
Soit \(n\in\N\).

Calculer les sommes suivantes : \[S_1=\sum_{i=1}^{n}\sum_{j=1}^{n}\dfrac{1}{j}\qquad S_2=\sum_{i=1}^{n}\sum_{j=1}^{n}\dfrac{i+j}{j}\qquad S_3=\sum_{i=1}^{n}\sum_{j=1}^{n}\dfrac{\paren{i+j}^2}{j}.\]
\end{exo}

\begin{corr}
\note{à venir}
\end{corr}

\begin{exo}
Soit \(n\in\N\).

Calculer \[\sum_{i=1}^{n}\sum_{j=1}^{n}\min\accol{i;j}.\]
\end{exo}

\begin{corr}
\note{à venir}
\end{corr}

\begin{exo}
On définit la suite \(\paren{u_n}_{n\in\Ns}\) en posant : \[u_1=1\qquad u_2=u_3=2\qquad u_4=u_5=u_6=3\qquad u_7=u_8=u_9=u_{10}=4\qquad \dots\]

Combien vaut \(u_{30}\) ?

Exprimer \(u_n\) en fonction de \(n\in\Ns\) à l'aide de la fonction partie entière.
\end{exo}

\begin{corr}
\note{à venir}
\end{corr}

\begin{exo}
Soit \(n\in\N\).

Montrer : \[\floor{\paren{\sqrt{n}+\sqrt{n+1}}^2}=4n+1.\]
\end{exo}

\begin{corr}
\note{à venir}
\end{corr}

\begin{exo}
Soit \(n\in\Ns\).

Montrer que la partie entière de \(\paren{2+\sqrt{3}}^n\) est un entier impair.
\end{exo}

\begin{corr}
\note{à venir}
\end{corr}

\begin{exo}[Classique]
Soit \(n\in\Ns\).

Calculer \[\prod_{k=1}^{n}2k\qquad\text{puis}\qquad\prod_{k=1}^{n}\paren{2k-1}.\]

\textit{NB : on exprimera le résultat à l'aide de factorielles, sans symbole \(\prod\) ni points de suspension.}
\end{exo}

\begin{corr}
\note{à venir}
\end{corr}