\chapter{Fonctions de deux variables réelles}

\minitoc

\begin{exo}[Exercice 1, ouverts de \(\R^2\)]
\begin{enumerate}
    \item Soit \(\paren{U_i}_{i\in I}\) une famille d'ouverts de \(\R^2\) (où \(I\) est un ensemble). Montrer que \(\bigunion_{i\in I}U_i\) est un ouvert de \(\R^2\). \\
    \item Soient \(U_1,\dots,U_n\) des ouverts de \(\R^2\) (où \(n\in\Ns\)). Montrer que \(\biginter_{k=1}^n U_i\) est un ouvert de \(\R^2\).
\end{enumerate}
\end{exo}

\begin{corr}
\note{À venir}
\end{corr}

\begin{exo}[Exercice 2]
En quels points la fonction \[\fonction{f}{\R^2}{\R}{\paren{x,y}}{\begin{dcases}
\dfrac{y^2}{x^2+y^2} &\text{si }\paren{x,y}\not=\paren{0,0} \\
0 &\text{sinon}
\end{dcases}}\] est-elle continue ?
\end{exo}

\begin{corr}
\note{À venir}
\end{corr}

\begin{exo}[Exercice 3]
En quels points la fonction \[\fonction{f}{\R^2}{\R}{\paren{x,y}}{\begin{dcases}
\dfrac{xy}{x^2+y^2} &\text{si }\paren{x,y}\not=\paren{0,0} \\
0 &\text{sinon}
\end{dcases}}\]

\begin{enumerate}
    \item En quels points \(f\) est-elle continue ? \\
    \item En quels points \(f\) admet-elle des dérivées partielles ? \\
    \item \(f\) est-elle de classe \(\classe{1}\) ?
\end{enumerate}
\end{exo}

\begin{corr}
\note{À venir}
\end{corr}

\begin{exo}[Exercice 4, à propos du théorème de Schwarz]
On pose : \[\quantifs{\forall x,y\in\R}f\paren{x,y}=\begin{dcases}
\dfrac{xy\paren{x^2-y^2}}{x^2+y^2} &\text{si }\paren{x,y}\not=\paren{0,0} \\
0 &\text{sinon}
\end{dcases}\]

\begin{enumerate}
    \item La fonction \(f:\R^2\to\R\) est-elle continue ? \\
    \item Admet-elle des dérivées partielles ? Le cas échéant, les calculer. \\
    \item Est-elle de classe \(\classe{1}\) ? \\
    \item Calculer \(\pdv{f}{y,x}\paren{0,0}\) et \(\pdv{f}{x,y}\paren{0,0}\).
\end{enumerate}
\end{exo}

\begin{corr}[1]
On a \(\quantifs{\forall\paren{x,y}\in\R^2\excluant\accol{\paren{0,0}}}\abs{\dfrac{x^2-y^2}{x^2+y^2}}\leq1\).

Donc \(\quantifs{\forall\paren{x,y}\in\R^2\excluant\accol{\paren{0,0}}}\abs{f\paren{x,y}}\leq\abs{xy}\).

Si \(\begin{dcases}
\abs{x}\leq\epsilon \\
\abs{y}\leq1
\end{dcases}\) alors \(\abs{f\paren{x,y}}\leq\epsilon\) donc \(\delta=\epsilon\) convient.

Donc \(f\) est continue.
\end{corr}

\begin{corr}[2]
On a : \[\begin{aligned}
\quantifs{\forall\paren{x,y}\in\R^2\excluant\accol{\paren{0,0}}}\pdv{f}{x}\paren{x,y}&=y\paren{\dfrac{x^2-y^2}{x^2+y^2}}+yx\paren{2y^2\times2x\times\paren{x^2+y^2}^{-2}} \\
&=\dfrac{y\paren{x^2-y^2}\paren{x^2+y^2}+4y^3x^2}{\paren{x^2+y^2}^2} \\
&=\dfrac{y\paren{x^4-y^4+4y^2x^2}}{\paren{x^2+y^2}^2}
\end{aligned}\] et \(\quantifs{\forall x\in\R}f\paren{x,0}=0\).

Donc \(\pdv{f}{x}\paren{0,0}=0\).

On obtient de même : \[\quantifs{\forall\paren{x,y}\in\R^2}\pdv{f}{y}\paren{x,y}=\begin{dcases}
-\dfrac{x\paren{y^4-x^4+4x^2y^2}}{\paren{y^2+x^2}^2} &\text{si }\paren{x,y}\not=\paren{0,0} \\
0 &\text{sinon}
\end{dcases}\] car on a : \[\begin{WithArrows}
\quantifs{\forall\paren{x,y}\in\R^2}f\paren{x,y}&=-f\paren{y,x} \Arrow{\(\pdv{}{y}\)} \\
\pdv{f}{y}\paren{x,y}&=-\pdv{f}{x}\paren{y,x}
\end{WithArrows}\]
\end{corr}

\begin{exo}[Exercice 5]
On pose : \[f\paren{x,y}=\Arctan x+\Arctan y-\Arctan\dfrac{x+y}{1-xy}.\]

\begin{enumerate}
    \item Donner trois ouverts \guillemets{naturels} \(U_1\), \(U_2\) et \(U_3\) tels que l'ensemble de définition de \(f\) soit \(U_1\union U_2\union U_3\). \\
    \item Montrer que \(f\) est de classe \(\classe{1}\). \\
    \item Déterminer \(f\).
\end{enumerate}
\end{exo}

\begin{corr}
\note{À venir}
\end{corr}

\begin{exo}[Exercice 6]
Dire pour chacune des équations suivantes s'il existe une solution (fonction de classe \(\classe{1}\) sur l'ensemble indiqué).

\[\quantifs{\forall\paren{x,y}\in\R^2\excluant\accol{\paren{0,0}}}\paren{E_1}~\nabla f\paren{x,y}=\dfrac{1}{x^2+y^2}\dcoords{-y}{x}\]

\[\quantifs{\forall\paren{x,y}\in\Rps\times\R}\paren{E_2}~\nabla f\paren{x,y}=\dfrac{1}{x^2+y^2}\dcoords{-y}{x}\]

\[\quantifs{\forall\paren{x,y}\in\R^2}\paren{E_3}~\nabla f\paren{x,y}=\dcoords{-y}{x}\]

\[\quantifs{\forall\paren{x,y}\in\Rps\times\R}\paren{E_4}~\nabla f\paren{x,y}=\dcoords{-y}{x}\]

\textit{Indication :} pour \(\paren{E_1}\), supposer par l'absurde qu'une telle fonction \(f\) existe et calculer la dérivée de \[\fonction{g}{\R}{\R}{\theta}{f\paren{\cos\theta,\sin\theta}}\]

\textit{Remarque :} concernant \(\paren{E_3}\) et \(\paren{E_4}\), la façon la plus simple de conclure serait d'utiliser le théorème de Schwarz que vous verrez en deuxième année.
\end{exo}

\begin{corr}
\note{À venir}
\end{corr}

\begin{exo}[Exercice 7]
Soit \(f\in\ensclasse{1}{\R^2}{\R}\) une fonction telle que : \[\quantifs{\forall t,x,y\in\R}f\paren{tx,ty}=tf\paren{x,y}.\]

\begin{enumerate}
    \item Montrer : \[\quantifs{\forall t,x,y\in\R}f\paren{x,y}=x\pdv{f}{x}\paren{tx,ty}+y\pdv{f}{y}\paren{tx,ty}.\] \\ \textit{Indication :} dériver de deux façons la fonction \[\fonction{g}{\R}{\R}{t}{f\paren{tx,ty}}\]
    \item En déduire que la fonction \(f\) est linéaire.
\end{enumerate}
\end{exo}

\begin{corr}
\note{À venir}
\end{corr}

\begin{rem}[Compléments de cours]
\begin{itemize}
    \item On dit qu'une partie \(A\subset\R^2\) est un fermé de \(\R^2\) si son complémentaire est un ouvert de \(\R^2\). \\
    \item On admet le résultat suivant, qui sera un théorème du cours de deuxième année.
\end{itemize}
\end{rem}

\begin{theo}
Soit une fonction continue \(f:A\to\R\) sur une partie \(A\subset\R^2\) non-vide, fermée et bornée.

Alors \(f\) est \guillemets{bornée et atteint ses bornes}, ce qui signifie qu'elle admet un maximum et un minimum (globaux).
\end{theo}

\begin{exo}[Exercice 8]\thlabel{exo:exempleExoExtrema}
Étudier les extrema de la fonction \[\fonction{f}{\R^2}{\R}{\paren{x,y}}{x^2+xy+y^2-2x-y}\]
\end{exo}

\begin{corr}
\analyse

Soit \(\paren{x,y}\in\R^2\).

Supposons que \(f\) admet un extremum local en \(\paren{x,y}\).

Comme \(\R^2\) est un ouvert, on a \(\nabla f\paren{x,y}=0\), \cad : \[\begin{dcases}
\pdv{f}{x}\paren{x,y}=0 \\
\pdv{f}{y}\paren{x,y}=0
\end{dcases}\]

Or on a : \[\begin{dcases}
\pdv{f}{x}\paren{x,y}=2x+y-2 \\
\pdv{f}{y}\paren{x,y}=x+2y-1
\end{dcases}\]

D'où le système \(\begin{dcases}
2x+y=2 \\
x+2y=1
\end{dcases}\)

On a \(\begin{vmatrix}
2 & 1 \\
1 & 2
\end{vmatrix}\not=0\) donc d'après les formules de Cramer, on a : \[x=\dfrac{\begin{vmatrix}2 & 1 \\ 1 & 2\end{vmatrix}}{\begin{vmatrix}2 & 1 \\ 1 & 2\end{vmatrix}}=1\qquad\text{et}\qquad y=\dfrac{\begin{vmatrix}2 & 2 \\ 1 & 1\end{vmatrix}}{\begin{vmatrix}2 & 1 \\ 1 & 2\end{vmatrix}}=0.\]

\synthese

Soient \(h,k\in\R\).

On a : \[\begin{aligned}
f\paren{1+h,0+k}&=\paren{1+h}^2+k\paren{1+h}+k^2-2\paren{1+h}-k+1 \\
&=h^2+k^2+kh \\
&=\dfrac{h^2}{2}+\dfrac{k^2}{2}+\dfrac{1}{2}\paren{h+k}^2 \\
&\geq0.
\end{aligned}\]

Donc \(f\) admet un minimum en \(\paren{1,0}\).

\conclusion

\(f\) admet un minimum en \(\paren{1,0}\) et n'admet pas de maximum.
\end{corr}

\begin{exo}[Exercice 9]
Étudier les extrema de la fonction \[\fonction{f}{\R^2}{\R}{\paren{x,y}}{\dfrac{x+y}{\paren{1+x^2}\paren{1+y^2}}}\]
\end{exo}

\begin{corr}
\note{À venir}
\end{corr}

\begin{exo}[Exercice 10]
Étudier les extrema de la fonction \[\fonction{f}{\R^2}{\R}{\paren{x,y}}{x^3y^2\paren{1+x-y}}\]
\end{exo}

\begin{corr}
\note{À venir}
\end{corr}

\begin{exo}[Exercice 11, CCP PSI 2015 BEOS]
Soient les ensembles \[K=\accol{\paren{x,y}\in\R^2\tq0\leq x\leq\pi\text{ et }0\leq y\leq\pi}\qquad\text{et}\qquad T=\accol{\paren{x,y}\in\R^2\tq0<x<y<\pi}.\]

On pose : \[\quantifs{\forall\paren{x,y}\in K}F\paren{x,y}=\begin{dcases}
x\paren{\pi-y} &\text{si }0\leq x\leq y\leq\pi \\
y\paren{\pi-x} &\text{si }0\leq y\leq x\leq\pi
\end{dcases}\]

\begin{enumerate}
    \item Représenter \(K\) et \(T\). \\
    \item La fonction \(F\) admet-elle des extrema locaux sur \(T\) ? \\
    \item La fonction \(F\) admet-elle un minimum sur \(K\) ? un maximum sur \(K\) ? Le cas échéant, déterminer leur valeur.
\end{enumerate}
\end{exo}

\begin{corr}
\note{À venir}
\end{corr}

\begin{exo}[Exercice 12]\thlabel{exo:EDPChangementDeVariable}
Résoudre l'équation aux dérivées partielles : \[\paren{E}~x\pdv{f}{x}+y\pdv{f}{y}=x^3y^3\] d'inconnue \(f\in\ensclasse{1}{\paren{\Rps}^2}{\R}\) à l'aide du changement de variables : \[\paren{u,v}=\paren{xy,\dfrac{x}{y}}.\]
\end{exo}

\begin{corr}~\\
\begin{brouill}~\\
On a \(\begin{dcases}
x,y\in\Rps \\
u=xy \\
v=\dfrac{x}{y}
\end{dcases}\ssi\begin{dcases}
u,v\in\Rps \\
x=\sqrt{uv} \\
y=\sqrt{\dfrac{u}{v}}
\end{dcases}\) et \(f\paren{x,y}=f\paren{\sqrt{uv},\sqrt{\dfrac{u}{v}}}=g\paren{u,v}=g\paren{xy,\dfrac{x}{y}}\).
\end{brouill}

Soit \(f\in\ensclasse{1}{\paren{\Rps}^2}{\R}\).

On pose \(\fonction{g}{\paren{\Rps}^2}{\R}{\paren{u,v}}{f\paren{\sqrt{uv},\sqrt{\dfrac{u}{v}}}}\). \(g\) est de classe \(\classe{1}\).

On a \(\quantifs{\forall\paren{x,y}\in\paren{\Rps}^2}\begin{dcases}
f\paren{x,y}=g\paren{xy,\dfrac{x}{y}} \\
\pdv{f}{x}\paren{x,y}=y\pdv{g}{u}\paren{xy,\dfrac{x}{y}}+\dfrac{1}{y}\pdv{g}{v}\paren{xy,\dfrac{x}{y}} \\
\pdv{f}{y}\paren{x,y}=x\pdv{g}{u}\paren{xy,\dfrac{x}{y}}-\dfrac{x}{y^2}\pdv{g}{v}\paren{xy,\dfrac{x}{y}}
\end{dcases}\)

Donc : \[\begin{aligned}
f\text{ est solution de }\paren{E}&\ssi\quantifs{\forall\paren{x,y}\in\paren{\Rps}^2}2xy\pdv{g}{u}\paren{xy,\dfrac{x}{y}}=x^3y^3 \\
&\ssi\quantifs{\forall\paren{u,v}\in\paren{\Rps}^2}2\pdv{g}{u}\paren{u,v}=u^2 \\
&\ssi\quantifs{\exists h\in\F{\Rps}{\R};\forall\paren{u,v}\in\paren{\Rps}^2}2g\paren{u,v}=\dfrac{u^3}{3}+h\paren{v} \\
&\ssi\quantifs{\exists h\in\F{\Rps}{\R};\forall\paren{x,y}\in\paren{\Rps}^2}2f\paren{x,y}=\dfrac{x^3y^3}{3}+h\paren{\dfrac{x}{y}}.
\end{aligned}\]

Donc les solutions de \(\paren{E}\) sont les fonctions de la forme \[\paren{x,y}\mapsto\dfrac{x^3y^3}{6}+h\paren{\dfrac{x}{y}}\] avec \(h\in\ensclasse{1}{\Rps}{\R}\).
\end{corr}

\begin{exo}[Exercice 13]
Résoudre l'équation aux dérivées partielles : \[\paren{E}~x\pdv{f}{x}-y\pdv{f}{y}=\paren{x^2+y^2}\dfrac{y}{x}\] d'inconnue \(f\in\ensclasse{1}{\Rps\times\R}{\R}\), en utilisant les coordonnées polaires : \[\paren{x,y}=\paren{r\cos\theta,r\sin\theta}.\] 
\end{exo}

\begin{corr}
\note{À venir}
\end{corr}