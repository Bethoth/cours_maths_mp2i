\chapter{Fonctions de deux variables réelles}

\minitoc

\begin{exo}[Exercice 1, ouverts de \(\R^2\)]
\begin{enumerate}
    \item Soit \(\paren{U_i}_{i\in I}\) une famille d'ouverts de \(\R^2\) (où \(I\) est un ensemble). Montrer que \(\bigunion_{i\in I}U_i\) est un ouvert de \(\R^2\). \\
    \item Soient \(U_1,\dots,U_n\) des ouverts de \(\R^2\) (où \(n\in\Ns\)). Montrer que \(\biginter_{k=1}^n U_i\) est un ouvert de \(\R^2\).
\end{enumerate}
\end{exo}

\begin{corr}
\note{À venir}
\end{corr}

\begin{exo}[Exercice 2]
En quels points la fonction \[\fonction{f}{\R^2}{\R}{\paren{x,y}}{\begin{dcases}
\dfrac{y^2}{x^2+y^2} &\text{si }\paren{x,y}\not=\paren{0,0} \\
0 &\text{sinon}
\end{dcases}}\] est-elle continue ?
\end{exo}

\begin{corr}
\note{À venir}
\end{corr}

\begin{exo}[Exercice 3]
En quels points la fonction \[\fonction{f}{\R^2}{\R}{\paren{x,y}}{\begin{dcases}
\dfrac{xy}{x^2+y^2} &\text{si }\paren{x,y}\not=\paren{0,0} \\
0 &\text{sinon}
\end{dcases}}\]

\begin{enumerate}
    \item En quels points \(f\) est-elle continue ? \\
    \item En quels points \(f\) admet-elle des dérivées partielles ? \\
    \item \(f\) est-elle de classe \(\classe{1}\) ?
\end{enumerate}
\end{exo}

\begin{corr}
\note{À venir}
\end{corr}

\begin{exo}[Exercice 4, à propos du théorème de Schwarz]
On pose : \[\quantifs{\forall x,y\in\R}f\paren{x,y}=\begin{dcases}
\dfrac{xy\paren{x^2-y^2}}{x^2+y^2} &\text{si }\paren{x,y}\not=\paren{0,0} \\
0 &\text{sinon}
\end{dcases}\]

\begin{enumerate}
    \item La fonction \(f:\R^2\to\R\) est-elle continue ? \\
    \item Admet-elle des dérivées partielles ? Le cas échéant, les calculer. \\
    \item Est-elle de classe \(\classe{1}\) ? \\
    \item Calculer \(\pdv{f}{y,x}\paren{0,0}\) et \(\pdv{f}{x,y}\paren{0,0}\).
\end{enumerate}
\end{exo}

\begin{corr}
\note{À venir}
\end{corr}

\begin{exo}[Exercice 5]
On pose : \[f\paren{x,y}=\Arctan x+\Arctan y-\Arctan\dfrac{x+y}{1-xy}.\]

\begin{enumerate}
    \item Donner trois ouverts \guillemets{naturels} \(U_1\), \(U_2\) et \(U_3\) tels que l'ensemble de définition de \(f\) soit \(U_1\union U_2\union U_3\). \\
    \item Montrer que \(f\) est de classe \(\classe{1}\). \\
    \item Déterminer \(f\).
\end{enumerate}
\end{exo}

\begin{corr}
\note{À venir}
\end{corr}

\begin{exo}[Exercice 6]
Dire pour chacune des équations suivantes s'il existe une solution (fonction de classe \(\classe{1}\) sur l'ensemble indiqué).

\[\quantifs{\forall\paren{x,y}\in\R^2\excluant\accol{\paren{0,0}}}\paren{E_1}~\nabla f\paren{x,y}=\dfrac{1}{x^2+y^2}\dcoords{-y}{x}\]

\[\quantifs{\forall\paren{x,y}\in\Rps\times\R}\paren{E_2}~\nabla f\paren{x,y}=\dfrac{1}{x^2+y^2}\dcoords{-y}{x}\]

\[\quantifs{\forall\paren{x,y}\in\R^2}\paren{E_3}~\nabla f\paren{x,y}=\dcoords{-y}{x}\]

\[\quantifs{\forall\paren{x,y}\in\Rps\times\R}\paren{E_4}~\nabla f\paren{x,y}=\dcoords{-y}{x}\]

\textit{Indication :} pour \(\paren{E_1}\), supposer par l'absurde qu'une telle fonction \(f\) existe et calculer la dérivée de \[\fonction{g}{\R}{\R}{\theta}{f\paren{\cos\theta,\sin\theta}}\]

\textit{Remarque :} concernant \(\paren{E_3}\) et \(\paren{E_4}\), la façon la plus simple de conclure serait d'utiliser le théorème de Schwarz que vous verrez en deuxième année.
\end{exo}

\begin{corr}
\note{À venir}
\end{corr}

\begin{exo}[Exercice 7]
Soit \(f\in\ensclasse{1}{\R^2}{\R}\) une fonction telle que : \[\quantifs{\forall t,x,y\in\R}f\paren{tx,ty}=tf\paren{x,y}.\]

\begin{enumerate}
    \item Montrer : \[\quantifs{\forall t,x,y\in\R}f\paren{x,y}=x\pdv{f}{x}\paren{tx,ty}+y\pdv{f}{y}\paren{tx,ty}.\] \\ \textit{Indication :} dériver de deux façons la fonction \[\fonction{g}{\R}{\R}{t}{f\paren{tx,ty}}\]
    \item En déduire que la fonction \(f\) est linéaire.
\end{enumerate}
\end{exo}

\begin{corr}
\note{À venir}
\end{corr}

\begin{rem}[Compléments de cours]
\begin{itemize}
    \item On dit qu'une partie \(A\subset\R^2\) est un fermé de \(\R^2\) si son complémentaire est un ouvert de \(\R^2\). \\
    \item On admet le résultat suivant, qui sera un théorème du cours de deuxième année.
\end{itemize}
\end{rem}

\begin{theo}
Soit une fonction continue \(f:A\to\R\) sur une partie \(A\subset\R^2\) non-vide, fermée et bornée.

Alors \(f\) est \guillemets{bornée et atteint ses bornes}, ce qui signifie qu'elle admet un maximum et un minimum (globaux).
\end{theo}

\begin{exo}[Exercice 8]
Étudier les extrema de la fonction \[\fonction{f}{\R^2}{\R}{\paren{x,y}}{x^2+xy+y^2-2x-y}\]
\end{exo}

\begin{corr}
\note{À venir}
\end{corr}

\begin{exo}[Exercice 9]
Étudier les extrema de la fonction \[\fonction{f}{\R^2}{\R}{\paren{x,y}}{\dfrac{x+y}{\paren{1+x^2}\paren{1+y^2}}}\]
\end{exo}

\begin{corr}
\note{À venir}
\end{corr}

\begin{exo}[Exercice 10]
Étudier les extrema de la fonction \[\fonction{f}{\R^2}{\R}{\paren{x,y}}{x^3y^2\paren{1+x-y}}\]
\end{exo}

\begin{corr}
\note{À venir}
\end{corr}

\begin{exo}[Exercice 11, CCP PSI 2015 BEOS]
Soient les ensembles \[K=\accol{\paren{x,y}\in\R^2\tq0\leq x\leq\pi\text{ et }0\leq y\leq\pi}\qquad\text{et}\qquad T=\accol{\paren{x,y}\in\R^2\tq0<x<y<\pi}.\]

On pose : \[\quantifs{\forall\paren{x,y}\in K}F\paren{x,y}=\begin{dcases}
x\paren{\pi-y} &\text{si }0\leq x\leq y\leq\pi \\
y\paren{\pi-x} &\text{si }0\leq y\leq x\leq\pi
\end{dcases}\]

\begin{enumerate}
    \item Représenter \(K\) et \(T\). \\
    \item La fonction \(F\) admet-elle des extrema locaux sur \(T\) ? \\
    \item La fonction \(F\) admet-elle un minimum sur \(K\) ? un maximum sur \(K\) ? Le cas échéant, déterminer leur valeur.
\end{enumerate}
\end{exo}

\begin{corr}
\note{À venir}
\end{corr}

\begin{exo}[Exercice 12]
Résoudre l'équation aux dérivées partielles : \[\paren{E}~x\pdv{f}{x}+y\pdv{f}{y}=x^3y^3\] d'inconnue \(f\in\ensclasse{1}{\paren{\Rps}^2}{\R}\) à l'aide du changement de variables : \[\paren{u,v}=\paren{xy,\dfrac{x}{y}}.\]
\end{exo}

\begin{corr}
\note{À venir}
\end{corr}

\begin{exo}[Exercice 13]
Résoudre l'équation aux dérivées partielles : \[\paren{E}~x\pdv{f}{x}-y\pdv{f}{y}=\paren{x^2+y^2}\dfrac{y}{x}\] d'inconnue \(f\in\ensclasse{1}{\Rps\times\R}{\R}\), en utilisant les coordonnées polaires : \[\paren{x,y}=\paren{r\cos\theta,r\sin\theta}.\] 
\end{exo}

\begin{corr}
\note{À venir}
\end{corr}