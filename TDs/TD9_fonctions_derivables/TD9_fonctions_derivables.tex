\chapter{Fonctions dérivables}

\minitoc

\section{Étude locale}

\begin{exo}
Étudier la dérivabilité des fonctions suivantes (en précisant leurs ensembles de définition) :

\begin{enumerate}
\item \(f:x\mapsto\cos\sqrt{x}\) \\

\item \(f:x\mapsto\sin\abs{x}\) \\

\item \(f:x\mapsto\ln\paren{1+\sqrt{x}}\) \\

\item \(f:x\mapsto x\abs{x}\) \\

\item \(f:x\mapsto\sqrt{x^2-x^3}\) \\

\item \(f:x\mapsto\paren{1-x}\Arccos x\)
\end{enumerate}
\end{exo}

\begin{corr}
\note{À venir}
\end{corr}

\begin{exo}
Montrer que la fonction \[f:x\mapsto\ln\paren{\dfrac{\e{x}-1}{x}}\] est définie sur \(\Rs\) et se prolonge en une fonction de classe \(\classe{1}\) sur \(\R\).
\end{exo}

\begin{corr}
\note{À venir}
\end{corr}

\begin{exo}
Soient \(f:\R\to\R\) et \(a\in\R\).

On suppose que \(f\) est dérivable en \(a\).

Déterminer \(\lim_{\substack{h\to0 \\ h\not=0}}\dfrac{f\paren{a+2h}-f\paren{a-h}}{h}\).
\end{exo}

\begin{corr}
\note{À venir}
\end{corr}

\begin{exo}
Pour tout \(k\in\N\), calculer la dérivée \(k\)-ème des fonctions suivantes :

\begin{enumerate}
\item \(f:x\mapsto\ln x\) \\

\item \(f:x\mapsto x^2\ln x\) \\

\item \(f:x\mapsto\cos^4x\) \\

\item \(f:x\mapsto\dfrac{1}{\sqrt{x}}\) \\

\item \(f:x\mapsto\dfrac{x-1}{x+1}\)
\end{enumerate}

On simplifiera les résultats en utilisant des factorielles si c'est utile.
\end{exo}

\begin{corr}
\note{À venir}
\end{corr}

\begin{exo}
Soit \(n\in\N\).

\begin{enumerate}
\item Pour tout \(k\in\N\), calculer la dérivée \(k\)-ème de \(f:x\mapsto x^n\). \\

\item En calculant de deux façons la dérivée \(n\)-ème de \(g:x\mapsto x^{2n}\), calculer : \[S=\sum_{k=0}^n\binom{k}{n}^2.\]
\end{enumerate}
\end{exo}

\begin{corr}
\note{À venir}
\end{corr}

\section{Étude globale}

\begin{exo}
Étudier la fonction \(f:x\mapsto\Arccos\paren{\dfrac{1-x^2}{1+x^2}}\).

On précisera l'ensemble de définition de \(f\), en quels points \(f\) est continue, en quels points \(f\) est dérivable, les variations de \(f\) (avec ses limites), et le graphe de \(f\).
\end{exo}

\begin{corr}
\note{À venir}
\end{corr}

\begin{exo}
Dessiner le graphe de la fonction : \[f:x\mapsto\Arccos\sqrt{\dfrac{1+\sin x}{2}}-\Arcsin\sqrt{\dfrac{1+\cos x}{2}}.\]
\end{exo}

\begin{corr}
\note{À venir}
\end{corr}

\begin{exo}
Montrer \[\quantifs{\forall x\in\Rps}\Arcsin\paren{\dfrac{1-x}{1+x}}=\Arctan\paren{\dfrac{1-x}{2\sqrt{x}}}\] de deux façons :

\begin{enumerate}
\item En calculant \(\tan\paren{\Arcsin\paren{\dfrac{1-x}{1+x}}}\). \\

\item En dérivant.
\end{enumerate}
\end{exo}

\begin{corr}
\note{À venir}
\end{corr}

\begin{exo}
Soit \(\alpha\in\intervee{1}{\pinf}\).

\begin{enumerate}
\item Déterminer toutes les foncions \(f:\R\to\R\) telles que : \[\quantifs{\forall x,y\in\R}\abs{f\paren{x}-f\paren{y}}\leq\abs{x-y}^{\alpha}.\]

\textit{Indication :} montrer qu'une telle fonction est nécessairement dérivable. \\

\item Déterminer toutes les fonctions \(f:\Rs\to\R\) telles que : \[\quantifs{\forall x,y\in\Rs}\abs{f\paren{x}-f\paren{y}}\leq\abs{x-y}^{\alpha}.\]
\end{enumerate}
\end{exo}

\begin{corr}
\note{À venir}
\end{corr}

\begin{exo}
Soit \(f:\Rp\to\R\) une fonction dérivable telle que : \[f\paren{0}=0\qquad\text{et}\qquad\lim_{x\to\pinf}f\paren{x}=0.\]

\begin{enumerate}
\item Montrer que \(f\) est bornée. \\

\item À l'aide du théorème de Rolle, montrer : \[\quantifs{\exists c\in\Rps}f\prim\paren{c}=0.\]
\end{enumerate}
\end{exo}

\begin{corr}
\note{À venir}
\end{corr}

\begin{exo}
Soient \(a,b\in\R\) tels que \(a<b\) et \(f,g\in\F{\intervii{a}{b}}{\R}\) continues.

On suppose que \(f\) et \(g\) sont  dérivables sur \(\intervee{a}{b}\).

Montrer : \[\quantifs{\exists c\in\intervee{a}{b}}f\prim\paren{c}\paren{g\paren{b}-g\paren{a}}=g\prim\paren{c}\paren{f\paren{b}-f\paren{a}}.\]
\end{exo}

\begin{corr}
\note{À venir}
\end{corr}

\begin{exo}
On pose : \[\fonction{f}{\intervii{0}{1}}{\R}{x}{\dfrac{\e{x}}{x+2}}\]

\begin{enumerate}
\item Soit \(a\in\intervii{0}{1}\). Montrer que l'on définit une suite \(\paren{u_n}_{n\in\N}\in\intervii{0}{1}^\N\) en posant : \[u_0=a\qquad\text{et}\qquad\quantifs{\forall n\in\N}u_{n+1}=f\paren{u_n}.\]

\item Montrer que \(f\) est \(\dfrac{2\e{}}{9}\)-lipschitzienne. \\

\item Montrer que \(f\) admet un unique point fixe \(\alpha\) dans \(\intervii{0}{1}\). \\

\item Montrer que la suite \(\paren{u_n}_n\) converge vers \(\alpha\) et que l'on a : \[\quantifs{\forall n\in\N}\abs{u_n-\alpha}\leq\paren{\dfrac{2\e{}}{9}}^n.\]

\item Donner un rang \(n\in\N\) tel que \(u_n\) soit une approximation de \(\alpha\) à \(10^{-3}\) près.
\end{enumerate}
\end{exo}

\begin{corr}
\note{À venir}
\end{corr}

\section{Convexité}

\begin{exo}
Soient \(a,b,c\in\R\).

Donner une CNS pour que la fonction \[\fonction{f}{\R}{\R}{x}{ax^2+bx+c}\] soit convexe.
\end{exo}

\begin{corr}
\note{À venir}
\end{corr}

\begin{exo}
Soit \(\alpha\in\Rps\).

On pose : \[\fonction{f_\alpha}{\R}{\R}{x}{\exp\paren{\dfrac{-x^2}{2\alpha^2}}}\]

Étudier la convexité de \(f_\alpha\) en fonction de \(\alpha\).
\end{exo}

\begin{corr}
\note{À venir}
\end{corr}

\begin{exo}
Soit \(x\in\intervii{0}{1}\).

Montrer : \[\dfrac{\pi}{4}x\leq\Arctan x\leq x\leq\Arcsin x\leq\dfrac{\pi}{2}x.\]
\end{exo}

\begin{corr}
\note{À venir}
\end{corr}

\begin{exo}
Montrer : \[\quantifs{\forall x_1,\dots,x_n\in\Rp}\sqrt{x_1+\dots+x_n}\leq\sqrt{x_1}+\dots+\sqrt{x_n}\leq\sqrt{n\paren{x_1+\dots+x_n}}.\]
\end{exo}

\begin{corr}
\note{À venir}
\end{corr}

\begin{exo}
\begin{enumerate}
\item Donner un exemple de fonction dérivable de \(\Rp\) dans \(\R\) qui soit convexe, majorée et non-constante. \\

\item Soit \(f:\R\to\R\) une fonction dérivable, convexe et majorée. Montrer que \(f\) est constante. \\

\item Soit \(f:\R\to\R\) une fonction convexe et majorée. Montrer que \(f\) est constante.
\end{enumerate}
\end{exo}

\begin{corr}
\note{À venir}
\end{corr}

\begin{exo}
Soit \(f:\intervii{0}{1}\to\R\) convexe.

\begin{enumerate}
\item Montrer que \(f\) est dérivable à droite et à gauche en tout point de \(\intervee{0}{1}\). \\

\item La fonction \(f\) est-elle nécessairement dérivable en \(0\) ? en \(1\) ? en \(\dfrac{1}{2}\) ? \\

\item La fonction \(f\) est-elle nécessairement continue en \(0\) ? en \(1\) ? en \(\dfrac{1}{2}\) ?
\end{enumerate}
\end{exo}

\begin{corr}
\note{À venir}
\end{corr}

\begin{exo}
Soit \(f:\Rp\to\R\) convexe.

\begin{enumerate}
\item Montrer que \(\lim_{x\to\pinf}\dfrac{f\paren{x}}{x}\) existe. On la note \(l\). \\

\item Montrer que si \(l\leq0\) alors \(f\) est décroissante sur \(\Rp\).

\textit{Indication :} raisonner par l'absurde. \\

\item Montrer que \(f\) admet une limite en \(\pinf\). \\

\item Établir, pour chaque valeur de \(l\), quelles sont les différentes limites possibles pour \(f\) en \(\pinf\), en illustrant chaque possibilité par un exemple.
\end{enumerate}
\end{exo}

\begin{corr}
\note{À venir}
\end{corr}

\begin{exo}
\begin{enumerate}
\item Montrer que la fonction \(\fonction{f}{\R}{\R}{x}{\ln\paren{1+\e{x}}}\) est convexe. \\

\item En déduire : \[\quantifs{\forall x_1,\dots,x_n\in\Rps}1+\paren{\prod_{k=1}^nx_k}^{\nicefrac{1}{n}}\leq\paren{\prod_{k=1}^n\paren{1+x_k}}^{\nicefrac{1}{n}}\] puis : \[\quantifs{\forall a_1,\dots,a_n,b_1,\dots,b_n\in\Rps}\paren{\prod_{k=1}^na_k}^{\nicefrac{1}{n}}+\paren{\prod_{k=1}^nb_k}^{\nicefrac{1}{n}}\leq\paren{\prod_{k=1}^n\paren{a_k+b_k}}^{\nicefrac{1}{n}}.\]
\end{enumerate}
\end{exo}

\begin{corr}
\note{À venir}
\end{corr}

\begin{exo}
Soient \(n\in\Ns\) et \(x\in\intervie{1}{\pinf}\).

Montrer : \[n\paren{x^{\nicefrac{n+1}{2}}-x^{\nicefrac{n-1}{2}}}\leq x^n-1.\]

\textit{Indication :} factoriser par \(x-1\).
\end{exo}

\begin{corr}
\note{À venir}
\end{corr}

\begin{exo}
Soient \(a,b\in\R\) tels que \(a<b\) et \(f\in\ensclasse{2}{\intervii{a}{b}}{\R}\) telle que \(f\paren{a}=f\paren{b}=0\).

\begin{enumerate}
\item Justifier que \[M=\max_{\intervii{a}{b}}\abs{f\seconde}\] est bien défini. \\

\item Étudier la convexité des fonctions \[\fonction{g}{\intervii{a}{b}}{\R}{x}{f\paren{x}+M\dfrac{\paren{x-a}\paren{b-x}}{2}}\quad\text{et}\quad\fonction{h}{\intervii{a}{b}}{\R}{x}{f\paren{x}-M\dfrac{\paren{x-a}\paren{b-x}}{2}}\]

\item En déduire : \[\quantifs{\forall x\in\intervii{a}{b}}\abs{f\paren{x}}\leq M\dfrac{\paren{x-a}\paren{b-x}}{2}.\]
\end{enumerate}
\end{exo}

\begin{corr}
\note{À venir}
\end{corr}

\begin{exo}[Inégalités de Hölder et de Minkowski]
Soient \(p,q\in\intervee{1}{\pinf}\) tels que : \[\dfrac{1}{p}+\dfrac{1}{q}=1.\]

\begin{enumerate}
\item Montrer : \[\quantifs{\forall x,y\in\Rps}xy\leq\dfrac{1}{p}x^p+\dfrac{1}{q}y^q.\]

\item Soient \(x_1,\dots,x_n,y_1,\dots,y_n\in\Rps\) tels que \(\sum_{k=1}^nx_k^p=\sum_{k=1}^ny_k^q=1\).

Montrer : \[\sum_{k=1}^nx_ky_k\leq1.\]

\item En déduire l'inégalité de Hölder : \[\quantifs{\forall a_1,\dots,a_n,b_1,\dots,b_n\in\Rp}\sum_{k=1}^na_kb_k\leq\paren{\sum_{k=1}^na_k^p}^{\nicefrac{1}{p}}\paren{\sum_{k=1}^nb_k^q}^{\nicefrac{1}{q}}.\]

\item En déduire l'inégalité de Minkowski : \[\quantifs{\forall a_1,\dots,a_n,b_1,\dots,b_n\in\Rp}\paren{\sum_{k=1}^n\paren{a_k+b_k}^p}^{\nicefrac{1}{p}}\leq\paren{\sum_{k=1}^na_k^p}^{\nicefrac{1}{p}}+\paren{\sum_{k=1}^nb_k^p}^{\nicefrac{1}{p}}.\]
\end{enumerate}
\end{exo}

\begin{corr}
\note{À venir}
\end{corr}