\chapter{Espaces vectoriels en dimension finie}

\minitoc

\begin{exo}[Exercice 1]
Si \(p\in\Ns\), on note \(E_p\) l'espace vectoriel des suites \(p\)-périodiques d'éléments de \(\K\) : \[E_p=\accol{\paren{x_n}_n\in\K^\N\tq\quantifs{\forall n\in\N}u_{n+p}=u_n}.\]

On note \(E\) l'espace vectoriel des suites périodiques d'éléments de \(\K\) : \[E=\accol{\paren{x_n}_n\in\K^\N\tq\quantifs{\exists p\in\Ns;\forall n\in\N}u_{n+p}=u_n}.\]

\begin{enumerate}
\item Soit \(p\in\Ns\). Quelle est la dimension de \(E_p\) ? \\

\item Quelle est la dimension de \(E\) ?
\end{enumerate}
\end{exo}

\begin{corr}
\note{À venir}
\end{corr}

\begin{exo}[Exercice 2, TPE MP 2018]
On pose : \[E=\accol{P\in\poly[\C]\tq\paren{X^4+1}P=P\paren{X^2}}.\]

\begin{enumerate}
\item Montrer que \(E\) est un sous-espace vectoriel de \(\poly[\C]\). \\

\item Si \(P\in E\excluant\accol{0}\), déterminer le degré de \(P\). \\

\item Donner un polynôme non-nul appartenant à \(E\). \\

\item Déterminer la dimension de \(E\).
\end{enumerate}
\end{exo}

\begin{corr}
\note{À venir}
\end{corr}

\begin{exo}[Exercice 3, relations de récurrence d'ordre 2]\thlabel{exo:relationsDeRécurrenceD'Ordre2}
Soient \(a,b,c\in\C\) tels que \(a\not=0\).

On note \(E\) le \(\C\)-espace vectoriel des suites \(\paren{u_n}_{n\in\N}\in\C^\N\) à coefficients dans \(\C\) vérifiant la relation de récurrence linéaire homogène d'ordre 2 à coefficients constants suivant : \[\quantifs{\forall n\in\N}au_{n+2}+bu_{n+1}+cu_n=0.\]

\begin{enumerate}
\item Montrer que l'application \[\fonction{\phi}{E}{\C^2}{\paren{u_n}_n}{\paren{u_0,u_1}}\] est un isomorphisme d'espaces vectoriels. \\

\item Quelle est la dimension de \(E\) ? \\

\item En déduire le théorème suivant (à retenir) :

\begin{itemize}
\item Si le polynôme \(aX^2+bX+c\) admet deux racines distinctes \(x,y\in\C\), alors \(E\) est l'ensemble des suites de la forme : \[\paren{\lambda x^n+\mu y^n}_{n\in\N}\] où \(\lambda,\mu\in\C\). \\

\item Si le polynôme \(aX^2+bX+c\) admet une racine double \(x\in\C\), alors \(E\) est l'ensemble des suites de la forme : \[\paren{\paren{\lambda n+\mu}x^n}_{n\in\N}\] où \(\lambda,\mu\in\C\).
\end{itemize}
\end{enumerate}
\end{exo}

\begin{corr}
\note{À venir}
\end{corr}

\begin{exo}[Exercice 4]
Donner un exemple d'espace vectoriel \(E\) et d'endomorphisme \(u\in\Lendo{E}\) tels que \(u\) soit inversible à droite mais pas à gauche (dans l'anneau \(\anneau{\Lendo{E}}[+][\rond]\)).
\end{exo}

\begin{corr}
\note{À venir}
\end{corr}

\begin{exo}[Exercice 5]
Soient \(E\) et \(F\) deux \(\K\)-espaces vectoriels de dimension finie.

\begin{enumerate}
\item Montrer qu'il existe une application linéaire injective \(u:E\to F\) si, et seulement si, \(\dim E\leq\dim F\). \\

\item Montrer qu'il existe une application linéaire surjective \(u:E\to F\) si, et seulement si, \(\dim E\geq\dim F\). \\

\item Montrer qu'il existe un isomorphisme (d'espaces vectoriels) \(u:E\to F\) si, et seulement si, \(\dim E=\dim F\).
\end{enumerate}
\end{exo}

\begin{corr}
\note{À venir}
\end{corr}

\begin{exo}[Exercice 6, endomorphismes de rang 1]
Soient \(E\) un \(\K\)-espace vectoriel et \(u\in\Lendo{E}\).

\begin{enumerate}[series=endoRG1]
\item Justifier que l'endomorphisme \(u\) est de rang 1 si, et seulement si, il s'écrit : \[\fonction{u}{E}{E}{x}{l\paren{x}a}\] où \(a\) est un vecteur non-nul de \(E\) et \(l\in E\etoile\) est une forme linéaire non-nulle. \\
\end{enumerate}

On suppose désormais que \(u\) est de rang 1.

\begin{enumerate}[resume=endoRG1]
\item Montrer : \(\quantifs{\exists\lambda\in\K}u^2=\lambda u\). \\

\item Montrer que les trois propositions suivantes sont équivalentes :

\begin{enumerate}
\item \(\quantifs{\exists\lambda\in\K\excluant\accol{0}}\lambda u\text{ est un projecteur}\) \\

\item \(E=\ker u\oplus\Im u\) \\

\item \(u^2\not=0\).
\end{enumerate}
\end{enumerate}
\end{exo}

\begin{corr}
\note{À venir}
\end{corr}

\begin{exo}[Exercice 7]
Soient \(E\) un \(\K\)-espace vectoriel et \(u\in\Lendo{E}\).

On suppose : \[\dim E=3\qquad\text{et}\qquad u^2=0.\]

Montrer : \(\rg u\leq1\).
\end{exo}

\begin{corr}
\note{À venir}
\end{corr}

\begin{exo}[Exercice 8]
Soient \(E\) un espace vectoriel de dimension finie et \(u,v\in\Lendo{E}\).

Montrer : \[\abs{\rg u-\rg v}\leq\rg\paren{u+v}\leq\rg u+\rg v.\]
\end{exo}

\begin{corr}
\note{À venir}
\end{corr}

\begin{exo}[Exercice 9]
Soient \(E\) un espace vectoriel de dimension finie et \(u,v\in\Lendo{E}\).

Montrer : \[\rg\paren{uv}\leq\min\accol{\rg u;\rg v}.\]
\end{exo}

\begin{corr}
\note{À venir}
\end{corr}

\begin{exo}[Exercice 10]
Soient \(u\in\L{\R^2}{\R^3}\) et \(v\in\L{\R^3}{\R^2}\).

On suppose que \(u\rond v\) est un projecteur de rang 2 de \(\R^3\).

Montrer que \(v\) est une surjection et que \(u\) est une injection puis que \(v\rond u=\id{\R^2}\).
\end{exo}

\begin{corr}
\note{À venir}
\end{corr}

\begin{exo}[Exercice 11]
Soient \(E\) un \(\K\)-espace vectoriel de dimension finie et \(u,v\in\Lendo{E}\).

Montrer l'équivalence : \[\ker u=\Im u\ssi\begin{dcases}
u^2=0 \\
\dim E=2\rg u
\end{dcases}\]
\end{exo}

\begin{corr}
\note{À venir}
\end{corr}

\begin{exo}[Exercice 12, formule de Grassmann]
Soient \(E\) un \(\K\)-espace vectoriel et \(F\) et \(G\) deux sous-espaces vectoriels de \(E\) de dimension finie.

\begin{enumerate}
\item Montrer qu'il existe \(a,b,c\in\N\) et une base de \(F+G\) : \[\paren{e_1,\dots,e_a,e_1\prim,\dots,e_b\prim,e_1\seconde,\dots,e_c\seconde}\] tels que \[\begin{dcases}
F\inter G=\Vect{e_1,\dots,e_a} \\
F=\Vect{x_1,\dots,e_a,e_1\prim,\dots,e_b\prim} \\
G=\Vect{e_1,\dots,e_a,e_1\seconde,\dots,e_b\seconde}
\end{dcases}\]

\item En déduire une nouvelle démonstration de la formule de Grassmann.
\end{enumerate}
\end{exo}

\begin{corr}
\note{À venir}
\end{corr}

\begin{exo}[Exercice 13, Centrale 2008 (extrait)]
Soient \(E\) et \(F\) deux espaces vectoriels de dimension finie, \(f\in\L{E}{F}\) et \(G\) un sous-espace vectoriel de \(F\).

Montrer : \[\dim f\inv\paren{G}=\dim E-\rg f+\dim\paren{\Im f\inter G}.\]
\end{exo}

\begin{corr}
\note{À venir}
\end{corr}

\begin{exo}[Exercice 14]
Soient \(E\) un espace vectoriel de dimension finie et \(u,v\in\Lendo{E}\).

Montrer : \[\rg\paren{u+v}=\rg u+\rg v\ssi\begin{dcases}
\Im u\inter\Im v=\accol{0_E} \\
\ker u+\ker v=E
\end{dcases}\]
\end{exo}

\begin{corr}
\note{À venir}
\end{corr}

\begin{exo}[Exercice 15]
Soient \(E\) un espace vectoriel de dimension finie et \(l_1,\dots,l_m\in E\etoile\).

On considère l'application : \[\fonction{\phi}{E}{\K^m}{x}{\paren{l_1\paren{x},\dots,l_m\paren{x}}}\]

\begin{enumerate}
\item Montrer que la famille \(\paren{l_1,\dots,l_m}\) est une famille libre si, et seulement si, l'application \(\phi\) est surjective. \\

\item Montrer : \[\rg\phi=\rg\paren{l_1,\dots,l_m}.\]

\textit{Indication :} considérer une base \(\paren{l_{i_1},\dots,l_{i_r}}\) de \(\Vect{l_1,\dots,l_m}\). \\

\item En déduire que la famille \(\paren{l_1,\dots,l_m}\) engendre \(E\etoile\) si, et seulement si, l'application \(\phi\) est injective.
\end{enumerate}
\end{exo}

\begin{corr}
\note{À venir}
\end{corr}

\begin{exo}[Exercice 16]
Soient \(E\) un espace vectoriel de dimension finie et \(u\in\Lendo{E}\).

Montrer que les propositions suivantes sont équivalentes :

\begin{enumerate}
\item \(\Im u^2=\Im u\) \\

\item \(\ker u^2=\ker u\) \\

\item \(E=\ker u\oplus\Im u\).
\end{enumerate}
\end{exo}

\begin{corr}
\note{À venir}
\end{corr}

\begin{exo}[Exercice 17]
Soient \(E\) un espace vectoriel de dimension finie et \(u,v\in\Lendo{E}\).

On suppose : \[u\rond v=0\qquad\text{et}\qquad u+v\in\GL{}[E].\]

Montrer : \[\rg u+\rg v=\dim E\qquad\text{et}\qquad\ker u\oplus\ker v=E.\]
\end{exo}

\begin{corr}
\note{À venir}
\end{corr}