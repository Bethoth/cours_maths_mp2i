\chapter{Matrices I}

\minitoc

\begin{exo}[Exercice 1]
Soient \(a,b,c\in\K\excluant\accol{0}\).

On pose \[A=\begin{pmatrix}
a & 0 & 0 \\
0 & b & 0 \\
0 & 0 & c
\end{pmatrix}\qquad\text{et}\qquad B=\begin{pmatrix}
1 & a \\
0 & 1
\end{pmatrix}.\]

Calculer \(A^n\) et \(B^n\) pour tout \(n\in\Z\).
\end{exo}

\begin{corr}
\note{À venir}
\end{corr}

\begin{exo}[Exercice 2]
Soit \(n\in\Ns\).

L'ensemble des matrices symétriques de \(\M{n}\) est-il stable par le produit matriciel ? Qu'en est-il de l'ensemble des matrices antisymétriques ?
\end{exo}

\begin{corr}
\note{À venir}
\end{corr}

\begin{exo}[Exercice 3]
Dire pour chaque système quel est son rang et ce qu'on le peut en déduire sur l'espace \(\fami{S}\) de ses solutions. Le résoudre ensuite.

\begin{enumerate}
\item Système d'inconnue \(\paren{x,y,z}\in\R^3\) : \[\begin{dcases}
x+2z=1 \\
x+y+z=2
\end{dcases}\]

\item Système d'inconnue \(\paren{a,b,c,d}\in\R^4\), où \(\lambda\) est un paramètre réel fixé : \[\begin{dcases}
a+b+c+d=0 \\
a-b+2c-d=1 \\
3a+b+4c+d=\lambda
\end{dcases}\]
\end{enumerate}
\end{exo}

\begin{corr}
\note{À venir}
\end{corr}

\begin{exo}[Exercice 4]
Soient \(n,p\in\Ns\) et \(A\in\M{np}[\R]\).

On considère les équations matricielles d'inconnue \(X\in\M{p1}[\R]\) : \[\paren{E_1}~AX=\begin{pmatrix}1\\0\\0\end{pmatrix}\qquad\text{et}\qquad\paren{E_2}~AX=\begin{pmatrix}0\\1\\0\end{pmatrix}.\]

On suppose que \(\paren{E_1}\) n'admet aucune solution et que \(\paren{E_2}\) admet une unique solution.

Quelles sont les valeurs possibles pour les entiers \(n\) et \(p\) ?

Donner, pour chaque possibilité, un exemple de matrice \(A\).
\end{exo}

\begin{corr}
\note{À venir}
\end{corr}

\begin{exo}[Exercice 5]
Soit \(\paren{S}\) un système linéaire de \(n\) équations à \(p\) inconnues. On note \(r\) son rang.

Justifier les propositions suivantes :

\begin{enumerate}
\item Si \(r=p\) alors \(\paren{S}\) admet au plus une solution. \\

\item Si \(r=n\) alors \(\paren{S}\) admet au moins une solution.
\end{enumerate}
\end{exo}

\begin{corr}
\note{À venir}
\end{corr}

\begin{exo}[Exercice 6]
Montrer que les matrices \(\begin{pmatrix}
1 & 1 \\
0 & 1
\end{pmatrix}\) et \(\begin{pmatrix}
1 & 0 \\
1 & 1
\end{pmatrix}\) sont semblables.
\end{exo}

\begin{corr}
\note{À venir}
\end{corr}

\begin{exo}[Exercice 7]
Calculer les inverses des matrices suivantes : \[A=\begin{pmatrix}
2 & 5 \\
3 & 7
\end{pmatrix}\qquad B=\begin{pmatrix}
1 & 2 \\
2 & 4
\end{pmatrix}\qquad C=\begin{pmatrix}
0 & 7 & 8 \\
0 & 1 & 1 \\
1 & 2 & 3
\end{pmatrix}\qquad D=\begin{pmatrix}
1 & 0 & 2 \\
1 & 3 & 1 \\
1 & 7 & 0
\end{pmatrix}\qquad E=\begin{pmatrix}
1 & 0 & 2 & 0 \\
3 & 1 & 1 & 0 \\
0 & 1 & 2 & 1 \\
2 & 2 & 2 & 1
\end{pmatrix}\]
\end{exo}

\begin{corr}
\note{À venir}
\end{corr}

\begin{exo}[Exercice 8]\thlabel{exo:matrice1...n}
Soit \(n\in\Ns\).

On pose : \[A=\begin{pmatrix}
1 & \dots & \dots & 1 \\
0 & \ddots &  & \vdots \\
\vdots & \ddots & \ddots & \vdots \\
0 & \dots & 0 & 1
\end{pmatrix}\in\M{n}\qquad\text{et}\qquad B=\begin{pmatrix}
1 & \dots & \dots & \dots & 1 \\
\vdots & 2 & \dots & \dots & 2 \\
\vdots & \vdots & 3 & \dots & 3 \\
\vdots & \vdots & \vdots & \ddots & \vdots \\
1 & 2 & 3 & \dots & n
\end{pmatrix}\]

\begin{enumerate}
\item Déterminer l'inverse de \(A\). \\

\item Calculer \(\trans{A}A\). \\

\item En déduire l'inverse de \(B\).
\end{enumerate}
\end{exo}

\begin{corr}
\note{À venir}
\end{corr}

\begin{exo}[Exercice 9]
Soient \(a,b,c,d,e,f,g,h,i\in\K\) et \(\lambda\in\K\excluant\accol{0}\).

Montrer que les matrices \(\begin{pmatrix}
a & b & c \\
d & e & f \\
g & h & i
\end{pmatrix}\) et \(\begin{pmatrix}
a & b\lambda\inv & c\lambda^{-2} \\
d\lambda & e & f\lambda\inv \\
g\lambda^2 & h\lambda & i
\end{pmatrix}\) sont semblables.
\end{exo}

\begin{corr}
\note{À venir}
\end{corr}

\begin{exo}[Exercice 10]
Montrer que l'application \[\fonction{\phi}{\M{n}[\R]}{\M{n}[\R]\etoile}{A}{B\mapsto\tr\paren{AB}}\] est un isomorphisme de l'espace vectoriel \(\M{n}[\R]\) vers son dual.
\end{exo}

\begin{corr}
\note{À venir}
\end{corr}

\begin{exo}[Exercice 11]\thlabel{exo:td15exo11}
On suppose que le corps \(\K\) est fini et on note \(q\) son cardinal : \[q=\Card\K<\pinf.\]

Soit \(n\in\Ns\).

Déterminer le cardinal des ensembles suivants :

\begin{enumerate}
\item \(\K^n\) \\

\item \(\M{n}\) \\

\item \(\GL{n}\)
\end{enumerate}

\textit{Indications pour le (3) :}

\begin{itemize}
\item Commencer par \(n=2\) ou \(3\). \\

\item Une matrice carrée est inversible si, et seulement si, la famille de ses vecteurs colonnes est libre. Compter les matrices inversibles de taille \(n\) revient donc à compter les familles libres de \(n\) vecteurs de \(\K^n\).
\end{itemize}
\end{exo}

\begin{corr}
\note{À venir}
\end{corr}

\begin{exo}[Exercice 12]
Soient \(n\in\Ns\) et \(f:\M{n}\to\K\) une fonction non-constante telle que : \[\quantifs{\forall A,B\in\M{n}}f\paren{AB}=f\paren{A}f\paren{B}.\]

Montrer que pour toute matrice \(A\in\M{n}\), on a l'équivalence : \[A\in\GL{n}\ssi f\paren{A}\not=0.\]
\end{exo}

\begin{corr}
\note{À venir}
\end{corr}