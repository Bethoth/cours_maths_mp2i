\chapter{Algèbre générale}

\minitoc

\section{Lois de composition internes}

\begin{exo}[Exercice 1]
On considère la loi de composition interne \(*\) sur \(\R\) définie par : \[\quantifs{\forall x,y\in\R}x*y=\ln\paren{\e{x}+\e{y}}.\]

Est-elle associative ? commutative ? Possède-t-elle un élément neutre ?
\end{exo}

\begin{corr}
\note{à venir}
\end{corr}

\begin{exo}[Exercice 2]
Soit \(E\) un ensemble muni d'une loi \(*\).

On suppose que la loi \(*\) est associative et qu'elle possède un élément neutre \(e\).

Un élément \(x\in E\) est dit idempotent s'il vérifie \(x*x=x\).

\begin{enumerate}
\item Soient \(x,y\in E\). On suppose que \(x\) et \(y\) sont idempotents et qu'ils commutent. Montrer que \(x*y\) est idempotent. \\

\item Soit \(x\in E\) un élément inversible et idempotent. Montrer que son inverse \(x\inv\) est idempotent.
\end{enumerate}
\end{exo}

\begin{corr}
\note{à venir}
\end{corr}

\begin{exo}[Exercice 3, plus difficile]
Soient \(E\) un ensemble fini et \(*\) une loi de composition interne associative sur \(E\).

Soit \(x\in E\). On suppose que \(x\) est régulier pour la loi \(*\).

\begin{enumerate}
\item Montrer que toute puissance de \(x\) est régulière : \[\quantifs{\forall n\in\Ns}x^n\text{ est régulier}.\] \\

\item Montrer que la loi \(*\) admet un élément neutre. \\

\item Montrer que \(x\) est inversible.
\end{enumerate}
\end{exo}

\begin{corr}
\note{à venir}
\end{corr}

\section{Groupes}

\begin{exo}[Exercice 4, exemple important]
Soit \(E\) un ensemble.

On appelle permutation de \(E\) toute bijection \(f:E\to E\).

On note \(S_E\) l'ensemble des permutations de \(E\).

Montrer que \(\groupe{S_E}[\rond]\) est un groupe. Est-il commutatif ?
\end{exo}

\begin{corr}
\note{à venir}
\end{corr}

\begin{exo}[Exercice 5]
Montrer que l'ensemble des applications de la forme \[\fonction{f_{ab}}{\C}{\C}{z}{az+b}\] avec \(\paren{a,b}\in\Cs\times\C\) est un groupe pour la composition.
\end{exo}

\begin{corr}
\note{à venir}
\end{corr}

\begin{exo}[Exercice 6]
On pose \(E=\accol{0;1;2}\).

\begin{enumerate}
\item Combien y a-t-il de lois de composition internes sur \(E\) ? \\

\item Combien y a-t-il de lois de composition internes commutatives sur \(E\) ? \\

\item Combien y a-t-il de lois de composition internes admettant \(0\) comme élément neutre sur \(E\) ? \\

\item Combien y a-t-il de lois de composition internes \guillemets{unitaires} (\cad admettant un élément neutre) sur \(E\) ? \\

\item Combien y a-t-il de structures de groupe pour lesquelles \(0\) est l'élément neutre sur \(E\) ? \\

\item Combien y a-t-il de structures de groupe sur \(E\) ? \\

\item Ces dernières sont-elles commutatives ? \\

\item On munit \(E\) de sa structure de groupe telle que \(0\) soit l'élément neutre. Quels sont les sous-groupes de \(E\) ? Déterminer \(\Aut{E}\).
\end{enumerate}
\end{exo}

\begin{corr}
\note{à venir}
\end{corr}

\begin{exo}[Exercice 7]
On note \(\U\) l'ensemble des nombres complexes de module \(1\).

Rappeler les structures naturelles de groupe de \(\Cs\) et \(\U\) et montrer que la fonction \[\fonction{f}{\Cs}{\U}{z}{\dfrac{z}{\abs{z}}}\] est un morphisme de groupes.
\end{exo}

\begin{corr}
\note{à venir}
\end{corr}

\begin{exo}[Exercice 8, intersection de sous-groupes]\thlabel{exo:6.15}
Soient \(G\) un groupe et \(\paren{H_i}_{i\in I}\) une famille de sous-groupes de \(G\).

Montrer que l'intersection \(\biginter_{i\in I}H_i\) est un sous-groupe de \(G\).
\end{exo}

\begin{corr}
\note{à venir}
\end{corr}

\begin{exo}[Exercice 9]
Soit \(\groupe{G}[*]\) un groupe d'élément neutre \(e\) tel que \[\quantifs{\forall g\in G}g*g=e.\]

Montrer que \(G\) est abélien.
\end{exo}

\begin{corr}
\note{à venir}
\end{corr}

\begin{exo}[Exercice 10]
Soit \(\groupe{G}[*]\) un groupe de cardinal pair et d'élément neutre \(e\).

\begin{enumerate}
\item Vérifier que la relation \(\rel\) sur \(G\) définie par \[\quantifs{\forall g,h\in G}g\rel h\ssi\orenv{g=h \\ g=h\inv}\] est une relation d'équivalence. \\

\item En déduire que \(G\) possède un élément \(g\) tel que \(g\not=e\) et \(g*g=e\).
\end{enumerate}
\end{exo}

\begin{corr}
\note{à venir}
\end{corr}

\begin{exo}[Exercice 11]
Soit \(G\) un groupe et \(g\) un élément de \(G\).

\begin{enumerate}
\item Montrer qu'il existe un unique morphisme de groupes \(\phi:\Z\to G\) tel que \[\phi\paren{1}=g.\] \\

Que vient-on de montrer à propos de l'application \[\fonction{f}{\Hom{\Z}{G}}{G}{\phi}{\phi\paren{1}}\text{ ?}\] \\

\item Montrer par des exemples que l'image d'un morphisme de groupes \(\phi:\Z\to G\) peut être finie ou infinie.
\end{enumerate}
\end{exo}

\begin{corr}
\note{à venir}
\end{corr}

\section{Anneaux}

\begin{exo}[Exercice 12]
Quels sont les sous-anneaux de \(\Z\) ?
\end{exo}

\begin{corr}
\note{à venir}
\end{corr}

\begin{exo}[Exercice 13]
Montrer que toute structure d'anneau sur un ensemble à trois éléments est une structure d'anneau commutatif.
\end{exo}

\begin{corr}
\note{à venir}
\end{corr}

\begin{exo}[Exercice 14]
Montrer que l'ensemble des suites réelles convergentes (indicées par \(\N\)) est un sous-anneau de \(\R^\N\).

Que dire des suites complexes convergentes ?
\end{exo}

\begin{corr}
\note{à venir}
\end{corr}

\begin{exo}[Exercice 15]
\begin{enumerate}
\item Les anneaux \(\anneau{\Z}\) et \(\anneau{\Q}\) sont-ils isomorphes ? \\

\item Les anneaux \(\anneau{\R}\) et \(\anneau{\Q}\) sont-ils isomorphes ? \\

\item Les anneaux \(\anneau{\R}\) et \(\anneau{\C}\) sont-ils isomorphes ?
\end{enumerate}
\end{exo}

\begin{corr}
\note{à venir}
\end{corr}

\begin{exo}[Exercice 16]
On pose : \[\Z\croch{\sqrt{2}}=\accol{x+\sqrt{2}y}_{\paren{x,y}\in\Z^2}=\accol{a\in\R\tq\quantifs{\exists x,y\in\Z}a=x+\sqrt{2}y}.\]

\begin{enumerate}
\item Montrer que \(\Z\croch{\sqrt{2}}\) est un anneau commutatif. \\

\item On pose : \[\quantifs{\forall x,y\in\Z}N\paren{x+\sqrt{2}y}=x^2-2y^2.\] \\

Montrer que cela définit une application \(N:\Z\croch{\sqrt{2}}\to\Z\). \\

\item Montrer que si \(a\) et \(b\) sont des éléments de \(\Z\croch{\sqrt{2}}\) alors \[N\paren{ab}=N\paren{a}N\paren{b}.\] \\

\item Montrer que \[\quantifs{\forall a\in\Z\croch{\sqrt{2}}}a=0\ssi N\paren{a}=0.\] \\

\item Montrer que \[\quantifs{\forall a\in\Z\croch{\sqrt{2}}}a\in\Z\croch{\sqrt{2}}\croix\ssi N\paren{a}\in\Z\croix.\] \\

\item On pose \[\begin{dcases}u_0=1 \\ v_0=0\end{dcases}\qquad\text{et}\qquad\begin{dcases}u_{n+1}=u_n+2v_n \\ v_{n+1}=u_n+v_n\end{dcases}\] \\

Montrer que pour tout entier naturel \(n\), l'élément \(u_n+\sqrt{2}v_n\) est inversible dans \(\Z\croch{\sqrt{2}}\).
\end{enumerate}
\end{exo}

\begin{corr}
\note{à venir}
\end{corr}

\begin{exo}[Exercice 17]
Soient \(\anneau{A}\) un anneau intègre et \(a,b\in A\).

On dit que \(a\) divise \(b\) et on note \(a\divise b\) si on a : \[\quantifs{\exists c\in A}ac=b.\]

Montrer : \[\paren{\quantifs{\exists\lambda\in A\croix}a=\lambda b}\ssi\begin{dcases}a\divise b \\ b\divise a\end{dcases}\]
\end{exo}

\begin{corr}
\note{à venir}
\end{corr}

\begin{exo}[Exercice 18]
On note \(\D\) l'ensemble des nombres décimaux : \[\D=\accol{x\in\R\tq\quantifs{\exists\alpha\in\N;\exists\lambda\in\Z}x=\dfrac{\lambda}{10^\alpha}}.\]

\begin{enumerate}
\item Montrer que \(\D\) est naturellement muni d'une structure d'anneau. \\

\item Quels sont ses éléments inversibles ?
\end{enumerate}
\end{exo}

\begin{corr}
\note{à venir}
\end{corr}

\begin{exo}[Exercice 19]
Soit \(\anneau{A}\) un anneau. On note \(0\) et \(1\) ses éléments neutres.

Un élément \(a\in A\) est dit nilpotent s'il vérifie : \[\quantifs{\exists n\in\Ns}a^n=0.\]

\begin{enumerate}
\item Si \(A\) est intègre, quels sont ses éléments nilpotents ? \\

\item Soient \(a,b\in A\). On suppose que \(a\) et \(b\) sont nilpotents et qu'ils commutent. \\

Montrer que \(a+b\) et \(ab\) sont nilpotents. \\

\item Soit \(a\in A\) un élément nilpotent. \\

Montrer que \(1+a\) est inversible et déterminer son inverse.
\end{enumerate}
\end{exo}

\begin{corr}
\note{à venir}
\end{corr}

\section{Corps}

\begin{exo}[Exercice 20]
Quels sont les sous-corps de \(\Q\) ?
\end{exo}

\begin{corr}
\note{à venir}
\end{corr}

\begin{exo}[Exercice 21]
On pose : \[\Q\croch{\sqrt{2}}=\accol{x+\sqrt{2}y}_{\paren{x,y}\in\Q^2}=\accol{a\in\R\tq\quantifs{\exists x,y\in\Q}a=x+\sqrt{2}y}.\]

Montrer que \(\Q\croch{\sqrt{2}}\) est un corps. Quels sont ses sous-corps ?
\end{exo}

\begin{corr}
\note{à venir}
\end{corr}

\begin{exo}[Exercice 22]
Soient \(K\) un corps et \(A\) un anneau non-nul.

Montrer que tout morphisme d'anneaux \(\phi:F\to A\) est injectif.
\end{exo}

\begin{corr}
\note{à venir}
\end{corr}

\begin{exo}[Exercice 23]
Montrer que tout anneau intègre fini est un corps.
\end{exo}

\begin{corr}
\note{À venir}
\end{corr}