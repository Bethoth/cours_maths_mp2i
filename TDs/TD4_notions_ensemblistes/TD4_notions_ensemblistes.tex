\chapter{Notions ensemblistes}

\minitoc

\section{Ensembles}

\begin{exo}[Exercice 1]
Soient \(A\), \(B\) et \(C\) des ensembles.

Montrer les propositions suivantes :

\begin{enumerate}
\item \(\paren{A\union B}\inter C=\paren{A\inter C}\union\paren{B\inter C}\) \\

\item \(\paren{A\inter B}\union C=\paren{A\union C}\inter\paren{B\union C}\) \\

\item \(C\excluant\paren{A\union B}=\paren{C\excluant A}\inter\paren{C\excluant B}\) \\

\item \(\paren{A\union B}\excluant\paren{A\inter B}=\paren{A\excluant B}\union\paren{B\excluant A}\)
\end{enumerate}
\end{exo}

\begin{corr}
\note{à venir}
\end{corr}

\begin{exo}[Exercice 2]
Soient \(E\) et \(F\) deux ensembles.

\begin{enumerate}
\item Montrer l'équivalence \[E\subset F\ssi\P{E}\subset\P{F}\] \\

\item La proposition suivante est-elle vraie ? \[\P{E\union F}=\P{E}\union\P{F}\] \\

\item La proposition suivante est-elle vraie ? \[\P{E\inter F}=\P{E}\inter\P{F}\]
\end{enumerate}
\end{exo}

\begin{corr}
\note{à venir}
\end{corr}

\begin{exo}[Exercice 3]
On pose \(\classe{}=\accol{\paren{x,y}\in\R^2\tq x^2+y^2=1}\).

Existe-t-il deux parties \(A,B\subset\R\) telles que \(\classe{}=A\times B\) ?
\end{exo}

\begin{corr}
\note{à venir}
\end{corr}

\begin{exo}[Exercice 4, pas d'\guillemets{ensemble de tous les ensembles}]
Montrer qu'il n'existe pas d'ensemble dont les éléments sont les ensembles.

Pour cela, on pourra supposer par l'absurde qu'un tel ensemble \(E\) existe, considérer l'ensemble \[A=\accol{X\in E\tq X\not\in X}\] et aboutir à une contradiction.
\end{exo}

\begin{corr}
\note{à venir}
\end{corr}

\section{Fonctions}

\begin{exo}[Exercice 5]
Soient \(E\) et \(F\) deux ensembles et deux fonctions \(f:E\to F\) et \(g:F\to E\).

On suppose que la fonction \(f\rond g\rond f\) est une bijection de \(E\) dans \(F\).

Montrer que \(f\) et \(g\) sont des bijections.
\end{exo}

\begin{corr}
\note{à venir}
\end{corr}

\begin{exo}[Exercice 6, images directes, images réciproques]
Soient \(E\) et \(F\) deux ensembles, la fonction \(f:E\to F\), la partie \(A\subset E\) et la partie \(B\subset F\).

\begin{enumerate}
\item Montrer l'équivalence \[A\subset f\inv\paren{B}\ssi f\paren{A}\subset B.\] \\

\item Quelle inclusion est toujours vraie entre \(A\) et \(f\inv\paren{f\paren{A}}\) ? Donner un contre-exemple pour l'autre inclusion. \\

\item Même chose entre \(B\) et \(f\paren{f\inv\paren{B}}\).
\end{enumerate}
\end{exo}

\begin{corr}
\note{à venir}
\end{corr}

\begin{exo}[Exercice 7]
Soient \(a\) et \(b\) des réels.

Donner une CNS sur \(a\) et \(b\) pour que la fonction \[\fonction{f}{\R}{\R}{x}{ax+b}\] soit une bijection.

Déterminer alors sa bijection réciproque.
\end{exo}

\begin{corr}
\note{à venir}
\end{corr}

\begin{exo}[Exercice 8]
Montrer que la fonction \guillemets{sinus hyperbolique} : \[\fonction{\sh}{\R}{\R}{x}{\dfrac{\e{x}-\e{-x}}{2}}\] est bijective et déterminer sa bijection réciproque.
\end{exo}

\begin{corr}
\note{à venir}
\end{corr}

\begin{exo}[Exercice 9]
Soit \(E\) un ensemble non-vide et \(A\subset E\) une partie de \(E\).

Montrer que la fonction \[\fonction{f}{\P{E}}{\P{E}^2}{X}{\paren{X\inter A,X\union A}}\] est une injection.

Est-ce une surjection ?
\end{exo}

\begin{corr}
\note{à venir}
\end{corr}

\begin{exo}[Exercice 10]
Soient \(E\) et \(F\) deux ensembles et deux fonctions \(f:E\to F\) et \(g:F\to E\).

On suppose \[g\rond f=\id{E}\qquad\text{et}\qquad f\rond g=\id{F}.\]

Montrer que \(f\) et \(g\) sont des bijections, réciproques l'une de l'autre.
\end{exo}

\begin{corr}
\note{à venir}
\end{corr}

\begin{exo}[Exercice 11, fonctions indicatrices]
Soient \(E\) un ensemble et \(A,B,C\in\P{E}\).

\begin{enumerate}
\item Montrer que la fonction \[\fonction{\Phi}{\P{E}}{\accol{0;1}^E}{A}{\ind{A}}\] est une bijection et déterminer sa bijection réciproque. \\

\item Déterminer, en fonction de \(\ind{A}\) et \(\ind{B}\), les fonctions indicatrices des parties suivantes : \[A\inter B\qquad A\union B\qquad E\excluant A.\] \\

\item Utiliser ce qui précède pour montrer l'égalité : \[A\inter\paren{B\union C}=\paren{A\inter B}\union\paren{A\inter C}.\]
\end{enumerate}
\end{exo}

\begin{corr}
\note{à venir}
\end{corr}

\begin{exo}[Exercice 12]
Soient \(E\) et \(F\) deux ensembles non-vides.

Montrer que les deux propositions suivantes sont équivalentes :

\begin{enumerate}
\item Il existe une injection \(f:E\to F\). \\

\item Il existe une surjection \(g:F\to E\).
\end{enumerate}
\end{exo}

\begin{corr}
\note{à venir}
\end{corr}

\begin{exo}[Exercice 13]
Soit \(E\) un ensemble.

Montrer qu'il n'existe pas de surjection \(f:E\to\P{E}\).
\end{exo}

\begin{corr}
\note{à venir}
\end{corr}

\begin{exo}[Exercice 14]
Soient \(E\), \(E\prim\), \(F\) et \(F\prim\) des ensembles et \(f:E\to F\).

On suppose que l'ensemble \(E\prim\) est non-vide et que l'ensemble \(F\prim\) possède au moins deux éléments distincts.

On définit les fonctions : \[\fonction{F_1}{\F{E\prim}{E}}{\F{E\prim}{F}}{\phi}{f\rond\phi}\qquad\text{et}\qquad\fonction{F_2}{\F{F}{F\prim}}{\F{E}{F\prim}}{\phi}{\phi\rond f}\]

\begin{enumerate}
\item Donner une CNS sur \(f\) pour que \(F_1\) soit injective. \\

\item Donner une CNS sur \(f\) pour que \(F_1\) soit surjective. \\

\item Donner une CNS sur \(f\) pour que \(F_2\) soit injective. \\

\item Donner une CNS sur \(f\) pour que \(F_2\) soit surjective.
\end{enumerate}
\end{exo}

\begin{corr}
\note{à venir}
\end{corr}

\section{Ensembles ordonnés}

\begin{exo}[Exercice 15, ordre produit sur \(E\times F\)]
Soient \(\groupe{E}[\leq_E]\) et \(\groupe{F}[\leq_F]\) deux ensembles ordonnés.

On définit une relation binaire sur \(E\times F\) en posant : \[\quantifs{\forall\paren{x_1,y_1},\paren{x_2,y_2}\in E\times F}\paren{x_1,y_1}\leq\paren{x_2,y_2}\ssi\begin{dcases}x_1\leq_Ex_2 \\ y_1\leq_Fy_2\end{dcases}\]

Montrer que \(\leq\) est une relation d'ordre sur \(E\times F\).
\end{exo}

\begin{corr}
\note{à venir}
\end{corr}

\begin{exo}[Exercice 16, ordre lexicographique sur \(\R^2\)]
On définit une relation binaire \(\sqsubseteq\) sur \(\R^2\) en posant : \[\quantifs{\forall\paren{x_1,y_1},\paren{x_2,y_2}\in\R^2}\paren{x_1,y_1}\sqsubseteq\paren{x_2,y_2}\ssi\orenv{x_1<x_2 \\ x_1=x_2\quad\text{et}\quad y_1\leq y_2}\]

Montrer que \(\sqsubseteq\) est une relation d'ordre sur \(\R^2\) et que cet ordre est total.
\end{exo}

\begin{corr}
\note{à venir}
\end{corr}

\begin{exo}[Exercice 17]
Soient \(A,B\in\P{\R}\) deux parties de \(\R\) admettant chacune une borne supérieure.

\begin{enumerate}
\item Montrer que \(A\union B\) admet une borne supérieure, et déterminer cette borne supérieure en fonction de \(\sup A\) et \(\sup B\). \\

\item Montrer que la partie \[A+B=\accol{a+b}_{\paren{a,b}\in A\times B}\] admet une borne supérieure, et déterminer cette borne supérieure en fonction de \(\sup A\) et \(\sup B\).
\end{enumerate}
\end{exo}

\begin{corr}
\note{à venir}
\end{corr}

\begin{exo}[Exercice 18]
Soit \(X\) un ensemble.

On considère l'ensemble \(E=\P{X}\), ordonné par la relation d'inclusion \(\subset\).

\begin{enumerate}
\item Soient \(A,B\in E\). La partie \(\accol{A;B}\subset E\) admet-elle une borne supérieure dans \(E\) ? une borne inférieure dans \(E\) ? \\

\item Soit \(\paren{A_i}_{i\in I}\in E^I\) une famille d'éléments de \(E\). Cette famille admet-elle une borne supérieure dans \(E\) ? une borne inférieure dans \(E\) ? \\

\textit{NB : dans le cas présent, où tous les ensembles \(A_i\) sont des parties de \(E\), on convient que l'intersection \(\biginter_{i\in I}A_i\) vaut \(E\) si \(I\) est vide.}
\end{enumerate}
\end{exo}

\begin{corr}
\note{à venir}
\end{corr}

\begin{exo}[Exercice 19]\thlabel{exo:toutePartieDeRbAdmetUnSupEtUnInfDansRb}
On considère la droite réelle achevée \(\Rb\), munie de sa relation d'ordre usuelle.

Montrer que toute partie de \(\Rb\) admet une borne supérieure et une borne inférieure dans \(\Rb\).
\end{exo}

\begin{corr}
\note{à venir}
\end{corr}

\section{Entiers naturels}

\begin{exo}[Exercice 20, descente infinie de Fermat]
Montrer qu'il n'existe pas de suite \(\paren{u_n}_{n\in\N}\in\N^\N\) d'entiers naturels strictement décroissante.
\end{exo}

\begin{corr}
\note{à venir}
\end{corr}

\begin{exo}[Exercice 21]
On considère la suite \(\paren{u_n}_{n\in\N}\in\R^\N\) définie par : \[\begin{dcases}u_0=1 \\ \quantifs{\forall n\in\N}u_{n+1}=\sum_{k=0}^{n}u_k\end{dcases}\]

Montrer la proposition suivante : \[\quantifs{\forall n\in\Ns}u_n=2^{n-1}.\]
\end{exo}

\begin{corr}
\note{à venir}
\end{corr}