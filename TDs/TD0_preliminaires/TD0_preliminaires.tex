\chapter{Préliminaires}

\minitoc

\section{Logique}

\begin{exo}[Exercice 1]
Soient \(P\), \(Q\) et \(R\) des propositions.

\begin{enumerate}
\item Montrer que les propositions suivantes sont vraies à l'aide de tables de vérité :

\begin{enumerate}
\item \(\paren{P\imp Q}\ou P\) \\

\item \(\paren{P\imp Q}\ssi\paren{Q\ou\non P}\) \\

\item \(\croch{P\et\paren{P\imp Q}}\imp Q\) \\
\end{enumerate}

\item Déduire de (b) une nouvelle démonstration de l'équivalence \[\paren{P\imp Q}\ssi\paren{\non Q\imp\non P}.\]
\end{enumerate}
\end{exo}

\begin{corr}
\note{à venir}
\end{corr}

\section{Quantificateurs}

\begin{exo}[Exercice 2]
Les propositions suivantes sont-elles vraies ?

\begin{enumerate}
\item \(\quantifs{\forall x\in\Rps}\ln\paren{x}=0\) \\

\item \(\quantifs{\exists x\in\Rps}\ln\paren{x}=0\) \\

\item \(\quantifs{\exists x\in\R}\exp\paren{x}=0\) \\

\item \(\quantifs{\forall x,y\in\R}2xy\geq x^2+y^2\) (en justifiant) \\

\item \(\quantifs{\forall x\in\R;\exists y\in\R}x<y\) \\

\item \(\quantifs{\exists y\in\R;\forall x\in\R}x<y\)
\end{enumerate}
\end{exo}

\begin{corr}
\note{à venir}
\end{corr}

\begin{exo}[Exercice 3]
Soit \(f:\R\to\R\) une fonction et \(T\) un réel strictement positif.

Écrire à l'aide de quantificateurs les propositions suivantes :

\begin{enumerate}
\item La fonction \(f\) est paire \\

\item La fonction \(f\) est impaire \\

\item La fonction \(f\) est périodique, de période \(T\) \\

\item La fonction \(f\) est périodique
\end{enumerate}
\end{exo}

\begin{corr}
\note{à venir}
\end{corr}

\begin{exo}[Exercice 4]
Écrire à l'aide de quantificateurs les négations des propositions des exercices 2 et 3.
\end{exo}

\begin{corr}
\note{à venir}
\end{corr}

\begin{exo}[Exercice 5]
Soit \(f:\R\to\R\) une fonction.

Écrire à l'aide de quantificateurs les propositions suivantes :

\begin{enumerate}
\item La fonction \(f\) est constante \\

\item La fonction \(f\) est croissante \\

\item La fonction \(f\) est strictement décroissante
\end{enumerate}
\end{exo}

\begin{corr}
\note{à venir}
\end{corr}

\begin{exo}[Exercice 6]
Les propositions suivantes sont-elles vraies ?

\begin{enumerate}
\item \(\quantifs{\forall a,b\in\Rs}a\leq b\imp\dfrac{1}{a}\geq\dfrac{1}{b}\) \\

\item \(\quantifs{\forall a,b\in\Rps}a\leq b\imp\dfrac{1}{a}\geq\dfrac{1}{b}\) \\

\item \(\quantifs{\forall a,b\in\Rps}a<b\imp\dfrac{1}{a}>\dfrac{1}{b}\) \\

\item \(\quantifs{\forall a,b,c,d\in\R}\paren{a\leq b\quad\text{et}\quad c\leq d}\imp a+c\leq b+d\) \\

\item \(\quantifs{\forall a,b,c,d\in\R}\paren{a\leq b\quad\text{et}\quad c\leq d}\imp ac\leq bd\) \\

\item \(\quantifs{\forall a,b,c,d\in\Rp}\paren{a\leq b\quad\text{et}\quad c\leq d}\imp ac\leq bd\)
\end{enumerate}
\end{exo}

\begin{corr}
\note{à venir}
\end{corr}

\section{Raisonnement par analyse-synthèse}

\begin{exo}[Exercice 7]
Déterminer les suites \(\paren{u_n}_{n\in\N}\) de réels vérifiant : \[\quantifs{\forall m,n\in\N}u_{m+n}=u_m+u_n.\]
\end{exo}

\begin{corr}
\note{à venir}
\end{corr}

\section{Congruences}

\begin{exo}[Exercice 8]
Les propositions suivantes sont-elles vraies ?

\begin{enumerate}
\item \(\quantifs{\forall x,y\in\Z}x\equiv y\croch{5}\imp x^2\equiv y^2\croch{5}\) \\

\item \(\quantifs{\forall x,y\in\R}x\equiv y\croch{5}\imp x^2\equiv y^2\croch{5}\) \\

\item \(\quantifs{\exists x\in\Z}x^2\equiv-1\croch{5}\) \\

\item \(\quantifs{\exists x\in\Z}x^2\equiv-1\croch{7}\) \\

\item \(\quantifs{\exists x\in\R}x^2\equiv-1\croch{7}\)
\end{enumerate}
\end{exo}

\begin{corr}
\note{à venir}
\end{corr}

\begin{exo}[Exercice 9]
Soient \(x\) et \(y\) deux nombres réels.

Chercher quelles implications sont vraies entre les propositions suivantes :

\begin{enumerate}
\item \(x\equiv y\croch{\pi}\) \\

\item \(x\equiv y+\pi\croch{\pi}\) \\

\item \(x\equiv y\croch{2\pi}\) \\

\item \(x\equiv y+\pi\croch{2\pi}\) \\

\item \(2x\equiv 2y\croch{2\pi}\) \\

\item \(\dfrac{x}{2}\equiv\dfrac{y}{2}\croch{\dfrac{\pi}{2}}\) \\

\item \(\dfrac{x}{2}\equiv\dfrac{y}{2}+\dfrac{\pi}{2}\croch{\pi}\)
\end{enumerate}
\end{exo}

\begin{corr}
\note{à venir}
\end{corr}