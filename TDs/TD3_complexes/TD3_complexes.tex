\chapter{Nombres complexes}

\minitoc

\begin{exo}[Exercice 1]
Écrire les nombres complexes suivants sous forme algébrique et sous forme trigonométrique : \[1+\i\qquad1+j\qquad\dfrac{\i}{1-\i}\qquad\dfrac{\sqrt{3}+3\i}{1+\i}.\]
\end{exo}

\begin{corr}
\note{à venir}
\end{corr}

\begin{exo}[Exercice 2]
Calculer \[\i^{2022}\qquad j^{2022}\qquad\paren{1+j}^{2022}\qquad\paren{1+\i}^{2022}.\]
\end{exo}

\begin{corr}
\note{à venir}
\end{corr}

\begin{exo}[Exercice 3]~\\
Écrire \(z=\sqrt{2+\sqrt{2}}+\i\sqrt{2-\sqrt{2}}\) sous forme trigonométrique.

\textit{Indication :} déterminer \(\cos\paren{2\arg z}\).
\end{exo}

\begin{corr}
\note{à venir}
\end{corr}

\begin{exo}[Exercice 4]
Résoudre les équations suivantes, d'inconnue \(z\in\C\) :

\begin{enumerate}
\item \(z^2+2z+5=0\) \\

\item \(z^2+\paren{1+\i}z-\i=0\) \\

\item \(z^4+z^2+1=0\) \\

\item \(z^4-4\i z^2-4=0\)
\end{enumerate}
\end{exo}

\begin{corr}
\note{à venir}
\end{corr}

\begin{exo}[Exercice 5]
Résoudre le système suivant, d'inconnues \(x,y\in\C\) : \[\begin{dcases}x+y=1+\i \\ xy=2-\i\end{dcases}\]
\end{exo}

\begin{corr}
\note{à venir}
\end{corr}

\begin{exo}[Exercice 6, CCP 2016]
On pose \[\omega=\exp\dfrac{2\i\pi}{7}\qquad S=\omega+\omega^2+\omega^4\qquad T=\omega^3+\omega^5+\omega^6.\]

\begin{enumerate}
\item Calculer \(S+T\) et \(ST\). \\

\item En déduire les valeurs de \(S\) et \(T\).
\end{enumerate}
\end{exo}

\begin{corr}
\note{à venir}
\end{corr}

\begin{exo}[Exercice 7]
Décrire l'ensemble \(\U_{10}\) des racines dixièmes de l'unité.

Quels éléments de \(\U_{10}\) sont racines carrées de l'unité ? racines cinquièmes ? racines septièmes ? racines quinzièmes ? racines vingtièmes ?
\end{exo}

\begin{corr}
\note{à venir}
\end{corr}

\begin{exo}[Exercice 8]
Soient \(x,y\in\C\).

Montrer les propositions suivantes :

\begin{enumerate}
\item \(\abs{x}+\abs{y}\leq\abs{x+y}+\abs{x-y}\) \\

\item \(\abs{x+y}^2+\abs{x-y}^2=2\abs{x}^2+2\abs{y}^2\qquad\text{\guillemets{identité du parallélogramme} dans }\C\)
\end{enumerate}
\end{exo}

\begin{corr}
\note{à venir}
\end{corr}

\begin{exo}[Exercice 9]
Soient \(A\), \(B\) et \(C\) des points du plan d'affixes respectives \(a\), \(b\) et \(c\).

\begin{enumerate}
\item Montrer que le triangle \(ABC\) est équilatéral direct si, et seulement si \[a+jb+j^2c=0.\] \textit{\small Si \(A=B=C\), on convient que le triangle \(ABC\) est équilatéral direct et indirect.} \\

\item Donner une CNS analogue pour que le triangle \(ABC\) soit équilatéral indirect. \\

\item Montrer que le triangle \(ABC\) est équilatéral si, et seulement si \[a^2+b^2+c^2=ab+bc+ca.\]
\end{enumerate}
\end{exo}

\begin{corr}
\note{à venir}
\end{corr}

\begin{exo}[Exercice 10]
Soit \(\paren{n,a,b}\in\N\times\R\times\R\).

Calculer \[\sum_{k=0}^{n}\sin\paren{a+kb}\qquad\text{et}\qquad\sum_{k=0}^{n}\cos\paren{a+kb}.\]
\end{exo}

\begin{corr}
\note{à venir}
\end{corr}

\begin{exo}[Exercice 11]
Soit \(\theta\in\R\).

Développer \[\sin\paren{5\theta}\qquad\cos\paren{5\theta}\qquad\tan\paren{5\theta}\qquad\sin\paren{7\theta}\qquad\cos\paren{7\theta}\qquad\tan\paren{7\theta}.\]
\end{exo}

\begin{corr}
\note{à venir}
\end{corr}

\begin{exo}[Exercice 12]
Soit \(\theta\in\R\).

Linéariser \[\cos^5\theta\qquad\sin^5\theta\qquad\cos^4\theta\sin^2\theta.\]
\end{exo}

\begin{corr}
\note{à venir}
\end{corr}

\begin{exo}[Exercice 13]
Soit \(n\in\Ns\).

Calculer \[S=\sum_{\omega\in\U_n}\omega\qquad\text{et}\qquad P=\prod_{\omega\in\U_n}\omega.\]
\end{exo}

\begin{corr}
\note{à venir}
\end{corr}

\begin{exo}[Exercice 14]~\\
Calculer \(\cos\dfrac{2\pi}{5}\).

Indication : on pourra utiliser la somme \(\sum_{\omega\in\U_5}\omega\).
\end{exo}

\begin{corr}
\note{à venir}
\end{corr}

\begin{exo}[Exercice 15]
Résoudre les équations suivantes, d'inconnue \(z\in\C\) :

\begin{enumerate}
\item \(\e{z}=0\) \\

\item \(\e{z}=1\) \\

\item \(\e{z}=\i\) \\

\item \(\e{z}=2j\)
\end{enumerate}
\end{exo}

\begin{corr}
\note{à venir}
\end{corr}

\begin{exo}[Exercice 16]
Déterminer les nombres complexes \(z\in\C\) tels que \[z+\conj{z}=\abs{z}.\]
\end{exo}

\begin{corr}
\note{à venir}
\end{corr}

\begin{exo}[Exercice 17]
Soit \(n\in\Ns\).

Résoudre l'équation d'inconnue \(z\in\C\) : \[\paren{z+1}^n=\paren{z-1}^n.\]
\end{exo}

\begin{corr}
\note{à venir}
\end{corr}

\begin{exo}[Exercice 18]
Soient \(a,b,c\in\R\) avec \(a\not=0\).

Montrer que les racines du polynôme \(P=aX^2+bX+c\) ont leurs parties réelles respectives strictement négatives si, et seulement si, on a : \[a,b,c\in\Rps\qquad\text{ou}\qquad a,b,c\in\Rms.\]
\end{exo}

\begin{corr}
\note{à venir}
\end{corr}

\begin{exo}[Exercice 19]
Soit \(z\in\C\).

Donner une CNS pour que les points \(z\), \(z^2\) et \(z^4\) soient alignés.
\end{exo}

\begin{corr}
\note{à venir}
\end{corr}

\begin{exo}[Exercice 20]
On pose \(\mathbb{H}=\accol{z\in\C\tq\Im z>0}\) et \(\D=\accol{z\in\C\tq\abs{z}<1}\).

Montrer que la fonction \(\fonction{f}{\mathbb{H}}{\D}{z}{\dfrac{z-\i}{z+\i}}\) est une bijection de \(\mathbb{H}\) vers \(\D\).

\textit{NB : On vérifiera que la fonction \(f\) est bien définie.}
\end{exo}

\begin{corr}
\note{à venir}
\end{corr}