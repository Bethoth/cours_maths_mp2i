\chapter{Déterminants}

\minitoc

\begin{exo}
Soient \(a,b,c,d\in\R\).

Calculer les déterminants suivants quand ils sont définis : \[A=\begin{vmatrix}
a\inv & a & a^3 \\
b\inv & b & b^3 \\
c\inv & c & c^3
\end{vmatrix}\qquad B=\begin{vmatrix}
a & c & c & b \\
c & a & b & c \\
c & b & a & c \\
b & c & c & a
\end{vmatrix}\qquad C=\begin{vmatrix}
a+b & b+c & c+a \\
a^2+b^2 & b^2+c^2 & c^2+a^2 \\
a^3+b^3 & b^3+c^3 & c^3+a^3
\end{vmatrix}\]

On factorisera autant que possible les résultats.
\end{exo}

\begin{corr}
\note{À venir}
\end{corr}

\begin{exo}
\begin{enumerate}
\item Donner une CNS sur \(\lambda\in\R\) pour que \[\fonction{u}{\polydeg[\R]{2}}{\polydeg[\R]{2}}{P}{P+\lambda XP\prim}\] soit un automorphisme de \(\polydeg[\R]{2}\). \\

\item Donner une CNS sur \(\lambda\in\R\) pour que \[\fonction{u}{\polydeg[\R]{3}}{\polydeg[\R]{3}}{P}{P+\paren{\lambda X+1}P\prim+\lambda X^2P\paren{0}}\] soit un automorphisme de \(\polydeg[\R]{3}\). \\

\item (Mines-Telecom 2016) Soit \(n\in\Ns\). Calculer le déterminant de \[\fonction{u}{\polydeg[\R]{n}}{\polydeg[\R]{n}}{P}{XP\prim+P}\]
\end{enumerate}
\end{exo}

\begin{corr}
\note{À venir}
\end{corr}

\begin{exo}
\begin{enumerate}
\item Donner une CNS sur \(\lambda,\mu\in\R\) pour que \[\fonction{u}{\polydeg[\R]{2}}{\polydeg[\R]{2}}{P}{\paren{\lambda X^2+\mu}P\paren{1}+XP\prim}\] soit un automorphisme de \(\polydeg[\R]{2}\). \\

\item Plus généralement, donner une CNS sur \(\lambda,\mu\in\R\) et \(n\in\Ns\) pour que \[\fonction{u}{\polydeg[\R]{n}}{\polydeg[\R]{n}}{P}{\paren{\lambda X^n+\mu}P\paren{1}+XP\prim}\] soit un automorphisme de \(\polydeg[\R]{n}\).
\end{enumerate}
\end{exo}

\begin{corr}
\note{À venir}
\end{corr}

\begin{exo}
Soient \(a,b,c,t\in\R\).

On suppose que \(a\), \(b\) et \(c\) sont deux à deux distincts.

Résoudre le système d'inconnue \(\paren{x,y,z}\in\R^3\) : \[\paren{S}\begin{dcases}
x+y+z=t \\
ax+by+cz=t^2 \\
a^2x+b^2y+c^2z=t^3
\end{dcases}\]
\end{exo}

\begin{corr}
\note{À venir}
\end{corr}

\begin{exo}
Soient \(a,b\in\R\).

Résoudre le système d'inconnues \(x,y,z\in\R\) : \[\paren{S}\begin{dcases}
ax+\paren{a-1}y+\paren{a+b}z=1 \\
ax+ay+bz=a \\
bx+by+az=b
\end{dcases}\]
\end{exo}

\begin{corr}
\note{À venir}
\end{corr}

\begin{exo}
Soit \(n\in\Ns\).

Calculer le déterminant de la matrice \(A=\paren{a_{ij}}_{\paren{i,j}\in\interventierii{1}{n}^2}\in\M{n}[\R]\) définie par : \[\quantifs{\forall i,j\in\interventierii{1}{n}}a_{ij}=\abs{i-j}.\]
\end{exo}

\begin{corr}
\note{À venir}
\end{corr}

\begin{exo}
Soient \(x_1,\dots,x_n\in\C\).

Calculer le déterminant \[\begin{vmatrix}
0 & \dots & 0 & x_1 \\
\vdots & \iddots & \iddots & 0 \\
0 & \iddots & \iddots & \vdots \\
x_n & 0 & \dots & 0
\end{vmatrix}\]
\end{exo}

\begin{corr}
\note{À venir}
\end{corr}

\begin{exo}
\begin{enumerate}
\item Soient \(x_1,\dots,x_n\in\C\). On pose \(\quantifs{\forall k\in\interventierii{1}{n}}s_k=\sum_{j=1}^kx_j\).

Calculer le déterminant \[\begin{vmatrix}
s_1 & \dots & \dots & \dots & s_1 \\
\vdots & s_2 & \dots & \dots & s_2 \\
\vdots & \vdots & s_3 & \dots & s_3 \\
\vdots & \vdots & \vdots & \ddots & \vdots \\
s_1 & s_2 & s_3 & \dots & s_n
\end{vmatrix}\]

\item Que vaut le déterminant suivant ? \[\begin{vmatrix}
1 & \dots & \dots & \dots & 1 \\
\vdots & 2 & \dots & \dots & 2 \\
\vdots & \vdots & 3 & \dots & 3 \\
\vdots & \vdots & \vdots & \ddots & \vdots \\
1 & 2 & 3 & \dots & n
\end{vmatrix}\]

Retrouver ce résultat en utilisant l'\thref{exo:matrice1...n}.
\end{enumerate}
\end{exo}

\begin{corr}
\note{À venir}
\end{corr}

\begin{exo}
Calculer le déterminant et la trace des endomorphismes \[\fonction{f}{\M{n}[\C]}{\M{n}[\C]}{M}{\trans{M}}\qquad\text{et}\qquad\fonction{g}{\M{n}[\C]}{\M{n}[\C]}{M}{\paren{\tr M}I_n}\]
\end{exo}

\begin{corr}
\note{À venir}
\end{corr}

\begin{exo}
Soient \(n\in\Ns\) et \(A\in\M{n}[\R]\) antisymétrique et inversible.

Montrer que l'entier \(n\) est pair.

Donner un exemple de matrice antisymétrique inversible.
\end{exo}

\begin{corr}
\note{À venir}
\end{corr}

\begin{exo}
Soient \(n\in\Ns\) et \(A\in\M{n}[\R]\) telle que \(A^2=-I_n\).

Montrer que l'entier \(n\) est pair.

Donner un exemple

\begin{itemize}
\item de matrice \(A\in\M{n}[\R]\) telle que \(A^2=-I_n\). \\

\item de matrice \(A\in\M{n}[\C]\) telle que \(A^2=-I_n\) avec \(n\) impair.
\end{itemize}
\end{exo}

\begin{corr}
\note{À venir}
\end{corr}

\begin{exo}
Soient \(A,B\in\M{n}[\K]\) avec \(\K=\R\) ou \(\C\).

Montrer que \(A\) et \(B\) sont semblables sur \(\R\) si, et seulement si, elles sont semblables sur \(\C\), \cad : \[\croch{\quantifs{\exists P\in\GL{n}[\R]}B=PAP\inv}\ssi\croch{\quantifs{\exists Q\in\GL{n}[\C]}B=QAQ\inv}.\]
\end{exo}

\begin{corr}
\note{À venir}
\end{corr}

\begin{exo}[Déterminant d'une matrice \guillemets{tridiagonale}]
Soient \(x,y\in\C\) et \(n\in\interventierie{2}{\pinf}\).

On considère la matrice \(M_n=\paren{m_{ij}}_{\paren{i,j}\in\interventierii{1}{n}^2}\in\M{n}[\C]\) définie par : \[\quantifs{\forall\paren{i,j}\in\interventierii{1}{n}^2}m_{ij}=\begin{dcases}
0 &\text{si }\abs{i-j}\geq2 \\
xy &\text{si }i-j=-1 \\
x+y &\text{si }i-j=0 \\
1 &\text{si }i-j=1
\end{dcases}\]

Calculer le déterminant de \(M_n\).

\textit{Indication :} poser \(u_n=\det M_n\) et montrer que la suite \(\paren{u_n}_n\) vérifie une relation de récurrence linéaire d'ordre 2 : \(\quantifs{\forall n\in\interventierie{2}{\pinf}}au_{n+2}+bu_{n+1}+cu_n=0\) (avec \(a,b,c\in\C\)) puis en déduire la valeur de \(u_n\) en fonction de \(n\) (utiliser l'exercice \thref{exo:relationsDeRécurrenceD'Ordre2}).
\end{exo}

\begin{corr}
\note{À venir}
\end{corr}

\begin{exo}
Soit \(x\in\C\).

On considère la matrice \(M=\paren{m_{ij}}_{\paren{i,j}\in\interventierii{1}{n}^2}\in\M{n}[\C]\) définie par : \[\quantifs{\forall\paren{i,j}\in\interventierii{1}{n}^2}m_{ij}=\begin{dcases}
0 &\text{si }\abs{i-j}\geq2 \\
x &\text{si }\abs{i-j}=1 \\
2x &\text{si }i=j
\end{dcases}\]

Calculer le déterminant de \(M\).
\end{exo}

\begin{corr}
\note{À venir}
\end{corr}

\begin{exo}
Soit \(n\in\Ns\).

On note \(\fami{B}=\paren{e_1,\dots,e_n}\) la base canonique de \(\R^n\).

\begin{enumerate}[series=det15]
\item Soit \(\sigma\in\S{n}\). Justifier qu'il existe un unique endomorphisme \(u\in\Lendo{\R^n}\) tel que \[\quantifs{\forall j\in\interventierii{1}{n}}u\paren{e_j}=e_{\sigma\paren{j}}.\]
\end{enumerate}

Dans la suite, on note \(u_\sigma\) cet endomorphisme et on note \(M_\sigma\) sa matrice dans \(\fami{B}\). Les matrices de la forme \(M_\sigma\), où \(\sigma\in\S{n}\), sont appelées matrices de permutation.

\begin{enumerate}[resume=det15]
\item On suppose ici \(n=3\). Donner la matrice de permutation \(M_\sigma\) pour toute permutation \(\sigma\in\S{3}\). \\

\item Montrer que l'application \[\fonction{\phi}{\S{n}}{\GL{}[\R^n]}{\sigma}{u_\sigma}\] est bien définie et est un morphisme de groupes. \\

\item En déduire un morphisme de groupes de \(\S{n}\) vers \(\GL{n}[\R]\). \\

\item Soit \(\sigma\in\S{n}\). Calculer le déterminant de \(u_\sigma\).
\end{enumerate}
\end{exo}

\begin{corr}
\note{À venir}
\end{corr}

\begin{exo}
Soit \(n\in\Ns\).

On note \(\M{n}[\Z]\) l'ensemble des matrices carrées de taille \(n\) à coefficients dans \(\Z\).

\begin{enumerate}
\item Vérifier que \(\M{n}[\Z]\) est un sous-anneau de \(\anneau{\M{n}[\R]}\). \\

\item Soit \(A\in\M{n}[\Z]\). Donner une CNS sur \(\det A\) pour que \(A\) soit inversible dans \(\M{n}[\Z]\). \\

\item Montrer que \(\M{n}[\Z]\) possède une infinité d'éléments inversibles si \(n\geq2\).
\end{enumerate}
\end{exo}

\begin{corr}
\note{À venir}
\end{corr}

\begin{exo}[ENSAM PSI 2016 (BEOS 2727)]
Soit \(P\in\poly[\R]\) un polynôme de degré \(3\).

On lui associe la famille de vecteurs de \(\polydeg[\R]{4}\) : \[\fami{F}=\paren{P,XP,P\prim,XP\prim,X^2P\prim}.\]

On note \(D\) le déterminant de cette famille dans la base canonique de \(\polydeg[\R]{4}\).

\begin{enumerate}
\item Montrer que \(D\) est nul si, et seulement si, il existe deux polynômes non-nuls \(U\in\polydeg[\R]{1}\) et \(V\in\polydeg[\R]{2}\) tels que \(PU=P\prim V\). \\

\item Montrer que \(D\) est nul si, et seulement si, le polynôme \(P\) admet une racine multiple. \\

\item Calculer \(D\) si \(P=aX^3+bX^2+cX\).
\end{enumerate}
\end{exo}

\begin{corr}
\note{À venir}
\end{corr}

\begin{exo}[Polynômes interpolateurs de Lagrange]
Soient \(x_0,\dots,x_n\in\K\) des scalaires deux à deux distincts.

On note \(\fami{B}=\paren{1,X,\dots,X^n}\) la base canonique de \(\polydeg{n}\).

On note \(L_0,\dots,L_n\in\polydeg{n}\) les polynômes définis par : \[\quantifs{\forall j\in\interventierii{0}{n}}L_j=\prod_{k\in\interventierii{0}{n}\excluant\accol{j}}\dfrac{X-x_k}{x_j-x_k}.\]

\begin{enumerate}
\item Montrer que \(\paren{L_0,\dots,L_n}\) est une base de \(\polydeg{n}\). \\

\item Calculer \(\detb{\fami{B}}\paren{L_0,\dots,L_n}\).
\end{enumerate}
\end{exo}

\begin{corr}
\note{À venir}
\end{corr}

\begin{exo}[ENSEA PSI 2018]
Soient \(E\) un espace vectoriel de dimension \(3\), \(u\in\Lendo{E}\) et \(\fami{B}\) une base de \(E\).

Étudier l'application \[\fonction{\phi}{E^3}{\K}{\paren{x,y,z}}{\detb{\fami{B}}\paren{u\paren{x},y,z}+\detb{\fami{B}}\paren{x,u\paren{y},z}+\detb{\fami{B}}\paren{x,y,u\paren{z}}}\]
\end{exo}

\begin{corr}
\note{À venir}
\end{corr}

\begin{exo}[TPE PSI 2018]
Soient \(a,b,r_1,\dots,r_n\in\R\).

On pose : \[A=\begin{pmatrix}
r_1 & a & \dots & a \\
b & \ddots & \ddots & \vdots \\
\vdots & \ddots & \ddots & a \\
b & \dots & b & r_n
\end{pmatrix}\qquad\text{et}\qquad\quantifs{\forall x\in\R}\Delta\paren{x}=\begin{vmatrix}
r_1+x & a+x & \dots & a+x \\
b+x & \ddots & \ddots & \vdots \\
\vdots & \ddots & \ddots & a+x \\
b+x & \dots & b+x & r_n+x
\end{vmatrix}\]

\begin{enumerate}
\item Montrer que la fonction \(\Delta\) est polynomiale de degré au plus \(1\). \\

\item En déduire \(\det A\) si \(a\not=b\). \\

\item Supposons \(a=b\). Comment calculer \(\det A\) ?
\end{enumerate}
\end{exo}

\begin{corr}
\note{À venir}
\end{corr}