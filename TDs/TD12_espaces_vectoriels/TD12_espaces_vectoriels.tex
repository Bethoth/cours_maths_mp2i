\chapter{Espaces vectoriels}

\minitoc

Soit \(\K\) un corps.

\section{Espaces vectoriels}

\begin{exo}
Les ensembles suivants sont-ils (naturellement) des \(\R\)-espaces vectoriels ?

\begin{enumerate}
\item \(\accol{\paren{a,b,c,d}\in\R^4\tq a+2b-c-d=\lambda}\) où \(\lambda\) est un réel fixé. \\

\item \(\accol{y\in\ensclasse{2}{\R}{\R}\tq y\seconde+2y\prim+fy=g}\) où \(f\) et \(g\) sont deux fonctions fixées. \\

\item \(\accol{y\in\ensclasse{\infty}{\R}{\R}\tq y\text{ est croissante sur }\R}\). \\

\item \(\accol{y\in\ensclasse{\infty}{\R}{\R}\tq y\text{ est monotone sur }\R}\).
\end{enumerate}
\end{exo}

\begin{corr}
\note{À venir}
\end{corr}

\begin{exo}[Mines-Telecom PSI 2016]
Soient \(E\) un espace vectoriel et \(F\) et \(G\) deux sous-espaces vectoriels de \(E\).

Montrer : \[F\union G\text{ est un sous-espace vectoriel de }E\ssi\orenv{F\subset G \\ G\subset F}\]
\end{exo}

\begin{corr}
\note{À venir}
\end{corr}

\section{Applications linéaires}

\begin{exo}
Les applications suivantes sont-elles des applications linéaires (pour les structures naturelles d'espaces vectoriels) ?

\[\fonction{u_1}{\R}{\R}{x}{x^2}\qquad\fonction{u_2}{\ensclasse{\infty}{\R}{\R}}{\R\times\ensclasse{\infty}{\R}{\R}}{f}{\paren{f\prim\paren{1},f\seconde}}\qquad\fonction{u_3}{\poly[\R]}{\poly[\R]}{P}{P\paren{X^2}}\]
\end{exo}

\begin{corr}
\note{À venir}
\end{corr}

\begin{exo}
On note \(B\) l'ensemble des suites réelles bornées et \(C\) l'ensemble des suites réelles convergentes.

\begin{enumerate}
\item Montrer que \(B\) et \(C\) sont naturellement des \(\R\)-espaces vectoriels. \\

\item L'application \[\fonction{u}{B}{\R}{\paren{u_n}_{n\in\N}}{\sup_{n\in\N}u_n}\] est-elle une forme linéaire ? \\

\item L'application \[\fonction{v}{C}{\R}{\paren{u_n}_{n\in\N}}{\lim_{n\to\pinf}u_n}\] est-elle une forme linéaire ?
\end{enumerate}
\end{exo}

\begin{corr}
\note{À venir}
\end{corr}

\begin{exo}
Soient \(E\) un \(\K\)-espace vectoriel et \(u,v\in\Lendo{E}\).

On suppose que les endomorphismes \(u\) et \(v\) commutent.

Montrer que \(u\) stabilise \(\ker v\) et \(\Im v\), \cad : \[u\paren{\ker v}\subset\ker v\qquad\text{et}\qquad u\paren{\Im v}\subset\Im v.\]
\end{exo}

\begin{corr}
\note{À venir}
\end{corr}

\begin{exo}[Résultats à bien connaître]
Soient \(E\) un \(\K\)-espace vectoriel et \(u,v\in\Lendo{E}\).

\begin{enumerate}
\item Montrer : \[v\rond u=0\ssi\Im u\subset\ker v.\]

\item Donner les inclusions qui sont vraies entre les sous-espaces vectoriels de \(E\) suivants :

\begin{enumerate}
\item \(\Im\paren{v\rond u}\) et \(\Im v\). \\

\item \(\ker\paren{v\rond u}\) et \(\ker u\). \\

\item \(\Im\paren{u+v}\) et \(\Im u+\Im v\). \\

\item \(\ker\paren{u+v}\) et \(\ker u\inter\ker v\). \\
\end{enumerate}

\item Que peut-on en déduire pour les suites de sous-espaces vectoriels de \(E\) : \[\paren{\Im u^k}_{k\in\N}\qquad\text{et}\qquad\paren{\ker u^k}_{k\in\N}\text{ ?}\]
\end{enumerate}
\end{exo}

\begin{corr}
\note{À venir}
\end{corr}

\begin{exo}[Classique]
Soient \(E\) un \(\K\)-espace vectoriel et \(u\in\Lendo{E}\).

On suppose que pour tout vecteur \(x\in E\), le vecteur \(u\paren{x}\) est colinéaire à \(x\).

\begin{enumerate}
\item Justifier que \(u\) vérifie : \[\quantifs{\forall x\in E;\exists\lambda_x\in\K}u\paren{x}=\lambda_xx.\]

\item Montrer que \(u\) est une homothétie.
\end{enumerate}
\end{exo}

\begin{corr}
\note{À venir}
\end{corr}

\begin{exo}[Arts \& Métiers PSI 2016]
On pose \[E=\accol{f\in\ensclasse{\infty}{\R}{\R}\tq\quantifs{\forall x\in\R}f\paren{x+2\pi}=f\paren{x}}\qquad\text{et}\qquad\quantifs{\forall f\in E}u\paren{f}=f\seconde.\]

\begin{enumerate}
\item Montrer que \(E\) est un espace vectoriel et que \(u\in\Lendo{E}\). \\

\item Déterminer \(\ker u\). \\

\item Déterminer les vecteurs \(f\in E\) tels que \(u\paren{f}=\sin\) et les vecteurs \(f\in E\) tels que \(u\paren{f}=\sin^2\). \\

\item Déterminer \(\Im u\). \\

\item Montrer \(\ker u\oplus\Im u=E\).
\end{enumerate}
\end{exo}

\begin{corr}
\note{À venir}
\end{corr}

\begin{exo}
Soient \(E\), \(F\) et \(G\) des \(\K\)-espaces vectoriels et \(f:E\to F\) et \(g:F\to G\) des applications linéaires.

\begin{enumerate}
\item Montrer : \[f\paren{\ker\paren{g\rond f}}=\Im f\inter\ker g.\]

\item Montrer : \[\ker\paren{g\rond f}=\ker f\ssi\Im f\inter\ker g=\accol{0_F}.\]
\end{enumerate}
\end{exo}

\begin{corr}
\note{À venir}
\end{corr}

\section{Sommes directes, projecteurs}

\begin{exo}
Soient \(E\) un espace vectoriel et \(p,q\in\Lendo{E}\) des projecteurs.

Montrer que les propositions suivantes sont équivalentes :

\begin{enumerate}
\item \(p+q\) est un projecteur \\

\item \(pq+qp=0\) \\

\item \(pq=qp=0\)
\end{enumerate}
\end{exo}

\begin{corr}
\note{À venir}
\end{corr}

\begin{exo}
Soient \(E\) un espace vectoriel et \(f,g\in\Lendo{E}\) tels que \(f\rond g=\id{E}\).

Montrer que \(g\rond f\) est un projecteur. Déterminer son image et son noyau.
\end{exo}

\begin{corr}
\note{À venir}
\end{corr}

\begin{exo}
Soient \(E\) un \(\K\)-espace vectoriel et \(p\) et \(q\) des projecteurs de \(E\) qui commutent.

\begin{enumerate}
\item Montrer que \(pq\) est un projecteur. \\

\item Montrer \(\Im\paren{pq}=\Im p\inter\Im q\). \\

\item Montrer \(\ker\paren{pq}=\ker p+\ker q\).
\end{enumerate}
\end{exo}

\begin{corr}
\note{À venir}
\end{corr}

\begin{exo}[Polynômes interpolateurs de Lagrange]
Soient \(x_0,\dots,x_n\in\K\) deux à deux distincts.

On note \(L_0,\dots,L_n\in\polydeg{n}\) les polynômes définis par : \[\quantifs{\forall j\in\interventierii{0}{n}}L_j=\prod_{k\in\interventierii{0}{n}\excluant\accol{j}}\dfrac{X-x_k}{x_j-x_k}.\]

\begin{enumerate}
\item Montrer que \(\paren{L_0,\dots,L_n}\) est une base de \(\polydeg{n}\). \\

\item Montrer que l'application \[\fonction{u}{\poly}{\poly}{P}{P\paren{x_0}L_0+\dots+P\paren{x_n}L_n}\] est un projecteur. \\

\item Déterminer l'image et le noyau de \(u\).
\end{enumerate}
\end{exo}

\begin{corr}
\note{À venir}
\end{corr}

\begin{exo}
Donner un supplémentaire de \(\poly\) dans \(\fracrat\).
\end{exo}

\begin{corr}
\note{À venir}
\end{corr}

\section{Familles libres / génératrices}

\begin{exo}
On considère le système linéaire homogène : \[\paren{S}\begin{dcases}
a+\i b+c+\i d=0 \\
\i a-b+\i c-d=0 \\
\paren{1-\i}a+\paren{1+\i}b+\paren{1+\i}c-\paren{1-\i}d=0
\end{dcases}\]

On note \(\fami{S}\subset\C^4\) son ensemble solution.

C'est un \(\C\)-espace vectoriel et donc également un \(\R\)-espace vectoriel.

\begin{enumerate}
\item Donner une base du \(\C\)-espace vectoriel \(\fami{S}\). \\

\item Donner une base du \(\R\)-espace vectoriel \(\fami{S}\).
\end{enumerate}
\end{exo}

\begin{corr}
\note{À venir}
\end{corr}

\begin{exo}
On pose \[\quantifs{\forall\lambda\in\R}\fonction{f_\lambda}{\R}{\R}{t}{\e{\lambda t}}\]

On considère la famille \(\fami{F}=\paren{f_\lambda}_{\lambda\in\R}\) de vecteurs de \(\F{\R}{\R}\).

\begin{enumerate}
\item La famille \(\fami{F}\) est-elle une base de \(\F{\R}{\R}\) ? \\

\item La famille \(\fami{F}\) est-elle une base de \(\ensclasse{\infty}{\R}{\R}\) ? \\

\item Montrer que la famille \(\fami{F}\) n'est pas libre

\begin{enumerate}
\item en utilisant des limites en \(\pinf\). \\

\item en utilisant des polynômes interpolateurs de Lagrange.
\end{enumerate}
\end{enumerate}
\end{exo}

\begin{corr}
\note{À venir}
\end{corr}

\begin{exo}
On se place dans le \(\R\)-espace vectoriel \(\F{\R}{\R}\).

On définit les vecteurs suivants : \[\quantifs{\forall\lambda\in\R}\fonction{f_\lambda}{\R}{\R}{t}{\sin\paren{\lambda t}}\]

Les familles suivantes sont-elles libres ?

\begin{enumerate}
\item \(\paren{f_\lambda}_{\lambda\in\R}\) \\

\item \(\paren{f_\lambda}_{\lambda\in\Rs}\) \\

\item \(\paren{f_\lambda}_{\lambda\in\Rps}\)
\end{enumerate}
\end{exo}

\begin{corr}
\note{À venir}
\end{corr}

\begin{exo}
Donner une base de l'image et du noyau des endomorphismes suivants :

\begin{enumerate}
\item \[\fonction{u}{\R^3}{\R^3}{\paren{a,b,c}}{\paren{a-b,b-c,c-a}}\]

\item \[\fonction{u}{\polydeg[\R]{3}}{\polydeg[\R]{3}}{P}{P\paren{X+1}-P}\]
\end{enumerate}
\end{exo}

\begin{corr}
\note{À venir}
\end{corr}

\begin{exo}[Base duale]\thlabel{exo:baseDuale}
Soient \(E\) un \(\K\)-espace vectoriel et \(\fami{B}=\paren{e_1,\dots,e_n}\) une base de \(E\).

Pour tout \(k\in\interventierii{1}{n}\), on note \(e_k\etoile:E\to\K\) l'application qui associe à tout vecteur de \(E\) sa \(k\)-ème coordonnée dans la base \(\fami{B}\).

Les applications \(e_1\etoile,\dots,e_n\etoile\) sont donc caractérisées par : \[\quantifs{\forall x\in E}x=\sum_{k=1}^ne_k\etoile\paren{x}e_k.\]

Montrer que la famille \(\fami{B}\etoile=\paren{e_1\etoile,\dots,e_n\etoile}\) est une base de \(E\etoile\) (appelée base duale de \(\fami{B}\)).
\end{exo}

\begin{corr}
\note{À venir}
\end{corr}

\begin{exo}
Montrer que la famille infinie \(\paren{\ln p}_{p\in\prem}\) est une famille libre du \(\Q\)-espace vectoriel \(\R\).
\end{exo}

\begin{corr}
\note{À venir}
\end{corr}

\begin{exo}
On se place dans le \(\Q\)-espace vectoriel \(\R\).

\begin{enumerate}
\item Soit \(p\in\prem\). Montrer que la famille \(\paren{1,\sqrt{p}}\) est libre sur \(\Q\). \\

\item En déduire que la famille \(\paren{1,\sqrt{2},\sqrt{3},\sqrt{6}}\) est libre sur \(\Q\).
\end{enumerate}
\end{exo}

\begin{corr}
\note{À venir}
\end{corr}

\begin{exo}
Résoudre les systèmes linéaires suivants en appliquant l'algorithme du pivot de Gauss.

Lorsque l'ensemble solution est un espace affine, on précisera une base de sa direction.

\[\paren{S_1}\begin{dcases}
a+b+2c+d=7 \\
4a+5b+3c+4d=17 \\
a+2b-3c+d=-4
\end{dcases}\qquad\paren{S_2}\begin{dcases}
a+b+2c=7 \\
4a+5b+3c=17 \\
a+2b-3c=7
\end{dcases}\]
\end{exo}

\begin{corr}
\note{À venir}
\end{corr}